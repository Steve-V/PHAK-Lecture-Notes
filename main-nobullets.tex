%%%%%%%%%%%%%%%%%%%%%%%%%%%%%%%%%%%%%%%%%%%%%%%%%%%%%%%%%%%%%%%%%%%%%%

%%%%%%%%%%%%%%%%%%%%%%%%%%%%%%%%%%%%%%%%%%%%%%%%%%%%%%%%%%%%%%%%%%%%%%
% Standard Palatino font, enable to use "force lining rather than old style" below
\documentclass[sfsidenotes,oneside,justified,marginals=raggedright]{tufte-handout}

% % TisaPro (Requires Tisa font)
% \documentclass[nofonts,oneside,marginals=raggedright]{tufte-handout}
% \usepackage{fontspec}
% \setmainfont{[TisaPro-Regular.otf]}
% \renewcommand\allcapsspacing[1]{{\addfontfeature{LetterSpace=15}#1}}
% \renewcommand\smallcapsspacing[1]{{\addfontfeature{LetterSpace=10}#1}}

\hypersetup{colorlinks}% uncomment this line if you prefer colored hyperlinks (e.g., for onscreen viewing)

%%
% Book metadata
\title[PHAK Lecture~Notes | V2 ]{Lecture~Notes \\ PHAK}
\author[Steve Vaught]{Steve Vaught}
\publisher{}

%\usepackage{microtype}

\usepackage{fancybox}

% For nicely typeset tabular material
\usepackage{booktabs}

% For fancy equations and symbols
\usepackage{amsmath}
\usepackage{gensymb}

%%
% For graphics / images
\usepackage{graphicx}
\setkeys{Gin}{width=\linewidth,totalheight=\textheight,keepaspectratio}
\graphicspath{{graphics/}}

% The fancyvrb package lets us customize the formatting of verbatim
% environments.  We use a slightly smaller font.
\usepackage{fancyvrb}
\fvset{fontsize=\normalsize}

%%
% Prints argument within hanging parentheses (i.e., parentheses that take
% up no horizontal space).  Useful in tabular environments.
\newcommand{\hangp}[1]{\makebox[0pt][r]{(}#1\makebox[0pt][l]{)}}

% Prints a trailing space in a smart way.
\usepackage{xspace}

% Inserts a blank page
\newcommand{\blankpage}{\newpage\hbox{}\thispagestyle{empty}\newpage}

\usepackage{units}

% Typesets the font size, leading, and measure in the form of 10/12x26 pc.
\newcommand{\measure}[3]{#1/#2$\times$\unit[#3]{pc}}

% Generates the index
\usepackage{makeidx}
\makeindex

% Use this command to force lining numbers rather than old-style - May not work with Tisa Pro
\newcommand{\liningnums}[1]{{\fontfamily{pplx}\selectfont #1}}

% Allow QR codes
\usepackage{qrcode}

\begin{document}

% Front matter
%\frontmatter

% r.3 full title page
%\maketitle

% r.5 contents
%\tableofcontents

%%
% Start the main matter (normal chapters)
%\mainmatter

Pilot's Handbook of
Aeronautical Knowledge

2016

U.S. Department of Transportation
FEDERAL AVIATION ADMINISTRATION
Flight Standards Service

ii

Preface
The Pilot's Handbook of Aeronautical Knowledge provides basic knowledge that is essential for pilots. This handbook
introduces pilots to the broad spectrum of knowledge that will be needed as they progress in their pilot training. Except for
the Code of Federal Regulations pertinent to civil aviation, most of the knowledge areas applicable to pilot certification are
presented. This handbook is useful to beginning pilots, as well as those pursuing more advanced pilot certificates.
Occasionally the word "must" or similar language is used where the desired action is deemed critical. The use of such language
is not intended to add to, interpret, or relieve a duty imposed by Title 14 of the Code of Federal Regulations (14 CFR).
It is essential for persons using this handbook to become familiar with and apply the pertinent parts of 14 CFR and the
Aeronautical Information Manual (AIM). The AIM is available online at \url{www.faa.gov}. The current Flight Standards
Service airman training and testing material and learning statements for all airman certificates and ratings can be obtained
from \url{www.faa.gov}.
This handbook supersedes FAA-H-8083-25A, Pilot's Handbook of Aeronautical Knowledge, dated 2008.
This handbook is available for download, in PDF format, from \url{faa.gov}.
This handbook is published by the United States Department of Transportation, Federal Aviation Administration, Airman
Testing Standards Branch, AFS-630, P.O. Box 25082, Oklahoma City, OK 73125.
Comments regarding this publication should be sent, in email form, to the following address:
AFS630comments@faa.gov

John S. Duncan
Director, Flight Standards Service

iii

iv

Acknowledgments
The Pilot's Handbook of Aeronautical Knowledge was produced by the Federal Aviation Administration (FAA) with the
assistance of Safety Research Corporation of America. The FAA wishes to acknowledge the following contributors:
Mrs. Nancy A. Wright for providing imagery of a de Haviland DH-4 inaugural air mail flight (Chapter 1)
The Raab Collection, Philadelphia, Pennsylvania, for images of the first pilot license (Chapter 1)
Sandy Kenyon and Rod Magner (magicair.com) for photo of 1929 TravelAir 4000 (Chapter 1)
Dr. Pat Veillette for information used on decision-making (Chapter 2)
Adventure Seaplanes for photos of a ski and float training plane (Chapter 3)
Jack Davis, Stearman Restorers Asociation, for photo of a 1941 PT-17 Army Air Corps trainer (Chapter 3)
Michael J. Hoke, Abaris Training Resources, Inc., for images and information about composite aircraft (Chapter 3)
Colin Cutler, Boldmethod, for images and content on the topic of ground effect (Chapter 5)
Mark R. Korin, Alpha Systems, for images of AOA disaplys (Chapter 5)
M. van Leeuwen (\url{zap16.com}) for image of Piaggio P-180 (Chapter 6)
Greg Richter, Blue Mountain Avionics, for autopilot information and imagery (Chapter 6)
Mountain High E\&S Company for various images provided regarding oxygen systems (Chapter 7)
Jeff Callahan, Aerox, for image of MSK-AS Silicone Mask without Microphone (Chapter 7)
Nonin Medical, Inc. for image of Onyx pulse oximeter (Chapter 7)
Pilotfriend.com for photo of a TKS Weeping Wing (Chapter 7)
Chelton Flight Systems for image of FlightLogic (Chapter 8)
Avidyne Corporation for image of the Entegra (Chapter 8)
Teledyne Controls for image of an air data computer (Chapter 8)
Watson Industries, Inc. (\url{watson-gyro.com}) for image of Attitude and Heading Reference system (Chapter 8)
Engineering Arresting Systems Corporation (\url{esco.zodiacaerospace.com}) for EMAS imagery and EMASMAX
technical digrams (Chapter 14)
Caasey Rose and Jose Roggeveen (burningholesinthesky.wordpress.com) for flight checklist image (Chapter 14)
Tim Murnahan for images of EMAS at Yeager Airport, Charleston, West Virginia, and EMAS arrested
aircraft (Chapter 14)
Cessna Aircraft Company, Columbia Aircraft Manufacturing Corporation, Eclipse Aviation Corporation, Garmin Ltd.,
The Boeing Company for images provided and used throughout the Handbook.
Additional appreciation is extended to the Aircraft Owners and Pilots Association (AOPA), the AOPA Air Safety Foundation,
the General Aviation Manufacturers Association (GAMA), and the National Business Aviation Association (NBAA) for
their technical support and input.
Disclaimer: Information in Chapter 14 pertaining to Runway Incursion Avoidance was created using FAA orders, documents,
and Advisory Circulars that were current at the date of publication. Users should not assume that all references are current
and should check often for reference updates.

v

vi

Table of Contents
Preface....................................................................iii

Acknowledgments..................................................v

Table of Contents .................................................vii

Chapter 1

Introduction To Flying.........................................1-1

Introduction....................................................................1-1

History of Flight.............................................................1-2

History of the Federal Aviation Administration (FAA) 1-3

Transcontinental Air Mail Route................................1-4

Federal Certification of Pilots and Mechanics ...........1-4

The Federal Aviation Act of 1958..............................1-6

Department of Transportation (DOT) ........................1-6

ATC Automation........................................................1-6

The Professional Air Traffic Controllers

Organization (PATCO) Strike....................................1-6

The Airline Deregulation Act of 1978 .......................1-7

The Role of the FAA......................................................1-7

The Code of Federal Regulations (CFR)....................1-7

Primary Locations of the FAA ...................................1-8

Field Offices ...........................................................1-8

Aviation Safety Inspector (ASI) ................................1-9

FAA Safety Team (FAASTeam)................................1-9

Obtaining Assistance from the FAA ..........................1-9

Aeronautical Information Manual (AIM)...............1-9

Handbooks ............................................................1-10

Advisory Circulars (ACs) .....................................1-10

Flight Publications ................................................1-11

Pilot and Aeronautical Information .........................1-12

Notices to Airmen (NOTAMs) ............................1-12

Safety Program Airmen Notification System

(SPANS) ...............................................................1-14

Aircraft Classifications and Ultralight Vehicles ..........1-14

Pilot Certifications .......................................................1-16

Privileges: .............................................................1-16

Limitations:...........................................................1-17

Recreational Pilot .....................................................1-17

Privileges: .............................................................1-17


Limitations:...........................................................1-17

Private Pilot ..............................................................1-17

Commercial Pilot......................................................1-18

Airline Transport Pilot..............................................1-18

Selecting a Flight School ............................................1-18

How To Find a Reputable Flight Program ...............1-19

How To Choose a Certificated Flight

Instructor (CFI).........................................................1-19

The Student Pilot .........................................................1-20

Basic Requirements..................................................1-20

Medical Certification Requirements.....................1-20

Becoming a Pilot..........................................................1-21

Knowledge and Skill Tests ..........................................1-21

Knowledge Tests ......................................................1-21

When To Take the Knowledge Test .....................1-22

Practical Test ............................................................1-22

When To Take the Practical Test..........................1-23

Who Administers the FAA Practical Tests?.........1-23

Role of the Certificated Flight Instructor .................1-23

Role of the Designated Pilot Examiner ...................1-24

Chapter Summary .......................................................1-24

Chapter 2

Aeronautical Decision-Making ...........................2-1

Introduction....................................................................2-1

History of ADM.............................................................2-2

Risk Management ..........................................................2-3

Crew Resource Management (CRM) and Single-

Pilot Resource Management ..........................................2-4

Hazard and Risk.............................................................2-4

Hazardous Attitudes and Antidotes............................2-5

Risk.............................................................................2-6

Assessing Risk ........................................................2-6

Mitigating Risk .......................................................2-8

The PAVE Checklist .................................................2-8

P = Pilot in Command (PIC) ..................................2-8

A = Aircraft ............................................................2-8

V = EnVironment ...................................................2-9

E = External Pressures ............................................2-9

Human Factors .............................................................2-10

vii

Human Behavior ..........................................................2-11

The Decision-Making Process .....................................2-12

Single-Pilot Resource Management (SRM) ...........2-13

The 5 Ps Check ........................................................2-13

The Plan ...............................................................2-14

The Plane .............................................................2-14

The Pilot ...............................................................2-14

The Passengers .....................................................2-14

The Programming ................................................2-15

Perceive, Process, Perform (3P) Model....................2-15

PAVE Checklist: Identify Hazards and

Personal Minimums ..............................................2-15

CARE Checklist: Review Hazards and

Evaluate Risks ......................................................2-16

TEAM Checklist: Choose and Implement

Risk Controls ........................................................2-16

The DECIDE Model .............................................2-18

Detect (the Problem).............................................2-20

Estimate (the Need To React)...............................2-20

Choose (a Course of Action) ................................2-20

Identify (Solutions)...............................................2-20

Do (the Necessary Actions) ..................................2-20

Evaluate (the Effect of the Action) ......................2-20

Decision-Making in a Dynamic Environment ............2-21

Automatic Decision-Making ...................................2-21

Operational Pitfalls ...............................................2-21

Stress Management...................................................2-21

Use of Resources ......................................................2-21

Internal Resources ................................................2-23

External Resources ...............................................2-23

Situational Awareness..................................................2-24

Obstacles to Maintaining Situational Awareness.....2-24

Workload Management ........................................2-24

Managing Risks ....................................................2-25

Automation ..................................................................2-25

Results of the Study..................................................2-27

Equipment Use .........................................................2-27

Autopilot Systems.................................................2-27

Familiarity.............................................................2-27

Respect for Onboard Systems...............................2-29

Getting Beyond Rote Workmanship.....................2-29

Understand the Platform ......................................2-29

Managing Aircraft Automation ...............................2-29

Information Management .....................................2-30

Enhanced Situational Awareness .............................2-30

Automation Management .........................................2-31

Risk Management.....................................................2-31

Chapter Summary .......................................................2-32


viii

Chapter 3

Aircraft Construction ..........................................3-1

Introduction....................................................................3-1

Aircraft Design, Certification, and Airworthiness.........3-2

A Note About Light Sport Aircraft ............................3-2

Lift and Basic Aerodynamics.........................................3-2

Major Components.........................................................3-3

Fuselage......................................................................3-3

Wings .........................................................................3-3

Empennage .................................................................3-6

Landing Gear..............................................................3-7

The Powerplant...........................................................3-7

Subcomponents ..............................................................3-8

Types of Aircraft Construction ......................................3-8

Truss Structure ...........................................................3-8

Semimonocoque .........................................................3-9

Composite Construction.............................................3-9

History ....................................................................3-9

Advantages of Composites ...................................3-10

Disadvantages of Composites...............................3-10

Fluid Spills on Composites...................................3-11

Lightning Strike Protection...................................3-11

The Future of Composites ....................................3-12

Instrumentation: Moving into the Future ....................3-12

Control Instruments .................................................3-13

Navigation Instruments ...........................................3-13

Global Positioning System (GPS)................................3-13

Chapter Summary ........................................................3-13

Chapter 4

Principles of Flight ..............................................4-1

Introduction....................................................................4-1

Structure of the Atmosphere ..........................................4-1

Air is a Fluid ..............................................................4-2

Viscosity .................................................................4-2

Friction....................................................................4-2

Pressure...................................................................4-3

Atmospheric Pressure.................................................4-3

Pressure Altitude ........................................................4-4

Density Altitude .........................................................4-4

Effect of Pressure on Density .................................4-4

Effect of Temperature on Density ..........................4-4

Effect of Humidity (Moisture) on Density .............4-5

Theories in the Production of Lift..................................4-5

Newton's Basic Laws of Motion................................4-5

Bernoulli's Principle of Differential Pressure ............4-6

Airfoil Design ................................................................4-6

Low Pressure Above ..................................................4-7

High Pressure Below ..................................................4-8


Pressure Distribution ..................................................4-8

Airfoil Behavior .........................................................4-8

A Third Dimension ........................................................4-9

Chapter Summary ..........................................................4-9

Chapter 5

Aerodynamics of Flight.......................................5-1

Forces Acting on the Aircraft ........................................5-1

Thrust .........................................................................5-2

Lift..............................................................................5-3

Lift/Drag Ratio........................................................5-5

Drag ............................................................................5-6

Parasite Drag...........................................................5-6

Induced Drag ..........................................................5-7

Weight ........................................................................5-8

Wingtip Vortices............................................................5-8

Formation of Vortices ................................................5-8

Avoiding Wake Turbulence .......................................5-9

Ground Effect...............................................................5-11

Axes of an Aircraft.......................................................5-12

Moment and Moment Arm ..........................................5-13

Aircraft Design Characteristics ...................................5-14

Stability ....................................................................5-14

Static Stability.......................................................5-14

Dynamic Stability .................................................5-14

Longitudinal Stability (Pitching) ..........................5-15

Lateral Stability (Rolling).....................................5-17

Directional Stability (Yawing) .............................5-19

Free Directional Oscillations (Dutch Roll) ..............5-20

Spiral Instability .......................................................5-20

Effect of Wing Planform .............................................5-20

Aerodynamic Forces in Flight Maneuvers...................5-22

Forces in Turns.........................................................5-22

Forces in Climbs.......................................................5-23

Forces in Descents....................................................5-24

Stalls ............................................................................5-25

Angle of Attack Indicators...........................................5-26

Basic Propeller Principles ............................................5-28

Torque and P-Factor.................................................5-30

Torque Reaction .......................................................5-31

Corkscrew Effect......................................................5-31

Gyroscopic Action....................................................5-31

Asymmetric Loading (P-Factor) ..............................5-32

Load Factors.................................................................5-33

Load Factors in Aircraft Design...............................5-33

Load Factors in Steep Turns.....................................5-34

Load Factors and Stalling Speeds ............................5-34

Load Factors and Flight Maneuvers.........................5-36

Vg Diagram ..............................................................5-37

Rate of Turn..............................................................5-38

Radius of Turn..........................................................5-39


Weight and Balance .....................................................5-40

Effect of Weight on Flight Performance ..................5-42

Effect of Weight on Aircraft Structure.....................5-42

Effect of Weight on Stability and Controllability ....5-42

Effect of Load Distribution ......................................5-43

High Speed Flight ........................................................5-44

Subsonic Versus Supersonic Flow ...........................5-44

Speed Ranges ...........................................................5-44

Mach Number Versus Airspeed ...............................5-45

Boundary Layer........................................................5-46

Laminar Boundary Layer Flow ............................5-46

Turbulent Boundary Layer Flow ..........................5-46

Boundary Layer Separation ..................................5-46

Shock Waves ............................................................5-46

Sweepback................................................................5-48

Mach Buffet Boundaries ..........................................5-49

High Speed Flight Controls......................................5-49

Chapter Summary ........................................................5-51

Chapter 6

Flight Controls .....................................................6-1

Introduction....................................................................6-1

Flight Control Systems .................................................6-2

Flight Controls............................................................6-2

Primary Flight Controls..............................................6-2

Elevator...................................................................6-5

T-Tail ......................................................................6-6

Stabilator.................................................................6-7

Canard.....................................................................6-7

Rudder.....................................................................6-8

V-Tail......................................................................6-8

Secondary Flight Controls..........................................6-8

Flaps........................................................................6-8

Leading Edge Devices ............................................6-9

Spoilers .................................................................6-10

Trim Tabs..............................................................6-10

Balance Tabs.........................................................6-11

Servo Tabs ............................................................6-11

Antiservo Tabs......................................................6-11

Ground Adjustable Tabs .......................................6-11

Adjustable Stabilizer.............................................6-12

Autopilot ......................................................................6-12

Chapter Summary ........................................................6-12

Chapter 7

Aircraft Systems ..................................................7-1

Introduction....................................................................7-1

Powerplant .....................................................................7-1

Reciprocating Engines................................................7-2

Propeller .....................................................................7-4

ix

Fixed-Pitch Propeller ..............................................7-5

Adjustable-Pitch Propeller......................................7-6

Propeller Overspeed in Piston Engine Aircraft ......7-7

Induction Systems ......................................................7-7

Carburetor Systems ....................................................7-8

Mixture Control ......................................................7-9

Carburetor Icing......................................................7-9

Carburetor Heat ....................................................7-10

Carburetor Air Temperature Gauge......................7-11

Outside Air Temperature Gauge ..............................7-11

Fuel Injection Systems .............................................7-11

Superchargers and Turbosuperchargers.......................7-12

Superchargers ...........................................................7-12

Turbosuperchargers ..................................................7-13

System Operation..................................................7-14

High Altitude Performance...................................7-14

Ignition System ............................................................7-15

Oil Systems ..................................................................7-16

Engine Cooling Systems ..............................................7-17

Exhaust Systems ..........................................................7-18

Starting System ............................................................7-18

Combustion ..................................................................7-18

Full Authority Digital Engine Control (FADEC) ........7-20

Turbine Engines ...........................................................7-20

Types of Turbine Engines ........................................7-20

Turbojet.................................................................7-20

Turboprop .............................................................7-21

Turbofan ...............................................................7-21

Turboshaft.............................................................7-21

Turbine Engine Instruments .....................................7-22

Engine Pressure Ratio (EPR) ...............................7-22

Exhaust Gas Temperature (EGT) ........................7-22

Torquemeter..........................................................7-22

N1 Indicator...........................................................7-23

N2 Indicator...........................................................7-23

Turbine Engine Operational Considerations ............7-23

Engine Temperature Limitations ..........................7-23

Thrust Variations ..................................................7-23

Foreign Object Damage (FOD) ............................7-23

Turbine Engine Hot/Hung Start............................7-23

Compressor Stalls .................................................7-23

Flameout ...............................................................7-24

Performance Comparison .........................................7-24

Airframe Systems ........................................................7-25

Fuel Systems ................................................................7-25

Gravity-Feed System ...............................................7-25

Fuel-Pump System ...................................................7-25

Fuel Primer...............................................................7-25

Fuel Tanks ................................................................7-25


x

Fuel Gauges..............................................................7-26

Fuel Selectors ...........................................................7-26

Fuel Strainers, Sumps, and Drains ...........................7-27

Fuel Grades...............................................................7-27

Fuel Contamination ..................................................7-27

Fuel System Icing.....................................................7-28

Prevention Procedures ..........................................7-28

Refueling Procedures ...................................................7-29

Heating System ............................................................7-29

Fuel Fired Heaters ....................................................7-29

Exhaust Heating Systems .........................................7-29

Combustion Heater Systems ....................................7-29

Bleed Air Heating Systems ......................................7-30

Electrical System .........................................................7-30

Hydraulic Systems .......................................................7-31

Landing Gear............................................................7-33

Tricycle Landing Gear..........................................7-33

Tailwheel Landing Gear .......................................7-33

Fixed and Retractable Landing Gear ....................7-34

Brakes.......................................................................7-34

Pressurized Aircraft .....................................................7-34

Oxygen Systems...........................................................7-37

Oxygen Masks..........................................................7-38

Cannula.....................................................................7-38

Pressure-Demand Oxygen Systems..........................7-38

Continuous-Flow Oxygen System ...........................7-38

Electrical Pulse-Demand Oxygen System................7-38

Pulse Oximeters........................................................7-39

Servicing of Oxygen Systems ..................................7-39

Anti-Ice and Deice Systems.........................................7-40

Airfoil Anti-Ice and Deice .......................................7-40

Windscreen Anti-Ice.................................................7-41

Propeller Anti-Ice .....................................................7-41

Other Anti-Ice and Deice Systems ...........................7-41

Chapter Summary ........................................................7-41

Chapter 8

Flight Instruments ...............................................8-1

Introduction....................................................................8-1

Pitot-Static Flight Instruments .......................................8-1

Impact Pressure Chamber and Lines ..........................8-2

Static Pressure Chamber and Lines ............................8-2

Altimeter.....................................................................8-3

Principle of Operation.............................................8-3

Effect of Nonstandard Pressure and Temperature ..8-4

Setting the Altimeter...............................................8-5

Altimeter Operation ................................................8-6

Types of Altitude ....................................................8-6

Instrument Check....................................................8-7

Vertical Speed Indicator (VSI)...................................8-7


Principle of Operation.............................................8-7

Instrument Check....................................................8-8

Airspeed Indicator (ASI) ............................................8-8

Airspeed Indicator Markings ..................................8-9

Other Airspeed Limitations ....................................8-9

Instrument Check..................................................8-10

Blockage of the Pitot-Static System.........................8-10

Blocked Pitot System............................................8-10

Blocked Static System ..........................................8-11

Electronic Flight Display (EFD)..................................8-12

Airspeed Tape...........................................................8-12

Attitude Indicator .....................................................8-13

Altimeter...................................................................8-13

Vertical Speed Indicator (VSI).................................8-13

Heading Indicator .....................................................8-13

Turn Indicator...........................................................8-13

Tachometer...............................................................8-13

Slip/Skid Indicator....................................................8-13

Turn Rate Indicator ..................................................8-13

Air Data Computer (ADC) .......................................8-14

Trend Vectors ...........................................................8-14

Gyroscopic Flight Instruments.....................................8-15

Gyroscopic Principles...............................................8-15

Rigidity in Space...................................................8-15

Precession .............................................................8-15

Sources of Power......................................................8-16

Turn Indicators .........................................................8-16

Turn-and-Slip Indicator ........................................8-16

Turn Coordinator ..................................................8-17

Inclinometer..............................................................8-18

Yaw String ............................................................8-18

Instrument Check..................................................8-18

Attitude Indicator .....................................................8-18

Heading Indicator .....................................................8-19

Attitude and Heading Reference System (AHRS) ...8-20

The Flux Gate Compass System ..............................8-20

Remote Indicating Compass.....................................8-21

Instrument Check..................................................8-22

Angle of Attack Indicators...........................................8-22

Compass Systems.........................................................8-23

Magnetic Compass ...................................................8-23

Magnetic Compass Induced Errors.......................8-24

The Vertical Card Magnetic Compass .....................8-27

Lags or Leads........................................................8-27

Eddy Current Damping.........................................8-27

Outside Air Temperature (OAT) Gauge ......................8-28

Chapter Summary ........................................................8-28


Chapter 9

Flight Manuals and Other Documents...............9-1

Introduction....................................................................9-1

Preliminary Pages.......................................................9-2

General (Section 1).....................................................9-2

Limitations (Section 2)...............................................9-2

Airspeed..................................................................9-2

Powerplant ..............................................................9-3

Weight and Loading Distribution ...........................9-3

Flight Limits ...........................................................9-4

Placards...................................................................9-4

Emergency Procedures (Section 3) ............................9-4

Normal Procedures (Section 4) ..................................9-4

Performance (Section 5).............................................9-4

Weight and Balance/Equipment List (Section 6) .......9-4

Systems Description (Section 7) ................................9-4

Handling, Service, and Maintenance (Section 8) .......9-5

Supplements (Section 9).............................................9-5

Safety Tips (Section 10) .............................................9-6

Aircraft Documents........................................................9-6

Certificate of Aircraft Registration.............................9-6

Airworthiness Certificate ...........................................9-7

Aircraft Maintenance..................................................9-8

Aircraft Inspections........................................................9-8

Annual Inspection.......................................................9-8

100-Hour Inspection...................................................9-8

Other Inspection Programs.........................................9-9

Altimeter System Inspection ..................................9-9

Transponder Inspection ..........................................9-9

Emergency Locator Transmitter .............................9-9

Preflight Inspections ...............................................9-9

Minimum Equipment Lists (MEL) and Operations

With Inoperative Equipment .........................................9-9

Preventive Maintenance...............................................9-10

Maintenance Entries .............................................9-10

Examples of Preventive Maintenance ..................9-10

Repairs and Alterations ............................................9-12

Special Flight Permits ..............................................9-12

Airworthiness Directives (ADs) ..................................9-12

Aircraft Owner/Operator Responsibilities ...................9-13

Chapter Summary ........................................................9-13

Chapter 10

Weight and Balance ..........................................10-1

Introduction..................................................................10-1

Weight Control.............................................................10-1

Effects of Weight......................................................10-2

Weight Changes .......................................................10-2


xi

Balance, Stability, and Center of Gravity ....................10-2

Effects of Adverse Balance ......................................10-3

Stability.................................................................10-3

Control ..................................................................10-3

Management of Weight and Balance Control ..........10-4

Terms and Definitions ..............................................10-4

Principles of Weight and Balance Computations.....10-5

Weight and Balance Restrictions .............................10-6

Determining Loaded Weight and CG ..........................10-7

Computational Method.............................................10-7

Graph Method...........................................................10-7

Table Method ...........................................................10-9

Computations With a Negative Arm ......................10-10

Computations With Zero Fuel Weight ...................10-10

Shifting, Adding, and Removing Weight...............10-10

Weight Shifting...................................................10-10

Weight Addition or Removal..............................10-11

Chapter Summary ......................................................10-11

Chapter 11

Aircraft Performance.........................................11-1

Introduction..................................................................11-1

Importance of Performance Data .................................11-1

Structure of the Atmosphere ........................................11-2

Atmospheric Pressure ..................................................11-2

Pressure Altitude..........................................................11-3

Density Altitude ...........................................................11-3

Effects of Pressure on Density .................................11-4

Effects of Temperature on Density ..........................11-5

Effects of Humidity (Moisture) on Density .............11-5

Performance .................................................................11-5

Straight-and-Level Flight .........................................11-5

Climb Performance...................................................11-6

Angle of Climb (AOC) .........................................11-7

Rate of Climb (ROC)............................................11-7

Climb Performance Factors ..................................11-8

Range Performance ..................................................11-9

Region of Reversed Command...............................11-11

Takeoff and Landing Performance.........................11-12

Runway Surface and Gradient................................11-12

Water on the Runway and Dynamic

Hydroplaning..........................................................11-13

Takeoff Performance..............................................11-14

Landing Performance .............................................11-16

Performance Speeds...................................................11-18

Performance Charts....................................................11-19

Interpolation ...........................................................11-20

Density Altitude Charts..........................................11-20

Takeoff Charts........................................................11-20

Climb and Cruise Charts ........................................11-21

Crosswind and Headwind Component Chart .........11-25

xii

Landing Charts .......................................................11-26

Stall Speed Performance Charts .............................11-27

Transport Category Aircraft Performance .................11-28

Air Carrier Obstacle Clearance Requirements...........11-28

Chapter Summary ......................................................11-28

Chapter 12

Weather Theory .................................................12-1

Introduction..................................................................12-1

Atmosphere ..................................................................12-2

Composition of the Atmosphere...............................12-2

Atmospheric Circulation ..........................................12-3

Atmospheric Pressure...............................................12-3

Coriolis Force...............................................................12-3

Measurement of Atmosphere Pressure ........................12-4

Altitude and Atmospheric Pressure .............................12-5

Altitude and Flight .......................................................12-6

Altitude and the Human Body .....................................12-6

Wind and Currents .......................................................12-7

Wind Patterns ...........................................................12-7

Convective Currents .................................................12-7

Effect of Obstructions on Wind................................12-8

Low-Level Wind Shear ..........................................12-11

Wind and Pressure Representation on Surface

Weather Maps.........................................................12-12

Atmospheric Stability ................................................12-12

Inversion.................................................................12-13

Moisture and Temperature .....................................12-13

Relative Humidity ..................................................12-13

Temperature/Dew Point Relationship ....................12-13

Methods by Which Air Reaches the Saturation

Point .......................................................................12-14

Dew and Frost ........................................................12-15

Fog..........................................................................12-15

Clouds.....................................................................12-15

Ceiling ....................................................................12-17

Visibility.................................................................12-17

Precipitation............................................................12-17

Air Masses .................................................................12-17

Fronts .........................................................................12-18

Warm Front ............................................................12-18

Flight Toward an Approaching Warm Front ......12-19

Cold Front ..............................................................12-20

Fast-Moving Cold Front .....................................12-20

Flight Toward an Approaching Cold Front ........12-20

Comparison of Cold and Warm Fronts ..................12-20

Wind Shifts.............................................................12-21

Stationary Front......................................................12-21

Occluded Front .......................................................12-21

Thunderstorms........................................................12-22

Hazards ..............................................................12-23


Squall Line .........................................................12-23

Tornadoes ..........................................................12-23

Turbulence .........................................................12-24

Icing ...................................................................12-24

Hail ....................................................................12-25

Ceiling and Visibility .........................................12-25

Effect on Altimeters ...........................................12-25

Lightning.............................................................12-25

Engine Water Ingestion .....................................12-25

Chapter Summary ......................................................12-25


Weather Products Age and Expiration ...................13-18

What Can Pilots Do? ..........................................13-19

NEXRAD Abnormalities....................................13-21

NEXRAD Limitations ........................................13-21

AIRMET/SIGMET Display ...................................13-21

Graphical METARs................................................13-21

Data Link Weather .................................................13-21

Data Link Weather Products ..................................13-23

Flight Information Service- Broadcast (FIS-B)..13-23

Pilot Responsibility....................................................13-24

Chapter Summary ......................................................13-24


Chapter 13

Aviation Weather Services ...............................13-1

Introduction..................................................................13-1

Observations ................................................................13-2

Surface Aviation Weather Observations ..................13-2

Air Route Traffic Control Center (ARTCC) ........13-2

Upper Air Observations............................................13-2

Radar Observations ..................................................13-3

Satellite.....................................................................13-4

Service Outlets .............................................................13-4

Flight Service Station (FSS).....................................13-4

Telephone Information Briefing Service (TIBS) .....13-4

Hazardous Inflight Weather Advisory

Service (HIWAS) .....................................................13-4

Transcribed Weather Broadcast (TWEB)

(Alaska Only) ..........................................................13-4

Weather Briefings ........................................................13-5

Standard Briefing .....................................................13-5

Abbreviated Briefing................................................13-5

Outlook Briefing ......................................................13-5

Aviation Weather Reports............................................13-5

Aviation Routine Weather Report (METAR) ..........13-6

Pilot Weather Reports (PIREPs) ..............................13-8

Aviation Forecasts....................................................13-9

Terminal Aerodrome Forecasts (TAF).....................13-9

Area Forecasts (FA) ...............................................13-10

Inflight Weather Advisories ...................................13-11

AIRMET .............................................................13-11

SIGMET .............................................................13-12

Convective Significant Meteorological

Information (WST) .............................................13-12

Winds and Temperature Aloft Forecast (FB).........13-13

Weather Charts...........................................................13-13

Surface Analysis Chart...........................................13-13

Weather Depiction Chart........................................13-15

Significant Weather Prognostic Charts ..................13-15

ATC Radar Weather Displays ..................................13-16

Weather Avoidance Assistance ..............................13-18

Electronic Flight Displays (EFD) /Multi-Function

Display (MFD) Weather ...........................................13-18


Chapter 14

Airport Operations.............................................14-1

Introduction..................................................................14-1

Airport Categories........................................................14-1

Types of Airports......................................................14-2

Towered Airport ...................................................14-2

Nontowered Airport..............................................14-2

Sources for Airport Data..............................................14-3

Aeronautical Charts..................................................14-3

Chart Supplement U.S. (formerly Airport/Facility

Directory) .................................................................14-3

Notices to Airmen (NOTAM) ..................................14-4

Automated Terminal Information Service (ATIS)...14-5

Airport Markings and Signs.........................................14-5

Runway Markings and Signs....................................14-5

Relocated Runway Threshold...............................14-5

Displaced Threshold .............................................14-5

Runway Safety Area ............................................14-6

Runway Safety Area Boundary Sign....................14-6

Runway Holding Position Sign ............................14-6

Runway Holding Position Marking ......................14-8

Runway Distance Remaining Signs......................14-8

Runway Designation Marking..............................14-8

Land and Hold Short Operations (LAHSO) .......14-10

Taxiway Markings and Signs .................................14-11

Enhanced Taxiway Centerline Markings............14-12

Destination Signs ................................................14-12

Holding Position Signs and Markings for an

Instrument Landing System (ILS) Critical Area ..14-12

Holding Position Markings for Taxiway/Taxiway

Intersections........................................................14-14

Marking and Lighting of Permanently Closed

Runways and Taxiways ......................................14-14

Temporarily Closed Runways and Taxiways .....14-15

Other Markings.......................................................14-15

Airport Signs ..........................................................14-15

Airport Lighting .........................................................14-16

Airport Beacon .......................................................14-16

xiii

Approach Light Systems ........................................14-16

Visual Glideslope Indicators ..................................14-16

Visual Approach Slope Indicator (VASI)...........14-16

Other Glidepath Systems ....................................14-16

Runway Lighting....................................................14-17

Runway End Identifier Lights (REIL)................14-17

Runway Edge Lights...........................................14-17

In-Runway Lighting............................................14-18

Control of Airport Lighting....................................14-18

Taxiway Lights.......................................................14-19

Omnidirectional .................................................14-19

Clearance Bar Lights ..........................................14-19

Runway Guard Lights.........................................14-19

Stop Bar Lights ...................................................14-19

Obstruction Lights..................................................14-19

New Lighting Technologies ...................................14-19

Wind Direction Indicators..........................................14-20

Traffic Patterns ..........................................................14-20

Example: Key to Traffic Pattern Operations—

Single Runway .......................................................14-21

Example: Key to Traffic Pattern Operations—

Parallel Runways....................................................14-21

Radio Communications..............................................14-22

Radio License .........................................................14-22

Radio Equipment....................................................14-22

Using Proper Radio Procedures ............................14-22

Lost Communication Procedures ...........................14-23

Air Traffic Control (ATC) Services...........................14-24

Primary Radar.........................................................14-24

ATC Radar Beacon System (ATCRBS) ................14-24

Transponder............................................................14-25

Automatic Dependent Surveillance–

Broadcast (ADS-B) ................................................14-26

Radar Traffic Advisories ........................................14-26

Wake Turbulence .......................................................14-26

Vortex Generation ..................................................14-26

Terminal Area ....................................................14-27

En Route .............................................................14-27

Vortex Behavior .....................................................14-27

Vortex Avoidance Procedures................................14-28

Collision Avoidance...................................................14-28

Clearing Procedures ...............................................14-28

Pilot Deviations (PDs)............................................14-30

Runway Incursion Avoidance ................................14-30

Causal Factors of Runway Incursions ....................14-31

Runway Confusion.................................................14-31

Causal Factors of Runway Confusion ................14-31

ATC Instructions ....................................................14-32

ATC Instructions—"Hold Short" ......................14-32


xiv

ATC Instructions—Explicit Runway Crossing.... 14-33

ATC Instructions—"Line Up and Wait"

(LUAW)..............................................................14-33

ATC Instructions—"Runway Shortened" ..........14-34

Pre-Landing, Landing, and After-Landing.............14-34

Engineered Materials Arresting Systems (EMAS) ....14-35

Incidents .................................................................14-35

EMAS Installations and Information .....................14-35

Pilot Considerations ...............................................14-36

Chapter Summary ......................................................14-37

Chapter 15

Airspace .............................................................15-1

Introduction..................................................................15-1

Controlled Airspace .....................................................15-2

Class A Airspace ......................................................15-2

Class B Airspace ......................................................15-2

Class C Airspace ......................................................15-2

Class D Airspace ......................................................15-2

Class E Airspace.......................................................15-2

Uncontrolled Airspace .................................................15-3

Class G Airspace ......................................................15-3

Special Use Airspace ...................................................15-3

Prohibited Areas .......................................................15-3

Restricted Areas .......................................................15-3

Warning Areas..........................................................15-4

Military Operation Areas (MOAs) ...........................15-4

Alert Areas ...............................................................15-4

Controlled Firing Areas (CFAs)...............................15-4

Other Airspace Areas...................................................15-4

Local Airport Advisory (LAA) ................................15-6

Military Training Routes (MTRs) ............................15-6

Temporary Flight Restrictions (TFR).......................15-6

Published VFR Routes .............................................15-6

Terminal Radar Service Areas (TRSAs) ..................15-7

National Security Areas (NSAs) ..............................15-7

Air Traffic Control and the National Airspace System ..15-7

Coordinating the Use of Airspace ............................15-7

Operating in the Various Types of Airspace ............15-7

Basic VFR Weather Minimums............................15-7

Operating Rules and Pilot/Equipment

Requirements ........................................................15-8

Ultralight Vehicles..............................................15-11

Unmanned Free Balloons ...................................15-11

Unmanned Aircraft Systems...............................15-11

Parachute Jumps .................................................15-11

Chapter Summary ......................................................15-11


Chapter 16

Navigation ..........................................................16-1

Introduction..................................................................16-1

Aeronautical Charts .....................................................16-2

Sectional Charts........................................................16-2

VFR Terminal Area Charts ......................................16-2

World Aeronautical Charts.......................................16-2

Latitude and Longitude (Meridians and Parallels).......16-3

Time Zones...............................................................16-3

Measurement of Direction........................................16-5

Variation...................................................................16-6

Magnetic Variation ...............................................16-7

Magnetic Deviation ..............................................16-7

Deviation ..................................................................16-8

Effect of Wind..............................................................16-8

Basic Calculations......................................................16-11

Converting Minutes to Equivalent Hours...............16-11

Time T = D/GS ...................................................16-11

Distance D = GS X T..........................................16-11

GS GS = D/T ......................................................16-11

Converting Knots to Miles Per Hour......................16-11

Fuel Consumption ..................................................16-11

Flight Computers....................................................16-12

Plotter .....................................................................16-12

Pilotage ......................................................................16-12

Dead Reckoning.........................................................16-13

Wind Triangle or Vector Analysis .........................16-13

Step 1 ..................................................................16-14

Step 2 ..................................................................16-15

Step 3 ..................................................................16-15

Step 4 ..................................................................16-15

Flight Planning...........................................................16-17

Assembling Necessary Material.............................16-17

Weather Check .......................................................16-17

Use of Chart Supplement U.S. (formerly

Airport/Facility Directory) .....................................16-17

Airplane Flight Manual or Pilot's Operating

Handbook (AFM/POH)..........................................16-17

Charting the Course ...................................................16-18

Steps in Charting the Course..................................16-18

Filing a VFR Flight Plan............................................16-21

Ground-Based Navigation .........................................16-22

Very High Frequency (VHF) Omnidirectional

Range (VOR)..........................................................16-22

Using the VOR ...................................................16-23

Course Deviation Indicator (CDI)..........................16-23

Horizontal Situation Indicator ................................16-24

Radio Magnetic Indicator (RMI)............................16-24

Tracking With VOR ...............................................16-25

Tips on Using the VOR..........................................16-26


Time and Distance Check From a Station Using

a RMI......................................................................16-26

Time and Distance Check From a Station Using

a CDI ......................................................................16-27

Course Intercept ....................................................16-27

Rate of Intercept .................................................16-27

Angle of Intercept ..............................................16-27

Distance Measuring Equipment (DME).................16-27

VOR/DME RNAV .................................................16-28

Automatic Direction Finder (ADF)........................16-29

Global Positioning System .....................................16-30

Selective Availability..........................................16-31

VFR Use of GPS ................................................16-32

RAIM Capability ................................................16-32

Tips for Using GPS for VFR Operations ...............16-33

VFR Waypoints .....................................................16-33

Lost Procedures..........................................................16-34

Flight Diversion .........................................................16-34

Chapter Summary ......................................................16-35

Chapter 17

Aeromedical Factors .........................................17-1

Introduction..................................................................17-1

Obtaining a Medical Certificate...................................17-2

Health and Physiological Factors Affecting Pilot

Performance .................................................................17-3

Hypoxia ....................................................................17-3

Hypoxic Hypoxia..................................................17-3

Hypemic Hypoxia.................................................17-3

Stagnant Hypoxia..................................................17-3

Histotoxic Hypoxia...............................................17-4

Symptoms of Hypoxia..............................................17-4

Treatment of Hypoxia...........................................17-4

Hyperventilation.......................................................17-4

Middle Ear and Sinus Problems ...............................17-5

Spatial Disorientation and Illusions .........................17-6

Vestibular Illusions...............................................17-7

Visual Illusions .....................................................17-8

Postural Considerations............................................17-8

Demonstration of Spatial Disorientation..................17-8

Climbing While Accelerating ...............................17-9

Climbing While Turning.......................................17-9

Diving While Turning...........................................17-9

Tilting to Right or Left .........................................17-9

Reversal of Motion ...............................................17-9

Diving or Rolling Beyond the Vertical Plane.......17-9

Coping with Spatial Disorientation ..........................17-9

Optical Illusions .....................................................17-10

Runway Width Illusion.......................................17-10


xv

Runway and Terrain Slopes Illusion...................17-10

Featureless Terrain Illusion ................................17-10

Water Refraction.................................................17-10

Haze ....................................................................17-10

Fog ......................................................................17-10

Ground Lighting Illusions...................................17-10

How To Prevent Landing Errors Due to

Optical Illusions .....................................................17-10

Motion Sickness .....................................................17-12

Carbon Monoxide (CO) Poisoning ........................17-12

Stress ......................................................................17-12

Fatigue....................................................................17-13

Exposure to Chemicals...........................................17-13

Hydraulic Fluid...................................................17-13

Engine Oil...........................................................17-14

Fuel .....................................................................17-14

Dehydration and Heatstroke...................................17-14

Alcohol ...................................................................17-15

Drugs ......................................................................17-16

Altitude-Induced Decompression Sickness (DCS).. 17-18

DCS After Scuba Diving ....................................17-18

Vision in Flight ..........................................................17-19

Vision Types ..........................................................17-20

Photopic Vision ..................................................17-20

Mesopic Vision...................................................17-21

Scotopic Vision...................................................17-21

Central Blind Spot..................................................17-21

Empty-Field Myopia ..............................................17-22

Night Vision ...........................................................17-22

Night Blind Spot .................................................17-22

Dark Adaptation..................................................17-23

Scanning Techniques .........................................17-23

Night Vision Protection ......................................17-23

Self-Imposed Stress ...........................................17-25

Distance Estimation and Depth Perception .......17-25

Binocular Cues....................................................17-26

Night Vision Illusions ............................................17-26

Autokinesis .........................................................17-26

False Horizon......................................................17-26

Reversible Perspective Illusion...........................17-26

Size-Distance Illusion.........................................17-27

Fascination (Fixation).........................................17-27

Flicker Vertigo....................................................17-27

Night Landing Illusions..........................................17-27

Enhanced Night Vision Systems ............................17-27

Synthetic Vision System.....................................17-28

Enhanced Flight Vision System..........................17-28

Chapter Summary ......................................................17-29


xvi

Appendix A

Performance Data for Cessna Model 172R

and Challenger 605............................................. A-1

Appendix B

Acronyms, Abbreviations, and NOTAM

Contractions ....................................................... B-1

Appendix C

Airport Signs and Markings............................... C-1

Glossary ..............................................................G-1

Index ......................................................................I-1


Chapter 1

Introduction
To Flying
Introduction
The Pilot's Handbook of Aeronautical Knowledge provides
basic knowledge for the student pilot learning to fly, as well
as pilots seeking advanced pilot certification. For detailed
information on a variety of specialized flight topics, see
specific Federal Aviation Administration (FAA) handbooks
and Advisory Circulars (ACs).
This chapter offers a brief history of flight, introduces the
history and role of the FAA in civil aviation, FAA regulations
and standards, government references and publications,
eligibility for pilot certificates, available routes to flight
instruction, the role of the Certificated Flight Instructor (CFI)
and Designated Pilot Examiner (DPE) in flight training,
Practical Test Standards (PTS), and new, industry-developed
Airman Certification Standards (ACS) framework that will
eventually replace the PTS.

1-1

History of Flight
From prehistoric times, humans have watched the flight of
birds, and longed to imitate them, but lacked the power to do
so. Logic dictated that if the small muscles of birds can lift
them into the air and sustain them, then the larger muscles
of humans should be able to duplicate the feat. No one knew
about the intricate mesh of muscles, sinew, heart, breathing
system, and devices not unlike wing flaps, variable-camber
and spoilers of the modern airplane that enabled a bird to
fly. Still, thousands of years and countless lives were lost in
attempts to fly like birds.
The identity of the first "bird-men" who fitted themselves
with wings and leapt off of cliffs in an effort to fly are lost in
time, but each failure gave those who wished to fly questions
that needed to be answered. Where had the wing flappers
gone wrong? Philosophers, scientists, and inventors offered
solutions, but no one could add wings to the human body
and soar like a bird. During the 1500s, Leonardo da Vinci
filled pages of his notebooks with sketches of proposed
flying machines, but most of his ideas were flawed because
he clung to the idea of birdlike wings. [Figure 1-1] By
1655, mathematician, physicist, and inventor Robert Hooke
concluded that the human body does not possess the strength
to power artificial wings. He believed human flight would
require some form of artificial propulsion.

billowing heap of cloth capable of no more than a one-way,
downwind journey.
Balloons solved the problem of lift, but that was only one of
the problems of human flight. The ability to control speed and
direction eluded balloonists. The solution to that problem lay
in a child's toy familiar to the East for 2,000 years, but not
introduced to the West until the 13th century—the kite. The
kites used by the Chinese for aerial observation, to test winds
for sailing, as a signaling device, and as a toy, held many of
the answers to lifting a heavier-than-air device into the air.
One of the men who believed the study of kites unlocked
the secrets of winged flight was Sir George Cayley. Born
in England 10 years before the Mongolfier balloon flight,
Cayley spent his 84 years seeking to develop a heavier-than­
air vehicle supported by kite-shaped wings. [Figure 1-2] The
"Father of Aerial Navigation," Cayley discovered the basic
principles on which the modern science of aeronautics is
founded; built what is recognized as the first successful flying
model; and tested the first full-size man-carrying airplane.

The quest for human flight led some practitioners in another
direction. In 1783, the first manned hot air balloon, crafted
by Joseph and Etienne Montgolfier, flew for 23 minutes.
Ten days later, Professor Jacques Charles flew the first gas
balloon. A madness for balloon flight captivated the public's
imagination and for a time flying enthusiasts turned their
expertise to the promise of lighter-than-air flight. But for
all its majesty in the air, the balloon was little more than a

Figure 1-1. Leonardo da Vinci's ornithopter wings.

1-2

Figure 1-2. Glider from 1852 by Sir George Cayley, British aviator
(1773–1857).

For the half-century after Cayley's death, countless scientists,
flying enthusiasts, and inventors worked toward building
a powered flying machine. Men, such as William Samuel
Henson, who designed a huge monoplane that was propelled
by a steam engine housed inside the fuselage, and Otto
Lilienthal, who proved human flight in aircraft heavier than
air was practical, worked toward the dream of powered flight.
A dream turned into reality by Wilbur and Orville Wright at
Kitty Hawk, North Carolina, on December 17, 1903.
The bicycle-building Wright brothers of Dayton, Ohio, had
experimented for 4 years with kites, their own homemade
wind tunnel, and different engines to power their biplane. One
of their great achievements in flight was proving the value of
the scientific, rather than a build-it-and-see approach. Their
biplane, The Flyer, combined inspired design and engineering
with superior craftsmanship. [Figure 1-3] By the afternoon
of December 17th, the Wright brothers had flown a total of
98 seconds on four flights. The age of flight had arrived.

History of the Federal Aviation
Administration (FAA)
During the early years of manned flight, aviation was a
free for all because no government body was in place to
establish policies or regulate and enforce safety standards.
Individuals were free to conduct flights and operate aircraft
with no government oversight. Most of the early flights were
conducted for sport. Aviation was expensive and became the
playground of the wealthy. Since these early airplanes were
small, many people doubted their commercial value. One
group of individuals believed otherwise and they became
the genesis for modern airline travel.
P. E. Fansler, a Florida businessman living in St. Petersburg,
approached Tom Benoist of the Benoist Aircraft Company
in St. Louis, Missouri, about starting a flight route from St.

Figure 1-3. First flight by the Wright brothers.

Petersburg across the waterway to Tampa. Benoist suggested
using his "Safety First" airboat and the two men signed an
agreement for what would become the first scheduled airline
in the United States. The first aircraft was delivered to St.
Petersburg and made the first test flight on December 31,
1913. [Figure 1-4]
A public auction decided who would win the honor of
becoming the first paying airline customer. The former
mayor of St. Petersburg, A. C. Pheil, made the winning bid
of \$400.00, which secured his place in history as the first
paying airline passenger.
On January 1, 1914, the first scheduled airline flight was
conducted. The flight length was 21 miles and lasted 23
minutes due to a headwind. The return trip took 20 minutes.
The line, which was subsidized by Florida businessmen,
continued for 4 months and offered regular passage for \$5.00
per person or \$5.00 per 100 pounds of cargo. Shortly after the
opening of the line, Benoist added a new airboat that afforded
more protection from spray during takeoff and landing.
The routes were also extended to Manatee, Bradenton, and
Sarasota giving further credence to the idea of a profitable
commercial airline.
The St. Petersburg-Tampa Airboat Line continued throughout
the winter months with flights finally being suspended when
the winter tourist industry began to dry up. The airline
operated for only 4 months, but 1,205 passengers were
carried without injury. This experiment proved commercial
passenger airline travel was viable.
The advent of World War I offered the airplane a chance
to demonstrate its varied capabilities. It began the war as a
reconnaissance platform, but by 1918, airplanes were being

Figure 1-4. Benoist airboat.

1-3

mass produced to serve as fighters, bombers, trainers, as well
as reconnaissance platforms.

1
2
3

Aviation advocates continued to look for ways to use
airplanes. Airmail service was a popular idea, but the
war prevented the Postal Service from having access to
airplanes. The War Department and Postal Service reached an
agreement in 1918. The Army would use the mail service to
train its pilots in flying cross-country. The first airmail flight
was conducted on May 15, 1918, between New York and
Washington, DC. The flight was not considered spectacular;
the pilot became lost and landed at the wrong airfield. In
August of 1918, the United States Postal Service took control
of the airmail routes and brought the existing Army airmail
pilots and their planes into the program as postal employees.
Transcontinental Air Mail Route
Airmail routes continued to expand until the Transcontinental
Mail Route was inaugurated. [Figure 1-5] This route spanned
from San Francisco to New York for a total distance of 2,612
miles with 13 intermediate stops along the way. [Figure 1-6]
On May 20, 1926, Congress passed the Air Commerce Act,
which served as the cornerstone for aviation within the
United States. This legislation was supported by leaders in
the aviation industry who felt that the airplane could not
reach its full potential without assistance from the Federal
Government in improving safety.
The Air Commerce Act charged the Secretary of Commerce
with fostering air commerce, issuing and enforcing air traffic
rules, licensing pilots, certificating aircraft, establishing
airways, and operating and maintaining aids to air navigation.
The Department of Commerce created a new Aeronautics
Branch whose primary mission was to provide oversight for the
aviation industry. In addition, the Aeronautics Branch took over
the construction and operation of the nation's system of lighted
airways. The Postal Service, as part of the Transcontinental
Air Mail Route system, had initiated this system. The

4
5

6

New York
Bellefonte
Cleveland
Bryan
Chicago

13
14

12

7
8
9
10

11 10
9

11

Iowa City
Omaha
North Platte
Cheyenne
Rawlins

8

7

6

12
13
14
15

5

4

Rock Springs
Salt Lake City
Elko
Reno
San Francisco

3

2

1

15

Figure 1-6. The transcontinental airmail route ran from New York

to San Francisco.

Department of Commerce made significant advances in
aviation communications, including the introduction of radio
beacons as an effective means of navigation.
Built at intervals of approximately 10 miles apart, the
standard beacon tower was 51 feet high, and was topped
with a powerful rotating light. Below the rotating light, two
course lights pointed forward and back along the airway. The
course lights flashed a code to identify the beacon's number.
The tower usually stood in the center of a concrete arrow
70 feet long. A generator shed, where required, stood at the
"feather" end of the arrow. [Figure 1-7]
Federal Certification of Pilots and Mechanics
The Aeronautics Branch of the Department of Commerce
began pilot certification with the first license issued on April
6, 1927. The recipient was the Chief of the Aeronautics
Branch, William P. MacCracken, Jr. [Figure 1-8] (Orville
Wright, who was no longer an active flier, had declined the
honor.) MacCracken's license was the first issued to a pilot
by a civilian agency of the Federal Government. Some 3
months later, the Aeronautics Branch issued the first Federal
aircraft mechanic license.
Equally important for safety was the establishment of a
system of certification for aircraft. On March 29, 1927,
the Aeronautics Branch issued the first airworthiness
type certificate to the Buhl Airster CA-3, a three-place
open biplane.

Figure 1-5. The de Haviland DH-4 on the New York to San
Francisco inaugural route in 1921.

1-4

In 1934, to recognize the tremendous strides made in aviation
and to display the enhanced status within the department,
the Aeronautics Branch was renamed the Bureau of Air
Commerce. [Figure 1-9] Within this time frame, the Bureau
of Air Commerce brought together a group of airlines

-D KC
Figure
1-7.
A standard
airway
beacon
tower.
Figure
1-8.
Standard
airway
beacon
installation.

Figure 1-9. The third head of the Aeronautics Branch, Eugene
L. Vidal, is flanked by President Franklin D. Roosevelt (left) and
Secretary of Agriculture Henry A. Wallace (right). The photograph
was taken in 1933. During Vidal's tenure, the Aeronautics Branch
was renamed the Bureau of Air Commerce on July 1, 1934. The
new name more accurately reflected the status of the organization
within the Department of Commerce.

and encouraged them to form the first three Air Traffic
Control (ATC) facilities along the established air routes.
Then in 1936, the Bureau of Air Commerce took over the
responsibilities of operating the centers and continued to
advance the ATC facilities. ATC has come a long way from
the early controllers using maps, chalkboards, and performing
mental math calculations in order to separate aircraft along
flight routes.
The Civil Aeronautics Act of 1938
In 1938, the Civil Aeronautics Act transferred the civil
aviation responsibilities to a newly created, independent
body, named the Civil Aeronautics Authority (CAA). This
Act empowered the CAA to regulate airfares and establish
new routes for the airlines to service.

Figure 1-8. The first pilot license was issued to William P.

President Franklin Roosevelt split the CAA into two
agencies—the Civil Aeronautics Administration (CAA)
and the Civil Aeronautics Board (CAB). Both agencies
were still part of the Department of Commerce but the CAB
functioned independently of the Secretary of Commerce.
The role of the CAA was to facilitate ATC, certification of
airmen and aircraft, rule enforcement, and the development
of new airways. The CAB was charged with rule making to
enhance safety, accident investigation, and the economic
regulation of the airlines. Then in 1946, Congress gave the
CAA the responsibility of administering the Federal Aid

MacCracken, Jr.

1-5

Airport Program. This program was designed to promote
the establishment of civil airports throughout the country.
The Federal Aviation Act of 1958
By mid-century, air traffic had increased and jet aircraft had
been introduced into the civil aviation arena. A series of
mid-air collisions underlined the need for more regulation
of the aviation industry. Aircraft were not only increasing in
numbers, but were now streaking across the skies at much
higher speeds. The Federal Aviation Act of 1958 established
a new independent body that assumed the roles of the CAA
and transferred the rule making authority of the CAB to the
newly created Federal Aviation Agency (FAA). In addition,
the FAA was given complete control of the common civilmilitary system of air navigation and ATC. The man who
was given the honor of being the first Administrator of the
FAA was former Air Force General Elwood Richard "Pete"
Quesada. He served as the administrator from 1959–1961.
[Figure 1-10]
Department of Transportation (DOT)
On October 15, 1966, Congress established the Department
of Transportation (DOT), which was given oversight of the
transportation industry within the United States. The result
was a combination of both air and surface transportation. Its
mission was and is to serve the United States by ensuring a
fast, safe, efficient, accessible, and convenient transportation
system meeting vital national interests and enhancing the
quality of life of the American people, then, now, and into

the future. The DOT began operation on April 1, 1967. At
this same time, the Federal Aviation Agency was renamed
to the Federal Aviation Administration (FAA).
The role of the CAB was assumed by the newly created
National Transportation Safety Board (NTSB), which was
charged with the investigation of all transportation accidents
within the United States.
As aviation continued to grow, the FAA took on additional
duties and responsibilities. With the highjacking epidemic
of the 1960s, the FAA was responsible for increasing the
security duties of aviation both on the ground and in the air.
After September 11, 2001, the duties were transferred to
a newly created body called the Department of Homeland
Security (DHS).
With numerous aircraft flying in and out of larger cities, the
FAA began to concentrate on the environmental aspect of
aviation by establishing and regulating the noise standards
of aircraft. Additionally, in the 1960s and 1970s, the FAA
began to regulate high altitude (over 500 feet) kite and balloon
flying. In 1970, more duties were assumed by the FAA in the
addition of a new federal airport aid program and increased
responsibility for airport safety.
ATC Automation
By the mid-1970s, the FAA had achieved a semi-automated
ATC system based on a marriage of radar and computer
technology. By automating certain routine tasks, the system
allowed controllers to concentrate more efficiently on the
vital task of providing aircraft separation. Data appearing
directly on the controllers' scopes provided the identity,
altitude, and groundspeed of aircraft carrying radar beacons.
Despite its effectiveness, this system required enhancement
to keep pace with the increased air traffic of the late 1970s.
The increase was due in part to the competitive environment
created by the Airline Deregulation Act of 1978. This law
phased out CAB's economic regulation of the airlines, and
CAB ceased to exist at the end of 1984.
To meet the challenge of traffic growth, the FAA unveiled
the National Airspace System (NAS) Plan in January
1982. The new plan called for more advanced systems
for en route and terminal ATC, modernized flight service
stations, and improvements in ground-to-air surveillance
and communication.

Figure 1-10. First Administrator of the FAA was General Elwood

Richard "Pete" Quesada, 1959–1961.

1-6

The Professional Air Traffic Controllers
Organization (PATCO) Strike
While preparing the NAS Plan, the FAA faced a strike
by key members of its workforce. An earlier period of
discord between management and the Professional Air

Traffic Controllers Organization (PATCO) culminated in a
1970 "sickout" by 3,000 controllers. Although controllers
subsequently gained additional wage and retirement
benefits, another period of tension led to an illegal strike in
August 1981. The government dismissed over 11,000 strike
participants and decertified PATCO. By the spring of 1984,
the FAA ended the last of the special restrictions imposed to
keep the airspace system operating safely during the strike.
The Airline Deregulation Act of 1978
Until 1978, the CAB regulated many areas of commercial
aviation such as fares, routes, and schedules. The Airline
Deregulation Act of 1978, however, removed many of
these controls, thus changing the face of civil aviation in the
United States. After deregulation, unfettered free competition
ushered in a new era in passenger air travel.
The CAB had three main functions: to award routes to
airlines, to limit the entry of air carriers into new markets,
and to regulate fares for passengers. Much of the established
practices of commercial passenger travel within the United
States went back to the policies of Walter Folger Brown, the
United States Postmaster General during the administration
of President Herbert Hoover. Brown had changed the mail
payments system to encourage the manufacture of passenger
aircraft instead of mail-carrying aircraft. His influence
was crucial in awarding contracts and helped create four
major domestic airlines: United, American, Eastern, and
Transcontinental and Western Air (TWA). Similarly,
Brown had also helped give Pan American a monopoly on
international routes.
The push to deregulate, or at least to reform the existing laws
governing passenger carriers, was accelerated by President
Jimmy Carter, who appointed economist and former
professor Alfred Kahn, a vocal supporter of deregulation, to
head the CAB. A second force to deregulate emerged from
abroad. In 1977, Freddie Laker, a British entrepreneur who
owned Laker Airways, created the Skytrain service, which
offered extraordinarily cheap fares for transatlantic flights.
Laker's offerings coincided with a boom in low-cost domestic
flights as the CAB eased some limitations on charter flights
(i.e., flights offered by companies that do not actually own
planes but leased them from the major airlines). The big air
carriers responded by proposing their own lower fares. For
example, American Airlines, the country's second largest
airline, obtained CAB approval for "SuperSaver" tickets.
All of these events proved to be favorable for large-scale
deregulation. In November 1977, Congress formally
deregulated air cargo. In late 1978, Congress passed the
Airline Deregulation Act of 1978, legislation that had been
principally authored by Senators Edward Kennedy and

Howard Cannon. [Figure 1-11] There was stiff opposition to
the bill—from the major airlines who feared free competition,
from labor unions who feared non-union employees, and
from safety advocates who feared that safety would be
sacrificed. Public support was, however, strong enough to
pass the act. The act appeased the major airlines by offering
generous subsidies and pleased workers by offering high
unemployment benefits if they lost their jobs as a result. The
most important effect of the act, whose laws were slowly
phased in, was on the passenger market. For the first time
in 40 years, airlines could enter the market or (from 1981)
expand their routes as they saw fit. Airlines (from 1982)
also had full freedom to set their fares. In 1984, the CAB
was finally abolished since its primary duty of regulating the
airline industry was no longer necessary.

The Role of the FAA
The Code of Federal Regulations (CFR)
The FAA is empowered by regulations to promote aviation
safety and establish safety standards for civil aviation. The
FAA achieves these objectives under the Code of Federal
Regulations (CFR), which is the codification of the general
and permanent rules published by the executive departments
and agencies of the United States Government. The
regulations are divided into 50 different codes, called Titles,
that represent broad areas subject to Federal regulation.
FAA regulations are listed under Title 14, "Aeronautics and
Space," which encompasses all aspects of civil aviation from
how to earn a pilot's certificate to maintenance of an aircraft.
Title 14 CFR Chapter 1, Federal Aviation Administration,
is broken down into subchapters A through N as illustrated
in Figure 1-12.
For the pilot, certain parts of 14 CFR are more relevant
than others. During flight training, it is helpful for the pilot
to become familiar with the parts and subparts that relate

Figure 1-11. President Jimmy Carter signs the Airline Deregulation

Act in late 1978.

1-7

Code of Federal Regulations
Title
Title 14
Aeronautics
and Space

Volume
1

Chapter
I

2

3

4

II

III

5

V
VI

Subchapters
A
B
C
D
E
F
G
H
I
J
K
L–M
N
A
B
C
D
E
F
A
B
C

Definitions and Abbreviations
Procedural Rules
Aircraft
Airmen
Airspace
Air Traffic and General Rules
Air Carriers and Operators for Compensation or Hire: Certification and
Operations
Schools and Other Certified Agencies
Airports
Navigational Facilities
Administrative Regulations
Reserved
War Risk Insurance
Economic Regulations
Procedural Regulations
Reserved
Special Regulations
Organization
Policy Statements
General
Procedure
Licensing

A
B

Office of Management and Budget
Air Transportation Stabilization Board

Figure 1-12. Overview of 14 CFR, available online free from the FAA and for purchase through commercial sources.

to flight training and pilot certification. For instance, 14
CFR part 61 pertains to the certification of pilots, flight
instructors, and ground instructors. It also defines the
eligibility, aeronautical knowledge, and flight proficiency,
as well as training and testing requirements for each type of
pilot certificate issued. 14 CFR part 91 provides guidance in
the areas of general flight rules, visual flight rules (VFR), and
instrument flight rules (IFR), while 14 CFR part 43 covers
aircraft maintenance, preventive maintenance, rebuilding,
and alterations.
Primary Locations of the FAA
The FAA headquarters are in Washington, DC, and there are
nine regional offices strategically located across the United
States. The agency's two largest field facilities are the Mike
Monroney Aeronautical Center (MMAC) in Oklahoma
City, Oklahoma, and the William J. Hughes Technical
Center (WJHTC) in Atlantic City, New Jersey. Home to
FAA training and logistics services, the MMAC provides
a number of aviation safety-related and business support
services. The WJHTC is the premier aviation research and
development and test and evaluation facility in the country.
The center's programs include testing and evaluation in ATC,
communication, navigation, airports, aircraft safety, and
security. Furthermore, the WJHTC is active in long-range
1-8

development of innovative aviation systems and concepts,
development of new ATC equipment and software, and
modification of existing systems and procedures.

Field Offices
Flight Standards Service
Within the FAA, the Flight Standards Service promotes safe
air transportation by setting the standards for certification
and oversight of airmen, air operators, air agencies, and
designees. It also promotes safety of flight of civil aircraft
and air commerce by:


Accomplishing certification, inspection, surveillance,
investigation, and enforcement.



Setting regulations and standards.



Managing the system for registration of civil aircraft
and all airmen records.

The focus of interaction between Flight Standards Service
and the aviation community/general public is the Flight
Standards District Office (FSDO).

Flight Standards District Office (FSDO)
The FAA has approximately 80 FSDOs. [Figure 1-13] These
offices provide information and services for the aviation
community. FSDO phone numbers are listed in the telephone
directory under Government Offices, DOT, FAA. Another
convenient method of finding a local office is to use the
FSDO locator available at: \url{faa.gov/about/office_org/
field_offices/fsdo}.
In addition to accident investigation and the enforcement of
aviation regulations, the FSDO is also responsible for the
certification and surveillance of air carriers, air operators,
flight schools/training centers, and airmen including pilots
and flight instructors. Each FSDO is staffed by Aviation
Safety Inspectors (ASIs) who play a key role in making the
nation's aviation system safe.
Aviation Safety Inspector (ASI)
The ASIs administer and enforce safety regulations and
standards for the production, operation, maintenance, and/
or modification of aircraft used in civil aviation. They also
specialize in conducting inspections of various aspects of the
aviation system, such as aircraft and parts manufacturing,
aircraft operation, aircraft airworthiness, and cabin safety.
ASIs must complete a training program at the FAA Academy
in Oklahoma City, Oklahoma, which includes airman
evaluation and pilot testing techniques and procedures. ASIs
also receive extensive on-the-job training and recurrent
training on a regular basis. The FAA has approximately
3,700 inspectors located in its FSDO offices. All questions
concerning pilot certification (and/or requests for other
aviation information or services) should be directed to the
local FSDO.
FAA Safety Team (FAASTeam)
The FAA is dedicated to improving the safety of United
States civilian aviation by conveying safety principles and
practices through training, outreach, and education. The FAA

Safety Team (FAASTeam) exemplifies this commitment.
The FAASTeam has replaced the Aviation Safety Program
(ASP), whose education of airmen on all types of safety
subjects successfully reduced accidents. Its success led to
its demise because the easy-to-fix accident causes have been
addressed. To take aviation safety one step further, Flight
Standards Service created the FAASTeam, which is devoted
to reducing aircraft accidents by using a coordinated effort
to focus resources on elusive accident causes.
Each of the FAA's nine regions has a Regional FAASTeam
Office dedicated to this new safety program and managed by
the Regional FAASTeam Manager (RFM). The FAASTeam
is "teaming" up with individuals and the aviation industry
to create a unified effort against accidents and tip the safety
culture in the right direction. To learn more about this effort
to improve aviation safety, to take a course at their online
learning center, or to join the FAASTeam, visit their website
at \url{faasafety.gov}.
Obtaining Assistance from the FAA
Information can be obtained from the FAA by phone,
Internet/e-mail, or mail. To talk to the FAA toll-free 24
hours a day, call 1-866-TELL-FAA (1-866-835-5322). To
visit the FAA's website, go to \url{faa.gov}. Individuals can
also e-mail an FAA representative at a local FSDO office by
accessing the staff e-mail address available via the "Contact
FAA" link at the bottom of the FAA home page. Letters can
be sent to:
Federal Aviation Administration
800 Independence Ave, SW
Washington, DC 20591
FAA Reference Material
The FAA provides a variety of important reference material
for the student, as well as the advanced civil aviation pilot.
In addition to the regulations provided online by the FAA,
several other publications are available to the user. Almost
all reference material is available online at \url{faa.gov} in
downloadable format. Commercial aviation publishers also
provide published and online reference material to further
aid the aviation pilot.

Aeronautical Information Manual (AIM)

Figure 1-13. Atlanta Flight Standards District Office (FSDO).

The Aeronautical Information Manual (AIM) is the official
guide to basic flight information and ATC procedures for the
aviation community flying in the NAS of the United States.
[Figure 1-14] An international version, containing parallel
information as well as specific information on international
airports, is also available. The AIM also contains information
of interest to pilots, such as health and medical facts, flight

1-9

Aeronautical Information Manual (AIM)
The Aeronautical Information Manual is designed to provide
the aviation community with basic flight information and
ATC procedures for use in the NAS of the United States. It
also contains the fundamentals required in order to fly in the
United States NAS, including items of interest to pilots
concerning health/medical facts, factors affecting flight
safety, etc.
Aircraft Flying Handbooks (by category)
The Aircraft Flying Handbooks are designed as technical
manuals to introduce basic pilot skills and knowledge that
are essential for piloting aircraft. They provide information
on transition to other aircraft and the operation of various
aircraft systems.
Aviation Instructor's Handbook

Figure 1-14. Aeronautical Information Manual.

safety, a pilot/controller glossary of terms used in the
system, and information on safety, accidents, and reporting
of hazards. This manual is offered for sale on a subscription
basis or is available online at: http://bookstore.gpo.gov.
Order forms are provided at the beginning of the manual or
online and should be sent to the Superintendent of Documents,
United States Government Printing Office (GPO). The AIM
is complemented by other operational publications that are
available via separate subscriptions or online.

The Aviation Instructor's Handbook provides the foundation
for beginning instructors to understand and apply the
fundamentals of instructing. This handbook also provides
aviation instructors with up-to-date information on learning
and teaching, and how to relate this information to the task
of conveying aeronautical knowledge and skills to students.
Experienced aviation instructors also find the new and
updated information useful for improving their effectiveness
in training activities.
Instrument Flying Handbook
The Instrument Flying Handbook is designed for use by
instrument flight instructors and pilots preparing for
instrument rating tests. Instructors find this handbook a
valuable training aid as it includes basic reference material
for knowledge testing and instrument flight training.
Instrument Procedures Handbook
The Instrument Procedures Handbook is designed as a
technical reference for professional pilots who operate
under IFR in the NAS and expands on information contained
in the Instrument Flying Handbook.

Handbooks
Handbooks are developed to provide specific information
about a particular topic that enhances training or understanding.
The FAA publishes a variety of handbooks that generally fall
into three categories: aircraft, aviation, and examiners and
inspectors. [Figure 1-15] These handbooks can be purchased
from the Superintendent of Documents or downloaded at \url{
faa.gov/regulations_policies}. Aviation handbooks are also
published by various commercial aviation companies. Aircraft
flight manuals commonly called Pilot Operating Handbooks
(POH) are documents developed by the airplane manufacturer,
approved by the FAA, and are specific to a particular make
and model aircraft by serial number. This subject is covered
in greater detail in Chapter 8, "Flight Manuals and Other
Documents," of this handbook. [Figure 1-16]

Advisory Circulars (ACs)
An AC is an informational document that the FAA wants to
distribute to the aviation community. This can be in the form

1-10

Figure 1-15. A sample of handbooks available to the public. Most
can be downloaded free of charge from the FAA website.

of a text book used in a classroom or a one page document.
Some ACs are free while others cost money. They are to
be used for information only and are not regulations. The
FAA website \url{faa.gov/regulations_policies/advisory_
circulars/} provides a database that is a searchable repository
of all aviation safety ACs. All ACs, current and historical,
are provided and can be viewed as a portable document
format (PDF) copy.
ACs provide a single, uniform, agency-wide system that the
FAA uses to deliver advisory material to FAA customers,
industry, the aviation community, and the public. An AC
may be needed to:


Provide an acceptable, clearly understood method for
complying with a regulation

Figure 1-16. Pilot Operating Handbooks from manufacturers.



Standardize implementation of a regulation or
harmonize implementation for the international
aviation community



Resolve a general misunderstanding of a regulation



Respond to a request from some government entity,
such as General Accounting Office, NTSB, or the
Office of the Inspector General



Help the industry and FAA effectively implement a
regulation



Explain requirements and limits of an FAA grant
program



Expand on standards needed to promote aviation
safety, including the safe operation of airports

There are three parts to an AC number, as in 25-42C. The
first part of the number identifies the subject matter area
of the AC and corresponds to the appropriate 14 CFR part.
For example, an AC on "Certification: Pilots and Flight and
Ground Instructors" is numbered as AC 61-65E. Since ACs
are numbered sequentially within each subject area, the
second part of the number beginning with the dash identifies
this sequence. The third part of the number is a letter assigned
by the originating office and shows the revision sequence if
an AC is revised. The first version of an AC does not have
a revision letter. In Figure 1-17, this is the fifth revision, as
designated by the "E."

Flight Publications
The FAA, in concert with other government agencies,
orchestrates the publication and changes to publications
that are key to safe flight. Figure 1-18 illustrates some
publications a pilot may use.

Figure 1-17. Example of an Advisory Circular in its fifth revision.

1-11

NEVADA

245

ALAMO LANDING FLD
(L92)

2W
UTC 8( 7DT)
N37 °21.75 W115°11.67
LAS VEGAS
3719
NOTAM FILE RNO

RWY 14–32:5000X120 (DIRT)

RWY 14:Brush.
RWY 32:Berm.

RWY 15–33:2500X70 (DIRT)

RWY 15:Berm.
RWY 33:Berm.

AIRPORT REMARKS:

Unattended. Uncontrolled vehicle access. No line of sight between rwy ends. Rwys 15–33 and Rwy
14–32 livestock in vicinity of rwys.
COMMUNICATIONS: CTAF
122.9

AUSTIN

(TMT) 4 SW UTC 8( 7DT) N39 °28.08 W117°11.72
LAS VEGAS
5735
B NOTAM FILE RNO
H–3C, L–9B
RWY 18–36:H6000X75 (ASPH) S–30 MIRL
RWY 36:REIL. PAPI(P2L)—GA 3.0 ° TCH 40 . Fence.

RWY 18:REIL. PAPI(P2L)—GA 3.0 ° TCH 40 .
AIRPORT REMARKS:

Unattended. Military acft opr in vicinity of arpt. ACTIVATE MIRL Rwy 18–36, PAPI Rwys 18 and 36,
REIL Rwy 18 and 36—CTAF.
WEATHER DATA SOURCES:
AWOS–3PT 132.925 (775) 964–1144.
COMMUNICATIONS: CTAF
122.9
RADIO AIDS TO NAVIGATION:
NOTAM FILE RNO.

019° 66.7 NM to fld. 7860/17E.
MINA (H) VORTAC
115.1	 MVA Chan 98 N38 °33.92
 W118°01.97
HIWAS.

BATTLE MOUNTAIN
(BAM)

3 SE
UTC 8( 7DT)
N40 °35.94 W116°52.46
4536 B S4
FUEL 100LL, JET A NOTAM FILE RNO
RWY 12–30:H7302X150 (ASPH) S–30, D–104, 2S–132 MIRL
RWY 03–21:H7299X150 (ASPH) S–30, D–125, 2S–159 MIRL
RWY 03:VASI(V2R)—GA 3.0 ° TCH 26 .
RWY 21:PAPI(P4L)—GA 3.0 ° TCH 45 .
AIRPORT REMARKS:
Attended Oct–May 1500–0100Z ‡, Jun–Sep

1500–0200Z‡. After hrs call 775–635–2245. ACTIVATE MIRL Rwy

03–21 and Rwy 12–30, and perimeter lgts H1—CTAF.

WEATHER DATA SOURCES:
AWOS–3 119.45 (775) 635–8419.
COMMUNICATIONS: CTAF/UNICOM
122.8

MT LEWIS RCO
122.65 (RENO RADIO)

SALT LAKE CENTER APP/DEP CON

132.25
RADIO AIDS TO NAVIGATION:
NOTAM FILE RNO.
(H) VORTACW
112.2 BAM Chan 59 N40 °34.15

W116°55.34

033° 2.8 NM to fld. 4536/18E.
VORTAC unusable:

050°–060° byd 30 NM blo 12,000

115°–165° byd 15 NM blo 12,000

255°–290° byd 15 NM blo 12,000

DME unusable 246 °–255° byd 34 NM blo 14,000































SALT LAKE CITY
H–3C, L–9B, 11B
IAP



HELIPAD H1:
H60X60 (CONC)
HELIPAD H2:
H60X60 (CONC)
HELIPORT REMARKS:
Rwy H1 perimeter lights. ACTIVATE MIRL Rwy 03–21 and Rwy 12–30, and perimeter lgts
H1—CTAF.

,

Figure 1-18. From left to right, a sectional VFR chart, IFR chart, and chart supplement U.S. (formerly Airport/Facility Directory) with
a sample of a page from the supplement.

Pilot and Aeronautical Information



Notification of an operationally significant change in
volcanic ash or other dust contamination (an ASHTAM)



Software code risk announcements with associated
patches to reduce specific vulnerabilities

Notices to Airmen (NOTAMs)
Notices to Airmen, or NOTAMs, are time-critical aeronautical
information either temporary in nature or not sufficiently
known in advance to permit publication on aeronautical
charts or in other operational publications. The information
receives immediate dissemination via the National Notice to
Airmen (NOTAM) System. NOTAMs contain current notices
to airmen that are considered essential to the safety of flight,
as well as supplemental data affecting other operational
publications. There are many different reasons that NOTAMs
are issued. Following are some of those reasons:


Hazards, such as air shows, parachute jumps, kite
flying, and rocket launches



Flights by important people such as heads of state



Closed runways



Inoperable radio navigational aids



Military exercises with resulting airspace restrictions



Inoperable lights on tall obstructions



Temporary erection of obstacles near airfields



Passage of flocks of birds through airspace (a NOTAM
in this category is known as a BIRDTAM)



Notifications of runway/taxiway/apron status with
respect to snow, ice, and standing water (a SNOWTAM)

1-12

NOTAM information is generally classified into four
categories: NOTAM (D) or NOTAMs that receive distant
dissemination, distant and Flight Data Center (FDC)
NOTAMs, Pointer NOTAMs, and Military NOTAMs
pertaining to military airports or NAVAIDs that are part of the
NAS. NOTAMs are available through Flight Service Station
(FSS), Direct User Access Terminal Service (DUATS),
private vendors, and many online websites.
NOTAM (D) Information
NOTAM (D) information is disseminated for all navigational
facilities that are part of the NAS, and all public use airports,
seaplane bases, and heliports listed in the Chart Supplement
U.S. (formerly Airport/Facility Directory). NOTAM (D)
information now includes such data as taxiway closures,
personnel and equipment near or crossing runways, and
airport lighting aids that do not affect instrument approach
criteria, such as visual approach slope indicator (VASI).
All D NOTAMs are required to have one of the following
keywords as the first part of the text: RWY, TWY, RAMP,
APRON, AD, OBST, NAV, COM, SVC, AIRSPACE, (U),
or (O). [Figure 1-19]

FDC NOTAMs
FDC NOTAMs are issued by the National Flight Data
Center and contain information that is regulatory in nature
pertaining to flight including, but not limited to, changes
to charts, procedures, and airspace usage. FDC NOTAMs
refer to information that is regulatory in nature and includes
the following:


Flight restrictions in the proximity of the President
and other parties



14 CFR part 139 certificated airport condition changes



Snow conditions affecting glide slope operation



Air defense emergencies



Emergency flight rules

Interim IFR flight procedures:



Substitute airway routes

1.	 Airway structure changes



Special data

2.	 Instrument approach procedure changes (excludes
Departure Procedures (DPs) and Standard
Terminal Arrivals (STARs)



U.S. Government charting corrections



Laser activity

3.	 Airspace changes in general
4.	 Special instrument approach procedure changes




Temporary flight restrictions (discussed in Chapter 15):

NOTAM Composition
NOTAMs contain the elements below from left to right in
the following order:

1.	 Disaster areas



An exclamation point (!)

2.	 Special events generating a high degree of interest



Accountability Location (the identifier of the
accountability location)

3.	 Hijacking
Keyword

Example

Meaning

RWY
TWY
RAMP

RWY 3/21 CLSD
TWY F LGTS OTS
RAMP TERMINAL EAST SIDE
CONSTRUCTION
APRON SW TWY C NEAR
HANGARS CLSD
AD ABN OTS
OBST TOWER 283 (245 AGL) 2.2
S LGTS OTS (ASR 1065881) TIL
0707272300

Runways 3 and 21 are closed to aircraft.
Taxiway F lights are out of service.
The ramp in front of the east side of the terminal has ongoing
construction.
The apron near the southwest taxiway C in front of the hangars
is closed.
Aerodromes: The airport beacon is out of service.
Obstruction: The lights are out of service on a tower that is 283 feet
above mean sea level (MSL) or 245 feet above ground level (AGL)
2.2 statute miles south of the field. The FCC antenna structure
registration (ASR) number is 1065881. The lights will be returned to
service 2300 UTC (Coordinated Universal Time) on July 27, 2007.
Navigation: The VOR located on this airport is out of service.
Communications: The Automatic Terminal information Service
(ATIS) is out of service.
Service: The control tower has new operating hours, 1215-0330
UTC Monday Thru Friday. 1430-2300 UTC on Saturday and
1600-0100 UTC on Sunday until 0100 on July 30, 2007.
Service: All fuel for this airport is unavailable until July 29, 2007,
at 1600 UTC.
Service: United States Customs service for this airport will not be
available until August 15, 2007, at 0800 UTC.
Airspace. There is an airshow being held at this airport with aircraft
flying 5,000 feet and below within a 5 nautical mile radius.
Avoidance is advised from 2000 UTC on July 15, 2007, until 2200
on July 15, 2007.
Unverified aeronautical information.
Other aeronautical information received from any authorized source
that may be beneficial to aircraft operations and does not meet
defined NOTAM criteria.

APRON
AD
OBST

NAV
COM

NAV VOR OTS
COM ATIS OTS

SVC

SVC TWR 1215-0330
MON -FRI/1430-2300 SAT/1600-0100
SUN TIL 0707300100
SVC FUEL UNAVBL TIL 0707291600
SVC CUSTOMS UNAVBL TIL 0708150800

AIRSPACE

U
O

AIRSPACE AIRSHOW ACFT
5000/BLW 5 NMR AIRPORT
AVOIDANCE ADZD WEF
0707152000-0707152200
ORT 6K8 (U) RWY ABANDONED VEHICLE
LOZ LOZ (O) CONTROLLED BURN OF
HOUSE 8 NE APCH END RWY 23 WEF
0710211300-0710211700

Figure 1-19. NOTAM (D) Information.

1-13



Affected Location (the identifier of the affected facility
or location)



KEYWORD (one of the following: RWY, TWY,
RAMP, APRON, AD, COM, NAV, SVC, OBST,
AIRSPACE, (U) and (O))



Surface Identification (optional—this shall be the
runway identification for runway related NOTAMs,
the taxiway identification for taxiway-related
NOTAMs, or the ramp/apron identification for ramp/
apron-related NOTAMs)



Condition (the condition being reported)



Time (identifies the effective time(s) of the NOTAM
condition)

Altitude and height are in feet mean sea level (MSL) up to
17,999; e.g., 275, 1225 (feet and MSL is not written), and in
flight levels (FL) for 18,000 and above; e.g., FL180, FL550.
When MSL is not known, above ground level (AGL) will be
written (304 AGL).
When time is expressed in a NOTAM, the day begins at 0000
and ends at 2359. Times used in the NOTAM system are
universal time coordinated (UTC) and shall be stated in 10
digits (year, month, day, hour, and minute). The following
are two examples of how the time would be presented:
!DCA LDN NAV VOR OTS WEF
0708051600-0708052359
!DCA LDN NAV VOR OTS WEF
0709050000-0709050400
NOTAM Dissemination and Availability
The system for disseminating aeronautical information is
made up of two subsystems: the Airmen's Information System
(AIS) and the NOTAM System. The AIS consists of charts and
publications and is disseminated by the following methods:
Aeronautical charts depicting permanent baseline data:


IFR Charts—Enroute High Altitude ConterminousU.S.,
Enroute Low Altitude Conterminous U.S., Alaska
Charts, and Pacific Charts



U.S. Terminal Procedures—Departure Procedures
(DPs), Standard Terminal Arrivals (STARs) and
Standard Instrument Approach Procedures (SIAPs)



VFR Charts—Sectional Aeronautical Charts, Terminal
Area Charts (TAC), and World Aeronautical Charts
(WAC)

1-14

Flight information publications outlining baseline data:


Notices to Airmen (NTAP)—Published by System
Operations Services, System Operations and Safety,
Publications, every 28 days)



Chart Supplement U.S. (formerly Airport/Facility
Directory)



Pacific Chart Supplement



Alaska Supplement



Alaska Terminal



Aeronautical Information Manual (AIM)

NOTAMs are available in printed form through subscription
from the Superintendent of Documents, from an FSS, or
online at PilotWeb (\url{pilotweb.nas.faa.gov}), which
provides access to current NOTAM information. Local
airport NOTAMs can be obtained online from various
websites. Some examples are \url{fltplan.com} and \url{
aopa.org/whatsnew/notams.html}. Most sites require a free
registration and acceptance of terms but offer pilots updated
NOTAMs and TFRs.

Safety Program Airmen Notification System (SPANS)
In 2004, the FAA launched the Safety Program Airmen
Notification System (SPANS), an online event notification
system that provides timely and easy-to-assess seminar
and event information notification for airmen. The SPANS
system is taking the place of the current paper-based mail
system. This provides better service to airmen while reducing
costs for the FAA. Anyone can search the SPANS system
and register for events. To read more about SPANS, visit
\url{faasafety.gov/spans}.

Aircraft Classifications and Ultralight
Vehicles
The FAA uses various ways to classify or group machines
operated or flown in the air. The most general grouping uses
the term aircraft. This term is in 14 CFR 1.1 and means a
device that is used or intended to be used for flight in the air.
Ultralight vehicle is another general term the FAA uses.
This term is defined in 14 CFR 103. As the term implies,
powered ultralight vehicles must weigh less than 254 pounds
empty weight and unpowered ultralight vehicles must
weigh less than 155 pounds. Rules for ultralight vehicles
are significantly different from rules for aircraft; ultralight
vehicle certification, registration, and operation rules are also
contained in 14 CFR 103.

The FAA differentiates aircraft by their characteristics and
physical properties. Key groupings defined in 14 CFR 1.1
include:


Airplane—an engine-driven fixed-wing aircraft
heavier than air, that is supported in flight by the
dynamic reaction of the air against its wings.



Glider—a heavier-than-air aircraft, that is supported
in flight by the dynamic reaction of the air against its
lifting surfaces and whose free flight does not depend
principally on an engine.











Size and weight are other methods used in 14 CFR 1.1 to
group aircraft:


Large aircraft—an aircraft of more than 12,500
pounds, maximum certificated takeoff weight.



Light-sport aircraft (LSA)—an aircraft, other than
a helicopter or powered-lift that, since its original
certification, has continued to meet the definition in
14 CFR 1.1. (LSA can include airplanes, airships,
balloons, gliders, gyro planes, powered parachutes,
and weight-shift-control.)



Small Aircraft—aircraft of 12,500 pounds or less,
maximum certificated takeoff weight.

Lighter-than-air aircraft—an aircraft that can rise and
remain suspended by using contained gas weighing
less than the air that is displaced by the gas.
-

Airship—an engine-driven lighter-than-air
aircraft that can be steered.

-

Balloon—a lighter-than-air aircraft that is not
engine driven, and that sustains flight through the
use of either gas buoyancy or an airborne heater.

Powered-lift—a heavier-than-air aircraft capable of
vertical takeoff, vertical landing, and low speed flight
that depends principally on engine-driven lift devices
or engine thrust for lift during these flight regimes and
on nonrotating airfoil(s) for lift during horizontal flight.

We also use broad classifications of aircraft with respect to
the certification of airmen or with respect to the certification
of the aircraft themselves. See the next section, Pilot
Certifications, and Chapter 3, for further discussion of
certification. These definitions are in 14 CFR 1.1:


Rocket—an aircraft propelled by ejected expanding
gases generated in the engine from self-contained
propellants and not dependent on the intake of outside
substances. It includes any part which becomes
separated during the operation.

2.	 As used with respect to the certification of
aircraft, means a grouping of aircraft based upon
intended use or operating limitations. Examples
include: transport, normal, utility, acrobatic,
limited, restricted, and provisional.
 	 Class
1. 	 As used with respect to the certification, ratings,
privileges, and limitations of airmen, means a
classification of aircraft within a category having
similar operating characteristics. Examples
Include: single engine; multiengine; land; water;
gyroplane, helicopter, airship, and free balloon;
and

Rotorcraft—a heavier-than-air aircraft that depends
principally for its support in flight on the lift generated
by one or more rotors.

-

Gyroplane—a rotorcraft whose rotors are not
engine-driven, except for Initial starting, but
are made to rotate by action of the air when
the rotorcraft Is moving; and whose means of
propulsion, consisting usually of conventional
propellers, is Independent of the rotor system.
Helicopter—a rotorcraft that, for its horizontal
motion, depends principally on its engine-driven
rotors.

Weight-shift-control—a powered aircraft with a framed
pivoting wing and a fuselage controllable only in pitch
and roll by the pilot's ability to change the aircraft's

Category
1. 	 As used with respect to the certification, ratings,
privileges, and limitations of airmen, means a
broad classification of aircraft. Examples include:
airplane; rotorcraft; glider; and lighter-than-air;
and

Powered parachute—a powered aircraft comprised of
a flexible or semi-rigid wing connected to a fuselage
so that the wing is not in position for flight until
the aircraft is in motion. The fuselage of a powered
parachute contains the aircraft engine, a seat for each
occupant and Is attached to the aircraft's landing gear.

-



center of gravity with respect to the wing. Flight control
of the aircraft depends on the wing's ability to flexibly
deform rather than the use of control surfaces.

2.	 As used with respect to the certification of
aircraft, means a broad grouping of aircraft having
similar characteristics of propulsion, flight, or
landing. Examples include: airplane, rotorcraft,
gilder, balloon, landplane, and seaplane.


Type
1. 	 As used with respect to the certification, ratings,
privileges, and limitations of airmen, means
1-15

a specific make and basic model of aircraft,
Including modifications thereto that do not
change its handling or flight characteristics.
Examples include: 737-700, G-IV, and 1900; and



Privileges—define where and when the pilot may fly,
with whom they may fly, the purpose of the flight, and
the type of aircraft they are allowed to fly.



Limitations—the FAA may impose limitations on a
pilot certificate if, during training or the practical test,
the pilot does not demonstrate all skills necessary to
exercise all privileges of a privilege level, category,
class, or type rating.

2.	 As used with respect to the certification of
aircraft, means those aircraft which are similar
in design. Examples include: 737-700 and 737­
700C; G-IV and G-IV-X; and 1900 and 1900C.
This system of definitions allows the FAA to group and
regulate aircraft to provide for their safe operation.

Pilot Certifications
The type of intended flying influences what type of pilot's
certificate is required. Eligibility, training, experience,
and testing requirements differ depending on the type of
certificates sought. [Figure 1-20] Each type of pilot's
certificate has privileges and limitations that are inherent
within the certificate itself. However, other privileges and
limitations may be applicable based on the aircraft type,
operation being conducted, and the type of certificate.
For example, a certain certificate may have privileges and
limitations under 14 CFR part 61 and part 91.

Endorsements, a form of authorization, are written to establish
that the certificate holder has received training in specific skill
areas. Endorsements are written and signed by an authorized
individual, usually a certificated flight instructor (CFI), and
are based on aircraft classification. [Figure 1-21]
Sport Pilot
To become a sport pilot, the student pilot is required to have
flown, at a minimum, the following hours depending upon
the aircraft:


Airplane: 20 hours



Powered Parachute: 12 hours



Weight-Shift Control (Trikes): 20 hours



Glider: 10 hours



Rotorcraft (gyroplane only): 20 hours



Lighter-Than-Air: 20 hours (airship) or 7 hours
(balloon)

To earn a Sport Pilot Certificate, one must:


Be at least 16 years old to become a student sport pilot
(14 years old for gliders or balloons)



Be at least 17 years old to test for a sport pilot
certificate (16 years old for gliders or balloons)



Be able to read, write, and understand the English
language



Hold a current and valid driver's license as evidence
of medical eligibility

When operating as a sport pilot, some of the following
privileges and limitations may apply.

Privileges:

Figure 1-20. Front side (top) and back side (bottom) of an airman

certificate issued by the FAA.

1-16



Operate as pilot in command (PIC) of a light-sport
aircraft



Carry a passenger and share expenses (fuel, oil, airport
expenses, and aircraft rental)



Fly during the daytime using VFR, a minimum of
3 statute miles visibility and visual contact with the
ground are required

Recreational pilot to conduct solo flights for the purpose of obtaining an additional certificate or rating while under
the supervision of an authorized flight instructor: section 61.101(i).
I certify that (First name, MI, Last name) has received the required training of section 61.87 in a (make and model
aircraft). I have determined he/she is prepared to conduct a solo flight on (date) under the following conditions: (List
all conditions which require endorsement, e.g., flight which requires communication with air traffic control, flight in an
aircraft for which the pilot does not hold a category/class rating, etc.).

Figure 1-21. Example endorsement for a recreational pilot to conduct solo flights for the purpose of determining an additional certificate

or rating.

Limitations:


Prohibited from flying in Class A airspace



Prohibited from flying in Class B, C, or D airspace
until you receive training and a logbook endorsement
from an instructor



No flights outside the United States without prior
permission from the foreign aviation authority



May not tow any object



No flights while carrying a passenger or property for
compensation or hire



Prohibited from flying in furtherance of a business

The sport pilot certificate does not list aircraft category
and class ratings. After successfully passing the practical
test for a sport pilot certificate, regardless of the light-sport
aircraft privileges you seek, the FAA will issue you a sport
pilot certificate without any category and class ratings. The
Instructor will provide you with the appropriate logbook
endorsement for the category and class of aircraft in which
you are authorized to act as pilot in command.
Recreational Pilot
To become a recreational pilot, one must:


Be at least 17 years old



Be able to read, write, speak, and understand the
English language



Pass the required knowledge test



Meet the aeronautical experience requirements in
either a single-engine airplane, a helicopter, or a
gyroplane.



Obtain a logbook endorsement from an instructor



Pass the required practical test



Obtain a third-class medical certificate issued under
14 CFR part 67

As a recreational pilot, cross-country flight is limited to a 50
NM range from the departure airport but is permitted with
additional training per 14 CFR part 61, section 61.101(c).
Additionally, recreational pilots are restricted from flying
at night and flying in airspace where communications with
ATC are required.
The minimum aeronautical experience requirements for a
recreational pilot license involve:


30 hours of flight time including at least:


15 hours of dual instruction



2 hours of en route training



3 hours in preparation for the practical test



3 hours of solo flight

When operating as a recreational pilot, some of the following
privileges and limitations may apply.

Privileges:


Carry no more than one passenger;



Not pay less than the pro rata share of the operating
expenses of a flight with a passenger, provided the
expenses involve only fuel, oil, airport expenses, or
aircraft rental fees

Limitations:


A recreational pilot may not act as PIC of an aircraft
that is certificated for more than four occupants or has
more than one powerplant.

Private Pilot
A private pilot is one who flies for pleasure or personal
business without accepting compensation for flying except
in some very limited, specific circumstances. The Private
Pilot Certificate is the certificate held by the majority of

1-17

active pilots. It allows command of any aircraft (subject
to appropriate ratings) for any noncommercial purpose
and gives almost unlimited authority to fly under VFR.
Passengers may be carried and flight in furtherance of a
business is permitted; however, a private pilot may not be
compensated in any way for services as a pilot, although
passengers can pay a pro rata share of flight expenses, such
as fuel or rental costs. If training under 14 CFR part 61,
experience requirements include at least 40 hours of piloting
time, including 20 hours of flight with an instructor and 10
hours of solo flight. [Figure 1-22]
Commercial Pilot
A commercial pilot may be compensated for flying. Training
for the certificate focuses on a better understanding of
aircraft systems and a higher standard of airmanship. The
Commercial Pilot Certificate itself does not allow a pilot
to fly in instrument meteorological conditions (IMC), and
commercial pilots without an instrument rating are restricted
to daytime flight within 50 NM when flying for hire.
A commercial airplane pilot must be able to operate
a complex airplane, as a specific number of hours of
complex (or turbine-powered) aircraft time are among
the prerequisites, and at least a portion of the practical
examination is performed in a complex aircraft. A complex
aircraft must have retractable landing gear, movable flaps,
and a controllable-pitch propeller. See 14 CFR part 61,
section 61.31(e) for additional information. [Figure 1-23]

Figure 1-23. A complex aircraft.

and understand the English language, and be "of good moral
standing." A pilot may obtain an ATP certificate with restricted
privileges enabling him/her to serve as an SIC in scheduled
airline operations. The minimum pilot experience is reduced
based upon specific academic and flight training experience.
The minimum age to be eligible is 21 years. [Figure 1-24]

Selecting a Flight School

Airline Transport Pilot
The airline transport pilot (ATP) is tested to the highest level
of piloting ability. The ATP certificate is a prerequisite for
serving as a PIC and second in command (SIC) of scheduled
airline operations. It is also a prerequisite for serving as a PIC
in select charter and fractional operations. The minimum pilot
experience is 1,500 hours of flight time. In addition, the pilot
must be at least 23 years of age, be able to read, write, speak,

Selecting a flight school is an important consideration in
the flight training process. FAA-approved training centers,
FAA-approved pilot schools, noncertificated flying schools,
and independent flight instructors conduct flight training in
the United States. All flight training is conducted under the
auspices of the FAA following the regulations outlined in
14 CFR parts 142, 141, or 61. Training centers, also referred
to as flight academies, operate under 14 CFR part 142 and
are certificated by the FAA. Application for certification
is voluntary and the training center must meet stringent
requirements for personnel, equipment, maintenance,
facilities, and must teach a curriculum approved by the
FAA. Training centers typically utilize a number of flight
simulation training devices as part of its curricula. Flight
training conducted at a training center is primarily done
under contract to airlines and other commercial operators
in transport or turbine aircraft, however many also provide

Figure 1-22. A typical aircraft a private pilot might fly.

Figure 1-24. Type of aircraft flown by an airline transport pilot.

1-18

flight training for the private pilot certificate, commercial
pilot certificate, instrument rating, and ATP certificate.
Flight schools operating under 14 CFR part 141 are
certificated by the FAA. Application for certification is
voluntary and the school must meet stringent requirements
for personnel, equipment, maintenance, facilities, and must
teach an established curriculum, which includes a training
course outline (TCO) approved by the FAA. The certificated
schools may qualify for a ground school rating and a flight
school rating. In addition, the school may be authorized
to give its graduates practical (flight) tests and knowledge
(computer administered written) tests. The FAA Pilot School
Search database located at http://av-info.faa.gov/PilotSchool.
asp, lists certificated ground and flight schools and the pilot
training courses each school offers.
Enrollment in a 14 CFR part 141 flight school ensures
quality, continuity, and offers a structured approach to flight
training because these facilities must document the training
curriculum and have their flight courses approved by the
FAA. These strictures allow 14 CFR part 141 schools to
complete certificates and ratings in fewer flight hours, which
can mean a savings on the cost of flight training for the
student pilot. For example, the minimum requirement for a
Private Pilot Certificate is 35 hours in a part 141-certificated
school and 40 hours in a part 61-certificated school. (This
difference may be insignificant for a Private Pilot Certificate
because the national average indicates most pilots require 60
to 75 hours of flight training.)
Many excellent flight schools find it impractical to qualify
for the FAA part 141 certificates and are referred to as part
61 schools. 14 CFR part 61 outlines certificate and rating
requirements for pilot certification through noncertificated
schools and individual flight instructors. It also states what
knowledge-based training must be covered and how much
flight experience is required for each certificate and rating.
Flight schools and flight instructors who train must adhere
to the statutory requirements and train pilots to the standards
found in 14 CFR part 61.
One advantage of flight training under 14 CFR part 61 is its
flexibility. Flight lessons can be tailored to the individual
student, because 14 CFR part 61 dictates the required
minimum flight experience and knowledge-based training
necessary to gain a specific pilot's license, but it does not
stipulate how the training is to be organized. This flexibility
can also be a disadvantage because a flight instructor who
fails to organize the flight training can cost a student pilot
time and expense through repetitious training. One way for
a student pilot to avoid this problem is to ensure the flight
instructor has a well-documented training syllabus.

How To Find a Reputable Flight Program
To obtain information about pilot training, contact the local
FSDO, which maintains a current file on all schools within its
district. The choice of a flight school depends on what type of
certificate is sought, and whether an individual wishes to fly
as a sport pilot or wishes to pursue a career as a professional
pilot. Another consideration is the amount of time that can
be devoted to training. Ground and flight training should
be obtained as regularly and frequently as possible because
this assures maximum retention of instruction and the
achievement of requisite proficiency.
Do not make the determination based on financial concerns
alone, because the quality of training is very important.
Prior to making a final decision, visit the schools under
consideration and talk with management, instructors, and
students. Request a personal tour of the flight school facility.
Be inquisitive and proactive when searching for a flight
school, do some homework, and develop a checklist of
questions by talking to pilots and reading articles in flight
magazines. The checklist should include questions about
aircraft reliability and maintenance practices, and questions
for current students such as whether or not there is a safe,
clean aircraft available when they are scheduled to fly.
Questions for the training facility should be aimed at
determining if the instruction fits available personal time.
What are the school's operating hours? Does the facility have
dedicated classrooms available for ground training required
by the FAA? Is there an area available for preflight briefings,
postflight debriefings, and critiques? Are these rooms private
in nature in order to provide a nonthreatening environment
in which the instructor can explain the content and outcome
of the flight without making the student feel self-conscious?
Examine the facility before committing to any flight training.
Evaluate the answers on the checklist, and then take time to
think things over before making a decision. This proactive
approach to choosing a flight school will ensure a student
pilot contracts with a flight school or flight instructor best
suited to their individual needs.
How To Choose a Certificated Flight Instructor (CFI)
Whether an individual chooses to train under 14 CFR part
141 or part 61, the key to an effective flight program is the
quality of the ground and flight training received from the
CFI. The flight instructor assumes total responsibility for
training an individual to meet the standards required for
certification within an ever-changing operating environment.
A CFI should possess an understanding of the learning
process, knowledge of the fundamentals of teaching, and

1-19

the ability to communicate effectively with the student pilot.
During the certification process, a flight instructor applicant
is tested on the practical application of these skills in specific
teaching situations. The flight instructor is crucial to the
scenario-based training program endorsed by the FAA. He
or she is trained to function in the learning environment as an
advisor and guide for the learner. The duties, responsibilities,
and authority of the CFI include the following:


Orient the student to the scenario-based training
system



Help the student become a confident planner and
inflight manager of each flight and a critical evaluator
of their own performance



Help the student understand the knowledge
requirements present in real world applications



Diagnose learning difficulties and help the student
overcome them



Evaluate student progress and maintain appropriate
records



Provide continuous review of student learning

Should a student pilot find the selected CFI is not training in
a manner conducive for learning, or the student and CFI do
not have compatible schedules, the student pilot should find
another CFI. Choosing the right CFI is important because the
quality of instruction and the knowledge and skills acquired
from their flight instructor affect a student pilot's entire
flying career.

The Student Pilot
The first step in becoming a pilot is to select a type of aircraft.
FAA rules for obtaining a pilot's certificate differ depending
on the type of aircraft flown. Individuals can choose among
airplanes, gyroplanes, weight-shift, helicopters, powered
parachutes, gliders, balloons, or airships. A pilot does not
need a certificate to fly ultralight vehicles.
Basic Requirements
A student pilot is one who is being trained by an instructor
pilot for his or her first full certificate, and is permitted
to fly alone (solo) under specific, limited circumstances.
Before a student pilot may be endorsed to fly solo, that
student must have a Student Pilot Certificate. There are
multiple ways that an aspiring pilot can obtain their Student
Pilot Certificate. The application may be processed by an
FAA inspector or technician, an FAA-Designated Pilot
Examiner, a Certified Flight Instructor (CFI), or an Airman
Certification Representative (ACR). If the application is
completed electronically, the authorized person will submit
the application to the FAA's Airman Certification Branch
(AFS-760) in Oklahoma City, OK, via the Integrated
1-20

Airman Certification and Rating Application (IACRA). If
the application is completed on paper, it must be sent to
the local Flight Standards District Office (FSDO), who will
forward it to AFS-760. Once the application is processed, the
applicant will receive the Student Pilot Certificate by mail at
the address provided on the application.
The aforementioned process will become effective on April
1, 2016. The new certificate will be printed on a plastic card,
which will replace the paper certificate that was issued in the
past. The plastic card certificate will not have an expiration
date. Paper certificates issued prior to the new process will
still expire according to the date on the certificate; however,
under the new process, paper certificates cannot be renewed.
Once the paper certificate expires, the Student Pilot must
submit a new application under the new process. Another
significant change in the new process is that flight instructors
will now make endorsements for solo privileges in the
Student Pilot's logbook, instead of endorsing the Student
Pilot Certificate.
To be eligible for a Student Pilot Certificate, the applicant
must:


Be at least 16 years of age (14 years of age to pilot a
glider or balloon).



Be able to read, speak, write, and understand the
English language.

Medical Certification Requirements
The second step in becoming a pilot is to obtain a medical
certificate (if the choice of aircraft is an airplane, helicopter,
gyroplane, or an airship). (The FAA suggests the individual
get a medical certificate before beginning flight training to
avoid the expense of flight training that cannot be continued
due to a medical condition.) Balloon or glider pilots do not
need a medical certificate, but do need to write a statement
certifying that no medical defect exists that would prevent
them from piloting a balloon or glider. The new sport pilot
category does not require a medical examination; a driver's
license can be used as proof of medical competence.
Applicants who fail to meet certain requirements or who
have physical disabilities which might limit, but not
prevent, their acting as pilots, should contact the nearest
FAA office. Anyone requesting an FAA Medical Clearance,
Medical Certificate, or Student Pilot Medical Certificate can
electronically complete an application through the FAA's
MedXPress system available at https://medxpress.faa.gov/.
A medical certificate is obtained by passing a physical
examination administered by a doctor who is an FAAauthorized AME. There are approximately 6,000
FAA-authorized AMEs in the nation. To find an AME near

you, go to the FAA's AME locator at \url{faa.gov/pilots/
amelocator/}. Medical certificates are designated as first class,
second class, or third class. Generally, first class is designed
for the airline transport pilot; second class for the commercial
pilot; and third class for the student, recreational, and private
pilot. A Student Pilot Certificate can be processed by an FAA
inspector or technician, an FAA Designated pilot examiner
(DPE), an Airman Certification Representative (ACR), or a
Certified Flight Instructor (CFI). This certificate allows an
individual who is being trained by a flight instructor to fly
alone (solo) under specific, limited circumstances and must
be carried with the student pilot while exercising solo flight
privileges. The Student Pilot Certificate is only required
when exercising solo flight privileges. The new plastic
student certificate does not have an expiration date. For
airmen who were issued a paper certificate, that certificate
will remain valid until its expiration date. A paper certificate
cannot be renewed. When the paper certificate expires, a new
application must be completed via the IACRA system, and
a new plastic certificate will be issued.

Student Pilot Solo Requirements
Once a student has accrued sufficient training and experience,
a CFI can endorse the student's logbook to authorize limited
solo flight in a specific type (make and model) of aircraft.
A student pilot may not carry passengers, fly in furtherance
of a business, or operate an aircraft outside of the various
endorsements provided by the flight instructor. There is no
minimum aeronautical knowledge or experience requirement
for the issuance of a Student Pilot Certificate, however, the
applicant must be at least 16 years of age (14 years of age for
a pilot for glider or balloon), and they must be able to read,
speak, write and understand the English language. There are,
however, minimum aeronautical knowledge and experience
requirements for student pilots to solo.

Becoming a Pilot
The course of instruction a student pilot follows depends on
the type of certificate sought. It should include the ground and
flight training necessary to acquire the knowledge and skills
required to safely and efficiently function as a certificated
pilot in the selected category and class of aircraft. The
specific knowledge and skill areas for each category and
class of aircraft are outlined in 14 CFR part 61. Eligibility,
aeronautical knowledge, proficiency, and aeronautical
requirements can be found in 14 CFR part 61.


Recreational Pilot, see subpart D



Private Pilot, see subpart E



Sport Pilot, see subpart J

of training and testing materials which are available in print
form from the Superintendent of Documents, GPO, and
online at the Regulatory Support Division: \url{faa.gov/
about/office_org/headquarters_offices/avs/offices/afs/afs600}.
The CFI may also use commercial publications as a source
of study materials, especially for aircraft categories where
government materials are limited. A student pilot should
follow the flight instructor's advice on what and when to
study. Planning a definite study program and following it as
closely as possible will help in scoring well on the knowledge
test. Haphazard or disorganized study habits usually result
in an unsatisfactory score.
In addition to learning aeronautical knowledge, such as the
principles of flight, a student pilot is also required to gain
skill in flight maneuvers. The selected category and class of
aircraft determines the type of flight skills and number of
flight hours to be obtained. There are four steps involved in
learning a flight maneuver:


The CFI introduces and demonstrates flight maneuver
to the student.



The CFI talks the student pilot through the maneuver.



The student pilot practices the maneuver under CFI
supervision.



The CFI authorizes the student pilot to practice the
maneuver solo.

Once the student pilot has shown proficiency in the required
knowledge areas, flight maneuvers, and accrued the required
amount of flight hours, the CFI endorses the student pilot
logbook, which allows the student pilot to take the written
and practical tests for pilot certification.

Knowledge and Skill Tests
Knowledge Tests
The knowledge test is the computer portion of the tests taken
to obtain pilot certification. The test contains questions of
the objective, multiple-choice type. This testing method
conserves the applicant's time, eliminates any element of
individual judgment in determining grades, and saves time
in scoring.
FAA Airman Knowledge Test Guides for every type of pilot
certificate address most questions you may have regarding
the knowledge test process. The guides are available online (free of charge) at http://\url{faa.gov/training_testing/
testing/test_guides/}.

The knowledge-based portion of training is obtained through
FAA handbooks such as this one, textbooks, and other sources
1-21

When To Take the Knowledge Test
The knowledge test is more meaningful to the applicant
and more likely to result in a satisfactory grade if it is taken
after beginning the flight portion of the training. Therefore,
the FAA recommends the knowledge test be taken after the
student pilot has completed a solo cross-country flight. The
operational knowledge gained by this experience can be used
to the student's advantage in the knowledge test. The student
pilot's CFI is the best person to determine when the applicant
is ready to take the knowledge test.
Practical Test
The FAA has developed PTS for FAA pilot certificates
and associated ratings. [Figure 1-25] In 2015, the FAA
began transitioning to the ACS approach. The ACS is
essentially an "enhanced" version of the PTS. It adds taskspecific knowledge and risk management elements to each
PTS Area of Operation and Task. The result is a holistic,
integrated presentation of specific knowledge, skills, and
risk management elements and performance metrics for each
Area of Operation and Task The ACS evaluation program
will eventually replace the PTS program for evaluating and
certifying pilots.
The practical tests are administered by FAA ASIs and DPEs.
Title 14 CFR part 61 specifies the areas of operation in which

knowledge and skill must be demonstrated by the applicant.
Since the FAA requires all practical tests be conducted in
accordance with the appropriate PTS and the policies set forth
in the Introduction section of the PTS book. The pilot applicant
should become familiar with this book during training.
The PTS book is a testing document and not intended to be
a training syllabus. An appropriately-rated flight instructor
is responsible for training the pilot applicant to acceptable
standards in all subject matter areas, procedures, and
maneuvers. Descriptions of tasks and information on how to
perform maneuvers and procedures are contained in reference
and teaching documents such as this handbook. A list of
reference documents is contained in the Introduction section
of each PTS book. Copies may obtained by:


Downloading from the FAA website at \url{faa.gov}



Purchasing print copies from the GPO, Pittsburgh,
Pennsylvania, or via their official online bookstore
at \url{access.gpo.gov}

The flight proficiency maneuvers listed in 14 CFR part 61
are the standard skill requirements for certification. They
are outlined in the PTS as "areas of operation." These are
phases of the practical test arranged in a logical sequence
within the standard. They begin with preflight preparation
and end with postflight procedures. Each area of operation

FAA-S-8081-12C
14B
A-S-8081-

FA

FAA-S-8

081-8B

Figure 1-25. Examples of Practical Test Standards.

1-22

contains "tasks," which are comprised of knowledge areas,
flight procedures, and/or flight maneuvers appropriate to the
area of operation. The candidate is required to demonstrate
knowledge and proficiency in all tasks for the original
issuance of all pilot certificates.

When To Take the Practical Test
14 CFR part 61 establishes the ground school and flight
experience requirements for the type of certification and
aircraft selected. However, the CFI best determines when an
applicant is qualified for the practical test. A practice practical
test is an important step in the flight training process.
The applicant will be asked to present the following
documentation:


FAA Form 8710-1 (8710.11 for sport pilot applicants),
Application for an Airman Certificate and/or Rating,
with the flight instructor's recommendation



An Airman Knowledge Test Report with a
satisfactory grade



A medical certificate (not required for glider or
balloon), a Student Pilot Certificate, and a pilot
logbook endorsed by a flight instructor for solo, solo
cross-country (airplane and rotorcraft), and for the
make and model aircraft to be used for the practical
test (driver's license or medical certificate for sport
pilot applicants)



The pilot log book records



A graduation certificate from an FAA-approved school
(if applicable)

The applicant must provide an airworthy aircraft with
equipment relevant to the areas of operation required for
the practical test. He or she will also be asked to produce
and explain the:


Aircraft's registration certificate



Aircraft's airworthiness certificate



Aircraft's operating limitations or FAA-approved
aircraft flight manual (if required)



Aircraft equipment list



Required weight and balance data



Maintenance records



Applicable airworthiness directives (ADs)

For a detailed explanation of the required pilot maneuvers and
performance standards, refer to the PTS pertaining to the type
of certification and aircraft selected. These standards may be
downloaded free of charge from the FAA at \url{faa.gov}. They
may also be purchased from the Superintendent of Documents

or GPO bookstores. Most airport fixed-base operators and flight
schools carry a variety of government publications and charts,
as well as commercially published materials.

Who Administers the FAA Practical Tests?
Due to the varied responsibilities of the FSDOs, practical tests
are usually given by DPEs. An applicant should schedule the
practical test by appointment to avoid conflicts and wasted
time. A list of examiner names can be obtained from the local
FSDO. Since a DPE serves without pay from the government
for conducting practical tests and processing the necessary
reports, the examiner is allowed to charge a reasonable fee.
There is no charge for the practical test when conducted by
an FAA inspector.
Role of the Certificated Flight Instructor
To become a CFI, a pilot must meet the provisions of 14 CFR
part 61. The FAA places full responsibility for student flight
training on the shoulders of the CFI, who is the cornerstone
of aviation safety. It is the job of the flight instructor to train
the student pilot in all the knowledge areas and teach the
skills necessary for the student pilot to operate safely and
competently as a certificated pilot in the NAS. The training
includes airmanship skills, pilot judgment and decisionmaking, and good operating practices.
A pilot training program depends on the quality of the
ground and flight instruction the student pilot receives. The
flight instructor must possess a thorough understanding of
the learning process, knowledge of the fundamentals of
teaching, and the ability to communicate effectively with the
student pilot. Use of a structured training program and formal
course syllabus is crucial for effective and comprehensive
flight training. It should be clear to the student in advance of
every lesson what the course of training will involve and the
criteria for successful completion. This should include the
flight instructor briefing and debriefing the student before and
after every lesson. Additionally, scenario-based training has
become the preferred method of flight instruction today. This
involves presenting the student with realistic flight scenarios
and recommended actions for mitigating risk.
Insistence on correct techniques and procedures from the
beginning of training by the flight instructor ensures that the
student pilot develops proper flying habits. Any deficiencies
in the maneuvers or techniques must immediately be
emphasized and corrected. A flight instructor serves as a role
model for the student pilot who observes the flying habits of
his or her flight instructor during flight instruction, as well
as when the instructor conducts other pilot operations. Thus,
the flight instructor becomes a model of flying proficiency
for the student who, consciously or unconsciously, attempts
to imitate the instructor. For this reason, a flight instructor
1-23

should observe recognized safety practices, as well as
regulations during all flight operations.
The student pilot who enrolls in a pilot training program
commits considerable time, effort, and expense to achieve a
pilot certificate. Students often judge the effectiveness of the
flight instructor and the success of the pilot training program
based on their ability to pass the requisite FAA practical
test. A competent flight instructor stresses to the student that
practical tests are a sampling of pilot ability compressed into
a short period of time. The goal of a flight instructor is to
train the "total" pilot.
Role of the Designated Pilot Examiner
The Designated Pilot Examiner (DPE) plays an important
role in the FAA's mission of promoting aviation safety
by administering FAA practical tests for pilot and Flight
Instructor Certificates and associated ratings. Although
administering these tests is a responsibility of the ASI, the
FAA's highest priority is making air travel safer by inspecting
aircraft that fly in the United States. To satisfy the need for
pilot testing and certification services, the FAA delegates
certain responsibilities to private individuals who are not
FAA employees.
Appointed in accordance with 14 CFR part 183, section
183.23, a DPE is an individual who meets the qualification
requirements of the Pilot Examiner's Handbook, FAA Order
8710.3, and who:


Is technically qualified



Holds all pertinent category, class, and type ratings
for each aircraft related to their designation



Meets requirements of 14 CFR part 61, sections 61.56,
61.57, and 61.58, as appropriate



Is current and qualified to act as PIC of each aircraft
for which he or she is authorized



Maintains at least a Third-Class Medical Certificate,
if required



Maintains a current Flight Instructor Certificate, if
required

Designated to perform specific pilot certification tasks
on behalf of the FAA, a DPE may charge a reasonable
fee. Generally, a DPE's authority is limited to accepting
applications and conducting practical tests leading to the
issuance of specific pilot certificates and/or ratings. The
majority of FAA practical tests at the private and commercial
pilot levels are administered by DPEs.
DPE candidates must have good industry reputations for
professionalism, integrity, a demonstrated willingness to
1-24

serve the public, and must adhere to FAA policies and
procedures in certification matters. The FAA expects the
DPE to administer practical tests with the same degree of
professionalism, using the same methods, procedures, and
standards as an FAA ASI.

Chapter Summary
The FAA has entered the second century of civil aviation as a
robust government organization and is taking full advantage of
technology, such as Global Positioning System (GPS) satellite
technology to enhance the safety of civil aviation. The Internet
has also become an important tool in promoting aviation safety
and providing around-the-clock resources for the aviation
community. Handbooks, regulations, standards, references,
and online courses are now available at \url{faa.gov}.
In keeping with the FAA's belief that safety is a learned
behavior, the FAA offers many courses and seminars to
enhance air safety. The FAA puts the burden of instilling safe
flying habits on the flight instructor, who should follow basic
flight safety practices and procedures in every flight operation
he or she undertakes with a student pilot. Operational safety
practices include, but are not limited to, collision avoidance
procedures consisting of proper scanning techniques, use of
checklists, runway incursion avoidance, positive transfer of
controls, and workload management. These safety practices
are discussed more fully within this handbook. Safe flight also
depends on Scenario-Based Training (SBT) that teaches the
student pilot how to respond in different flight situations. The
FAA has incorporated these techniques along with decisionmaking methods, such as aeronautical decision-making
(ADM), risk management, and crew resource management
(CRM), which are covered more completely in Chapter 2,
Aeronautical Decision-Making.

Chapter 2

Aeronautical
Decision-Making
Introduction
Aeronautical decision-making (ADM) is decision-making
in a unique environment—aviation. It is a systematic
approach to the mental process used by pilots to consistently
determine the best course of action in response to a given set
of circumstances. It is what a pilot intends to do based on the
latest information he or she has.

2-1

The importance of learning and understanding effective
ADM skills cannot be overemphasized. While progress is
continually being made in the advancement of pilot training
methods, aircraft equipment and systems, and services
for pilots, accidents still occur. Despite all the changes in
technology to improve flight safety, one factor remains the
same: the human factor which leads to errors. It is estimated
that approximately 80 percent of all aviation accidents are
related to human factors and the vast majority of these
accidents occur during landing (24.1 percent) and takeoff
(23.4 percent). [Figure 2-1]
ADM is a systematic approach to risk assessment and stress
management. To understand ADM is to also understand
how personal attitudes can influence decision-making and
how those attitudes can be modified to enhance safety in the
flight deck. It is important to understand the factors that cause
humans to make decisions and how the decision-making
process not only works, but can be improved.
This chapter focuses on helping the pilot improve his or
her ADM skills with the goal of mitigating the risk factors
associated with flight. Advisory Circular (AC) 60-22,
"Aeronautical Decision-Making," provides background
references, definitions, and other pertinent information about
ADM training in the general aviation (GA) environment.
[Figure 2-2]

History of ADM
For over 25 years, the importance of good pilot judgment, or
aeronautical decision-making (ADM), has been recognized
as critical to the safe operation of aircraft, as well as accident
avoidance. The airline industry, motivated by the need to
reduce accidents caused by human factors, developed the first
training programs based on improving ADM. Crew resource
management (CRM) training for flight crews is focused on
the effective use of all available resources: human resources,
hardware, and information supporting ADM to facilitate crew
cooperation and improve decision-making. The goal of all
flight crews is good ADM and the use of CRM is one way
to make good decisions.
Research in this area prompted the Federal Aviation
Administration (FAA) to produce training directed at
improving the decision-making of pilots and led to current
FAA regulations that require that decision-making be taught
as part of the pilot training curriculum. ADM research,
development, and testing culminated in 1987 with the
publication of six manuals oriented to the decision-making
needs of variously rated pilots. These manuals provided
multifaceted materials designed to reduce the number
of decision-related accidents. The effectiveness of these
materials was validated in independent studies where student
pilots received such training in conjunction with the standard
flying curriculum. When tested, the pilots who had received
ADM-training made fewer in-flight errors than those who had

Percentage of General Aviation Accidents
Flight Time
2\%

Flight Time
83\%

Flight Time
15\%

24.1\%

23.4\%

15.7\%
13\%
9.7\%

3.5\%

Preflights/
Taxi

4.7\%

3.3\%

Takeoff/
Initial Climb

Climb

2.6\%

Cruise

Descent

Maneuvering

Approach

Landing

Other

Figure 2-1. The percentage of aviation accidents as they relate to the different phases of flight. Note that the greatest percentage of
accidents take place during a minor percentage of the total flight.

2-2

Figure 2-2. Advisory Circular (AC) 60-22, "Aeronautical Decision Making," carries a wealth of information for the pilot to learn.

not received ADM training. The differences were statistically
significant and ranged from about 10 to 50 percent fewer
judgment errors. In the operational environment, an operator
flying about 400,000 hours annually demonstrated a 54
percent reduction in accident rate after using these materials
for recurrency training.
Contrary to popular opinion, good judgment can be taught.
Tradition held that good judgment was a natural by-product
of experience, but as pilots continued to log accident-free
flight hours, a corresponding increase of good judgment
was assumed. Building upon the foundation of conventional
decision-making, ADM enhances the process to decrease the
probability of human error and increase the probability of a
safe flight. ADM provides a structured, systematic approach
to analyzing changes that occur during a flight and how these
changes might affect the safe outcome of a flight. The ADM
process addresses all aspects of decision-making in the flight
deck and identifies the steps involved in good decision-making.
Steps for good decision-making are:
1.

Identifying personal attitudes hazardous to safe flight

2.

Learning behavior modification techniques

3.

Learning how to recognize and cope with stress

4.

Developing risk assessment skills

5.

Using all resources

6.

Evaluating the effectiveness of one's ADM skills

Risk Management
The goal of risk management is to proactively identify
safety-related hazards and mitigate the associated risks. Risk
management is an important component of ADM. When a
pilot follows good decision-making practices, the inherent risk
in a flight is reduced or even eliminated. The ability to make
good decisions is based upon direct or indirect experience
and education. The formal risk management decision-making
process involves six steps as shown in Figure 2-3.
Consider automotive seat belt use. In just two decades, seat
belt use has become the norm, placing those who do not
wear seat belts outside the norm, but this group may learn to
wear a seat belt by either direct or indirect experience. For
example, a driver learns through direct experience about the
value of wearing a seat belt when he or she is involved in a car
accident that leads to a personal injury. An indirect learning
experience occurs when a loved one is injured during a car
accident because he or she failed to wear a seat belt.
As you work through the ADM cycle, it is important to
remember the four fundamental principles of risk management.
2-3

START

Monitor
Results

Crew Resource Management (CRM) and
Single-Pilot Resource Management
Identify
Hazards

RISK

Assess
Risks

MANAGEMENT

PROCESS
Use
Controls

Analyze
Controls

Make Control
Decisions

Figure 2-3. Risk management decision-making process.

1.	 Accept no unnecessary risk. Flying is not possible
without risk, but unnecessary risk comes without a
corresponding return. If you are flying a new airplane
for the first time, you might determine that the risk
of making that flight in low visibility conditions is
unnecessary.
2.	 Make risk decisions at the appropriate level. Risk
decisions should be made by the person who can
develop and implement risk controls. Remember
that you are pilot-in-command, so never let anyone
else—not ATC and not your passengers—make risk
decisions for you.
3. 	 Accept risk when benefits outweigh dangers (costs).
In any flying activity, it is necessary to accept some
degree of risk. A day with good weather, for example,
is a much better time to fly an unfamiliar airplane for
the first time than a day with low IFR conditions.
4. 	 Integrate risk management into planning at all levels.
Because risk is an unavoidable part of every flight,
safety requires the use of appropriate and effective risk
management not just in the preflight planning stage,
but in all stages of the flight.
While poor decision-making in everyday life does not always
lead to tragedy, the margin for error in aviation is thin. Since
ADM enhances management of an aeronautical environment,
all pilots should become familiar with and employ ADM.

2-4

While CRM focuses on pilots operating in crew environments,
many of the concepts apply to single-pilot operations. Many
CRM principles have been successfully applied to single-pilot
aircraft and led to the development of Single-Pilot Resource
Management (SRM). SRM is defined as the art and science
of managing all the resources (both on-board the aircraft
and from outside sources) available to a single pilot (prior
to and during flight) to ensure the successful outcome of the
flight. SRM includes the concepts of ADM, risk management
(RM), task management (TM), automation management
(AM), controlled flight into terrain (CFIT) awareness, and
situational awareness (SA). SRM training helps the pilot
maintain situational awareness by managing the automation
and associated aircraft control and navigation tasks. This
enables the pilot to accurately assess and manage risk and
make accurate and timely decisions.
SRM is all about helping pilots learn how to gather
information, analyze it, and make decisions. Although the
flight is coordinated by a single person and not an onboard
flight crew, the use of available resources such as auto-pilot
and air traffic control (ATC) replicates the principles of CRM.

Hazard and Risk
Two defining elements of ADM are hazard and risk. Hazard
is a real or perceived condition, event, or circumstance that a
pilot encounters. When faced with a hazard, the pilot makes
an assessment of that hazard based upon various factors. The
pilot assigns a value to the potential impact of the hazard,
which qualifies the pilot's assessment of the hazard—risk.
Therefore, risk is an assessment of the single or cumulative
hazard facing a pilot; however, different pilots see hazards
differently. For example, the pilot arrives to preflight and
discovers a small, blunt type nick in the leading edge at the
middle of the aircraft's prop. Since the aircraft is parked on
the tarmac, the nick was probably caused by another aircraft's
prop wash blowing some type of debris into the propeller.
The nick is the hazard (a present condition). The risk is prop
fracture if the engine is operated with damage to a prop blade.
The seasoned pilot may see the nick as a low risk. He
realizes this type of nick diffuses stress over a large area, is
located in the strongest portion of the propeller, and based
on experience; he does not expect it to propagate a crack that
can lead to high risk problems. He does not cancel his flight.

The inexperienced pilot may see the nick as a high risk factor
because he is unsure of the affect the nick will have on the
operation of the prop, and he has been told that damage to
a prop could cause a catastrophic failure. This assessment
leads him to cancel his flight.
Therefore, elements or factors affecting individuals are
different and profoundly impact decision-making. These
are called human factors and can transcend education,
experience, health, physiological aspects, etc.
Another example of risk assessment was the flight of a
Beechcraft King Air equipped with deicing and anti-icing.
The pilot deliberately flew into moderate to severe icing
conditions while ducking under cloud cover. A prudent pilot
would assess the risk as high and beyond the capabilities of
the aircraft, yet this pilot did the opposite. Why did the pilot
take this action?
Past experience prompted the action. The pilot had
successfully flown into these conditions repeatedly although
the icing conditions were previously forecast 2,000 feet above
the surface. This time, the conditions were forecast from the
surface. Since the pilot was in a hurry and failed to factor

in the difference between the forecast altitudes, he assigned
a low risk to the hazard and took a chance. He and the
passengers died from a poor risk assessment of the situation.
Hazardous Attitudes and Antidotes
Being fit to fly depends on more than just a pilot's physical
condition and recent experience. For example, attitude
affects the quality of decisions. Attitude is a motivational
predisposition to respond to people, situations, or events in a
given manner. Studies have identified five hazardous attitudes
that can interfere with the ability to make sound decisions
and exercise authority properly: anti-authority, impulsivity,
invulnerability, macho, and resignation. [Figure 2-4]
Hazardous attitudes contribute to poor pilot judgment but
can be effectively counteracted by redirecting the hazardous
attitude so that correct action can be taken. Recognition of
hazardous thoughts is the first step toward neutralizing them.
After recognizing a thought as hazardous, the pilot should
label it as hazardous, then state the corresponding antidote.
Antidotes should be memorized for each of the hazardous
attitudes so they automatically come to mind when needed.

The Five Hazardous Attitudes
Anti-authority: "Don't tell me."
This attitude is found in people who do not like anyone telling them what to do. In a
sense, they are saying, "No one can tell me what to do." They may be resentful of
having someone tell them what to do or may regard rules, regulations, and procedures
as silly or unnecessary. However, it is always your prerogative to question authority
if you feel it is in error.
Impulsivity: "Do it quickly."
This is the attitude of people who frequently feel the need to do something, anything,
immediately. They do not stop to think about what they are about to do, they do not
select the best alternative, and they do the first thing that comes to mind.
Invulnerability: "It won't happen to me."
Many people falsely believe that accidents happen to others, but never to them. They
know accidents can happen, and they know that anyone can be affected. However,
they never really feel or believe that they will be personally involved. Pilots who think
this way are more likely to take chances and increase risk.
Macho: "I can do it."
Pilots who are always trying to prove that they are better than anyone else think, "I can
do it—I'll show them." Pilots with this type of attitude will try to prove themselves by
taking risks in order to impress others. While this pattern is thought to be a male
characteristic, women are equally susceptible.
Resignation: "What's the use?"
Pilots who think, "What's the use?" do not see themselves as being able to make a
great deal of difference in what happens to them. When things go well, the pilot is apt
to think that it is good luck. When things go badly, the pilot may feel that someone is
out to get them or attribute it to bad luck. The pilot will leave the action to others, for
better or worse. Sometimes, such pilots will even go along with unreasonable requests
just to be a "nice guy."

Antidote

Follow the rules. They are usually right.

Not so fast. Think first.

It could happen to me.

Taking chances is foolish.

I'm not helpless. I can make a difference.

Figure 2-4. The five hazardous attitudes identified through past and contemporary study.

2-5

Risk
During each flight, the single pilot makes many decisions
under hazardous conditions. To fly safely, the pilot needs
to assess the degree of risk and determine the best course of
action to mitigate the risk.

Risk Assessment Matrix
Severity

Assessing Risk
For the single pilot, assessing risk is not as simple as it sounds.
For example, the pilot acts as his or her own quality control
in making decisions. If a fatigued pilot who has flown 16
hours is asked if he or she is too tired to continue flying, the
answer may be "no." Most pilots are goal oriented and when
asked to accept a flight, there is a tendency to deny personal
limitations while adding weight to issues not germane to the
mission. For example, pilots of helicopter emergency services
(EMS) have been known (more than other groups) to make
flight decisions that add significant weight to the patient's
welfare. These pilots add weight to intangible factors (the
patient in this case) and fail to appropriately quantify actual
hazards, such as fatigue or weather, when making flight
decisions. The single pilot who has no other crew member
for consultation must wrestle with the intangible factors that
draw one into a hazardous position. Therefore, he or she has
a greater vulnerability than a full crew.
Examining National Transportation Safety Board (NTSB)
reports and other accident research can help a pilot learn to
assess risk more effectively. For example, the accident rate
during night visual flight rules (VFR) decreases by nearly
50 percent once a pilot obtains 100 hours and continues to
decrease until the 1,000 hour level. The data suggest that for
the first 500 hours, pilots flying VFR at night might want to
establish higher personal limitations than are required by the
regulations and, if applicable, apply instrument flying skills
in this environment.
Several risk assessment models are available to assist in the
process of assessing risk. The models, all taking slightly
different approaches, seek a common goal of assessing risk
in an objective manner. The most basic tool is the risk matrix.
[Figure 2-5] It assesses two items: the likelihood of an event
occurring and the consequence of that event.
Likelihood of an Event
Likelihood is nothing more than taking a situation and
determining the probability of its occurrence. It is rated as
probable, occasional, remote, or improbable. For example, a
pilot is flying from point A to point B (50 miles) in marginal
visual flight rules (MVFR) conditions. The likelihood of
encountering potential instrument meteorological conditions
(IMC) is the first question the pilot needs to answer. The
experiences of other pilots, coupled with the forecast, might

2-6

Likelihood

Catastrophic

Critical

Marginal

Probable

High

High

Serious

Occasional

High

Serious

Remote

Serious

Medium

Negligible

Low

Improbable

Figure 2-5. This risk matrix can be used for almost any operation
by assigning likelihood and consequence. In the case presented,
the pilot assigned a likelihood of occasional and the severity as
catastrophic. As one can see, this falls in the high risk area.

cause the pilot to assign "occasional" to determine the
probability of encountering IMC.
The following are guidelines for making assignments.


Probable—an event will occur several times



Occasional—an event will probably occur sometime



Remote—an event is unlikely to occur, but is possible



Improbable—an event is highly unlikely to occur

Severity of an Event
The next element is the severity or consequence of a pilot's
action(s). It can relate to injury and/or damage. If the
individual in the example above is not an instrument rated
pilot, what are the consequences of him or her encountering
inadvertent IMC conditions? In this case, because the pilot
is not IFR rated, the consequences are catastrophic. The
following are guidelines for this assignment.


Catastrophic—results in fatalities, total loss



Critical—severe injury, major damage



Marginal—minor injury, minor damage



Negligible—less than minor injury, less than minor
system damage

Simply connecting the two factors as shown in Figure 2-5
indicates the risk is high and the pilot must either not fly or
fly only after finding ways to mitigate, eliminate, or control
the risk.
Although the matrix in Figure 2-5 provides a general viewpoint
of a generic situation, a more comprehensive program can be
made that is tailored to a pilot's flying. [Figure 2-6] This
program includes a wide array of aviation-related activities
specific to the pilot and assesses health, fatigue, weather,

RISK ASSESSMENT
Pilot's Name

Flight From

SLEEP

To

HOW IS THE DAY GOING?

1. Did not sleep well or less than 8 hours
2. Slept well

2
0

1. Seems like one thing after another (late,
making errors, out of step)
2. Great day

3
0

HOW DO YOU FEEL?
1. Have a cold or ill
2. Feel great
3. Feel a bit off

4
0
2

IS THE FLIGHT
1. Day?
2. Night?

WEATHER AT TERMINATION

1
3

PLANNING

1. Greater than 5 miles visibility and 3,000 feet
ceilings
2. At least 3 miles visibility and 1,000 feet ceilings,
but less than 3,000 feet ceilings and 5 miles
visibility
3. IMC conditions
Column total

1

3
4

1. Rush to get off ground
2. No hurry
3. Used charts and computer to assist
4. Used computer program for all planning
5. Did you verify weight and balance?
6. Did you evaluate performance?
7. Do you brief your passangers on the
ground and in flight?

Yes
No
Yes
No
Yes
No
Yes
No

3
1
0
3
0
0
3
0
3
0
2

Column total

Low risk

Endangerment

TOTAL SCORE

0

Not complex flight

10

Exercise caution

20

Area of concern

30

Figure 2-6. Example of a more comprehensive risk assessment program.

2-7

capabilities, etc. The scores are added and the overall score
falls into various ranges, with the range representative of
actions that a pilot imposes upon himself or herself.

Mitigating Risk
Risk assessment is only part of the equation. After
determining the level of risk, the pilot needs to mitigate the
risk. For example, the pilot flying from point A to point B (50
miles) in MVFR conditions has several ways to reduce risk:
 	 Wait for the weather to improve to good visual flight
rules (VFR) conditions.
 	 Take an instrument-rated pilot.
 	 Delay the flight.
 	 Cancel the flight.
 	 Drive.
One of the best ways single pilots can mitigate risk is to use
the IMSAFE checklist to determine physical and mental
readiness for flying:
1. 	 Illness—Am I sick? Illness is an obvious pilot risk.
2. 	 Medication—Am I taking any medicines that might
affect my judgment or make me drowsy?

Once a pilot identifies the risks of a flight, he or she needs
to decide whether the risk, or combination of risks, can be
managed safely and successfully. If not, make the decision to
cancel the flight. If the pilot decides to continue with the flight,
he or she should develop strategies to mitigate the risks. One
way a pilot can control the risks is to set personal minimums
for items in each risk category. These are limits unique to that
individual pilot's current level of experience and proficiency.
For example, the aircraft may have a maximum crosswind
component of 15 knots listed in the aircraft flight manual
(AFM), and the pilot has experience with 10 knots of direct
crosswind. It could be unsafe to exceed a 10 knot crosswind
component without additional training. Therefore, the 10 knot
crosswind experience level is that pilot's personal limitation
until additional training with a certificated flight instructor
(CFI) provides the pilot with additional experience for flying
in crosswinds that exceed 10 knots.
One of the most important concepts that safe pilots
understand is the difference between what is "legal" in terms
of the regulations, and what is "smart" or "safe" in terms of
pilot experience and proficiency.

P = Pilot in Command (PIC)

3. 	 Stress—Am I under psychological pressure from the
job? Do I have money, health, or family problems?
Stress causes concentration and performance problems.
While the regulations list medical conditions that
require grounding, stress is not among them. The pilot
should consider the effects of stress on performance.

The pilot is one of the risk factors in a flight. The pilot must
ask, "Am I ready for this trip?" in terms of experience,
recency, currency, physical, and emotional condition. The
IMSAFE checklist provides the answers.

4.	 Alcohol—Have I been drinking within 8 hours?
Within 24 hours? As little as one ounce of liquor, one
bottle of beer, or four ounces of wine can impair flying
skills. Alcohol also renders a pilot more susceptible
to disorientation and hypoxia.

What limitations will the aircraft impose upon the trip? Ask
the following questions:

5.	 Fatigue—Am I tired and not adequately rested?
Fatigue continues to be one of the most insidious
hazards to flight safety, as it may not be apparent to
a pilot until serious errors are made.
6. 	 Emotion—Am I emotionally upset?
The PAVE Checklist
Another way to mitigate risk is to perceive hazards. By
incorporating the PAVE checklist into preflight planning,
the pilot divides the risks of flight into four categories: Pilot­
in-command (PIC), Aircraft, enVironment, and External
pressures (PAVE) which form part of a pilot's decisionmaking process.
With the PAVE checklist, pilots have a simple way to
remember each category to examine for risk prior to each flight.
2-8

A = Aircraft

 	 Is this the right aircraft for the flight?


Am I familiar with and current in this aircraft? Aircraft
performance figures and the AFM are based on a brand
new aircraft flown by a professional test pilot. Keep
that in mind while assessing personal and aircraft
performance.

 	 Is this aircraft equipped for the flight? Instruments?
Lights? Navigation and communication equipment
adequate?


Can this aircraft use the runways available for the trip
with an adequate margin of safety under the conditions
to be flown?

 	 Can this aircraft carry the planned load?


Can this aircraft operate at the altitudes needed for the
trip?



Does this aircraft have sufficient fuel capacity, with
reserves, for trip legs planned?



Does the fuel quantity delivered match the fuel
quantity ordered?

V = EnVironment
Weather
Weather is a major environmental consideration. Earlier it was
suggested pilots set their own personal minimums, especially
when it comes to weather. As pilots evaluate the weather for
a particular flight, they should consider the following:




What is the current ceiling and visibility? In
mountainous terrain, consider having higher
minimums for ceiling and visibility, particularly if
the terrain is unfamiliar.
Consider the possibility that the weather may be
different than forecast. Have alternative plans and
be ready and willing to divert, should an unexpected
change occur.

 	 Consider the winds at the airports being used and the
strength of the crosswind component.


If flying in mountainous terrain, consider whether
there are strong winds aloft. Strong winds in
mountainous terrain can cause severe turbulence and
downdrafts and be very hazardous for aircraft even
when there is no other significant weather.

 	 Are there any thunderstorms present or forecast?




If there are clouds, is there any icing, current or
forecast? What is the temperature/dew point spread
and the current temperature at altitude? Can descent
be made safely all along the route?
If icing conditions are encountered, is the pilot
experienced at operating the aircraft's deicing or
anti-icing equipment? Is this equipment in good
condition and functional? For what icing conditions
is the aircraft rated, if any?

Terrain
Evaluation of terrain is another important component of
analyzing the flight environment.




To avoid terrain and obstacles, especially at night or
in low visibility, determine safe altitudes in advance
by using the altitudes shown on VFR and IFR charts
during preflight planning.
Use maximum elevation figures (MEFs) and other
easily obtainable data to minimize chances of an
inflight collision with terrain or obstacles.

Airport


What lights are available at the destination and
alternate airports? VASI/PAPI or ILS glideslope

guidance? Is the terminal airport equipped with them?
Are they working? Will the pilot need to use the radio
to activate the airport lights?


Check the Notices to Airmen (NOTAM) for closed
runways or airports. Look for runway or beacon lights
out, nearby towers, etc.



Choose the flight route wisely. An engine failure gives
the nearby airports supreme importance.



Are there shorter or obstructed fields at the destination
and/or alternate airports?

Airspace


If the trip is over remote areas, is there appropriate
clothing, water, and survival gear onboard in the event
of a forced landing?

 	 If the trip includes flying over water or unpopulated
areas with the chance of losing visual reference to the
horizon, the pilot must be prepared to fly IFR.


Check the airspace and any temporary flight restriction
(TFRs) along the route of flight.

Nighttime
Night flying requires special consideration.


If the trip includes flying at night over water or
unpopulated areas with the chance of losing visual
reference to the horizon, the pilot must be prepared
to fly IFR.



Will the flight conditions allow a safe emergency
landing at night?



Perform preflight check of all aircraft lights, interior
and exterior, for a night flight. Carry at least two
flashlights—one for exterior preflight and a smaller
one that can be dimmed and kept nearby.

E = External Pressures
External pressures are influences external to the flight that
create a sense of pressure to complete a flight—often at the
expense of safety. Factors that can be external pressures
include the following:
 	 Someone waiting at the airport for the flight's arrival
 	 A passenger the pilot does not want to disappoint
 	 The desire to demonstrate pilot qualifications


The desire to impress someone (Probably the two most
dangerous words in aviation are "Watch this!")

 	 The desire to satisfy a specific personal goal ("get­
home-itis," "get-there-itis," and "let's-go-itis")


The pilot's general goal-completion orientation
2-9



Emotional pressure associated with acknowledging
that skill and experience levels may be lower than a
pilot would like them to be. Pride can be a powerful
external factor!

Managing External Pressures
Management of external pressure is the single most important
key to risk management because it is the one risk factor
category that can cause a pilot to ignore all the other risk
factors. External pressures put time-related pressure on the
pilot and figure into a majority of accidents.
The use of personal standard operating procedures (SOPs) is
one way to manage external pressures. The goal is to supply a
release for the external pressures of a flight. These procedures
include but are not limited to:


Allow time on a trip for an extra fuel stop or to make
an unexpected landing because of weather.



Have alternate plans for a late arrival or make backup
airline reservations for must-be-there trips.

 	 For really important trips, plan to leave early enough
so that there would still be time to drive to the
destination, if necessary.
Pilot
A pilot must continually make decisions about competency,
condition of health, mental and emotional state, level of fatigue,
and many other variables. For example, a pilot may be called
early in the morning to make a long flight. If a pilot has had only
a few hours of sleep and is concerned that the congestion
being experienced could be the onset of a cold, it would be
prudent to consider if the flight could be accomplished safely.
A pilot had only 4 hours of sleep the night before being asked
by the boss to fly to a meeting in a city 750 miles away. The
reported weather was marginal and not expected to improve.
After assessing fitness as a pilot, it was decided that it would
not be wise to make the flight. The boss was initially unhappy,
but later convinced by the pilot that the risks involved were
unacceptable.
Environment
This encompasses many elements not pilot or airplane related.
It can include such factors as weather, air traffic control,
navigational aids (NAVAIDS), terrain, takeoff and landing
areas, and surrounding obstacles. Weather is one element that
can change drastically over time and distance.
A pilot was landing a small airplane just after a heavy jet had
departed a parallel runway. The pilot assumed that wake
turbulence would not be a problem since landings had been
performed under similar circumstances. Due to a combination
of prevailing winds and wake turbulence from the heavy jet
drifting across the landing runway, the airplane made a hard
landing. The pilot made an error when assessing the flight
environment.

Figure 2-7. The PAVE checklist.

2-10



Advise those who are waiting at the destination that
the arrival may be delayed. Know how to notify them
when delays are encountered.



Manage passengers' expectations. Make sure
passengers know that they might not arrive on a firm
schedule, and if they must arrive by a certain time,
they should make alternative plans.

 	 Eliminate pressure to return home, even on a casual
day flight, by carrying a small overnight kit containing
prescriptions, contact lens solutions, toiletries, or other
necessities on every flight.
The key to managing external pressure is to be ready for
and accept delays. Remember that people get delayed when
traveling on airlines, driving a car, or taking a bus. The pilot's
goal is to manage risk, not create hazards. [Figure 2-7]

Human Factors
Why are human conditions, such as fatigue, complacency
and stress, so important in aviation? These conditions, along
with many others, are called human factors. Human factors
directly cause or contribute to many aviation accidents and

Aircraft
A pilot will frequently base decisions on the evaluations of the

aircraft, such as performance, equipment, or airworthiness.

During a preflight, a pilot noticed a small amount of oil dripping
from the bottom of the cowling. Although the quantity of oil
seemed insignificant at the time, the pilot decided to delay the
takeoff and have a mechanic check the source of the oil.
The pilot's good judgment was confirmed when the mechanic
found that one of the oil cooler hose fittings was loose.
External pressures
The interaction between the pilot, airplane, and the environment
is greatly influenced by the purpose of each flight operation.
The pilot must evaluate the three previous areas to decide on
the desirability of undertaking or continuing the flight as planned.
It is worth asking why the flight is being made, how critical is it to
maintain the schedule, and is the trip worth the risks?
On a ferry flight to deliver an airplane from the factory, in
marginal weather conditions, the pilot calculated the
groundspeed and determined that the airplane would arrive at
the destination with only 10 minutes of fuel remaining. The pilot
was determined to keep on schedule by trying to "stretch" the
fuel supply instead of landing to refuel. After landing with low
fuel state, the pilot realized that this could have easily resulted
in an emergency landing in deteriorating weather conditions.
This was a chance that was not worth taking to keep the
planned schedule.

have been documented as a primary contributor to more than
70 percent of aircraft accidents.
Typically, human factor incidents/accidents are associated
with flight operations but recently have also become a major
concern in aviation maintenance and air traffic management
as well. [Figure 2-8] Over the past several years, the FAA has
made the study and research of human factors a top priority
by working closely with engineers, pilots, mechanics, and
ATC to apply the latest knowledge about human factors in
an effort to help operators and maintainers improve safety
and efficiency in their daily operations.
Human factors science, or human factors technologies,
is a multidisciplinary field incorporating contributions
from psychology, engineering, industrial design, statistics,
operations research, and anthropometry. It is a term that
covers the science of understanding the properties of
human capability, the application of this understanding to
the design, development and deployment of systems and
services, and the art of ensuring successful application of
human factor principles into all aspects of aviation to include
pilots, ATC, and aviation maintenance. Human factors is
often considered synonymous with CRM or maintenance
resource management (MRM) but is really much broader in

both its knowledge base and scope. Human factors involves
gathering research specific to certain situations (i.e., flight,
maintenance, stress levels, knowledge) about human abilities,
limitations, and other characteristics and applying it to tool
design, machines, systems, tasks, jobs, and environments
to produce safe, comfortable, and effective human use. The
entire aviation community benefits greatly from human
factors research and development as it helps better understand
how humans can most safely and efficiently perform their
jobs and improve the tools and systems in which they interact.

Human Behavior
Studies of human behavior have tried to determine an
individual's predisposition to taking risks and the level of
an individual's involvement in accidents. In 1951, a study
regarding injury-prone children was published by Elizabeth
Mechem Fuller and Helen B. Baune, of the University of
Minnesota. The study was comprised of two separate groups
of second grade students. Fifty-five students were considered
accident repeaters and 48 students had no accidents. Both
groups were from the same school of 600 and their family
demographics were similar.
The accident-free group showed a superior knowledge
of safety, was considered industrious and cooperative

Figure 2-8. Human factors effects pilots, aviation maintenance technicians (AMTs) and air traffic control (ATC).

2-11

with others, but were not considered physically inclined.
The accident-repeater group had better gymnastic skills,
was considered aggressive and impulsive, demonstrated
rebellious behavior when under stress, were poor losers, and
liked to be the center of attention. One interpretation of this
data—an adult predisposition to injury stems from childhood
behavior and environment—leads to the conclusion that
any pilot group should be comprised only of pilots who are
safety-conscious, industrious, and cooperative.
Clearly, this is not only an inaccurate inference, it is
impossible. Pilots are drawn from the general population and
exhibit all types of personality traits. Thus, it is important that
good decision-making skills be taught to all pilots.
Historically, the term "pilot error" has been used to describe
an accident in which an action or decision made by the
pilot was the cause or a contributing factor that led to the
accident. This definition also includes the pilot's failure
to make a correct decision or take proper action. From a
broader perspective, the phrase "human factors related" more
aptly describes these accidents. A single decision or event
does not lead to an accident, but a series of events and the
resultant decisions together form a chain of events leading
to an outcome.
In his article "Accident-Prone Pilots," Dr. Patrick R. Veillette
uses the history of "Captain Everyman" to demonstrate how
aircraft accidents are caused more by a chain of poor choices
rather than one single poor choice. In the case of Captain
Everyman, after a gear-up landing accident, he became
involved in another accident while taxiing a Beech 58P Baron
out of the ramp. Interrupted by a radio call from the dispatcher,
Everyman neglected to complete the fuel cross-feed check
before taking off. Everyman, who was flying solo, left the
right-fuel selector in the cross-feed position. Once aloft and
cruising, he noticed a right roll tendency and corrected with
aileron trim. He did not realize that both engines were feeding
off the left wing's tank, making the wing lighter.
After two hours of flight, the right engine quit when
Everyman was flying along a deep canyon gorge. While he
was trying to troubleshoot the cause of the right engine's
failure, the left engine quit. Everyman landed the aircraft on
a river sand bar but it sank into ten feet of water.
Several years later Everyman flew a de Havilland Twin
Otter to deliver supplies to a remote location. When he
returned to home base and landed, the aircraft veered sharply
to the left, departed the runway, and ran into a marsh 375
feet from the runway. The airframe and engines sustained
considerable damage. Upon inspecting the wreck, accident
investigators found the nose wheel steering tiller in the fully
2-12

deflected position. Both the after takeoff and before landing
checklists require the tiller to be placed in the neutral position.
Everyman had overlooked this item.
Now, is Everyman accident prone or just unlucky? Skipping
details on a checklist appears to be a common theme in the
preceding accidents. While most pilots have made similar
mistakes, these errors were probably caught prior to a mishap
due to extra margin, good warning systems, a sharp copilot,
or just good luck. What makes a pilot less prone to accidents?
The successful pilot possesses the ability to concentrate,
manage workloads, and monitor and perform several
simultaneous tasks. Some of the latest psychological
screenings used in aviation test applicants for their ability
to multitask, measuring both accuracy, as well as the
individual's ability to focus attention on several subjects
simultaneously. The FAA oversaw an extensive research
study on the similarities and dissimilarities of accident-free
pilots and those who were not. The project surveyed over
4,000 pilots, half of whom had "clean" records while the
other half had been involved in an accident.
Five traits were discovered in pilots prone to having
accidents. These pilots:


Have disdain toward rules



Have very high correlation between accidents on their
flying records and safety violations on their driving
records



Frequently fall into the "thrill and adventure seeking"
personality category



Are impulsive rather than methodical and disciplined,
both in their information gathering and in the speed
and selection of actions to be taken



Have a disregard for or tend to under utilize outside
sources of information, including copilots, flight
attendants, flight service personnel, flight instructors,
and ATC

The Decision-Making Process
An understanding of the decision-making process provides
the pilot with a foundation for developing ADM and SRM
skills. While some situations, such as engine failure, require an
immediate pilot response using established procedures, there is
usually time during a flight to analyze any changes that occur,
gather information, and assess risks before reaching a decision.
Risk management and risk intervention is much more than the
simple definitions of the terms might suggest. Risk management
and risk intervention are decision-making processes designed
to systematically identify hazards, assess the degree of risk, and

determine the best course of action. These processes involve
the identification of hazards, followed by assessments of the
risks, analysis of the controls, making control decisions, using
the controls, and monitoring the results.

arises. These decision points include preflight, pretakeoff,
hourly or at the midpoint of the flight, pre-descent, and just
prior to the final approach fix or for VFR operations, just
prior to entering the traffic pattern.

The steps leading to this decision constitute a decisionmaking process. Three models of a structured framework
for problem-solving and decision-making are the 5P, the 3P
using PAVE, CARE and TEAM, and the DECIDE models.
They provide assistance in organizing the decision process.
All these models have been identified as helpful to the single
pilot in organizing critical decisions.

The 5 Ps are based on the idea that pilots have essentially
five variables that impact his or her environment and forcing
him or her to make a single critical decision, or several less
critical decisions, that when added together can create a
critical outcome. These variables are the Plan, the Plane, the
Pilot, the Passengers, and the Programming. This concept
stems from the belief that current decision-making models
tended to be reactionary in nature. A change has to occur
and be detected to drive a risk management decision by the
pilot. For instance, many pilots complete risk management
sheets prior to takeoff. These form a catalog of risks
that may be encountered that day. Each of these risks is
assigned a numerical value. If the total of these numerical
values exceeds a predetermined level, the flight is altered or
cancelled. Informal research shows that while these are useful
documents for teaching risk factors, they are almost never
used outside of formal training programs. The 5P concept is
an attempt to take the information contained in those sheets
and in the other available models and use it.

Single-Pilot Resource Management (SRM)
Single-Pilot Resource Management (SRM) is about how to
gather information, analyze it, and make decisions. Learning
how to identify problems, analyze the information, and make
informed and timely decisions is not as straightforward as the
training involved in learning specific maneuvers. Learning
how to judge a situation and "how to think" in the endless
variety of situations encountered while flying out in the "real
world" is more difficult.
There is no one right answer in ADM, rather each pilot is
expected to analyze each situation in light of experience
level, personal minimums, and current physical and mental
readiness level, and make his or her own decision.
The 5 Ps Check
SRM sounds good on paper, but it requires a way for pilots
to understand and use it in their daily flights. One practical
application is called the "Five Ps (5 Ps)." [Figure 2-9] The
5 Ps consist of "the Plan, the Plane, the Pilot, the Passengers,
and the Programming." Each of these areas consists of a set
of challenges and opportunities that every pilot encounters.
Each challenge and opportunity can substantially increase or
decrease the risk of successfully completing the flight based
on the pilot's ability to make informed and timely decisions.
The 5 Ps are used to evaluate the pilot's current situation at
key decision points during the flight or when an emergency
The SRM Five "Ps" Check

THE PLAN

THE PLANE

THE PILOT

THE

THE

PASSENGERS

PROGRAMMING

Figure 2-9. The Five Ps checklist.

The 5P concept relies on the pilot to adopt a "scheduled"
review of the critical variables at points in the flight where
decisions are most likely to be effective. For instance, the
easiest point to cancel a flight due to bad weather is before the
pilot and passengers walk out the door and load the aircraft.
So the first decision point is preflight in the flight planning
room, where all the information is readily available to make
a sound decision, and where communication and Fixed
Base Operator (FBO) services are readily available to make
alternate travel plans.
The second easiest point in the flight to make a critical safety
decision is just prior to takeoff. Few pilots have ever had
to make an "emergency takeoff." While the point of the 5P
check is to help the pilot fly, the correct application of the 5
P before takeoff is to assist in making a reasoned go/no-go
decision based on all the information available. That decision
will usually be to "go," with certain restrictions and changes,
but may also be a "no-go." The key idea is that these two
points in the process of flying are critical go/no-go points on
each and every flight.
The third place to review the 5 Ps is at the midpoint of the
flight. Often, pilots may wait until the Automated Terminal
information Service (ATIS) is in range to check weather, yet,
at this point in the flight, many good options have already
passed behind the aircraft and pilot. Additionally, fatigue

2-13

and low-altitude hypoxia serve to rob the pilot of much of
his or her energy by the end of a long and tiring flight day.
This leads to a transition from a decision-making mode to an
acceptance mode on the part of the pilot. If the flight is longer
than 2 hours, the 5 P check should be conducted hourly.
The last two decision points are just prior to descent into the
terminal area and just prior to the final approach fix, or if
VFR, just prior to entering the traffic pattern as preparations
for landing commence. Most pilots execute approaches with
the expectation that they will land out of the approach every
time. A healthier approach requires the pilot to assume that
changing conditions (the 5 Ps again) will cause the pilot to
divert or execute the missed approach on every approach.
This keeps the pilot alert to all manner of conditions that
may increase risk and threaten the safe conduct of the flight.
Diverting from cruise altitude saves fuel, allows unhurried
use of the autopilot and is less reactive in nature. Diverting
from the final approach fix, while more difficult, still allows
the pilot to plan and coordinate better, rather than executing
a futile missed approach. Let's look at a detailed discussion
of each of the Five Ps.

The Plan
The "Plan" can also be called the mission or the task. It
contains the basic elements of cross-country planning,
weather, route, fuel, publications currency, etc. The "Plan"
should be reviewed and updated several times during the
course of the flight. A delayed takeoff due to maintenance,
fast moving weather, and a short notice TFR may all radically
alter the plan. The "plan" is not only about the flight plan,
but also all the events that surround the flight and allow the
pilot to accomplish the mission. The plan is always being
updated and modified and is especially responsive to changes
in the other four remaining Ps. If for no other reason, the 5 P
check reminds the pilot that the day's flight plan is real life
and subject to change at any time.
Obviously, weather is a huge part of any plan. The addition
of datalink weather information gives the advanced avionics
pilot a real advantage in inclement weather, but only if the
pilot is trained to retrieve and evaluate the weather in real
time without sacrificing situational awareness. And of course,
weather information should drive a decision, even if that
decision is to continue on the current plan. Pilots of aircraft
without datalink weather should get updated weather in flight
through an FSS and/or Flight Watch.

The Plane
Both the "plan" and the "plane" are fairly familiar to most
pilots. The "plane" consists of the usual array of mechanical
and cosmetic issues that every aircraft pilot, owner, or
operator can identify. With the advent of advanced avionics,
2-14

the "plane" has expanded to include database currency,
automation status, and emergency backup systems that were
unknown a few years ago. Much has been written about
single pilot IFR flight, both with and without an autopilot.
While this is a personal decision, it is just that—a decision.
Low IFR in a non-autopilot equipped aircraft may depend
on several of the other Ps to be discussed. Pilot proficiency,
currency, and fatigue are among them.

The Pilot
Flying, especially when business transportation is involved,
can expose a pilot to risks such as high altitudes, long trips
requiring significant endurance, and challenging weather.
Advanced avionics, when installed, can expose a pilot to high
stresses because of the inherent additional capabilities which
are available. When dealing with pilot risk, it is always best
to consult the "IMSAFE" checklist (see page 2-6).
The combination of late nights, pilot fatigue, and the effects
of sustained flight above 5,000 feet may cause pilots to
become less discerning, less critical of information, less
decisive, and more compliant and accepting. Just as the most
critical portion of the flight approaches (for instance a night
instrument approach, in the weather, after a 4-hour flight),
the pilot's guard is down the most. The 5 P process helps a
pilot recognize the physiological challenges that they may
face towards the end of the flight prior to takeoff and allows
them to update personal conditions as the flight progresses.
Once risks are identified, the pilot is in a better place to
make alternate plans that lessen the effect of these factors
and provide a safer solution.

The Passengers
One of the key differences between CRM and SRM is the
way passengers interact with the pilot. The pilot of a highly
capable single-engine aircraft maintains a much more personal
relationship with the passengers as he/she is positioned within
an arm's reach of them throughout the flight.
The necessity of the passengers to make airline connections
or important business meetings in a timely manner enters
into this pilot's decision-making loop. Consider a flight
to Dulles Airport in which and the passengers, both close
friends and business partners, need to get to Washington,
D.C. for an important meeting. The weather is VFR all the
way to southern Virginia, then turns to low IFR as the pilot
approaches Dulles. A pilot employing the 5 P approach
might consider reserving a rental car at an airport in northern
North Carolina or southern Virginia to coincide with a
refueling stop. Thus, the passengers have a way to get to
Washington, and the pilot has an out to avoid being pressured
into continuing the flight if the conditions do not improve.

Passengers can also be pilots. If no one is designated as pilot
in command (PIC) and unplanned circumstances arise, the
decision-making styles of several self-confident pilots may
come into conflict.
Pilots also need to understand that non-pilots may not
understand the level of risk involved in flight. There is an
element of risk in every flight. That is why SRM calls it risk
management, not risk elimination. While a pilot may feel
comfortable with the risk present in a night IFR flight, the
passengers may not. A pilot employing SRM should ensure
the passengers are involved in the decision-making and given
tasks and duties to keep them busy and involved. If, upon a
factual description of the risks present, the passengers decide
to buy an airline ticket or rent a car, then a good decision has
generally been made. This discussion also allows the pilot
to move past what he or she thinks the passengers want to
do and find out what they actually want to do. This removes
self-induced pressure from the pilot.

The Programming
The advanced avionics aircraft adds an entirely new
dimension to the way GA aircraft are flown. The electronic
instrument displays, GPS, and autopilot reduce pilot
workload and increase pilot situational awareness. While
programming and operation of these devices are fairly
simple and straightforward, unlike the analog instruments
they replace, they tend to capture the pilot's attention and
hold it for long periods of time. To avoid this phenomenon,
the pilot should plan in advance when and where the
programming for approaches, route changes, and airport
information gathering should be accomplished, as well as
times it should not. Pilot familiarity with the equipment, the
route, the local ATC environment, and personal capabilities
vis-à-vis the automation should drive when, where, and how
the automation is programmed and used.
The pilot should also consider what his or her capabilities
are in response to last minute changes of the approach (and
the reprogramming required) and ability to make large-scale
changes (a reroute for instance) while hand flying the aircraft.
Since formats are not standardized, simply moving from one
manufacturer's equipment to another should give the pilot
pause and require more conservative planning and decisions.

Purpose

Strategic

Used in a complex operation (e.g.,
introduction of new equipment); involves
research, use of analysis tools, formal
testing, or long term tracking of risks.

The SRM process is simple. At least five times before
and during the flight, the pilot should review and consider
the "Plan, the Plane, the Pilot, the Passengers, and the
Programming" and make the appropriate decision required
by the current situation. It is often said that failure to make
a decision is a decision. Under SRM and the 5 Ps, even the
decision to make no changes to the current plan is made
through a careful consideration of all the risk factors present.
Perceive, Process, Perform (3P) Model
The Perceive, Process, Perform (3P) model for ADM offers
a simple, practical, and systematic approach that can be used
during all phases of flight. To use it, the pilot will:


Perceive the given set of circumstances for a flight



Process by evaluating their impact on flight safety



Perform by implementing the best course of action

Use the Perceive, Process, Perform, and Evaluate method as
a continuous model for every aeronautical decision that you
make. Although human beings will inevitably make mistakes,
anything that you can do to recognize and minimize potential
threats to your safety will make you a better pilot.
Depending upon the nature of the activity and the time
available, risk management processing can take place in
any of three timeframes. [Figure 2-10] Most flight training
activities take place in the "time-critical" timeframe for
risk management. The six steps of risk management can be
combined into an easy-to-remember 3P model for practical
risk management: Perceive, Process, Perform with the PAVE,
CARE and TEAM checklists. Pilots can help perceive
hazards by using the PAVE checklist of: Pilot, Aircraft,
enVironment, and External pressures. They can process
hazards by using the CARE checklist of: Consequences,
Alternatives, Reality, External factors. Finally, pilots can
perform risk management by using the TEAM choice list
of: Transfer, Eliminate, Accept, or Mitigate.

PAVE Checklist: Identify Hazards and Personal
Minimums
In the first step, the goal is to develop situational awareness
by perceiving hazards, which are present events, objects, or
circumstances that could contribute to an undesired future
event. In this step, the pilot will systematically identify and

Deliberate

Time-Critical

Uses experience and brainstorming to identify "On the fly" mental or verbal review using
hazards, assess risks, and develop controls
the basic risk management process
for planning operations, review of standard
during the execution phase of an activity.
operating or training procedures, etc.

Figure 2-10. Risk management processing can take place in any of three timeframes.

2-15

list hazards associated with all aspects of the flight: Pilot,
Aircraft, enVironment, and External pressures, which makes
up the PAVE checklist. [Figure 2-11] For each element,
ask "what could hurt me, my passengers, or my aircraft?"
All four elements combine and interact to create a unique
situation for any flight. Pay special attention to the pilotaircraft combination, and consider whether the combined
"pilot-aircraft team" is capable of the mission you want to fly.
For example, you may be a very experienced and proficient
pilot, but your weather flying ability is still limited if you
are flying a 1970s-model aircraft with no weather avoidance
gear. On the other hand, you may have a new technically
advanced aircraft with moving map GPS, weather datalink,
and autopilot—but if you do not have much weather flying
experience or practice in using this kind of equipment, you
cannot rely on the airplane's capability to compensate for
your own lack of experience.

CARE Checklist: Review Hazards and Evaluate Risks
In the second step, the goal is to process this information to
determine whether the identified hazards constitute risk, which
is defined as the future impact of a hazard that is not controlled
or eliminated. The degree of risk posed by a given hazard
can be measured in terms of exposure (number of people or
resources affected), severity (extent of possible loss), and
probability (the likelihood that a hazard will cause a loss). The
goal is to evaluate their impact on the safety of your flight,
and consider "why must I CARE about these circumstances?"
For each hazard that you perceived in step one, process by
using the CARE checklist of: Consequences, Alternatives,
Reality, External factors. [Figure 2-12] For example, let's
evaluate a night flight to attend a business meeting:
Consequences—departing after a full workday creates
fatigue and pressure

Pilots can perceive hazards by using the PAVE checklist:

Pilot
Gayle is a healthy and well-rested private pilot with
approximately 300 hours total flight time. Hazards include her
lack of overall and cross-country experience and the fact that
she has not flown at all in 2 months.
Aircraft
Although it does not have a panel-mount GPS or weather

avoidance gear, the aircraft—a C182 Skylane with long-range

fuel tanks—is in good mechanical condition with no inoperative

equipment. The instrument panel is a standard "six-pack."


Alternatives—delay until morning; reschedule
meeting; drive
Reality —dangers and distractions of fatigue could
lead to an accident
External pressures—business meeting at destination
might influence me
A good rule of thumb for the processing phase: if you find
yourself saying that it will "probably" be okay, it is definitely
time for a solid reality check. If you are worried about missing
a meeting, be realistic about how that pressure will affect
not just your initial go/no-go decision, but also your inflight
decisions to continue the flight or divert.

TEAM Checklist: Choose and Implement Risk
Controls
Once you have perceived a hazard (step one) and processed
its impact on flight safety (step two), it is time to move to the
third step, perform. Perform risk management by using the
TEAM checklist of: Transfer, Eliminate, Accept, Mitigate
to deal with each factor. [Figure 2-13]
Transfer—Should this risk decision be transferred to
someone else (e.g., do you need to consult the chief
flight instructor?)
Eliminate—Is there a way to eliminate the hazard?
Accept—Do the benefits of accepting risk outweigh
the costs?
Mitigate—What can you do to mitigate the risk?
The goal is to perform by taking action to eliminate hazards
or mitigate risk, and then continuously evaluate the outcome
of this action. With the example of low ceilings at destination,
for instance, the pilot can perform good ADM by selecting
a suitable alternate, knowing where to find good weather,
EnVironment
Departure and destination airports have long runways.
Weather is the main hazard. Although it is VFR, it is a typical
summer day in the Mid-Atlantic region: hot (near 90 °F) hazy
(visibility 7 miles), and humid with a density altitude of 2,500
feet. Weather at the destination airport (located in the
mountains) is still IMC but forecast to improve to visual
meteorological conditions (VMC) prior to her arrival. En route
weather is VMC, but there is an AIRMET Sierra for pockets of
IMC over mountain ridges along the proposed route of flight.
External pressures
Gayle is making the trip to spend a weekend with relatives she
does not see very often. Her family is very excited and has
made a number of plans for the visit.

Figure 2-11. A real-world example of how the 3P model guides decisions on a cross-country trip using the PAVE checklist.

2-16

Pilots can perceive hazards by using the CARE checklist:
Pilot
 Consequences: Gayle's inexperience and lack of recent
flight time create some risks for an accident, primarily because
she plans to travel over mountains on a hazy day and land
at an unfamiliar mountain airport that is still in IMC
conditions.
 Alternatives: Gayle might mitigate the pilot-related risk by
hiring a CFI to accompany her and provide dual crosscountry instruction. An added benefit is the opportunity to
broaden her flying experience in safe conditions.
 Reality: Accepting the reality that limited experience can
create additional risks is a key part of sound risk management
and mitigation.
 External Factors: Like many pilots, Gayle must contend with
the emotional pressure associated with acknowledging that
her skill and experience levels may be lower than she would
like them to be. Pride can be a powerful external factor!
Environment
 	 Consequences: For a pilot whose experience consists
mostly of local flights in good VMC, launching a long crosscountry flight over mountainous terrain in hazy conditions
could lead to pilot disorientation and increase the risk of an
accident.
 	 Alternatives: Options include postponing the trip until the
visibility improves, or modifying the route to avoid extended
periods of time over the mountains.
 	 Reality: Hazy conditions and mountainous terrain clearly
create risks for an inexperienced VFR-only pilot.
 	 External Factors: Few pilots are immune to the pressure of
"get-there-itis," which can sometimes induce a decision to
launch or continue in less than ideal weather conditions.

Aircraft
 	 Consequences: This area presents low risk because the
aircraft is in excellent mechanical condition and Gayle is
familiar with its avionics.
 	 Alternatives: Had there been a problem with her aircraft,
Gayle might have considered renting another plane from her
flight school. Bear in mind, however, that alternatives
sometimes create new hazards. In this instance, there may
be hazards associated with flying an unfamiliar aircraft with
different avionics.
 	 Reality: It is important to recognize the reality of an aircraft's
mechanical condition. If you find a maintenance discrepancy
and then find yourself saying that it is "probably" okay to fly
with it anyway, you need to revisit the consequences part of
this checklist.
 	 External Factors: Pilot decision-making can sometimes be
influenced by the external pressure of needing to return the
airplane to the FBO by a certain date and time. Because
Gayle owns the airplane, there was no such pressure in this
case.
External pressures
 	 Consequences: Any number of factors can create the risk of
emotional pressure from a "get-there" mentality. In Gayle's
case, the consequences of her strong desire to visit family,
her family's expectations, and personal pride could induce
her to accept unnecessary risks.
 	 Alternatives: Gayle clearly needs to develop a mitigating
strategy for each of the external factors associated with this
trip.
 	 Reality: Pilots sometimes tend to discount or ignore the
potential impact of these external factors. Gayle's open
acknowledgement of these factors (e.g., "I might be
pressured into pressing on so my mother won't have to
worry about our late arrival.") is a critical element of effective
risk management.
 	 External Factors: (see above)

Figure 2-12. A real-world examples of how the 3P model guides decisions on a cross-country trip using the CARE checklist.

and carrying sufficient fuel to reach it. This course of action
would mitigate the risk. The pilot also has the option to
eliminate it entirely by waiting for better weather.
Once the pilot has completed the 3P decision process and
selected a course of action, the process begins anew because
now the set of circumstances brought about by the course of
action requires analysis. The decision-making process is a
continuous loop of perceiving, processing, and performing.
With practice and consistent use, running through the 3P
cycle can become a habit that is as smooth, continuous, and
automatic as a well-honed instrument scan. This basic set
of practical risk management tools can be used to improve
risk management.
Your mental willingness to follow through on safe decisions,
especially those that require delay or diversion is critical. You
can bulk up your mental muscles by:



Using personal minimums checklist to make some
decisions in advance of the flight. To develop a good
personal minimums checklist, you need to assess your
abilities and capabilities in a non-flying environment,
when there is no pressure to make a specific trip. Once
developed, a personal minimums checklist will give
you a clear and concise reference point for making
your go/no-go or continue/discontinue decisions.



In addition to having personal minimums, some pilots
also like to use a preflight risk assessment checklist to
help with the ADM and risk management processes.
This kind of form assigns numbers to certain risks
and situations, which can make it easier to see when
a particular flight involves a higher level of risk



Develop a list of good alternatives during your
processing phase. In marginal weather, for instance,
you might mitigate the risk by identifying a reasonable

2-17

Pilots can perform risk management by using the TEAM checklist:
Pilot
To manage the risk associated with her inexperience and lack
of recent flight time, Gayle can:
 Transfer the risk entirely by having another pilot act as PIC.
 Eliminate the risk by canceling the trip.
 Accept the risk and fly anyway.
 Mitigate the risk by flying with another pilot.
Gayle chooses to mitigate the major risk by hiring a CFI to
accompany her and provide dual cross-country instruction.
An added benefit is the opportunity to broaden her flying
experience.

Aircraft
To manage risk associated with any doubts about the aircraft's
mechanical condition, Gayle can:
 	 Transfer the risk by using a different airplane.
 	 Eliminate the risk by canceling the trip.
 	 Accept the risk.
 	 Mitigate the remaining (residual) risk through review of
aircraft performance and careful preflight inspection.
Since she finds no problems with the aircraft's mechanical
condition, Gayle chooses to mitigate any remaining risk
through careful preflight inspection of the aircraft.

Environment
To manage the risk associated with hazy conditions and
mountainous terrain, Gayle can:
 	 Transfer the risk of VFR in these conditions by asking an
instrument-rated pilot to fly the trip under IFR.
 	 Eliminate the risk by canceling the trip.
 	 Accept the risk.
 	 Mitigate the risk by careful preflight planning, filing a VFR
flight plan, requesting VFR flight following, and using
resources such as Flight Watch.

External pressures
To mitigate the risk of emotional pressure from family
expectations that can drive a "get-there" mentality, Gayle can:
 	 Transfer the risk by having her co-pilot act as PIC and make
the continue/divert decision.
 	 Eliminate the risk by canceling the trip.
 	 Accept the risk.
 	 Mitigate the risk by managing family expectations and
making alternative arrangements in the event of diversion to
another airport.

Detailed preflight planning must be a vital part of Gayle's
weather risk mitigation strategy. The most direct route would
put her over mountains for most of the trip. Because of the
thick haze and pockets of IMC over mountains, Gayle might
mitigate the risk by modifying the route to fly over valleys. This
change will add 30 minutes to her estimated time of arrival
(ETA), but the extra time is a small price to pay for avoiding
possible IMC over mountains. Because her destination airport
is IMC at the time of departure, Gayle needs to establish that
VFR conditions exist at other airports within easy driving
distance of her original destination. In addition, Gayle should
review basic information (e.g., traffic pattern altitude, runway
layout, frequencies) for these alternate airports. To further
mitigate risk and practice good cockpit resource management,
Gayle should file a VFR flight plan, use VFR flight following,
and call Flight Watch to get weather updates en route. Finally,
basic functions on her handheld GPS should also be practiced.

Gayle and her co-pilot choose to address this risk by agreeing
that each pilot has a veto on continuing the flight, and that they
will divert if either becomes uncomfortable with flight conditions.
Because the destination airport is still IMC at the time of
departure, Gayle establishes a specific point in the trip—an en
route VORTAC located between the destination airport and the
two alternates—as the logical place for her "final" continue/
divert decision. Rather than give her family a specific ETA that
might make Gayle feel pressured to meet the schedule, she
manages her family's expectations by advising them that she
will call when she arrives.

Figure 2-13. A real-world example of how the 3P model guides decisions on a cross-country trip using the TEAM checklist.

alternative airport for every 25–30 nautical mile
segment of your route.
 	 Preflight your passengers by preparing them for the
possibility of delay and diversion, and involve them
in your evaluation process.


2-18

Another important tool—overlooked by many pilots—
is a good post-flight analysis. When you have safely
secured the airplane, take the time to review and
analyze the flight as objectively as you can. Mistakes
and judgment errors are inevitable; the most important
thing is for you to recognize, analyze, and learn from
them before your next flight.

The DECIDE Model
Using the acronym "DECIDE," the six-step process DECIDE
Model is another continuous loop process that provides the
pilot with a logical way of making decisions. [Figure 2-14]
DECIDE means to Detect, Estimate, Choose a course of
action, Identify solutions, Do the necessary actions, and
Evaluate the effects of the actions.
First, consider a recent accident involving a Piper Apache (PA­
23). The aircraft was substantially damaged during impact
with terrain at a local airport in Alabama. The certificated
airline transport pilot (ATP) received minor injuries and the
certificated private pilot was not injured. The private pilot

was receiving a checkride from the ATP (who was also a
designated examiner) for a commercial pilot certificate with
a multi-engine rating. After performing airwork at altitude,
they returned to the airport and the private pilot performed a
single-engine approach to a full stop landing. He then taxied
back for takeoff, performed a short field takeoff, and then
joined the traffic pattern to return for another landing. During
the approach for the second landing, the ATP simulated a right

engine failure by reducing power on the right engine to zero
thrust. This caused the aircraft to yaw right.
The procedure to identify the failed engine is a two-step
process. First, adjust the power to the maximum controllable
level on both engines. Because the left engine is the only
engine delivering thrust, the yaw increases to the right, which
necessitates application of additional left rudder application.

Aeronautical Decision-Making
A. Analytical

B. Automatic/Naturalistic

Situation

Pilot

Enviroment

Aircraft

Pilot

Aircraft

Enviroment

External factors

External factors

Detection

Detection

Evaluation of event
Evaluation of event






Risk or hazard
Potential outcomes
Capabilities of pilot
Aircraft capabilities
Outside factors

 Risk to flight
 Pilot training
 Pilot experience

Outcome desired

Solutions to get you there
Solution 1
Solution 2
Solution 3
Solution 4

Outcome desired

Take action
What is best action to do

Effect of decision
Successful
Problem remains

Done

The DECIDE model
1. Detect. The decision maker detects the fact that change has occurred.
2. Estimate. The decision maker estimates the need to counter or react to the change.
3. Choose. The decision maker chooses a desirable outcome (in terms of success) for the flight.
4. Identify. The decision maker identifies actions which could successfully control the change.
5. Do. The decision maker takes the necessary action.
6. Evaluate. The decision maker evaluates the effect(s) of his/her action countering the change.

Figure 2-14. The DECIDE model has been recognized worldwide. Its application is illustrated in column A while automatic/naturalistic

decision-making is shown in column B.

2-19

The failed engine is the side that requires no rudder pressure,
in this case the right engine. Second, having identified the
failed right engine, the procedure is to feather the right engine
and adjust power to maintain descent angle to a landing.
However, in this case the pilot feathered the left engine because
he assumed the engine failure was a left engine failure. During
twin-engine training, the left engine out is emphasized more
than the right engine because the left engine on most light
twins is the critical engine. This is due to multiengine airplanes
being subject to P-factor, as are single-engine airplanes.
The descending propeller blade of each engine will produce
greater thrust than the ascending blade when the airplane is
operated under power and at positive angles of attack. The
descending propeller blade of the right engine is also a greater
distance from the center of gravity, and therefore has a longer
moment arm than the descending propeller blade of the left
engine. As a result, failure of the left engine will result in the
most asymmetrical thrust (adverse yaw) because the right
engine will be providing the remaining thrust. Many twins are
designed with a counter-rotating right engine. With this design,
the degree of asymmetrical thrust is the same with either engine
inoperative. Neither engine is more critical than the other.
Since the pilot never executed the first step of identifying
which engine failed, he feathered the left engine and set the
right engine at zero thrust. This essentially restricted the
aircraft to a controlled glide. Upon realizing that he was
not going to make the runway, the pilot increased power to
both engines causing an enormous yaw to the left (the left
propeller was feathered) whereupon the aircraft started to turn
left. In desperation, the instructor closed both throttles and
the aircraft hit the ground and was substantially damaged.

detecting the problem. In the previous example, the change
that occurred was a yaw.

Estimate (the Need To React)
In the engine-out example, the aircraft yawed right, the pilot
was on final approach, and the problem warranted a prompt
solution. In many cases, overreaction and fixation excludes
a safe outcome. For example, what if the cabin door of a
Mooney suddenly opened in flight while the aircraft climbed
through 1,500 feet on a clear sunny day? The sudden opening
would be alarming, but the perceived hazard the open door
presents is quickly and effectively assessed as minor. In
fact, the door's opening would not impact safe flight and
can almost be disregarded. Most likely, a pilot would return
to the airport to secure the door after landing.
The pilot flying on a clear day faced with this minor problem
may rank the open cabin door as a low risk. What about
the pilot on an IFR climb out in IMC conditions with light
intermittent turbulence in rain who is receiving an amended
clearance from ATC? The open cabin door now becomes
a higher risk factor. The problem has not changed, but the
perception of risk a pilot assigns it changes because of the
multitude of ongoing tasks and the environment. Experience,
discipline, awareness, and knowledge influences how a pilot
ranks a problem.

Choose (a Course of Action)
After the problem has been identified and its impact
estimated, the pilot must determine the desirable outcome
and choose a course of action. In the case of the multiengine
pilot given the simulated failed engine, the desired objective
is to safely land the airplane.

This case is interesting because it highlights two particular
issues. First, taking action without forethought can be just
as dangerous as taking no action at all. In this case, the
pilot's actions were incorrect; yet, there was sufficient
time to take the necessary steps to analyze the simulated
emergency. The second and more subtle issue is that decisions
made under pressure are sometimes executed based upon
limited experience and the actions taken may be incorrect,
incomplete, or insufficient to handle the situation.

Identify (Solutions)

Detect (the Problem)

Do (the Necessary Actions)

Problem detection is the first step in the decision-making
process. It begins with recognizing a change occurred or an
expected change did not occur. A problem is perceived first
by the senses and then it is distinguished through insight
and experience. These same abilities, as well as an objective
analysis of all available information, are used to determine
the nature and severity of the problem. One critical error
made during the decision-making process is incorrectly

Evaluate (the Effect of the Action)

2-20

The pilot formulates a plan that will take him or her to the
objective. Sometimes, there may be only one course of action
available. In the case of the engine failure already at 500
feet or below, the pilot solves the problem by identifying
one or more solutions that lead to a successful outcome. It is
important for the pilot not to become fixated on the process
to the exclusion of making a decision.

Once pathways to resolution are identified, the pilot selects the
most suitable one for the situation. The multiengine pilot given
the simulated failed engine must now safely land the aircraft.

Finally, after implementing a solution, evaluate the decision
to see if it was correct. If the action taken does not provide
the desired results, the process may have to be repeated.

Decision-Making in a Dynamic Environment pitfalls that come with the development of pilot experience.
A solid approach to decision-making is through the use of
analytical models, such as the 5 Ps, 3P, and DECIDE. Good
decisions result when pilots gather all available information,
review it, analyze the options, rate the options, select a course
of action, and evaluate that course of action for correctness.
In some situations, there is not always time to make decisions
based on analytical decision-making skills. A good example
is a quarterback whose actions are based upon a highly fluid
and changing situation. He intends to execute a plan, but new
circumstances dictate decision-making on the fly. This type
of decision-making is called automatic decision-making or
naturalized decision-making. [Figure 2-14B]
Automatic Decision-Making
In an emergency situation, a pilot might not survive if he or
she rigorously applies analytical models to every decision
made as there is not enough time to go through all the options.
Under these circumstances he or she should attempt to find
the best possible solution to every problem.
For the past several decades, research into how people
actually make decisions has revealed that when pressed for
time, experts faced with a task loaded with uncertainty first
assess whether the situation strikes them as familiar. Rather
than comparing the pros and cons of different approaches,
they quickly imagine how one or a few possible courses of
action in such situations will play out. Experts take the first
workable option they can find. While it may not be the best of
all possible choices, it often yields remarkably good results.
The terms "naturalistic" and "automatic decision-making"
have been coined to describe this type of decision-making.
The ability to make automatic decisions holds true for a
range of experts from firefighters to chess players. It appears
the expert's ability hinges on the recognition of patterns and
consistencies that clarify options in complex situations. Experts
appear to make provisional sense of a situation, without
actually reaching a decision, by launching experience-based
actions that in turn trigger creative revisions.
This is a reflexive type of decision-making anchored in
training and experience and is most often used in times of
emergencies when there is no time to practice analytical
decision-making. Naturalistic or automatic decision-making
improves with training and experience, and a pilot will find
himself or herself using a combination of decision-making
tools that correlate with individual experience and training.

Operational Pitfalls
Although more experienced pilots are likely to make more
automatic decisions, there are tendencies or operational

These are classic behavioral traps into which pilots have
been known to fall. More experienced pilots, as a rule, try
to complete a flight as planned, please passengers, and meet
schedules. The desire to meet these goals can have an adverse
effect on safety and contribute to an unrealistic assessment
of piloting skills. All experienced pilots have fallen prey to,
or have been tempted by, one or more of these tendencies in
their flying careers. These dangerous tendencies or behavior
patterns, which must be identified and eliminated, include
the operational pitfalls shown in Figure 2-15.
Stress Management
Everyone is stressed to some degree almost all of the time. A
certain amount of stress is good since it keeps a person alert
and prevents complacency. Effects of stress are cumulative
and, if the pilot does not cope with them in an appropriate
way, they can eventually add up to an intolerable burden.
Performance generally increases with the onset of stress,
peaks, and then begins to fall off rapidly as stress levels
exceed a person's ability to cope. The ability to make
effective decisions during flight can be impaired by stress.
There are two categories of stress—acute and chronic. These
are both explained in Chapter 17, "Aeromedical Factors."
Factors referred to as stressors can increase a pilot's risk of
error in the flight deck. [Figure 2-16] Remember the cabin
door that suddenly opened in flight on the Mooney climbing
through 1,500 feet on a clear sunny day? It may startle the
pilot, but the stress would wane when it became apparent
the situation was not a serious hazard. Yet, if the cabin door
opened in IMC conditions, the stress level makes significant
impact on the pilot's ability to cope with simple tasks. The
key to stress management is to stop, think, and analyze before
jumping to a conclusion. There is usually time to think before
drawing unnecessary conclusions.
There are several techniques to help manage the accumulation
of life stresses and prevent stress overload. For example, to
help reduce stress levels, set aside time for relaxation each
day or maintain a program of physical fitness. To prevent
stress overload, learn to manage time more effectively to
avoid pressures imposed by getting behind schedule and not
meeting deadlines.
Use of Resources
To make informed decisions during flight operations, a pilot
must also become aware of the resources found inside and
outside the flight deck. Since useful tools and sources of
information may not always be readily apparent, learning
to recognize these resources is an essential part of ADM
training. Resources must not only be identified, but a pilot
must also develop the skills to evaluate whether there is
2-21

OperationalPitfalls
pitfalls
Operational
Peer pressure
Poor decision-making may be based upon an emotional response to peers, rather than evaluating a situation objectively.
Mindset
A pilot displays mind set through an inability to recognize and cope with changes in a given situation.
Get-there-itis
This disposition impairs pilot judgment through a fixation on the original goal or destination, combined with a disregard for any
alternative course of action.
Duck-under syndrome
A pilot may be tempted to make it into an airport by descending below minimums during an approach. There may be a belief that
there is a built-in margin of error in every approach procedure, or a pilot may want to admit that the landing cannot be completed
and a missed approach must be initiated.
Scud running
This occurs when a pilot tries to maintain visual contact with the terrain at low altitudes while instrument conditions exist.
Continuing visual flight rules (VFR) into instrument conditions
Spatial disorientation or collision with ground/obstacles may occur when a pilot continues VFR into instrument conditions. This can
be even more dangerous if the pilot is not instrument rated or current.
Getting behind the aircraft
This pitfall can be caused by allowing events or the situation to control pilot actions. A constant state of surprise at what happens
next may be exhibited when the pilot is getting behind the aircraft.
Loss of positional or situational awareness
In extreme cases, when a pilot gets behind the aircraft, a loss of positional or situational awareness may result. The pilot may not
know the aircraft's geographical location or may be unable to recognize deteriorating circumstances.
Operating without adequate fuel reserves
Ignoring minimum fuel reserve requirements is generally the result of overconfidence, lack of flight planning, or disregarding
applicable regulations.
Descent below the minimum en route altitude
The duck-under syndrome, as mentioned above, can also occur during the en route portion of an IFR flight.
Flying outside the envelope
The assumed high performance capability of a particular aircraft may cause a mistaken belief that it can meet the demands
imposed by a pilot's overestimated flying skills.
Neglect of flight planning, preflight inspections, and checklists
A pilot may rely on short- and long-term memory, regular flying skills, and familiar routes instead of established procedures and
published checklists. This can be particularly true of experienced pilots.
Figure 2-15. Typical operational pitfalls requiring pilot awareness.
Stressors
Environmental
Conditions associated with the environment, such as temperature and humidity extremes, noise, vibration, and lack of oxygen.
Physiological stress
Physical conditions, such as fatigue, lack of physical fitness, sleep loss, missed meals (leading to low blood sugar levels), and
illness.
Psychological stress
Social or emotional factors, such as a death in the family, a divorce, a sick child, or a demotion at work. This type of stress may
also be related to mental workload, such as analyzing a problem, navigating an aircraft, or making decisions.
Figure 2-16. System stressors. Environmental, physiological, and psychological stress are factors that affect decision-making skills.

These stressors have a profound impact especially during periods of high workload.

2-22

time to use a particular resource and the impact its use will
have upon the safety of flight. For example, the assistance
of ATC may be very useful if a pilot becomes lost, but in
an emergency situation, there may be no time available to
contact ATC.

Internal Resources
One of the most underutilized resources may be the
person in the right seat, even if the passenger has no flying
experience. When appropriate, the PIC can ask passengers
to assist with certain tasks, such as watching for traffic or
reading checklist items. The following are some other ways
a passenger can assist:


Provide information in an irregular situation,
especially if familiar with flying. A strange smell or
sound may alert a passenger to a potential problem.

 	 Confirm after the pilot that the landing gear is down.
 	 Learn to look at the altimeter for a given altitude in a
descent.
 	 Listen to logic or lack of logic.
Also, the process of a verbal briefing (which can happen
whether or not passengers are aboard) can help the PIC in
the decision-making process. For example, assume a pilot
provides a lone passenger a briefing of the forecast landing
weather before departure. When the Automatic Terminal
Information Service (ATIS) is picked up, the weather
has significantly changed. The discussion of this forecast
change can lead the pilot to reexamine his or her activities
and decision-making. [Figure 2-17] Other valuable internal
resources include ingenuity, aviation knowledge, and flying
skill. Pilots can increase flight deck resources by improving
these characteristics.
When flying alone, another internal resource is verbal
communication. It has been established that verbal
communication reinforces an activity; touching an object
while communicating further enhances the probability an
activity has been accomplished. For this reason, many solo
pilots read the checklist out loud; when they reach critical
items, they touch the switch or control. For example, to
ascertain the landing gear is down, the pilot can read the
checklist. But, if he or she touches the gear handle during the
process, a safe extension of the landing gear is confirmed.
It is necessary for a pilot to have a thorough understanding
of all the equipment and systems in the aircraft being flown.
Lack of knowledge, such as knowing if the oil pressure
gauge is direct reading or uses a sensor, is the difference
between making a wise decision or poor one that leads to
a tragic error.

Figure 2-17. When possible, have a passenger reconfirm that critical
tasks are completed.

Checklists are essential flight deck internal resources. They
are used to verify the aircraft instruments and systems are
checked, set, and operating properly, as well as ensuring
the proper procedures are performed if there is a system
malfunction or in-flight emergency. Students reluctant to
use checklists can be reminded that pilots at all levels of
experience refer to checklists, and that the more advanced the
aircraft is, the more crucial checklists become. In addition, the
pilot's operating handbook (POH) is required to be carried on
board the aircraft and is essential for accurate flight planning
and resolving in-flight equipment malfunctions. However,
the most valuable resource a pilot has is the ability to manage
workload whether alone or with others.

External Resources
ATC and flight service specialists are the best external
resources during flight. In order to promote the safe, orderly
flow of air traffic around airports and, along flight routes, the
ATC provides pilots with traffic advisories, radar vectors,
and assistance in emergency situations. Although it is the
PIC's responsibility to make the flight as safe as possible,
a pilot with a problem can request assistance from ATC.
[Figure 2-18] For example, if a pilot needs to level off, be

Figure 2-18. Controllers work to make flights as safe as possible.

2-23

given a vector, or decrease speed, ATC assists and becomes
integrated as part of the crew. The services provided by ATC
can not only decrease pilot workload, but also help pilots
make informed in-flight decisions.
The Flight Service Stations (FSSs) are air traffic facilities
that provide pilot briefing, en route communications, VFR
search and rescue services, assist lost aircraft and aircraft
in emergency situations, relay ATC clearances, originate
Notices to Airmen (NOTAM), broadcast aviation weather
and National Airspace System (NAS) information, receive
and process IFR flight plans, and monitor navigational aids
(NAVAIDs). In addition, at selected locations, FSSs provide
En Route Flight Advisory Service (Flight Watch), issue
airport advisories, and advise Customs and Immigration of
transborder flights. Selected FSSs in Alaska also provide
TWEB recordings and take weather observations.

Situational Awareness
Situational awareness is the accurate perception and
understanding of all the factors and conditions within
the five fundamental risk elements (flight, pilot, aircraft,
environment, and type of operation that comprise any given
aviation situation) that affect safety before, during, and after
the flight. Monitoring radio communications for traffic,
weather discussion, and ATC communication can enhance
situational awareness by helping the pilot develop a mental
picture of what is happening.
Maintaining situational awareness requires an understanding
of the relative significance of all flight related factors and their
future impact on the flight. When a pilot understands what is
going on and has an overview of the total operation, he or she
is not fixated on one perceived significant factor. Not only
is it important for a pilot to know the aircraft's geographical
location, it is also important he or she understand what is
happening. For instance, while flying above Richmond,
Virginia, toward Dulles Airport or Leesburg, the pilot
should know why he or she is being vectored and be able to
anticipate spatial location. A pilot who is simply making turns
without understanding why has added an additional burden
to his or her management in the event of an emergency. To
maintain situational awareness, all of the skills involved in
ADM are used.
Obstacles to Maintaining Situational Awareness
Fatigue, stress, and work overload can cause a pilot to fixate
on a single perceived important item and reduce an overall
situational awareness of the flight. A contributing factor
in many accidents is a distraction that diverts the pilot's
attention from monitoring the instruments or scanning
outside the aircraft. Many flight deck distractions begin as a
minor problem, such as a gauge that is not reading correctly,
2-24

but result in accidents as the pilot diverts attention to the
perceived problem and neglects proper control of the aircraft.

Workload Management
Effective workload management ensures essential operations
are accomplished by planning, prioritizing, and sequencing
tasks to avoid work overload. [Figure 2-19] As experience
is gained, a pilot learns to recognize future workload
requirements and can prepare for high workload periods
during times of low workload. Reviewing the appropriate
chart and setting radio frequencies well in advance of when
they are needed helps reduce workload as the flight nears the
airport. In addition, a pilot should listen to ATIS, Automated
Surface Observing System (ASOS), or Automated Weather
Observing System (AWOS), if available, and then monitor
the tower frequency or Common Traffic Advisory Frequency
(CTAF) to get a good idea of what traffic conditions to
expect. Checklists should be performed well in advance so
there is time to focus on traffic and ATC instructions. These
procedures are especially important prior to entering a highdensity traffic area, such as Class B airspace.
Recognizing a work overload situation is also an important
component of managing workload. The first effect of
high workload is that the pilot may be working harder but
accomplishing less. As workload increases, attention cannot
be devoted to several tasks at one time, and the pilot may
begin to focus on one item. When a pilot becomes task
saturated, there is no awareness of input from various sources,
so decisions may be made on incomplete information and the
possibility of error increases. [Figure 2-20]
When a work overload situation exists, a pilot needs to stop,
think, slow down, and prioritize. It is important to understand
how to decrease workload. For example, in the case of the
cabin door that opened in VFR flight, the impact on workload
should be insignificant. If the cabin door opens under
IFR different conditions, its impact on workload changes.
Therefore, placing a situation in the proper perspective,

Figure 2-19. Balancing workloads can be a difficult task.

remaining calm, and thinking rationally are key elements in
reducing stress and increasing the capacity to fly safely. This
ability depends upon experience, discipline, and training.



In addition to the SAFETY list, discuss with
passengers whether or not smoking is permitted, flight
route altitudes, time en route, destination, weather
during flight, expected weather at the destination,
controls and what they do, and the general capabilities
and limitations of the aircraft.



Use a sterile flight deck (one that is completely silent
with no pilot communication with passengers or by
passengers) from the time of departure to the first
intermediate altitude and clearance from the local
airspace.



Use a sterile flight deck during arrival from the first
radar vector for approach or descent for the approach.



Keep the passengers informed during times when the
workload is low.



Consider using the passenger in the right seat for
simple tasks, such as holding the chart. This relieves
the pilot of a task.

Managing Risks
The ability to manage risks begins with preparation. Here
are some things a pilot can do to manage risks:


Assess the flight's risk based upon experience. Use
some form of risk assessment. For example, if the
weather is marginal and the pilot has little IMC
training, it is probably a good idea to cancel the flight.



Brief passengers using the SAFETY list:
S	 Seat belts fastened for taxi, takeoff, landing
Shoulder harness fastened for takeoff, landing
Seat position adjusted and locked in place
A	 Air vents (location and operation)
All environmental controls (discussed)
Action in case of any passenger discomfort
F	 Fire extinguisher (location and operation)
E	 Exit doors (how to secure; how to open)
Emergency evacuation plan
Emergency/survival kit (location and contents)
T	 Traffic (scanning, spotting, notifying pilot)
Talking, ("sterile flight deck" expectations)
Y

Your questions? (Speak up!)

Automation
In the GA community, an automated aircraft is generally
comprised of an integrated advanced avionics system
consisting of a primary flight display (PFD), a multifunction
flight display (MFD) including an instrument-certified global
positioning system (GPS) with traffic and terrain graphics,
and a fully integrated autopilot. This type of aircraft is
commonly known as a technically advanced aircraft (TAA).
In a TAA aircraft, there are typically two display (computer)
screens: PFD (left display screen) and MFD.

Task load

High

Low

ff

Time

ments

Takeo

equire

ht

Pilot

s

Cruise
Appro

ach \&

landin

g

Task r

Preflig

tie
i
l
i
b
a
cap

Figure 2-20. The pilot has a certain capacity of doing work and handling tasks. However, there is a point where the tasking exceeds the
pilot's capability. When this happens, tasks are either not performed properly or some are not performed at all.

2-25

Automation is the single most important advance in aviation
technologies. Electronic flight displays (EFDs) have made
vast improvements in how information is displayed and
what information is available to the pilot. Pilots can access
electronic databases that contain all of the information
traditionally contained in multiple handbooks, reducing
clutter in the flight deck. [Figure 2-21]
MFDs are capable of displaying moving maps that mirror
sectional charts. These detailed displays depict all airspace,
including Temporary Flight Restrictions (TFRs). MFDs are
so descriptive that many pilots fall into the trap of relying
solely on the moving maps for navigation. Pilots also draw
upon the database to familiarize themselves with departure
and destination airport information.
More pilots now rely on electronic databases for flight
planning and use automated flight planning tools rather
than planning the flight by the traditional methods of laying
out charts, drawing the course, identifying navigation
points (assuming a VFR flight), and using the POH to
figure out the weight and balance and performance charts.
Whichever method a pilot chooses to plan a flight, it is

important to remember to check and confirm calculations.
Always remember that it is up to the pilot to maintain basic
airmanship skills and use those skills often to maintain
proficiency in all tasks.
Although automation has made flying safer, automated
systems can make some errors more evident and sometimes
hide other errors or make them less evident. There are
concerns about the effect of automation on pilots. In a study
published in 1995, the British Airline Pilots Association
officially voiced its concern that "Airline pilots increasingly
lack ‘basic flying skills' as a result of reliance on automation."
This reliance on automation translates into a lack of basic flying
skills that may affect the pilot's ability to cope with an in-flight
emergency, such as sudden mechanical failure. The worry that
pilots are becoming too reliant on automated systems and are
not being encouraged or trained to fly manually has grown
with the increase in the number of MFD flight decks.
As automated flight decks began entering everyday line
operations, instructors and check airmen grew concerned
about some of the unanticipated side effects. Despite the
promise of reducing human mistakes, the flight managers
reported the automation actually created much larger errors
at times. In the terminal environment, the workload in an
automated flight deck actually seemed higher than in the older
analog flight decks. At other times, the automation seemed
to lull the flight crews into complacency. Over time, concern
surfaced that the manual flying skills of the automated flight
crews deteriorated due to over-reliance on computers. The
flight crew managers said they worried that pilots would
have less "stick-and-rudder" proficiency when those skills
were needed to manually resume direct control of the aircraft.
A major study was conducted to evaluate the performance
of two groups of pilots. The control group was composed of
pilots who flew an older version of a common twin-jet airliner
equipped with analog instrumentation and the experimental
group was composed of pilots who flew the same aircraft,
but newer models equipped with an electronic flight
instrument system (EFIS) and a flight management system
(FMS). The pilots were evaluated in maintaining aircraft
parameters, such as heading, altitude, airspeed, glideslope,
and localizer deviations, as well as pilot control inputs. These
were recorded during a variety of normal, abnormal, and
emergency maneuvers during 4 hours of simulator sessions.

Figure 2-21. Electronic flight instrumentation comes in many
systems and provides a myriad of information to the pilot.

2-26

Results of the Study
When pilots who had flown EFIS for several years were
required to fly various maneuvers manually, the aircraft
parameters and flight control inputs clearly showed some
erosion of flying skills. During normal maneuvers, such as
turns to headings without a flight director, the EFIS group
exhibited somewhat greater deviations than the analog group.
Most of the time, the deviations were within the practical test
standards (PTS), but the pilots definitely did not keep on the
localizer and glideslope as smoothly as the analog group.
The differences in hand-flying skills between the two groups
became more significant during abnormal maneuvers, such as
accelerated descent profiles known as "slam-dunks." When
given close crossing restrictions, the analog crews were more
adept at the mental math and usually maneuvered the aircraft
in a smoother manner to make the restriction. On the other
hand, the EFIS crews tended to go "heads down" and tried
to solve the crossing restriction on the FMS. [Figure 2-22]
Another situation used in the simulator experiment reflected
real world changes in approach that are common and can
be assigned on short notice. Once again, the analog crews
transitioned more easily to the parallel runway's localizer,
whereas the EFIS crews had a much more difficult time with
the pilot going head down for a significant amount of time
trying to program the new approach into the FMS.
While a pilot's lack of familiarity with the EFIS is often
an issue, the approach would have been made easier by
disengaging the automated system and manually flying the
approach. At the time of this study, the general guidelines
in the industry were to let the automated system do as much
of the flying as possible. That view has since changed and
it is recommended that pilots use their best judgment when
choosing which level of automation will most efficiently do
the task considering the workload and situational awareness.
Emergency maneuvers clearly broadened the difference in
manual flying skills between the two groups. In general, the
analog pilots tended to fly raw data, so when they were given
an emergency, such as an engine failure, and were instructed
to fly the maneuver without a flight director, they performed
it expertly. By contrast, SOP for EFIS operations at the time
was to use the flight director. When EFIS crews had their flight
directors disabled, their eye scan again began a more erratic
searching pattern and their manual flying subsequently suffered.
Those who reviewed the data saw that the EFIS pilots who
better managed the automation also had better flying skills.
While the data did not reveal whether those skills preceded
or followed automation, it did indicate that automation
management needed to be improved. Recommended "best

practices" and procedures have remedied some of the earlier
problems with automation.
Pilots must maintain their flight skills and ability to maneuver
aircraft manually within the standards set forth in the PTS. It
is recommended that pilots of automated aircraft occasionally
disengage the automation and manually fly the aircraft to
maintain stick-and-rudder proficiency. It is imperative that
the pilots understand that the EFD adds to the overall quality
of the flight experience, but it can also lead to catastrophe if
not utilized properly. At no time is the moving map meant to
substitute for a VFR sectional or low altitude en route chart.
Equipment Use

Autopilot Systems
In a single-pilot environment, an autopilot system can greatly
reduce workload. [Figure 2-23] As a result, the pilot is free
to focus his or her attention on other flight deck duties. This
can improve situational awareness and reduce the possibility
of a CFIT accident. While the addition of an autopilot may
certainly be considered a risk control measure, the real
challenge comes in determining the impact of an inoperative
unit. If the autopilot is known to be inoperative prior to
departure, this may factor into the evaluation of other risks.
For example, the pilot may be planning for a VHF
omnidirectional range (VOR) approach down to minimums
on a dark night into an unfamiliar airport. In such a case, the
pilot may have been relying heavily on a functioning autopilot
capable of flying a coupled approach. This would free the
pilot to monitor aircraft performance. A malfunctioning
autopilot could be the single factor that takes this from a
medium to a serious risk. At this point, an alternative needs
to be considered. On the other hand, if the autopilot were to
fail at a critical (high workload) portion of this same flight,
the pilot must be prepared to take action. Instead of simply
being an inconvenience, this could quickly turn into an
emergency if not properly handled. The best way to ensure
a pilot is prepared for such an event is to carefully study the
issue prior to departure and determine well in advance how
an autopilot failure is to be handled.

Familiarity
As previously discussed, pilot familiarity with all equipment
is critical in optimizing both safety and efficiency. If a pilot is
unfamiliar with any aircraft systems, this will add to workload
and may contribute to a loss of situational awareness. This
level of proficiency is critical and should be looked upon as
a requirement, not unlike carrying an adequate supply of fuel.
As a result, pilots should not look upon unfamiliarity with the
aircraft and its systems as a risk control measure, but instead as
a hazard with high risk potential. Discipline is key to success.

2-27

N

33

3

30

6

W
24

12

NAV

21

S

15

33

N

3

6

E

W

30

OBS

E

GS

24

12

S

21

N

3

E

W

6

30

33

15

OBS

24

12

WPT ______ DIS __._NM DTK ___° TRK 360°

134.000 118.000 COM1
123.800 118.000 COM2

130

4
300
4000

120

4200

110

4100

1
100
9

2

1

60

44000
000
20

90

3900

80

3800

70

21

113.00
110.60

S

NAV1 108.00
NAV2 108.00

15

HDG

270°

TAS 106KT
OAT 7°C

4300

1

2

3600
VOR 1

3500
3400
3300
XPDR 5537 IDNT LCL 10:12:34
3200
ALERTS
3100

Figure 2-22. Two similar flight decks equipped with the same information two different ways, analog and digital. What are they indicating?

Chances are that the analog pilot will review the top display before the bottom display. Conversely, the digitally trained pilot will review
the instrument panel on the bottom first.

2-28

The following are two simple rules for use of an EFD:
 	 Be able to fly the aircraft to the standards in the PTS.
Although this may seem insignificant, knowing how to
fly the aircraft to a standard makes a pilot's airmanship
smoother and allows him or her more time to attend
to the system instead of managing multiple tasks.


Figure 2-23. An

example of an autopilot system.

Respect for Onboard Systems
Automation can assist the pilot in many ways, but a
thorough understanding of the system(s) in use is essential
to gaining the benefits it can offer. Understanding leads
to respect, which is achieved through discipline and the
mastery of the onboard systems. It is important to fly the
aircraft using minimal information from the primary flight
display (PFD). This includes turns, climbs, descents, and
being able to fly approaches.

Read and understand the installed electronic flight
systems manuals to include the use of the autopilot
and the other onboard electronic management tools.

Managing Aircraft Automation
Before any pilot can master aircraft automation, he or she
must first know how to fly the aircraft. Maneuvers training
remains an important component of flight training because
almost 40 percent of all GA accidents take place in the

Reinforcement of Onboard Suites
The use of an EFD may not seem intuitive, but competency
becomes better with understanding and practice. Computerbased software and incremental training help the pilot become
comfortable with the onboard suites. Then the pilot needs
to practice what was learned in order to gain experience.
Reinforcement not only yields dividends in the use of
automation, it also reduces workload significantly.

Getting Beyond Rote Workmanship
The key to working effectively with automation is getting
beyond the sequential process of executing an action. If a
pilot has to analyze what key to push next, or always uses
the same sequence of keystrokes when others are available,
he or she may be trapped in a rote process. This mechanical
process indicates a shallow understanding of the system.
Again, the desire is to become competent and know what to
do without having to think about, "what keystroke is next."
Operating the system with competency and comprehension
benefits a pilot when situations become more diverse and
tasks increase.

Understand the Platform
Contrary to popular belief, flight in aircraft equipped with
different electronic management suites requires the same
attention as aircraft equipped with analog instrumentation
and a conventional suite of avionics. The pilot should review
and understand the different ways in which EFD are used in
a particular aircraft. [Figure 2-24]

Figure 2-24. Examples of different platforms. Top to bottom are the

Beechcraft Baron G58, Cirrus SR22, and Cirrus Entega.

2-29

landing phase, one realm of flight that still does not involve
programming a computer to execute. Another 15 percent
of all GA accidents occurs during takeoff and initial climb.
An advanced avionics safety issue identified by the FAA
concerns pilots who apparently develop an unwarranted
over-reliance in their avionics and the aircraft, believing
that the equipment will compensate for pilot shortcomings.
Related to the over-reliance is the role of ADM, which is
probably the most significant factor in the GA accident record
of high performance aircraft used for cross-country flight.
The FAA advanced avionics aircraft safety study found that
poor decision-making seems to afflict new advanced avionics
pilots at a rate higher than that of GA as a whole. The review
of advanced avionics accidents cited in this study shows the
majority are not caused by something directly related to the
aircraft, but by the pilot's lack of experience and a chain of
poor decisions. One consistent theme in many of the fatal
accidents is continued VFR flight into IMC.
Thus, pilot skills for normal and emergency operations hinge
not only on mechanical manipulation of the stick and rudder,
but also include the mental mastery of the EFD. Three key
flight management skills are needed to fly the advanced
avionics safely: information, automation, and risk.

Information Management
For the newly transitioning pilot, the PFD, MFD, and GPS/
VHF navigator screens seem to offer too much information
presented in colorful menus and submenus. In fact, the pilot
may be drowning in information but unable to find a specific
piece of information. It might be helpful to remember these
systems are similar to computers that store some folders on
a desktop and some within a hierarchy.
The first critical information management skill for flying with
advanced avionics is to understand the system at a conceptual
level. Remembering how the system is organized helps the
pilot manage the available information. It is important to
understanding that learning knob-and-dial procedures is not
enough. Learning more about how advanced avionics systems
work leads to better memory for procedures and allows pilots
to solve problems they have not seen before.
There are also limits to understanding. It is generally impossible
to understand all of the behaviors of a complex avionics
system. Knowing to expect surprises and to continually learn
new things is more effective than attempting to memorize
mechanical manipulation of the knobs. Simulation software
and books on the specific system used are of great value.
The second critical information management skill is stop,
look, and read. Pilots new to advanced avionics often become
2-30

fixated on the knobs and try to memorize each and every
sequence of button pushes, pulls, and turns. A far better
strategy for accessing and managing the information available
in advanced avionics computers is to stop, look, and read.
Reading before pushing, pulling, or twisting can often save
a pilot some trouble.
Once behind the display screens on an advanced avionics
aircraft, the pilot's goal is to meter, manage, and prioritize
the information flow to accomplish specific tasks.
Certificated flight instructors (CFIs), as well as pilots
transitioning to advanced avionics, will find it helpful to
corral the information flow. This is possible through such
tactics as configuring the aspects of the PFD and MFD
screens according to personal preferences. For example,
most systems offer map orientation options that include
"north up," "track up," "DTK" (desired track up), and
"heading up." Another tactic is to decide, when possible,
how much (or how little) information to display. Pilots can
also tailor the information displayed to suit the needs of a
specific flight.
Information flow can also be managed for a specific
operation. The pilot has the ability to prioritize information
for a timely display of exactly the information needed for any
given flight operation. Examples of managing information
display for a specific operation include:


Program map scale settings for en route versus
terminal area operation.

 	 Utilize the terrain awareness page on the MFD for a
night or IMC flight in or near the mountains.
 	 Use the nearest airports inset on the PFD at night or
over inhospitable terrain.
 	 Program the weather datalink set to show echoes and
METAR status flags.
Enhanced Situational Awareness
An advanced avionics aircraft offers increased safety with
enhanced situational awareness. Although aircraft flight
manuals (AFM) explicitly prohibit using the moving map,
topography, terrain awareness, traffic, and weather datalink
displays as the primary data source, these tools nonetheless
give the pilot unprecedented information for enhanced
situational awareness. Without a well-planned information
management strategy, these tools also make it easy for an
unwary pilot to slide into the complacent role of passenger
in command.
Consider the pilot whose navigational information
management strategy consists solely of following the
magenta line on the moving map. He or she can easily fly
into geographic or regulatory disaster, if the straight-line GPS

course goes through high terrain or prohibited airspace, or if
the moving map display fails.

for proper operation, and promptly take appropriate action if
the system does not perform as expected.

A good strategy for maintaining situational awareness
information management should include practices that help
ensure that awareness is enhanced, not diminished, by the use
of automation. Two basic procedures are to always doublecheck the system and verbal callouts. At a minimum, ensure
the presentation makes sense. Was the correct destination fed
into the navigation system? Callouts—even for single-pilot
operations—are an excellent way to maintain situational
awareness, as well as manage information.

For example, at the most basic level, managing the autopilot
means knowing at all times which modes are engaged
and which modes are armed to engage. The pilot needs to
verify that armed functions (e.g., navigation tracking or
altitude capture) engage at the appropriate time. Automation
management is another good place to practice the callout
technique, especially after arming the system to make a
change in course or altitude.

Other ways to maintain situational awareness include:


Perform verification check of all programming. Before
departure, check all information programmed while
on the ground.

 	 Check the flight routing. Before departure, ensure all
routing matches the planned flight route. Enter the
planned route and legs, to include headings and leg
length, on a paper log. Use this log to evaluate what
has been programmed. If the two do not match, do not
assume the computer data is correct, double check the
computer entry.


Verify waypoints.

 	 Make use of all onboard navigation equipment. For
example, use VOR to back up GPS and vice versa.


Match the use of the automated system with pilot
proficiency. Stay within personal limitations.



Plan a realistic flight route to maintain situational
awareness. For example, although the onboard
equipment allows a direct flight from Denver,
Colorado, to Destin, Florida, the likelihood of
rerouting around Eglin Air Force Base's airspace is
high.



Be ready to verify computer data entries. For example,
incorrect keystrokes could lead to loss of situational
awareness because the pilot may not recognize errors
made during a high workload period.

Automation Management
Advanced avionics offer multiple levels of automation, from
strictly manual flight to highly automated flight. No one level
of automation is appropriate for all flight situations, but in
order to avoid potentially dangerous distractions when flying
with advanced avionics, the pilot must know how to manage
the course deviation indicator (CDI), the navigation source,
and the autopilot. It is important for a pilot to know the
peculiarities of the particular automated system being used.
This ensures the pilot knows what to expect, how to monitor

In advanced avionics aircraft, proper automation management
also requires a thorough understanding of how the autopilot
interacts with the other systems. For example, with some
autopilots, changing the navigation source on the e-HSI from
GPS to LOC or VOR while the autopilot is engaged in NAV
(course tracking mode) causes the autopilot's NAV mode to
disengage. The autopilot's lateral control will default to ROL
(wing level) until the pilot takes action to reengage the NAV
mode to track the desired navigation source.
Risk Management
Risk management is the last of the three flight management
skills needed for mastery of the glass flight deck aircraft. The
enhanced situational awareness and automation capabilities
offered by a glass flight deck airplane vastly expand its safety
and utility, especially for personal transportation use. At the
same time, there is some risk that lighter workloads could
lead to complacency.
Humans are characteristically poor monitors of automated
systems. When asked to passively monitor an automated
system for faults, abnormalities, or other infrequent events,
humans perform poorly. The more reliable the system, the
poorer the human performance. For example, the pilot only
monitors a backup alert system, rather than the situation
that the alert system is designed to safeguard. It is a paradox
of automation that technically advanced avionics can both
increase and decrease pilot awareness.
It is important to remember that EFDs do not replace basic
flight knowledge and skills. They are a tool for improving
flight safety. Risk increases when the pilot believes the gadgets
compensate for lack of skill and knowledge. It is especially
important to recognize there are limits to what the electronic
systems in any light GA aircraft can do. Being PIC requires
sound ADM, which sometimes means saying "no" to a flight.
Risk is also increased when the pilot fails to monitor the
systems. By failing to monitor the systems and failing to
check the results of the processes, the pilot becomes detached

2-31

from the aircraft operation and slides into the complacent role
of passenger in command. Complacency led to tragedy in a
1999 aircraft accident.
In Colombia, a multi-engine aircraft crewed with two pilots
struck the face of the Andes Mountains. Examination of
their FMS revealed they entered a waypoint into the FMS
incorrectly by one degree resulting in a flight path taking
them to a point 60 NM off their intended course. The pilots
were equipped with the proper charts, their route was posted
on the charts, and they had a paper navigation log indicating
the direction of each leg. They had all the tools to manage
and monitor their flight, but instead allowed the automation
to fly and manage itself. The system did exactly what it was
programmed to do; it flew on a programmed course into a
mountain resulting in multiple deaths. The pilots simply failed
to manage the system and inherently created their own hazard.
Although this hazard was self-induced, what is notable is the
risk the pilots created through their own inattention. By failing
to evaluate each turn made at the direction of automation, the
pilots maximized risk instead of minimizing it. In this case, a
totally avoidable accident become a tragedy through simple
pilot error and complacency.
For the GA pilot transitioning to automated systems, it is
helpful to note that all human activity involving technical
devices entails some element of risk. Knowledge, experience,
and mission requirements tilt the odds in favor of safe and
successful flights. The advanced avionics aircraft offers
many new capabilities and simplifies the basic flying tasks,
but only if the pilot is properly trained and all the equipment
is working as advertised.

Chapter Summary
This chapter focused on helping the pilot improve his or
her ADM skills with the goal of mitigating the risk factors
associated with flight in both classic and automated aircraft.
In the end, the discussion is not so much about aircraft, but
about the people who fly them.

2-32


Chapter 3

Aircraft
Construction
Introduction
An aircraft is a device that is used, or intended to be used, for
flight according to the current Title 14 of the Code of Federal
Regulations (14 CFR) part 1, Definitions and Abbreviations.
Categories of aircraft for certification of airmen include
airplane, rotorcraft, glider, lighter-than-air, powered-lift,
powered parachute, and weight-shift control aircraft. Title
14 CFR part 1 also defines airplane as an engine-driven,
fixed-wing aircraft that is supported in flight by the dynamic
reaction of air against its wings. Another term, not yet
codified in 14 CFR part 1, is advanced avionics aircraft,
which refers to an aircraft that contains a global positioning
system (GPS) navigation system with a moving map display,
in conjunction with another system, such as an autopilot.
This chapter provides a brief introduction to the structure
of aircraft and uses an airplane for most illustrations. Light
Sport Aircraft (LSA), such as weight-shift control aircraft,
balloon, glider, powered parachute, and gyroplane, have their
own handbooks to include detailed information regarding
aerodynamics and control.

3-1

Aircraft Design, Certification, and
Airworthiness
The FAA certifies three types of aviation products: aircraft,
aircraft engines, and propellers. Each of these products
has been designed to a set of airworthiness standards.
These standards are parts of Title 14 of the Code of
Federal Regulations (14 CFR), published by the FAA. The
airworthiness standards were developed to help ensure that
aviation products are designed with no unsafe features.
Different airworthiness standards apply to the different
categories of aviation products as follows:
 	 Normal, Utility, Acrobatic, and Commuter Category
Airplanes- 14 CFR part 23
 	 Transport Category Airplanes—14 CFR part 25
 	 Normal Category—14 CFR part 27
 	 Transport Category Rotorcraft—14 CFR part 29
 	 Manned Free Balloons—14 CFR part 31
 	 Aircraft Engines—14 CFR part 33
 	 Propellers—14 CFR part 35
Some aircraft are considered "special classes" of aircraft and
do not have their own airworthiness standards, such as gliders
and powered lift. The airworthiness standards used for these
aircraft are a combination of requirements in 14 CFR parts
23, 25, 27, and 29 that the FAA and the designer have agreed
are appropriate for the proposed aircraft.
The FAA issues a Type Certificate (TC) for the product
when they are satisfied it complies with the applicable
airworthiness standards. When the TC is issued, a Type
Certificate Data Sheet (TCDS) is generated that specifies
the important design and operational characteristics of the
aircraft, aircraft engine, or propeller. The TCDS defines the
product and are available to the public from the FAA website
at \url{faa.gov}.
A Note About Light Sport Aircraft
Light sport aircraft are not designed according to FAA
airworthiness standards. Instead, they are designed to a
consensus of standards agreed upon in the aviation industry.
The FAA has agreed the consensus of standards is acceptable
as the design criteria for these aircraft. Light sport aircraft do
not necessarily have individually type certificated engines
and propellers. Instead, a TC is issued to the aircraft as a
whole. It includes the airframe, engine, and propeller.
Aircraft, aircraft engines, and propellers can be manufactured
one at a time from the design drawings, or through an FAA
approved manufacturing process, depending on the size and
capabilities of the manufacturer. During the manufacturing
3-2

process, each part is inspected to ensure that it has been built
exactly according to the approved design. This inspection is
called a conformity inspection.
When the aircraft is complete, with the airframe, engine, and
propeller, it is inspected and the FAA issues an airworthiness
certificate for the aircraft. Having an airworthiness
certificate means the complete aircraft meets the design and
manufacturing standards, and is in a condition for safe flight.
This airworthiness certificate must be carried in the aircraft
during all flight operations. The airworthiness certificate
remains valid as long as the required maintenance and
inspections are kept up to date for the aircraft.
Airworthiness certificates are classified as either "Standard"
or "Special." Standard airworthiness certificates are white,
and are issued for normal, utility, acrobatic, commuter, or
transport category aircraft. They are also issued for manned
free balloons and aircraft designated as "Special Class."
Special airworthiness certificates are pink, and are issued
for primary, restricted, and limited category aircraft, and
light sport aircraft. They are also issued as provisional
airworthiness certificates, special flight permits (ferry
permits), and for experimental aircraft.
More information on airworthiness certificates can be found
in Chapter 9, in 14 CFR parts 175-225, and also on the FAA
website at \url{faa.gov}.

Lift and Basic Aerodynamics
In order to understand the operation of the major components
and subcomponents of an aircraft, it is important to
understand basic aerodynamic concepts. This chapter briefly
introduces aerodynamics; a more detailed explanation can be
found in Chapter 5, Aerodynamics of Flight.
Four forces act upon an aircraft in relation to straight-and­
level, unaccelerated flight. These forces are thrust, lift,
weight, and drag. [Figure 3-1]
Thrust is the forward force produced by the powerplant/
propeller. It opposes or overcomes the force of drag. As a
general rule, it is said to act parallel to the longitudinal axis.
This is not always the case as explained later.
Drag is a rearward, retarding force and is caused by disruption
of airflow by the wing, fuselage, and other protruding objects.
Drag opposes thrust and acts rearward parallel to the relative
wind.

Lift

One of the most significant components of aircraft design is
CG. It is the specific point where the mass or weight of an
aircraft may be said to center; that is, a point around which,
if the aircraft could be suspended or balanced, the aircraft
would remain relatively level. The position of the CG of
an aircraft determines the stability of the aircraft in flight.
As the CG moves rearward (towards the tail), the aircraft
becomes more and more dynamically unstable. In aircraft
with fuel tanks situated in front of the CG, it is important
that the CG is set with the fuel tank empty. Otherwise, as the
fuel is used, the aircraft becomes unstable. [Figure 3-3] The
CG is computed during initial design and construction and
is further affected by the installation of onboard equipment,
aircraft loading, and other factors.

Drag

Thrust

Weight

Figure 3-1. The four forces.

Major Components

Weight is the combined load of the aircraft itself, the crew,
the fuel, and the cargo or baggage. Weight pulls the aircraft
downward because of the force of gravity. It opposes lift
and acts vertically downward through the aircraft's center
of gravity (CG).

Although airplanes are designed for a variety of purposes, most
of them have the same major components. [Figure 3-4] The
overall characteristics are largely determined by the original
design objectives. Most airplane structures include a fuselage,
wings, an empennage, landing gear, and a powerplant.

Lift opposes the downward force of weight, is produced by
the dynamic effect of the air acting on the wing, and acts
perpendicular to the flight path through the wing's center
of lift (CL).

Fuselage
The fuselage is the central body of an airplane and is designed
to accommodate the crew, passengers, and cargo. It also
provides the structural connection for the wings and tail
assembly. Older types of aircraft design utilized an open truss
structure constructed of wood, steel, or aluminum tubing.
[Figure 3-5] The most popular types of fuselage structures
used in today's aircraft are the monocoque (French for
"single shell") and semimonocoque. These structure types
are discussed in more detail under aircraft construction later
in the chapter.

An aircraft moves in three dimensions and is controlled by
moving it about one or more of its axes. The longitudinal,
or roll, axis extends through the aircraft from nose to tail,
with the line passing through the CG. The lateral or pitch
axis extends across the aircraft on a line through the wing
tips, again passing through the CG. The vertical, or yaw, axis
passes through the aircraft vertically, intersecting the CG. All
control movements cause the aircraft to move around one or
more of these axes and allows for the control of the aircraft
in flight. [Figure 3-2]
Pitching

Lateral axis

Wings
The wings are airfoils attached to each side of the fuselage
and are the main lifting surfaces that support the airplane in

Rolling

Longitudinal axis

Yawing

Vertical axis

Figure 3-2. Illustrates the pitch, roll, and yaw motion of the aircraft along the lateral, longitudinal, and vertical axes, respectively.

3-3

Lift

Longerons

Variable

CL

Fixed

CG

Nose-down force
independent of airspeed

Struts

Nose-up force
dependent upon airspeed

Longerons

Vertical forces acting on an airplane in flight.

Lift

Insufficient elevator
nose-down force

CL
CG

Bulkhead

Stringers

CG too far aft

Lift

If the CG is too far aft, there might not be enough elevator nose-down
force at the low stall airspeed to get the nose down for recovery.

CG

flight. There are numerous wing designs, sizes, and shapes
used by the various manufacturers. Each fulfills a certain need
with respect to the expected performance for the particular
airplane. How the wing produces lift is explained in Chapter
5, Aerodynamics of Flight.

CL

Insufficient elevator
nose-up force

CG too far forward

If the CG is too far forward, there will not be enough elevator
nose-up force to flare the airplane for landing.

Figure 3-3. Center of gravity (CG).
Wing

Empennage

Powerplant
Fuselage

Landing gear
Figure 3-4. Airplane components.

3-4

Figure 3-5. Truss-type fuselage structure.

Wings may be attached at the top, middle, or lower portion
of the fuselage. These designs are referred to as high-, mid-,
and low-wing, respectively. The number of wings can also
vary. Airplanes with a single set of wings are referred to as
monoplanes, while those with two sets are called biplanes.
[Figure 3-6]
Many high-wing airplanes have external braces, or wing
struts that transmit the flight and landing loads through the
struts to the main fuselage structure. Since the wing struts
are usually attached approximately halfway out on the wing,
this type of wing structure is called semi-cantilever. A few
high-wing and most low-wing airplanes have a full cantilever
wing designed to carry the loads without external struts.
The principal structural parts of the wing are spars, ribs,
and stringers. [Figure 3-7] These are reinforced by trusses,
I-beams, tubing, or other devices, including the skin. The
wing ribs determine the shape and thickness of the wing
(airfoil). In most modern airplanes, the fuel tanks are either
an integral part of the wing's structure or consist of flexible
containers mounted inside of the wing.

Figure 3-6. Monoplane (left) and biplane (right).

Attached to the rear, or trailing edges, of the wings are two
types of control surfaces referred to as ailerons and flaps.
Ailerons extend from about the midpoint of each wing
outward toward the tip, and move in opposite directions to
create aerodynamic forces that cause the airplane to roll.
Flaps extend outward from the fuselage to near the midpoint
of each wing. The flaps are normally flush with the wing's
surface during cruising flight. When extended, the flaps move
simultaneously downward to increase the lifting force of the
wing for takeoffs and landings. [Figure 3-8]

Alternate Types of Wings
Alternate types of wings are often found on aircraft. The
shape and design of a wing is dependent upon the type of

operation for which an aircraft is intended and is tailored
to specific types of flying. These design variations are
discussed in Chapter 5, Aerodynamics of Flight, which
provides information on the effect controls have on lifting
surfaces from traditional wings to wings that use both flexing
(due to billowing) and shifting (through the change of the
aircraft's CG). For example, the wing of the weight-shift
control aircraft is highly swept in an effort to reduce drag
and allow for the shifting of weight to provide controlled
flight. [Figure 3-9] Handbooks specific to most categories
of aircraft are available for the interested pilot and can be
found on the Federal Aviation Administration (FAA) website
at \url{faa.gov}.

Wing flap
Spar

Aileron
Fuel tank

Skin
Ribs

Stringers

Wing tip

Figure 3-7. Wing components.

3-5

Basic section

Plain flap

Figure 3-9. Weight-shift control aircraft use the shifting of weight
Split flap

for control.

portions of the trailing edge of the control surface. These
movable trim tabs, which are controlled from the flight deck,
reduce control pressures. Trim tabs may be installed on the
ailerons, the rudder, and/or the elevator.
Slotted flap

Fowler flap

Slotted Fowler flap

A second type of empennage design does not require an
elevator. Instead, it incorporates a one-piece horizontal
stabilizer that pivots from a central hinge point. This type of
design is called a stabilator and is moved using the control
wheel, just as the elevator is moved. For example, when a
pilot pulls back on the control wheel, the stabilator pivots so
the trailing edge moves up. This increases the aerodynamic
tail load and causes the nose of the airplane to move up.
Stabilators have an antiservo tab extending across their
trailing edge. [Figure 3-11]
The antiservo tab moves in the same direction as the trailing
edge of the stabilator and helps make the stabilator less
sensitive. The antiservo tab also functions as a trim tab to
relieve control pressures and helps maintain the stabilator in
the desired position.
Vertical stabilizer

Figure 3-8. Types of flaps.

Horizontal stabilizer

Empennage
The empennage includes the entire tail group and consists
of fixed surfaces, such as the vertical stabilizer and the
horizontal stabilizer. The movable surfaces include the
rudder, the elevator, and one or more trim tabs. [Figure 3-10]
The rudder is attached to the back of the vertical stabilizer.
During flight, it is used to move the airplane's nose left
and right. The elevator, which is attached to the back of the
horizontal stabilizer, is used to move the nose of the airplane
up and down during flight. Trim tabs are small, movable

3-6

Rudder
Trim tabs

Elevator
Figure 3-10. Empennage components.

Antiservo tab

Stabilator pivot point

Figure 3-11. Stabilator components.

Landing Gear
The landing gear is the principal support of the airplane when
parked, taxiing, taking off, or landing. The most common type
of landing gear consists of wheels, but airplanes can also be
equipped with floats for water operations or skis for landing
on snow. [Figure 3-12]
Wheeled landing gear consists of three wheels—two main
wheels and a third wheel positioned either at the front or rear
of the airplane. Landing gear with a rear mounted wheel is
called conventional landing gear.
Airplanes with conventional landing gear are sometimes
referred to as tailwheel airplanes. When the third wheel is
located on the nose, it is called a nosewheel, and the design
is referred to as a tricycle gear. A steerable nosewheel or
tailwheel permits the airplane to be controlled throughout all
operations while on the ground. Most aircraft are steered by
moving the rudder pedals, whether nosewheel or tailwheel.
Additionally, some aircraft are steered by differential braking.
The Powerplant
The powerplant usually includes both the engine and the
propeller. The primary function of the engine is to provide
the power to turn the propeller. It also generates electrical
power, provides a vacuum source for some flight instruments,
and in most single-engine airplanes, provides a source of
heat for the pilot and passengers. [Figure 3-13] The engine
is covered by a cowling, or a nacelle, which are both types
of covered housing. The purpose of the cowling or nacelle
is to streamline the flow of air around the engine and to help
cool the engine by ducting air around the cylinders.
The propeller, mounted on the front of the engine, translates
the rotating force of the engine into thrust, a forward acting
force that helps move the airplane through the air. A propeller

Figure 3-12. Types of landing gear: floats (top), skis (middle), and

wheels (bottom).

is a rotating airfoil that produces thrust through aerodynamic
action. A high-pressure area is formed at the back of the
propeller's airfoil, and low pressure is produced at the face of
the propeller, similar to the way lift is generated by an airfoil
used as a lifting surface or wing. This pressure differential
develops thrust from the propeller, which in turn pulls the
airplane forward. Engines may be turned around to be pushers
with the propeller at the rear.
There are two significant factors involved in the design
of a propeller that impact its effectiveness. The angle of a
3-7

operate the flight instruments, essential systems, such as
anti-icing, and passenger services, such as cabin lighting.
Engine

Propeller

Cowling

Figure 3-13. Engine compartment.

propeller blade, as measured against the hub of the propeller,
keeps the angle of attack (AOA) (See definition in Glossary)
relatively constant along the span of the propeller blade,
reducing or eliminating the possibility of a stall. The amount
of lift being produced by the propeller is directly related to
the AOA, which is the angle at which the relative wind meets
the blade. The AOA continuously changes during the flight
depending upon the direction of the aircraft.
The pitch is defined as the distance a propeller would travel in
one revolution if it were turning in a solid. These two factors
combine to allow a measurement of the propeller's efficiency.
Propellers are usually matched to a specific aircraft/
powerplant combination to achieve the best efficiency at a
particular power setting, and they pull or push depending on
how the engine is mounted.

Subcomponents
The subcomponents of an airplane include the airframe,
electrical system, flight controls, and brakes.
The airframe is the basic structure of an aircraft and is
designed to withstand all aerodynamic forces, as well as the
stresses imposed by the weight of the fuel, crew, and payload.
The primary function of an aircraft electrical system is to
generate, regulate, and distribute electrical power throughout
the aircraft. There are several different power sources on
aircraft to power the aircraft electrical systems. These
power sources include: engine-driven alternating current
(AC) generators, auxiliary power units (APUs), and external
power. The aircraft's electrical power system is used to

3-8

The flight controls are the devices and systems that govern
the attitude of an aircraft and, as a result, the flight path
followed by the aircraft. In the case of many conventional
airplanes, the primary flight controls utilize hinged, trailingedge surfaces called elevators for pitch, ailerons for roll, and
the rudder for yaw. These surfaces are operated by the pilot
in the flight deck or by an automatic pilot.
In the case of most modern airplanes, airplane brakes consist
of multiple pads (called caliper pads) that are hydraulically
squeezed toward each other with a rotating disk (called a
rotor) between them. The pads place pressure on the rotor
which is turning with the wheels. As a result of the increased
friction on the rotor, the wheels inherently slow down and
stop turning. The disks and brake pads are made either from
steel, like those in a car, or from a carbon material that weighs
less and can absorb more energy. Because airplane brakes are
used principally during landings and must absorb enormous
amounts of energy, their life is measured in landings rather
than miles.

Types of Aircraft Construction
The construction of aircraft fuselages evolved from the early
wood truss structural arrangements to monocoque shell
structures to the current semimonocoque shell structures.
Truss Structure
The main drawback of truss structure is its lack of a
streamlined shape. In this construction method, lengths of
tubing, called longerons, are welded in place to form a wellbraced framework. Vertical and horizontal struts are welded
to the longerons and give the structure a square or rectangular
shape when viewed from the end. Additional struts are needed
to resist stress that can come from any direction. Stringers
and bulkheads, or formers, are added to shape the fuselage
and support the covering.
As technology progressed, aircraft designers began to enclose
the truss members to streamline the airplane and improve
performance. This was originally accomplished with cloth
fabric, which eventually gave way to lightweight metals such
as aluminum. In some cases, the outside skin can support all
or a major portion of the flight loads. Most modern aircraft
use a form of this stressed skin structure known as monocoque
or semimonocoque construction. [Figure 3-14]
Monocoque
Monocoque construction uses stressed skin to support almost
all loads much like an aluminum beverage can. Although
very strong, monocoque construction is not highly tolerant

automobile manufacturing where the unibody is considered
standard in manufacturing.

Monocoque
Bulkhead

Stressed skin
Formers

Semimonocoque
Stringers

Bulkhead

Semimonocoque
Semimonocoque construction, partial or one-half, uses a
substructure to which the airplane's skin is attached. The
substructure, which consists of bulkheads and/or formers
of various sizes and stringers, reinforces the stressed skin
by taking some of the bending stress from the fuselage. The
main section of the fuselage also includes wing attachment
points and a firewall. On single-engine airplanes, the engine
is usually attached to the front of the fuselage. There is a
fireproof partition between the rear of the engine and the
flight deck or cabin to protect the pilot and passengers from
accidental engine fires. This partition is called a firewall and
is usually made of heat-resistant material such as stainless
steel. However, a new emerging process of construction is
the integration of composites or aircraft made entirely of
composites.

Skin

Composite Construction
Formers

Figure 3-14. Semimonocoque and monocoque fuselage design.

to deformation of the surface. For example, an aluminum
beverage can supports considerable forces at the ends of
the can, but if the side of the can is deformed slightly while
supporting a load, it collapses easily.
Because most twisting and bending stresses are carried by
the external skin rather than by an open framework, the need
for internal bracing was eliminated or reduced, saving weight
and maximizing space. One of the notable and innovative
methods for using monocoque construction was employed by
Jack Northrop. In 1918, he devised a new way to construct
a monocoque fuselage used for the Lockheed S-1 Racer.
The technique utilized two molded plywood half-shells that
were glued together around wooden hoops or stringers. To
construct the half shells, rather than gluing many strips of
plywood over a form, three large sets of spruce strips were
soaked with glue and laid in a semi-circular concrete mold
that looked like a bathtub. Then, under a tightly clamped
lid, a rubber balloon was inflated in the cavity to press
the plywood against the mold. Twenty-four hours later,
the smooth half-shell was ready to be joined to another to
create the fuselage. The two halves were each less than a
quarter inch thick. Although employed in the early aviation
period, monocoque construction would not reemerge for
several decades due to the complexities involved. Every
day examples of monocoque construction can be found in

History
The use of composites in aircraft construction can be dated
to World War II aircraft when soft fiberglass insulation was
used in B-29 fuselages. By the late 1950s, European high
performance sailplane manufacturers were using fiberglass
as primary structures. In 1965, the FAA type certified the
first all-fiberglass aircraft in the normal category, a Swiss
sailplane called a Diamant HBV. Four years later, the FAA
certified a four-seat, single-engine Windecker Eagle in the
normal category. By 2005, over 35 percent of new aircraft
were constructed of composite materials.
Composite is a broad term and can mean materials such as
fiberglass, carbon fiber cloth, Kevlar™ cloth, and mixtures
of all of the above. Composite construction offers two
advantages: extremely smooth skins and the ability to easily
form complex curved or streamlined structures. [Figure 3-15]

Composite Materials in Aircraft
Composite materials are fiber-reinforced matrix systems.
The matrix is the "glue" used to hold the fibers together
and, when cured, gives the part its shape, but the fibers carry
most of the load. There are many different types of fibers
and matrix systems.
In aircraft, the most common matrix is epoxy resin, which is
a type of thermosetting plastic. Compared to other choices
such as polyester resin, epoxy is stronger and has good hightemperature properties. There are many different types of
epoxies available with a wide range of structural properties,
cure times and temperatures, and costs.

3-9

and Columbia line of production aircraft, leading to their high
performance despite their fixed landing gear. Composites also
help mask the radar signature of "stealth" aircraft designs,
such as the B-2 and the F-22. Today, composites can be
found in aircraft as varied as gliders to most new helicopters.
Lack of corrosion is a third advantage of composites. Boeing
is designing the 787, with its all-composite fuselage, to have
both a higher pressure differential and higher humidity in
the cabin than previous airliners. Engineers are no longer as
concerned about corrosion from moisture condensation on the
hidden areas of the fuselage skins, such as behind insulation
blankets. This should lead to lower long-term maintenance
costs for the airlines.
Figure 3-15. Composite aircraft.

The most common reinforcing fibers used in aircraft
construction are fiberglass and carbon fiber. Fiberglass
has good tensile and compressive strength, good impact
resistance, is easy to work with, and is relatively inexpensive
and readily available. Its main disadvantage is that it is
somewhat heavy, and it is difficult to make a fiberglass loadcarrying structure lighter than a well designed equivalent
aluminum structure.
Carbon fiber is generally stronger in tensile and compressive
strength than fiberglass and has much higher bending
stiffness. It is also considerably lighter than fiberglass.
However, it is relatively poor in impact resistance; the fibers
are brittle and tend to shatter under sharp impact. This can
be greatly improved with a "toughened" epoxy resin system,
as used in the Boeing 787 horizontal and vertical stabilizers.
Carbon fiber is more expensive than fiberglass, but the price
has dropped due to innovations driven by the B-2 program
in the 1980s and Boeing 777 work in the 1990s. Very welldesigned carbon fiber structures can be significantly lighter
than an equivalent aluminum structure, sometimes by 30
percent or so.

Advantages of Composites
Composite construction offers several advantages over
metal, wood, or fabric, with its lighter weight being the most
frequently cited. Lighter weight is not always automatic. It
must be remembered that building an aircraft structure out of
composites does not guarantee it will be lighter; it depends
on the structure, as well as the type of composite being used.
A more important advantage is that a very smooth, compound
curved, aerodynamic structure made from composites
reduces drag. This is the main reason sailplane designers
switched from metal and wood to composites in the 1960s.
In aircraft, the use of composites reduces drag for the Cirrus
3-10

Another advantage of composites is their good performance
in a flexing environment, such as in helicopter rotor blades.
Composites do not suffer from metal fatigue and crack growth
as do metals. While it takes careful engineering, composite
rotor blades can have considerably higher design lives than
metal blades, and most new large helicopter designs have all
composite blades, and in many cases, composite rotor hubs.

Disadvantages of Composites
Composite construction comes with its own set of
disadvantages, the most important of which is the lack of
visual proof of damage. Composites respond differently from
other structural materials to impact, and there is often no
obvious sign of damage. For example, if a car backs into an
aluminum fuselage, it might dent the fuselage. If the fuselage
is not dented, there is no damage. If the fuselage is dented,
the damage is visible and repairs are made.
In a composite structure, a low energy impact, such as a
bump or a tool drop, may not leave any visible sign of the
impact on the surface. Underneath the impact site there may
be extensive delaminations, spreading in a cone-shaped area
from the impact location. The damage on the backside of
the structure can be significant and extensive, but it may be
hidden from view. Anytime one has reason to think there
may have been an impact, even a minor one, it is best to
get an inspector familiar with composites to examine the
structure to determine underlying damage. The appearance
of "whitish" areas in a fiberglass structure is a good tip-off
that delaminations of fiber fracture has occurred.
A medium energy impact (perhaps the car backing into the
structure) results in local crushing of the surface, which
should be visible to the eye. The damaged area is larger than
the visible crushed area and will need to be repaired. A high
energy impact, such as a bird strike or hail while in flight,
results in a puncture and a severely damaged structure. In

medium and high energy impacts, the damage is visible to the
eye, but low energy impact is difficult to detect. [Figure 3-16]
If an impact results in delaminations, crushing of the surface,
or a puncture, then a repair is mandatory. While waiting
for the repair, the damaged area should be covered and
protected from rain. Many composite parts are composed
of thin skins over a honeycomb core, creating a "sandwich"
structure. While excellent for structural stiffness reasons,
such a structure is an easy target for water ingress (entering),
leading to further problems later. A piece of "speed tape"
over the puncture is a good way to protect it from water, but
it is not a structural repair. The use of a paste filler to cover
up the damage, while acceptable for cosmetic purposes, is
not a structural repair, either.
0
0
+45
-45
0
90
90
0
-45
+45
0
0

0
0
+45
-45
0
90
90
0
-45
+45
0
0

Pyramid pattern matrix crack from impact
Low Energy Impact
Local fiber/matrix crushing
0
0
+45
-45
0
90
90
0
-45
+45
0
0

0
0
+45
-45
0
90
90
0
-45
+45
0
0

Delaminations

Back side fiber fracture
Medium Energy Impact

0
0
+45
-45
0
90
90
0
-45
+45
0
0

Through
penetration
small damage
zone

Delaminations

Loose
fiber ends

The potential for heat damage to the resin is another
disadvantage of using composites. While "too hot" depends
on the particular resin system chosen, many epoxies begin
to weaken over 150 °F. White paint on composites is often
used to minimize this issue. For example, the bottom of
a wing that is painted black facing a black asphalt ramp
on a hot, sunny day can get as hot as 220 °F. The same
structure, painted white, rarely exceeds 140 °F. As a result,
composite aircraft often have specific recommendations
on allowable paint colors. If the aircraft is repainted, these
recommendations must be followed. Heat damage can also
occur due to a fire. Even a quickly extinguished small brake
fire can damage bottom wing skins, composite landing gear
legs, or wheel pants.
Also, chemical paint strippers are very harmful to composites
and must not be used on them. If paint needs to be removed
from composites, only mechanical methods are allowed, such
as gentle grit blasting or sanding. Many expensive composite
parts have been ruined by the use of paint stripper and such
damage is generally not repairable.

Fluid Spills on Composites
Some owners are concerned about fuel, oil, or hydraulic fluid
spills on composite surfaces. These are generally not a problem
with modern composites using epoxy resin. Usually, if the
spill does not attack the paint, it will not hurt the underlying
composite. Some aircraft use fiberglass fuel tanks, for example,
in which the fuel rides directly against the composite surface
with no sealant being used. If the fiberglass structure is made
with some of the more inexpensive types of polyester resin,
there can be a problem when using auto gas with ethanol
blended into the mixture. The more expensive types of
polyester resin, as well as epoxy resin, can be used with auto
gas, as well as 100 octane aviation gas (avgas) and jet fuel.

Lightning Strike Protection
0
0
+45
-45
0
90
90
0
-45
+45
0
0

High Energy Impact
Figure 3-16. Impact energy affects the visibility, as well as the
severity, of damage in composite structures. High and medium
energy impacts, while severe, are easy to detect. Low energy impacts
can easily cause hidden damage.

Lightning strike protection is an important consideration in
aircraft design. When an aircraft is hit by lightning, a very
large amount of energy is delivered to the structure. Whether
flying a light general aviation (GA) aircraft or a large airliner,
the basic principle of lightning strike protection is the same.
For any size aircraft, the energy from the strike must be spread
over a large surface area to lower the amps per square inch
to a harmless level.
If lightning strikes an aluminum airplane, the electrical
energy naturally conducts easily through the aluminum
structure. The challenge is to keep the energy out of avionics,
fuel systems, etc., until it can be safely conducted overboard.
The outer skin of the aircraft is the path of least resistance.

3-11

In a composite aircraft, fiberglass is an excellent electrical
insulator, while carbon fiber conducts electricity, but not
as easily as aluminum. Therefore, additional electrical
conductivity needs to be added to the outside layer of
composite skin. This is done typically with fine metal meshes
bonded to the skin surfaces. Aluminum and copper mesh
are the two most common types, with aluminum used on
fiberglass and copper on carbon fiber. Any structural repairs
on lightning-strike protected areas must also include the mesh
as well as the underlying structure.
For composite aircraft with internal radio antennas, there
must be "windows" in the lightning strike mesh in the area
of the antenna. Internal radio antennas may be found in
fiberglass composites because fiberglass is transparent to
radio frequencies, where carbon fiber is not.

The Future of Composites
In the decades since World War II, composites have earned
an important role in aircraft structure design. Their design
flexibility and corrosion resistance, as well as the high
strength-to-weight ratios possible, will undoubtedly continue
to lead to more innovative aircraft designs in the future.
From the Cirrus SR-20 to the Boeing 787, it is obvious that
composites have found a home in aircraft construction and
are here to stay. [Figure 3-17]

Instrumentation: Moving into the Future
Until recently, most GA aircraft were equipped with
individual instruments utilized collectively to safely operate
and maneuver the aircraft. With the release of the electronic
flight display (EFD) system, conventional instruments have
been replaced by multiple liquid crystal display (LCD)
screens. The first screen is installed in front of the pilot
position and is referred to as the primary flight display (PFD).
The second screen, positioned approximately in the center
of the instrument panel, is referred to as the multi-function
display (MFD). These two screens de-clutter instrument
panels while increasing safety. This has been accomplished
through the utilization of solid state instruments that have
a failure rate far less than those of conventional analog
instrumentation. [Figure 3-18]
With today's improvements in avionics and the introduction
of EFDs, pilots at any level of experience need an astute
knowledge of the onboard flight control systems, as well as
an understanding of how automation melds with aeronautical
decision-making (ADM). These subjects are covered in detail
in Chapter 2, Aeronautical Decision-Making.
Whether an aircraft has analog or digital (glass) instruments,
the instrumentation falls into three different categories:
performance, control, and navigation.
3-12

Figure 3-17. Composite materials in aircraft, such as Columbia 350
(top), Boeing 787 (middle), and a Coast Guard HH-65 (bottom).

Performance Instruments
The performance instruments indicate the aircraft's actual
performance. Performance is determined by reference to the
altimeter, airspeed or vertical speed indicator (VSI), heading
indicator, and turn-and-slip indicator. The performance
instruments directly reflect the performance the aircraft
is achieving. The speed of the aircraft can be referenced
on the airspeed indicator. The altitude can be referenced
on the altimeter. The aircraft's climb performance can be
determined by referencing the VSI. Other performance
instruments available are the heading indicator, angle of
attack indicator, and the slip-skid indicator. [Figure 3-19]

selected navigation facility or fix. They also provide pilotage
information so the aircraft can be maneuvered to keep it on
a predetermined path. The pilotage information can be in
either two or three dimensions relative to the ground-based or
space-based navigation information. [Figures 3-21 and 3-22]

Global Positioning System (GPS)
GPS is a satellite-based navigation system composed of a
network of satellites placed into orbit by the United States
Department of Defense (DOD). GPS was originally intended
for military applications, but in the 1980s the government
made the system available for civilian use. GPS works in
all weather conditions, anywhere in the world, 24 hours a
day. A GPS receiver must be locked onto the signal of at
least three satellites to calculate a two-dimensional position
(latitude and longitude) and track movement. With four or
more satellites in view, the receiver can determine the user's
three-dimensional position (latitude, longitude, and altitude).
Other satellites must also be in view to offset signal loss
and signal ambiguity. The use of the GPS is discussed in
more detail in Chapter 17, Navigation. Additionally, GPS is
discussed in the Aeronautical Information Manual (AIM).

Chapter Summary

Figure 3-18. Analog display (top) and digital display (bottom) from

a Cessna 172.

Control Instruments
The control instruments display immediate attitude and power
changes and are calibrated to permit adjustments in precise
increments. [Figure 3-20] The instrument for attitude display
is the attitude indicator. The control instruments do not
indicate aircraft speed or altitude. In order to determine these
variables and others, a pilot must reference the performance
instruments.

This chapter provides an overview of aircraft structures.
A more in-depth understanding of aircraft structures and
controls can be gained through the use of flight simulation
software or interactive programs available online through
aviation organizations, such as the Aircraft Owners and Pilots
Association (AOPA). Pilots are also encouraged to subscribe
to or review the various aviation periodicals that contain
valuable flying information. As discussed in Chapter 1, the
National Aeronautics and Space Administration (NASA) and
the FAA also offer free information for pilots.

Navigation Instruments
The navigation instruments indicate the position of the
aircraft in relation to a selected navigation facility or fix.
This group of instruments includes various types of course
indicators, range indicators, glideslope indicators, and
bearing pointers. Newer aircraft with more technologically
advanced instrumentation provide blended information,
giving the pilot more accurate positional information.
Navigation instruments are comprised of indicators that
display GPS, very high frequency (VHF) omni-directional
radio range (VOR), nondirectional beacon (NDB),
and instrument landing system (ILS) information. The
instruments indicate the position of the aircraft relative to a
3-13

Airspeed indicator

Attitude indicator

Altimeter indicator

Turn coordinator

Heading indicator

Vertical speed indicator

NAV1
NAV2

108.00
108.00

113.00
110.60

WPT

_ _ _ _ _ _ DIS _ _ ._ NM

DTK

_ _Slip/Skid
_°T TRK indicator
360°T 134.000

123.800

Airspeed tape indicator

130


4000

4300


120


4200

2


Altimeter and
altitude trend tape

110


1

100

9


118.000
118.000

COM1
COM2

Vertical speed and
tape indicator

1


4100
60


500


44000

000

20


3900


90

80


270°

70


3800


Turn rate trend vector
2


4300

TAS 100KT

1


270°

Heading bug
3600

VOR 1

Course3500
arrow
Turn rate indicator
3400
3300

XPDR 5537 IDNT LCL23:00:34
3200

ALERTS

3100

Figure 3-19. Performance instruments.

3-14

Manifold pressure gauge

Attitude indicator

NAV1
NAV2

108.00
108.00

23.0

113.00
110.60

WPT

_ _ _ _ _ _ DIS _ _ ._ NM

DTK

_ _ _°

TRK

360°

134.000
123.800

130

4000
4300

120

4200

110

4100

118.000
118.000

COM1
COM2

2

23.0

1
100
9

2300

60

44000
000
3900

90
80

13.7
46

TAS

70
100KT

270°

2300
1

338
5

20
1

3800
2

4300

Tachometer

200
1652

1

VOR 1

3600
3500
3400
3300

XPDR 5537 IDNT LCL23:00:34
3200

ALERTS

3100

Figure 3-20. Control instruments.

3-15

A

2
The aircraft illustrated is heading
020°, but the course is to the left.

33

N
3

30

2

020°

6

E

24

340°

21

CRS

S

3

6

30 3
3

15

OBS

VOR 1

12

360°

W

TO

HDG

I2
I5

2I 2
4

1

1
Present position, inbound on
160° radial.
33

N
3

30

24

340°

2I

30

W

CRS

12

15

S

21

24

340°

W

TO

I5

E

360°

15

S

OBS

12

21

E

24

6

View is from the pilot's
perspective, and the
movable card is reset
after each turn.

I2

6

HDG

6

3

GS

NAV

33
3

N

33

OBS

30

°

160

Figure 3-21. A comparison of navigation information as depicted on both analog and digital displays.

Fly down

Fly up


Glideslope needle indicates "fly

down" to intercept glideslope

E
6

3

FR

N

6

3

20

W

30

33

N

20

OBS

4300

NAV 1

33

FR

NAV 1

I0

20

3500

000
44
000

80

3800
3600

25

3500
3400

Figure 3-22. Analog and digital indications for glideslope interception.
3200
3100

3-16

3900

3600

4100

3300

20
000
44
000

3800

4200

3900

30

OBS

4100

24

4400

20

24

I0

I0

0
0

21

2
I

21

I0
20

TO

S

TO
0
0

15

S

E
2
I

4000
4300
4200

12

15

W

12

Glideslope needle indicates "fly up" to intercept glideslope

4400

Glid

e
eslop

3400
3300
3200
3100

80

Chapter 4

Principles of
Flight
Introduction
This chapter examines the fundamental physical laws
governing the forces acting on an aircraft in flight, and
what effect these natural laws and forces have on the
performance characteristics of aircraft. To control an aircraft,
be it an airplane, helicopter, glider, or balloon, the pilot
must understand the principles involved and learn to use or
counteract these natural forces.

Structure of the Atmosphere
The atmosphere is an envelope of air that surrounds the Earth
and rests upon its surface. It is as much a part of the Earth as
the seas or the land, but air differs from land and water as it is
a mixture of gases. It has mass, weight, and indefinite shape.
The atmosphere is composed of 78 percent nitrogen, 21
percent oxygen, and 1 percent other gases, such as argon
or helium. Some of these elements are heavier than others.
The heavier elements, such as oxygen, settle to the surface
of the Earth, while the lighter elements are lifted up to the
region of higher altitude. Most of the atmosphere's oxygen
is contained below 35,000 feet altitude.

4-1

Air is a Fluid
When most people hear the word "fluid," they usually think
of liquid. However, gasses, like air, are also fluids. Fluids
take on the shape of their containers. Fluids generally do not
resist deformation when even the smallest stress is applied,
or they resist it only slightly. We call this slight resistance
viscosity. Fluids also have the ability to flow. Just as a liquid
flows and fills a container, air will expand to fill the available
volume of its container. Both liquids and gasses display these
unique fluid properties, even though they differ greatly in
density. Understanding the fluid properties of air is essential
to understanding the principles of flight.

Viscosity
Viscosity is the property of a fluid that causes it to resist
flowing. The way individual molecules of the fluid tend to
adhere, or stick, to each other determines how much a fluid
resists flow. High-viscosity fluids are "thick" and resist flow;
low-viscosity fluids are "thin" and flow easily. Air has a low
viscosity and flows easily.
Using two liquids as an example, similar amounts of oil and
water poured down two identical ramps will flow at different
rates due to their different viscosity. The water seems to flow
freely while the oil flows much more slowly.
As another example, different types of similar liquids will
display different behaviors because of different viscosities.
Grease is very viscous. Given time, grease will flow, even
though the flow rate will be slow. Motor oil is less viscous
than grease and flows much more easily, but it is more viscous
and flows more slowly than gasoline.

the viscosity of air. However, since air is a fluid and has
viscosity properties, it resists flow around any object to
some extent.

Friction
Another factor at work when a fluid flows over or around
an object is called friction. Friction is the resistance that one
surface or object encounters when moving over another.
Friction exists between any two materials that contact each
other.
The effects of friction can be demonstrated using a similar
example as before. If identical fluids are poured down two
identical ramps, they flow in the same manner and at the
same speed. If the surface of one ramp is rough, and the other
smooth, the flow down the two ramps differs significantly.
The rough surface ramp impedes the flow of the fluid due
to resistance from the surface (friction). It is important to
remember that all surfaces, no matter how smooth they
appear, are not smooth on a microscopic level and impede
the flow of a fluid.
The surface of a wing, like any other surface, has a certain
roughness at the microscopic level. The surface roughness
causes resistance and slows the velocity of the air flowing
over the wing. [Figure 4-1]
Molecules of air pass over the surface of the wing and actually
adhere (stick, or cling) to the surface because of friction. Air
molecules near the surface of the wing resist motion and have
a relative velocity near zero. The roughness of the surface
impedes their motion. The layer of molecules that adhere to
the wing surface is referred to as the boundary layer.

All fluids are viscous and have a resistance to flow, whether
or not we observe this resistance. We cannot easily observe

Leading edge of wing under 1,500x magnification
Figure 4-1. Microscopic surface of a wing.

4-2

Once the boundary layer of the air adheres to the wing by
friction, further resistance to the airflow is caused by the
viscosity, the tendency of the air to stick to itself. When
these two forces act together to resist airflow over a wing,
it is called drag.

Pressure

Standard
Sea Level
Pressure

29.92 Hg

Pressure is the force applied in a perpendicular direction to the
surface of an object. Often, pressure is measured in pounds of
force exerted per square inch of an object, or PSI. An object
completely immersed in a fluid will feel pressure uniformly
around the entire surface of the object. If the pressure on one
surface of the object becomes less than the pressure exerted
on the other surfaces, the object will move in the direction
of the lower pressure.
Atmospheric Pressure
Although there are various kinds of pressure, pilots are
mainly concerned with atmospheric pressure. It is one of
the basic factors in weather changes, helps to lift an aircraft,
and actuates some of the important flight instruments. These
instruments are the altimeter, airspeed indicator, vertical
speed indicator, and manifold pressure gauge.
Air is very light, but it has mass and is affected by the
attraction of gravity. Therefore, like any other substance,
it has weight, and because of its weight, it has force. Since
air is a fluid substance, this force is exerted equally in all
directions. Its effect on bodies within the air is called pressure.
Under standard conditions at sea level, the average pressure
exerted by the weight of the atmosphere is approximately
14.70 pounds per square inch (psi) of surface, or 1,013.2
millibars (mb). The thickness of the atmosphere is limited;
therefore, the higher the altitude, the less air there is above.
For this reason, the weight of the atmosphere at 18,000 feet
is one-half what it is at sea level.
The pressure of the atmosphere varies with time and location.
Due to the changing atmospheric pressure, a standard
reference was developed. The standard atmosphere at sea
level is a surface temperature of 59 °F or 15 °C and a surface
pressure of 29.92 inches of mercury ("Hg) or 1,013.2 mb.
[Figure 4-2]
A standard temperature lapse rate is when the temperature
decreases at the rate of approximately 3.5 °F or 2 °C per
thousand feet up to 36,000 feet, which is approximately –65
°F or –55 °C. Above this point, the temperature is considered
constant up to 80,000 feet. A standard pressure lapse rate is
when pressure decreases at a rate of approximately 1 "Hg
per 1,000 feet of altitude gain to 10,000 feet. [Figure 4-3]
The International Civil Aviation Organization (ICAO) has
established this as a worldwide standard, and it is often

Inches of
Mercury

Millibars

30

1016

25

847

20

677

15

508

10

339

Standard
Sea Level
Pressure

1013 mb

5
170
Atmospheric Pressure
0
0

Figure 4-2. Standard sea level pressure.

Standard Atmosphere
Altitude (ft)

Pressure (Hg)

0
1,000
2,000
3,000
4,000
5,000
6,000
7,000
8,000
9,000
10,000
11,000
12,000
13,000
14,000
15,000
16,000
17,000
18,000
19,000
20,000

29.92
28.86
27.82
26.82
25.84
24.89
23.98
23.09
22.22
21.38
20.57
19.79
19.02
18.29
17.57
16.88
16.21
15.56
14.94
14.33
13.74

Temperature
(°C)

(°F)

15.0
13.0
11.0
9.1
7.1
5.1
3.1
1.1
-0.9
-2.8
-4.8
-6.8
-8.8
-10.8
-12.7
-14.7
-16.7
-18.7
-20.7
-22.6
-24.6

59.0
55.4
51.9
48.3
44.7
41.2
37.6
34.0
30.5
26.9
23.3
19.8
16.2
12.6
9.1
5.5
1.9
-1.6
-5.2
-8.8
-12.3

Figure 4-3. Properties of standard atmosphere.

referred to as International Standard Atmosphere (ISA) or
ICAO Standard Atmosphere. Any temperature or pressure
that differs from the standard lapse rates is considered
nonstandard temperature and pressure.

4-3

Since aircraft performance is compared and evaluated with
respect to the standard atmosphere, all aircraft instruments are
calibrated for the standard atmosphere. In order to properly
account for the nonstandard atmosphere, certain related terms
must be defined.
Pressure Altitude
Pressure altitude is the height above a standard datum plane
(SDP), which is a theoretical level where the weight of the
atmosphere is 29.92 "Hg (1,013.2 mb) as measured by a
barometer. An altimeter is essentially a sensitive barometer
calibrated to indicate altitude in the standard atmosphere. If
the altimeter is set for 29.92 "Hg SDP, the altitude indicated
is the pressure altitude. As atmospheric pressure changes, the
SDP may be below, at, or above sea level. Pressure altitude
is important as a basis for determining airplane performance,
as well as for assigning flight levels to airplanes operating at
or above 18,000 feet.
The pressure altitude can be determined by one of the
following methods:
1.

Setting the barometric scale of the altimeter to 29.92
and reading the indicated altitude

2.

Applying a correction factor to the indicated altitude
according to the reported altimeter setting

Density Altitude
SDP is a theoretical pressure altitude, but aircraft operate in a
nonstandard atmosphere and the term density altitude is used
for correlating aerodynamic performance in the nonstandard
atmosphere. Density altitude is the vertical distance above sea
level in the standard atmosphere at which a given density is
to be found. The density of air has significant effects on the
aircraft's performance because as air becomes less dense,
it reduces:


Power because the engine takes in less air



Thrust because a propeller is less efficient in thin air



Lift because the thin air exerts less force on the airfoils

Density altitude is pressure altitude corrected for nonstandard
temperature. As the density of the air increases (lower
density altitude), aircraft performance increases; conversely
as air density decreases (higher density altitude), aircraft
performance decreases. A decrease in air density means
a high density altitude; an increase in air density means a
lower density altitude. Density altitude is used in calculating
aircraft performance because under standard atmospheric
conditions, air at each level in the atmosphere not only has
a specific density, its pressure altitude and density altitude
identify the same level.

4-4

The computation of density altitude involves consideration
of pressure (pressure altitude) and temperature. Since aircraft
performance data at any level is based upon air density under
standard day conditions, such performance data apply to
air density levels that may not be identical with altimeter
indications. Under conditions higher or lower than standard,
these levels cannot be determined directly from the altimeter.
Density altitude is determined by first finding pressure
altitude, and then correcting this altitude for nonstandard
temperature variations. Since density varies directly with
pressure and inversely with temperature, a given pressure
altitude may exist for a wide range of temperatures by
allowing the density to vary. However, a known density
occurs for any one temperature and pressure altitude. The
density of the air has a pronounced effect on aircraft and
engine performance. Regardless of the actual altitude of the
aircraft, it will perform as though it were operating at an
altitude equal to the existing density altitude.
Air density is affected by changes in altitude, temperature,
and humidity. High density altitude refers to thin air, while
low density altitude refers to dense air. The conditions that
result in a high density altitude are high elevations, low
atmospheric pressures, high temperatures, high humidity, or
some combination of these factors. Lower elevations, high
atmospheric pressure, low temperatures, and low humidity
are more indicative of low density altitude.

Effect of Pressure on Density
Since air is a gas, it can be compressed or expanded. When
air is compressed, a greater amount of air can occupy a given
volume. Conversely, when pressure on a given volume of air
is decreased, the air expands and occupies a greater space.
At a lower pressure, the original column of air contains a
smaller mass of air. The density is decreased because density
is directly proportional to pressure. If the pressure is doubled,
the density is doubled; if the pressure is lowered, the density is
lowered. This statement is true only at a constant temperature.

Effect of Temperature on Density
Increasing the temperature of a substance decreases its
density. Conversely, decreasing the temperature increases
the density. Thus, the density of air varies inversely with
temperature. This statement is true only at a constant pressure.
In the atmosphere, both temperature and pressure decrease
with altitude and have conflicting effects upon density.
However, a fairly rapid drop in pressure as altitude increases
usually has a dominating effect. Hence, pilots can expect the
density to decrease with altitude.

Effect of Humidity (Moisture) on Density
The preceding paragraphs refer to air that is perfectly dry. In
reality, it is never completely dry. The small amount of water
vapor suspended in the atmosphere may be almost negligible
under certain conditions, but in other conditions humidity
may become an important factor in the performance of an
aircraft. Water vapor is lighter than air; consequently, moist
air is lighter than dry air. Therefore, as the water content
of the air increases, the air becomes less dense, increasing
density altitude and decreasing performance. It is lightest or
least dense when, in a given set of conditions, it contains the
maximum amount of water vapor.
Humidity, also called relative humidity, refers to the amount
of water vapor contained in the atmosphere and is expressed
as a percentage of the maximum amount of water vapor the
air can hold. This amount varies with temperature. Warm air
holds more water vapor, while cold air holds less. Perfectly
dry air that contains no water vapor has a relative humidity
of zero percent, while saturated air, which cannot hold any
more water vapor, has a relative humidity of 100 percent.
Humidity alone is usually not considered an important factor
in calculating density altitude and aircraft performance, but
it is a contributing factor.
As temperature increases, the air can hold greater amounts
of water vapor. When comparing two separate air masses,
the first warm and moist (both qualities tending to lighten
the air) and the second cold and dry (both qualities making
it heavier), the first must be less dense than the second.
Pressure, temperature, and humidity have a great influence
on aircraft performance because of their effect upon density.
There are no rules of thumb that can be easily applied, but
the affect of humidity can be determined using several online
formulas. In the first example, the pressure is needed at the
altitude for which density altitude is being sought. Using
Figure 4-2, select the barometric pressure closest to the
associated altitude. As an example, the pressure at 8,000 feet
is 22.22 "Hg. Using the National Oceanic and Atmospheric
Administration (NOAA) website (\url{srh.noaa.gov/
epz/?n=wxcalc_densityaltitude}) for density altitude, enter
the 22.22 for 8,000 feet in the station pressure window. Enter
a temperature of 80° and a dew point of 75°. The result is a
density altitude of 11,564 feet. With no humidity, the density
altitude would be almost 500 feet lower.
Another website (\url{wahiduddin.net/calc/density_
altitude.htm}) provides a more straight forward method of
determining the effects of humidity on density altitude
without using additional interpretive charts. In any case, the
effects of humidity on density altitude include a decrease in
overall performance in high humidity conditions.

Theories in the Production of Lift
In order to achieve flight in a machine that is heavier than air,
there are several obstacles we must overcome. One of those
obstacles, discussed previously, is the resistance to movement
called drag. The most challenging obstacle to overcome in
aviation, however, is the force of gravity. A wing moving
through air generates the force called lift, also previously
discussed. Lift from the wing that is greater than the force of
gravity, directed opposite to the direction of gravity, enables
an aircraft to fly. Generating this force called lift is based on
some important principles, Newton's basic laws of motion,
and Bernoulli's principle of differential pressure.
Newton's Basic Laws of Motion
The formulation of lift has historically been an adaptation
over the past few centuries of basic physical laws. These
laws, although seemingly applicable to all aspects of lift,
do not explain how lift is formulated. In fact, one must
consider the many airfoils that are symmetrical, yet produce
significant lift.
The fundamental physical laws governing the forces acting
upon an aircraft in flight were adopted from postulated
theories developed before any human successfully flew
an aircraft. The use of these physical laws grew out of the
Scientific Revolution, which began in Europe in the 1600s.
Driven by the belief the universe operated in a predictable
manner open to human understanding, many philosophers,
mathematicians, natural scientists, and inventors spent their
lives unlocking the secrets of the universe. One of the most
well-known was Sir Isaac Newton, who not only formulated
the law of universal gravitation, but also described the three
basic laws of motion.
Newton's First Law: "Every object persists in its state of rest
or uniform motion in a straight line unless it is compelled to
change that state by forces impressed on it."
This means that nothing starts or stops moving until some
outside force causes it to do so. An aircraft at rest on the ramp
remains at rest unless a force strong enough to overcome
its inertia is applied. Once it is moving, its inertia keeps
it moving, subject to the various other forces acting on it.
These forces may add to its motion, slow it down, or change
its direction.
Newton's Second Law: "Force is equal to the change in
momentum per change in time. For a constant mass, force
equals mass times acceleration."

4-5

When a body is acted upon by a constant force, its resulting
acceleration is inversely proportional to the mass of the body
and is directly proportional to the applied force. This takes
into account the factors involved in overcoming Newton's
First Law. It covers both changes in direction and speed,
including starting up from rest (positive acceleration) and
coming to a stop (negative acceleration or deceleration).

Since air is recognized as a body, and it is understood that
air will follow the above laws, one can begin to see how
and why an airplane wing develops lift. As the wing moves
through the air, the flow of air across the curved top surface
increases in velocity creating a low-pressure area.
Although Newton, Bernoulli, and hundreds of other early
scientists who studied the physical laws of the universe did
not have the sophisticated laboratories available today, they
provided great insight to the contemporary viewpoint of how
lift is created.

Newton's Third Law: "For every action, there is an equal
and opposite reaction."
In an airplane, the propeller moves and pushes back the
air; consequently, the air pushes the propeller (and thus the
airplane) in the opposite direction—forward. In a jet airplane,
the engine pushes a blast of hot gases backward; the force of
equal and opposite reaction pushes against the engine and
forces the airplane forward.

Airfoil Design
An airfoil is a structure designed to obtain reaction upon its
surface from the air through which it moves or that moves
past such a structure. Air acts in various ways when submitted
to different pressures and velocities; but this discussion
is confined to the parts of an aircraft that a pilot is most
concerned with in flight—namely, the airfoils designed to
produce lift. By looking at a typical airfoil profile, such as
the cross section of a wing, one can see several obvious
characteristics of design. [Figure 4-5] Notice that there is
a difference in the curvatures (called cambers) of the upper
and lower surfaces of the airfoil. The camber of the upper
surface is more pronounced than that of the lower surface,
which is usually somewhat flat.

Bernoulli's Principle of Differential Pressure
A half-century after Newton formulated his laws, Daniel
Bernoulli, a Swiss mathematician, explained how the pressure
of a moving fluid (liquid or gas) varies with its speed of
motion. Bernoulli's Principle states that as the velocity of a
moving fluid (liquid or gas) increases, the pressure within
the fluid decreases. This principle explains what happens to
air passing over the curved top of the airplane wing.
A practical application of Bernoulli's Principle is the venturi
tube. The venturi tube has an air inlet that narrows to a
throat (constricted point) and an outlet section that increases
in diameter toward the rear. The diameter of the outlet is
the same as that of the inlet. The mass of air entering the
tube must exactly equal the mass exiting the tube. At the
constriction, the speed must increase to allow the same
amount of air to pass in the same amount of time as in all
other parts of the tube. When the air speeds up, the pressure
also decreases. Past the constriction, the airflow slows and
the pressure increases. [Figure 4-4]

4
2

6

4

VELOCITY

8
0

I0

2

6

4

PRESSURE

8
0

I0

Figure 4-4. Air pressure decreases in a venturi tube.

4-6

2

A reference line often used in discussing the airfoil is
the chord line, a straight line drawn through the profile
connecting the extremities of the leading and trailing edges.
The distance from this chord line to the upper and lower
surfaces of the wing denotes the magnitude of the upper and
lower camber at any point. Another reference line, drawn

6

4

VELOCITY

8
0

NOTE: The two extremities of the airfoil profile also differ in
appearance. The rounded end, which faces forward in flight,
is called the leading edge; the other end, the trailing edge, is
quite narrow and tapered.

I0

2

6

4

PRESSURE

8
0

I0

2

6

4

VELOCITY

8
0

I0

2

6

PRESSURE

8
0

I0

Mean camber line

Trailing edge

Camber of upper surface
Camber of lower surface

Leading edge

lift and is not suitable for high-speed flight. Advancements
in engineering have made it possible for today's high-speed
jets to take advantage of the concave airfoil's high lift
characteristics. Leading edge (Kreuger) flaps and trailing
edge (Fowler) flaps, when extended from the basic wing
structure, literally change the airfoil shape into the classic
concave form, thereby generating much greater lift during
slow flight conditions.

Chord line

Figure 4-5. Typical airfoil section.

from the leading edge to the trailing edge, is the mean camber
line. This mean line is equidistant at all points from the upper
and lower surfaces.
An airfoil is constructed in such a way that its shape takes
advantage of the air's response to certain physical laws. This
develops two actions from the air mass: a positive pressure
lifting action from the air mass below the wing, and a negative
pressure lifting action from lowered pressure above the wing.
As the air stream strikes the relatively flat lower surface of
a wing or rotor blade when inclined at a small angle to its
direction of motion, the air is forced to rebound downward,
causing an upward reaction in positive lift. At the same time,
the air stream striking the upper curved section of the leading
edge is deflected upward. An airfoil is shaped to cause an
action on the air, and forces air downward, which provides
an equal reaction from the air, forcing the airfoil upward. If
a wing is constructed in such form that it causes a lift force
greater than the weight of the aircraft, the aircraft will fly.
If all the lift required were obtained merely from the
deflection of air by the lower surface of the wing, an aircraft
would only need a flat wing like a kite. However, the balance
of the lift needed to support the aircraft comes from the flow
of air above the wing. Herein lies the key to flight.
It is neither accurate nor useful to assign specific values to the
percentage of lift generated by the upper surface of an airfoil
versus that generated by the lower surface. These are not
constant values. They vary, not only with flight conditions,
but also with different wing designs.
Different airfoils have different flight characteristics. Many
thousands of airfoils have been tested in wind tunnels and in
actual flight, but no one airfoil has been found that satisfies
every flight requirement. The weight, speed, and purpose
of each aircraft dictate the shape of its airfoil. The most
efficient airfoil for producing the greatest lift is one that has
a concave or "scooped out" lower surface. As a fixed design,
this type of airfoil sacrifices too much speed while producing

On the other hand, an airfoil that is perfectly streamlined
and offers little wind resistance sometimes does not have
enough lifting power to take the airplane off the ground.
Thus, modern airplanes have airfoils that strike a medium
between extremes in design. The shape varies according to
the needs of the airplane for which it is designed. Figure 4-6
shows some of the more common airfoil designs.
Low Pressure Above
In a wind tunnel or in flight, an airfoil is simply a streamlined
object inserted into a moving stream of air. If the airfoil
profile were in the shape of a teardrop, the speed and the
pressure changes of the air passing over the top and bottom
would be the same on both sides. But if the teardrop shaped
airfoil were cut in half lengthwise, a form resembling the
basic airfoil (wing) section would result. If the airfoil were
then inclined so the airflow strikes it at an angle, the air
moving over the upper surface would be forced to move
faster than the air moving along the bottom of the airfoil.
This increased velocity reduces the pressure above the airfoil.
Applying Bernoulli's Principle of Pressure, the increase in
the speed of the air across the top of an airfoil produces a

Early airfoil

Later airfoil

Clark 'Y' airfoil
(Subsonic)

Laminar flow airfoil
(Subsonic)

Circular arc airfoil
(Supersonic)
Double wedge airfoil
(Supersonic)
Figure 4-6. Airfoil designs.

4-7

drop in pressure. This lowered pressure is a component of
total lift. The pressure difference between the upper and
lower surface of a wing alone does not account for the total
lift force produced.

Airfoil Behavior
Although specific examples can be cited in which each of
the principles predict and contribute to the formation of lift,
4-8

-8°

Angle
of
attack

+4°

CP

High angle of attack

CP

+10°

Pressure Distribution
From experiments conducted on wind tunnel models and on
full size airplanes, it has been determined that as air flows
along the surface of a wing at different angles of attack
(AOA), there are regions along the surface where the pressure
is negative, or less than atmospheric, and regions where the
pressure is positive, or greater than atmospheric. This negative
pressure on the upper surface creates a relatively larger force
on the wing than is caused by the positive pressure resulting
from the air striking the lower wing surface. Figure 4-7 shows
the pressure distribution along an airfoil at three different
angles of attack. The average of the pressure variation for
any given AOA is referred to as the center of pressure (CP).
Aerodynamic force acts through this CP. At high angles of
attack, the CP moves forward, while at low angles of attack
the CP moves aft. In the design of wing structures, this CP
travel is very important, since it affects the position of the
air loads imposed on the wing structure in both low and high
AOA conditions. An airplane's aerodynamic balance and
controllability are governed by changes in the CP.

Normal angle of attack

Angle
of
attack

High Pressure Below
A certain amount of lift is generated by pressure conditions
underneath the airfoil. Because of the manner in which air
flows underneath the airfoil, a positive pressure results,
particularly at higher angles of attack. However, there is
another aspect to this airflow that must be considered. At a
point close to the leading edge, the airflow is virtually stopped
(stagnation point) and then gradually increases speed. At
some point near the trailing edge, it again reaches a velocity
equal to that on the upper surface. In conformance with
Bernoulli's principle, where the airflow was slowed beneath
the airfoil, a positive upward pressure was created (i.e., as
the fluid speed decreases, the pressure must increase). Since
the pressure differential between the upper and lower surface
of the airfoil increases, total lift increases. Both Bernoulli's
Principle and Newton's Laws are in operation whenever lift
is being generated by an airfoil.

CP

Angle
o
attack f

The downward backward flow from the top surface of an
airfoil creates a downwash. This downwash meets the flow
from the bottom of the airfoil at the trailing edge. Applying
Newton's third law, the reaction of this downward backward
flow results in an upward forward force on the airfoil.

Low angle of attack

Figure 2-8. Pressure distribution on an airfoil \& CP changes
with an angle of attack.
with AOA.

Figure 4-7. Pressure distribution on an airfoil and CP changes

lift is a complex subject. The production of lift is much more
complex than a simple differential pressure between upper
and lower airfoil surfaces. In fact, many lifting airfoils do
not have an upper surface longer than the bottom, as in the
case of symmetrical airfoils. These are seen in high-speed
aircraft having symmetrical wings, or on symmetrical rotor
blades for many helicopters whose upper and lower surfaces

are identical. In both examples, the only difference is the
relationship of the airfoil with the oncoming airstream
(angle). A paper airplane, which is simply a flat plate, has a
bottom and top exactly the same shape and length. Yet, these
airfoils do produce lift, and "flow turning" is partly (or fully)
responsible for creating lift.

To this point, the discussion has centered on the flow across
the upper and lower surfaces of an airfoil. While most of the
lift is produced by these two dimensions, a third dimension,
the tip of the airfoil also has an aerodynamic effect. The highpressure area on the bottom of an airfoil pushes around the tip
to the low-pressure area on the top. [Figure 4-8] This action
creates a rotating flow called a tip vortex. The vortex flows
behind the airfoil creating a downwash that extends back to
the trailing edge of the airfoil. This downwash results in an
overall reduction in lift for the affected portion of the airfoil.
Manufacturers have developed different methods to
counteract this action. Winglets can be added to the tip of
an airfoil to reduce this flow. The winglets act as a dam
preventing the vortex from forming. Winglets can be on the
top or bottom of the airfoil. Another method of countering
the flow is to taper the airfoil tip, reducing the pressure
differential and smoothing the airflow around the tip.

tex
vor

A Third Dimension

p
Ti

As an airfoil moves through air, the airfoil is inclined
against the airflow, producing a different flow caused by the
airfoil's relationship to the oncoming air. Think of a hand
being placed outside the car window at a high speed. If the
hand is inclined in one direction or another, the hand will
move upward or downward. This is caused by deflection,
which in turn causes the air to turn about the object within
the air stream. As a result of this change, the velocity about
the object changes in both magnitude and direction, in turn
resulting in a measurable velocity force and direction.

Figure 4-8. Tip vortex.

Chapter Summary
Modern general aviation aircraft have what may be considered
high performance characteristics. Therefore, it is increasingly
necessary that pilots appreciate and understand the principles
upon which the art of flying is based. For additional
information on the principles discussed in this chapter, visit
the National Aeronautics and Space Administration (NASA)
Beginner's Guide to Aerodynamics at \url{grc.nasa.gov/
www/k-12/airplane/bga.html}.

4-9

4-10


Chapter 5

Aerodynamics
of Flight
Forces Acting on the Aircraft
Thrust, drag, lift, and weight are forces that act upon all
aircraft in flight. Understanding how these forces work and
knowing how to control them with the use of power and
flight controls are essential to flight. This chapter discusses
the aerodynamics of flight—how design, weight, load factors,
and gravity affect an aircraft during flight maneuvers.
The four forces acting on an aircraft in straight-and-level,
unaccelerated flight are thrust, drag, lift, and weight. They
are defined as follows:


Thrust—the forward force produced by the powerplant/
propeller or rotor. It opposes or overcomes the force
of drag. As a general rule, it acts parallel to the
longitudinal axis. However, this is not always the case,
as explained later.



Drag—a rearward, retarding force caused by disruption
of airflow by the wing, rotor, fuselage, and other
protruding objects. As a general rule, drag opposes
thrust and acts rearward parallel to the relative wind.



Lift—is a force that is produced by the dynamic effect
of the air acting on the airfoil, and acts perpendicular
to the flight path through the center of lift (CL) and
perpendicular to the lateral axis. In level flight, lift
opposes the downward force of weight.

5-1



Weight—the combined load of the aircraft itself, the
crew, the fuel, and the cargo or baggage. Weight is
a force that pulls the aircraft downward because of
the force of gravity. It opposes lift and acts vertically
downward through the aircraft's center of gravity (CG).

In steady flight, the sum of these opposing forces is always
zero. There can be no unbalanced forces in steady, straight
flight based upon Newton's Third Law, which states that for
every action or force there is an equal, but opposite, reaction
or force. This is true whether flying level or when climbing
or descending.
It does not mean the four forces are equal. It means the
opposing forces are equal to, and thereby cancel, the effects of
each other. In Figure 5-1, the force vectors of thrust, drag, lift,
and weight appear to be equal in value. The usual explanation
states (without stipulating that thrust and drag do not equal
weight and lift) that thrust equals drag and lift equals weight.
Although true, this statement can be misleading. It should be
understood that in straight, level, unaccelerated flight, it is
true that the opposing lift/weight forces are equal. They are
also greater than the opposing forces of thrust/drag that are
equal only to each other. Therefore, in steady flight:




The sum of all upward components of forces (not just
lift) equals the sum of all downward components of
forces (not just weight)
The sum of all forward components of forces (not just
thrust) equals the sum of all backward components of
forces (not just drag)

This refinement of the old "thrust equals drag; lift equals
weight" formula explains that a portion of thrust is directed
upward in climbs and slow flight and acts as if it were lift
while a portion of weight is directed backward opposite to the
direction of flight and acts as if it were drag. In slow flight,

thrust has an upward component. But because the aircraft is in
level flight, weight does not contribute to drag. [Figure 5-2]
In glides, a portion of the weight vector is directed along
the forward flight path and, therefore, acts as thrust. In other
words, any time the flight path of the aircraft is not horizontal,
lift, weight, thrust, and drag vectors must each be broken down
into two components.
Another important concept to understand is angle of attack
(AOA). Since the early days of flight, AOA is fundamental to
understanding many aspects of airplane performance, stability,
and control. The AOA is defined as the acute angle between the
chord line of the airfoil and the direction of the relative wind.
Discussions of the preceding concepts are frequently omitted
in aeronautical texts/handbooks/manuals. The reason is
not that they are inconsequential, but because the main
ideas with respect to the aerodynamic forces acting upon
an aircraft in flight can be presented in their most essential
elements without being involved in the technicalities of the
aerodynamicist. In point of fact, considering only level flight,
and normal climbs and glides in a steady state, it is still true
that lift provided by the wing or rotor is the primary upward
force, and weight is the primary downward force.
By using the aerodynamic forces of thrust, drag, lift, and
weight, pilots can fly a controlled, safe flight. A more detailed
discussion of these forces follows.
Thrust
For an aircraft to start moving, thrust must be exerted and be
greater than drag. The aircraft continues to move and gain
speed until thrust and drag are equal. In order to maintain a
Fligh

t pa

Rela

tive

th

wind

Lift
CL

Thru

Lift

st

CG

Drag

Thrust

Weight

Figure 5-1. Relationship of forces acting on an aircraft.

5-2

Component of weight
opposed to lift

Rearward component of weight
Figure 5-2. Force vectors during a stabilized climb.

Dra

g

constant airspeed, thrust and drag must remain equal, just as
lift and weight must be equal to maintain a constant altitude.
If in level flight, the engine power is reduced, the thrust is
lessened, and the aircraft slows down. As long as the thrust
is less than the drag, the aircraft continues to decelerate. To
a point, as the aircraft slows down, the drag force will also
decrease. The aircraft will continue to slow down until thrust
again equals drag at which point the airspeed will stabilize.

In level flight, when thrust is increased, the aircraft speeds
up and the lift increases. The aircraft will start to climb
unless the AOA is decreased just enough to maintain the
relationship between lift and weight. The timing of this
decrease in AOA needs to be coordinated with the increase
in thrust and airspeed. Otherwise, if the AOA is decreased too
fast, the aircraft will descend, and if the AOA is decreased
too slowly, the aircraft will climb.

Likewise, if the engine power is increased, thrust becomes
greater than drag and the airspeed increases. As long as
the thrust continues to be greater than the drag, the aircraft
continues to accelerate. When drag equals thrust, the aircraft
flies at a constant airspeed.

As the airspeed varies due to thrust, the AOA must also vary
to maintain level flight. At very high speeds and level flight,
it is even possible to have a slightly negative AOA. As thrust
is reduced and airspeed decreases, the AOA must increase
in order to maintain altitude. If speed decreases enough, the
required AOA will increase to the critical AOA. Any further
increase in the AOA will result in the wing stalling. Therefore,
extra vigilance is required at reduced thrust settings and low
speeds so as not to exceed the critical angle of attack. If the
airplane is equipped with an AOA indicator, it should be
referenced to help monitor the proximity to the critical AOA.

Straight-and-level flight may be sustained at a wide range
of speeds. The pilot coordinates AOA and thrust in all
speed regimes if the aircraft is to be held in level flight. An
important fact related to the principal of lift (for a given
airfoil shape) is that lift varies with the AOA and airspeed.
Therefore, a large AOA at low airspeeds produces an equal
amount of lift at high airspeeds with a low AOA. The speed
regimes of flight can be grouped in three categories: lowspeed flight, cruising flight, and high-speed flight.

Some aircraft have the ability to change the direction of the
thrust rather than changing the AOA. This is accomplished
either by pivoting the engines or by vectoring the exhaust
gases. [Figure 5-4]

When the airspeed is low, the AOA must be relatively high
if the balance between lift and weight is to be maintained.
[Figure 5-3] If thrust decreases and airspeed decreases, lift
will become less than weight and the aircraft will start to
descend. To maintain level flight, the pilot can increase the
AOA an amount that generates a lift force again equal to the
weight of the aircraft. While the aircraft will be flying more
slowly, it will still maintain level flight. The AOA is adjusted
to maintain lift equal weight. The airspeed will naturally
adjust until drag equals thrust and then maintain that airspeed
(assumes the pilot is not trying to hold an exact speed).

Lift
The pilot can control the lift. Any time the control yoke
or stick is moved fore or aft, the AOA is changed. As the
AOA increases, lift increases (all other factors being equal).
When the aircraft reaches the maximum AOA, lift begins
to diminish rapidly. This is the stalling AOA, known as
CL-MAX critical AOA. Examine Figure 5-5, noting how the
CL increases until the critical AOA is reached, then decreases
rapidly with any further increase in the AOA.

Straight-and-level flight in the slow-speed regime provides
some interesting conditions relative to the equilibrium of
forces. With the aircraft in a nose-high attitude, there is a
vertical component of thrust that helps support it. For one
thing, wing loading tends to be less than would be expected.

Before proceeding further with the topic of lift and how it
can be controlled, velocity must be discussed. The shape of
the wing or rotor cannot be effective unless it continually
keeps "attacking" new air. If an aircraft is to keep flying, the
lift-producing airfoil must keep moving. In a helicopter or
gyroplane, this is accomplished by the rotation of the rotor
blades. For other types of aircraft, such as airplanes, weight-

Level cruise speed

Level low speed

3°

6°

1 2°

Level high speed

Flight path

Flight path

Flight path

Relative wind

Relative wind

Relative wind

Figure 5-3. Angle of attack at various speeds.

5-3

Figure 5-4. Some aircraft have the ability to change the direction of thrust.

shift control, or gliders, air must be moving across the lifting
surface. This is accomplished by the forward speed of the
aircraft. Lift is proportional to the square of the aircraft's
velocity. For example, an airplane traveling at 200 knots has
four times the lift as the same airplane traveling at 100 knots,
if the AOA and other factors remain constant.
CL . ρ . V2 . S
2

The above lift equation exemplifies this mathematically
and supports that doubling of the airspeed will result in four
times the lift. As a result, one can see that velocity is an
important component to the production of lift, which itself
can be affected through varying AOA. When examining the
equation, lift (L) is determined through the relationship of the
air density (ρ), the airfoil velocity (V), the surface area of the
wing (S) and the coefficient of lift (CL) for a given airfoil.

All other factors being constant, for every AOA there is
a corresponding airspeed required to maintain altitude in
steady, unaccelerated flight (true only if maintaining level
flight). Since an airfoil always stalls at the same AOA, if
increasing weight, lift must also be increased. The only

CL

.2000
CL

Coefficient of drag (CD)

.1800
.1600
L/D

.1400

MAX

.1200

CL

.1000
L/D

.0800
.0600

CD

.0400

Stall

.0200
0

0°

2°

4°

6°

8°

10°

12° 14° 16° 18°
Angle of attack

Figure 5-5. Coefficients of lift and drag at various angles of attack.

5-4

MAX

20°

22°

1.8

18

1.6

16

1.4

14

1.2

12

1.0

10

0.8

8

0.6

6

0.4

4

0.2

2

0

0

Lift/drag

L=

Taking the equation further, one can see an aircraft could
not continue to travel in level flight at a constant altitude and
maintain the same AOA if the velocity is increased. The lift
would increase and the aircraft would climb as a result of
the increased lift force or speed up. Therefore, to keep the
aircraft straight and level (not accelerating upward) and in a
state of equilibrium, as velocity is increased, lift must be kept
constant. This is normally accomplished by reducing the AOA
by lowering the nose. Conversely, as the aircraft is slowed, the
decreasing velocity requires increasing the AOA to maintain
lift sufficient to maintain flight. There is, of course, a limit to
how far the AOA can be increased, if a stall is to be avoided.

method of increasing lift is by increasing velocity if the AOA
is held constant just short of the "critical," or stalling, AOA
(assuming no flaps or other high lift devices).
Lift and drag also vary directly with the density of the air.
Density is affected by several factors: pressure, temperature,
and humidity. At an altitude of 18,000 feet, the density of
the air has one-half the density of air at sea level. In order to
maintain its lift at a higher altitude, an aircraft must fly at a
greater true airspeed for any given AOA.
Warm air is less dense than cool air, and moist air is less
dense than dry air. Thus, on a hot humid day, an aircraft
must be flown at a greater true airspeed for any given AOA
than on a cool, dry day.
If the density factor is decreased and the total lift must equal
the total weight to remain in flight, it follows that one of the
other factors must be increased. The factor usually increased
is the airspeed or the AOA because these are controlled
directly by the pilot.
Lift varies directly with the wing area, provided there is no
change in the wing's planform. If the wings have the same
proportion and airfoil sections, a wing with a planform area
of 200 square feet lifts twice as much at the same AOA as a
wing with an area of 100 square feet.
Two major aerodynamic factors from the pilot's viewpoint
are lift and airspeed because they can be controlled readily
and accurately. Of course, the pilot can also control density by
adjusting the altitude and can control wing area if the aircraft
happens to have flaps of the type that enlarge wing area.
However, for most situations, the pilot controls lift and airspeed
to maneuver an aircraft. For instance, in straight-and-level flight,
cruising along at a constant altitude, altitude is maintained by
adjusting lift to match the aircraft's velocity or cruise airspeed,
while maintaining a state of equilibrium in which lift equals
weight. In an approach to landing, when the pilot wishes to
land as slowly as practical, it is necessary to increase AOA near
maximum to maintain lift equal to the weight of the aircraft.

flow around the body, and a reference area associated with
the body. The coefficient of drag is also dimensionless and is
used to quantify the drag of an object in a fluid environment,
such as air, and is always associated with a particular surface
area.
The L/D ratio is determined by dividing the CL by the CD,
which is the same as dividing the lift equation by the drag
equation as all of the variables, aside from the coefficients,
cancel out. The lift and drag equations are as follows (L =
Lift in pounds; D = Drag; CL = coefficient of lift; ρ = density
(expressed in slugs per cubic feet); V = velocity (in feet per
second); q = dynamic pressure per square foot (q = 1⁄2 ρv2);
S = the area of the lifting body (in square feet); and
CD = Ratio of drag pressure to dynamic pressure):
D=

CD . ρ . V2 . S
2

Typically at low AOA, the coefficient of drag is low and
small changes in AOA create only slight changes in the
coefficient of drag. At high AOA, small changes in the AOA
cause significant changes in drag. The shape of an airfoil, as
well as changes in the AOA, affects the production of lift.
Notice in Figure 5-5 that the coefficient of lift curve (red)
reaches its maximum for this particular wing section at 20°
AOA and then rapidly decreases. 20° AOA is therefore the
critical angle of attack. The coefficient of drag curve (orange)
increases very rapidly from 14° AOA and completely
overcomes the lift curve at 21° AOA. The lift/drag ratio
(green) reaches its maximum at 6° AOA, meaning that at this
angle, the most lift is obtained for the least amount of drag.
Note that the maximum lift/drag ratio (L/DMAX) occurs at
one specific CL and AOA. If the aircraft is operated in steady
flight at L/DMAX, the total drag is at a minimum. Any AOA
lower or higher than that for L/DMAX reduces the L/D and
consequently increases the total drag for a given aircraft's

The coefficient of lift is dimensionless and relates the lift
generated by a lifting body, the dynamic pressure of the fluid

Drag

ras

Total drag

pa

The lift-to-drag ratio (L/D) is the amount of lift generated by
a wing or airfoil compared to its drag. A ratio of L/D indicates
airfoil efficiency. Aircraft with higher L/D ratios are more
efficient than those with lower L/D ratios. In unaccelerated
flight with the lift and drag data steady, the proportions of
the coefficient of lift (CL) and coefficient of drag (CD) can
be calculated for specific AOA. [Figure 5-5]

ite
dra
g

Lift/Drag Ratio

Minimum
drag

Induce

d drag

Airspeed
Figure 5-6. Drag versus speed.

5-5

lift. Figure 5-6 depicts the L/DMAX by the lowest portion of
the blue line labeled "total drag." The configuration of an
aircraft has a great effect on the L/D.
Drag
Drag is the force that resists movement of an aircraft through
the air. There are two basic types: parasite drag and induced
drag. The first is called parasite because it in no way functions
to aid flight, while the second, induced drag, is a result of an
airfoil developing lift.

Parasite Drag
Parasite drag is comprised of all the forces that work to slow
an aircraft's movement. As the term parasite implies, it is the
drag that is not associated with the production of lift. This
includes the displacement of the air by the aircraft, turbulence
generated in the airstream, or a hindrance of air moving over
the surface of the aircraft and airfoil. There are three types of
parasite drag: form drag, interference drag, and skin friction.
Form Drag
Form drag is the portion of parasite drag generated by the
aircraft due to its shape and airflow around it. Examples include
the engine cowlings, antennas, and the aerodynamic shape of
other components. When the air has to separate to move around
a moving aircraft and its components, it eventually rejoins
after passing the body. How quickly and smoothly it rejoins is
representative of the resistance that it creates, which requires
additional force to overcome. [Figure 5-7]
Notice how the flat plate in Figure 5-7 causes the air to swirl
around the edges until it eventually rejoins downstream. Form

FLAT PLATE

drag is the easiest to reduce when designing an aircraft. The
solution is to streamline as many of the parts as possible.
Interference Drag
Interference drag comes from the intersection of airstreams
that creates eddy currents, turbulence, or restricts smooth
airflow. For example, the intersection of the wing and the
fuselage at the wing root has significant interference drag.
Air flowing around the fuselage collides with air flowing over
the wing, merging into a current of air different from the two
original currents. The most interference drag is observed when
two surfaces meet at perpendicular angles. Fairings are used
to reduce this tendency. If a jet fighter carries two identical
wing tanks, the overall drag is greater than the sum of the
individual tanks because both of these create and generate
interference drag. Fairings and distance between lifting
surfaces and external components (such as radar antennas
hung from wings) reduce interference drag. [Figure 5-8]
Skin Friction Drag
Skin friction drag is the aerodynamic resistance due to the
contact of moving air with the surface of an aircraft. Every
surface, no matter how apparently smooth, has a rough,
ragged surface when viewed under a microscope. The air
molecules, which come in direct contact with the surface of
the wing, are virtually motionless. Each layer of molecules
above the surface moves slightly faster until the molecules
are moving at the velocity of the air moving around the
aircraft. This speed is called the free-stream velocity. The area
between the wing and the free-stream velocity level is about as
wide as a playing card and is called the boundary layer. At the
top of the boundary layer, the molecules increase velocity and
move at the same speed as the molecules outside the boundary
layer. The actual speed at which the molecules move depends
upon the shape of the wing, the viscosity (stickiness) of
the air through which the wing or airfoil is moving, and its
compressibility (how much it can be compacted).

SPHERE

SPHERE WITH
A FAIRING

SPHERE INSIDE
A HOUSING

Figure 5-7. Form drag.

5-6

Figure 5-8. A wing root can cause interference drag.

The airflow outside of the boundary layer reacts to the
shape of the edge of the boundary layer just as it would
to the physical surface of an object. The boundary layer
gives any object an "effective" shape that is usually slightly
different from the physical shape. The boundary layer may
also separate from the body, thus creating an effective shape
much different from the physical shape of the object. This
change in the physical shape of the boundary layer causes a
dramatic decrease in lift and an increase in drag. When this
happens, the airfoil has stalled.
In order to reduce the effect of skin friction drag, aircraft
designers utilize flush mount rivets and remove any
irregularities that may protrude above the wing surface. In
addition, a smooth and glossy finish aids in transition of
air across the surface of the wing. Since dirt on an aircraft
disrupts the free flow of air and increases drag, keep the
surfaces of an aircraft clean and waxed.

Induced Drag
The second basic type of drag is induced drag. It is an
established physical fact that no system that does work in the
mechanical sense can be 100 percent efficient. This means
that whatever the nature of the system, the required work
is obtained at the expense of certain additional work that is
dissipated or lost in the system. The more efficient the system,
the smaller this loss.
In level flight, the aerodynamic properties of a wing or rotor
produce a required lift, but this can be obtained only at the
expense of a certain penalty. The name given to this penalty
is induced drag. Induced drag is inherent whenever an airfoil
is producing lift and, in fact, this type of drag is inseparable
from the production of lift. Consequently, it is always present
if lift is produced.

Figure 5-9. Wingtip vortex from a crop duster.

altitude versus near the ground. Bearing in mind the direction
of rotation of these vortices, it can be seen that they induce
an upward flow of air beyond the tip and a downwash flow
behind the wing's trailing edge. This induced downwash has
nothing in common with the downwash that is necessary to
produce lift. It is, in fact, the source of induced drag.
Downwash points the relative wind downward, so the more
downwash you have, the more your relative wind points
downward. That's important for one very good reason: lift is
always perpendicular to the relative wind. In Figure 5-11, you
can see that when you have less downwash, your lift vector
is more vertical, opposing gravity. And when you have more
downwash, your lift vector points back more, causing induced
drag. On top of that, it takes energy for your wings to create
downwash and vortices, and that energy creates drag.

An airfoil (wing or rotor blade) produces the lift force by
making use of the energy of the free airstream. Whenever
an airfoil is producing lift, the pressure on the lower surface
of it is greater than that on the upper surface (Bernoulli's
Principle). As a result, the air tends to flow from the high
pressure area below the tip upward to the low pressure area
on the upper surface. In the vicinity of the tips, there is a
tendency for these pressures to equalize, resulting in a lateral
flow outward from the underside to the upper surface. This
lateral flow imparts a rotational velocity to the air at the tips,
creating vortices that trail behind the airfoil.
When the aircraft is viewed from the tail, these vortices
circulate counterclockwise about the right tip and clockwise
about the left tip. [Figure 5-9] As the air (and vortices) roll off
the back of your wing, they angle down, which is known as
downwash. Figure 5-10 shows the difference in downwash at

Figure 5-10. The difference in wingtip vortex size at altitude versus
near the ground.

5-7

Weight
Gravity is the pulling force that tends to draw all bodies to
the center of the earth. The CG may be considered as a point
at which all the weight of the aircraft is concentrated. If the
aircraft were supported at its exact CG, it would balance in
any attitude. It will be noted that CG is of major importance in
an aircraft, for its position has a great bearing upon stability.
The allowable location of the CG is determined by the general
design of each particular aircraft. The designers determine
how far the center of pressure (CP) will travel. It is important
to understand that an aircraft's weight is concentrated at the
CG and the aerodynamic forces of lift occur at the CP. When
the CG is forward of the CP, there is a natural tendency for
the aircraft to want to pitch nose down. If the CP is forward
of the CG, a nose up pitching moment is created. Therefore,
designers fix the aft limit of the CG forward of the CP for the
corresponding flight speed in order to retain flight equilibrium.

Figure 5-11. The difference in downwash at altitude versus near

the ground.

The greater the size and strength of the vortices and
consequent downwash component on the net airflow over
the airfoil, the greater the induced drag effect becomes. This
downwash over the top of the airfoil at the tip has the same
effect as bending the lift vector rearward; therefore, the lift
is slightly aft of perpendicular to the relative wind, creating
a rearward lift component. This is induced drag.
In order to create a greater negative pressure on the top of an
airfoil, the airfoil can be inclined to a higher AOA. If the AOA
of a symmetrical airfoil were zero, there would be no pressure
differential, and consequently, no downwash component and
no induced drag. In any case, as AOA increases, induced
drag increases proportionally. To state this another way—the
lower the airspeed, the greater the AOA required to produce
lift equal to the aircraft's weight and, therefore, the greater
induced drag. The amount of induced drag varies inversely
with the square of the airspeed.
Conversely, parasite drag increases as the square of the
airspeed. Thus, in steady state, as airspeed decreases to
near the stalling speed, the total drag becomes greater, due
mainly to the sharp rise in induced drag. Similarly, as the
aircraft reaches its never-exceed speed (VNE), the total drag
increases rapidly due to the sharp increase of parasite drag.
As seen in Figure 5-6, at some given airspeed, total drag is
at its minimum amount. In figuring the maximum range of
aircraft, the thrust required to overcome drag is at a minimum
if drag is at a minimum. The minimum power and maximum
endurance occur at a different point.

5-8

Weight has a definite relationship to lift. This relationship
is simple, but important in understanding the aerodynamics
of flying. Lift is the upward force on the wing acting
perpendicular to the relative wind and perpendicular to
the aircraft's lateral axis. Lift is required to counteract the
aircraft's weight. In stabilized level flight, when the lift force is
equal to the weight force, the aircraft is in a state of equilibrium
and neither accelerates upward or downward. If lift becomes
less than weight, the vertical speed will decrease. When lift is
greater than weight, the vertical speed will increase.

Wingtip Vortices
Formation of Vortices
The action of the airfoil that gives an aircraft lift also causes
induced drag. When an airfoil is flown at a positive AOA,
a pressure differential exists between the upper and lower
surfaces of the airfoil. The pressure above the wing is less
than atmospheric pressure and the pressure below the wing
is equal to or greater than atmospheric pressure. Since air
always moves from high pressure toward low pressure,
and the path of least resistance is toward the airfoil's tips,
there is a spanwise movement of air from the bottom of the
airfoil outward from the fuselage around the tips. This flow
of air results in "spillage" over the tips, thereby setting up a
whirlpool of air called a vortex. [Figure 5-12]
At the same time, the air on the upper surface has a tendency
to flow in toward the fuselage and off the trailing edge. This
air current forms a similar vortex at the inboard portion of the
trailing edge of the airfoil, but because the fuselage limits the
inward flow, the vortex is insignificant. Consequently, the
deviation in flow direction is greatest at the outer tips where
the unrestricted lateral flow is the strongest.

vortices lead to a particularly dangerous hazard to flight,
wake turbulence.
Avoiding Wake Turbulence
Wingtip vortices are greatest when the generating aircraft is
"heavy, clean, and slow." This condition is most commonly
encountered during approaches or departures because an
aircraft's AOA is at the highest to produce the lift necessary
to land or take off. To minimize the chances of flying through
an aircraft's wake turbulence:

tex
r
o
V

Figure 5-12. Wingtip vortices.

As the air curls upward around the tip, it combines with the
downwash to form a fast-spinning trailing vortex. These
vortices increase drag because of energy spent in producing
the turbulence. Whenever an airfoil is producing lift, induced
drag occurs and wingtip vortices are created.
Just as lift increases with an increase in AOA, induced
drag also increases. This occurs because as the AOA is
increased, there is a greater pressure difference between the
top and bottom of the airfoil, and a greater lateral flow of air;
consequently, this causes more violent vortices to be set up,
resulting in more turbulence and more induced drag.
In Figure 5-12, it is easy to see the formation of wingtip
vortices. The intensity or strength of the vortices is directly
proportional to the weight of the aircraft and inversely
proportional to the wingspan and speed of the aircraft. The
heavier and slower the aircraft, the greater the AOA and the
stronger the wingtip vortices. Thus, an aircraft will create
wingtip vortices with maximum strength occurring during
the takeoff, climb, and landing phases of flight. These



Avoid flying through another aircraft's flight path.



Rotate prior to the point at which the preceding aircraft
rotated when taking off behind another aircraft.



Avoid following another aircraft on a similar flight
path at an altitude within 1,000 feet. [Figure 5-13]



Approach the runway above a preceding aircraft's
path when landing behind another aircraft and touch
down after the point at which the other aircraft wheels
contacted the runway. [Figure 5-14]

A hovering helicopter generates a down wash from its main
rotor(s) similar to the vortices of an airplane. Pilots of small
aircraft should avoid a hovering helicopter by at least three
rotor disc diameters to avoid the effects of this down wash. In
forward flight, this energy is transformed into a pair of strong,
high-speed trailing vortices similar to wing-tip vortices of larger
fixed-wing aircraft. Helicopter vortices should be avoided
because helicopter forward flight airspeeds are often very
slow and can generate exceptionally strong wake turbulence.
Wind is an important factor in avoiding wake turbulence
because wingtip vortices drift with the wind at the speed of the
wind. For example, a wind speed of 10 knots causes the vortices
to drift at about 1,000 feet in a minute in the wind direction.
When following another aircraft, a pilot should consider wind
speed and direction when selecting an intended takeoff or
landing point. If a pilot is unsure of the other aircraft's takeoff
or landing point, approximately 3 minutes provides a margin of
Nominally
500–1,000 ft

AVOID

Sink rate
several hundred ft/min

Figure 5-13. Avoid following another aircraft at an altitude within 1,000 feet.

5-9

25

Wake ends

Wake begins
Rotation

Touchdown

Figure 5-14. Avoid turbulence from another aircraft.

3K

3K
No Wind

Vortex Movement Near Ground - No Wind

3K Wind

6K

0 (3K - 3K)

(3K + 3K)

Vortex Movement Near Ground - with Cross Winds

Figure 5-15. When the vortices of larger aircraft sink close to the ground (within 100 to 200 feet), they tend to move laterally over the
ground at a speed of 2 or 3 knots (top). A crosswind will decrease the lateral movement of the upwind vortex and increase the movement
of the downwind vortex. Thus a light wind with a cross runway component of 1 to 5 knots could result in the upwind vortex remaining in
the touchdown zone for a period of time and hasten the drift of the downwind vortex toward another runway (bottom).

5-10

safety that allows wake turbulence dissipation. [Figure 5-15]
For more information on wake turbulence, see Advisory
Circular (AC) 90-23, Aircraft Wake Turbulence.

Ever since the beginning of manned flight, pilots realized
that just before touchdown it would suddenly feel like the
aircraft did not want to go lower, and it would just want to go
on and on. This is due to the air that is trapped between the
wing and the landing surface, as if there were an air cushion.
This phenomenon is called ground effect.
When an aircraft in flight comes within several feet of the
surface, ground or water, a change occurs in the threedimensional flow pattern around the aircraft because the
vertical component of the airflow around the wing is
restricted by the surface. This alters the wing's upwash,
downwash, and wingtip vortices. [Figure 5-16] Ground
effect, then, is due to the interference of the ground (or water)
surface with the airflow patterns about the aircraft in flight.
While the aerodynamic characteristics of the tail surfaces
and the fuselage are altered by ground effect, the principal
effects due to proximity of the ground are the changes in
the aerodynamic characteristics of the wing. As the wing
encounters ground effect and is maintained at a constant
AOA, there is consequent reduction in the upwash,
downwash, and wingtip vortices.
Induced drag is a result of the airfoil's work of sustaining
the aircraft, and a wing or rotor lifts the aircraft simply by
accelerating a mass of air downward. It is true that reduced
pressure on top of an airfoil is essential to lift, but that is
only one of the things contributing to the overall effect of
pushing an air mass downward. The more downwash there
is, the harder the wing pushes the mass of air down. At high
angles of attack, the amount of induced drag is high; since this
corresponds to lower airspeeds in actual flight, it can be said
that induced drag predominates at low speed. However, the
reduction of the wingtip vortices due to ground effect alters

14

Ground Effect

Figure 5-16. Ground effect changes airflow.

the spanwise lift distribution and reduces the induced AOA
and induced drag. Therefore, the wing will require a lower
AOA in ground effect to produce the same CL. If a constant
AOA is maintained, an increase in CL results. [Figure 5-17]
Ground effect also alters the thrust required versus velocity.
Since induced drag predominates at low speeds, the reduction
of induced drag due to ground effect will cause a significant
reduction of thrust required (parasite plus induced drag) at low
speeds. Due to the change in upwash, downwash, and wingtip
vortices, there may be a change in position (installation) error
of the airspeed system associated with ground effect. In the
majority of cases, ground effect causes an increase in the local
pressure at the static source and produces a lower indication
of airspeed and altitude. Thus, an aircraft may be airborne at
an indicated airspeed less than that normally required.
In order for ground effect to be of significant magnitude, the
wing must be quite close to the ground. One of the direct
results of ground effect is the variation of induced drag with
wing height above the ground at a constant CL. When the
wing is at a height equal to its span, the reduction in induced
drag is only 1.4 percent. However, when the wing is at a
height equal to one-fourth its span, the reduction in induced
drag is 23.5 percent and, when the wing is at a height equal
to one-tenth its span, the reduction in induced drag is 47.6
percent. Thus, a large reduction in induced drag takes place
only when the wing is very close to the ground. Because of
this variation, ground effect is most usually recognized during
the liftoff for takeoff or just prior to touchdown when landing.

Out of ground effect

Lift coefficient CL

Thrust required

In ground effect

Out of ground effect

In ground effect
Velocity

Angle of attack

Figure 5-17. Ground effect changes drag and lift.

5-11

During the takeoff phase of flight, ground effect produces
some important relationships. An aircraft leaving ground
effect after takeoff encounters just the reverse of an aircraft
entering ground effect during landing. The aircraft leaving
ground effect will:


Require an increase in AOA to maintain the same CL



Experience an increase in induced drag and thrust
required



Experience a decrease in stability and a nose-up
change in moment



Experience a reduction in static source pressure and
increase in indicated airspeed

Ground effect must be considered during takeoffs and landings.
For example, if a pilot fails to understand the relationship
between the aircraft and ground effect during takeoff, a
hazardous situation is possible because the recommended
takeoff speed may not be achieved. Due to the reduced drag
in ground effect, the aircraft may seem capable of takeoff well
below the recommended speed. As the aircraft rises out of
ground effect with a deficiency of speed, the greater induced
drag may result in marginal initial climb performance. In
extreme conditions, such as high gross weight, high density
altitude, and high temperature, a deficiency of airspeed during
takeoff may permit the aircraft to become airborne but be
incapable of sustaining flight out of ground effect. In this case,
the aircraft may become airborne initially with a deficiency
of speed and then settle back to the runway.
A pilot should not attempt to force an aircraft to become
airborne with a deficiency of speed. The manufacturer's
recommended takeoff speed is necessary to provide adequate
initial climb performance. It is also important that a definite
climb be established before a pilot retracts the landing gear
or flaps. Never retract the landing gear or flaps prior to

Pitching

Lateral axis
Figure 5-18. Axes of an airplane.

5-12

establishing a positive rate of climb and only after achieving
a safe altitude.
If, during the landing phase of flight, the aircraft is brought
into ground effect with a constant AOA, the aircraft
experiences an increase in CL and a reduction in the thrust
required, and a "floating" effect may occur. Because of the
reduced drag and lack of power-off deceleration in ground
effect, any excess speed at the point of flare may incur a
considerable "float" distance. As the aircraft nears the point
of touchdown, ground effect is most realized at altitudes less
than the wingspan. During the final phases of the approach
as the aircraft nears the ground, a reduction of power is
necessary to offset the increase in lift caused from ground
effect otherwise the aircraft will have a tendency to climb
above the desired glidepath (GP).

Axes of an Aircraft
The axes of an aircraft are three imaginary lines that pass
through an aircraft's CG. The axes can be considered as
imaginary axles around which the aircraft turns. The three
axes pass through the CG at 90° angles to each other. The
axis passes through the CG and parallel to a line from nose
to tail is the longitudinal axis, the axis that passes parallel
to a line from wingtip to wingtip is the lateral axis, and the
axis that passes through the CG at right angles to the other
two axes is the vertical axis. Whenever an aircraft changes
its flight attitude or position in flight, it rotates about one or
more of the three axes. [Figure 5-18]
The aircraft's motion about its longitudinal axis resembles
the roll of a ship from side to side. In fact, the names
used to describe the motion about an aircraft's three axes
were originally nautical terms. They have been adapted to
aeronautical terminology due to the similarity of motion of
aircraft and seagoing ships. The motion about the aircraft's
longitudinal axis is "roll," the motion about its lateral axis is

Rolling

Longitudinal axis

Yawing

Vertical axis

"pitch," and the motion about its vertical axis is "yaw." Yaw
is the left and right movement of the aircraft's nose.
The three motions of the conventional airplane (roll, pitch,
and yaw) are controlled by three control surfaces. Roll is
controlled by the ailerons; pitch is controlled by the elevators;
yaw is controlled by the rudder. The use of these controls
is explained in Chapter 6, Flight Controls. Other types of
aircraft may utilize different methods of controlling the
movements about the various axes.
For example, weight-shift control aircraft control two axes
(roll and pitch) using an "A" frame suspended from the
flexible wing attached to a three-wheeled carriage. These
aircraft are controlled by moving a horizontal bar (called a
control bar) in roughly the same way hang glider pilots fly.
[Figure 5-19] They are termed weight-shift control aircraft
because the pilot controls the aircraft by shifting the CG.
For more information on weight-shift control aircraft, see
the Federal Aviation Administration (FAA) Weight-Shift
Control Flying Handbook, FAA-H-8083-5. In the case of
powered parachutes, aircraft control is accomplished by
altering the airfoil via steering lines.
A powered parachute wing is a parachute that has a cambered
upper surface and a flatter under surface. The two surfaces are
separated by ribs that act as cells, which open to the airflow
at the leading edge and have internal ports to allow lateral
airflow. The principle at work holds that the cell pressure is
greater than the outside pressure, thereby forming a wing that
maintains its airfoil shape in flight. The pilot and passenger
sit in tandem in front of the engine, which is located at the
rear of a vehicle. The airframe is attached to the parachute
via two attachment points and lines. Control is accomplished
by both power and the changing of the airfoil via the control
lines. [Figure 5-20]

Figure 5-19. A weight-shift control aircraft.

Figure 5-20. A powered parachute.

Moment and Moment Arm
A study of physics shows that a body that is free to rotate
will always turn about its CG. In aerodynamic terms, the
mathematical measure of an aircraft's tendency to rotate
about its CG is called a "moment." A moment is said to be
equal to the product of the force applied and the distance at
which the force is applied. (A moment arm is the distance
from a datum [reference point or line] to the applied force.)
For aircraft weight and balance computations, "moments"
are expressed in terms of the distance of the arm times the
aircraft's weight, or simply, inch-pounds.
Aircraft designers locate the fore and aft position of the
aircraft's CG as nearly as possible to the 20 percent point
of the mean aerodynamic chord (MAC). If the thrust line
is designed to pass horizontally through the CG, it will not
cause the aircraft to pitch when power is changed, and there
will be no difference in moment due to thrust for a power-on
or power-off condition of flight. Although designers have
some control over the location of the drag forces, they are not
always able to make the resultant drag forces pass through the
CG of the aircraft. However, the one item over which they
have the greatest control is the size and location of the tail.
The objective is to make the moments (due to thrust, drag, and
lift) as small as possible and, by proper location of the tail,
to provide the means of balancing an aircraft longitudinally
for any condition of flight.
The pilot has no direct control over the location of forces
acting on the aircraft in flight, except for controlling the
center of lift by changing the AOA. The pilot can control
the magnitude of the forces. Such a change, however,
immediately involves changes in other forces. Therefore,
the pilot cannot independently change the location of one
force without changing the effect of others. For example,
a change in airspeed involves a change in lift, as well as a
change in drag and a change in the up or down force on the

5-13

tail. As forces such as turbulence and gusts act to displace
the aircraft, the pilot reacts by providing opposing control
forces to counteract this displacement.
Some aircraft are subject to changes in the location of the CG
with variations of load. Trimming devices, such as elevator
trim tabs and adjustable horizontal stabilizers, are used to
counteract the moments set up by fuel burnoff and loading
or off-loading of passengers or cargo.

Static Stability

Aircraft Design Characteristics
Each aircraft handles somewhat differently because each
resists or responds to control pressures in its own way. For
example, a training aircraft is quick to respond to control
applications, while a transport aircraft feels heavy on the
controls and responds to control pressures more slowly.
These features can be designed into an aircraft to facilitate
the particular purpose of the aircraft by considering certain
stability and maneuvering requirements. The following
discussion summarizes the more important aspects of an
aircraft's stability, maneuverability, and controllability
qualities; how they are analyzed; and their relationship to
various flight conditions.
Stability
Stability is the inherent quality of an aircraft to correct for
conditions that may disturb its equilibrium and to return to
or to continue on the original flight path. It is primarily an
aircraft design characteristic. The flight paths and attitudes an
aircraft flies are limited by the aerodynamic characteristics of
the aircraft, its propulsion system, and its structural strength.
These limitations indicate the maximum performance and

Positive Static Stability

Static stability refers to the initial tendency, or direction of
movement, back to equilibrium. In aviation, it refers to the
aircraft's initial response when disturbed from a given pitch,
yaw, or bank.


Positive static stability—the initial tendency of the
aircraft to return to the original state of equilibrium
after being disturbed. [Figure 5-21]



Neutral static stability—the initial tendency of
the aircraft to remain in a new condition after its
equilibrium has been disturbed. [Figure 5-21]



Negative static stability—the initial tendency of the
aircraft to continue away from the original state of
equilibrium after being disturbed. [Figure 5-21]

Dynamic Stability
Static stability has been defined as the initial tendency to
return to equilibrium that the aircraft displays after being
disturbed from its trimmed condition. Occasionally, the
initial tendency is different or opposite from the overall
tendency, so a distinction must be made between the two.
Dynamic stability refers to the aircraft response over time

Neutral Static Stability

CG
CG

CG

Figure 5-21. Types of static stability.

Negative Static Stability

Applied
force

CG

Applied
force

Applied
force

5-14

maneuverability of the aircraft. If the aircraft is to provide
maximum utility, it must be safely controllable to the full
extent of these limits without exceeding the pilot's strength
or requiring exceptional flying ability. If an aircraft is to fly
straight and steady along any arbitrary flight path, the forces
acting on it must be in static equilibrium. The reaction of
any body when its equilibrium is disturbed is referred to as
stability. The two types of stability are static and dynamic.

when disturbed from a given pitch, yaw, or bank. This type
of stability also has three subtypes: [Figure 5-22]


Positive dynamic stability—over time, the motion
of the displaced object decreases in amplitude and,
because it is positive, the object displaced returns
toward the equilibrium state.



Neutral dynamic stability—once displaced, the
displaced object neither decreases nor increases in
amplitude. A worn automobile shock absorber exhibits
this tendency.



Negative dynamic stability—over time, the motion
of the displaced object increases and becomes more
divergent.

Stability in an aircraft affects two areas significantly:


Maneuverability—the quality of an aircraft that
permits it to be maneuvered easily and to withstand
the stresses imposed by maneuvers. It is governed by
the aircraft's weight, inertia, size and location of flight
controls, structural strength, and powerplant. It too is
an aircraft design characteristic.

 	 Controllability—the capability of an aircraft to
respond to the pilot's control, especially with regard
to flight path and attitude. It is the quality of the
aircraft's response to the pilot's control application
when maneuvering the aircraft, regardless of its
stability characteristics.

Longitudinal Stability (Pitching)
In designing an aircraft, a great deal of effort is spent in
developing the desired degree of stability around all three
axes. But longitudinal stability about the lateral axis is

considered to be the most affected by certain variables in
various flight conditions.
Longitudinal stability is the quality that makes an aircraft
stable about its lateral axis. It involves the pitching motion
as the aircraft's nose moves up and down in flight. A
longitudinally unstable aircraft has a tendency to dive or
climb progressively into a very steep dive or climb, or even
a stall. Thus, an aircraft with longitudinal instability becomes
difficult and sometimes dangerous to fly.
Static longitudinal stability, or instability in an aircraft, is
dependent upon three factors:
1.	 Location of the wing with respect to the CG
2.	 Location of the horizontal tail surfaces with respect
to the CG
3.	 Area or size of the tail surfaces
In analyzing stability, it should be recalled that a body free
to rotate always turns about its CG.
To obtain static longitudinal stability, the relation of the
wing and tail moments must be such that, if the moments
are initially balanced and the aircraft is suddenly nose up,
the wing moments and tail moments change so that the sum
of their forces provides an unbalanced but restoring moment
which, in turn, brings the nose down again. Similarly, if the
aircraft is nose down, the resulting change in moments brings
the nose back up.
The Center of Lift (CL) in most asymmetrical airfoils has a
tendency to change its fore and aft positions with a change in
the AOA. The CL tends to move forward with an increase in
AOA and to move aft with a decrease in AOA. This means

Damped oscillation
Undamped oscillation
Divergent oscillation

Displacement

Positive static
(positive dynamic)
Time

Positive static
(neutral dynamic)

Positive Static
(negative dynamic)

Figure 5-22. Damped versus undamped stability.

5-15

that when the AOA of an airfoil is increased, the CL, by
moving forward, tends to lift the leading edge of the wing
still more. This tendency gives the wing an inherent quality
of instability. (NOTE: CL is also known as the center of
pressure (CP).)
Figure 5-23 shows an aircraft in straight-and-level flight. The
line CG-CL-T represents the aircraft's longitudinal axis from
the CG to a point T on the horizontal stabilizer.

Cruise Speed

CG

Balanced tail load
Low Spe

ed

CG

Most aircraft are designed so that the wing's CL is to the rear
of the CG. This makes the aircraft "nose heavy" and requires
that there be a slight downward force on the horizontal
stabilizer in order to balance the aircraft and keep the nose
from continually pitching downward. Compensation for this
nose heaviness is provided by setting the horizontal stabilizer
at a slight negative AOA. The downward force thus produced
holds the tail down, counterbalancing the "heavy" nose. It
is as if the line CG-CL-T were a lever with an upward force
at CL and two downward forces balancing each other, one
a strong force at the CG point and the other, a much lesser
force, at point T (downward air pressure on the stabilizer).
To better visualize this physics principle: If an iron bar were
suspended at point CL, with a heavy weight hanging on it at
the CG, it would take downward pressure at point T to keep
the "lever" in balance.

High Speed

CG

Greater downward tail load
Figure 5-24. Effect of speed on downwash.

flow of air over the wing, the downwash is reduced, causing
a lesser downward force on the horizontal stabilizer. In turn,
the characteristic nose heaviness is accentuated, causing the
aircraft's nose to pitch down more. [Figure 5-25] This places
the aircraft in a nose-low attitude, lessening the wing's AOA
and drag and allowing the airspeed to increase. As the aircraft
continues in the nose-low attitude and its speed increases,
the downward force on the horizontal stabilizer is once again
increased. Consequently, the tail is again pushed downward
and the nose rises into a climbing attitude.

Lift

Even though the horizontal stabilizer may be level when the
aircraft is in level flight, there is a downwash of air from the
wings. This downwash strikes the top of the stabilizer and
produces a downward pressure, which at a certain speed is
just enough to balance the "lever." The faster the aircraft
is flying, the greater this downwash and the greater the
downward force on the horizontal stabilizer (except T-tails).
[Figure 5-24] In aircraft with fixed-position horizontal
stabilizers, the aircraft manufacturer sets the stabilizer at an
angle that provides the best stability (or balance) during flight
at the design cruising speed and power setting.

Lesser downward tail load

Thrust

CG
Weight

If the aircraft's speed decreases, the speed of the airflow
over the wing is decreased. As a result of this decreased

Normal downwash
CL
CG

Lift

CG

CL

Figure 5-23. Longitudinal stability.

5-16

CG
Weight

Thrust

Reduced downwash
Figure 5-25. Reduced power allows pitch down.

Power or thrust can also have a destabilizing effect in that
an increase of power may tend to make the nose rise. The
aircraft designer can offset this by establishing a "high
thrust line" wherein the line of thrust passes above the CG.
[Figures 5-26 and 5-27] In this case, as power or thrust is

Thrust

CG

Below center of gravity

Thrust

CG

Through center of gravity

Thrust

CG

Above center of gravity

CG

Cruise power

Thrust

CG

Idle power
Lift

A similar effect is noted upon closing the throttle. The
downwash of the wings is reduced and the force at T in
Figure 5-23 is not enough to hold the horizontal stabilizer
down. It seems as if the force at T on the lever were allowing
the force of gravity to pull the nose down. This is a desirable
characteristic because the aircraft is inherently trying to
regain airspeed and reestablish the proper balance.

Thrust

Lift

After several of these diminishing oscillations, in which
the nose alternately rises and lowers, the aircraft finally
settles down to a speed at which the downward force on the
tail exactly counteracts the tendency of the aircraft to dive.
When this condition is attained, the aircraft is once again in
balanced flight and continues in stabilized flight as long as
this attitude and airspeed are not changed.

Lift

As this climb continues, the airspeed again decreases, causing
the downward force on the tail to decrease until the nose
lowers once more. Because the aircraft is dynamically stable,
the nose does not lower as far this time as it did before. The
aircraft acquires enough speed in this more gradual dive to
start it into another climb, but the climb is not as steep as
the preceding one.

Thrust

CG

Full power
Figure 5-27. Power changes affect longitudinal stability.

increased a moment is produced to counteract the down load
on the tail. On the other hand, a very "low thrust line" would
tend to add to the nose-up effect of the horizontal tail surface.
Conclusion: with CG forward of the CL and with an
aerodynamic tail-down force, the aircraft usually tries to
return to a safe flying attitude.
The following is a simple demonstration of longitudinal
stability. Trim the aircraft for "hands off" control in level
flight. Then, momentarily give the controls a slight push to
nose the aircraft down. If, within a brief period, the nose rises
towards the original position, the aircraft is statically stable.
Ordinarily, the nose passes the original position (that of level
flight) and a series of slow pitching oscillations follows. If the
oscillations gradually cease, the aircraft has positive stability;
if they continue unevenly, the aircraft has neutral stability;
if they increase, the aircraft is unstable.

Lateral Stability (Rolling)
Stability about the aircraft's longitudinal axis, which extends
from the nose of the aircraft to its tail, is called lateral
stability. Positive lateral stability helps to stabilize the lateral
or "rolling effect" when one wing gets lower than the wing
on the opposite side of the aircraft. There are four main
design factors that make an aircraft laterally stable: dihedral,
sweepback, keel effect, and weight distribution.

Figure 5-26. Thrust line affects longitudinal stability.

5-17

Dihedral
Some aircraft are designed so that the outer tips of the wings
are higher than the wing roots. The upward angle thus formed
by the wings is called dihedral. [Figure 5-28] When a gust
causes a roll, a sideslip will result. This sideslip causes the
relative wind affecting the entire airplane to be from the
direction of the slip. When the relative wind comes from the
side, the wing slipping into the wind is subject to an increase
in AOA and develops an increase in lift. The wing away
from the wind is subject to a decrease in angle of attack, and
develops a decrease in lift. The changes in lift effect a rolling
moment tending to raise the windward wing, hence dihedral
contributes to a stable roll due to sideslip. [Figure 5-29]
Sweepback and Wing Location
Many aspects of an aircraft's configuration can affect its
effective dihedral, but two major components are wing
sweepback and the wing location with respect to the fuselage
(such as a low wing or high wing). As a rough estimation,
10° of sweepback on a wing provides about 1° of effective
dihedral, while a high wing configuration can provide about
5° of effective dihedral over a low wing configuration.

Restoring lift

Wing has decreased
AOA, hence reduced lift
due to sideslip.
Wing has increased AOA, hence
increased lift due to sideslip.

Sideslip

Figure 5-29. Sideslip causing different AOA on each blade.

A sweptback wing is one in which the leading edge slopes
backward. [Figure 5-30] When a disturbance causes an
aircraft with sweepback to slip or drop a wing, the low
wing presents its leading edge at an angle that is more
perpendicular to the relative airflow. As a result, the low
wing acquires more lift, rises, and the aircraft is restored to
its original flight attitude.
Keel Effect and Weight Distribution
A high wing aircraft always has the tendency to turn the
longitudinal axis of the aircraft into the relative wind, which is
often referred to as the keel effect. These aircraft are laterally
stable simply because the wings are attached in a high
position on the fuselage, making the fuselage behave like a
keel exerting a steadying influence on the aircraft laterally
about the longitudinal axis. When a high-winged aircraft is

Dihedral

Figure 5-30. Sweepback wings.

disturbed and one wing dips, the fuselage weight acts like a
pendulum returning the aircraft to the horizontal level.
Laterally stable aircraft are constructed so that the greater
portion of the keel area is above the CG. [Figure 5-31] Thus,
when the aircraft slips to one side, the combination of the

Dihedral

Figure 5-28. Dihedral is the upward angle of the wings from a horizontal (front/rear view) axis of the plane as shown in the graphic

depiction and the rear view of a Ryanair Boeing 737.

5-18

CG

CG

CG centerline
Area
forward
of CG

CG

Area aft of CG

CG

Relative w
ind

Yaw

aircraft's weight and the pressure of the airflow against the
upper portion of the keel area (both acting about the CG)
tends to roll the aircraft back to wings-level flight.

yaw

Figure 5-31. Keel area for lateral stability.

Directional Stability (Yawing)
Stability about the aircraft's vertical axis (the sideways
moment) is called yawing or directional stability. Yawing
or directional stability is the most easily achieved stability
in aircraft design. The area of the vertical fin and the sides
of the fuselage aft of the CG are the prime contributors that
make the aircraft act like the well known weather vane or
arrow, pointing its nose into the relative wind.
In examining a weather vane, it can be seen that if exactly the
same amount of surface were exposed to the wind in front
of the pivot point as behind it, the forces fore and aft would
be in balance and little or no directional movement would
result. Consequently, it is necessary to have a greater surface
aft of the pivot point than forward of it.
Similarly, the aircraft designer must ensure positive
directional stability by making the side surface greater aft
than ahead of the CG. [Figure 5-32] To provide additional
positive stability to that provided by the fuselage, a vertical
fin is added. The fin acts similar to the feather on an arrow
in maintaining straight flight. Like the weather vane and the
arrow, the farther aft this fin is placed and the larger its size,
the greater the aircraft's directional stability.
If an aircraft is flying in a straight line, and a sideward gust
of air gives the aircraft a slight rotation about its vertical
axis (i.e., the right), the motion is retarded and stopped by
the fin because while the aircraft is rotating to the right, the
air is striking the left side of the fin at an angle. This causes
pressure on the left side of the fin, which resists the turning
motion and slows down the aircraft's yaw. In doing so, it
acts somewhat like the weather vane by turning the aircraft
into the relative wind. The initial change in direction of the
aircraft's flight path is generally slightly behind its change
of heading. Therefore, after a slight yawing of the aircraft

Figure 5-32. Fuselage and fin for directional stability.

to the right, there is a brief moment when the aircraft is still
moving along its original path, but its longitudinal axis is
pointed slightly to the right.
The aircraft is then momentarily skidding sideways and,
during that moment (since it is assumed that although the
yawing motion has stopped, the excess pressure on the left
side of the fin still persists), there is necessarily a tendency
for the aircraft to be turned partially back to the left. That is,
there is a momentary restoring tendency caused by the fin.
This restoring tendency is relatively slow in developing and
ceases when the aircraft stops skidding. When it ceases, the
aircraft is flying in a direction slightly different from the
original direction. In other words, it will not return of its own
accord to the original heading; the pilot must reestablish the
initial heading.
A minor improvement of directional stability may be obtained
through sweepback. Sweepback is incorporated in the design
of the wing primarily to delay the onset of compressibility
during high-speed flight. In lighter and slower aircraft,
sweepback aids in locating the center of pressure in the
correct relationship with the CG. A longitudinally stable
aircraft is built with the center of pressure aft of the CG.
Because of structural reasons, aircraft designers sometimes
cannot attach the wings to the fuselage at the exact desired

5-19

point. If they had to mount the wings too far forward, and at
right angles to the fuselage, the center of pressure would not
be far enough to the rear to result in the desired amount of
longitudinal stability. By building sweepback into the wings,
however, the designers can move the center of pressure toward
the rear. The amount of sweepback and the position of the
wings then place the center of pressure in the correct location.
When turbulence or rudder application causes the aircraft to
yaw to one side, the opposite wing presents a longer leading
edge perpendicular to the relative airflow. The airspeed of
the forward wing increases and it acquires more drag than
the back wing. The additional drag on the forward wing pulls
the wing back, turning the aircraft back to its original path.
The contribution of the wing to static directional stability is
usually small. The swept wing provides a stable contribution
depending on the amount of sweepback, but the contribution
is relatively small when compared with other components.
Free Directional Oscillations (Dutch Roll)
Dutch roll is a coupled lateral/directional oscillation that is
usually dynamically stable but is unsafe in an aircraft because
of the oscillatory nature. The damping of the oscillatory mode
may be weak or strong depending on the properties of the
particular aircraft.
If the aircraft has a right wing pushed down, the positive
sideslip angle corrects the wing laterally before the nose is
realigned with the relative wind. As the wing corrects the
position, a lateral directional oscillation can occur resulting in
the nose of the aircraft making a figure eight on the horizon as
a result of two oscillations (roll and yaw), which, although of
about the same magnitude, are out of phase with each other.
In most modern aircraft, except high-speed swept wing
designs, these free directional oscillations usually die out
automatically in very few cycles unless the air continues to
be gusty or turbulent. Those aircraft with continuing Dutch
roll tendencies are usually equipped with gyro-stabilized yaw
dampers. Manufacturers try to reach a midpoint between too
much and too little directional stability. Because it is more
desirable for the aircraft to have "spiral instability" than
Dutch roll tendencies, most aircraft are designed with that
characteristic.
Spiral Instability
Spiral instability exists when the static directional stability
of the aircraft is very strong as compared to the effect of its
dihedral in maintaining lateral equilibrium. When the lateral
equilibrium of the aircraft is disturbed by a gust of air and a
sideslip is introduced, the strong directional stability tends
to yaw the nose into the resultant relative wind while the
5-20

comparatively weak dihedral lags in restoring the lateral
balance. Due to this yaw, the wing on the outside of the
turning moment travels forward faster than the inside wing
and, as a consequence, its lift becomes greater. This produces
an overbanking tendency which, if not corrected by the pilot,
results in the bank angle becoming steeper and steeper. At
the same time, the strong directional stability that yaws the
aircraft into the relative wind is actually forcing the nose
to a lower pitch attitude. A slow downward spiral begins
which, if not counteracted by the pilot, gradually increases
into a steep spiral dive. Usually the rate of divergence in the
spiral motion is so gradual the pilot can control the tendency
without any difficulty.
Many aircraft are affected to some degree by this characteristic,
although they may be inherently stable in all other normal
parameters. This tendency explains why an aircraft cannot
be flown "hands off" indefinitely.
Much research has gone into the development of control
devices (wing leveler) to correct or eliminate this instability.
The pilot must be careful in application of recovery controls
during advanced stages of this spiral condition or excessive
loads may be imposed on the structure. Improper recovery
from spiral instability leading to inflight structural failures
has probably contributed to more fatalities in general aviation
aircraft than any other factor. Since the airspeed in the spiral
condition builds up rapidly, the application of back elevator
force to reduce this speed and to pull the nose up only
"tightens the turn," increasing the load factor. The results
of the prolonged uncontrolled spiral are inflight structural
failure, crashing into the ground, or both. Common recorded
causes for pilots who get into this situation are loss of horizon
reference, inability to control the aircraft by reference to
instruments, or a combination of both.

Effect of Wing Planform
Understanding the effects of different wing planforms
is important when learning about wing performance and
airplane flight characteristics. A planform is the shape of the
wing as viewed from directly above and deals with airflow
in three dimensions. Aspect ratio, taper ratio, and sweepback
are factors in planform design that are very important to the
overall aerodynamic characteristic of a wing. [Figure 5-33]
Aspect ratio is the ratio of wing span to wing chord. Taper
ratio can be either in planform or thickness, or both. In its
simplest terms, it is a decrease from wing root to wingtip in
wing chord or wing thickness. Sweepback is the rearward
slant of a wing, horizontal tail, or other airfoil surface.
There are two general means by which the designer can
change the planform of a wing and both will affect the

Elliptical wing

Regular wing

Moderate taper wing

High taper wing

Pointed tip wing

Sweepback wing

Figure 5-33. Different types of wing planforms.

aerodynamic characteristics of the wing. The first is to effect
a change in the aspect ratio. Aspect ratio is the primary factor
in determining the three dimensional characteristics of the
ordinary wing and its lift/drag ratio. An increase in aspect
ratio with constant velocity will decrease the drag, especially
at high angles of attack, improving the performance of the
wing when in a climbing attitude.

A decrease in aspect ratio will give a corresponding increase
in drag. It should be noted, however, that with an increase in
aspect ratio there is an increase in the length of span, with a
corresponding increase in the weight of the wing structure,
which means the wing must be heavier to carry the same load.
For this reason, part of the gain (due to a decrease in drag) is
lost because of the increased weight, and a compromise in
5-21

design is necessary to obtain the best results from these two
conflicting conditions.

In comparison, the rectangular wing has a tendency to stall
first at the wing root and provides adequate stall warning,
adequate aileron effectiveness, and is usually quite stable.
It is, therefore, favored in the design of low cost, low speed
airplanes.

The second means of changing the planform is by tapering
(decreasing the length of chord from the root to the tip of the
wing). In general, tapering causes a decrease in drag (most
effective at high speeds) and an increase in lift. There is also
a structural benefit due to a saving in weight of the wing.

Aerodynamic Forces in Flight Maneuvers
Forces in Turns
If an aircraft were viewed in straight-and-level flight from the
front [Figure 5-34], and if the forces acting on the aircraft
could be seen, lift and weight would be apparent: two forces.
If the aircraft were in a bank it would be apparent that lift
did not act directly opposite to the weight, rather it now acts
in the direction of the bank. A basic truth about turns is that
when the aircraft banks, lift acts inward toward the center of
the turn, perpendicular to the lateral axis as well as upward.

Most training and general aviation type airplanes are operated
at high coefficients of lift, and therefore require comparatively
high aspect ratios. Airplanes that are developed to operate at
very high speeds demand greater aerodynamic cleanness and
greater strength, which require low aspect ratios. Very low
aspect ratios result in high wing loadings and high stall speeds.
When sweepback is combined with low aspect ratio, it results
in flying qualities very different from a more conventional
high aspect ratio airplane configuration. Such airplanes
require very precise and professional flying techniques,
especially at slow speeds, while airplanes with a high aspect
ratio are usually more forgiving of improper pilot techniques.

Newton's First Law of Motion, the Law of Inertia, states that
an object at rest or moving in a straight line remains at rest
or continues to move in a straight line until acted on by some
other force. An aircraft, like any moving object, requires a
sideward force to make it turn. In a normal turn, this force
is supplied by banking the aircraft so that lift is exerted
inward, as well as upward. The force of lift during a turn is
separated into two components at right angles to each other.
One component, which acts vertically and opposite to the
weight (gravity), is called the "vertical component of lift."
The other, which acts horizontally toward the center of the
turn, is called the "horizontal component of lift" or centripetal
force. The horizontal component of lift is the force that
pulls the aircraft from a straight flight path to make it turn.
Centrifugal force is the "equal and opposite reaction" of the
aircraft to the change in direction and acts equal and opposite
to the horizontal component of lift. This explains why, in a
correctly executed turn, the force that turns the aircraft is
not supplied by the rudder. The rudder is used to correct any
deviation between the straight track of the nose and tail of the
aircraft into the relative wind. A good turn is one in which the

The elliptical wing is the ideal subsonic planform since it
provides for a minimum of induced drag for a given aspect
ratio, though as we shall see, its stall characteristics in
some respects are inferior to the rectangular wing. It is also
comparatively difficult to construct. The tapered airfoil is
desirable from the standpoint of weight and stiffness, but
again is not as efficient aerodynamically as the elliptical
wing. In order to preserve the aerodynamic efficiency of the
elliptical wing, rectangular and tapered wings are sometimes
tailored through use of wing twist and variation in airfoil
sections until they provide as nearly as possible the elliptical
wing's lift distribution. While it is true that the elliptical
wing provides the best coefficients of lift before reaching an
incipient stall, it gives little advance warning of a complete
stall, and lateral control may be difficult because of poor
aileron effectiveness.
Level flight

Medium banked turn
ta

Centrifugal
force

ift

Centrifugal
force

nt

ta

ul

es

R

Horizontal
component

ad

lo

Weight

ad

nt lo

ulta

Res

Weight

Weight

5-22

ll

Vertical
component

Vertical
component

l lift

Lift

Tota

To

Horizontal
component

Figure 5-34. Forces during normal, coordinated turn at constant altitude.

Steeply banked turn

nose and tail of the aircraft track along the same path. If no
rudder is used in a turn, the nose of the aircraft yaws to the
outside of the turn. The rudder is used rolling into the turn
to bring the nose back in line with the relative wind. Once
in the turn, the rudder should not be needed.

decreased, or the angle of bank increased, if a constant
altitude is to be maintained. If the angle of bank is held
constant and the AOA decreased, the ROT decreases. In order
to maintain a constant ROT as the airspeed is increased, the
AOA must remain constant and the angle of bank increased.

An aircraft is not steered like a boat or an automobile. In
order for an aircraft to turn, it must be banked. If it is not
banked, there is no force available to cause it to deviate
from a straight flight path. Conversely, when an aircraft is
banked, it turns provided it is not slipping to the inside of the
turn. Good directional control is based on the fact that the
aircraft attempts to turn whenever it is banked. Pilots should
keep this fact in mind when attempting to hold the aircraft
in straight-and-level flight.

An increase in airspeed results in an increase of the turn radius,
and centrifugal force is directly proportional to the radius of
the turn. In a correctly executed turn, the horizontal component
of lift must be exactly equal and opposite to the centrifugal
force. As the airspeed is increased in a constant-rate level turn,
the radius of the turn increases. This increase in the radius of
turn causes an increase in the centrifugal force, which must
be balanced by an increase in the horizontal component of lift,
which can only be increased by increasing the angle of bank.

Merely banking the aircraft into a turn produces no change in
the total amount of lift developed. Since the lift during the bank
is divided into vertical and horizontal components, the amount
of lift opposing gravity and supporting the aircraft's weight
is reduced. Consequently, the aircraft loses altitude unless
additional lift is created. This is done by increasing the AOA
until the vertical component of lift is again equal to the weight.
Since the vertical component of lift decreases as the bank
angle increases, the AOA must be progressively increased
to produce sufficient vertical lift to support the aircraft's
weight. An important fact for pilots to remember when making
constant altitude turns is that the vertical component of lift
must be equal to the weight to maintain altitude.

In a slipping turn, the aircraft is not turning at the rate
appropriate to the bank being used, since the aircraft is yawed
toward the outside of the turning flight path. The aircraft is
banked too much for the ROT, so the horizontal lift component
is greater than the centrifugal force. [Figure 5-35] Equilibrium
between the horizontal lift component and centrifugal force
is reestablished by either decreasing the bank, increasing the
ROT, or a combination of the two changes.

At a given airspeed, the rate at which an aircraft turns
depends upon the magnitude of the horizontal component
of lift. It is found that the horizontal component of lift is
proportional to the angle of bank—that is, it increases or
decreases respectively as the angle of bank increases or
decreases. As the angle of bank is increased, the horizontal
component of lift increases, thereby increasing the rate of
turn (ROT). Consequently, at any given airspeed, the ROT
can be controlled by adjusting the angle of bank.
To provide a vertical component of lift sufficient to hold
altitude in a level turn, an increase in the AOA is required.
Since the drag of the airfoil is directly proportional to its AOA,
induced drag increases as the lift is increased. This, in turn,
causes a loss of airspeed in proportion to the angle of bank.
A small angle of bank results in a small reduction in airspeed
while a large angle of bank results in a large reduction in
airspeed. Additional thrust (power) must be applied to prevent
a reduction in airspeed in level turns. The required amount of
additional thrust is proportional to the angle of bank.
To compensate for added lift, which would result if the
airspeed were increased during a turn, the AOA must be

A skidding turn results from an excess of centrifugal force
over the horizontal lift component, pulling the aircraft
toward the outside of the turn. The ROT is too great for the
angle of bank. Correction of a skidding turn thus involves a
reduction in the ROT, an increase in bank, or a combination
of the two changes.
To maintain a given ROT, the angle of bank must be varied
with the airspeed. This becomes particularly important in
high-speed aircraft. For instance, at 400 miles per hour (mph),
an aircraft must be banked approximately 44° to execute a
standard-rate turn (3° per second). At this angle of bank,
only about 79 percent of the lift of the aircraft comprises the
vertical component of the lift. This causes a loss of altitude
unless the AOA is increased sufficiently to compensate for
the loss of vertical lift.
Forces in Climbs
For all practical purposes, the wing's lift in a steady state
normal climb is the same as it is in a steady level flight at the
same airspeed. Although the aircraft's flight path changed
when the climb was established, the AOA of the wing with
respect to the inclined flight path reverts to practically the
same values, as does the lift. There is an initial momentary
change as shown in Figure 5-36. During the transition from
straight-and-level flight to a climb, a change in lift occurs
when back elevator pressure is first applied. Raising the
aircraft's nose increases the AOA and momentarily increases
5-23

Normal turn

Slipping turn

Vertical lift
Centrifugal force

Centrifugal
force

Centrifugal force
greater than
horizontal lift

Weight

Weight

ad
Lo

Centrifugal
force less than
horizontal lift

Horizontal
lift

d
Loa

Horizontal
lift
Weight

Centrifugal
force equals
horizontal lift

t
Lif

Vertical lift

t
Lif

Vertical lift

t
Lif

Centrifugal
force
Horizontal
lift

Skidding turn

Lo

ad

Figure 5-35. Normal, slipping, and skidding turns at a constant altitude.

If the climb is entered with no change in power setting, the
airspeed gradually diminishes because the thrust required
to maintain a given airspeed in level flight is insufficient to
maintain the same airspeed in a climb. When the flight path
is inclined upward, a component of the aircraft's weight
acts in the same direction as, and parallel to, the total drag
of the aircraft, thereby increasing the total effective drag.
Consequently, the total effective drag is greater than the
power, and the airspeed decreases. The reduction in airspeed
gradually results in a corresponding decrease in drag until
the total drag (including the component of weight acting
in the same direction) equals the thrust. [Figure 5-37] Due
to momentum, the change in airspeed is gradual, varying
considerably with differences in aircraft size, weight, total
drag, and other factors. Consequently, the total effective drag
is greater than the thrust, and the airspeed decreases.
Generally, the forces of thrust and drag, and lift and weight,
again become balanced when the airspeed stabilizes but at

a value lower than in straight-and-level flight at the same
power setting. Since the aircraft's weight is acting not only
downward but rearward with drag while in a climb, additional
power is required to maintain the same airspeed as in level
flight. The amount of power depends on the angle of climb.
When the climb is established steep enough that there is
insufficient power available, a slower speed results.
The thrust required for a stabilized climb equals drag plus a
percentage of weight dependent on the angle of climb. For
example, a 10° climb would require thrust to equal drag plus
17 percent of weight. To climb straight up would require
thrust to equal all of weight and drag. Therefore, the angle
of climb for climb performance is dependent on the amount
of excess thrust available to overcome a portion of weight.
Note that aircraft are able to sustain a climb due to excess
thrust. When the excess thrust is gone, the aircraft is no
longer able to climb. At this point, the aircraft has reached
its "absolute ceiling."
Forces in Descents
As in climbs, the forces that act on the aircraft go through
definite changes when a descent is entered from straight­
and-level flight. For the following example, the aircraft
Level flight
forces balanced
constant speed
D

L

L

W

L

T

L

the lift. Lift at this moment is now greater than weight and
starts the aircraft climbing. After the flight path is stabilized
on the upward incline, the AOA and lift again revert to about
the level flight values.

D

D

T
W

L

W

L

L

T

Steady climb
normal lift

Climb entry
increased lift

Figure 5-36. Changes in lift during climb entry.

5-24

Level flight
normal lift

Steady climb
forces balanced
constant speed

Climb entry drag
greater than thrust
speed slowing

Figure 5-37. Changes in speed during climb entry.

is descending at the same power as used in straight-and­
level flight.
As forward pressure is applied to the control yoke to initiate
the descent, the AOA is decreased momentarily. Initially,
the momentum of the aircraft causes the aircraft to briefly
continue along the same flight path. For this instant, the AOA
decreases causing the total lift to decrease. With weight now
being greater than lift, the aircraft begins to descend. At the
same time, the flight path goes from level to a descending
flight path. Do not confuse a reduction in lift with the inability
to generate sufficient lift to maintain level flight. The flight
path is being manipulated with available thrust in reserve
and with the elevator.
To descend at the same airspeed as used in straight-and­
level flight, the power must be reduced as the descent is
entered. Entering the descent, the component of weight
acting forward along the flight path increases as the angle
of descent increases and, conversely, when leveling off, the
component of weight acting along the flight path decreases
as the angle of descent decreases.

Stalls
An aircraft stall results from a rapid decrease in lift caused by
the separation of airflow from the wing's surface brought on
by exceeding the critical AOA. A stall can occur at any pitch
attitude or airspeed. Stalls are one of the most misunderstood
areas of aerodynamics because pilots often believe an airfoil
stops producing lift when it stalls. In a stall, the wing does
not totally stop producing lift. Rather, it cannot generate
adequate lift to sustain level flight.
Since the CL increases with an increase in AOA, at some
point the CL peaks and then begins to drop off. This peak is
called the CL-MAX. The amount of lift the wing produces drops
dramatically after exceeding the CL-MAX or critical AOA, but
as stated above, it does not completely stop producing lift.
In most straight-wing aircraft, the wing is designed to stall
the wing root first. The wing root reaches its critical AOA
first making the stall progress outward toward the wingtip.
By having the wing root stall first, aileron effectiveness is
maintained at the wingtips, maintaining controllability of
the aircraft. Various design methods are used to achieve
the stalling of the wing root first. In one design, the wing is
"twisted" to a higher AOA at the wing root. Installing stall
strips on the first 20–25 percent of the wing's leading edge
is another method to introduce a stall prematurely.
The wing never completely stops producing lift in a stalled
condition. If it did, the aircraft would fall to the Earth. Most
training aircraft are designed for the nose of the aircraft to

drop during a stall, reducing the AOA and "unstalling" the
wing. The nose-down tendency is due to the CL being aft of
the CG. The CG range is very important when it comes to
stall recovery characteristics. If an aircraft is allowed to be
operated outside of the CG range, the pilot may have difficulty
recovering from a stall. The most critical CG violation would
occur when operating with a CG that exceeds the rear limit.
In this situation, a pilot may not be able to generate sufficient
force with the elevator to counteract the excess weight aft of
the CG. Without the ability to decrease the AOA, the aircraft
continues in a stalled condition until it contacts the ground.
The stalling speed of a particular aircraft is not a fixed value
for all flight situations, but a given aircraft always stalls at
the same AOA regardless of airspeed, weight, load factor, or
density altitude. Each aircraft has a particular AOA where the
airflow separates from the upper surface of the wing and the
stall occurs. This critical AOA varies from approximately 16°
to 20° depending on the aircraft's design. But each aircraft
has only one specific AOA where the stall occurs.
There are three flight situations in which the critical AOA is
most frequently exceeded: low speed, high speed, and turning.
One way the aircraft can be stalled in straight-and-level flight
by flying too slowly. As the airspeed decreases, the AOA
must be increased to retain the lift required for maintaining
altitude. The lower the airspeed becomes, the more the AOA
must be increased. Eventually, an AOA is reached that results
in the wing not producing enough lift to support the aircraft,
which then starts settling. If the airspeed is reduced further,
the aircraft stalls because the AOA has exceeded the critical
angle and the airflow over the wing is disrupted.
Low speed is not necessary to produce a stall. The wing
can be brought into an excessive AOA at any speed. For
example, an aircraft is in a dive with an airspeed of 100
knots when the pilot pulls back sharply on the elevator
control. [Figure 5-38] Gravity and centrifugal force prevent
an immediate alteration of the flight path, but the aircraft's
AOA changes abruptly from quite low to very high. Since
the flight path of the aircraft in relation to the oncoming air
determines the direction of the relative wind, the AOA is
suddenly increased, and the aircraft would reach the stalling
angle at a speed much greater than the normal stall speed.
The stalling speed of an aircraft is also higher in a level turn
than in straight-and-level flight. [Figure 5-39] Centrifugal
force is added to the aircraft's weight and the wing must
produce sufficient additional lift to counterbalance the load
imposed by the combination of centrifugal force and weight.
In a turn, the necessary additional lift is acquired by applying
back pressure to the elevator control. This increases the wing's
5-25

Airfoil shape and degradation of that shape must also be
considered in a discussion of stalls. For example, if ice, snow,
and frost are allowed to accumulate on the surface of an aircraft,
the smooth airflow over the wing is disrupted. This causes the
boundary layer to separate at an AOA lower than that of the
critical angle. Lift is greatly reduced, altering expected aircraft
performance. If ice is allowed to accumulate on the aircraft
during flight, the weight of the aircraft is increased while the
ability to generate lift is decreased. [Figure 5-40] As little as
0.8 millimeter of ice on the upper wing surface increases drag
and reduces aircraft lift by 25 percent.

L

CF

W

L

L

W

L

CF

W

W

CF

CF

Pilots can encounter icing in any season, anywhere in the
country, at altitudes of up to 18,000 feet and sometimes
higher. Small aircraft, including commuter planes, are most
vulnerable because they fly at lower altitudes where ice is more
prevalent. They also lack mechanisms common on jet aircraft
that prevent ice buildup by heating the front edges of wings.

Figure 5-38. Forces exerted when pulling out of a dive.

AOA and results in increased lift. The AOA must increase
as the bank angle increases to counteract the increasing load
caused by centrifugal force. If at any time during a turn the
AOA becomes excessive, the aircraft stalls.

20
0
0°

r

Figure 5-39. Increase in stall speed and load factor.

5-26

Angle of Attack Indicators
The FAA along with the General Aviation Joint Steering
Committee (GAJSC) is promoting AOA indicators as one
of the many safety initiatives aimed at reducing the general
aviation accident rate. AOA indicators will specifically target
Loss of Control (LOC) accidents. Loss of control is the
number one root cause of fatalities in both general aviation
and commercial aviation. More than 25 percent of general
aviation fatal accidents occur during the maneuvering phase
of flight. Of those accidents, half involve stall/spin scenarios.
Technology such as AOA indicators can have a tremendous
impact on reversing this trend and are increasingly affordable
for general aviation airplanes. [Figure 5-41]
The purpose of an AOA indicator is to give the pilot better
situation awareness pertaining to the aerodynamic health

Load factor or "G"

40

d facto

60

L oa

80

nc
rea
se

90

13
12
11
10
9
8
7
6
s
5
l
l
a
4
St
3
2
1
10° 20° 30° 40° 50° 60° 70° 80° 90°
Bank Angle

ed
i

100

pe

Percent increase in stall speed

At this point, the action of the aircraft during a stall should be
examined. To balance the aircraft aerodynamically, the CL
is normally located aft of the CG. Although this makes the
aircraft inherently nose-heavy, downwash on the horizontal
stabilizer counteracts this condition. At the point of stall,
when the upward force of the wing's lift diminishes below
that required for sustained flight and the downward tail
force decreases to a point of ineffectiveness, or causes it to
have an upward force, an unbalanced condition exists. This
causes the aircraft to pitch down abruptly, rotating about its
CG. During this nose-down attitude, the AOA decreases and
the airspeed again increases. The smooth flow of air over
the wing begins again, lift returns, and the aircraft begins
to fly again. Considerable altitude may be lost before this
cycle is complete.

Icing can occur in clouds any time the temperature drops
below freezing and super-cooled droplets build up on an
aircraft and freeze. (Super-cooled droplets are still liquid
even though the temperature is below 32 °Fahrenheit (F),
or 0 °Celsius (C).

Figure 5-40. Inflight ice formation.

of the airfoil. This can also be referred to as stall margin
awareness. More simply explained, it is the margin that exists
between the current AOA that the airfoil is operating at, and
the AOA at which the airfoil will stall (critical AOA).
Angle of attack is taught to student pilots as theory in ground
training. When beginning flight training, students typically
rely solely on airspeed and the published 1G stall speed to
avoid stalls. This creates problems since this speed is only
valid when the following conditions are met:


Unaccelerated flight (a 1G load factor)



Coordinated flight (inclinometer centered)



At one weight (typically maximum gross weight)

Speed by itself is not a reliable parameter to avoid a stall.
An airplane can stall at any speed. Angle of attack is a better
parameter to use to avoid a stall. For a given configuration,
the airplane always stalls at the same AOA, referred to as
the critical AOA. This critical AOA does not change with:


Weight



Bank angle



Temperature



Density altitude



Center of gravity

An AOA indicator can have several benefits when installed
in general aviation aircraft, not the least of which is increased
situational awareness. Without an AOA indicator, the AOA
is "invisible" to pilots. These devices measure several
parameters simultaneously and determine the current AOA
providing a visual image to the pilot of the current AOA along
with representations of the proximity to the critical AOA.
[Figure 5-42] These devices can give a visual representation
of the energy management state of the airplane. The energy

state of an airplane is the balance between airspeed, altitude,
drag, and thrust and represents how efficiently the airfoil is
operating. The more efficiently the airfoil operates; the larger
stall margin that is present. With this increased situational
awareness pertaining to the energy condition of the airplane,
pilots will have information that they need to aid in preventing
a LOC scenario resulting from a stall/spin. Additionally, the
less energy that is utilized to maintain flight means greater
overall efficiency of the airplane, which is typically realized in
fuel savings. This equates to a lower operating cost to the pilot.
Just as training is required for any system on an aircraft,
AOA indicators have training considerations also. A more
comprehensive understanding of AOA in general should be
the goal of this training along with the specific operating
characteristics and limitations of the installed AOA indicator.
Ground and flight instructors should make every attempt
to receive training from an instructor knowledgeable about
AOA indicators prior to giving instruction pertaining to or in
airplanes equipped with AOA indicators. Pilot schools should
incorporate training on AOA indicators in their syllabi,
whether their training aircraft are equipped with them or not.
Installation of AOA indicators not required by type
certification in general aviation airplanes has recently been
streamlined by the FAA. The FAA established policy in
February 2014 pertaining to non-required AOA systems and
how they may be installed as a minor alteration, depending
upon their installation requirements and operational
utilization, and the procedures to take for certification of
these installations. For updated information, reference the
FAA website at \url{faa.gov}.
While AOA indicators provide a simple visual representation
of the current AOA and its proximity to the critical AOA, they
are not without their limitations. These limitations should
be understood by operators of general aviation airplanes

Figure 5-41. A variety of AOA indicators.

5-27

Coefficient of Lift Curve
Direction of
Relative Wind

1

2
Direction of
Relative Wind

Direction of
Relative Wind

3
CL

Direction of
Relative Wind

Calibration Set Points:
1 Ground
2 Optimum Alpha Angle (OAA)
3 Cruise Alpha

(Zero Set Point)
(1.3 x Vs)

(Va or Maneuvering Speed)

Alpha (angle of attack)
Figure 5-42. An AOA indicator has several benefits when installed in general aviation aircraft.

equipped with these devices. Like advanced automation,
such as autopilots and moving maps, the misunderstanding
or misuse of the equipment can have disastrous results. Some
items which may limit the effectiveness of an AOA indicator
are listed below:


Calibration techniques



Probes or vanes not being heated



The type of indicator itself



Flap setting



Wing contamination

Pilots of general aviation airplanes equipped with AOA
indicators should contact the manufacturer for specific
limitations applicable to that installation.

Basic Propeller Principles
The aircraft propeller consists of two or more blades and a
central hub to which the blades are attached. Each blade of
an aircraft propeller is essentially a rotating wing. As a result
of their construction, the propeller blades are like airfoils
5-28

and produce forces that create the thrust to pull, or push,
the aircraft through the air. The engine furnishes the power
needed to rotate the propeller blades through the air at high
speeds, and the propeller transforms the rotary power of the
engine into forward thrust.
A cross-section of a typical propeller blade is shown in
Figure 5-43. This section or blade element is an airfoil
comparable to a cross-section of an aircraft wing. One
surface of the blade is cambered or curved, similar to the
upper surface of an aircraft wing, while the other surface is
flat like the bottom surface of a wing. The chord line is an
imaginary line drawn through the blade from its leading edge
to its trailing edge. As in a wing, the leading edge is the thick
edge of the blade that meets the air as the propeller rotates.
Blade angle, usually measured in degrees, is the angle
between the chord of the blade and the plane of rotation
and is measured at a specific point along the length of the
blade. [Figure 5-44] Because most propellers have a flat
blade "face," the chord line is often drawn along the face
of the propeller blade. Pitch is not blade angle, but because
pitch is largely determined by blade angle, the two terms are

the vectors of propeller forces in Figure 5-44, each section of
a propeller blade moves downward and forward. The angle
at which this air (relative wind) strikes the propeller blade
is its AOA. The air deflection produced by this angle causes
the dynamic pressure at the engine side of the propeller blade
to be greater than atmospheric pressure, thus creating thrust.

Figure 5-43. Airfoil sections of propeller blade.

often used interchangeably. An increase or decrease in one is
usually associated with an increase or decrease in the other.
The pitch of a propeller may be designated in inches. A
propeller designated as a "74–48" would be 74 inches in
length and have an effective pitch of 48 inches. The pitch
is the distance in inches, which the propeller would screw
through the air in one revolution if there were no slippage.
When specifying a fixed-pitch propeller for a new type of
aircraft, the manufacturer usually selects one with a pitch
that operates efficiently at the expected cruising speed of the
aircraft. Every fixed-pitch propeller must be a compromise
because it can be efficient at only a given combination of
airspeed and revolutions per minute (rpm). Pilots cannot
change this combination in flight.
When the aircraft is at rest on the ground with the engine
operating, or moving slowly at the beginning of takeoff,
the propeller efficiency is very low because the propeller is
restrained from advancing with sufficient speed to permit
its fixed-pitch blades to reach their full efficiency. In this
situation, each propeller blade is turning through the air at
an AOA that produces relatively little thrust for the amount
of power required to turn it.
To understand the action of a propeller, consider first its
motion, which is both rotational and forward. As shown by

Rotational velocity

Thrust

or e
Pitch angl
blade

gle
An of k
ac
att

Forward velocity

Chord line

Relative wind

Figure 5-44. Propeller blade angle.

The shape of the blade also creates thrust because it is
cambered like the airfoil shape of a wing. As the air flows
past the propeller, the pressure on one side is less than that
on the other. As in a wing, a reaction force is produced in the
direction of the lesser pressure. The airflow over the wing
has less pressure, and the force (lift) is upward. In the case
of the propeller, which is mounted in a vertical instead of a
horizontal plane, the area of decreased pressure is in front of
the propeller, and the force (thrust) is in a forward direction.
Aerodynamically, thrust is the result of the propeller shape
and the AOA of the blade.
Thrust can be considered also in terms of the mass of air
handled by the propeller. In these terms, thrust equals mass
of air handled multiplied by slipstream velocity minus
velocity of the aircraft. The power expended in producing
thrust depends on the rate of air mass movement. On average,
thrust constitutes approximately 80 percent of the torque (total
horsepower absorbed by the propeller). The other 20 percent
is lost in friction and slippage. For any speed of rotation,
the horsepower absorbed by the propeller balances the
horsepower delivered by the engine. For any single revolution
of the propeller, the amount of air handled depends on the
blade angle, which determines how big a "bite" of air the
propeller takes. Thus, the blade angle is an excellent means of
adjusting the load on the propeller to control the engine rpm.
The blade angle is also an excellent method of adjusting the
AOA of the propeller. On constant-speed propellers, the blade
angle must be adjusted to provide the most efficient AOA at
all engine and aircraft speeds. Lift versus drag curves, which
are drawn for propellers as well as wings, indicate that the
most efficient AOA is small, varying from +2° to +4°. The
actual blade angle necessary to maintain this small AOA
varies with the forward speed of the aircraft.
Fixed-pitch and ground-adjustable propellers are designed
for best efficiency at one rotation and forward speed. They
are designed for a given aircraft and engine combination. A
propeller may be used that provides the maximum efficiency
for takeoff, climb, cruise, or high-speed flight. Any change in
these conditions results in lowering the efficiency of both the
propeller and the engine. Since the efficiency of any machine
is the ratio of the useful power output to the actual power
input, propeller efficiency is the ratio of thrust horsepower
to brake horsepower. Propeller efficiency varies from 50 to
5-29

Effective pitch
Geometric pitch
Figure 5-45. Propeller slippage.

5-30

Shor
t tr
a

M

ts
no

Slip

20 in.
2,50 rpm
0

ots
9 kn
25
—
ed

A constant-speed propeller automatically keeps the blade
angle adjusted for maximum efficiency for most conditions
encountered in flight. During takeoff, when maximum power
and thrust are required, the constant-speed propeller is at a
low propeller blade angle or pitch. The low blade angle keeps
the AOA small and efficient with respect to the relative wind.
At the same time, it allows the propeller to handle a smaller
mass of air per revolution. This light load allows the engine to
turn at high rpm and to convert the maximum amount of fuel
into heat energy in a given time. The high rpm also creates
maximum thrust because, although the mass of air handled
per revolution is small, the rpm and slipstream velocity are

stance—slo
l di
w
ve

sp
ee
d—

—12
9k

Usually 1° to 4° provides the most efficient lift/drag ratio,
but in flight the propeller AOA of a fixed-pitch propeller
varies—normally from 0° to 15°. This variation is caused
by changes in the relative airstream, which in turn results
from changes in aircraft speed. Thus, propeller AOA is the
product of two motions: propeller rotation about its axis and
its forward motion.

l distance—mod
rave
era
te t
te
a
r
sp
e
d
e
o

eed
sp

The reason a propeller is "twisted" is that the outer parts of the
propeller blades, like all things that turn about a central point,
travel faster than the portions near the hub. [Figure 5-46] If the
blades had the same geometric pitch throughout their lengths,
portions near the hub could have negative AOAs while the
propeller tips would be stalled at cruise speed. Twisting or
variations in the geometric pitch of the blades permits the
propeller to operate with a relatively constant AOA along its
length when in cruising flight. Propeller blades are twisted
to change the blade angle in proportion to the differences in
speed of rotation along the length of the propeller, keeping
thrust more nearly equalized along this length.

avel distance—ve
ater tr
ry h
Gre
igh

ts
no
9k
38

87 percent, depending on how much the propeller "slips."
Propeller slip is the difference between the geometric pitch of
the propeller and its effective pitch. [Figure 5-45] Geometric
pitch is the theoretical distance a propeller should advance
in one revolution; effective pitch is the distance it actually
advances. Thus, geometric or theoretical pitch is based on
no slippage, but actual or effective pitch includes propeller
slippage in the air.

40 in.

2,500 rpm
60 in.

2,500 rpm

Figure 5-46. Propeller tips travel faster than the hub.

high, and with the low aircraft speed, there is maximum thrust.
After liftoff, as the speed of the aircraft increases, the constantspeed propeller automatically changes to a higher angle (or
pitch). Again, the higher blade angle keeps the AOA small
and efficient with respect to the relative wind. The higher
blade angle increases the mass of air handled per revolution.
This decreases the engine rpm, reducing fuel consumption
and engine wear, and keeps thrust at a maximum.
After the takeoff climb is established in an aircraft having a
controllable-pitch propeller, the pilot reduces the power output
of the engine to climb power by first decreasing the manifold
pressure and then increasing the blade angle to lower the rpm.
At cruising altitude, when the aircraft is in level flight and
less power is required than is used in takeoff or climb, the
pilot again reduces engine power by reducing the manifold
pressure and then increasing the blade angle to decrease the
rpm. Again, this provides a torque requirement to match the
reduced engine power. Although the mass of air handled per
revolution is greater, it is more than offset by a decrease in
slipstream velocity and an increase in airspeed. The AOA is
still small because the blade angle has been increased with
an increase in airspeed.
Torque and P-Factor
To the pilot, "torque" (the left turning tendency of the
airplane) is made up of four elements that cause or produce
a twisting or rotating motion around at least one of the
airplane's three axes. These four elements are:
1.

Torque reaction from engine and propeller

2.

Corkscrewing effect of the slipstream

3.

Gyroscopic action of the propeller

4.

Asymmetric loading of the propeller (P-factor)

Torque Reaction
Torque reaction involves Newton's Third Law of Physics—
for every action, there is an equal and opposite reaction. As
applied to the aircraft, this means that as the internal engine
parts and propeller are revolving in one direction, an equal
force is trying to rotate the aircraft in the opposite direction.
[Figure 5-47]
When the aircraft is airborne, this force is acting around
the longitudinal axis, tending to make the aircraft roll. To
compensate for roll tendency, some of the older aircraft are
rigged in a manner to create more lift on the wing that is being
forced downward. The more modern aircraft are designed
with the engine offset to counteract this effect of torque.
NOTE: Most United States built aircraft engines rotate the
propeller clockwise, as viewed from the pilot's seat. The
discussion here is with reference to those engines.
Generally, the compensating factors are permanently set so
that they compensate for this force at cruising speed, since
most of the aircraft's operating time is at that speed. However,
aileron trim tabs permit further adjustment for other speeds.
When the aircraft's wheels are on the ground during the
takeoff roll, an additional turning moment around the vertical
axis is induced by torque reaction. As the left side of the
aircraft is being forced down by torque reaction, more weight
is being placed on the left main landing gear. This results in
more ground friction, or drag, on the left tire than on the right,
causing a further turning moment to the left. The magnitude
of this moment is dependent on many variables. Some of
these variables are:
1.

Size and horsepower of engine

2.

Size of propeller and the rpm

3.

Size of the aircraft

4.

Condition of the ground surface

Reac
tio
n

This yawing moment on the takeoff roll is corrected by the
pilot's proper use of the rudder or rudder trim.
Corkscrew Effect
The high-speed rotation of an aircraft propeller gives a
corkscrew or spiraling rotation to the slipstream. At high
propeller speeds and low forward speed (as in the takeoffs
and approaches to power-on stalls), this spiraling rotation
is very compact and exerts a strong sideward force on the
aircraft's vertical tail surface. [Figure 5-48]
When this spiraling slipstream strikes the vertical fin, it
causes a yawing moment about the aircraft's vertical axis.
The more compact the spiral, the more prominent this force
is. As the forward speed increases, however, the spiral
elongates and becomes less effective. The corkscrew
flow of the slipstream also causes a rolling moment
around the longitudinal axis.
Note that this rolling moment caused by the corkscrew flow
of the slipstream is to the right, while the yawing moment
caused by torque reaction is to the left—in effect one may
be counteracting the other. However, these forces vary
greatly and it is the pilot's responsibility to apply proper
corrective action by use of the flight controls at all times.
These forces must be counteracted regardless of which is
the most prominent at the time.
Gyroscopic Action
Before the gyroscopic effects of the propeller can be
understood, it is necessary to understand the basic principle
of a gyroscope. All practical applications of the gyroscope
are based upon two fundamental properties of gyroscopic
action: rigidity in space and precession. The one of interest
for this discussion is precession.
Precession is the resultant action, or deflection, of a spinning
rotor when a deflecting force is applied to its rim. As can be
seen in Figure 5-49, when a force is applied, the resulting
force takes effect 90° ahead of and in the direction of rotation.
The rotating propeller of an airplane makes a very good

Yaw

m
trea
Slips

F orc

A ct

Figure 5-47. Torque reaction.

io

e

n

Figure 5-48. Corkscrewing slipstream.

5-31

Effective
force

Resultant force 90°
Yaw

1. Intake
Applied force

Figure 5-49. Gyroscopic precession.

gyroscope and thus has similar properties. Any time a force
is applied to deflect the propeller out of its plane of rotation,
the resulting force is 90° ahead of and in the direction of
rotation and in the direction of application, causing a pitching
moment, a yawing moment, or a combination of the two
depending upon the point at which the force was applied.
This element of torque effect has always been associated with
and considered more prominent in tailwheel-type aircraft
and most often occurs when the tail is being raised during
the takeoff roll. [Figure 5-50] This change in pitch attitude
has the same effect as applying a force to the top of the
propeller's plane of rotation. The resultant force acting 90°
ahead causes a yawing moment to the left around the vertical
axis. The magnitude of this moment depends on several
variables, one of which is the abruptness with which the tail
is raised (amount of force applied). However, precession,
or gyroscopic action, occurs when a force is applied to any
point on the rim of the propeller's plane of rotation; the
resultant force will still be 90° from the point of application
in the direction of rotation. Depending on where the force is
applied, the airplane is caused to yaw left or right, to pitch
up or down, or a combination of pitching and yawing.
It can be said that, as a result of gyroscopic action, any yawing
around the vertical axis results in a pitching moment, and any
pitching around the lateral axis results in a yawing moment.
To correct for the effect of gyroscopic action, it is necessary
for the pilot to properly use elevator and rudder to prevent
undesired pitching and yawing.

Yaw

t

tan
sul
Re rce
fo

Applied force

ctive
Effe ce
for

Figure 5-50. Raising tail produces gyroscopic precession.

5-32

Asymmetric Loading (P-Factor)
When an aircraft is flying with a high AOA, the "bite" of
the downward moving blade is greater than the "bite" of the
upward moving blade. This moves the center of thrust to the
right of the prop disc area, causing a yawing moment toward
the left around the vertical axis. Proving this explanation is
complex because it would be necessary to work wind vector
problems on each blade while considering both the AOA of
the aircraft and the AOA of each blade.
This asymmetric loading is caused by the resultant velocity,
which is generated by the combination of the velocity of the
propeller blade in its plane of rotation and the velocity of the
air passing horizontally through the propeller disc. With the
aircraft being flown at positive AOAs, the right (viewed from
the rear) or downswinging blade, is passing through an area of
resultant velocity, which is greater than that affecting the left
or upswinging blade. Since the propeller blade is an airfoil,
increased velocity means increased lift. The downswinging
blade has more lift and tends to pull (yaw) the aircraft's nose
to the left.
When the aircraft is flying at a high AOA, the downward
moving blade has a higher resultant velocity, creating more
lift than the upward moving blade. [Figure 5-51] This might
be easier to visualize if the propeller shaft was mounted
perpendicular to the ground (like a helicopter). If there
were no air movement at all, except that generated by the
propeller itself, identical sections of each blade would have
the same airspeed. With air moving horizontally across this
vertically mounted propeller, the blade proceeding forward
into the flow of air has a higher airspeed than the blade
retreating with the airflow. Thus, the blade proceeding into
the horizontal airflow is creating more lift, or thrust, moving
the center of thrust toward that blade. Visualize rotating the
vertically mounted propeller shaft to shallower angles relative
to the moving air (as on an aircraft). This unbalanced thrust
then becomes proportionately smaller and continues getting
smaller until it reaches the value of zero when the propeller
shaft is exactly horizontal in relation to the moving air.
The effects of each of these four elements of torque vary
in value with changes in flight situations. In one phase of
flight, one of these elements may be more prominent than
another. In another phase of flight, another element may be
more prominent. The relationship of these values to each
other varies with different aircraft depending on the airframe,
engine, and propeller combinations, as well as other design
features. To maintain positive control of the aircraft in all
flight conditions, the pilot must apply the flight controls as
necessary to compensate for these varying values.

Load on
upward moving
propeller blade

Load on
upward moving
propeller blade

Load on
downward moving
propeller blade

Load on
downward moving
propeller blade

Low angle of attack

High angle of attack

Figure 5-51. Asymmetrical loading of propeller (P-factor).

Load Factors
In aerodynamics, the maximum load factor (at given bank
angle) is a proportion between lift and weight and has a
trigonometric relationship. The load factor is measured in
Gs (acceleration of gravity), a unit of force equal to the force
exerted by gravity on a body at rest and indicates the force to
which a body is subjected when it is accelerated. Any force
applied to an aircraft to deflect its flight from a straight line
produces a stress on its structure. The amount of this force
is the load factor. While a course in aerodynamics is not a
prerequisite for obtaining a pilot's license, the competent
pilot should have a solid understanding of the forces that act
on the aircraft, the advantageous use of these forces, and the
operating limitations of the aircraft being flown.
For example, a load factor of 3 means the total load on an
aircraft's structure is three times its weight. Since load factors
are expressed in terms of Gs, a load factor of 3 may be spoken
of as 3 Gs, or a load factor of 4 as 4 Gs.
If an aircraft is pulled up from a dive, subjecting the pilot to
3 Gs, he or she would be pressed down into the seat with a
force equal to three times his or her weight. Since modern
aircraft operate at significantly higher speeds than older
aircraft, increasing the potential for large load factors, this
effect has become a primary consideration in the design of
the structure of all aircraft.
With the structural design of aircraft planned to withstand
only a certain amount of overload, a knowledge of load
factors has become essential for all pilots. Load factors are
important for two reasons:
1.	 It is possible for a pilot to impose a dangerous overload
on the aircraft structures.
2.	 An increased load factor increases the stalling speed and
makes stalls possible at seemingly safe flight speeds.

Load Factors in Aircraft Design
The answer to the question "How strong should an aircraft
be?" is determined largely by the use to which the aircraft is
subjected. This is a difficult problem because the maximum
possible loads are much too high for use in efficient design.
It is true that any pilot can make a very hard landing or an
extremely sharp pull up from a dive, which would result in
abnormal loads. However, such extremely abnormal loads
must be dismissed somewhat if aircraft are built that take off
quickly, land slowly, and carry worthwhile payloads.
The problem of load factors in aircraft design becomes how
to determine the highest load factors that can be expected in
normal operation under various operational situations. These
load factors are called "limit load factors." For reasons of
safety, it is required that the aircraft be designed to withstand
these load factors without any structural damage. Although
the Code of Federal Regulations (CFR) requires the aircraft
structure be capable of supporting one and one-half times
these limit load factors without failure, it is accepted that
parts of the aircraft may bend or twist under these loads and
that some structural damage may occur.
This 1.5 load limit factor is called the "factor of safety" and
provides, to some extent, for loads higher than those expected
under normal and reasonable operation. This strength reserve
is not something that pilots should willfully abuse; rather, it is
there for protection when encountering unexpected conditions.
The above considerations apply to all loading conditions,
whether they be due to gusts, maneuvers, or landings. The
gust load factor requirements now in effect are substantially
the same as those that have been in existence for years.
Hundreds of thousands of operational hours have proven
them adequate for safety. Since the pilot has little control over
gust load factors (except to reduce the aircraft's speed when
rough air is encountered), the gust loading requirements are
substantially the same for most general aviation type aircraft
regardless of their operational use. Generally, the gust load
factors control the design of aircraft which are intended for
strictly nonacrobatic usage.
An entirely different situation exists in aircraft design with
maneuvering load factors. It is necessary to discuss this matter
separately with respect to: (1) aircraft designed in accordance
with the category system (i.e., normal, utility, acrobatic); and
(2) older designs built according to requirements that did not
provide for operational categories.
Aircraft designed under the category system are readily
identified by a placard in the flight deck, which states the
operational category (or categories) in which the aircraft

5-33

is certificated. The maximum safe load factors (limit load
factors) specified for aircraft in the various categories are:
CATEGORY
Normal1
Utility (mild acrobatics,
including spins)
Acrobatic

LIMIT LOAD FACTOR
3.8 to –1.52
4.4 to –1.76
6.0 to –3.00

1

For aircraft with gross weight of more than 4,000 pounds,
the limit load factor is reduced. To the limit loads given
above, a safety factor of 50 percent is added.
There is an upward graduation in load factor with the
increasing severity of maneuvers. The category system
provides for maximum utility of an aircraft. If normal
operation alone is intended, the required load factor (and
consequently the weight of the aircraft) is less than if the
aircraft is to be employed in training or acrobatic maneuvers
as they result in higher maneuvering loads.
Aircraft that do not have the category placard are designs that
were constructed under earlier engineering requirements in
which no operational restrictions were specifically given to
the pilots. For aircraft of this type (up to weights of about
4,000 pounds), the required strength is comparable to presentday utility category aircraft, and the same types of operation
are permissible. For aircraft of this type over 4,000 pounds,
the load factors decrease with weight. These aircraft should
be regarded as being comparable to the normal category
aircraft designed under the category system, and they should
be operated accordingly.
Load Factors in Steep Turns
At a constant altitude, during a coordinated turn in any
aircraft, the load factor is the result of two forces: centrifugal
force and weight. [Figure 5-52] For any given bank angle,
the ROT varies with the airspeed—the higher the speed, the

50

°
40

°
30

0°
0° 1
° 2

Gravity = 1G

It should be noted how rapidly the line denoting load factor
rises as it approaches the 90° bank line, which it never quite
reaches because a 90° banked, constant altitude turn is not
mathematically possible. An aircraft may be banked to 90°
in a coordinated turn if not trying to hold altitude. An aircraft
that can be held in a 90° banked slipping turn is capable of
straight knife-edged flight. At slightly more than 80°, the
load factor exceeds the limit of 6 Gs, the limit load factor of
an acrobatic aircraft.
For a coordinated, constant altitude turn, the approximate
maximum bank for the average general aviation aircraft is 60°.
This bank and its resultant necessary power setting reach the
limit of this type of aircraft. An additional 10° bank increases
the load factor by approximately 1 G, bringing it close to the
yield point established for these aircraft. [Figure 5-54]
Load Factors and Stalling Speeds
Any aircraft, within the limits of its structure, may be stalled
at any airspeed. When a sufficiently high AOA is imposed,
the smooth flow of air over an airfoil breaks up and separates,
producing an abrupt change of flight characteristics and a
sudden loss of lift, which results in a stall.
A study of this effect has revealed that an aircraft's stalling
speed increases in proportion to the square root of the

7
6

°

Lo

ad

fac

Centrifugal
force = 1.73 Gs

tor

=2

5
4
3
2
1

Gs

Figure 5-52. Two forces cause load factor during turns.

5-34

Figure 5-53 reveals an important fact about turns—the load
factor increases at a terrific rate after a bank has reached
45° or 50°. The load factor for any aircraft in a coordinated
level turn at 60° bank is 2 Gs. The load factor in an 80° bank
is 5.76 Gs. The wing must produce lift equal to these load
factors if altitude is to be maintained.

Load factor (G units)

60

slower the ROT. This compensates for added centrifugal
force, allowing the load factor to remain the same.

0
0°

10° 20° 30° 40° 50° 60° 70° 80° 90°
Bank angle

Figure 5-53. Angle of bank changes load factor in level flight.

load factor. This means that an aircraft with a normal
unaccelerated stalling speed of 50 knots can be stalled at 100
knots by inducing a load factor of 4 Gs. If it were possible
for this aircraft to withstand a load factor of nine, it could
be stalled at a speed of 150 knots. A pilot should be aware
of the following:


The danger of inadvertently stalling the aircraft by
increasing the load factor, as in a steep turn or spiral;



When intentionally stalling an aircraft above its
design maneuvering speed, a tremendous load factor
is imposed.

Figures 5-53 and 5-54 show that banking an aircraft greater
than 72° in a steep turn produces a load factor of 3, and the
stalling speed is increased significantly. If this turn is made
in an aircraft with a normal unaccelerated stalling speed of
45 knots, the airspeed must be kept greater than 75 knots to
prevent inducing a stall. A similar effect is experienced in a
quick pull up or any maneuver producing load factors above
1 G. This sudden, unexpected loss of control, particularly in
a steep turn or abrupt application of the back elevator control
near the ground, has caused many accidents.
Since the load factor is squared as the stalling speed doubles,
tremendous loads may be imposed on structures by stalling
an aircraft at relatively high airspeeds.
The following information primarily applies to fixed-wing
airplanes. The maximum speed at which an airplane may
be stalled safely is now determined for all new designs.

This speed is called the "design maneuvering speed" (VA),
which is the speed below which you can move a single
flight control, one time, to its full deflection, for one axis
of airplane rotation only (pitch, roll or yaw), in smooth
air, without risk of damage to the airplane. VA must be
entered in the FAA-approved Airplane Flight Manual/
Pilot's Operating Handbook (AFM/POH) of all recently
designed airplanes. For older general aviation airplanes,
this speed is approximately 1.7 times the normal stalling
speed. Thus, an older airplane that normally stalls at 60
knots must never be stalled at above 102 knots (60 knots ×
1.7 = 102 knots). An airplane with a normal stalling speed
of 60 knots stalled at 102 knots undergoes a load factor
equal to the square of the increase in speed, or 2.89 Gs (1.7
× 1.7 = 2.89 Gs). (The above figures are approximations to
be considered as a guide, and are not the exact answers to
any set of problems. The design maneuvering speed should
be determined from the particular airplane's operating
limitations provided by the manufacturer.) Operating at or
below design maneuvering speed does not provide structural
protection against multiple full control inputs in one axis or
full control inputs in more than one axis at the same time.
Since the leverage in the control system varies with different
aircraft (some types employ "balanced" control surfaces while
others do not), the pressure exerted by the pilot on the controls
cannot be accepted as an index of the load factors produced
in different aircraft. In most cases, load factors can be judged
by the experienced pilot from the feel of seat pressure. Load
factors can also be measured by an instrument called an
"accelerometer," but this instrument is not common in general

50
50
60
60

4

70
70
80
80

3

100
100
120
120

2

150
150

Unaccelerated stall speed

Ratio of acceleration Vs to unaccelerated Vs

40
40
5

1

0

0

1

2

3

4

5

6

7

"G" Load

8

20

40

60

80

100 120 140 160 180 200 220 240 260
Accelerated stall speed

Figure 5-54. Load factor changes stall speed.

5-35

aviation training aircraft. The development of the ability to
judge load factors from the feel of their effect on the body is
important. A knowledge of these principles is essential to the
development of the ability to estimate load factors.
A thorough knowledge of load factors induced by varying
degrees of bank and the VA aids in the prevention of two of
the most serious types of accidents:
1.	 Stalls from steep turns or excessive maneuvering near
the ground
2.	 Structural failures during acrobatics or other violent
maneuvers resulting from loss of control
Load Factors and Flight Maneuvers
Critical load factors apply to all flight maneuvers except
unaccelerated straight flight where a load factor of 1 G is
always present. Certain maneuvers considered in this section
are known to involve relatively high load factors. Full
application of pitch, roll, or yaw controls should be confined
to speeds below the maneuvering speed. Avoid rapid and
large alternating control inputs, especially in combination
with large changes in pitch, roll, or yaw (e.g., large sideslip
angles) as they may result in structural failures at any speed,
including below VA.
Turns
Increased load factors are a characteristic of all banked
turns. As noted in the section on load factors in steep turns,
load factors become significant to both flight performance
and load on wing structure as the bank increases beyond
approximately 45°.
The yield factor of the average light plane is reached
at a bank of approximately 70° to 75°, and the stalling
speed is increased by approximately one-half at a bank of
approximately 63°.
Stalls
The normal stall entered from straight-and-level flight, or an
unaccelerated straight climb, does not produce added load
factors beyond the 1 G of straight-and-level flight. As the
stall occurs, however, this load factor may be reduced toward
zero, the factor at which nothing seems to have weight. The
pilot experiences a sensation of "floating free in space." If
recovery is effected by snapping the elevator control forward,
negative load factors (or those that impose a down load on
the wings and raise the pilot from the seat) may be produced.
During the pull up following stall recovery, significant
load factors are sometimes induced. These may be further
increased inadvertently during excessive diving (and
consequently high airspeed) and abrupt pull ups to level
flight. One usually leads to the other, thus increasing the load
5-36

factor. Abrupt pull ups at high diving speeds may impose
critical loads on aircraft structures and may produce recurrent
or secondary stalls by increasing the AOA to that of stalling.
As a generalization, a recovery from a stall made by diving
only to cruising or design maneuvering airspeed, with a
gradual pull up as soon as the airspeed is safely above stalling,
can be effected with a load factor not to exceed 2 or 2.5 Gs. A
higher load factor should never be necessary unless recovery
has been effected with the aircraft's nose near or beyond the
vertical attitude or at extremely low altitudes to avoid diving
into the ground.
Spins
A stabilized spin is not different from a stall in any element
other than rotation and the same load factor considerations
apply to spin recovery as apply to stall recovery. Since spin
recoveries are usually effected with the nose much lower than is
common in stall recoveries, higher airspeeds and consequently
higher load factors are to be expected. The load factor in a
proper spin recovery usually is found to be about 2.5 Gs.
The load factor during a spin varies with the spin characteristics
of each aircraft, but is usually found to be slightly above the
1 G of level flight. There are two reasons for this:
1.	 Airspeed in a spin is very low, usually within 2 knots
of the unaccelerated stalling speeds.
2.	 An aircraft pivots, rather than turns, while it is in a spin.
High Speed Stalls
The average light plane is not built to withstand the repeated
application of load factors common to high speed stalls.
The load factor necessary for these maneuvers produces a
stress on the wings and tail structure, which does not leave
a reasonable margin of safety in most light aircraft.
The only way this stall can be induced at an airspeed above
normal stalling involves the imposition of an added load
factor, which may be accomplished by a severe pull on the
elevator control. A speed of 1.7 times stalling speed (about
102 knots in a light aircraft with a stalling speed of 60 knots)
produces a load factor of 3 Gs. Only a very narrow margin
for error can be allowed for acrobatics in light aircraft. To
illustrate how rapidly the load factor increases with airspeed,
a high-speed stall at 112 knots in the same aircraft would
produce a load factor of 4 Gs.
Chandelles and Lazy Eights
A chandelle is a maximum performance climbing turn
beginning from approximately straight-and-level flight,
and ending at the completion of a precise 180° turn in a
wings-level, nose-high attitude at the minimum controllable

airspeed. In this flight maneuver, the aircraft is in a steep
climbing turn and almost stalls to gain altitude while changing
direction. A lazy eight derives its name from the manner in
which the extended longitudinal axis of the aircraft is made
to trace a flight pattern in the form of a figure "8" lying on
its side. It would be difficult to make a definite statement
concerning load factors in these maneuvers as both involve
smooth, shallow dives and pull-ups. The load factors incurred
depend directly on the speed of the dives and the abruptness
of the pull-ups during these maneuvers.
Generally, the better the maneuver is performed, the less
extreme the load factor induced. A chandelle or lazy eight
in which the pull-up produces a load factor greater than 2 Gs
will not result in as great a gain in altitude; in low-powered
aircraft, it may result in a net loss of altitude.
The smoothest pull-up possible, with a moderate load factor,
delivers the greatest gain in altitude in a chandelle and results
in a better overall performance in both chandelles and lazy
eights. The recommended entry speed for these maneuvers
is generally near the manufacturer's design maneuvering
speed, which allows maximum development of load factors
without exceeding the load limits.
Rough Air
All standard certificated aircraft are designed to withstand
loads imposed by gusts of considerable intensity. Gust load
factors increase with increasing airspeed, and the strength used
for design purposes usually corresponds to the highest level
flight speed. In extremely rough air, as in thunderstorms or
frontal conditions, it is wise to reduce the speed to the design
maneuvering speed. Regardless of the speed held, there may
be gusts that can produce loads that exceed the load limits.
Each specific aircraft is designed with a specific G loading
that can be imposed on the aircraft without causing structural
damage. There are two types of load factors factored into
aircraft design: limit load and ultimate load. The limit load
is a force applied to an aircraft that causes a bending of the
aircraft structure that does not return to the original shape.
The ultimate load is the load factor applied to the aircraft
beyond the limit load and at which point the aircraft material
experiences structural failure (breakage). Load factors lower
than the limit load can be sustained without compromising
the integrity of the aircraft structure.
Speeds up to, but not exceeding, the maneuvering speed
allow an aircraft to stall prior to experiencing an increase in
load factor that would exceed the limit load of the aircraft.
Most AFM/POH now include turbulent air penetration
information, which help today's pilots safely fly aircraft

capable of a wide range of speeds and altitudes. It is important
for the pilot to remember that the maximum "never-exceed"
placard dive speeds are determined for smooth air only. High
speed dives or acrobatics involving speed above the known
maneuvering speed should never be practiced in rough or
turbulent air.
Vg Diagram
The flight operating strength of an aircraft is presented
on a graph whose vertical scale is based on load factor.
[Figure 5-55] The diagram is called a Vg diagram—velocity
versus G loads or load factor. Each aircraft has its own Vg
diagram that is valid at a certain weight and altitude.
The lines of maximum lift capability (curved lines) are the
first items of importance on the Vg diagram. The aircraft in
Figure 5-53 is capable of developing no more than +1 G at
64 mph, the wing level stall speed of the aircraft. Since the
maximum load factor varies with the square of the airspeed,
the maximum positive lift capability of this aircraft is 2 G at
92 mph, 3 G at 112 mph, 4.4 G at 137 mph, and so forth. Any
load factor above this line is unavailable aerodynamically
(i.e., the aircraft cannot fly above the line of maximum lift
capability because it stalls). The same situation exists for
negative lift flight with the exception that the speed necessary
to produce a given negative load factor is higher than that to
produce the same positive load factor.
If the aircraft is flown at a positive load factor greater than the
positive limit load factor of 4.4, structural damage is possible.
When the aircraft is operated in this region, objectionable
permanent deformation of the primary structure may take place
and a high rate of fatigue damage is incurred. Operation above
the limit load factor must be avoided in normal operation.
There are two other points of importance on the Vg diagram.
One point is the intersection of the positive limit load factor
and the line of maximum positive lift capability. The airspeed
at this point is the minimum airspeed at which the limit load
can be developed aerodynamically. Any airspeed greater than
this provides a positive lift capability sufficient to damage
the aircraft. Conversely, any airspeed less than this does not
provide positive lift capability sufficient to cause damage
from excessive flight loads. The usual term given to this speed
is "maneuvering speed," since consideration of subsonic
aerodynamics would predict minimum usable turn radius or
maneuverability to occur at this condition. The maneuver
speed is a valuable reference point, since an aircraft operating
below this point cannot produce a damaging positive flight
load. Any combination of maneuver and gust cannot create
damage due to excess airload when the aircraft is below the
maneuver speed.

5-37

7
6

Structural damage

5

Maneuvering speed

4

Load factor

2

ra
ele

Caution range

Normal operating range

Normal stall speed

1

l
al
st

Level flight at 1 G

0

Never exceed speed

c
Ac

d
te

Structural failure

3

–1
–2

Structural Damage

–3

20

40

60

80

100

120

140

160

180

200

220

240

Indicated airspeed (mph)
Figure 5-55. Typical Vg diagram.

The other point of importance on the Vg diagram is the
intersection of the negative limit load factor and line of
maximum negative lift capability. Any airspeed greater than
this provides a negative lift capability sufficient to damage
the aircraft; any airspeed less than this does not provide
negative lift capability sufficient to damage the aircraft from
excessive flight loads.
The limit airspeed (or redline speed) is a design reference point
for the aircraft—this aircraft is limited to 225 mph. If flight
is attempted beyond the limit airspeed, structural damage or
structural failure may result from a variety of phenomena.
The aircraft in flight is limited to a regime of airspeeds
and Gs that do not exceed the limit (or redline) speed, do
not exceed the limit load factor, and cannot exceed the
maximum lift capability. The aircraft must be operated
within this "envelope" to prevent structural damage and
ensure the anticipated service lift of the aircraft is obtained.
The pilot must appreciate the Vg diagram as describing the
allowable combination of airspeeds and load factors for
safe operation. Any maneuver, gust, or gust plus maneuver
outside the structural envelope can cause structural damage
and effectively shorten the service life of the aircraft.

5-38

Rate of Turn
The rate of turn (ROT) is the number of degrees (expressed in
degrees per second) of heading change that an aircraft makes.
The ROT can be determined by taking the constant of 1,091,
multiplying it by the tangent of any bank angle and dividing
that product by a given airspeed in knots as illustrated in
Figure 5-55. If the airspeed is increased and the ROT desired
is to be constant, the angle of bank must be increased,
otherwise, the ROT decreases. Likewise, if the airspeed is
held constant, an aircraft's ROT increases if the bank angle
is increased. The formula in Figures 5-56 through 5-58
depicts the relationship between bank angle and airspeed as
they affect the ROT.
NOTE: All airspeed discussed in this section is true airspeed
(TAS).
Airspeed significantly effects an aircraft's ROT. If airspeed is
increased, the ROT is reduced if using the same angle of bank
used at the lower speed. Therefore, if airspeed is increased
as illustrated in Figure 5-57, it can be inferred that the angle
of bank must be increased in order to achieve the same ROT
achieved in Figure 5-58.

ROT =
Example

1,091 x tangent of the bank angle
airspeed (in knots)

Example

The rate of turn for an aircraft in a
coordinated turn of 30° and traveling at
120 knots would have a ROT as follows.

ROT =

1,091 x tangent of 30°
120 knots

ROT =

1,091 x 0.5773 (tangent of 30°)
120 knots

ROT (5.25) =

bank angle.

240 x 5.25 = tangent of X
1,091
1.1549 = tangent of X
49° = X
Figure 5-58. To achieve the same rate of turn of an aircraft traveling

at 120 knots, an increase of bank angle is required.

What does this mean on a practicable side? If a given
airspeed and bank angle produces a specific ROT, additional
conclusions can be made. Knowing the ROT is a given number
of degrees of change per second, the number of seconds it
takes to travel 360° (a circle) can be determined by simple
division. For example, if moving at 120 knots with a 30° bank
angle, the ROT is 5.25° per second and it takes 68.6 seconds
(360° divided by 5.25 = 68.6 seconds) to make a complete
circle. Likewise, if flying at 240 knots TAS and using a 30°
angle of bank, the ROT is only about 2.63° per second and it
takes about 137 seconds to complete a 360° circle. Looking at
the formula, any increase in airspeed is directly proportional
to the time the aircraft takes to travel an arc.
So why is this important to understand? Once the ROT is
understood, a pilot can determine the distance required to
make that particular turn, which is explained in radius of turn.
Radius of Turn
The radius of turn is directly linked to the ROT, which
explained earlier is a function of both bank angle and
airspeed. If the bank angle is held constant and the airspeed
is increased, the radius of the turn changes (increases). A
higher airspeed causes the aircraft to travel through a longer
Example

1,091 x tangent of X
240 knots

240 x 5.25 = 1,091 x tangent of X

ROT = 5.25 degrees per second
Figure 5-56. Rate of turn for a given airspeed (knots, TAS) and

Suppose we wanted to know what bank angle
would give us a rate of turn of 5.25° per second
at 240 knots. A slight rearrangement of the formula
would indicate it will take a 49° angle of bank to
achieve the same ROT used at the lower airspeed
of 120 knots.

Suppose we were to increase the speed to 240
knots, what is the ROT? Using the same
formula from above we see that:

ROT =

1,091 x tangent of 30°
240 knots

ROT = 2.62 degrees per second
An increase in speed causes a decrease in the
ROT when using the same bank angle.
Figure 5-57. Rate of turn when increasing speed.

arc due to a greater speed. An aircraft traveling at 120 knots
is able to turn a 360° circle in a tighter radius than an aircraft
traveling at 240 knots. In order to compensate for the increase
in airspeed, the bank angle would need to be increased.
The radius of turn (R) can be computed using a simple
formula. The radius of turn is equal to the velocity squared
(V2) divided by 11.26 times the tangent of the bank angle.
R=

V2
11.26 × tangent of bank angle

Using the examples provided in Figures 5-56 through 5-58, the
turn radius for each of the two speeds can be computed.
Note that if the speed is doubled, the radius is quadrupled.
[Figures 5-59 and 5-60]
Another way to determine the radius of turn is speed using
feet per second (fps), π (3.1415), and the ROT. In one of the
previous examples, it was determined that an aircraft with
a ROT of 5.25 degrees per second required 68.6 seconds to
make a complete circle. An aircraft's speed (in knots) can
120 knots

R=

V2
11.26 x tangent of bank angle
2

120
11.26 x tangent of 30°
14,400
R=
11.26 x 0.5773

R=

R = 2,215 feet
The radius of a turn required by an aircraft traveling at 120 knots
and using a bank angle of 30° is 2,215 feet
Figure 5-59. Radius at 120 knots with bank angle of 30°.

5-39

240 knots

R=

V2
11.26 x tangent of bank angle
2

R=

240
11.26 x tangent of 30°

R=

57,600
11.26 x 0.57735

R = 8,861 feet
(four times the radius at 120 knots)
The radius of a turn required by an aircraft traveling at 240 knots
using the same bank angle in Figure 4-51 is 8,861 feet. Speed
is a major factor in a turn.
Figure 5-60. Radius at 240 knots.

be converted to fps by multiplying it by a constant of 1.69.
Therefore, an aircraft traveling at 120 knots (TAS) travels
at 202.8 fps. Knowing the speed in fps (202.8) multiplied by
the time an aircraft takes to complete a circle (68.6 seconds)
can determine the size of the circle; 202.8 times 68.6 equals
13,912 feet. Dividing by π yields a diameter of 4,428 feet,
which when divided by 2 equals a radius of 2,214 feet
[Figure 5-61], a foot within that determined through use of
the formula in Figure 5-59.
In Figure 5-62, the pilot enters a canyon and decides to turn
180° to exit. The pilot uses a 30° bank angle in his turn.

Weight and Balance
The aircraft's weight and balance data is important
information for a pilot that must be frequently reevaluated.
Although the aircraft was weighed during the certification
process, this information is not valid indefinitely. Equipment
changes or modifications affect the weight and balance data.
Too often pilots reduce the aircraft weight and balance into
a rule of thumb, such as: "If I have three passengers, I can
load only 100 gallons of fuel; four passengers, 70 gallons."
Weight and balance computations should be part of every
preflight briefing. Never assume three passengers are always

r = speed (fps) x

360
ROT

Pi (π)
2
r=

202.8 x 68.6
π
2

r=

13,912
π
2

r=

4,428
2

= 2,214 feet

Figure 5-61. Another formula that can be used for radius.

5-40

of equal weight. Instead, do a full computation of all items
to be loaded on the aircraft, including baggage, as well as
the pilot and passenger. It is recommended that all bags be
weighed to make a precise computation of how the aircraft
CG is positioned.
The importance of the CG was stressed in the discussion
of stability, controllability, and performance. Unequal load
distribution causes accidents. A competent pilot understands
and respects the effects of CG on an aircraft.
Weight and balance are critical components in the utilization
of an aircraft to its fullest potential. The pilot must know
how much fuel can be loaded onto the aircraft without
violating CG limits, as well as weight limits to conduct
long or short flights with or without a full complement of
allowable passengers. For example, an aircraft has four seats
and can carry 60 gallons of fuel. How many passengers can
the aircraft safely carry? Can all those seats be occupied at
all times with the varying fuel loads? Four people who each
weigh 150 pounds leads to a different weight and balance
computation than four people who each weigh 200 pounds.
The second scenario loads an additional 200 pounds onto the
aircraft and is equal to about 30 gallons of fuel.
The additional weight may or may not place the CG outside
of the CG envelope, but the maximum gross weight could
be exceeded. The excess weight can overstress the aircraft
and degrade the performance.
Aircraft are certificated for weight and balance for two
principal reasons:
1.	 The effect of the weight on the aircraft's primary
structure and its performance characteristics
2.	 The effect of the location of this weight on flight
characteristics, particularly in stall and spin recovery
and stability
Aircraft, such as balloons and weight-shift control, do not
require weight and balance computations because the load
is suspended below the lifting mechanism. The CG range
in these types of aircraft is such that it is difficult to exceed
loading limits. For example, the rear seat position and fuel
of a weight-shift control aircraft are as close as possible to
the hang point with the aircraft in a suspended attitude. Thus,
load variations have little effect on the CG. This also holds
true for the balloon basket or gondola. While it is difficult
to exceed CG limits in these aircraft, pilots should never
overload an aircraft because overloading causes structural
damage and failures. Weight and balance computations are
not required, but pilots should calculate weight and remain
within the manufacturer's established limit.

120 knots
20
I0
I0
20

20

I0
I0
20

STBY PWR

TEST

4,430 feet
2,215 feet
et
5,000 fe

R=

V2
11.26 x tangent of the bank angle 30°

R=

14,400
1202
=
= 2,215 feet
11.26 x 0.5773
6.50096

140 knots

20
I0
I0
20

20

I0
I0
20

STBY PWR

6,028 feet
5,000

R=

TEST

3,014 feet
feet

V2
11.26 x tangent of the bank angle 30°

R=

19,600
1402
=
= 3,014 feet
11.26 x 0.5773 6.50096

Figure 5-62. Two aircraft have flown into a canyon by error. The canyon is 5,000 feet across and has sheer cliffs on both sides. The pilot

in the top image is flying at 120 knots. After realizing the error, the pilot banks hard and uses a 30° bank angle to reverse course. This
aircraft requires about 4,000 feet to turn 180°, and makes it out of the canyon safely. The pilot in the bottom image is flying at 140 knots
and also uses a 30° angle of bank in an attempt to reverse course. The aircraft, although flying just 20 knots faster than the aircraft in the
top image, requires over 6,000 feet to reverse course to safety. Unfortunately, the canyon is only 5,000 feet across and the aircraft will hit
the canyon wall. The point is that airspeed is the most influential factor in determining how much distance is required to turn. Many pilots
have made the error of increasing the steepness of their bank angle when a simple reduction of speed would have been more appropriate.

5-41

Effect of Weight on Flight Performance
The takeoff/climb and landing performance of an aircraft are
determined on the basis of its maximum allowable takeoff and
landing weights. A heavier gross weight results in a longer
takeoff run and shallower climb, and a faster touchdown
speed and longer landing roll. Even a minor overload may
make it impossible for the aircraft to clear an obstacle that
normally would not be a problem during takeoff under more
favorable conditions.
The detrimental effects of overloading on performance are
not limited to the immediate hazards involved with takeoffs
and landings. Overloading has an adverse effect on all
climb and cruise performance, which leads to overheating
during climbs, added wear on engine parts, increased fuel
consumption, slower cruising speeds, and reduced range.
The manufacturers of modern aircraft furnish weight and
balance data with each aircraft produced. Generally, this
information may be found in the FAA-approved AFM/POH
and easy-to-read charts for determining weight and balance
data are now provided. Increased performance and loadcarrying capability of these aircraft require strict adherence
to the operating limitations prescribed by the manufacturer.
Deviations from the recommendations can result in structural
damage or complete failure of the aircraft's structure. Even
if an aircraft is loaded well within the maximum weight
limitations, it is imperative that weight distribution be
within the limits of CG location. The preceding brief study
of aerodynamics and load factors points out the reasons for
this precaution. The following discussion is background
information into some of the reasons why weight and balance
conditions are important to the safe flight of an aircraft.
In some aircraft, it is not possible to fill all seats, baggage
compartments, and fuel tanks, and still remain within
approved weight or balance limits. For example, in several
popular four-place aircraft, the fuel tanks may not be filled to
capacity when four occupants and their baggage are carried.
In a certain two-place aircraft, no baggage may be carried
in the compartment aft of the seats when spins are to be
practiced. It is important for a pilot to be aware of the weight
and balance limitations of the aircraft being flown and the
reasons for these limitations.
Effect of Weight on Aircraft Structure
The effect of additional weight on the wing structure of an
aircraft is not readily apparent. Airworthiness requirements
prescribe that the structure of an aircraft certificated in the
normal category (in which acrobatics are prohibited) must
be strong enough to withstand a load factor of 3.8 Gs to take
care of dynamic loads caused by maneuvering and gusts. This
means that the primary structure of the aircraft can withstand
5-42

a load of 3.8 times the approved gross weight of the aircraft
without structural failure occurring. If this is accepted as
indicative of the load factors that may be imposed during
operations for which the aircraft is intended, a 100-pound
overload imposes a potential structural overload of 380
pounds. The same consideration is even more impressive
in the case of utility and acrobatic category aircraft, which
have load factor requirements of 4.4 and 6.0, respectively.
Structural failures that result from overloading may be
dramatic and catastrophic, but more often they affect
structural components progressively in a manner that
is difficult to detect and expensive to repair. Habitual
overloading tends to cause cumulative stress and damage
that may not be detected during preflight inspections and
result in structural failure later during completely normal
operations. The additional stress placed on structural parts
by overloading is believed to accelerate the occurrence of
metallic fatigue failures.
A knowledge of load factors imposed by flight maneuvers
and gusts emphasizes the consequences of an increase in the
gross weight of an aircraft. The structure of an aircraft about to
undergo a load factor of 3 Gs, as in recovery from a steep dive,
must be prepared to withstand an added load of 300 pounds
for each 100-pound increase in weight. It should be noted that
this would be imposed by the addition of about 16 gallons
of unneeded fuel in a particular aircraft. FAA-certificated
civil aircraft have been analyzed structurally and tested for
flight at the maximum gross weight authorized and within the
speeds posted for the type of flights to be performed. Flights at
weights in excess of this amount are quite possible and often
are well within the performance capabilities of an aircraft.
This fact should not mislead the pilot, as the pilot may not
realize that loads for which the aircraft was not designed are
being imposed on all or some part of the structure.
In loading an aircraft with either passengers or cargo, the
structure must be considered. Seats, baggage compartments,
and cabin floors are designed for a certain load or
concentration of load and no more. For example, a light
plane baggage compartment may be placarded for 20 pounds
because of the limited strength of its supporting structure
even though the aircraft may not be overloaded or out of CG
limits with more weight at that location.
Effect of Weight on Stability and Controllability
Overloading also affects stability. An aircraft that is stable
and controllable when loaded normally may have very
different flight characteristics when overloaded. Although
the distribution of weight has the most direct effect on this,
an increase in the aircraft's gross weight may be expected
to have an adverse effect on stability, regardless of location

of the CG. The stability of many certificated aircraft is
completely unsatisfactory if the gross weight is exceeded.
Effect of Load Distribution
The effect of the position of the CG on the load imposed
on an aircraft's wing in flight is significant to climb and
cruising performance. An aircraft with forward loading is
"heavier" and consequently, slower than the same aircraft
with the CG further aft.
Figure 5-63 illustrates why this is true. With forward loading,
"nose-up" trim is required in most aircraft to maintain level
cruising flight. Nose-up trim involves setting the tail surfaces
to produce a greater down load on the aft portion of the
fuselage, which adds to the wing loading and the total lift
required from the wing if altitude is to be maintained. This
requires a higher AOA of the wing, which results in more
drag and, in turn, produces a higher stalling speed.
With aft loading and "nose-down" trim, the tail surfaces
exert less down load, relieving the wing of that much wing
loading and lift required to maintain altitude. The required
AOA of the wing is less, so the drag is less, allowing for a
faster cruise speed. Theoretically, a neutral load on the tail
surfaces in cruising flight would produce the most efficient
overall performance and fastest cruising speed, but it would
also result in instability. Modern aircraft are designed to
require a down load on the tail for stability and controllability.
A zero indication on the trim tab control is not necessarily
the same as "neutral trim" because of the force exerted by
downwash from the wings and the fuselage on the tail surfaces.
The effects of the distribution of the aircraft's useful load
have a significant influence on its flight characteristics, even
when the load is within the CG limits and the maximum
Load imposed by tail
Gross weight
CG

Forward CG

Center of lift

Stronger
Down load on tail

Load imposed by tail
Gross weight

CG

AFT CG

Lighter
Down load on tail

Figure 5-63. Effect of load distribution on balance.

permissible gross weight. Important among these effects
are changes in controllability, stability, and the actual load
imposed on the wing.
Generally, an aircraft becomes less controllable, especially
at slow flight speeds, as the CG is moved further aft. An
aircraft that cleanly recovers from a prolonged spin with
the CG at one position may fail completely to respond to
normal recovery attempts when the CG is moved aft by one
or two inches.
It is common practice for aircraft designers to establish
an aft CG limit that is within one inch of the maximum,
which allows normal recovery from a one-turn spin. When
certificating an aircraft in the utility category to permit
intentional spins, the aft CG limit is usually established
at a point several inches forward of that permissible for
certification in the normal category.
Another factor affecting controllability, which has become
more important in current designs of large aircraft, is the
effect of long moment arms to the positions of heavy
equipment and cargo. The same aircraft may be loaded to
maximum gross weight within its CG limits by concentrating
fuel, passengers, and cargo near the design CG, or by
dispersing fuel and cargo loads in wingtip tanks and cargo
bins forward and aft of the cabin.
With the same total weight and CG, maneuvering the
aircraft or maintaining level flight in turbulent air requires
the application of greater control forces when the load is
dispersed. The longer moment arms to the positions of the
heavy fuel and cargo loads must be overcome by the action
of the control surfaces. An aircraft with full outboard wing
tanks or tip tanks tends to be sluggish in roll when control
situations are marginal, while one with full nose and aft cargo
bins tends to be less responsive to the elevator controls.
The rearward CG limit of an aircraft is determined largely
by considerations of stability. The original airworthiness
requirements for a type certificate specify that an aircraft in
flight at a certain speed dampens out vertical displacement of
the nose within a certain number of oscillations. An aircraft
loaded too far rearward may not do this. Instead, when the
nose is momentarily pulled up, it may alternately climb and
dive becoming steeper with each oscillation. This instability
is not only uncomfortable to occupants, but it could even
become dangerous by making the aircraft unmanageable
under certain conditions.
The recovery from a stall in any aircraft becomes progressively
more difficult as its CG moves aft. This is particularly
important in spin recovery, as there is a point in rearward
5-43

loading of any aircraft at which a "flat" spin develops. A
flat spin is one in which centrifugal force, acting through a
CG located well to the rear, pulls the tail of the aircraft out
away from the axis of the spin, making it impossible to get
the nose down and recover.
An aircraft loaded to the rear limit of its permissible CG
range handles differently in turns and stall maneuvers and
has different landing characteristics than when it is loaded
near the forward limit.
The forward CG limit is determined by a number of
considerations. As a safety measure, it is required that the
trimming device, whether tab or adjustable stabilizer, be
capable of holding the aircraft in a normal glide with the power
off. A conventional aircraft must be capable of a full stall,
power-off landing in order to ensure minimum landing speed
in emergencies. A tailwheel-type aircraft loaded excessively
nose-heavy is difficult to taxi, particularly in high winds. It
can be nosed over easily by use of the brakes, and it is difficult
to land without bouncing since it tends to pitch down on the
wheels as it is slowed down and flared for landing. Steering
difficulties on the ground may occur in nosewheel-type
aircraft, particularly during the landing roll and takeoff. The
effects of load distribution are summarized as follows:


The CG position influences the lift and AOA of the
wing, the amount and direction of force on the tail,
and the degree of deflection of the stabilizer needed
to supply the proper tail force for equilibrium. The
latter is very important because of its relationship to
elevator control force.



The aircraft stalls at a higher speed with a forward CG
location. This is because the stalling AOA is reached
at a higher speed due to increased wing loading.



Higher elevator control forces normally exist with a
forward CG location due to the increased stabilizer
deflection required to balance the aircraft.



The aircraft cruises faster with an aft CG location
because of reduced drag. The drag is reduced because
a smaller AOA and less downward deflection of the
stabilizer are required to support the aircraft and
overcome the nose-down pitching tendency.



The aircraft becomes less stable as the CG is moved
rearward. This is because when the CG is moved
rearward, it causes a decrease in the AOA. Therefore,
the wing contribution to the aircraft's stability is
now decreased, while the tail contribution is still
stabilizing. When the point is reached that the wing
and tail contributions balance, then neutral stability
exists. Any CG movement further aft results in an
unstable aircraft.

5-44



A forward CG location increases the need for greater
back elevator pressure. The elevator may no longer
be able to oppose any increase in nose-down pitching.
Adequate elevator control is needed to control the
aircraft throughout the airspeed range down to the stall.

A detailed discussion and additional information relating
to weight and balance can be found in Chapter 10, Weight
and Balance.

High Speed Flight
Subsonic Versus Supersonic Flow
In subsonic aerodynamics, the theory of lift is based upon the
forces generated on a body and a moving gas (air) in which
it is immersed. At speeds of approximately 260 knots or
less, air can be considered incompressible in that, at a fixed
altitude, its density remains nearly constant while its pressure
varies. Under this assumption, air acts the same as water and
is classified as a fluid. Subsonic aerodynamic theory also
assumes the effects of viscosity (the property of a fluid that
tends to prevent motion of one part of the fluid with respect
to another) are negligible and classifies air as an ideal fluid
conforming to the principles of ideal-fluid aerodynamics such
as continuity, Bernoulli's principle, and circulation.
In reality, air is compressible and viscous. While the effects of
these properties are negligible at low speeds, compressibility
effects in particular become increasingly important as speed
increases. Compressibility (and to a lesser extent viscosity) is
of paramount importance at speeds approaching the speed of
sound. In these speed ranges, compressibility causes a change
in the density of the air around an aircraft.
During flight, a wing produces lift by accelerating the airflow
over the upper surface. This accelerated air can, and does,
reach sonic speeds even though the aircraft itself may be flying
subsonic. At some extreme AOAs, in some aircraft, the speed
of the air over the top surface of the wing may be double the
aircraft's speed. It is therefore entirely possible to have both
supersonic and subsonic airflow on an aircraft at the same time.
When flow velocities reach sonic speeds at some location on
an aircraft (such as the area of maximum camber on the wing),
further acceleration results in the onset of compressibility
effects, such as shock wave formation, drag increase, buffeting,
stability, and control difficulties. Subsonic flow principles are
invalid at all speeds above this point. [Figure 5-64]
Speed Ranges
The speed of sound varies with temperature. Under standard
temperature conditions of 15 °C, the speed of sound at sea
level is 661 knots. At 40,000 feet, where the temperature is
–55 °C, the speed of sound decreases to 574 knots. In high­

Maximum local velocity is less than sonic
Critical mach number
M = 0.72

S

Maximum local velocity equal to sonic
onic
ers
up flow

Normal shock wave
Subsonic

M = 0.77

Possible

separation

Figure 5-64. Wing airflow.

speed flight and/or high-altitude flight, the measurement of
speed is expressed in terms of a "Mach number"—the ratio
of the true airspeed of the aircraft to the speed of sound in
the same atmospheric conditions. An aircraft traveling at
the speed of sound is traveling at Mach 1.0. Aircraft speed
regimes are defined approximately as follows:
Subsonic—Mach numbers below 0.75
Transonic—Mach numbers from 0.75 to 1.20
Supersonic—Mach numbers from 1.20 to 5.00
Hypersonic—Mach numbers above 5.00
While flights in the transonic and supersonic ranges are
common occurrences for military aircraft, civilian jet aircraft
normally operate in a cruise speed range of Mach 0.7 to
Mach 0.90.
The speed of an aircraft in which airflow over any part of
the aircraft or structure under consideration first reaches
(but does not exceed) Mach 1.0 is termed "critical Mach
number" or "Mach Crit." Thus, critical Mach number is
the boundary between subsonic and transonic flight and is
largely dependent on the wing and airfoil design. Critical
Mach number is an important point in transonic flight. When
shock waves form on the aircraft, airflow separation followed
by buffet and aircraft control difficulties can occur. Shock
waves, buffet, and airflow separation take place above critical
Mach number. A jet aircraft typically is most efficient when
cruising at or near its critical Mach number. At speeds 5–10
percent above the critical Mach number, compressibility
effects begin. Drag begins to rise sharply. Associated with
the "drag rise" are buffet, trim, and stability changes and a
decrease in control surface effectiveness. This is the point
of "drag divergence." [Figure 5-65]
VMO/MMO is defined as the maximum operating limit speed.
VMO is expressed in knots calibrated airspeed (KCAS), while

Adherence to these speeds prevents structural problems due
to dynamic pressure or flutter, degradation in aircraft control
response due to compressibility effects (e.g., Mach Tuck,
aileron reversal, or buzz), and separated airflow due to shock
waves resulting in loss of lift or vibration and buffet. Any of
these phenomena could prevent the pilot from being able to
adequately control the aircraft.
For example, an early civilian jet aircraft had a VMO limit of
306 KCAS up to approximately FL 310 (on a standard day).
At this altitude (FL 310), an MMO of 0.82 was approximately
equal to 306 KCAS. Above this altitude, an MMO of 0.82
always equaled a KCAS less than 306 KCAS and, thus,
became the operating limit as you could not reach the VMO
limit without first reaching the MMO limit. For example, at
FL 380, an MMO of 0.82 is equal to 261 KCAS.
Mach Number Versus Airspeed
It is important to understand how airspeed varies with Mach
number. As an example, consider how the stall speed of a
jet transport aircraft varies with an increase in altitude. The
increase in altitude results in a corresponding drop in air
density and outside temperature. Suppose this jet transport
is in the clean configuration (gear and flaps up) and weighs
550,000 pounds. The aircraft might stall at approximately 152
KCAS at sea level. This is equal to (on a standard day) a true
velocity of 152 KTAS and a Mach number of 0.23. At FL 380,
the aircraft will still stall at approximately 152 KCAS, but the
true velocity is about 287 KTAS with a Mach number of 0.50.

CD (Drag coefficient)

M = 0.50

MMO is expressed in Mach number. The VMO limit is usually
associated with operations at lower altitudes and deals with
structural loads and flutter. The MMO limit is associated with
operations at higher altitudes and is usually more concerned
with compressibility effects and flutter. At lower altitudes,
structural loads and flutter are of concern; at higher altitudes,
compressibility effects and flutter are of concern.

Force divergence Mach number

CL= 0.3

Critical Mach number

0.5
M (Mach number)

1.0

Figure 5-65. Critical Mach.

5-45

Although the stalling speed has remained the same for our
purposes, both the Mach number and TAS have increased.
With increasing altitude, the air density has decreased; this
requires a faster true airspeed in order to have the same
pressure sensed by the pitot tube for the same KCAS, or KIAS
(for our purposes, KCAS and KIAS are relatively close to
each other). The dynamic pressure the wing experiences at
FL 380 at 287 KTAS is the same as at sea level at 152 KTAS.
However, it is flying at higher Mach number.
Another factor to consider is the speed of sound. A decrease
in temperature in a gas results in a decrease in the speed of
sound. Thus, as the aircraft climbs in altitude with outside
temperature dropping, the speed of sound is dropping. At
sea level, the speed of sound is approximately 661 KCAS,
while at FL 380 it is 574 KCAS. Thus, for our jet transport
aircraft, the stall speed (in KTAS) has gone from 152 at sea
level to 287 at FL 380. Simultaneously, the speed of sound
(in KCAS) has decreased from 661 to 574 and the Mach
number has increased from 0.23 (152 KTAS divided by 661
KTAS) to 0.50 (287 KTAS divided by 574 KTAS). All the
while, the KCAS for stall has remained constant at 152. This
describes what happens when the aircraft is at a constant
KCAS with increasing altitude, but what happens when the
pilot keeps Mach constant during the climb? In normal jet
flight operations, the climb is at 250 KIAS (or higher (e.g.
heavy)) to 10,000 feet and then at a specified en route climb
airspeed (about 330 if a DC10) until reaching an altitude in
the "mid-twenties" where the pilot then climbs at a constant
Mach number to cruise altitude.
Assuming for illustration purposes that the pilot climbs at a
MMO of 0.82 from sea level up to FL 380. KCAS goes from
543 to 261. The KIAS at each altitude would follow the
same behavior and just differ by a few knots. Recall from
the earlier discussion that the speed of sound is decreasing
with the drop in temperature as the aircraft climbs. The Mach
number is simply the ratio of the true airspeed to the speed
of sound at flight conditions. The significance of this is that
at a constant Mach number climb, the KCAS (and KTAS or
KIAS as well) is falling off.
If the aircraft climbed high enough at this constant MMO
with decreasing KIAS, KCAS, and KTAS, it would begin to
approach its stall speed. At some point, the stall speed of the
aircraft in Mach number could equal the MMO of the aircraft,
and the pilot could neither slow down (without stalling) nor
speed up (without exceeding the max operating speed of the
aircraft). This has been dubbed the "coffin corner."
Boundary Layer
The viscous nature of airflow reduces the local velocities on
a surface and is responsible for skin friction. As discussed
5-46

earlier in the chapter, the layer of air over the wing's surface
that is slowed down or stopped by viscosity is the boundary
layer. There are two different types of boundary layer flow:
laminar and turbulent.

Laminar Boundary Layer Flow
The laminar boundary layer is a very smooth flow, while
the turbulent boundary layer contains swirls or eddies.
The laminar flow creates less skin friction drag than the
turbulent flow but is less stable. Boundary layer flow over a
wing surface begins as a smooth laminar flow. As the flow
continues back from the leading edge, the laminar boundary
layer increases in thickness.

Turbulent Boundary Layer Flow
At some distance back from the leading edge, the smooth
laminar flow breaks down and transitions to a turbulent flow.
From a drag standpoint, it is advisable to have the transition
from laminar to turbulent flow as far aft on the wing as
possible or have a large amount of the wing surface within
the laminar portion of the boundary layer. The low energy
laminar flow, however, tends to break down more suddenly
than the turbulent layer.

Boundary Layer Separation
Another phenomenon associated with viscous flow is
separation. Separation occurs when the airflow breaks away
from an airfoil. The natural progression is from laminar
boundary layer to turbulent boundary layer and then to
airflow separation. Airflow separation produces high drag
and ultimately destroys lift. The boundary layer separation
point moves forward on the wing as the AOA is increased.
[Figure 5-66]
Vortex generators are used to delay or prevent shock wave
induced boundary layer separation encountered in transonic
flight. They are small low aspect ratio airfoils placed at a 12°
to 15° AOA to the airstream. Usually spaced a few inches
apart along the wing ahead of the ailerons or other control
surfaces, vortex generators create a vortex that mixes the
boundary airflow with the high energy airflow just above the
surface. This produces higher surface velocities and increases
the energy of the boundary layer. Thus, a stronger shock wave
is necessary to produce airflow separation.
Shock Waves
When an airplane flies at subsonic speeds, the air ahead is
"warned" of the airplane's coming by a pressure change
transmitted ahead of the airplane at the speed of sound.
Because of this warning, the air begins to move aside before
the airplane arrives and is prepared to let it pass easily. When
the airplane's speed reaches the speed of sound, the pressure

Turbulent boundary layer
Transition region
Laminar boundary layer

Laminar sublayer
Figure 5-66. Boundary layer.

As the airplane's speed increases beyond the speed of sound,
the pressure and density of the compressed air ahead of it
increase, the area of compression extending some distance
ahead of the airplane. At some point in the airstream, the air
particles are completely undisturbed, having had no advanced
warning of the airplane's approach, and in the next instant the
same air particles are forced to undergo sudden and drastic
changes in temperature, pressure, density, and velocity.
The boundary between the undisturbed air and the region
of compressed air is called a shock or "compression" wave.
This same type of wave is formed whenever a supersonic
airstream is slowed to subsonic without a change in direction,
such as when the airstream is accelerated to sonic speed
over the cambered portion of a wing, and then decelerated
to subsonic speed as the area of maximum camber is passed.
A shock wave forms as a boundary between the supersonic
and subsonic ranges.

part of the velocity energy of the airstream is converted to
heat as it flows through the wave, is a contributing factor
in the drag increase, but the drag resulting from airflow
separation is much greater. If the shock wave is strong,
the boundary layer may not have sufficient kinetic energy
to withstand airflow separation. The drag incurred in the
transonic region due to shock wave formation and airflow
separation is known as "wave drag." When speed exceeds
the critical Mach number by about 10 percent, wave drag
increases sharply. A considerable increase in thrust (power)
is required to increase flight speed beyond this point into the
supersonic range where, depending on the airfoil shape and
the AOA, the boundary layer may reattach.
Normal shock waves form on the wing's upper surface and
form an additional area of supersonic flow and a normal shock
wave on the lower surface. As flight speed approaches the
speed of sound, the areas of supersonic flow enlarge and the
shock waves move nearer the trailing edge. [Figure 5-67]

S

change can no longer warn the air ahead because the airplane
is keeping up with its own pressure waves. Rather, the air
particles pile up in front of the airplane causing a sharp
decrease in the flow velocity directly in front of the airplane
with a corresponding increase in air pressure and density.

o
ers
up

low
nic f

Normal shock wave

M = 0.82

Whenever a shock wave forms perpendicular to the airflow, it
is termed a "normal" shock wave, and the flow immediately
behind the wave is subsonic. A supersonic airstream passing
through a normal shock wave experiences these changes:


The airstream is slowed to subsonic.



The airflow immediately behind the shock wave does
not change direction.



The static pressure and density of the airstream behind
the wave is greatly increased.

 	 The energy of the airstream (indicated by total
pressure—dynamic plus static) is greatly reduced.
Shock wave formation causes an increase in drag. One of
the principal effects of a shock wave is the formation of a
dense high pressure region immediately behind the wave.
The instability of the high pressure region, and the fact that

Normal shock wave

S

ers
up

flow
onic

M = 0.95

Bow wave
M = 1.05

Subsonic airflow

Figure 5-67. Shock waves.

5-47

Associated with "drag rise" are buffet (known as Mach
buffet), trim, and stability changes and a decrease in control
force effectiveness. The loss of lift due to airflow separation
results in a loss of downwash and a change in the position of
the center pressure on the wing. Airflow separation produces
a turbulent wake behind the wing, which causes the tail
surfaces to buffet (vibrate). The nose-up and nose-down pitch
control provided by the horizontal tail is dependent on the
downwash behind the wing. Thus, an increase in downwash
decreases the horizontal tail's pitch control effectiveness
since it effectively increases the AOA that the tail surface is
seeing. Movement of the wing CP affects the wing pitching
moment. If the CP moves aft, a diving moment referred to
as "Mach tuck" or "tuck under" is produced, and if it moves
forward, a nose-up moment is produced. This is the primary
reason for the development of the T-tail configuration on
many turbine-powered aircraft, which places the horizontal
stabilizer as far as practical from the turbulence of the wings.
Sweepback
Most of the difficulties of transonic flight are associated with
shock wave induced flow separation. Therefore, any means of
delaying or alleviating the shock induced separation improves
aerodynamic performance. One method is wing sweepback.
Sweepback theory is based upon the concept that it is only the
component of the airflow perpendicular to the leading edge
of the wing that affects pressure distribution and formation
of shock waves. [Figure 5-68]
On a straight wing aircraft, the airflow strikes the wing
leading edge at 90°, and its full impact produces pressure and
lift. A wing with sweepback is struck by the same airflow at
an angle smaller than 90°. This airflow on the swept wing has
the effect of persuading the wing into believing that it is flying
slower than it really is; thus the formation of shock waves is
delayed. Advantages of wing sweep include an increase in
critical Mach number, force divergence Mach number, and
the Mach number at which drag rise peaks. In other words,
sweep delays the onset of compressibility effects.
The Mach number that produces a sharp change in coefficient
of drag is termed the "force divergence" Mach number and,
for most airfoils, usually exceeds the critical Mach number by
5 to 10 percent. At this speed, the airflow separation induced
by shock wave formation can create significant variations in
the drag, lift, or pitching moment coefficients. In addition to
the delay of the onset of compressibility effects, sweepback
reduces the magnitude in the changes of drag, lift, or moment
coefficients. In other words, the use of sweepback "softens"
the force divergence.
A disadvantage of swept wings is that they tend to stall at the
wingtips rather than at the wing roots. [Figure 5-69] This is
5-48

Spanwise flow

Airspeed sensed
by wing Mach 0.70
True airspeed
Mach 0.85

Figure 5-68. Sweepback effect.

because the boundary layer tends to flow spanwise toward
the tips and to separate near the leading edges. Because the
tips of a swept wing are on the aft part of the wing (behind
the CL), a wingtip stall causes the CL to move forward on
the wing, forcing the nose to rise further. The tendency for
tip stall is greatest when wing sweep and taper are combined.
The stall situation can be aggravated by a T-tail configuration,
which affords little or no pre-stall warning in the form of tail
control surface buffet. [Figure 5-70] The T-tail, being above
the wing wake remains effective even after the wing has begun
to stall, allowing the pilot to inadvertently drive the wing
into a deeper stall at a much greater AOA. If the horizontal
tail surfaces then become buried in the wing's wake, the
elevator may lose all effectiveness, making it impossible to
reduce pitch attitude and break the stall. In the pre-stall and
immediate post-stall regimes, the lift/drag qualities of a swept
wing aircraft (specifically the enormous increase in drag
at low speeds) can cause an increasingly descending flight
path with no change in pitch attitude, further increasing the

to cause a low-speed Mach buffet. This very high AOA has
the effect of increasing airflow velocity over the upper surface
of the wing until the same effects of the shock waves and
buffet occur as in the high-speed buffet situation. The AOA
of the wing has the greatest effect on inducing the Mach
buffet at either the high-speed or low-speed boundaries for
the aircraft. The conditions that increase the AOA, the speed
of the airflow over the wing, and chances of Mach buffet are:
Pre-stall
Prestall
Figure 5-69. Wingtip pre-stall.

Stalled
Stalled
Figure 5-70. T-tail stall.

AOA. In this situation, without reliable AOA information,
a nose-down pitch attitude with an increasing airspeed is no
guarantee that recovery has been affected, and up-elevator
movement at this stage may merely keep the aircraft stalled.
It is a characteristic of T-tail aircraft to pitch up viciously
when stalled in extreme nose-high attitudes, making
recovery difficult or violent. The stick pusher inhibits this
type of stall. At approximately one knot above stall speed,
pre-programmed stick forces automatically move the stick
forward, preventing the stall from developing. A G-limiter
may also be incorporated into the system to prevent the pitch
down generated by the stick pusher from imposing excessive
loads on the aircraft. A "stick shaker," on the other hand,
provides stall warning when the airspeed is five to seven
percent above stall speed.
Mach Buffet Boundaries
Mach buffet is a function of the speed of the airflow over the
wing—not necessarily the speed of the aircraft. Any time that
too great a lift demand is made on the wing, whether from too
fast an airspeed or from too high an AOA near the MMO, the
"high-speed" buffet occurs. There are also occasions when
the buffet can be experienced at much lower speeds known
as the "low-speed Mach buffet."
An aircraft flown at a speed too slow for its weight and
altitude necessitating a high AOA is the most likely situation



High altitudes—the higher an aircraft flies, the thinner
the air and the greater the AOA required to produce
the lift needed to maintain level flight.



Heavy weights—the heavier the aircraft, the greater
the lift required of the wing, and all other factors being
equal, the greater the AOA.



G loading—an increase in the G loading on the aircraft
has the same effect as increasing the weight of the
aircraft. Whether the increase in G forces is caused
by turns, rough control usage, or turbulence, the effect
of increasing the wing's AOA is the same.

High Speed Flight Controls
On high-speed aircraft, flight controls are divided into
primary flight controls and secondary or auxiliary flight
controls. The primary flight controls maneuver the aircraft
about the pitch, roll, and yaw axes. They include the ailerons,
elevator, and rudder. Secondary or auxiliary flight controls
include tabs, leading edge flaps, trailing edge flaps, spoilers,
and slats.
Spoilers are used on the upper surface of the wing to spoil or
reduce lift. High speed aircraft, due to their clean low drag
design, use spoilers as speed brakes to slow them down.
Spoilers are extended immediately after touchdown to dump
lift and thus transfer the weight of the aircraft from the wings
onto the wheels for better braking performance. [Figure 5-71]
Jet transport aircraft have small ailerons. The space for
ailerons is limited because as much of the wing trailing
edge as possible is needed for flaps. Also, a conventional
size aileron would cause wing twist at high speed. For that
reason, spoilers are used in unison with ailerons to provide
additional roll control.
Some jet transports have two sets of ailerons, a pair of outboard
low-speed ailerons and a pair of high-speed inboard ailerons.
When the flaps are fully retracted after takeoff, the outboard
ailerons are automatically locked out in the faired position.
When used for roll control, the spoiler on the side of the
up-going aileron extends and reduces the lift on that side,
causing the wing to drop. If the spoilers are extended as speed
brakes, they can still be used for roll control. If they are the
5-49

Rudder

Inboard wing

Elevator
Tab
Stabilizer

Outboard wing
Foreflap
Midflap

Inboard flap

Aftflap

Flight spoilers
Ground spoiler
Outboard flap

Aileron

Tab

Landing Flaps

Aileron
Leading edge slats
Leading edge flaps

737 Control Surfaces
Control tab

Inboard wing

Outboard wing
Stabilizer

Inboard aileron

Elevator

Leading edge flaps

Control tab

Aileron

Upper rudder

Takeoff Flaps

Anti-balance tabs
Lower rudder
Vortex generators
Pitot tubes
Ground spoilers
Inboard flaps
Flight spoilers
Outboard flap
Balance tab
Outboard aileron

Foreflap

Midflap

Aftflap

Inboard wing

Outboard wing

Leading edge slat
Leading edge flap

Slats
Fence

727 Control Surfaces

Aileron

Flaps Retracted

Figure 5-71. Control surfaces.

differential type, they extend further on one side and retract
on the other side. If they are the non-differential type, they
extend further on one side but do not retract on the other side.
When fully extended as speed brakes, the non-differential
spoilers remain extended and do not supplement the ailerons.
To obtain a smooth stall and a higher AOA without airflow
separation, the wing's leading edge should have a wellrounded almost blunt shape that the airflow can adhere to
at the higher AOA. With this shape, the airflow separation
starts at the trailing edge and progresses forward gradually
as AOA is increased.
The pointed leading edge necessary for high-speed flight
results in an abrupt stall and restricts the use of trailing edge
flaps because the airflow cannot follow the sharp curve
5-50

around the wing leading edge. The airflow tends to tear loose
rather suddenly from the upper surface at a moderate AOA.
To utilize trailing edge flaps, and thus increase the CL-MAX,
the wing must go to a higher AOA without airflow separation.
Therefore, leading edge slots, slats, and flaps are used to
improve the low-speed characteristics during takeoff, climb,
and landing. Although these devices are not as powerful as
trailing edge flaps, they are effective when used full span in
combination with high-lift trailing edge flaps. With the aid
of these sophisticated high-lift devices, airflow separation is
delayed and the CL-MAX is increased considerably. In fact, a
50 knot reduction in stall speed is not uncommon.

The operational requirements of a large jet transport aircraft
necessitate large pitch trim changes. Some requirements are:


A large CG range



A large speed range



The ability to perform large trim changes due to
wing leading edge and trailing edge high-lift devices
without limiting the amount of elevator remaining



Maintaining trim drag to a minimum

Chapter Summary
In order to sustain an aircraft in flight, a pilot must understand
how thrust, drag, lift, and weight act on the aircraft. By
understanding the aerodynamics of flight, how design,
weight, load factors, and gravity affect an aircraft during
flight maneuvers from stalls to high speed flight, the pilot
learns how to control the balance between these forces. For
information on stall speeds, load factors, and other important
aircraft data, always consult the AFM/POH for specific
information pertaining to the aircraft being flown.

These requirements are met by the use of a variable incidence
horizontal stabilizer. Large trim changes on a fixed-tail
aircraft require large elevator deflections. At these large
deflections, little further elevator movement remains in the
same direction. A variable incidence horizontal stabilizer
is designed to take out the trim changes. The stabilizer is
larger than the elevator, and consequently does not need to
be moved through as large an angle. This leaves the elevator
streamlining the tail plane with a full range of movement up
and down. The variable incidence horizontal stabilizer can
be set to handle the bulk of the pitch control demand, with
the elevator handling the rest. On aircraft equipped with a
variable incidence horizontal stabilizer, the elevator is smaller
and less effective in isolation than it is on a fixed-tail aircraft.
In comparison to other flight controls, the variable incidence
horizontal stabilizer is enormously powerful in its effect.
Because of the size and high speeds of jet transport aircraft,
the forces required to move the control surfaces can be beyond
the strength of the pilot. Consequently, the control surfaces are
actuated by hydraulic or electrical power units. Moving the
controls in the flight deck signals the control angle required,
and the power unit positions the actual control surface. In the
event of complete power unit failure, movement of the control
surface can be affected by manually controlling the control
tabs. Moving the control tab upsets the aerodynamic balance,
which causes the control surface to move.

5-51

Chapter 6

Flight Controls
Introduction
This chapter focuses on the flight control systems a pilot uses
to control the forces of flight and the aircraft's direction and
attitude. It should be noted that flight control systems and
characteristics can vary greatly depending on the type of
aircraft flown. The most basic flight control system designs
are mechanical and date back to early aircraft. They operate
with a collection of mechanical parts, such as rods, cables,
pulleys, and sometimes chains to transmit the forces of the
flight deck controls to the control surfaces. Mechanical flight
control systems are still used today in small general and
sport category aircraft where the aerodynamic forces are not
excessive. [Figure 6-1]

6-1

of this project is to develop an adaptive neural network-based
flight control system. Applied directly to flight control system
feedback errors, IFCS provides adjustments to improve
aircraft performance in normal flight, as well as with system
failures. With IFCS, a pilot is able to maintain control and
safely land an aircraft that has suffered a failure to a control
surface or damage to the airframe. It also improves mission
capability, increases the reliability and safety of flight, and
eases the pilot workload.

Control stick
Elevator
Pulleys

Push rod

Figure 6-1. Mechanical flight control system.

As aviation matured and aircraft designers learned more about
aerodynamics, the industry produced larger and faster aircraft.
Therefore, the aerodynamic forces acting upon the control
surfaces increased exponentially. To make the control force
required by pilots manageable, aircraft engineers designed
more complex systems. At first, hydromechanical designs,
consisting of a mechanical circuit and a hydraulic circuit,
were used to reduce the complexity, weight, and limitations
of mechanical flight controls systems. [Figure 6-2]
As aircraft became more sophisticated, the control surfaces
were actuated by electric motors, digital computers, or fiber
optic cables. Called "fly-by-wire," this flight control system
replaces the physical connection between pilot controls and
the flight control surfaces with an electrical interface. In
addition, in some large and fast aircraft, controls are boosted
by hydraulically or electrically actuated systems. In both
the fly-by-wire and boosted controls, the feel of the control
reaction is fed back to the pilot by simulated means.
Current research at the National Aeronautics and Space
Administration (NASA) Dryden Flight Research Center
involves Intelligent Flight Control Systems (IFCS). The goal

Today's aircraft employ a variety of flight control systems.
For example, some aircraft in the sport pilot category rely on
weight-shift control to fly while balloons use a standard burn
technique. Helicopters utilize a cyclic to tilt the rotor in the
desired direction along with a collective to manipulate rotor
pitch and anti-torque pedals to control yaw. [Figure 6-3]
For additional information on flight control systems, refer
to the appropriate handbook for information related to the
flight control systems and characteristics of specific types
of aircraft.

Flight Control Systems
Flight Controls
Aircraft flight control systems consist of primary and
secondary systems. The ailerons, elevator (or stabilator),
and rudder constitute the primary control system and are
required to control an aircraft safely during flight. Wing flaps,
leading edge devices, spoilers, and trim systems constitute
the secondary control system and improve the performance
characteristics of the airplane or relieve the pilot of excessive
control forces.
Primary Flight Controls
Aircraft control systems are carefully designed to provide
adequate responsiveness to control inputs while allowing a

Yaw

Neutral
Cyclic stick

LEGEND

c

Control stick (AFT—nose up)

Cycli

Cable

Cycli

ve

Collecti

c

Hydraulic pressure
Hydraulic return
Pivot point

Elevator (UP)

Yaw
Control valves

Control cables

Cy

clic

clic

Cy

e

ectiv

Coll

Neutral
Power disconnect linkage

Neutral

Power cylinder

Figure 6-2. Hydromechanical flight control system.

6-2

Anti-torque pedals

Collective lever

Figure 6-3. Helicopter flight control system.

natural feel. At low airspeeds, the controls usually feel soft
and sluggish, and the aircraft responds slowly to control
applications. At higher airspeeds, the controls become
increasingly firm and aircraft response is more rapid.
Movement of any of the three primary flight control surfaces
(ailerons, elevator or stabilator, or rudder), changes the
airflow and pressure distribution over and around the airfoil.
These changes affect the lift and drag produced by the airfoil/
control surface combination, and allow a pilot to control the
aircraft about its three axes of rotation.
Design features limit the amount of deflection of flight
control surfaces. For example, control-stop mechanisms may
be incorporated into the flight control linkages, or movement
of the control column and/or rudder pedals may be limited.
The purpose of these design limits is to prevent the pilot from
inadvertently overcontrolling and overstressing the aircraft
during normal maneuvers.
A properly designed aircraft is stable and easily controlled
during normal maneuvering. Control surface inputs cause
movement about the three axes of rotation. The types of
stability an aircraft exhibits also relate to the three axes of
rotation. [Figure 6-4]
El

ev

Rudder—Yaw

at

or

La
—
(lo tera Pit
n
l
sta gi ax ch
bil tud is
ity ina
) l

Vertical axis
(directional
stability)

Roll
on—
Ailer
al
itudin
Long(lateral
axis ity)
stabil

Ailerons
Ailerons control roll about the longitudinal axis. The ailerons
are attached to the outboard trailing edge of each wing and
move in the opposite direction from each other. Ailerons are
connected by cables, bellcranks, pulleys, and/or push-pull
tubes to a control wheel or control stick.
Moving the control wheel, or control stick, to the right
causes the right aileron to deflect upward and the left aileron
to deflect downward. The upward deflection of the right
aileron decreases the camber resulting in decreased lift on
the right wing. The corresponding downward deflection of
the left aileron increases the camber resulting in increased
lift on the left wing. Thus, the increased lift on the left wing
and the decreased lift on the right wing causes the aircraft
to roll to the right.

Adverse Yaw
Since the downward deflected aileron produces more lift as
evidenced by the wing raising, it also produces more drag.
This added drag causes the wing to slow down slightly.
This results in the aircraft yawing toward the wing which
had experienced an increase in lift (and drag). From the
pilot's perspective, the yaw is opposite the direction of the
bank. The adverse yaw is a result of differential drag and the
slight difference in the velocity of the left and right wings.
[Figure 6-5]
Adverse yaw becomes more pronounced at low airspeeds.
At these slower airspeeds, aerodynamic pressure on control
surfaces are low, and larger control inputs are required to

Lift
Drag

Airplane
Movement

Axes of
Rotation

Type of
Stability

Aileron

Roll

Longitudinal

Lateral

Elevator/
Stabilator

Pitch

Lateral

Longitudinal

Rudder

Yaw

Vertical

Directional

Figure 6-4. Airplane controls, movement, axes of rotation, and

A

dv

Lift

Primary
Control
Surface

g

Dra

e r s e yaw

Figure 6-5. Adverse yaw is caused by higher drag on the outside
wing that is producing more lift.

type of stability.

6-3

effectively maneuver the aircraft. As a result, the increase
in aileron deflection causes an increase in adverse yaw. The
yaw is especially evident in aircraft with long wing spans.
Application of the rudder is used to counteract adverse
yaw. The amount of rudder control required is greatest at
low airspeeds, high angles of attack, and with large aileron
deflections. Like all control surfaces at lower airspeeds,
the vertical stabilizer/rudder becomes less effective and
magnifies the control problems associated with adverse yaw.
All turns are coordinated by use of ailerons, rudder, and
elevator. Applying aileron pressure is necessary to place
the aircraft in the desired angle of bank, while simultaneous
application of rudder pressure is necessary to counteract the
resultant adverse yaw. Additionally, because more lift is
required during a turn than during straight-and-level flight,
the angle of attack (AOA) must be increased by applying
elevator back pressure. The steeper the turn, the more elevator
back pressure that is needed.
As the desired angle of bank is established, aileron and
rudder pressures should be relaxed. This stops the angle of
bank from increasing, because the aileron and rudder control
surfaces are in a neutral and streamlined position. Elevator
back pressure should be held constant to maintain altitude.
The roll-out from a turn is similar to the roll-in, except the
flight controls are applied in the opposite direction. The
aileron and rudder are applied in the direction of the roll-out
or toward the high wing. As the angle of bank decreases,
the elevator back pressure should be relaxed as necessary
to maintain altitude.

Aileron deflecte
d up

Differential aileron

Aileron deflected down

Figure 6-6. Differential ailerons.

the drag created by the lowered aileron on the opposite wing
and reduces adverse yaw. [Figure 6-7]
The frise-type aileron also forms a slot so air flows smoothly
over the lowered aileron, making it more effective at high
angles of attack. Frise-type ailerons may also be designed
to function differentially. Like the differential aileron, the
frise-type aileron does not eliminate adverse yaw entirely.
Coordinated rudder application is still needed when ailerons
are applied.
Coupled Ailerons and Rudder
Coupled ailerons and rudder are linked controls. This is
accomplished with rudder-aileron interconnect springs, which
help correct for aileron drag by automatically deflecting
the rudder at the same time the ailerons are deflected. For
Neutral

In an attempt to reduce the effects of adverse yaw,
manufacturers have engineered four systems: differential
ailerons, frise-type ailerons, coupled ailerons and rudder,
and flaperons.
Differential Ailerons
With differential ailerons, one aileron is raised a greater
distance than the other aileron and is lowered for a given
movement of the control wheel or control stick. This produces
an increase in drag on the descending wing. The greater drag
results from deflecting the up aileron on the descending wing
to a greater angle than the down aileron on the rising wing.
While adverse yaw is reduced, it is not eliminated completely.
[Figure 6-6]
Frise-Type Ailerons
With a frise-type aileron, when pressure is applied to the
control wheel, or control stick, the aileron that is being raised
pivots on an offset hinge. This projects the leading edge of
the aileron into the airflow and creates drag. It helps equalize
6-4

Raised

Drag
Lowered

Figure ailerons.
5-4. Frise-type ailerons.
Figure 6-7. Frise-type

example, when the control wheel, or control stick, is moved
to produce a left roll, the interconnect cable and spring pulls
forward on the left rudder pedal just enough to prevent the
nose of the aircraft from yawing to the right. The force applied
to the rudder by the springs can be overridden if it becomes
necessary to slip the aircraft. [Figure 6-8]
Flaperons
Flaperons combine both aspects of flaps and ailerons. In
addition to controlling the bank angle of an aircraft like
conventional ailerons, flaperons can be lowered together
to function much the same as a dedicated set of flaps. The
pilot retains separate controls for ailerons and flaps. A mixer
is used to combine the separate pilot inputs into this single
set of control surfaces called flaperons. Many designs that
incorporate flaperons mount the control surfaces away from
the wing to provide undisturbed airflow at high angles of
attack and/or low airspeeds. [Figure 6-9]

Elevator
The elevator controls pitch about the lateral axis. Like the
ailerons on small aircraft, the elevator is connected to the
control column in the flight deck by a series of mechanical

Rudder deflects with ailerons

Figure 6-9. Flaperons on a Skystar Kitfox MK 7.

linkages. Aft movement of the control column deflects
the trailing edge of the elevator surface up. This is usually
referred to as the up-elevator position. [Figure 6-10]
The up-elevator position decreases the camber of the elevator
and creates a downward aerodynamic force, which is greater
than the normal tail-down force that exists in straight-and­
level flight. The overall effect causes the tail of the aircraft
to move down and the nose to pitch up. The pitching moment
occurs about the center of gravity (CG). The strength of the
pitching moment is determined by the distance between
the CG and the horizontal tail surface, as well as by the
aerodynamic effectiveness of the horizontal tail surface.
Moving the control column forward has the opposite effect.
In this case, elevator camber increases, creating more lift
(less tail-down force) on the horizontal stabilizer/elevator.
This moves the tail upward and pitches the nose down. Again,
the pitching moment occurs about the CG.

Control column
aft
Up elevator
Nose up

CG

Tail down

As mentioned earlier, stability, power, thrustline, and the
position of the horizontal tail surfaces on the empennage
are factors in elevator effectiveness controlling pitch. For

Downward
aerodynamic force
Rudder/Aileron interconnecting springs

Figure 6-8. Coupled ailerons and rudder.

Figure 6-10. The elevator is the primary control for changing the

pitch attitude of an aircraft.

6-5

example, the horizontal tail surfaces may be attached near
the lower part of the vertical stabilizer, at the midpoint, or
at the high point, as in the T-tail design.

T-Tail
In a T-tail configuration, the elevator is above most of the
effects of downwash from the propeller, as well as airflow
around the fuselage and/or wings during normal flight
conditions. Operation of the elevators in this undisturbed air
allows control movements that are consistent throughout most
flight regimes. T-tail designs have become popular on many
light and large aircraft, especially those with aft fuselagemounted engines because the T-tail configuration removes
the tail from the exhaust blast of the engines. Seaplanes and
amphibians often have T-tails in order to keep the horizontal
surfaces as far from the water as possible. An additional
benefit is reduced noise and vibration inside the aircraft.
In comparison with conventional-tail aircraft, the elevator on a
T-tail aircraft must be moved a greater distance to raise the nose
a given amount when traveling at slow speeds. This is because
the conventional-tail aircraft has the downwash from the
propeller pushing down on the tail to assist in raising the nose.
Aircraft controls are rigged so that an increase in control force
is required to increase control travel. The forces required to
raise the nose of a T-tail aircraft are greater than the forces
required to raise the nose of a conventional-tail aircraft.
Longitudinal stability of a trimmed aircraft is the same for
both types of configuration, but the pilot must be aware that
the required control forces are greater at slow speeds during
takeoffs, landings, or stalls than for similar size aircraft
equipped with conventional tails.
T-tail aircraft also require additional design considerations
to counter the problem of flutter. Since the weight of the
horizontal surfaces is at the top of the vertical stabilizer, the
moment arm created causes high loads on the vertical stabilizer
that can result in flutter. Engineers must compensate for this by
increasing the design stiffness of the vertical stabilizer, usually
resulting in a weight penalty over conventional tail designs.
When flying at a very high AOA with a low airspeed and
an aft CG, the T-tail aircraft may be more susceptible to a
deep stall. In this condition, the wake of the wing impinges
on the tail surface and renders it almost ineffective. The
wing, if fully stalled, allows its airflow to separate right after
the leading edge. The wide wake of decelerated, turbulent
air blankets the horizontal tail and hence its effectiveness
diminished significantly. In these circumstances, elevator or
stabilator control is reduced (or perhaps eliminated) making
it difficult to recover from the stall. It should be noted that an
aft CG is often a contributing factor in these incidents, since
6-6

similar recovery problems are also found with conventional
tail aircraft with an aft CG. [Figure 6-11] Deep stalls can
occur on any aircraft but are more likely to occur on aircraft
with "T" tails as a high AOA may be more likely to place
the wings separated airflow into the path of the horizontal
surface of the tail. Additionally, the distance between the
wings and the tail, the position of the engines (such as being
mounted on the tail) may increase the susceptibility of deep
stall events. Therefore a deep stall may be more prevalent
on transport versus general aviation aircraft.
Since flight at a high AOA with a low airspeed and an aft
CG position can be dangerous, many aircraft have systems to
compensate for this situation. The systems range from control
stops to elevator down springs. On transport category jets, stick
pushers are commonly used. An elevator down spring assists in
lowering the nose of the aircraft to prevent a stall caused by the
aft CG position. The stall occurs because the properly trimmed
airplane is flying with the elevator in a trailing edge down
position, forcing the tail up and the nose down. In this unstable
condition, if the aircraft encounters turbulence and slows down
further, the trim tab no longer positions the elevator in the nosedown position. The elevator then streamlines, and the nose of
the aircraft pitches upward, possibly resulting in a stall.
The elevator down spring produces a mechanical load on the
elevator, causing it to move toward the nose-down position if not
otherwise balanced. The elevator trim tab balances the elevator
down spring to position the elevator in a trimmed position.
When the trim tab becomes ineffective, the down spring drives
the elevator to a nose-down position. The nose of the aircraft
lowers, speed builds up, and a stall is prevented. [Figure 6-12]
The elevator must also have sufficient authority to hold the
nose of the aircraft up during the roundout for a landing. In
this case, a forward CG may cause a problem. During the
landing flare, power is usually reduced, which decreases the

CG

Figure 6-11. Aircraft with a T-tail design at a high AOA and an aft CG.

Antiservo tab

Down spring

Stabilator
Pivot points
Elevator

Balance weight

Pivot point

Bell crank
Figure 6-13. The stabilator is a one-piece horizontal tail surface
Figure 6-12. When the aerodynamic efficiency of the horizontal tail

surface is inadequate due to an aft CG condition, an elevator down
spring may be used to supply a mechanical load to lower the nose.

airflow over the empennage. This, coupled with the reduced
landing speed, makes the elevator less effective.
As this discussion demonstrates, pilots must understand and
follow proper loading procedures, particularly with regard
to the CG position. More information on aircraft loading,
as well as weight and balance, is included in Chapter 10,
Weight and Balance.

Stabilator
As mentioned in Chapter 3, Aircraft Structure, a stabilator is
essentially a one-piece horizontal stabilizer that pivots from
a central hinge point. When the control column is pulled
back, it raises the stabilator's trailing edge, pulling the nose
of the aircraft. Pushing the control column forward lowers
the trailing edge of the stabilator and pitches the nose of the
aircraft down.

that pivots up and down about a central hinge point.

of the main wings. In effect, the canard is an airfoil similar to
the horizontal surface on a conventional aft-tail design. The
difference is that the canard actually creates lift and holds
the nose up, as opposed to the aft-tail design which exerts
downward force on the tail to prevent the nose from rotating
downward. [Figure 6-14]
The canard design dates back to the pioneer days of aviation.
Most notably, it was used on the Wright Flyer. Recently, the
canard configuration has regained popularity and is appearing
on newer aircraft. Canard designs include two types–one with
a horizontal surface of about the same size as a normal aft-tail
design, and the other with a surface of the same approximate
size and airfoil of the aft-mounted wing known as a tandem
wing configuration. Theoretically, the canard is considered
more efficient because using the horizontal surface to help
lift the weight of the aircraft should result in less drag for a
given amount of lift.

Because stabilators pivot around a central hinge point, they
are extremely sensitive to control inputs and aerodynamic
loads. Antiservo tabs are incorporated on the trailing edge to
decrease sensitivity. They deflect in the same direction as the
stabilator. This results in an increase in the force required to
move the stabilator, thus making it less prone to pilot-induced
overcontrolling. In addition, a balance weight is usually
incorporated in front of the main spar. The balance weight
may project into the empennage or may be incorporated on
the forward portion of the stabilator tips. [Figure 6-13]

Canard
The canard design utilizes the concept of two lifting surfaces.
The canard functions as a horizontal stabilizer located in front

Figure 6-14. The Piaggio P180 includes a variable-sweep canard
design that provides longitudinal stability about the lateral axis.

6-7

Rudder
The rudder controls movement of the aircraft about its vertical
axis. This motion is called yaw. Like the other primary control
surfaces, the rudder is a movable surface hinged to a fixed
surface in this case, to the vertical stabilizer or fin. The rudder
is controlled by the left and right rudder pedals.
When the rudder is deflected into the airflow, a horizontal
force is exerted in the opposite direction. [Figure 6-15] By
pushing the left pedal, the rudder moves left. This alters the
airflow around the vertical stabilizer/rudder and creates a
sideward lift that moves the tail to the right and yaws the nose
of the airplane to the left. Rudder effectiveness increases with
speed; therefore, large deflections at low speeds and small
deflections at high speeds may be required to provide the
desired reaction. In propeller-driven aircraft, any slipstream
flowing over the rudder increases its effectiveness.

V-Tail
The V-tail design utilizes two slanted tail surfaces to perform
the same functions as the surfaces of a conventional elevator
and rudder configuration. The fixed surfaces act as both
horizontal and vertical stabilizers. [Figure 6-16]
The movable surfaces, which are usually called ruddervators,
are connected through a special linkage that allows the control
wheel to move both surfaces simultaneously. On the other
hand, displacement of the rudder pedals moves the surfaces
differentially, thereby providing directional control.
When both rudder and elevator controls are moved by the
pilot, a control mixing mechanism moves each surface the

Y aw

Left rudder forward

CG

Figure 6-16. Beechcraft Bonanza V35.

appropriate amount. The control system for the V-tail is more
complex than the control system for a conventional tail. In
addition, the V-tail design is more susceptible to Dutch roll
tendencies than a conventional tail, and total reduction in
drag is minimal.
Secondary Flight Controls
Secondary flight control systems may consist of wing flaps,
leading edge devices, spoilers, and trim systems.

Flaps
Flaps are the most common high-lift devices used on aircraft.
These surfaces, which are attached to the trailing edge of
the wing, increase both lift and induced drag for any given
AOA. Flaps allow a compromise between high cruising
speed and low landing speed because they may be extended
when needed and retracted into the wing's structure when not
needed. There are four common types of flaps: plain, split,
slotted, and Fowler flaps. [Figure 6-17]
The plain flap is the simplest of the four types. It increases
the airfoil camber, resulting in a significant increase in the
coefficient of lift (CL) at a given AOA. At the same time, it
greatly increases drag and moves the center of pressure (CP)
aft on the airfoil, resulting in a nose-down pitching moment.
The split flap is deflected from the lower surface of the airfoil
and produces a slightly greater increase in lift than the plain
flap. More drag is created because of the turbulent air pattern
produced behind the airfoil. When fully extended, both plain
and split flaps produce high drag with little additional lift.

Left rudder

Ae

rody
namic force

Figure 6-15. The effect of left rudder pressure.

6-8

The most popular flap on aircraft today is the slotted flap.
Variations of this design are used for small aircraft, as well
as for large ones. Slotted flaps increase the lift coefficient
significantly more than plain or split flaps. On small aircraft,
the hinge is located below the lower surface of the flap, and

when the flap is lowered, a duct forms between the flap well
in the wing and the leading edge of the flap. When the slotted
flap is lowered, high energy air from the lower surface is
ducted to the flap's upper surface. The high energy air from
the slot accelerates the upper surface boundary layer and
delays airflow separation, providing a higher CL. Thus, the
slotted flap produces much greater increases in maximum
coefficient of lift (CL-MAX) than the plain or split flap. While
there are many types of slotted flaps, large aircraft often
have double- and even triple-slotted flaps. These allow the
maximum increase in drag without the airflow over the flaps
separating and destroying the lift they produce.
Basic section

Plain flap

Fowler flaps are a type of slotted flap. This flap design not
only changes the camber of the wing, it also increases the
wing area. Instead of rotating down on a hinge, it slides
backwards on tracks. In the first portion of its extension, it
increases the drag very little, but increases the lift a great
deal as it increases both the area and camber. Pilots should
be aware that flap extension may cause a nose-up or down
pitching moment, depending on the type of aircraft, which
the pilot will need to compensate for, usually with a trim
adjustment. As the extension continues, the flap deflects
downward. During the last portion of its travel, the flap
increases the drag with little additional increase in lift.

Leading Edge Devices
High-lift devices also can be applied to the leading edge of
the airfoil. The most common types are fixed slots, movable
slats, leading edge flaps, and cuffs. [Figure 6-18]
Fixed slots direct airflow to the upper wing surface and delay
airflow separation at higher angles of attack. The slot does not
Fixed slot

Split flap
Movable slot

Slotted flap

Leading edge flap

Fowler flap

Leading edge cuff
Slotted Fowler flap

Figure 6-17. Five common types of flaps.

Figure 6-18. Leading edge high lift devices.

6-9

increase the wing camber, but allows a higher maximum CL
because the stall is delayed until the wing reaches a greater AOA.
Movable slats consist of leading edge segments that move on
tracks. At low angles of attack, each slat is held flush against
the wing's leading edge by the high pressure that forms at
the wing's leading edge. As the AOA increases, the highpressure area moves aft below the lower surface of the wing,
allowing the slats to move forward. Some slats, however, are
pilot operated and can be deployed at any AOA. Opening a
slat allows the air below the wing to flow over the wing's
upper surface, delaying airflow separation.
Leading edge flaps, like trailing edge flaps, are used to
increase both CL-MAX and the camber of the wings. This type
of leading edge device is frequently used in conjunction with
trailing edge flaps and can reduce the nose-down pitching
movement produced by the latter. As is true with trailing edge
flaps, a small increment of leading edge flaps increases lift
to a much greater extent than drag. As flaps are extended,
drag increases at a greater rate than lift.
Leading edge cuffs, like leading edge flaps and trailing edge
flaps are used to increase both CL-MAX and the camber of
the wings. Unlike leading edge flaps and trailing edge flaps,
leading edge cuffs are fixed aerodynamic devices. In most
cases, leading edge cuffs extend the leading edge down and
forward. This causes the airflow to attach better to the upper
surface of the wing at higher angles of attack, thus lowering
an aircraft's stall speed. The fixed nature of leading edge cuffs
extracts a penalty in maximum cruise airspeed, but recent
advances in design and technology have reduced this penalty.

Spoilers
Found on some fixed-wing aircraft, high drag devices called
spoilers are deployed from the wings to spoil the smooth
airflow, reducing lift and increasing drag. On gliders, spoilers
are most often used to control rate of descent for accurate
landings. On other aircraft, spoilers are often used for roll
control, an advantage of which is the elimination of adverse
yaw. To turn right, for example, the spoiler on the right wing
is raised, destroying some of the lift and creating more drag
on the right. The right wing drops, and the aircraft banks
and yaws to the right. Deploying spoilers on both wings at
the same time allows the aircraft to descend without gaining
speed. Spoilers are also deployed to help reduce ground roll
after landing. By destroying lift, they transfer weight to the
wheels, improving braking effectiveness. [Figure 6-19]

Trim Systems
Although an aircraft can be operated throughout a wide range
of attitudes, airspeeds, and power settings, it can be designed to
fly hands-off within only a very limited combination of these
6-10

Figure 6-19. Spoilers reduce lift and increase drag during descent

and landing.

variables. Trim systems are used to relieve the pilot of the
need to maintain constant pressure on the flight controls, and
usually consist of flight deck controls and small hinged devices
attached to the trailing edge of one or more of the primary flight
control surfaces. Designed to help minimize a pilot's workload,
trim systems aerodynamically assist movement and position of
the flight control surface to which they are attached. Common
types of trim systems include trim tabs, balance tabs, antiservo
tabs, ground adjustable tabs, and an adjustable stabilizer.

Trim Tabs
The most common installation on small aircraft is a single
trim tab attached to the trailing edge of the elevator. Most trim
tabs are manually operated by a small, vertically mounted
control wheel. However, a trim crank may be found in some
aircraft. The flight deck control includes a trim tab position
indicator. Placing the trim control in the full nose-down
position moves the trim tab to its full up position. With
the trim tab up and into the airstream, the airflow over the
horizontal tail surface tends to force the trailing edge of the
elevator down. This causes the tail of the aircraft to move
up and the nose to move down. [Figure 6-20]
If the trim tab is set to the full nose-up position, the tab moves
to its full down position. In this case, the air flowing under the
horizontal tail surface hits the tab and forces the trailing edge
of the elevator up, reducing the elevator's AOA. This causes
the tail of the aircraft to move down and the nose to move up.
In spite of the opposing directional movement of the trim
tab and the elevator, control of trim is natural to a pilot. If
the pilot needs to exert constant back pressure on a control
column, the need for nose-up trim is indicated. The normal
trim procedure is to continue trimming until the aircraft is
balanced and the nose-heavy condition is no longer apparent.
Pilots normally establish the desired power, pitch attitude,
and configuration first, and then trim the aircraft to relieve

Nose-down trim
Elevator
Trim tab

Tab up—elevator down

helps to move the entire flight control surface in the direction
that the pilot wishes it to go. A servo tab is a dynamic device
that deploys to decrease the pilots work load and de-stabilize
the aircraft. Servo tabs are sometimes referred to as flight tabs
and are used primarily on large aircraft. They aid the pilot in
moving the control surface and in holding it in the desired
position. Only the servo tab moves in response to movement
of the pilot's flight control, and the force of the airflow on
the servo tab then moves the primary control surface.

Antiservo Tabs

Figure 6-20. The movement of the elevator is opposite to the

Antiservo tabs work in the same manner as balance tabs
except, instead of moving in the opposite direction, they move
in the same direction as the trailing edge of the stabilator.
In addition to decreasing the sensitivity of the stabilator, an
antiservo tab also functions as a trim device to relieve control
pressure and maintain the stabilator in the desired position.
The fixed end of the linkage is on the opposite side of the
surface from the horn on the tab; when the trailing edge of the
stabilator moves up, the linkage forces the trailing edge of the
tab up. When the stabilator moves down, the tab also moves
down. Conversely, trim tabs on elevators move opposite of
the control surface. [Figure 6-21]

direction
of movement
of the elevator
trim tab.
Figure 5-16.
The movement
of the elevator
is opposite to the
direction of movement of the elevator trim tab.

Ground Adjustable Tabs

control pressures that may exist for that flight condition. As
power, pitch attitude, or configuration changes, retrimming
is necessary to relieve the control pressures for the new
flight condition.

Many small aircraft have a nonmovable metal trim tab on the
rudder. This tab is bent in one direction or the other while on
the ground to apply a trim force to the rudder. The correct
displacement is determined by trial and error. Usually, small

Nose-up trim
Elevator
Trim tab

Tab up—elevator up

Balance Tabs
The control forces may be excessively high in some aircraft,
and, in order to decrease them, the manufacturer may use
balance tabs. They look like trim tabs and are hinged in
approximately the same places as trim tabs. The essential
difference between the two is that the balancing tab is coupled
to the control surface rod so that when the primary control
surface is moved in any direction, the tab automatically
moves in the opposite direction. The airflow striking the tab
counterbalances some of the air pressure against the primary
control surface and enables the pilot to move the control more
easily and hold the control surface in position.
If the linkage between the balance tab and the fixed
surface is adjustable from the flight deck, the tab acts as a
combination trim and balance tab that can be adjusted to a
desired deflection.

Servo Tabs
Servo tabs are very similar in operation and appearance to the
trim tabs previously discussed. A servo tab is a small portion
of a flight control surface that deploys in such a way that it

Stabilator
Pivot point

Antiservo tab

Figure 6-21. An antiservo tab attempts to streamline the control

surface and is used to make the stabilator less sensitive by opposing
the force exerted by the pilot.

6-11

adjustments are necessary until the aircraft no longer skids
left or right during normal cruising flight. [Figure 6-22]

Adjustable Stabilizer
Rather than using a movable tab on the trailing edge of the
elevator, some aircraft have an adjustable stabilizer. With this
arrangement, linkages pivot the horizontal stabilizer about
its rear spar. This is accomplished by the use of a jackscrew
mounted on the leading edge of the stabilator. [Figure 6-23]
On small aircraft, the jackscrew is cable operated with a trim
wheel or crank. On larger aircraft, it is motor driven. The
trimming effect and flight deck indications for an adjustable
stabilizer are similar to those of a trim tab.

Autopilot
Autopilot is an automatic flight control system that keeps an
aircraft in level flight or on a set course. It can be directed by
the pilot, or it may be coupled to a radio navigation signal.
Autopilot reduces the physical and mental demands on a pilot
and increases safety. The common features available on an
autopilot are altitude and heading hold.
The simplest systems use gyroscopic attitude indicators and
magnetic compasses to control servos connected to the flight
control system. [Figure 6-24] The number and location of
these servos depends on the complexity of the system. For
example, a single-axis autopilot controls the aircraft about the
longitudinal axis and a servo actuates the ailerons. A three-axis
autopilot controls the aircraft about the longitudinal, lateral, and
vertical axes. Three different servos actuate ailerons, elevator,
and rudder. More advanced systems often include a vertical
speed and/or indicated airspeed hold mode. Advanced autopilot
systems are coupled to navigational aids through a flight director.

Adjustable stabilizer
Nose down
Nose up

Jackscrew
Pivot

Trim motor or trim cable

Figure 6-23. Some aircraft, including most jet transports, use an
adjustable stabilizer to provide the required pitch trim forces.

The autopilot system also incorporates a disconnect safety
feature to disengage the system automatically or manually.
These autopilots work with inertial navigation systems,
global positioning systems (GPS), and flight computers to
control the aircraft. In fly-by-wire systems, the autopilot is
an integrated component.
Additionally, autopilots can be manually overridden. Because
autopilot systems differ widely in their operation, refer to
the autopilot operating instructions in the Airplane Flight
Manual (AFM) or the Pilot's Operating Handbook (POH).

Chapter Summary
Because flight control systems and aerodynamic
characteristics vary greatly between aircraft, it is essential
that a pilot become familiar with the primary and secondary
flight control systems of the aircraft being flown. The
primary source of this information is the AFM or the POH.
Various manufacturer and owner group websites can also be
a valuable source of additional information.

Figure 6-24. Basic autopilot system integrated into the flight

control system.
Figure 6-22. A ground adjustable tab is used on the rudder of many

small airplanes to correct for a tendency to fly with the fuselage
slightly misaligned with the relative wind.

6-12

Chapter 7

Aircraft Systems
Introduction
This chapter covers the primary systems found on most
aircraft. These include the engine, propeller, induction,
ignition, as well as the fuel, lubrication, cooling, electrical,
landing gear, and environmental control systems.

Powerplant
An aircraft engine, or powerplant, produces thrust to propel
an aircraft. Reciprocating engines and turboprop engines
work in combination with a propeller to produce thrust.
Turbojet and turbofan engines produce thrust by increasing
the velocity of air flowing through the engine. All of these
powerplants also drive the various systems that support the
operation of an aircraft.

7-1

Reciprocating Engines
Most small aircraft are designed with reciprocating
engines. The name is derived from the back-and-forth, or
reciprocating, movement of the pistons that produces the
mechanical energy necessary to accomplish work.
Driven by a revitalization of the general aviation (GA)
industry and advances in both material and engine design,
reciprocating engine technology has improved dramatically
over the past two decades. The integration of computerized
engine management systems has improved fuel efficiency,
decreased emissions, and reduced pilot workload.
Reciprocating engines operate on the basic principle of
converting chemical energy (fuel) into mechanical energy.
This conversion occurs within the cylinders of the engine
through the process of combustion. The two primary
reciprocating engine designs are the spark ignition and the
compression ignition. The spark ignition reciprocating engine
has served as the powerplant of choice for many years. In
an effort to reduce operating costs, simplify design, and
improve reliability, several engine manufacturers are turning
to compression ignition as a viable alternative. Often referred
to as jet fuel piston engines, compression ignition engines
have the added advantage of utilizing readily available and
lower cost diesel or jet fuel.
The main mechanical components of the spark ignition and
the compression ignition engine are essentially the same.
Both use cylindrical combustion chambers and pistons that
travel the length of the cylinders to convert linear motion
into the rotary motion of the crankshaft. The main difference
between spark ignition and compression ignition is the
process of igniting the fuel. Spark ignition engines use a
spark plug to ignite a pre-mixed fuel-air mixture. (Fuel-air
mixture is the ratio of the "weight" of fuel to the "weight"
of air in the mixture to be burned.) A compression ignition
engine first compresses the air in the cylinder, raising its
temperature to a degree necessary for automatic ignition
when fuel is injected into the cylinder.

Figure 7-1. Radial engine.

In-line engines have a comparatively small frontal area, but
their power-to-weight ratios are relatively low. In addition,
the rearmost cylinders of an air-cooled, in-line engine
receive very little cooling air, so these engines are normally
limited to four or six cylinders. V-type engines provide
more horsepower than in-line engines and still retain a small
frontal area.
Continued improvements in engine design led to the
development of the horizontally-opposed engine, which
remains the most popular reciprocating engines used on
smaller aircraft. These engines always have an even number
of cylinders, since a cylinder on one side of the crankcase
"opposes" a cylinder on the other side. [Figure 7-2] The
majority of these engines are air cooled and usually are
mounted in a horizontal position when installed on fixed-wing
airplanes. Opposed-type engines have high power-to-weight
ratios because they have a comparatively small, lightweight
crankcase. In addition, the compact cylinder arrangement
reduces the engine's frontal area and allows a streamlined
installation that minimizes aerodynamic drag.

These two engine designs can be further classified as:
1.	 Cylinder arrangement with respect to the crankshaft—
radial, in-line, v-type, or opposed
2. 	 Operating cycle—two or four
3. 	 Method of cooling—liquid or air
Radial engines were widely used during World War II and
many are still in service today. With these engines, a row or
rows of cylinders are arranged in a circular pattern around
the crankcase. The main advantage of a radial engine is the
favorable power-to-weight ratio. [Figure 7-1]
7-2

Opposed cylinders

Figure 7-2. Horizontally opposed engine.

Depending on the engine manufacturer, all of these
arrangements can be designed to utilize spark or compression
ignition and operate on either a two- or four-stroke cycle.
In a two-stroke engine, the conversion of chemical energy
into mechanical energy occurs over a two-stroke operating
cycle. The intake, compression, power, and exhaust processes
occur in only two strokes of the piston rather than the more
common four strokes. Because a two-stroke engine has
a power stroke upon each revolution of the crankshaft, it
typically has higher power-to-weight ratio than a comparable
four-stroke engine. Due to the inherent inefficiency and
disproportionate emissions of the earliest designs, use of the
two-stroke engine has been limited in aviation.

Cylinder

Exhaust valve
Intake valve

Recent advances in material and engine design have
reduced many of the negative characteristics associated
with two-stroke engines. Modern two-stroke engines often
use conventional oil sumps, oil pumps, and full pressure
fed lubrication systems. The use of direct fuel injection
and pressurized air, characteristic of advanced compression
ignition engines, make two-stroke compression ignition
engines a viable alternative to the more common four-stroke
spark ignition designs. [Figure 7-3]
Spark ignition four-stroke engines remain the most common
design used in GA today. [Figure 7-4] The main parts of a
spark ignition reciprocating engine include the cylinders,
crankcase, and accessory housing. The intake/exhaust valves,
spark plugs, and pistons are located in the cylinders. The
crankshaft and connecting rods are located in the crankcase.
The magnetos are normally located on the engine accessory
housing.
Forced air

Exhaust valve

Fuel injector

Spark plug
Piston
Crankcase

Connecting rod

Crankshaft

Figure 7-4. Main components of a spark ignition reciprocating

engine.

In a four-stroke engine, the conversion of chemical energy into
mechanical energy occurs over a four-stroke operating cycle.
The intake, compression, power, and exhaust processes occur
in four separate strokes of the piston in the following order.
1.	 The intake stroke begins as the piston starts its
downward travel. When this happens, the intake
valve opens and the fuel-air mixture is drawn into the
cylinder.
2. 	 The compression stroke begins when the intake valve
closes, and the piston starts moving back to the top of
the cylinder. This phase of the cycle is used to obtain
a much greater power output from the fuel-air mixture
once it is ignited.

Piston

1. Intake/compression
and exhaust

2. Power stroke

Figure 7-3. Two-stroke compression ignition.

3. 	 The power stroke begins when the fuel-air mixture is
ignited. This causes a tremendous pressure increase
in the cylinder and forces the piston downward away
from the cylinder head, creating the power that turns
the crankshaft.

7-3

4. 	 The exhaust stroke is used to purge the cylinder of
burned gases. It begins when the exhaust valve opens,
and the piston starts to move toward the cylinder head
once again.
Even when the engine is operated at a fairly low speed,
the four-stroke cycle takes place several hundred times
each minute. [Figure 7-5] In a four-cylinder engine, each
cylinder operates on a different stroke. Continuous rotation
of a crankshaft is maintained by the precise timing of the
power strokes in each cylinder. Continuous operation of the
engine depends on the simultaneous function of auxiliary
systems, including the induction, ignition, fuel, oil, cooling,
and exhaust systems.
The latest advance in aircraft reciprocating engines was
pioneered in the mid-1960s by Frank Thielert, who looked
to the automotive industry for answers on how to integrate
diesel technology into an aircraft engine. The advantage

Intake valve

Piston

1. Intake

3. Power

Exhaust valve

Connecting rod
2. Compression

4. Exhaust

Figure 7-5. The arrows in this illustration indicate the direction of

motion of the crankshaft and piston during the four-stroke cycle.

7-4

In 1999, Thielert formed Thielert Aircraft Engines (TAE)
to design, develop, certify, and manufacture a brand-new
Jet-A-burning diesel cycle engine (also known as jet-fueled
piston engine) for the GA industry. By March 2001, the first
prototype engine became the first certified diesel engine
since World War II. TAE continues to design and develop
diesel cycle engines and other engine manufacturers, such as
Société de Motorisations Aéronautiques (SMA), now offer
jet-fueled piston engines as well. TAE engines can be found
on the Diamond DA40 single and the DA42 Twin Star; the
first diesel engine to be part of the type certificate of a new
original equipment manufacturer (OEM) aircraft.
These engines have also gained a toehold in the retrofit
market with a supplemental type certificate (STC) to reengine the Cessna 172 models and the Piper PA-28 family.
The jet-fueled piston engine's technology has continued to
progress and a full authority digital engine control (FADEC,
discussed more fully later in the chapter) is standard on such
equipped aircraft, which minimizes complication of engine
control. By 2007, various jet-fueled piston aircraft had logged
well over 600,000 hours of service.

Spark plug

Crankshaft

of a diesel-fueled reciprocating engine lies in the physical
similarity of diesel and kerosene. Aircraft equipped with a
diesel piston engine runs on standard aviation fuel kerosene,
which provides more independence, higher reliability, lower
consumption, and operational cost saving.

Propeller
The propeller is a rotating airfoil, subject to induced drag,
stalls, and other aerodynamic principles that apply to any
airfoil. It provides the necessary thrust to pull, or in some
cases push, the aircraft through the air. The engine power is
used to rotate the propeller, which in turn generates thrust
very similar to the manner in which a wing produces lift.
The amount of thrust produced depends on the shape of the
airfoil, the angle of attack (AOA) of the propeller blade, and
the revolutions per minute (rpm) of the engine. The propeller
itself is twisted so the blade angle changes from hub to tip.
The greatest angle of incidence, or the highest pitch, is at the
hub while the smallest angle of incidence or smallest pitch
is at the tip. [Figure 7-6]
The reason for the twist is to produce uniform lift from the
hub to the tip. As the blade rotates, there is a difference in
the actual speed of the various portions of the blade. The tip
of the blade travels faster than the part near the hub, because
the tip travels a greater distance than the hub in the same
length of time. [Figure 7-7] Changing the angle of incidence
(pitch) from the hub to the tip to correspond with the speed
produces uniform lift throughout the length of the blade. A
propeller blade designed with the same angle of incidence

installed depends upon its intended use. The climb propeller
has a lower pitch, therefore less drag. Less drag results in
higher rpm and more horsepower capability, which increases
performance during takeoffs and climbs but decreases
performance during cruising flight.
The cruise propeller has a higher pitch, therefore more
drag. More drag results in lower rpm and less horsepower
capability, which decreases performance during takeoffs and
climbs but increases efficiency during cruising flight.
Figure 7-6. Changes in propeller blade angle from hub to tip.

throughout its entire length would be inefficient because as
airspeed increases in flight, the portion near the hub would
have a negative AOA while the blade tip would be stalled.
Small aircraft are equipped with either one of two types of
propellers: fixed-pitch or adjustable-pitch.

Fixed-Pitch Propeller
A propeller with fixed blade angles is a fixed-pitch propeller.
The pitch of this propeller is set by the manufacturer and
cannot be changed. Since a fixed-pitch propeller achieves
the best efficiency only at a given combination of airspeed
and rpm, the pitch setting is ideal for neither cruise nor
climb. Thus, the aircraft suffers a bit in each performance
category. The fixed-pitch propeller is used when low weight,
simplicity, and low cost are needed.
There are two types of fixed-pitch propellers: climb and
cruise. Whether the airplane has a climb or cruise propeller

The rpm is regulated by the throttle, which controls the fuelair flow to the engine. At a given altitude, the higher the
tachometer reading, the higher the power output of the engine.
When operating altitude increases, the tachometer may not
show correct power output of the engine. For example, 2,300
rpm at 5,000 feet produces less horsepower than 2,300 rpm
at sea level because power output depends on air density. Air
density decreases as altitude increases and a decrease in air

stance—slo
l di
w
ve

eed
sp

Shor
t tr
a

In a fixed-pitch propeller, the tachometer is the indicator of
engine power. [Figure 7-8] A tachometer is calibrated in
hundreds of rpm and gives a direct indication of the engine
and propeller rpm. The instrument is color coded with a green
arc denoting the maximum continuous operating rpm. Some
tachometers have additional markings to reflect engine and/or
propeller limitations. The manufacturer's recommendations
should be used as a reference to clarify any misunderstanding
of tachometer markings.

s
not
9k
25

Mo
de

s
ot

l distance—very high
spe
trave
ed—
ter
a
e
38
Gr
9k
n
—
m
e
odera
anc
te s
l dist
e
v
pe
tra
ed
te
—
ra

The propeller is usually mounted on a shaft, which may be
an extension of the engine crankshaft. In this case, the rpm
of the propeller would be the same as the crankshaft rpm. On
some engines, the propeller is mounted on a shaft geared to
the engine crankshaft. In this type, the rpm of the propeller
is different than that of the engine.

—12
9k

20 in.
2,50 rpm
0

ts
no

40 in.

2,500 rpm
60 in.

I5
I0
5
3

20
RPM

HUNDREDS

25
I
I0

HOURS
AVOID
CONTINUOUS
OPERATION
BETWEEN 2250
AND 2350 RPM

30

35

2,500 rpm

Figure 7-7. Relationship of travel distance and speed of various

portions of propeller blade.

Figure 7-8. Engine rpm is indicated on the tachometer.

7-5

density (higher density altitude) decreases the power output
of the engine. As altitude changes, the position of the throttle
must be changed to maintain the same rpm. As altitude is
increased, the throttle must be opened further to indicate the
same rpm as at a lower altitude.

Adjustable-Pitch Propeller
The adjustable-pitch propeller was the forerunner of the
constant-speed propeller. It is a propeller with blades whose
pitch can be adjusted on the ground with the engine not
running, but which cannot be adjusted in flight. It is also
referred to as a ground adjustable propeller. By the 1930s,
pioneer aviation inventors were laying the ground work for
automatic pitch-change mechanisms, which is why the term
sometimes refers to modern constant-speed propellers that
are adjustable in flight.
The first adjustable-pitch propeller systems provided only two
pitch settings: low and high. Today, most adjustable-pitch
propeller systems are capable of a range of pitch settings.
A constant-speed propeller is a controllable-pitch propeller
whose pitch is automatically varied in flight by a governor
maintaining constant rpm despite varying air loads. It is the
most common type of adjustable-pitch propeller. The main
advantage of a constant-speed propeller is that it converts
a high percentage of brake horsepower (BHP) into thrust
horsepower (THP) over a wide range of rpm and airspeed
combinations. A constant-speed propeller is more efficient
than other propellers because it allows selection of the most
efficient engine rpm for the given conditions.
An aircraft with a constant-speed propeller has two controls:
the throttle and the propeller control. The throttle controls
power output, and the propeller control regulates engine
rpm. This regulates propeller rpm, which is registered on
the tachometer.

will increase or decrease as appropriate, with changes in
airspeed and propeller load. For example, once a specific
rpm has been selected, if aircraft speed decreases enough to
rotate the propeller blades until they contact the low pitch
stop, any further decrease in airspeed will cause engine rpm
to decrease the same way as if a fixed-pitch propeller were
installed. The same holds true when an aircraft equipped with
a constant-speed propeller accelerates to a faster airspeed. As
the aircraft accelerates, the propeller blade angle increases to
maintain the selected rpm until the high pitch stop is reached.
Once this occurs, the blade angle cannot increase any further
and engine rpm increases.
On aircraft equipped with a constant-speed propeller, power
output is controlled by the throttle and indicated by a manifold
pressure gauge. The gauge measures the absolute pressure of
the fuel-air mixture inside the intake manifold and is more
correctly a measure of manifold absolute pressure (MAP). At
a constant rpm and altitude, the amount of power produced
is directly related to the fuel-air mixture being delivered to
the combustion chamber. As the throttle setting is increased,
more fuel and air flows to the engine and MAP increases.
When the engine is not running, the manifold pressure gauge
indicates ambient air pressure (i.e., 29.92 inches mercury
(29.92 "Hg)). When the engine is started, the manifold
pressure indication decreases to a value less than ambient
pressure (i.e., idle at 12 "Hg). Engine failure or power loss
is indicated on the manifold gauge as an increase in manifold
pressure to a value corresponding to the ambient air pressure
at the altitude where the failure occurred. [Figure 7-9]
The manifold pressure gauge is color coded to indicate the
engine's operating range. The face of the manifold pressure
gauge contains a green arc to show the normal operating
range and a red radial line to indicate the upper limit of
manifold pressure.

Once a specific rpm is selected, a governor automatically
adjusts the propeller blade angle as necessary to maintain
the selected rpm. For example, after setting the desired rpm
during cruising flight, an increase in airspeed or decrease in
propeller load causes the propeller blade angle to increase
as necessary to maintain the selected rpm. A reduction in
airspeed or increase in propeller load causes the propeller
blade angle to decrease.
The propeller's constant-speed range, defined by the high
and low pitch stops, is the range of possible blade angles for
a constant-speed propeller. As long as the propeller blade
angle is within the constant-speed range and not against
either pitch stop, a constant engine rpm is maintained. If
the propeller blades contact a pitch stop, the engine rpm
7-6

25
20

30

35

MANIFOLD
PRESS
IN Hg
AL g .

15
10

40
45

50

Figure 7-9. Engine power output is indicated on the manifold

pressure gauge.

For any given rpm, there is a manifold pressure that should
not be exceeded. If manifold pressure is excessive for a given
rpm, the pressure within the cylinders could be exceeded,
placing undue stress on the cylinders. If repeated too
frequently, this stress can weaken the cylinder components
and eventually cause engine failure.
A pilot can avoid conditions that overstress the cylinders
by being constantly aware of the rpm, especially when
increasing the manifold pressure. Consult the manufacturer's
recommendations for power settings of a particular engine to
maintain the proper relationship between manifold pressure
and rpm.
When both manifold pressure and rpm need to be changed,
avoid engine overstress by making power adjustments in
the proper order:


When power settings are being decreased, reduce
manifold pressure before reducing rpm. If rpm is
reduced before manifold pressure, manifold pressure
automatically increases, possibly exceeding the
manufacturer's tolerances.



When power settings are being increased, reverse the
order—increase rpm first, then manifold pressure.



To prevent damage to radial engines, minimize
operating time at maximum rpm and manifold
pressure, and avoid operation at maximum rpm and
low manifold pressure.

The engine and/or airframe manufacturer's recommendations
should be followed to prevent severe wear, fatigue, and
damage to high-performance reciprocating engines.

Propeller Overspeed in Piston Engine Aircraft
On March 17, 2010, the Federal Aviation Administration
(FAA) issued Special Airworthiness Information Bulletin
(SAIB) CE-10-21. The subject was Propellers/Propulsers;
Propeller Overspeed in Piston Engine Aircraft to alert
operators, pilots, and aircraft manufacturers of concerns
for an optimum response to a propeller overspeed in piston
engine aircraft with variable pitch propellers. Although a
SAIB is not regulatory in nature, the FAA recommends that
the information be read and taken into consideration for the
safety of flight.
The document explains that a single-engine aircraft
experienced a propeller overspeed during cruise flight at
7,000 feet. The pilot reported that the application of throttle
resulted in a propeller overspeed with no appreciable thrust.
The pilot attempted to glide to a nearby airport and established
the "best glide" speed of 110 knots, as published in the Pilot's

Operating Handbook (POH), but was unable to reach the
airport and was forced to conduct an off-field landing.
It was further explained that a determination was made that the
propeller experienced a failure causing the blade pitch change
mechanism to move to the low pitch stop position. This caused
the propeller to operate as a fixed-pitch propeller such that it
changes rpm with changes in power and airspeed. The low
pitch setting allows for maximum power during takeoff but
can result in a propeller overspeed at a higher airspeed.
A performance evaluation of the flight condition was
performed for the particular aircraft model involved in this
incident. This evaluation indicated that an airspeed lower
than the best glide speed would have resulted in increased
thrust enabling the pilot to maintain level flight. There are
numerous variables in aircraft, engines, and propellers
that affect aircraft performance. For some aircraft models,
the published best glide speed may not be low enough to
generate adequate thrust for a given propeller installation in
this situation (propeller blades at low pitch stop position).
The operators of aircraft with variable pitch propellers should
be aware that in certain instances of propeller overspeed, the
airspeed necessary to maintain level flight may be different
than the speed associated with engine-out best glide speed.
The appropriate emergency procedures should be followed to
mitigate the emergency situation in the event of a propeller
overspeed; however, pilots should be aware that some
reduction in airspeed may result in the ability for continued
safe flight and landing. The determination of an airspeed that
is more suitable than engine-out best glide speed should only
be conducted at a safe altitude when the pilot has time to
determine an alternative course of action other than landing
immediately.
Induction Systems
The induction system brings in air from the outside, mixes
it with fuel, and delivers the fuel-air mixture to the cylinder
where combustion occurs. Outside air enters the induction
system through an intake port on the front of the engine
cowling. This port normally contains an air filter that inhibits
the entry of dust and other foreign objects. Since the filter
may occasionally become clogged, an alternate source of
air must be available. Usually, the alternate air comes from
inside the engine cowling, where it bypasses a clogged air
filter. Some alternate air sources function automatically,
while others operate manually.

7-7

Two types of induction systems are commonly used in small
aircraft engines:
1.	 The carburetor system mixes the fuel and air in
the carburetor before this mixture enters the intake
manifold.
2.	 The fuel injection system mixes the fuel and air
immediately before entry into each cylinder or injects
fuel directly into each cylinder.
Carburetor Systems
Aircraft carburetors are separated into two categories: floattype carburetors and pressure-type carburetors. Float-type
carburetors, complete with idling, accelerating, mixture
control, idle cutoff, and power enrichment systems, are the
most common of the two carburetor types. Pressure-type
carburetors are usually not found on small aircraft. The basic
difference between a float-type and a pressure-type carburetor
is the delivery of fuel. The pressure-type carburetor delivers
fuel under pressure by a fuel pump.
In the operation of the float-type carburetor system, the
outside air first flows through an air filter, usually located
at an air intake in the front part of the engine cowling. This
filtered air flows into the carburetor and through a venturi, a
narrow throat in the carburetor. When the air flows through
the venturi, a low-pressure area is created that forces the

fuel to flow through a main fuel jet located at the throat. The
fuel then flows into the airstream where it is mixed with the
flowing air. [Figure 7-10]
The fuel-air mixture is then drawn through the intake
manifold and into the combustion chambers where it is
ignited. The float-type carburetor acquires its name from a
float that rests on fuel within the float chamber. A needle
attached to the float opens and closes an opening at the
bottom of the carburetor bowl. This meters the amount of
fuel entering into the carburetor, depending upon the position
of the float, which is controlled by the level of fuel in the
float chamber. When the level of the fuel forces the float
to rise, the needle valve closes the fuel opening and shuts
off the fuel flow to the carburetor. The needle valve opens
again when the engine requires additional fuel. The flow of
the fuel-air mixture to the combustion chambers is regulated
by the throttle valve, which is controlled by the throttle in
the flight deck.
The float-type carburetor has several distinct disadvantages.
First, they do not function well during abrupt maneuvers.
Secondly, the discharge of fuel at low pressure leads to
incomplete vaporization and difficulty in discharging
fuel into some types of supercharged systems. The chief
disadvantage of the float-type carburetor, however, is its
icing tendency. Since the float-type carburetor must discharge

Fuel-air mixture

Float chamber

Fuel inlet

The blend of fuel and
air is routed to the
combustion chambers
to be burned.

Fuel level is maintained
by a float-type device.

Fuel is received into
the carburetor through
the fuel inlet.

Throttle valve
The flow of the fuel-air
mixture is controlled by
the throttle valve. The
throttle valve is adjusted
from the flight deck by
the throttle.

Fuel

Venturi
The shape of the venturi
creates an area of low
pressure.

Mixture needle
The mixture needle
controls fuel to the
discharge nozzle.
Mixture needle position
can be adjusted using
the mixture control.

Discharge nozzle
Fuel is forced through
the discharge nozzle
into the venturi by
greater atmospheric
pressure in the float
chamber.

Air bleed
Air inlet
Air enters the carburetor
through the air inlet.

Figure 7-10. Float-type carburetor.

7-8

The air bleed allows air to be mixed
with fuel being drawn out of the
discharge nozzle to decrease fuel
density and promote fuel vaporization.

fuel at a point of low pressure, the discharge nozzle must be
located at the venturi throat, and the throttle valve must be
on the engine side of the discharge nozzle. This means that
the drop in temperature due to fuel vaporization takes place
within the venturi. As a result, ice readily forms in the venturi
and on the throttle valve.
A pressure-type carburetor discharges fuel into the airstream
at a pressure well above atmospheric pressure. This results
in better vaporization and permits the discharge of fuel into
the airstream on the engine side of the throttle valve. With the
discharge nozzle in this position fuel vaporization takes place
after the air has passed through the throttle valve and at a point
where the drop in temperature is offset by heat from the engine.
Thus, the danger of fuel vaporization icing is practically
eliminated. The effects of rapid maneuvers and rough air on
the pressure-type carburetors are negligible, since their fuel
chambers remain filled under all operating conditions.

Mixture Control
Carburetors are normally calibrated at sea-level air pressure
where the correct fuel-air mixture ratio is established with the
mixture control set in the FULL RICH position. However, as
altitude increases, the density of air entering the carburetor
decreases, while the density of the fuel remains the same. This
creates a progressively richer mixture that can result in engine
roughness and an appreciable loss of power. The roughness
normally is due to spark plug fouling from excessive carbon
buildup on the plugs. Carbon buildup occurs because the
rich mixture lowers the temperature inside the cylinder,
inhibiting complete combustion of the fuel. This condition
may occur during the runup prior to takeoff at high-elevation
airports and during climbs or cruise flight at high altitudes.
To maintain the correct fuel-air mixture, the mixture must
be leaned using the mixture control. Leaning the mixture
decreases fuel flow, which compensates for the decreased
air density at high altitude.

Carburetor Icing
As mentioned earlier, one disadvantage of the float-type
carburetor is its icing tendency. Carburetor ice occurs due
to the effect of fuel vaporization and the decrease in air
pressure in the venturi, which causes a sharp temperature
drop in the carburetor. If water vapor in the air condenses
when the carburetor temperature is at or below freezing, ice
may form on internal surfaces of the carburetor, including
the throttle valve. [Figure 7-11]
The reduced air pressure, as well as the vaporization of fuel,
contributes to the temperature decrease in the carburetor. Ice
generally forms in the vicinity of the throttle valve and in the
venturi throat. This restricts the flow of the fuel-air mixture
and reduces power. If enough ice builds up, the engine may
cease to operate. Carburetor ice is most likely to occur when
temperatures are below 70 degrees Fahrenheit (°F) or 21
degrees Celsius (°C) and the relative humidity is above 80
percent. Due to the sudden cooling that takes place in the
carburetor, icing can occur even in outside air temperatures
as high as 100 °F (38 °C) and humidity as low as 50 percent.
This temperature drop can be as much as 60 to 70 absolute
(versus relative) Fahrenheit degrees (70 x 100/180 = 38.89

To engine

Fuel-air mixture

Ice

Ice
Ice
Venturi

During a descent from high altitude, the fuel-air mixture
must be enriched, or it may become too lean. An overly lean
mixture causes detonation, which may result in rough engine
operation, overheating, and/or a loss of power. The best way
to maintain the proper fuel-air mixture is to monitor the
engine temperature and enrich the mixture as needed. Proper
mixture control and better fuel economy for fuel-injected
engines can be achieved by using an exhaust gas temperature
(EGT) gauge. Since the process of adjusting the mixture can
vary from one aircraft to another, it is important to refer to
the airplane flight manual (AFM) or the POH to determine
the specific procedures for a given aircraft.

Incoming air

Figure 7-11. The formation of carburetor ice may reduce or block
fuel-air flow to the engine.

7-9

Celsius degrees) (Remember there are 180 Fahrenheit degrees
from freezing to boiling versus 100 degrees for the Celsius
scale.) Therefore, an outside air temperature of 100 F (38 C),
a temperature drop of an absolute 70 F degrees (38.89 Celsius
degrees) results in an air temperature in the carburetor of 30
F (-1 C). [Figure 7-12]
The first indication of carburetor icing in an aircraft with
a fixed-pitch propeller is a decrease in engine rpm, which
may be followed by engine roughness. In an aircraft with a
constant-speed propeller, carburetor icing is usually indicated
by a decrease in manifold pressure, but no reduction in rpm.
Propeller pitch is automatically adjusted to compensate for
loss of power. Thus, a constant rpm is maintained. Although
carburetor ice can occur during any phase of flight, it is
particularly dangerous when using reduced power during a
descent. Under certain conditions, carburetor ice could build
unnoticed until power is added. To combat the effects of
carburetor ice, engines with float-type carburetors employ a
carburetor heat system.

Carburetor Heat
Carburetor heat is an anti-icing system that preheats the air
before it reaches the carburetor and is intended to keep the
fuel-air mixture above freezing to prevent the formation of
carburetor ice. Carburetor heat can be used to melt ice that has
already formed in the carburetor if the accumulation is not too
great, but using carburetor heat as a preventative measure is
the better option. Additionally, carburetor heat may be used
as an alternate air source if the intake filter clogs, such as in
sudden or unexpected airframe icing conditions. The carburetor
heat should be checked during the engine runup. When using
carburetor heat, follow the manufacturer's recommendations.
When conditions are conducive to carburetor icing during
flight, periodic checks should be made to detect its
100\%

Relative humidity

90\%

High carburetor
icing potential

80\%
70\%

Carburetor icing possible

60\%
50\%

20°F/-7°C 32°F/0°C

70°F/21°C

100°F/38°C

Outside air temperature
Figure 7-12. Although carburetor ice is most likely to form when

the temperature and humidity are in ranges indicated by this chart,
carburetor icing is possible under conditions not depicted.

7-10

presence. If detected, full carburetor heat should be applied
immediately, and it should be left in the ON position until
the pilot is certain that all the ice has been removed. If ice
is present, applying partial heat or leaving heat on for an
insufficient time might aggravate the situation. In extreme
cases of carburetor icing, even after the ice has been removed,
full carburetor heat should be used to prevent further ice
formation. If installed, a carburetor temperature gauge is
useful in determining when to use carburetor heat.
Whenever the throttle is closed during flight, the engine cools
rapidly and vaporization of the fuel is less complete than if
the engine is warm. Also, in this condition, the engine is more
susceptible to carburetor icing. If carburetor icing conditions
are suspected and closed-throttle operation anticipated, adjust
the carburetor heat to the full ON position before closing the
throttle and leave it on during the closed-throttle operation.
The heat aids in vaporizing the fuel and helps prevent the
formation of carburetor ice. Periodically, open the throttle
smoothly for a few seconds to keep the engine warm;
otherwise, the carburetor heater may not provide enough
heat to prevent icing.
The use of carburetor heat causes a decrease in engine
power, sometimes up to 15 percent, because the heated
air is less dense than the outside air that had been entering
the engine. This enriches the mixture. When ice is present
in an aircraft with a fixed-pitch propeller and carburetor
heat is being used, there is a decrease in rpm, followed by
a gradual increase in rpm as the ice melts. The engine also
should run more smoothly after the ice has been removed.
If ice is not present, the rpm decreases and then remains
constant. When carburetor heat is used on an aircraft with a
constant-speed propeller and ice is present, a decrease in the
manifold pressure is noticed, followed by a gradual increase.
If carburetor icing is not present, the gradual increase in
manifold pressure is not apparent until the carburetor heat
is turned off.
It is imperative for a pilot to recognize carburetor ice when it
forms during flight to prevent a loss in power, altitude, and/or
airspeed. These symptoms may sometimes be accompanied
by vibration or engine roughness. Once a power loss is
noticed, immediate action should be taken to eliminate ice
already formed in the carburetor and to prevent further ice
formation. This is accomplished by applying full carburetor
heat, which will further reduce power and may cause engine
roughness as melted ice goes through the engine. These
symptoms may last from 30 seconds to several minutes,
depending on the severity of the icing. During this period, the
pilot must resist the temptation to decrease the carburetor heat
usage. Carburetor heat must remain in the full-hot position
until normal power returns.

Since the use of carburetor heat tends to reduce the output
of the engine and to increase the operating temperature,
carburetor heat should not be used when full power is required
(as during takeoff) or during normal engine operation, except
to check for the presence of, or to remove, carburetor ice.

Carburetor Air Temperature Gauge
Some aircraft are equipped with a carburetor air temperature
gauge, which is useful in detecting potential icing conditions.
Usually, the face of the gauge is calibrated in degrees Celsius
with a yellow arc indicating the carburetor air temperatures
where icing may occur. This yellow arc typically ranges
between –15 °C and +5 °C (5 °F and 41 °F). If the air
temperature and moisture content of the air are such that
carburetor icing is improbable, the engine can be operated
with the indicator in the yellow range with no adverse effects.
If the atmospheric conditions are conducive to carburetor
icing, the indicator must be kept outside the yellow arc by
application of carburetor heat.

Celsius and Fahrenheit. It provides the outside or ambient
air temperature for calculating true airspeed and is useful in
detecting potential icing conditions.
Fuel Injection Systems
In a fuel injection system, the fuel is injected directly into
the cylinders, or just ahead of the intake valve. The air
intake for the fuel injection system is similar to that used
in a carburetor system, with an alternate air source located
within the engine cowling. This source is used if the external
air source is obstructed. The alternate air source is usually
operated automatically, with a backup manual system that
can be used if the automatic feature malfunctions.
A fuel injection system usually incorporates six basic
components: an engine-driven fuel pump, a fuel-air control
unit, a fuel manifold (fuel distributor), discharge nozzles,
an auxiliary fuel pump, and fuel pressure/flow indicators.
[Figure 7-13]

Certain carburetor air temperature gauges have a red radial
that indicates the maximum permissible carburetor inlet air
temperature recommended by the engine manufacturer. If
present, a green arc indicates the normal operating range.

The auxiliary fuel pump provides fuel under pressure to the
fuel-air control unit for engine starting and/or emergency
use. After starting, the engine-driven fuel pump provides fuel
under pressure from the fuel tank to the fuel-air control unit.

Outside Air Temperature Gauge
Most aircraft are also equipped with an outside air
temperature (OAT) gauge calibrated in both degrees

This control unit, which essentially replaces the carburetor,
meters fuel based on the mixture control setting and sends it
to the fuel manifold valve at a rate controlled by the throttle.

k

l tan

Fue

Auxiliary fuel pump

Engine-driven fuel pump

Fuel-air control unit
Fuel lines

Fuel manifold valve

Figure 7-13. Fuel injection system.

7-11

After reaching the fuel manifold valve, the fuel is distributed
to the individual fuel discharge nozzles. The discharge
nozzles, which are located in each cylinder head, inject the
fuel-air mixture directly into each cylinder intake port.
A fuel injection system is considered to be less susceptible
to icing than a carburetor system, but impact icing on the air
intake is a possibility in either system. Impact icing occurs
when ice forms on the exterior of the aircraft and blocks
openings, such as the air intake for the injection system.
The following are advantages of using fuel injection:


Reduction in evaporative icing



Better fuel flow



Faster throttle response



Precise control of mixture



Better fuel distribution



Easier cold weather starts

The following are disadvantages of using fuel injection:


Difficulty in starting a hot engine



Vapor locks during ground operations on hot days



Problems associated with restarting an engine that
quits because of fuel starvation

Superchargers and Turbosuperchargers
To increase an engine's horsepower, manufacturers have
developed forced induction systems called supercharger
and turbosupercharger systems. They both compress the
intake air to increase its density. The key difference lies in
the power supply. A supercharger relies on an engine-driven
air pump or compressor, while a turbocharger gets its power
from the exhaust stream that runs through a turbine, which in
turn spins the compressor. Aircraft with these systems have
a manifold pressure gauge, which displays MAP within the
engine's intake manifold.
On a standard day at sea level with the engine shut
down, the manifold pressure gauge indicates the ambient
absolute air pressure of 29.92 "Hg. Because atmospheric
pressure decreases approximately 1 "Hg per 1,000 feet of
altitude increase, the manifold pressure gauge indicates
approximately 24.92 "Hg at an airport that is 5,000 feet above
sea level with standard day conditions.
As a normally aspirated aircraft climbs, it eventually reaches
an altitude where the MAP is insufficient for a normal climb.
This altitude limit is known as the aircraft's service ceiling,
and it is directly affected by the engine's ability to produce
power. If the induction air entering the engine is pressurized,
7-12

or boosted, by either a supercharger or a turbosupercharger,
the aircraft's service ceiling can be increased. With these
systems, an aircraft can fly at higher altitudes with the
advantage of higher true airspeeds and the increased ability
to circumnavigate adverse weather.
Superchargers
A supercharger is an engine-driven air pump or compressor
that provides compressed air to the engine to provide
additional pressure to the induction air so that the engine can
produce additional power. It increases manifold pressure and
forces the fuel-air mixture into the cylinders. Higher manifold
pressure increases the density of the fuel-air mixture and
increases the power an engine can produce. With a normally
aspirated engine, it is not possible to have manifold pressure
higher than the existing atmospheric pressure. A supercharger
is capable of boosting manifold pressure above 30 "Hg.
For example, at 8,000 feet, a typical engine may be able to
produce 75 percent of the power it could produce at mean
sea level (MSL) because the air is less dense at the higher
altitude. The supercharger compresses the air to a higher
density allowing a supercharged engine to produce the same
manifold pressure at higher altitudes as it could produce
at sea level. Thus, an engine at 8,000 feet MSL could still
produce 25 "Hg of manifold pressure whereas, without a
supercharger, it could only produce 22 "Hg. Superchargers
are especially valuable at high altitudes (such as 18,000 feet)
where the air density is 50 percent that of sea level. The use
of a supercharger in many cases will supply air to the engine
at the same density it did at sea level.
The components in a supercharged induction system are similar
to those in a normally aspirated system, with the addition of
a supercharger between the fuel metering device and intake
manifold. A supercharger is driven by the engine through a
gear train at one speed, two speeds, or variable speeds. In
addition, superchargers can have one or more stages. Each
stage also provides an increase in pressure and superchargers
may be classified as single stage, two stage, or multistage,
depending on the number of times compression occurs.
An early version of a single-stage, single-speed supercharger
may be referred to as a sea-level supercharger. An engine
equipped with this type of supercharger is called a sea-level
engine. With this type of supercharger, a single gear-driven
impeller is used to increase the power produced by an engine
at all altitudes. The drawback with this type of supercharger is
a decrease in engine power output with an increase in altitude.
Single-stage, single-speed superchargers are found on many
high-powered radial engines and use an air intake that faces
forward so the induction system can take full advantage of

the ram air. Intake air passes through ducts to a carburetor,
where fuel is metered in proportion to the airflow. The
fuel-air charge is then ducted to the supercharger, or blower
impeller, which accelerates the fuel-air mixture outward.
Once accelerated, the fuel-air mixture passes through a
diffuser, where air velocity is traded for pressure energy.
After compression, the resulting high pressure fuel-air
mixture is directed to the cylinders.
Some of the large radial engines developed during World
War II have a single-stage, two-speed supercharger. With
this type of supercharger, a single impeller may be operated
at two speeds. The low impeller speed is often referred to
as the low blower setting, while the high impeller speed is
called the high blower setting. On engines equipped with a
two-speed supercharger, a lever or switch in the flight deck
activates an oil-operated clutch that switches from one speed
to the other.
Under normal operations, takeoff is made with the
supercharger in the low blower position. In this mode, the
engine performs as a ground-boosted engine, and the power
output decreases as the aircraft gains altitude. However, once
the aircraft reaches a specified altitude, a power reduction is
made, and the supercharger control is switched to the high
blower position. The throttle is then reset to the desired
manifold pressure. An engine equipped with this type of
supercharger is called an altitude engine. [Figure 7-14]

Brake horsepower
Sea level

A second advantage of turbochargers over superchargers is
the ability to maintain control over an engine's rated sealevel horsepower from sea level up to the engine's critical
altitude. Critical altitude is the maximum altitude at which
a turbocharged engine can produce its rated horsepower.
Above the critical altitude, power output begins to decrease
like it does for a normally aspirated engine.

ngine

Density altitude

Since the temperature of a gas rises when it is compressed,
turbocharging causes the temperature of the induction air to
increase. To reduce this temperature and lower the risk of
detonation, many turbocharged engines use an intercooler.
This small heat exchanger uses outside air to cool the hot
compressed air before it enters the fuel metering device.

Two-speed supercharged engine
High

blow
er

Nor
m

The major disadvantage of the gear-driven supercharger––use
of a large amount of the engine's power output for the amount
of power increase produced––is avoided with a turbocharger
because turbochargers are powered by an engine's exhaust
gases. This means a turbocharger recovers energy from hot
exhaust gases that would otherwise be lost.

Turbochargers increase the pressure of the engine's induction
air, which allows the engine to develop sea level or greater
horsepower at higher altitudes. A turbocharger is comprised
of two main elements: a compressor and turbine. The
compressor section houses an impeller that turns at a high rate
of speed. As induction air is drawn across the impeller blades,
the impeller accelerates the air, allowing a large volume of
air to be drawn into the compressor housing. The impeller's
action subsequently produces high-pressure, high-density
air that is delivered to the engine. To turn the impeller, the
engine's exhaust gases are used to drive a turbine wheel
that is mounted on the opposite end of the impeller's drive
shaft. By directing different amounts of exhaust gases to flow
over the turbine, more energy can be extracted, causing the
impeller to deliver more compressed air to the engine. The
waste gate, essentially an adjustable butterfly valve installed
in the exhaust system, is used to vary the mass of exhaust gas
flowing into the turbine. When closed, most of the exhaust
gases from the engine are forced to flow through the turbine.
When open, the exhaust gases are allowed to bypass the
turbine by flowing directly out through the engine's exhaust
pipe. [Figure 7-15]

Turbosuperchargers
The most efficient method of increasing horsepower in an
engine is by using a turbosupercharger or turbocharger.

Low

Installed on an engine, this booster uses the engine's exhaust
gases to drive an air compressor to increase the pressure of
the air going into the engine through the carburetor or fuel
injection system to boost power at higher altitude.

ally
aspi
rated
e

blow

er

Figure 7-14. Power output of normally aspirated engine compared
to a single-stage, two-speed supercharged engine.

7-13

Turbocharger
The turbocharger in­
corporates a turbine,
which is driven by ex­
haust gases and a com­
pressor that pressurizes
the incoming air.

Exhaust gas discharge

Throttle body

Intake manifold

This regulates airflow
to the engine.

Pressurized air from the
turbocharger is supplied
to the cylinders.

Waste gas

Exhaust manifold

This controls the amount
of exhaust through the
turbine. Waste gate
position is actuated by
engine oil pressure.

Exhaust gas is ducted
through the exhaust man­
ifold and is used to turn
the turbine which drives
the compressor.

Air intake
Intake air is ducted to the
turbocharger where it is
compressed.

Figure 7-15. Turbocharger components.

System Operation
On most modern turbocharged engines, the position of
the waste gate is governed by a pressure-sensing control
mechanism coupled to an actuator. Engine oil directed into
or away from this actuator moves the waste gate position.
On these systems, the actuator is automatically positioned to
produce the desired MAP simply by changing the position
of the throttle control.
Other turbocharging system designs use a separate manual
control to position the waste gate. With manual control,
the manifold pressure gauge must be closely monitored to
determine when the desired MAP has been achieved. Manual
systems are often found on aircraft that have been modified
with aftermarket turbocharging systems. These systems
require special operating considerations. For example, if the
waste gate is left closed after descending from a high altitude,
it is possible to produce a manifold pressure that exceeds the
engine's limitations. This condition, called an overboost,
may produce severe detonation because of the leaning effect
resulting from increased air density during descent.
Although an automatic waste gate system is less likely to
experience an overboost condition, it can still occur. If takeoff
power is applied while the engine oil temperature is below its
normal operating range, the cold oil may not flow out of the
7-14

waste gate actuator quickly enough to prevent an overboost.
To help prevent overboosting, advance the throttle cautiously
to prevent exceeding the maximum manifold pressure limits.
A pilot flying an aircraft with a turbocharger should be aware
of system limitations. For example, a turbocharger turbine
and impeller can operate at rotational speeds in excess of
80,000 rpm while at extremely high temperatures. To achieve
high rotational speed, the bearings within the system must be
constantly supplied with engine oil to reduce the frictional
forces and high temperature. To obtain adequate lubrication,
the oil temperature should be in the normal operating range
before high throttle settings are applied. In addition, allow
the turbocharger to cool and the turbine to slow down before
shutting the engine down. Otherwise, the oil remaining in
the bearing housing will boil, causing hard carbon deposits
to form on the bearings and shaft. These deposits rapidly
deteriorate the turbocharger's efficiency and service life. For
further limitations, refer to the AFM/POH.

High Altitude Performance
As an aircraft equipped with a turbocharging system climbs,
the waste gate is gradually closed to maintain the maximum
allowable manifold pressure. At some point, the waste gate
is fully closed and further increases in altitude cause the
manifold pressure to decrease. This is the critical altitude,

which is established by the aircraft or engine manufacturer.
When evaluating the performance of the turbocharging
system, be aware that if the manifold pressure begins
decreasing before the specified critical altitude, the engine
and turbocharging system should be inspected by a qualified
aviation maintenance technician (AMT) to verify that the
system is operating properly.

Ignition System
In a spark ignition engine, the ignition system provides a
spark that ignites the fuel-air mixture in the cylinders and is
made up of magnetos, spark plugs, high-tension leads, and
an ignition switch. [Figure 7-16]
A magneto uses a permanent magnet to generate an electrical
current completely independent of the aircraft's electrical
system. The magneto generates sufficiently high voltage
to jump a spark across the spark plug gap in each cylinder.
The system begins to fire when the starter is engaged and the
crankshaft begins to turn. It continues to operate whenever
the crankshaft is rotating.
Most standard certificated aircraft incorporate a dual ignition
system with two individual magnetos, separate sets of wires,
and spark plugs to increase reliability of the ignition system.
Each magneto operates independently to fire one of the two
spark plugs in each cylinder. The firing of two spark plugs
improves combustion of the fuel-air mixture and results in a
Upper magneto wires
Lower magneto wires

slightly higher power output. If one of the magnetos fails, the
other is unaffected. The engine continues to operate normally,
although a slight decrease in engine power can be expected.
The same is true if one of the two spark plugs in a cylinder fails.
The operation of the magneto is controlled in the flight deck
by the ignition switch. The switch has five positions:
1.

OFF

2.

R (right)

3.

L (left)

4.

BOTH

5.

START

With RIGHT or LEFT selected, only the associated magneto
is activated. The system operates on both magnetos when
BOTH is selected.
A malfunctioning ignition system can be identified during
the pretakeoff check by observing the decrease in rpm that
occurs when the ignition switch is first moved from BOTH
to RIGHT and then from BOTH to LEFT. A small decrease
in engine rpm is normal during this check. The permissible
decrease is listed in the AFM or POH. If the engine stops
running when switched to one magneto or if the rpm drop
exceeds the allowable limit, do not fly the aircraft until
the problem is corrected. The cause could be fouled plugs,

Upper spark plugs

Left magneto

Lower spark plugs

2

1

4

3

Right magneto

Figure 7-16. Ignition system components.

7-15

broken or shorted wires between the magneto and the plugs,
or improperly timed firing of the plugs. It should be noted
that "no drop" in rpm is not normal, and in that instance, the
aircraft should not be flown.
Following engine shutdown, turn the ignition switch to the
OFF position. Even with the battery and master switches
OFF, the engine can fire and turn over if the ignition switch
is left ON and the propeller is moved because the magneto
requires no outside source of electrical power. Be aware of
the potential for serious injury in this situation.
Even with the ignition switch in the OFF position, if
the ground wire between the magneto and the ignition
switch becomes disconnected or broken, the engine could
accidentally start if the propeller is moved with residual fuel
in the cylinder. If this occurs, the only way to stop the engine
is to move the mixture lever to the idle cutoff position, then
have the system checked by a qualified AMT.

Oil Systems
The engine oil system performs several important functions:


Lubrication of the engine's moving parts



Cooling of the engine by reducing friction



Removing heat from the cylinders



Providing a seal between the cylinder walls and pistons



Carrying away contaminants

Reciprocating engines use either a wet-sump or a dry-sump
oil system. In a wet-sump system, the oil is located in a sump
that is an integral part of the engine. In a dry-sump system,
the oil is contained in a separate tank and circulated through
the engine by pumps. [Figure 7-17]
The main component of a wet-sump system is the oil pump,
which draws oil from the sump and routes it to the engine. After
the oil passes through the engine, it returns to the sump. In
some engines, additional lubrication is supplied by the rotating
crankshaft, which splashes oil onto portions of the engine.
An oil pump also supplies oil pressure in a dry-sump
system, but the source of the oil is located external to the
engine in a separate oil tank. After oil is routed through
the engine, it is pumped from the various locations in the
engine back to the oil tank by scavenge pumps. Dry-sump
systems allow for a greater volume of oil to be supplied to
the engine, which makes them more suitable for very large
reciprocating engines.
The oil pressure gauge provides a direct indication of the oil
system operation. It ensures the pressure in pounds per square
inch (psi) of the oil supplied to the engine. Green indicates
the normal operating range, while red indicates the minimum
and maximum pressures. There should be an indication of
oil pressure during engine start. Refer to the AFM/POH for
manufacturer limitations.

Oil filler cap and dipstick

Engine
and
Accessory
Bearings

Sump oil and return
oil from relief valve
Pressure oil from
oil pump

Oil sump

Low pressure oil screen

Oil pressure relief valve

Oil pump
High pressure oil screen
Oil cooler and filter

245

75

Figure 7-17. Wet-sump oil system.

7-16

II5
I00 P
P R
60 S E

T 200
E °F
M I50
P I00

I
20 S

S

OIL

0

The oil temperature gauge measures the temperature of oil.
A green area shows the normal operating range, and the red
line indicates the maximum allowable temperature. Unlike
oil pressure, changes in oil temperature occur more slowly.
This is particularly noticeable after starting a cold engine,
when it may take several minutes or longer for the gauge to
show any increase in oil temperature.

Air cooling is accomplished by air flowing into the engine
compartment through openings in front of the engine
cowling. Baffles route this air over fins attached to the engine
cylinders, and other parts of the engine, where the air absorbs
the engine heat. Expulsion of the hot air takes place through
one or more openings in the lower, aft portion of the engine
cowling. [Figure 7-19]

Check oil temperature periodically during flight especially
when operating in high or low ambient air temperature.
High oil temperature indications may signal a plugged oil
line, a low oil quantity, a blocked oil cooler, or a defective
temperature gauge. Low oil temperature indications may
signal improper oil viscosity during cold weather operations.

The outside air enters the engine compartment through an
inlet behind the propeller hub. Baffles direct it to the hottest
parts of the engine, primarily the cylinders, which have fins
that increase the area exposed to the airflow.

The oil filler cap and dipstick (for measuring the oil quantity)
are usually accessible through a panel in the engine cowling. If
the quantity does not meet the manufacturer's recommended
operating levels, oil should be added. The AFM/POH or
placards near the access panel provide information about
the correct oil type and weight, as well as the minimum and
maximum oil quantity. [Figure 7-18]

Engine Cooling Systems
The burning fuel within the cylinders produces intense
heat, most of which is expelled through the exhaust system.
Much of the remaining heat, however, must be removed, or
at least dissipated, to prevent the engine from overheating.
Otherwise, the extremely high engine temperatures can lead
to loss of power, excessive oil consumption, detonation, and
serious engine damage.
While the oil system is vital to the internal cooling of the
engine, an additional method of cooling is necessary for the
engine's external surface. Most small aircraft are air cooled,
although some are liquid cooled.

The air cooling system is less effective during ground
operations, takeoffs, go-arounds, and other periods of highpower, low-airspeed operation. Conversely, high-speed
descents provide excess air and can shock cool the engine,
subjecting it to abrupt temperature fluctuations.
Operating the engine at higher than its designed temperature
can cause loss of power, excessive oil consumption, and
detonation. It will also lead to serious permanent damage,
such as scoring the cylinder walls, damaging the pistons and
rings, and burning and warping the valves. Monitoring the
flight deck engine temperature instruments aids in avoiding
high operating temperature.
Under normal operating conditions in aircraft not equipped
with cowl flaps, the engine temperature can be controlled

Cylinders
Baffle

Air inlet

Baffle

Fixed cowl opening

Figure 7-18. Always check the engine oil level during the preflight

Figure 7-19. Outside air aids in cooling the engine.

inspection.

7-17

by changing the airspeed or the power output of the engine.
High engine temperatures can be decreased by increasing the
airspeed and/or reducing the power.
The oil temperature gauge gives an indirect and delayed
indication of rising engine temperature, but can be used
for determining engine temperature if this is the only
means available.
Most aircraft are equipped with a cylinder-head temperature
gauge that indicates a direct and immediate cylinder
temperature change. This instrument is calibrated in degrees
Celsius or Fahrenheit and is usually color coded with a green
arc to indicate the normal operating range. A red line on
the instrument indicates maximum allowable cylinder head
temperature.
To avoid excessive cylinder head temperatures, increase
airspeed, enrich the fuel-air mixture, and/or reduce
power. Any of these procedures help to reduce the engine
temperature. On aircraft equipped with cowl flaps, use the
cowl flap positions to control the temperature. Cowl flaps
are hinged covers that fit over the opening through which the
hot air is expelled. If the engine temperature is low, the cowl
flaps can be closed, thereby restricting the flow of expelled
hot air and increasing engine temperature. If the engine
temperature is high, the cowl flaps can be opened to permit
a greater flow of air through the system, thereby decreasing
the engine temperature.

Exhaust Systems
Engine exhaust systems vent the burned combustion gases
overboard, provide heat for the cabin, and defrost the
windscreen. An exhaust system has exhaust piping attached
to the cylinders, as well as a muffler and a muffler shroud.
The exhaust gases are pushed out of the cylinder through
the exhaust valve and then through the exhaust pipe system
to the atmosphere.
For cabin heat, outside air is drawn into the air inlet and is
ducted through a shroud around the muffler. The muffler is
heated by the exiting exhaust gases and, in turn, heats the
air around the muffler. This heated air is then ducted to the
cabin for heat and defrost applications. The heat and defrost
are controlled in the flight deck and can be adjusted to the
desired level.
Exhaust gases contain large amounts of carbon monoxide,
which is odorless and colorless. Carbon monoxide is deadly,
and its presence is virtually impossible to detect. To ensure
that exhaust gases are properly expelled, the exhaust system
must be in good condition and free of cracks.

7-18

Some exhaust systems have an EGT probe. This probe
transmits the EGT to an instrument in the flight deck. The
EGT gauge measures the temperature of the gases at the
exhaust manifold. This temperature varies with the ratio of
fuel to air entering the cylinders and can be used as a basis
for regulating the fuel-air mixture. The EGT gauge is highly
accurate in indicating the correct fuel-air mixture setting.
When using the EGT to aid in leaning the fuel-air mixture,
fuel consumption can be reduced. For specific procedures,
refer to the manufacturer's recommendations for leaning the
fuel-air mixture.

Starting System
Most small aircraft use a direct-cranking electric starter
system. This system consists of a source of electricity, wiring,
switches, and solenoids to operate the starter and a starter
motor. Most aircraft have starters that automatically engage
and disengage when operated, but some older aircraft have
starters that are mechanically engaged by a lever actuated by
the pilot. The starter engages the aircraft flywheel, rotating
the engine at a speed that allows the engine to start and
maintain operation.
Electrical power for starting is usually supplied by an onboard
battery, but can also be supplied by external power through
an external power receptacle. When the battery switch is
turned on, electricity is supplied to the main power bus bar
through the battery solenoid. Both the starter and the starter
switch draw current from the main bus bar, but the starter
will not operate until the starting solenoid is energized by
the starter switch being turned to the "start" position. When
the starter switch is released from the "start" position, the
solenoid removes power from the starter motor. The starter
motor is protected from being driven by the engine through a
clutch in the starter drive that allows the engine to run faster
than the starter motor. [Figure 7-20]
When starting an engine, the rules of safety and courtesy
should be strictly observed. One of the most important safety
rules is to ensure there is no one near the propeller prior to
starting the engine. In addition, the wheels should be chocked
and the brakes set to avoid hazards caused by unintentional
movement. To avoid damage to the propeller and property,
the aircraft should be in an area where the propeller will not
stir up gravel or dust.

Combustion
During normal combustion, the fuel-air mixture burns in a
very controlled and predictable manner. In a spark ignition
engine, the process occurs in a fraction of a second. The
mixture actually begins to burn at the point where it is ignited

External
power plug

External
power
relay

+
+

M
A
I
N

tery

Bat

r

rte

Sta

B
U
S

Normal combustion

Explosion

Figure 7-21. Normal combustion and explosive combustion.
Battery
contactor
(solenoid)

A
L
T

B
A
T

Starter
contactor

L

R

B

OFF

S

Ignition switch



Detonation may be avoided by following these basic
guidelines during the various phases of ground and flight
operations:


Ensure that the proper grade of fuel is used.



Keep the cowl flaps (if available) in the full-open
position while on the ground to provide the maximum
airflow through the cowling.



Use an enriched fuel mixture, as well as a shallow
climb angle, to increase cylinder cooling during
takeoff and initial climb.



Avoid extended, high power, steep climbs.



Develop the habit of monitoring the engine instruments
to verify proper operation according to procedures
established by the manufacturer.

Figure 7-20. Typical starting circuit.

by the spark plugs. It then burns away from the plugs until it
is completely consumed. This type of combustion causes a
smooth build-up of temperature and pressure and ensures that
the expanding gases deliver the maximum force to the piston
at exactly the right time in the power stroke. [Figure 7-21]
Detonation is an uncontrolled, explosive ignition of the
fuel-air mixture within the cylinder's combustion chamber.
It causes excessive temperatures and pressures which, if not
corrected, can quickly lead to failure of the piston, cylinder,
or valves. In less severe cases, detonation causes engine
overheating, roughness, or loss of power.
Detonation is characterized by high cylinder head temperatures
and is most likely to occur when operating at high power
settings. Common operational causes of detonation are:


Use of a lower fuel grade than that specified by the
aircraft manufacturer



Operation of the engine with extremely high manifold
pressures in conjunction with low rpm



Operation of the engine at high power settings with
an excessively lean mixture

Maintaining extended ground operations or steep
climbs in which cylinder cooling is reduced

Preignition occurs when the fuel-air mixture ignites prior
to the engine's normal ignition event. Premature burning
is usually caused by a residual hot spot in the combustion
chamber, often created by a small carbon deposit on a spark
plug, a cracked spark plug insulator, or other damage in the
cylinder that causes a part to heat sufficiently to ignite the
fuel-air charge. Preignition causes the engine to lose power
and produces high operating temperature. As with detonation,
preignition may also cause severe engine damage because
the expanding gases exert excessive pressure on the piston
while still on its compression stroke.

7-19

Detonation and preignition often occur simultaneously and
one may cause the other. Since either condition causes high
engine temperature accompanied by a decrease in engine
performance, it is often difficult to distinguish between the
two. Using the recommended grade of fuel and operating
the engine within its proper temperature, pressure, and rpm
ranges reduce the chance of detonation or preignition.

Full Authority Digital Engine Control
(FADEC)
FADEC is a system consisting of a digital computer and
ancillary components that control an aircraft's engine
and propeller. First used in turbine-powered aircraft, and
referred to as full authority digital electronic control, these
sophisticated control systems are increasingly being used in
piston powered aircraft.
In a spark-ignition reciprocating engine, the FADEC uses
speed, temperature, and pressure sensors to monitor the status
of each cylinder. A digital computer calculates the ideal pulse
for each injector and adjusts ignition timing as necessary
to achieve optimal performance. In a compression-ignition
engine, the FADEC operates similarly and performs all of
the same functions, excluding those specifically related to
the spark ignition process.
FADEC systems eliminate the need for magnetos, carburetor
heat, mixture controls, and engine priming. A single throttle
lever is characteristic of an aircraft equipped with a FADEC
system. The pilot simply positions the throttle lever to a
desired detent, such as start, idle, cruise power, or max power,
and the FADEC system adjusts the engine and propeller
automatically for the mode selected. There is no need for the
pilot to monitor or control the fuel-air mixture.
During aircraft starting, the FADEC primes the cylinders,
adjusts the mixture, and positions the throttle based on engine
temperature and ambient pressure. During cruise flight, the
FADEC constantly monitors the engine and adjusts fuel flow
and ignition timing individually in each cylinder. This precise
control of the combustion process often results in decreased
fuel consumption and increased horsepower.
FADEC systems are considered an essential part of the
engine and propeller control and may be powered by the
aircraft's main electrical system. In many aircraft, FADEC
uses power from a separate generator connected to the engine.
In either case, there must be a backup electrical source
available because failure of a FADEC system could result in a
complete loss of engine thrust. To prevent loss of thrust, two
separate and identical digital channels are incorporated for
redundancy. Each channel is capable of providing all engine
and propeller functions without limitations.
7-20

Turbine Engines
An aircraft turbine engine consists of an air inlet, compressor,
combustion chambers, a turbine section, and exhaust. Thrust
is produced by increasing the velocity of the air flowing
through the engine. Turbine engines are highly desirable
aircraft powerplants. They are characterized by smooth
operation and a high power-to-weight ratio, and they
use readily available jet fuel. Prior to recent advances in
material, engine design, and manufacturing processes, the
use of turbine engines in small/light production aircraft was
cost prohibitive. Today, several aviation manufacturers are
producing or plan to produce small/light turbine-powered
aircraft. These smaller turbine-powered aircraft typically
seat between three and seven passengers and are referred to
as very light jets (VLJs) or microjets. [Figure 7-22]
Types of Turbine Engines
Turbine engines are classified according to the type of
compressors they use. There are three types of compressors—
centrifugal flow, axial flow, and centrifugal-axial flow.
Compression of inlet air is achieved in a centrifugal flow
engine by accelerating air outward perpendicular to the
longitudinal axis of the machine. The axial-flow engine
compresses air by a series of rotating and stationary
airfoils moving the air parallel to the longitudinal axis. The
centrifugal-axial flow design uses both kinds of compressors
to achieve the desired compression.
The path the air takes through the engine and how power is
produced determines the type of engine. There are four types
of aircraft turbine engines—turbojet, turboprop, turbofan,
and turboshaft.

Turbojet
The turbojet engine consists of four sections—compressor,
combustion chamber, turbine section, and exhaust. The
compressor section passes inlet air at a high rate of speed to

Figure 7-22. Eclipse 500 VLJ.

the combustion chamber. The combustion chamber contains
the fuel inlet and igniter for combustion. The expanding
air drives a turbine, which is connected by a shaft to the
compressor, sustaining engine operation. The accelerated
exhaust gases from the engine provide thrust. This is a basic
application of compressing air, igniting the fuel-air mixture,
producing power to self-sustain the engine operation, and
exhaust for propulsion. [Figure 7-23]
Turbojet engines are limited in range and endurance. They
are also slow to respond to throttle applications at slow
compressor speeds.

Turboprop
A turboprop engine is a turbine engine that drives a propeller
through a reduction gear. The exhaust gases drive a power
turbine connected by a shaft that drives the reduction gear
assembly. Reduction gearing is necessary in turboprop
engines because optimum propeller performance is achieved
at much slower speeds than the engine's operating rpm.
Turboprop engines are a compromise between turbojet
engines and reciprocating powerplants. Turboprop engines
are most efficient at speeds between 250 and 400 mph and
altitudes between 18,000 and 30,000 feet. They also perform
well at the slow airspeeds required for takeoff and landing and
are fuel efficient. The minimum specific fuel consumption
Fuel injector

of the turboprop engine is normally available in the altitude
range of 25,000 feet to the tropopause. [Figure 7-24]

Turbofan
Turbofans were developed to combine some of the best
features of the turbojet and the turboprop. Turbofan engines
are designed to create additional thrust by diverting a
secondary airflow around the combustion chamber. The
turbofan bypass air generates increased thrust, cools the
engine, and aids in exhaust noise suppression. This provides
turbojet-type cruise speed and lower fuel consumption.
The inlet air that passes through a turbofan engine is usually
divided into two separate streams of air. One stream passes
through the engine core, while a second stream bypasses the
engine core. It is this bypass stream of air that is responsible
for the term "bypass engine." A turbofan's bypass ratio refers
to the ratio of the mass airflow that passes through the fan
divided by the mass airflow that passes through the engine
core. [Figure 7-25]

Turboshaft
The fourth common type of jet engine is the turboshaft.
[Figure 7-26] It delivers power to a shaft that drives
something other than a propeller. The biggest difference
between a turbojet and turboshaft engine is that on a

Turbine

Inlet

Hot gases

Compressor

Combustion chamber

Nozzle

Figure 7-23. Turbojet engine.

Inlet

Prop

Gear box

Compressor

Combustion chamber

Fuel injector

Exhaust

Turbine

Figure 7-24. Turboprop engine.

7-21

Inlet

Fuel injector

Duct fan

Turbine
Hot gases

Primary air stream

Secondary air stream
Compressor

Combustion chamber

Nozzle

Figure 7-25. Turbofan engine.
Inlet

Compressor

Combustion chamber

Exhaust

Power shaft

Compressor turbine
Free (power) turbine
Figure 7-26. Turboshaft engine.

turboshaft engine, most of the energy produced by the
expanding gases is used to drive a turbine rather than produce
thrust. Many helicopters use a turboshaft gas turbine engine.
In addition, turboshaft engines are widely used as auxiliary
power units on large aircraft.
Turbine Engine Instruments
Engine instruments that indicate oil pressure, oil temperature,
engine speed, exhaust gas temperature, and fuel flow are
common to both turbine and reciprocating engines. However,
there are some instruments that are unique to turbine engines.
These instruments provide indications of engine pressure
ratio, turbine discharge pressure, and torque. In addition,
most gas turbine engines have multiple temperature-sensing
instruments, called thermocouples, which provide pilots with
temperature readings in and around the turbine section.

Engine Pressure Ratio (EPR)
An engine pressure ratio (EPR) gauge is used to indicate the
power output of a turbojet/turbofan engine. EPR is the ratio
of turbine discharge to compressor inlet pressure. Pressure
measurements are recorded by probes installed in the engine
inlet and at the exhaust. Once collected, the data is sent to
a differential pressure transducer, which is indicated on a
flight deck EPR gauge.
7-22

EPR system design automatically compensates for the effects
of airspeed and altitude. Changes in ambient temperature
require a correction be applied to EPR indications to provide
accurate engine power settings.

Exhaust Gas Temperature (EGT)
A limiting factor in a gas turbine engine is the temperature
of the turbine section. The temperature of a turbine section
must be monitored closely to prevent overheating the turbine
blades and other exhaust section components. One common
way of monitoring the temperature of a turbine section is
with an EGT gauge. EGT is an engine operating limit used
to monitor overall engine operating conditions.
Variations of EGT systems bear different names based on
the location of the temperature sensors. Common turbine
temperature sensing gauges include the turbine inlet
temperature (TIT) gauge, turbine outlet temperature (TOT)
gauge, interstage turbine temperature (ITT) gauge, and
turbine gas temperature (TGT) gauge.

Torquemeter
Turboprop/turboshaft engine power output is measured
by the torquemeter. Torque is a twisting force applied to a
shaft. The torquemeter measures power applied to the shaft.

Turboprop and turboshaft engines are designed to produce
torque for driving a propeller. Torquemeters are calibrated
in percentage units, foot-pounds, or psi.

N1 Indicator
N1 represents the rotational speed of the low pressure
compressor and is presented on the indicator as a percentage
of design rpm. After start, the speed of the low pressure
compressor is governed by the N1 turbine wheel. The N1
turbine wheel is connected to the low pressure compressor
through a concentric shaft.

N2 Indicator
N2 represents the rotational speed of the high pressure
compressor and is presented on the indicator as a percentage of
design rpm. The high pressure compressor is governed by the
N2 turbine wheel. The N2 turbine wheel is connected to the high
pressure compressor through a concentric shaft. [Figure 7-27]
Turbine Engine Operational Considerations
The great variety of turbine engines makes it impractical to
cover specific operational procedures, but there are certain
operational considerations common to all turbine engines.
They are engine temperature limits, foreign object damage,
hot start, compressor stall, and flameout.

Engine Temperature Limitations
The highest temperature in any turbine engine occurs at the
turbine inlet. TIT is therefore usually the limiting factor in
turbine engine operation.

Thrust Variations
Turbine engine thrust varies directly with air density. As air
density decreases, so does thrust. Additionally, because air
density decreases with an increase in temperature, increased

temperatures also results in decreased thrust. While both
turbine and reciprocating powered engines are affected to
some degree by high relative humidity, turbine engines will
experience a negligible loss of thrust, while reciprocating
engines a significant loss of brake horsepower.

Foreign Object Damage (FOD)
Due to the design and function of a turbine engine's air inlet,
the possibility of ingestion of debris always exists. This
causes significant damage, particularly to the compressor
and turbine sections. When ingestion of debris occurs, it is
called foreign object damage (FOD). Typical FOD consists
of small nicks and dents caused by ingestion of small objects
from the ramp, taxiway, or runway, but FOD damage caused
by bird strikes or ice ingestion also occur. Sometimes FOD
results in total destruction of an engine.
Prevention of FOD is a high priority. Some engine inlets
have a tendency to form a vortex between the ground and
the inlet during ground operations. A vortex dissipater may
be installed on these engines. Other devices, such as screens
and/or deflectors, may also be utilized. Preflight procedures
include a visual inspection for any sign of FOD.

Turbine Engine Hot/Hung Start
When the EGT exceeds the safe limit of an aircraft, it
experiences a "hot start." This is caused by too much fuel
entering the combustion chamber or insufficient turbine rpm.
Any time an engine has a hot start, refer to the AFM/POH or an
appropriate maintenance manual for inspection requirements.
If the engine fails to accelerate to the proper speed after
ignition or does not accelerate to idle rpm, a hung or false start
has occurred. A hung start may be caused by an insufficient
starting power source or fuel control malfunction.

Compressor Stalls
Low pressure
compressor (N1)

High pressure
compressor (N2)

High pressure compressor drive shaft
Low pressure compressor drive shaft
Figure 7-27. Dual-spool axial-flow compressor.

Compressor blades are small airfoils and are subject to the
same aerodynamic principles that apply to any airfoil. A
compressor blade has an AOA that is a result of inlet air
velocity and the compressor's rotational velocity. These two
forces combine to form a vector, which defines the airfoil's
actual AOA to the approaching inlet air.
A compressor stall is an imbalance between the two vector
quantities, inlet velocity, and compressor rotational speed.
Compressor stalls occur when the compressor blades' AOA
exceeds the critical AOA. At this point, smooth airflow
is interrupted and turbulence is created with pressure
fluctuations. Compressor stalls cause air flowing in the
compressor to slow down and stagnate, sometimes reversing
direction. [Figure 7-28]

7-23

Normal inlet airflow

Distorted inlet airflow

Figure 7-28. Comparison of normal and distorted airflow into the
compressor section.

Compressor stalls can be transient and intermittent or steady
and severe. Indications of a transient/intermittent stall are
usually an intermittent "bang" as backfire and flow reversal
take place. If the stall develops and becomes steady, strong
vibration and a loud roar may develop from the continuous
flow reversal. Often, the flight deck gauges do not show
a mild or transient stall, but they do indicate a developed
stall. Typical instrument indications include fluctuations
in rpm and an increase in exhaust gas temperature. Most
transient stalls are not harmful to the engine and often correct
themselves after one or two pulsations. The possibility of
severe engine damage from a steady state stall is immediate.
Recovery must be accomplished by quickly reducing power,
decreasing the aircraft's AOA, and increasing airspeed.
Although all gas turbine engines are subject to compressor
stalls, most models have systems that inhibit them. One
system uses a variable inlet guide vane (VIGV) and variable
stator vanes that direct the incoming air into the rotor blades
at an appropriate angle. To prevent air pressure stalls,
operate the aircraft within the parameters established by the
manufacturer. If a compressor stall does develop, follow the
procedures recommended in the AFM/POH.

Flameout
A flameout occurs in the operation of a gas turbine engine in
which the fire in the engine unintentionally goes out. If the
rich limit of the fuel-air ratio is exceeded in the combustion
chamber, the flame will blow out. This condition is often
referred to as a rich flameout. It generally results from
very fast engine acceleration where an overly rich mixture
causes the fuel temperature to drop below the combustion
temperature. It may also be caused by insufficient airflow
to support combustion.

7-24

A more common flameout occurrence is due to low fuel
pressure and low engine speeds, which typically are
associated with high-altitude flight. This situation may also
occur with the engine throttled back during a descent, which
can set up the lean-condition flameout. A weak mixture can
easily cause the flame to die out, even with a normal airflow
through the engine.
Any interruption of the fuel supply can result in a
flameout. This may be due to prolonged unusual attitudes,
a malfunctioning fuel control system, turbulence, icing, or
running out of fuel.
Symptoms of a flameout normally are the same as those
following an engine failure. If the flameout is due to a
transitory condition, such as an imbalance between fuel
flow and engine speed, an airstart may be attempted once
the condition is corrected. In any case, pilots must follow
the applicable emergency procedures outlined in the AFM/
POH. Generally these procedures contain recommendations
concerning altitude and airspeed where the airstart is most
likely to be successful.
Performance Comparison
It is possible to compare the performance of a reciprocating
powerplant and different types of turbine engines. For
the comparison to be accurate, thrust horsepower (usable
horsepower) for the reciprocating powerplant must be used
rather than brake horsepower, and net thrust must be used
for the turbine-powered engines. In addition, aircraft design
configuration and size must be approximately the same.
When comparing performance, the following definitions
are useful:


Brake horsepower (BHP)—the horsepower actually
delivered to the output shaft. Brake horsepower is the
actual usable horsepower.



Net thrust—the thrust produced by a turbojet or
turbofan engine.



Thrust horsepower (THP)—the horsepower equivalent
of the thrust produced by a turbojet or turbofan engine.

Equivalent shaft horsepower (ESHP)—with respect
to turboprop engines, the sum of the shaft horsepower
(SHP) delivered to the propeller and THP produced by the
exhaust gases.
Figure 7-29 shows how four types of engines compare in net
thrust as airspeed is increased. This figure is for explanatory

maximum speed than aircraft equipped with a turboprop or
reciprocating powerplant.
Reciprocating

Airframe Systems

Turboprop

Air
c ra
ft d

rag

Turbofan

Net thrust

Turbojet

A B

C D

E

F

Fuel, electrical, hydraulic, and oxygen systems make up the
airframe systems.

Fuel Systems
The fuel system is designed to provide an uninterrupted
flow of clean fuel from the fuel tanks to the engine. The
fuel must be available to the engine under all conditions
of engine power, altitude, attitude, and during all approved
flight maneuvers. Two common classifications apply to fuel
systems in small aircraft: gravity-feed and fuel-pump systems.

Gravity-Feed System
The gravity-feed system utilizes the force of gravity to
Figure 7-29. Engine net thrust versus aircraft speed and drag. Points transfer the fuel from the tanks to the engine. For example, on
high-wing airplanes, the fuel tanks are installed in the wings.
a through f are explained in the text below.
This places the fuel tanks above the carburetor, and the fuel
is gravity fed through the system and into the carburetor. If
purposes only and is not for specific models of engines. The the design of the aircraft is such that gravity cannot be used
following are the four types of engines:
to transfer fuel, fuel pumps are installed. For example, on
 Reciprocating powerplant
low-wing airplanes, the fuel tanks in the wings are located
below the carburetor. [Figure 7-30]
 Turbine, propeller combination (turboprop)
Airspeed



Turbine engine incorporating a fan (turbofan)



Turbojet (pure jet)

By plotting the performance curve for each engine, a
comparison can be made of maximum aircraft speed variation
with the type of engine used. Since the graph is only a means
of comparison, numerical values for net thrust, aircraft speed,
and drag are not included.
Comparison of the four powerplants on the basis of net thrust
makes certain performance capabilities evident. In the speed
range shown to the left of line A, the reciprocating powerplant
outperforms the other three types. The turboprop outperforms
the turbofan in the range to the left of line C. The turbofan
engine outperforms the turbojet in the range to the left of
line F. The turbofan engine outperforms the reciprocating
powerplant to the right of line B and the turboprop to the
right of line C. The turbojet outperforms the reciprocating
powerplant to the right of line D, the turboprop to the right
of line E, and the turbofan to the right of line F.
The points where the aircraft drag curve intersects the net
thrust curves are the maximum aircraft speeds. The vertical
lines from each of the points to the baseline of the graph
indicate that the turbojet aircraft can attain a higher maximum
speed than aircraft equipped with the other types of engines.
Aircraft equipped with the turbofan engine attains a higher

Fuel-Pump System
Aircraft with fuel-pump systems have two fuel pumps. The
main pump system is engine driven with an electricallydriven auxiliary pump provided for use in engine starting
and in the event the engine pump fails. The auxiliary pump,
also known as a boost pump, provides added reliability to
the fuel system. The electrically-driven auxiliary pump is
controlled by a switch in the flight deck.
Fuel Primer
Both gravity-feed and fuel-pump systems may incorporate a
fuel primer into the system. The fuel primer is used to draw
fuel from the tanks to vaporize fuel directly into the cylinders
prior to starting the engine. During cold weather, when
engines are difficult to start, the fuel primer helps because
there is not enough heat available to vaporize the fuel in the
carburetor. It is important to lock the primer in place when
it is not in use. If the knob is free to move, it may vibrate
out of position during flight which may cause an excessively
rich fuel-air mixture. To avoid overpriming, read the priming
instructions for the aircraft.
Fuel Tanks
The fuel tanks, normally located inside the wings of an
airplane, have a filler opening on top of the wing through
which they can be filled. A filler cap covers this opening.

7-25

accuracy in fuel gauges only when they read "empty." Any
reading other than "empty" should be verified. Do not depend
solely on the accuracy of the fuel quantity gauges. Always
visually check the fuel level in each tank during the preflight
inspection, and then compare it with the corresponding fuel
quantity indication.

nk

ht ta

Rig

tank

BOTH

LEFT

RIGHT

Left

Strainer

OFF

Vent
Selector valve

Carburetor

Primer
Gravity-feed system
Engine-driven pump

Carburetor

If a fuel pump is installed in the fuel system, a fuel pressure
gauge is also included. This gauge indicates the pressure in
the fuel lines. The normal operating pressure can be found
in the AFM/POH or on the gauge by color coding.
Fuel Selectors
The fuel selector valve allows selection of fuel from various
tanks. A common type of selector valve contains four
positions: LEFT, RIGHT, BOTH, and OFF. Selecting the
LEFT or RIGHT position allows fuel to feed only from the
respective tank, while selecting the BOTH position feeds
fuel from both tanks. The LEFT or RIGHT position may be
used to balance the amount of fuel remaining in each wing
tank. [Figure 7-31]
Fuel placards show any limitations on fuel tank usage, such
as "level flight only" and/or "both" for landings and takeoffs.

Strainer
Electric pump

Primer

Selector valve

BOTH

LEFT

Lef

RIGHT

k
t tan

nk

ht ta

Rig

OFF

Fuel-pump system

Regardless of the type of fuel selector in use, fuel
consumption should be monitored closely to ensure that a
tank does not run completely out of fuel. Running a fuel tank
dry does not only cause the engine to stop, but running for
prolonged periods on one tank causes an unbalanced fuel load
between tanks. Running a tank completely dry may allow air
to enter the fuel system and cause vapor lock, which makes
it difficult to restart the engine. On fuel-injected engines, the
fuel becomes so hot it vaporizes in the fuel line, not allowing
fuel to reach the cylinders.

Figure 7-30. Gravity-feed and fuel-pump systems.

The tanks are vented to the outside to maintain atmospheric
pressure inside the tank. They may be vented through the
filler cap or through a tube extending through the surface
of the wing. Fuel tanks also include an overflow drain that
may stand alone or be collocated with the fuel tank vent.
This allows fuel to expand with increases in temperature
without damage to the tank itself. If the tanks have been
filled on a hot day, it is not unusual to see fuel coming from
the overflow drain.
Fuel Gauges
The fuel quantity gauges indicate the amount of fuel
measured by a sensing unit in each fuel tank and is displayed
in gallons or pounds. Aircraft certification rules require
7-26

BOTH

OFF 38 GA LANDIN
L ATT
KE
HT
ITU G
TA FLIG
DE
L
S
L
A

LEFT

RIGHT

19 gal
LEVEL
FLIGHT
ONLY

19 gal
LEVEL
FLIGHT
ONLY

OFF

Figure 7-31. Fuel selector valve.

Fuel Strainers, Sumps, and Drains
After leaving the fuel tank and before it enters the carburetor,
the fuel passes through a strainer that removes any moisture
and other sediments in the system. Since these contaminants
are heavier than aviation fuel, they settle in a sump at the
bottom of the strainer assembly. A sump is a low point in a fuel
system and/or fuel tank. The fuel system may contain a sump,
a fuel strainer, and fuel tank drains, which may be collocated.
The fuel strainer should be drained before each flight. Fuel
samples should be drained and checked visually for water
and contaminants.
Water in the sump is hazardous because in cold weather the
water can freeze and block fuel lines. In warm weather, it
can flow into the carburetor and stop the engine. If water is
present in the sump, more water in the fuel tanks is probable,
and they should be drained until there is no evidence of water.
Never take off until all water and contaminants have been
removed from the engine fuel system.
Because of the variation in fuel systems, become thoroughly
familiar with the systems that apply to the aircraft being flown.
Consult the AFM/POH for specific operating procedures.
Fuel Grades
Aviation gasoline (AVGAS) is identified by an octane or
performance number (grade), which designates the antiknock
value or knock resistance of the fuel mixture in the engine
cylinder. The higher the grade of gasoline, the more pressure
the fuel can withstand without detonating. Lower grades of
fuel are used in lower-compression engines because these
fuels ignite at a lower temperature. Higher grades are used
in higher-compression engines because they ignite at higher
temperatures, but not prematurely. If the proper grade of fuel
is not available, use the next higher grade as a substitute. Never
use a grade lower than recommended. This can cause the
cylinder head temperature and engine oil temperature to exceed
their normal operating ranges, which may result in detonation.
Several grades of AVGAS are available. Care must be
exercised to ensure that the correct aviation grade is being
used for the specific type of engine. The proper fuel grade is
stated in the AFM/POH, on placards in the flight deck, and
next to the filler caps. Automobile gas should NEVER be
used in aircraft engines unless the aircraft has been modified
with a Supplemental Type Certificate (STC) issued by the
Federal Aviation Administration (FAA).
The current method identifies AVGAS for aircraft with
reciprocating engines by the octane and performance number,
along with the abbreviation AVGAS. These aircraft use
AVGAS 80, 100, and 100LL. Although AVGAS 100LL

performs the same as grade 100, the "LL" indicates it has
a low lead content. Fuel for aircraft with turbine engines is
classified as JET A, JET A-1, and JET B. Jet fuel is basically
kerosene and has a distinctive kerosene smell. Since use of
the correct fuel is critical, dyes are added to help identify the
type and grade of fuel. [Figure 7-32]
In addition to the color of the fuel itself, the color-coding
system extends to decals and various airport fuel handling
equipment. For example, all AVGAS is identified by name,
using white letters on a red background. In contrast, turbine
fuels are identified by white letters on a black background.
Special Airworthiness Information Bulleting (SAIB)
NE-11-15 advises that grade 100VLL AVGAS is acceptable
for use on aircraft and engines. 100VLL meets all
performance requirements of grades 80, 91, 100, and 100LL;
meets the approved operating limitations for aircraft and
engines certificated to operate with these other grades of
AVGAS; and is basically identical to 100LL AVGAS. The
lead content of 100VLL is reduced by about 19 percent.
100VLL is blue like 100LL and virtually indistinguishable.
Fuel Contamination
Accidents attributed to powerplant failure from fuel
contamination have often been traced to:


Inadequate preflight inspection by the pilot



Servicing aircraft with improperly filtered fuel from
small tanks or drums



Storing aircraft with partially filled fuel tanks



Lack of proper maintenance

Fuel should be drained from the fuel strainer quick drain and
from each fuel tank sump into a transparent container and
then checked for dirt and water. When the fuel strainer is
being drained, water in the tank may not appear until all the
fuel has been drained from the lines leading to the tank. This
indicates that water remains in the tank and is not forcing the
fuel out of the fuel lines leading to the fuel strainer. Therefore,
drain enough fuel from the fuel strainer to be certain that
fuel is being drained from the tank. The amount depends on

80
AVGAS

100
AVGAS

100LL
AVGAS

JET
A

RED

GREEN

BLUE

COLORLESS
OR STRAW

AVGAS
80

AVGAS
100

AVGAS
100LL

JET A

Figure 7-32. Aviation fuel color-coding system.

7-27

the length of fuel line from the tank to the drain. If water or
other contaminants are found in the first sample, drain further
samples until no trace appears.
Water may also remain in the fuel tanks after the drainage
from the fuel strainer has ceased to show any trace of water.
This residual water can be removed only by draining the fuel
tank sump drains.
Water is the principal fuel contaminant. Suspended water
droplets in the fuel can be identified by a cloudy appearance
of the fuel, or by the clear separation of water from the colored
fuel, which occurs after the water has settled to the bottom
of the tank. As a safety measure, the fuel sumps should be
drained before every flight during the preflight inspection.
Fuel tanks should be filled after each flight or after the last
flight of the day to prevent moisture condensation within the
tank. To prevent fuel contamination, avoid refueling from
cans and drums.
In remote areas or in emergency situations, there may be no
alternative to refueling from sources with inadequate anticontamination systems. While a chamois skin and funnel
may be the only possible means of filtering fuel, using
them is hazardous. Remember, the use of a chamois does
not always ensure decontaminated fuel. Worn-out chamois
do not filter water; neither will a new, clean chamois that is
already water-wet or damp. Most imitation chamois skins
do not filter water.
Fuel System Icing
Ice formation in the aircraft fuel system results from the
presence of water in the fuel system. This water may be
undissolved or dissolved. One condition of undissolved
water is entrained water that consists of minute water
particles suspended in the fuel. This may occur as a result of
mechanical agitation of free water or conversion of dissolved
water through temperature reduction. Entrained water settles
out in time under static conditions and may or may not be
drained during normal servicing, depending on the rate at
which it is converted to free water. In general, it is not likely
that all entrained water can ever be separated from fuel under
field conditions. The settling rate depends on a series of
factors including temperature, quiescence, and droplet size.
The droplet size varies depending upon the mechanics
of formation. Usually, the particles are so small as to be
invisible to the naked eye, but in extreme cases, can cause
slight haziness in the fuel. Water in solution cannot be
removed except by dehydration or by converting it through
temperature reduction to entrained, then to free water.

7-28

Another condition of undissolved water is free water that
may be introduced as a result of refueling or the settling of
entrained water that collects at the bottom of a fuel tank. Free
water is usually present in easily detected quantities at the
bottom of the tank, separated by a continuous interface from
the fuel above. Free water can be drained from a fuel tank
through the sump drains, which are provided for that purpose.
Free water, frozen on the bottom of reservoirs, such as the
fuel tanks and fuel filter, may render water drains useless
and can later melt releasing the water into the system thereby
causing engine malfunction or stoppage. If such a condition
is detected, the aircraft may be placed in a warm hangar to
reestablish proper draining of these reservoirs, and all sumps
and drains should be activated and checked prior to flight.
Entrained water (i.e., water in solution with petroleum fuels)
constitutes a relatively small part of the total potential water
in a particular system, the quantity dissolved being dependent
on fuel temperature and the existing pressure and the water
volubility characteristics of the fuel. Entrained water freezes
in mid fuel and tends to stay in suspension longer since the
specific gravity of ice is approximately the same as that of
AVGAS.
Water in suspension may freeze and form ice crystals of
sufficient size such that fuel screens, strainers, and filters
may be blocked. Some of this water may be cooled further as
the fuel enters carburetor air passages and causes carburetor
metering component icing, when conditions are not otherwise
conducive to this form of icing.

Prevention Procedures
The use of anti-icing additives for some aircraft has been
approved as a means of preventing problems with water
and ice in AVGAS. Some laboratory and flight testing
indicates that the use of hexylene glycol, certain methanol
derivatives, and ethylene glycol mononethyl ether (EGME)
in small concentrations inhibit fuel system icing. These tests
indicate that the use of EGME at a maximum 0.15 percent
by volume concentration substantially inhibits fuel system
icing under most operating conditions. The concentration
of additives in the fuel is critical. Marked deterioration in
additive effectiveness may result from too little or too much
additive. Pilots should recognize that anti-icing additives are
in no way a substitute or replacement for carburetor heat.
Aircraft operating instructions involving the use of carburetor
heat should be adhered to at all times when operating under
atmospheric conditions conducive to icing.

Refueling Procedures
Static electricity is formed by the friction of air passing
over the surfaces of an aircraft in flight and by the flow of
fuel through the hose and nozzle during refueling. Nylon,
Dacron, or wool clothing is especially prone to accumulate
and discharge static electricity from the person to the funnel
or nozzle. To guard against the possibility of static electricity
igniting fuel fumes, a ground wire should be attached to the
aircraft before the fuel cap is removed from the tank. Because
both the aircraft and refueler have different static charges,
bonding both components to each other is critical. By bonding
both components to each other, the static differential charge is
equalized. The refueling nozzle should be bonded to the aircraft
before refueling begins and should remain bonded throughout
the refueling process. When a fuel truck is used, it should be
grounded prior to the fuel nozzle contacting the aircraft.
If fueling from drums or cans is necessary, proper bonding
and grounding connections are important. Drums should be
placed near grounding posts, and the following sequence of
connections observed:
1. 	 Drum to ground
2. 	 Ground to aircraft
3. 	 Drum to aircraft or nozzle to aircraft before removing
the fuel cap
When disconnecting, reverse the order.
The passage of fuel through a chamois increases the charge
of static electricity and the danger of sparks. The aircraft
must be properly grounded and the nozzle, chamois filter,
and funnel bonded to the aircraft. If a can is used, it should
be connected to either the grounding post or the funnel.
Under no circumstances should a plastic bucket or similar
nonconductive container be used in this operation.

Heating System
There are many different types of aircraft heating systems that
are available depending on the type of aircraft. Regardless of
which type or the safety features that accompany them, it is
always important to reference the specific aircraft operator's
manual and become knowledgeable about the heating system.
Each has different repair and inspection criteria that should
be precisely followed.
Fuel Fired Heaters
A fuel fired heater is a small mounted or portable spaceheating device. The fuel is brought to the heater by using
piping from a fuel tank, or taps into the aircraft's fuel system.
A fan blows air into a combustion chamber, and a spark plug
or ignition device lights the fuel-air mixture. A built-in safety

switch prevents fuel from flowing unless the fan is working.
Outside the combustion chamber, a second, larger diameter
tube conducts air around the combustion tube's outer surface,
and a second fan blows the warmed air into tubing to direct
it towards the interior of the aircraft. Most gasoline heaters
can produce between 5,000 and 50,000 British Thermal Units
(BTU) per hour.
Fuel fired heaters require electricity to operate and are
compatible with a 12-volt and 24-volt aircraft electrical
system. The heater requires routine maintenance, such as
regular inspection of the combustion tube and replacement of
the igniter at periodic intervals. Because gasoline heaters are
required to be vented, special care must be made to ensure the
vents do not leak into the interior of the aircraft. Combustion
byproducts include soot, sulfur dioxide, carbon dioxide, and
some carbon monoxide. An improperly adjusted, fueled, or
poorly maintained fuel heater can be dangerous.
Exhaust Heating Systems
Exhaust heating systems are the simplest type of aircraft
heating system and are used on most light aircraft. Exhaust
heating systems are used to route exhaust gases away from
the engine and fuselage while reducing engine noise. The
exhaust systems also serve as a heat source for the cabin
and carburetor.
The risks of operating an aircraft with a defective exhaust
heating system include carbon monoxide poisoning, a
decrease in engine performance, and an increased potential
for fire. Because of these risks, technicians should be aware
of the rate of exhaust heating system deterioration and should
thoroughly inspect all areas of the exhaust heating system to
look for deficiencies inside and out.
Combustion Heater Systems
Combustion heaters or surface combustion heaters are often
used to heat the cabin of larger, more expensive aircraft.
This type of heater burns the aircraft's fuel in a combustion
chamber or tube to develop required heat, and the air
flowing around the tube is heated and ducted to the cabin.
A combustion heater is an airtight burner chamber with a
stainless-steel jacket. Fuel from the aircraft fuel system is
ignited and burns to provide heat. Ventilation air is forced
over the airtight burn chamber picking up heat, which is then
dispersed into the cabin area.
When the heater control switch is turned on, airflow, ignition,
and fuel are supplied to the heater. Airflow and ignition are
constant within the burner chamber while the heater control
switch is on. When heat is required, the temperature control
is advanced, activating the thermostat. The thermostat (which

7-29

senses ventilation air temperature) turns on the fuel solenoid
allowing fuel to spray into the burner chamber. Fuel mixes
with air inside the chamber and is ignited by the spark plug,
producing heat.
The by-product, carbon monoxide, leaves the aircraft through
the heater exhaust pipe. Air flowing over the outside of the
burner chamber and inside the jacket of the heater absorbs
the heat and carries it through ducts into the cabin. As the
thermostat reaches its preset temperature, it turns off the fuel
solenoid and stops the flow of fuel into the burner chamber.
When ventilation air cools to the point that the thermostat
again turns the fuel solenoid on, the burner starts again.
This method of heat is very safe as an overheat switch is
provided on all combustion heaters, which is wired into
the heater's electrical system to shut off the fuel in the case
of malfunction. In the unlikely event that the heater fuel
solenoid, located at the heater, remains open or the control
switches fail, the remote fuel solenoid and/or fuel pump is
shut off by the mechanical overheat switch, stopping all fuel
flow to the system.
As opposed to the fuel fired cabin heaters that are used
on most single-engine aircraft, it is unlikely for carbon
monoxide poisoning to occur in combustion heaters.
Combustion heaters have low pressure in the combustion
tube that is vented through its exhaust into the air stream. The
ventilation air on the outside of the combustion chamber is
of higher pressure than on the inside, and ram air increases
the pressure on the outside of the combustion tube. In the
event a leak would develop in the combustion chamber, the
higher-pressure air outside the chamber would travel into the
chamber and out the exhaust.
Bleed Air Heating Systems
Bleed air heating systems are used on turbine-engine
aircraft. Extremely hot compressor bleed air is ducted into
a chamber where it is mixed with ambient or re-circulated
air to cool the air to a useable temperature. The air mixture
is then ducted into the cabin. This type of system contains
several safety features to include temperature sensors that
prevent excessive heat from entering the cabin, check
valves to prevent a loss of compressor bleed air when
starting the engine and when full power is required, and
engine sensors to eliminate the bleed system if the engine
becomes inoperative.

Electrical System
Most aircraft are equipped with either a 14- or a 28-volt direct
current (DC) electrical system. A basic aircraft electrical
system consists of the following components:

7-30

Alternator/generator



Battery



Master/battery switch



Alternator/generator switch



Bus bar, fuses, and circuit breakers



Voltage regulator



Ammeter/loadmeter



Associated electrical wiring

Engine-driven alternators or generators supply electric
current to the electrical system. They also maintain a
sufficient electrical charge in the battery. Electrical energy
stored in a battery provides a source of electrical power for
starting the engine and a limited supply of electrical power
for use in the event the alternator or generator fails.
Most DC generators do not produce a sufficient amount of
electrical current at low engine rpm to operate the entire
electrical system. During operations at low engine rpm, the
electrical needs must be drawn from the battery, which can
quickly be depleted.
Alternators have several advantages over generators.
Alternators produce sufficient current to operate the entire
electrical system, even at slower engine speeds, by producing
alternating current (AC), which is converted to DC. The
electrical output of an alternator is more constant throughout
a wide range of engine speeds.
Some aircraft have receptacles to which an external ground
power unit (GPU) may be connected to provide electrical
energy for starting. These are very useful, especially
during cold weather starting. Follow the manufacturer's
recommendations for engine starting using a GPU.
The electrical system is turned on or off with a master switch.
Turning the master switch to the ON position provides
electrical energy to all the electrical equipment circuits
except the ignition system. Equipment that commonly uses
the electrical system for its source of energy includes:


Position lights



Anticollision lights



Landing lights



Taxi lights



Interior cabin lights



Instrument lights



Radio equipment



Turn indicator



Fuel gauges



Electric fuel pump



Stall warning system



Pitot heat



Starting motor

Many aircraft are equipped with a battery switch that
controls the electrical power to the aircraft in a manner
similar to the master switch. In addition, an alternator switch
is installed that permits the pilot to exclude the alternator
from the electrical system in the event of alternator failure.
[Figure 7-33]
With the alternator half of the switch in the OFF position, the
entire electrical load is placed on the battery. All nonessential
electrical equipment should be turned off to conserve
battery power.
A bus bar is used as a terminal in the aircraft electrical system
to connect the main electrical system to the equipment using
electricity as a source of power. This simplifies the wiring
system and provides a common point from which voltage can
be distributed throughout the system. [Figure 7-34]
Fuses or circuit breakers are used in the electrical system to
protect the circuits and equipment from electrical overload.
Spare fuses of the proper amperage limit should be carried in
the aircraft to replace defective or blown fuses. Circuit breakers
have the same function as a fuse but can be manually reset,
rather than replaced, if an overload condition occurs in the
electrical system. Placards at the fuse or circuit breaker panel
identify the circuit by name and show the amperage limit.

An ammeter is used to monitor the performance of the aircraft
electrical system. The ammeter shows if the alternator/
generator is producing an adequate supply of electrical power.
It also indicates whether or not the battery is receiving an
electrical charge.
Ammeters are designed with the zero point in the center
of the face and a negative or positive indication on either
side. [Figure 7-35] When the pointer of the ammeter is
on the plus side, it shows the charging rate of the battery.
A minus indication means more current is being drawn
from the battery than is being replaced. A full-scale minus
deflection indicates a malfunction of the alternator/generator.
A full-scale positive deflection indicates a malfunction of
the regulator. In either case, consult the AFM/POH for
appropriate action to be taken.
Not all aircraft are equipped with an ammeter. Some have
a warning light that, when lighted, indicates a discharge in
the system as a generator/alternator malfunction. Refer to the
AFM/POH for appropriate action to be taken.
Another electrical monitoring indicator is a loadmeter.
This type of gauge has a scale beginning with zero and
shows the load being placed on the alternator/generator.
[Figure 7-35] The loadmeter reflects the total percentage of
the load placed on the generating capacity of the electrical
system by the electrical accessories and battery. When all
electrical components are turned off, it reflects only the
amount of charging current demanded by the battery.
A voltage regulator controls the rate of charge to the battery
by stabilizing the generator or alternator electrical output. The
generator/alternator voltage output should be higher than the
battery voltage. For example, a 12-volt battery would be fed
by a generator/alternator system of approximately 14 volts.
The difference in voltage keeps the battery charged.

Hydraulic Systems
There are multiple applications for hydraulic use in aircraft,
depending on the complexity of the aircraft. For example, a
hydraulic system is often used on small airplanes to operate
wheel brakes, retractable landing gear, and some constantspeed propellers. On large airplanes, a hydraulic system is
used for flight control surfaces, wing flaps, spoilers, and
other systems.

Figure 7-33. On this master switch, the left half is for the alternator

and the right half is for the battery.

A basic hydraulic system consists of a reservoir, pump
(either hand, electric, or engine-driven), a filter to keep the
fluid clean, a selector valve to control the direction of flow,
a relief valve to relieve excess pressure, and an actuator.
[Figure 7-36]

7-31

Low-voltage
warning light

Alternator
control unit

To inst
LTS
circuit
breaker

Low volt out
Power in
Sense (+)
Field
Sense (-)

A
L
T

B

G

Ground

Clock

Alternator

F

B
A
T

ALT
Alternator field
circuit breaker

Master
switch

To flashing beacon
BCN PITOT

STROBE
RADIO FAN

Ground service
plug receptacle

To strobe lights
To landing and taxi lights

LDG LTS

To ignition switch
FLAP

Ammeter

0

-30

+30

To wing flap system

To low-voltage warning light

+ 60

- 60

INST LTS

AMP

r

To instrument, radio, compass
and post lights
To oil temperature gauge
To turn coordinator

To wing
flap circuit
breaker

Starter
contactor

Battery
contactor

To pitot heat
To radio cooling fan

PULL
OFF

rte

Flight hour
recorder

tery

B
U
S

To fuel quantity indicators
FUEL IND.

To red doorpost maplight

Sta

Oil pressure
switch

Bat

Pull off

B

P
R
I
M
A
R
Y

B
U
S

Magnetos
L

A
V
I
O
N
I
C
S

To low-vacuum warning light
Switch/circuit breaker to
standby vacuum pump
STBY VAC

To white doorpost light
To audio muting relay
To control wheel maplight

NAV
DOME

To navigation lights
To dome light
To radio

RADIO 1

To radio
R

RADIO 2

To radio or transponder
and encoding altimeter

CODE

Circuit breaker (auto-reset)

Fuse

Diode
Resistor

Circuit breaker (push to reset)
Circuit breaker (pull—off,
push to reset)

RADIO 3

To radio
RADIO 4

Capacitor (Noise Filter)

Figure 7-34. Electrical system schematic.

The hydraulic fluid is pumped through the system to an
actuator or servo. A servo is a cylinder with a piston inside
that turns fluid power into work and creates the power needed
to move an aircraft system or flight control. Servos can be
either single-acting or double-acting, based on the needs of
the system. This means that the fluid can be applied to one
7-32

or both sides of the servo, depending on the servo type. A
single-acting servo provides power in one direction. The
selector valve allows the fluid direction to be controlled.
This is necessary for operations such as the extension and
retraction of landing gear during which the fluid must work
in two different directions. The relief valve provides an outlet

-30

0

+30

0
30
60
ALT AMPS

+ 60

- 60
AMP

Loadmeter

Ammeter
Figure 7-35. Ammeter and loadmeter.

for the system in the event of excessive fluid pressure in the
system. Each system incorporates different components to
meet the individual needs of different aircraft.
A mineral-based hydraulic fluid is the most widely used type
for small aircraft. This type of hydraulic fluid, a kerosene-like
petroleum product, has good lubricating properties, as well
as additives to inhibit foaming and prevent the formation
of corrosion. It is chemically stable, has very little viscosity
change with temperature, and is dyed for identification. Since
several types of hydraulic fluids are commonly used, an aircraft
must be serviced with the type specified by the manufacturer.
Refer to the AFM/POH or the Maintenance Manual.
Landing Gear
The landing gear forms the principal support of an aircraft on
the surface. The most common type of landing gear consists
of wheels, but aircraft can also be equipped with floats for
water operations or skis for landing on snow. [Figure 7-37]
The landing gear on small aircraft consists of three wheels:
two main wheels (one located on each side of the fuselage)
and a third wheel positioned either at the front or rear of the

Hydraulic fluid supply
Return fluid
Hydraulic pressure

LEFT

RIGHT

Pump

Motion

BOTH

OFF

takeoff run, landing, taxiing, and when parked.

airplane. Landing gear employing a rear-mounted wheel is
called conventional landing gear. Airplanes with conventional
landing gear are often referred to as tailwheel airplanes. When
the third wheel is located on the nose, it is called a nosewheel,
and the design is referred to as a tricycle gear. A steerable
nosewheel or tailwheel permits the airplane to be controlled
throughout all operations while on the ground.

Tricycle Landing Gear
There are three advantages to using tricycle landing gear:
1.	 It allows more forceful application of the brakes during
landings at high speeds without causing the aircraft to
nose over.
2. 	 It permits better forward visibility for the pilot during
takeoff, landing, and taxiing.
3.	 It tends to prevent ground looping (swerving) by
providing more directional stability during ground
operation since the aircraft's center of gravity (CG)
is forward of the main wheels. The forward CG keeps
the airplane moving forward in a straight line rather
than ground looping.
Nosewheels are either steerable or castering. Steerable
nosewheels are linked to the rudders by cables or rods, while
castering nosewheels are free to swivel. In both cases, the
aircraft is steered using the rudder pedals. Airplanes with a
castering nosewheel may require the pilot to combine the
use of the rudder pedals with independent use of the brakes.

Tailwheel Landing Gear

System relief valve
Selector valve

Figure 7-37. The landing gear supports the airplane during the

Double acting cylinder

Tailwheel landing gear airplanes have two main wheels
attached to the airframe ahead of its CG that support most of
the weight of the structure. A tailwheel at the very back of the
fuselage provides a third point of support. This arrangement

Figure 7-36. Basic hydraulic system.

7-33

allows adequate ground clearance for a larger propeller
and is more desirable for operations on unimproved fields.
[Figure 7-38]
With the CG located behind the main landing gear, directional
control using this type of landing gear is more difficult while
on the ground. This is the main disadvantage of the tailwheel
landing gear. For example, if the pilot allows the aircraft to
swerve while rolling on the ground at a low speed, he or
she may not have sufficient rudder control and the CG will
attempt to get ahead of the main gear, which may cause the
airplane to ground loop.
Diminished forward visibility when the tailwheel is on or near
the ground is a second disadvantage of tailwheel landing gear
airplanes. Because of these disadvantages, specific training
is required to operate tailwheel airplanes.

Fixed and Retractable Landing Gear
Landing gear can also be classified as either fixed or
retractable. Fixed landing gear always remains extended
and has the advantage of simplicity combined with low

maintenance. Retractable landing gear is designed to
streamline the airplane by allowing the landing gear to
be stowed inside the structure during cruising flight.
[Figure 7-39]
Brakes
Airplane brakes are located on the main wheels and are
applied by either a hand control or by foot pedals (toe or heel).
Foot pedals operate independently and allow for differential
braking. During ground operations, differential braking can
supplement nosewheel/tailwheel steering.

Pressurized Aircraft
Aircraft are flown at high altitudes for two reasons. First, an
aircraft flown at high altitude consumes less fuel for a given
airspeed than it does for the same speed at a lower altitude
because the aircraft is more efficient at a high altitude.
Second, bad weather and turbulence may be avoided by flying
in relatively smooth air above the storms. Many modern
aircraft are being designed to operate at high altitudes,
taking advantage of that environment. In order to fly at
higher altitudes, the aircraft must be pressurized or suitable
supplemental oxygen must be provided for each occupant.
It is important for pilots who fly these aircraft to be familiar
with the basic operating principles.
In a typical pressurization system, the cabin, flight
compartment, and baggage compartments are incorporated
into a sealed unit capable of containing air under a pressure
higher than outside atmospheric pressure. On aircraft powered
by turbine engines, bleed air from the engine compressor
section is used to pressurize the cabin. Superchargers may
be used on older model turbine-powered aircraft to pump
air into the sealed fuselage. Piston-powered aircraft may use
air supplied from each engine turbocharger through a sonic
venturi (flow limiter). Air is released from the fuselage by

Figure 7-38. Tailwheel landing gear.

Figure 7-39. Fixed (left) and retractable (right) gear airplanes.

7-34

a device called an outflow valve. By regulating the air exit,
the outflow valve allows for a constant inflow of air to the
pressurized area. [Figure 7-40]
A cabin pressurization system typically maintains a cabin
pressure altitude of approximately 8,000 feet at the maximum
designed cruising altitude of an aircraft. This prevents rapid
changes of cabin altitude that may be uncomfortable or cause
injury to passengers and crew. In addition, the pressurization
system permits a reasonably fast exchange of air from
the inside to the outside of the cabin. This is necessary to
eliminate odors and to remove stale air. [Figure 7-41]
Pressurization of the aircraft cabin is necessary in order to
protect occupants against hypoxia. Within a pressurized
cabin, occupants can be transported comfortably and safely
for long periods of time, particularly if the cabin altitude
is maintained at 8,000 feet or below, where the use of
oxygen equipment is not required. The flight crew in this
type of aircraft must be aware of the danger of accidental
loss of cabin pressure and be prepared to deal with such an
emergency whenever it occurs.

Cabin heat
valve
Heat shroud

Forward
air outlets

At an altitude of 28,000
feet, standard atmo­
spheric pressure is 4.8
psi. By adding this
pressure to the cabin
pressure differential of
6.1 psi difference (psid),
a total air pressure of
10.9 psi is obtained.

Sea level
2,000
4,000
6,000
8,000
10,000
12,000
14,000
16,000
18,000
20,000
22,000
24,000
26,000
28,000
30,000

14.7
13.7
12.7
11.8
10.9
10.1
9.4
8.6
8.0
7.3
6.8
6.2
5.7
5.2
4.8
4.4

Figure 7-41. Standard atmospheric pressure chart.

The following terms will aid in understanding the operating
principles of pressurization and air conditioning systems:
Aircraft altitude—the actual height above sea level at
which the aircraft is flying

Air scoops



Ambient temperature—the temperature in the area
immediately surrounding the aircraft

Heat exchanger



Ambient pressure—the pressure in the area
immediately surrounding the aircraft

Turbocharger
compressor
section



Cabin altitude—cabin pressure in terms of equivalent
altitude above sea level



Differential pressure—the difference in pressure
between the pressure acting on one side of a wall
and the pressure acting on the other side of the
wall. In aircraft air-conditioning and pressurizing
systems, it is the difference between cabin pressure
and atmospheric pressure.

Floor level outlets
To cabin altitude
controller

Outflow valve

Ambient air
Compressor discharge air

CODE

The altitude at which the
standard air pressure is
equal to 10.9 psi can be
found at 8,000 feet.

Pressure (psi)



Flow control
venturi

Safety/dump valve

Atmosphere pressure
Altitude (ft)

Pressurization air
Pre-heated ambient air
Conditioned pressurization air
Pressurized cabin

Figure 7-40. High performance airplane pressurization system.

The cabin pressure control system provides cabin pressure
regulation, pressure relief, vacuum relief, and the means
for selecting the desired cabin altitude in the isobaric and
differential range. In addition, dumping of the cabin pressure
is a function of the pressure control system. A cabin pressure
regulator, an outflow valve, and a safety valve are used to
accomplish these functions.
The cabin pressure regulator controls cabin pressure to a
selected value in the isobaric range and limits cabin pressure
to a preset differential value in the differential range. When an
aircraft reaches the altitude at which the difference between
the pressure inside and outside the cabin is equal to the
highest differential pressure for which the fuselage structure
is designed, a further increase in aircraft altitude will result
7-35

in a corresponding increase in cabin altitude. Differential
control is used to prevent the maximum differential pressure,
for which the fuselage was designed, from being exceeded.
This differential pressure is determined by the structural
strength of the cabin and often by the relationship of the
cabin size to the probable areas of rupture, such as window
areas and doors.
The cabin air pressure safety valve is a combination
pressure relief, vacuum relief, and dump valve. The pressure
relief valve prevents cabin pressure from exceeding a
predetermined differential pressure above ambient pressure.
The vacuum relief prevents ambient pressure from exceeding
cabin pressure by allowing external air to enter the cabin
when ambient pressure exceeds cabin pressure. The flight
deck control switch actuates the dump valve. When this
switch is positioned to ram, a solenoid valve opens, causing
the valve to dump cabin air into the atmosphere.
The degree of pressurization and the operating altitude of
the aircraft are limited by several critical design factors.
Primarily, the fuselage is designed to withstand a particular
maximum cabin differential pressure.
Several instruments are used in conjunction with the
pressurization controller. The cabin differential pressure gauge
indicates the difference between inside and outside pressure.
This gauge should be monitored to assure that the cabin does
not exceed the maximum allowable differential pressure. A
cabin altimeter is also provided as a check on the performance
of the system. In some cases, these two instruments are
combined into one. A third instrument indicates the cabin rate
of climb or descent. A cabin rate-of-climb instrument and a
cabin altimeter are illustrated in Figure 7-42.

.5

I
I

2



Explosive decompression—a change in cabin pressure
faster than the lungs can decompress, possibly
resulting in lung damage. Normally, the time required
to release air from the lungs without restrictions, such
as masks, is 0.2 seconds. Most authorities consider any
decompression that occurs in less than 0.5 seconds to
be explosive and potentially dangerous.



Rapid decompression—a change in cabin pressure in
which the lungs decompress faster than the cabin.

During an explosive decompression, there may be noise,
and one may feel dazed for a moment. The cabin air fills
with fog, dust, or flying debris. Fog occurs due to the rapid
drop in temperature and the change of relative humidity.
Normally, the ears clear automatically. Air rushes from the
mouth and nose due to the escape of air from the lungs and
may be noticed by some individuals.
Rapid decompression decreases the period of useful
consciousness because oxygen in the lungs is exhaled rapidly,
reducing pressure on the body. This decreases the partial
pressure of oxygen in the blood and reduces the pilot's
effective performance time by one-third to one-fourth its
normal time. For this reason, an oxygen mask should be
worn when flying at very high altitudes (35,000 feet or
higher). It is recommended that the crewmembers select the
100 percent oxygen setting on the oxygen regulator at high
altitude if the aircraft is equipped with a demand or pressure
demand oxygen system.

4

35

6

4

Cabin rate-of-climb indicator
Figure 7-42. Cabin pressurization instruments.

7-36

Physiologically, decompressions fall into the following two
categories:

ALT
BIN et
CA 00 Fe
10

CABIN CLIMB
THOUSAND FT PER MIN

0
.5

2

Decompression is defined as the inability of the aircraft's
pressurization system to maintain its designed pressure
differential. This can be caused by a malfunction in the
pressurization system or structural damage to the aircraft.

30

0
0
DIFF
PRESS

6

25

2
5

20

5

1

PSI

4

3

10

15

Cabin/differential pressure indicator

Cabin differential
pressure indicator
(pounds per square
inch differential)

Cabin pressure
altitude indicator
(thousands of feet)

Maximum cabin
differential pressure
limit

The primary danger of decompression is hypoxia. Quick,
proper utilization of oxygen equipment is necessary to avoid
unconsciousness. Another potential danger that pilots, crew,
and passengers face during high altitude decompressions is
evolved gas decompression sickness. This occurs when the
pressure on the body drops sufficiently, nitrogen comes out
of solution, and forms bubbles inside the person that can have
adverse effects on some body tissues.

being stored in an unheated area of the aircraft rather than
an actual depletion of the oxygen supply. High pressure
oxygen containers should be marked with the psi tolerance
(i.e., 1,800 psi) before filling the container to that pressure.
The containers should be supplied with oxygen that meets
or exceeds SAE AS8010 (as revised), Aviator's Breathing
Oxygen Purity Standard. To assure safety, periodic inspection
and servicing of the oxygen system should be performed.

Decompression caused by structural damage to the aircraft
presents another type of danger to pilots, crew, and
passengers––being tossed or blown out of the aircraft if
they are located near openings. Individuals near openings
should wear safety harnesses or seatbelts at all times when
the aircraft is pressurized and they are seated. Structural
damage also has the potential to expose them to wind blasts
and extremely cold temperatures.

An oxygen system consists of a mask or cannula and a
regulator that supplies a flow of oxygen dependent upon
cabin altitude. Most regulators approved for use up to 40,000
feet are designed to provide zero percent cylinder oxygen
and 100 percent cabin air at cabin altitudes of 8,000 feet or
less, with the ratio changing to 100 percent oxygen and zero
percent cabin air at approximately 34,000 feet cabin altitude.
[Figure 7-43] Most regulators approved up to 45,000 feet
are designed to provide 40 percent cylinder oxygen and 60
percent cabin air at lower altitudes, with the ratio changing
to 100 percent at the higher altitude.

Rapid descent from altitude is necessary in order to minimize
these problems. Automatic visual and aural warning systems
are included in the equipment of all pressurized aircraft.

Crew and passengers use oxygen systems, in conjunction
with pressurization systems, to prevent hypoxia. Regulations
require, at a minimum, flight crews have and use supplemental
oxygen after 30 minutes exposure to cabin pressure altitudes
between 12,500 and 14,000 feet. Use of supplemental
oxygen is required immediately upon exposure to cabin
pressure altitudes above 14,000 feet. Every aircraft occupant,
above 15,000 feet cabin pressure altitude, must have
supplemental oxygen. However, based on a person's physical
characteristics and condition, a person may feel the effects
of oxygen deprivation at much lower altitudes. Some people
flying above 10,000 feet during the day may experience
disorientation due to the lack of adequate oxygen. At night,
especially when fatigued, these effects may occur as low
as 5,000 feet. Therefore, for optimum protection, pilots are
encouraged to use supplemental oxygen above 10,000 feet
cabin altitude during the day and above 5,000 feet at night.

Pilots should be aware of the danger of fire when using
oxygen. Materials that are nearly fireproof in ordinary air may
be susceptible to combustion in oxygen. Oils and greases may
ignite if exposed to oxygen and cannot be used for sealing
the valves and fittings of oxygen equipment. Smoking during
any kind of oxygen equipment use is prohibited. Before
each flight, the pilot should thoroughly inspect and test all
oxygen equipment. The inspection should include a thorough
examination of the aircraft oxygen equipment, including
available supply, an operational check of the system, and
assurance that the supplemental oxygen is readily accessible.
The inspection should be accomplished with clean hands and
should include a visual inspection of the mask and tubing
for tears, cracks, or deterioration; the regulator for valve
and lever condition and positions; oxygen quantity; and the
location and functioning of oxygen pressure gauges, flow
indicators, and connections. The mask should be donned and
the system should be tested. After any oxygen use, verify that
all components and valves are shut off.

Most high altitude aircraft come equipped with some type
of fixed oxygen installation. If the aircraft does not have
a fixed installation, portable oxygen equipment must be
readily accessible during flight. The portable equipment
usually consists of a container, regulator, mask outlet,
and pressure gauge. Aircraft oxygen is usually stored in
high pressure system containers of 1,800–2,200 psi. When
the ambient temperature surrounding an oxygen cylinder
decreases, pressure within that cylinder decreases because
pressure varies directly with temperature if the volume of a
gas remains constant. A drop in the indicated pressure of a
supplemental oxygen cylinder may be due to the container

Figure 7-43. Oxygen system regulator.

Oxygen Systems

7-37

Oxygen Masks
There are numerous types and designs of oxygen masks in
use. The most important factor in oxygen mask use is to
ensure that the masks and oxygen system are compatible.
Crew masks are fitted to the user's face with a minimum of
leakage and usually contain a microphone. Most masks are
the oronasal type that covers only the mouth and nose.
A passenger mask may be a simple, cup-shaped rubber
molding sufficiently flexible to obviate individual fitting. It
may have a simple elastic head strap or the passenger may
hold it to his or her face.
All oxygen masks should be kept clean to reduce the danger
of infection and prolong the life of the mask. To clean the
mask, wash it with a mild soap and water solution and rinse
it with clear water. If a microphone is installed, use a clean
swab, instead of running water, to wipe off the soapy solution.
The mask should also be disinfected. A gauze pad that has
been soaked in a water solution of Merthiolate can be used
to swab out the mask. This solution used should contain
one-fifth teaspoon of Merthiolate per quart of water. Wipe
the mask with a clean cloth and air dry.
Cannula
A cannula is an ergonomic piece of plastic tubing that runs
under the nose to administer oxygen to the user. [Figure 7-44]
Cannulas are typically more comfortable than masks, but
may not provide an adequate flow of oxygen as reliably as
masks when operating at higher altitudes. Airplanes certified
to older regulations had cannulas installed with an on-board

oxygen system. However, current regulations require aircraft
with oxygen systems installed and certified for operations
above 18,000 feet to be equipped with oxygen masks instead
of cannulas. Many cannulas have a flow meter in the oxygen
supply line. If equipped, a periodic check of the green flow
detector should be a part of the pilot's regular scan.
Diluter-Demand Oxygen Systems
Diluter-demand oxygen systems supply oxygen only when
the user inhales through the mask. An automix lever allows
the regulators to automatically mix cabin air and oxygen or
supply 100 percent oxygen, depending on the altitude. The
demand mask provides a tight seal over the face to prevent
dilution with outside air and can be used safely up to 40,000
feet. A pilot who has a beard or mustache should be sure it is
trimmed in a manner that will not interfere with the sealing
of the oxygen mask. The fit of the mask around the beard or
mustache should be checked on the ground for proper sealing.
Pressure-Demand Oxygen Systems
Pressure-demand oxygen systems are similar to diluter
demand oxygen equipment, except that oxygen is supplied to
the mask under pressure at cabin altitudes above 34,000 feet.
Pressure-demand regulators create airtight and oxygen-tight
seals, but they also provide a positive pressure application of
oxygen to the mask face piece that allows the user's lungs
to be pressurized with oxygen. This feature makes pressure
demand regulators safe at altitudes above 40,000 feet. Some
systems may have a pressure demand mask with the regulator
attached directly to the mask, rather than mounted on the
instrument panel or other area within the flight deck. The
mask-mounted regulator eliminates the problem of a long
hose that must be purged of air before 100 percent oxygen
begins flowing into the mask.
Continuous-Flow Oxygen System
Continuous-flow oxygen systems are usually provided for
passengers. The passenger mask typically has a reservoir
bag that collects oxygen from the continuous-flow oxygen
system during the time when the mask user is exhaling.
The oxygen collected in the reservoir bag allows a higher
aspiratory flow rate during the inhalation cycle, which
reduces the amount of air dilution. Ambient air is added to
the supplied oxygen during inhalation after the reservoir bag
oxygen supply is depleted. The exhaled air is released to the
cabin. [Figure 7-45]

Figure 7-44. Cannula with green flow detector.

7-38

Electrical Pulse-Demand Oxygen System
Portable electrical pulse-demand oxygen systems deliver
oxygen by detecting an individual's inhalation effort and
provide oxygen flow during the initial portion of inhalation.
Pulse demand systems do not waste oxygen during the

Figure 7-45. Continuous flow mask and rebreather bag.

breathing cycle because oxygen is only delivered during
inhalation. Compared to continuous-flow systems, the pulsedemand method of oxygen delivery can reduce the amount
of oxygen needed by 50–85 percent. Most pulse-demand
oxygen systems also incorporate an internal barometer
that automatically compensates for changes in altitude by
increasing the amount of oxygen delivered for each pulse as
altitude is increased. [Figure 7-46]
Pulse Oximeters
A pulse oximeter is a device that measures the amount of
oxygen in an individual's blood, in addition to heart rate.
This non-invasive device measures the color changes that
red blood cells undergo when they become saturated with
oxygen. By transmitting a special light beam through a
fingertip to evaluate the color of the red cells, a pulse
oximeter can calculate the degree of oxygen saturation
within one percent of directly measured blood oxygen.
Because of their portability and speed, pulse oximeters are
very useful for pilots operating in nonpressurized aircraft
above 12,500 feet where supplemental oxygen is required.
A pulse oximeter permits crewmembers and passengers of
an aircraft to evaluate their actual need for supplemental
oxygen. [Figure 7-47]
Servicing of Oxygen Systems
Before servicing any aircraft with oxygen, consult the
specific aircraft service manual to determine the type of
equipment required and procedures to be used. Certain
precautions should be observed whenever aircraft oxygen

Figure 7-46. EDS-011 portable pulse-demand oxygen system.

systems are to be serviced. Oxygen system servicing should
be accomplished only when the aircraft is located outside
of the hangars. Personal cleanliness and good housekeeping
are imperative when working with oxygen. Oxygen under
pressure creates spontaneous results when brought in contact
with petroleum products. Service people should be certain to
wash dirt, oil, and grease (including lip salves and hair oil)
from their hands before working around oxygen equipment. It
is also essential that clothing and tools are free of oil, grease,

Figure 7-47. Onyx pulse oximeter.

7-39

and dirt. Aircraft with permanently installed oxygen tanks
usually require two persons to accomplish servicing of the
system. One should be stationed at the service equipment
control valves, and the other stationed where he or she
can observe the aircraft system pressure gauges. Oxygen
system servicing is not recommended during aircraft fueling
operations or while other work is performed that could
provide a source of ignition. Oxygen system servicing while
passengers are on board the aircraft is not recommended.

Anti-Ice and Deice Systems
Anti-icing equipment is designed to prevent the formation
of ice, while deicing equipment is designed to remove ice
once it has formed. These systems protect the leading edge
of wing and tail surfaces, pitot and static port openings, fuel
tank vents, stall warning devices, windshields, and propeller
blades. Ice detection lighting may also be installed on some
aircraft to determine the extent of structural icing during
night flights.
Most light aircraft have only a heated pitot tube and are not
certified for flight in icing. These light aircraft have limited
cross-country capability in the cooler climates during late
fall, winter, and early spring. Noncertificated aircraft must
exit icing conditions immediately. Refer to the AFM/POH
for details.
Airfoil Anti-Ice and Deice
Inflatable deicing boots consist of a rubber sheet bonded to
the leading edge of the airfoil. When ice builds up on the
leading edge, an engine-driven pneumatic pump inflates the
rubber boots. Many turboprop aircraft divert engine bleed
air to the wing to inflate the rubber boots. Upon inflation,
the ice is cracked and should fall off the leading edge of the
wing. Deicing boots are controlled from the flight deck by
a switch and can be operated in a single cycle or allowed to
cycle at automatic, timed intervals. [Figure 7-48]
In the past, it was believed that if the boots were cycled
too soon after encountering ice, the ice layer would expand
instead of breaking off, resulting in a condition referred to as
ice "bridging." Consequently, subsequent deice boot cycles
would be ineffective at removing the ice buildup. Although
some residual ice may remain after a boot cycle, "bridging"
does not occur with any modern boots. Pilots can cycle the
boots as soon as an ice accumulation is observed. Consult
the AFM/POH for information on the operation of deice
boots on an aircraft.
Many deicing boot systems use the instrument system suction
gauge and a pneumatic pressure gauge to indicate proper boot
operation. These gauges have range markings that indicate
the operating limits for boot operation. Some systems may
7-40

Tubes deflated

Tubes inflated
Figure 6-48. Deicing boots on the leading edge of the wing.

also incorporate an annunciator light to indicate proper boot
operation.
Proper maintenance and care of deicing boots are important
for continued operation of this system. They need to be
carefully inspected during preflight.
Another type of leading edge protection is the thermal anti-ice
system. Heat provides one of the most effective methods for
preventing ice accumulation on an airfoil. High performance
turbine aircraft often direct hot air from the compressor
section of the engine to the leading edge surfaces. The hot
air heats the leading edge surfaces sufficiently to prevent the
formation of ice. A newer type of thermal anti-ice system
referred to as ThermaWing uses electrically heated graphite
foil laminate applied to the leading edge of the wing and
horizontal stabilizer. ThermaWing systems typically have
two zones of heat application. One zone on the leading edge
receives continuous heat; the second zone further aft receives
heat in cycles to dislodge the ice allowing aerodynamic forces
to remove it. Thermal anti-ice systems should be activated
prior to entering icing conditions.
An alternate type of leading edge protection that is not as
common as thermal anti-ice and deicing boots is known

as a weeping wing. The weeping-wing design uses small
holes located in the leading edge of the wing to prevent
the formation and build-up of ice. An antifreeze solution
is pumped to the leading edge and weeps out through the
holes. Additionally, the weeping wing is capable of deicing
an aircraft. When ice has accumulated on the leading edges,
application of the antifreeze solution chemically breaks down
the bond between the ice and airframe, allowing aerodynamic
forces to remove the ice. [Figure 7-49]
Windscreen Anti-Ice
There are two main types of windscreen anti-ice systems.
The first system directs a flow of alcohol to the windscreen.
If used early enough, the alcohol prevents ice from building
up on the windscreen. The rate of alcohol flow can be
controlled by a dial in the flight deck according to procedures
recommended by the aircraft manufacturer.
Another effective method of anti-icing equipment is the
electric heating method. Small wires or other conductive
material is imbedded in the windscreen. The heater can be
turned on by a switch in the flight deck, causing an electrical
current to be passed across the shield through the wires to
provide sufficient heat to prevent the formation of ice on
the windscreen. The heated windscreen should only be used
during flight. Do not leave it on during ground operations, as
it can overheat and cause damage to the windscreen. Warning:
the electrical current can cause compass deviation errors by
as much as 40°.
Propeller Anti-Ice
Propellers are protected from icing by the use of alcohol or
electrically heated elements. Some propellers are equipped
with a discharge nozzle that is pointed toward the root of the
blade. Alcohol is discharged from the nozzles, and centrifugal
force drives the alcohol down the leading edge of the blade.
The boots are also grooved to help direct the flow of alcohol.
This prevents ice from forming on the leading edge of the
propeller. Propellers can also be fitted with propeller anti-ice
boots. The propeller boot is divided into two sections—the

inboard and the outboard sections. The boots are imbedded
with electrical wires that carry current for heating the
propeller. The prop anti-ice system can be monitored for
proper operation by monitoring the prop anti-ice ammeter.
During the preflight inspection, check the propeller boots for
proper operation. If a boot fails to heat one blade, an unequal
blade loading can result and may cause severe propeller
vibration. [Figure 7-50]
Other Anti-Ice and Deice Systems
Pitot and static ports, fuel vents, stall-warning sensors,
and other optional equipment may be heated by electrical
elements. Operational checks of the electrically heated
systems are to be checked in accordance with the AFM /POH.
Operation of aircraft anti-icing and deicing systems should be
checked prior to encountering icing conditions. Encounters
with structural ice require immediate action. Anti-icing and
deicing equipment are not intended to sustain long-term flight
in icing conditions.

Chapter Summary
All aircraft have a requirement for essential systems such
as the engine, propeller, induction, ignition systems as well
as the fuel, lubrication, cooling, electrical, landing gear, and

10

Prop anti-ice ammeter
When the system is operating,
the prop ammeter indicates
normal operating range. As each
boot section cycles, the ammeter
fluctuates.

0

20

PROP DEICER
AMPS

Outboard section

Inboard section

Prop anti-ice boot
The boot is divided into two sections: inboard and outboard.
When the anti-ice is operating, the inboard section heats on
each blade, and then cycles to the outboard section. If a boot
fails to heat properly on one blade, unequal ice loading may
result causing severe vibration.

Figure 7-49. TKS weeping wing anti-ice/deicing system.

Figure 7-50. Prop ammeter and anti-ice boots.

7-41

environmental control systems to support flight. Understanding
the aircraft systems of the aircraft being flown is critical to
its safe operation and proper maintenance. Consult the AFM/
POH for specific information pertaining to the aircraft being
flown. Various manufacturer and owners group websites can
also be a valuable source of additional information.

7-42


Chapter 8

Flight
Instruments
Introduction
In order to safely fly any aircraft, a pilot must understand
how to interpret and operate the flight instruments. The
pilot also needs to be able to recognize associated errors and
malfunctions of these instruments. This chapter addresses the
pitot-static system and associated instruments, the vacuum
system and related instruments, gyroscopic instruments, and
the magnetic compass. When a pilot understands how each
instrument works and recognizes when an instrument is
malfunctioning, he or she can safely utilize the instruments
to their fullest potential.

Pitot-Static Flight Instruments
The pitot-static system is a combined system that utilizes the
static air pressure and the dynamic pressure due to the motion
of the aircraft through the air. These combined pressures are
utilized for the operation of the airspeed indicator (ASI),
altimeter, and vertical speed indicator (VSI). [Figure 8-1]

8-1

Airspeed indicator (ASI)

Vertical speed indicator (VSI)

Altimeter

29.8
29.9
30.0

Pressure chamber
Static port

Static chamber
Baffle plate
Pitot tube
Drain hole

Ram air

Static hole

Heater (35 watts)
Heater (100 watts)

Pitot heater switch

Alternate static source

Figure 8-1. Pitot-static system and instruments.

Impact Pressure Chamber and Lines
The pitot tube is utilized to measure the total combined
pressures that are present when an aircraft moves through
the air. Static pressure, also known as ambient pressure, is
always present whether an aircraft is moving or at rest. It is
simply the barometric pressure in the local area. Dynamic
pressure is present only when an aircraft is in motion;
therefore, it can be thought of as a pressure due to motion.
Wind also generates dynamic pressure. It does not matter if
the aircraft is moving through still air at 70 knots or if the
aircraft is facing a wind with a speed of 70 knots, the same
dynamic pressure is generated.
When the wind blows from an angle less than 90° off the
nose of the aircraft, dynamic pressure can be depicted on the
ASI. The wind moving across the airfoil at 20 knots is the
same as the aircraft moving through calm air at 20 knots.
The pitot tube captures the dynamic pressure, as well as the
static pressure that is always present.
The pitot tube has a small opening at the front that allows
the total pressure to enter the pressure chamber. The total
pressure is made up of dynamic pressure plus static pressure.
In addition to the larger hole in the front of the pitot tube,
there is a small hole in the back of the chamber that allows
moisture to drain from the system should the aircraft enter

8-2

precipitation. Both openings in the pitot tube must be checked
prior to flight to ensure that neither is blocked. Many aircraft
have pitot tube covers installed when they sit for extended
periods of time. This helps to keep bugs and other objects
from becoming lodged in the opening of the pitot tube.
The one instrument that utilizes the pitot tube is the ASI. The
total pressure is transmitted to the ASI from the pitot tube's
pressure chamber via a small tube. The static pressure is
also delivered to the opposite side of the ASI, which serves
to cancel out the two static pressures, thereby leaving the
dynamic pressure to be indicated on the instrument. When
the dynamic pressure changes, the ASI shows either increase
or decrease, corresponding to the direction of change. The
two remaining instruments (altimeter and VSI) utilize only
the static pressure that is derived from the static port.
Static Pressure Chamber and Lines
The static chamber is vented through small holes to the
free undisturbed air on the side(s) of the aircraft. As the
atmospheric pressure changes, the pressure is able to move
freely in and out of the instruments through the small lines
that connect the instruments to the static system. An alternate
static source is provided in some aircraft to provide static
pressure should the primary static source become blocked.
The alternate static source is normally found inside the flight

deck. Due to the venturi effect of the air flowing around the
fuselage, the air pressure inside the flight deck is lower than
the exterior pressure.
When the alternate static source pressure is used, the
following instrument indications are observed:

100 ft. pointer
10,000 ft. pointer
1,000 ft. pointer

Aneroid wafers

1. 	 The altimeter indicates a slightly higher altitude than
actual.
2. 	 The ASI indicates an airspeed greater than the actual
airspeed.
3.	 The VSI shows a momentary climb and then stabilizes
if the altitude is held constant.
Each pilot is responsible for consulting the Aircraft Flight
Manual (AFM) or the Pilot's Operating Handbook (POH)
to determine the amount of error that is introduced into the
system when utilizing the alternate static source. In an aircraft
not equipped with an alternate static source, an alternate
method of introducing static pressure into the system should
a blockage occur is to break the glass face of the VSI. This
most likely renders the VSI inoperative. The reason for
choosing the VSI as the instrument to break is that it is the
least important static source instrument for flight.
Altimeter
The altimeter is an instrument that measures the height of
an aircraft above a given pressure level. Pressure levels
are discussed later in detail. Since the altimeter is the only
instrument that is capable of indicating altitude, this is one of
the most vital instruments installed in the aircraft. To use the
altimeter effectively, the pilot must understand the operation
of the instrument, as well as the errors associated with the
altimeter and how each affect the indication.
A stack of sealed aneroid wafers comprise the main
component of the altimeter. An aneroid wafer is a sealed
wafer that is evacuated to an internal pressure of 29.92
inches of mercury ("Hg). These wafers are free to expand
and contract with changes to the static pressure. A higher
static pressure presses down on the wafers and causes them
to collapse. A lower static pressure (less than 29.92 "Hg)
allows the wafers to expand. A mechanical linkage connects
the wafer movement to the needles on the indicator face,
which translates compression of the wafers into a decrease
in altitude and translates an expansion of the wafers into an
increase in altitude. [Figure 8-2]
Notice how the static pressure is introduced into the rear of the
sealed altimeter case. The altimeter's outer chamber is sealed,
which allows the static pressure to surround the aneroid
wafers. If the static pressure is higher than the pressure in the
aneroid wafers (29.92 "Hg), then the wafers are compressed

Crosshatch flag

Static port

A crosshatched area appears on
some altimeters when displaying
an altitude below 10,000 feet MSL.

Barometric scale adjustment knob
Altimeter setting window
Figure 8-2. Altimeter.

until the pressure inside the wafers is equal to the surrounding
static pressure. Conversely, if the static pressure is less than
the pressure inside of the wafers, the wafers are able to expand
which increases the volume. The expansion and contraction
of the wafers moves the mechanical linkage which drives the
needles on the face of the altimeter.

Principle of Operation
The pressure altimeter is an aneroid barometer that measures
the pressure of the atmosphere at the level where the altimeter is
located and presents an altitude indication in feet. The altimeter
uses static pressure as its source of operation. Air is denser
at sea level than aloft—as altitude increases, atmospheric
pressure decreases. This difference in pressure at various levels
causes the altimeter to indicate changes in altitude.
The presentation of altitude varies considerably between
different types of altimeters. Some have one pointer while
others have two or more. Only the multipointer type is
discussed in this handbook. The dial of a typical altimeter
is graduated with numerals arranged clockwise from zero
to nine. Movement of the aneroid element is transmitted
through gears to the three hands that indicate altitude. In
Figure 8-2, the long, thin needle with the inverted triangle
at the end indicates tens of thousands of feet; the short, wide
needle indicates thousands of feet; and the long needle on
top indicates hundreds of feet.

8-3

This indicated altitude is correct, however, only when the
sea level barometric pressure is standard (29.92 "Hg), the sea
level free air temperature is standard (+15 degrees Celsius
(°C) or 59 degrees Fahrenheit (°F)), and the pressure and
temperature decrease at a standard rate with an increase
in altitude. Adjustments for nonstandard pressures are
accomplished by setting the corrected pressure into a
barometric scale located on the face of the altimeter. The
barometric pressure window is sometimes referred to as
the Kollsman window; only after the altimeter is set does it
indicate the correct altitude. The word "correct" will need
to be better explained when referring to types of altitudes,
but is commonly used in this case to denote the approximate
altitude above sea level. In other words, the indicated
altitude refers to the altitude read off of the altitude which is
uncorrected, after the barometric pressure setting is dialed
into the Kollsman window. The additional types of altitudes
are further explained later.

Effect of Nonstandard Pressure and Temperature
It is easy to maintain a consistent height above ground if
the barometric pressure and temperature remain constant,
but this is rarely the case. The pressure and temperature can
change between takeoff and landing even on a local flight.
If these changes are not taken into consideration, flight
becomes dangerous.
If altimeters could not be adjusted for nonstandard pressure, a
hazardous situation could occur. For example, if an aircraft is
flown from a high pressure area to a low pressure area without
adjusting the altimeter, a constant altitude will be displayed,
but the actual height of the aircraft above the ground would

5,00

0 foo

4,000

t pre

ssur

re lev

ot pres

el

sure le

vel

pressure le

1,000 foot pressure

Sea level

vel

level

30°C

Figure 8-3. Effects of nonstandard temperature on an altimeter.

8-4

Adjustments to compensate for nonstandard pressure do
not compensate for nonstandard temperature. Since cold
air is denser than warm air, when operating in temperatures
that are colder than standard, the altitude is lower than the
altimeter indication. [Figure 8-3] It is the magnitude of this
"difference" that determines the magnitude of the error. It is
the difference due to colder temperatures that concerns the
pilot. When flying into a cooler air mass while maintaining a
constant indicated altitude, true altitude is lower. If terrain or
obstacle clearance is a factor in selecting a cruising altitude,
particularly in mountainous terrain, remember to anticipate
that a colder-than-standard temperature places the aircraft
lower than the altimeter indicates. Therefore, a higher
indicated altitude may be required to provide adequate terrain
clearance. A variation of the memory aid used for pressure

el

ressu

2,000 foot

Many altimeters do not have an accurate means of being
adjusted for barometric pressures in excess of 31.00
"Hg. When the altimeter cannot be set to the higher
pressure setting, the aircraft actual altitude is higher than
the altimeter indicates. When low barometric pressure
conditions occur (below 28.00), flight operations by
aircraft unable to set the actual altimeter setting are
not recommended.

e lev

foot p

3,000 fo

be lower then the indicated altitude. There is an old aviation
axiom: "GOING FROM A HIGH TO A LOW, LOOK OUT
BELOW." Conversely, if an aircraft is flown from a low
pressure area to a high pressure area without an adjustment
of the altimeter, the actual altitude of the aircraft is higher
than the indicated altitude. Once in flight, it is important to
frequently obtain current altimeter settings en route to ensure
terrain and obstruction clearance.

15°C

0°C

can be employed: "FROM HOT TO COLD, LOOK OUT
BELOW." When the air is warmer than standard, the aircraft
is higher than the altimeter indicates. Altitude corrections for
temperature can be computed on the navigation computer.
Extremely cold temperatures also affect altimeter indications.
Figure 8-4, which was derived from ICAO formulas,
indicates how much error can exist when the temperature is
extremely cold.

Setting the Altimeter
Most altimeters are equipped with a barometric pressure
setting window (or Kollsman window) providing a means
to adjust the altimeter. A knob is located at the bottom of the
instrument for this adjustment.
To adjust the altimeter for variation in atmospheric pressure,
the pressure scale in the altimeter setting window, calibrated
in inches of mercury ("Hg) and/or millibars (mb), is adjusted
to match the given altimeter setting. Altimeter setting is
defined as station pressure reduced to sea level, but an
altimeter setting is accurate only in the vicinity of the
reporting station. Therefore, the altimeter must be adjusted as
the flight progresses from one station to the next. Air traffic
control (ATC) will advise when updated altimeter settings
are available. If a pilot is not utilizing ATC assistance,
local altimeter settings can be obtained by monitoring local
automated weather observing system/automated surface
observation system (AWOS/ASOS) or automatic terminal
information service (ATIS) broadcasts.
Many pilots confidently expect the current altimeter setting
will compensate for irregularities in atmospheric pressure at
all altitudes, but this is not always true. The altimeter setting
broadcast by ground stations is the station pressure corrected
to mean sea level. It does not account for the irregularities

90

20

20

20

30

40

50

50

60

90 120 170 230 280

-10

20

30

40

50

60

70

80

90 100 150 200 290 390 490

-20

30

50

60

70

90 100 120 130 140 210 280 420 570 710

-30

40

60

80 100 120 140 150 170 190 280 380 570 760 950

-40

50

80 100 120 150 170 190 220 240 360 480 720 970 1,210

-50

60

90 120 150 180 210 240 270 300 450 590 890 1,190 1,500

0
1,
00
0
1,
50
0
2,
00
0
3,
00
0
4,
00
0
5,
00
0

80

20

40

0

70

20

30

0

60

10

30

0

50

10

20

0

40

10

20

0

30

10

0

0

20

+10

0

Reported
Temp 0 °C

Height Above Airport in Feet

40

60

80

90

Figure 8-4. Look at the chart using a temperature of –10 °C and

an aircraft altitude of 1,000 feet above the airport elevation. The
chart shows that the reported current altimeter setting may place
the aircraft as much as 100 feet below the altitude indicated by
the altimeter.

at higher levels, particularly the effect of nonstandard
temperature. If each pilot in a given area is using the same
altimeter setting, each altimeter should be equally affected
by temperature and pressure variation errors, making it
possible to maintain the desired vertical separation between
aircraft. This does not guarantee vertical separation though.
It is still imperative to maintain a regimented visual scan for
intruding air traffic.
When flying over high, mountainous terrain, certain
atmospheric conditions cause the altimeter to indicate an
altitude of 1,000 feet or more higher than the actual altitude.
For this reason, a generous margin of altitude should be
allowed—not only for possible altimeter error, but also for
possible downdrafts that might be associated with high winds.
To illustrate the use of the altimeter setting system, follow a
flight from Dallas Love Field, Texas, to Abilene Municipal
Airport, Texas, via Mineral Wells. Before taking off from
Love Field, the pilot receives a current altimeter setting of
29.85 "Hg from the control tower or ATIS and sets this value
in the altimeter setting window. The altimeter indication
should then be compared with the known airport elevation of
487 feet. Since most altimeters are not perfectly calibrated,
an error may exist.
When over Mineral Wells, assume the pilot receives a current
altimeter setting of 29.94 "Hg and sets this in the altimeter
window. Before entering the traffic pattern at Abilene
Municipal Airport, a new altimeter setting of 29.69 "Hg
is received from the Abilene Control Tower and set in
the altimeter setting window. If the pilot desires to fly the
traffic pattern at approximately 800 feet above the terrain,
and the field elevation of Abilene is 1,791 feet, an indicated
altitude of 2,600 feet should be maintained (1,791 feet +
800 feet = 2,591 feet, rounded to 2,600 feet).
The importance of properly setting the altimeter cannot
be overemphasized. Assume the pilot did not adjust the
altimeter at Abilene to the current setting and continued using
the Mineral Wells setting of 29.94 "Hg. When entering the
Abilene traffic pattern at an indicated altitude of 2,600 feet,
the aircraft would be approximately 250 feet below the proper
traffic pattern altitude. Upon landing, the altimeter would
indicate approximately 250 feet higher than the field elevation.
Mineral Wells altimeter setting

29.94

Abilene altimeter setting

29.69

Difference

0.25

(Since 1 inch of pressure is equal to approximately 1,000 feet
of altitude, 0.25 × 1,000 feet = 250 feet.)

8-5

When determining whether to add or subtract the amount of
altimeter error, remember that when the actual pressure is lower
than what is set in the altimeter window, the actual altitude
of the aircraft is lower than what is indicated on the altimeter.
The following is another method of computing the altitude
deviation. Start by subtracting the current altimeter setting
from 29.94 "Hg. Always remember to place the original setting
as the top number. Then subtract the current altimeter setting.
Mineral Wells altimeter setting

29.94

Abilene altimeter setting

29.69

29.94 – 29.69 = Difference

0.25

(Since 1 inch of pressure is equal to approximately 1,000 feet
of altitude, 0.25 × 1,000 feet = 250 feet.) Always subtract
the number from the indicated altitude.
2,600 – 250 = 2,350
Now, try a lower pressure setting. Adjust from altimeter
setting 29.94 to 30.56 "Hg.
Mineral Wells altimeter setting

29.94

Altimeter setting

30.56

29.94 – 30.56 = Difference

–0.62

(Since 1 inch of pressure is equal to approximately 1,000 feet
of altitude, 0.62 × 1,000 feet = 620 feet.) Always subtract
the number from the indicated altitude.
2,600 – (–620) = 3,220
The pilot will be 620 feet high.
Notice the difference is a negative number. Starting with the
current indicated altitude of 2,600 feet, subtracting a negative
number is the same as adding the two numbers. By utilizing
this method, a pilot will better understand the importance of
using the current altimeter setting (miscalculation of where
and in what direction an error lies can affect safety; if altitude
is lower than indicated altitude, an aircraft could be in danger
of colliding with an obstacle).

Altimeter Operation
There are two means by which the altimeter pointers can
be moved. The first is a change in air pressure, while the
other is an adjustment to the barometric scale. When the
aircraft climbs or descends, changing pressure within the
altimeter case expands or contracts the aneroid barometer.
This movement is transmitted through mechanical linkage
to rotate the pointers.

8-6

A decrease in pressure causes the altimeter to indicate an
increase in altitude, and an increase in pressure causes the
altimeter to indicate a decrease in altitude. Accordingly, if
the aircraft is sitting on the ground with a pressure level of
29.98 "Hg and the pressure level changes to 29.68 "Hg, the
altimeter would show an increase of approximately 300 feet
in altitude. This pressure change is most noticeable when the
aircraft is left parked over night. As the pressure falls, the
altimeter interprets this as a climb. The altimeter indicates
an altitude above the actual field elevation. If the barometric
pressure setting is reset to the current altimeter setting of 29.68
"Hg, then the field elevation is again indicated on the altimeter.
This pressure change is not as easily noticed in flight since
aircraft fly at specific altitudes. The aircraft steadily decreases
true altitude while the altimeter is held constant through pilot
action as discussed in the previous section.
Knowing the aircraft's altitude is vitally important to a
pilot. The pilot must be sure that the aircraft is flying high
enough to clear the highest terrain or obstruction along the
intended route. It is especially important to have accurate
altitude information when visibility is restricted. To clear
obstructions, the pilot must constantly be aware of the altitude
of the aircraft and the elevation of the surrounding terrain. To
reduce the possibility of a midair collision, it is essential to
maintain altitude in accordance with air traffic rules.

Types of Altitude
Altitude in itself is a relevant term only when it is specifically
stated to which type of altitude a pilot is referring. Normally
when the term "altitude" is used, it is referring to altitude
above sea level since this is the altitude which is used to
depict obstacles and airspace, as well as to separate air traffic.
Altitude is vertical distance above some point or level used as
a reference. There are as many kinds of altitude as there are
reference levels from which altitude is measured, and each
may be used for specific reasons. Pilots are mainly concerned
with five types of altitudes:
1. 	 Indicated altitude—read directly from the altimeter
(uncorrected) when it is set to the current altimeter
setting.
2. 	 True altitude—the vertical distance of the aircraft
above sea level—the actual altitude. It is often
expressed as feet above mean sea level (MSL). Airport,
terrain, and obstacle elevations on aeronautical charts
are true altitudes.

3. 	 Absolute altitude—the vertical distance of an aircraft
above the terrain, or above ground level (AGL).
4. 	 Pressure altitude—the altitude indicated when
the altimeter setting window (barometric scale) is
adjusted to 29.92 "Hg. This is the altitude above
the standard datum plane, which is a theoretical
plane where air pressure (corrected to 15 °C) equals
29.92 "Hg. Pressure altitude is used to compute density
altitude, true altitude, true airspeed (TAS), and other
performance data.
5. 	 Density altitude—pressure altitude corrected
for variations from standard temperature. When
conditions are standard, pressure altitude and density
altitude are the same. If the temperature is above
standard, the density altitude is higher than pressure
altitude. If the temperature is below standard, the
density altitude is lower than pressure altitude. This
is an important altitude because it is directly related
to the aircraft's performance.
A pilot must understand how the performance of the aircraft
is directly related to the density of the air. The density of
the air affects how much power a naturally aspirated engine
produces, as well as how efficient the airfoils are. If there are
fewer air molecules (lower pressure) to accelerate through
the propeller, the acceleration to rotation speed is longer
and thus produces a longer takeoff roll, which translates to
a decrease in performance.
As an example, consider an airport with a field elevation
of 5,048 feet MSL where the standard temperature is 5 °C.
Under these conditions, pressure altitude and density altitude
are the same—5,048 feet. If the temperature changes to
30 °C, the density altitude increases to 7,855 feet. This
means an aircraft would perform on takeoff as though the
field elevation were 7,855 feet at standard temperature.
Conversely, a temperature of –25 °C would result in a density
altitude of 1,232 feet. An aircraft would perform much better
under these conditions.

Vertical Speed Indicator (VSI)
The VSI, which is sometimes called a vertical velocity
indicator (VVI), indicates whether the aircraft is climbing,
descending, or in level flight. The rate of climb or descent is
indicated in feet per minute (fpm). If properly calibrated, the
VSI indicates zero in level flight. [Figure 8-5]

Principle of Operation
Although the VSI operates solely from static pressure, it is a
differential pressure instrument. It contains a diaphragm with
connecting linkage and gearing to the indicator pointer inside
an airtight case. The inside of the diaphragm is connected
directly to the static line of the pitot-static system. The area
outside the diaphragm, which is inside the instrument case,
is also connected to the static line but through a restricted
orifice (calibrated leak).
Both the diaphragm and the case receive air from the static
line at existing atmospheric pressure. The diaphragm receives
unrestricted air, while the case receives the static pressure via
the metered leak. When the aircraft is on the ground or in level
flight, the pressures inside the diaphragm and the instrument
case are equal, and the pointer is at the zero indication.
When the aircraft climbs or descends, the pressure inside
the diaphragm changes immediately, but due to the metering
action of the restricted passage, the case pressure remains
higher or lower for a short time, causing the diaphragm to
contract or expand. This causes a pressure differential that
is indicated on the instrument needle as a climb or descent.
When the pressure differential stabilizes at a definite ratio,
the needle indicates the rate of altitude change.

Diaphragm

2
I VERTICAL SPEED3
0

IDOWN

Instrument Check
Prior to each flight, a pilot should examine the altimeter for
proper indications in order to verify its validity. To determine
the condition of an altimeter, set the barometric scale to the
current reported altimeter setting transmitted by the local
airport traffic control tower, flight service station (FSS), or
any other reliable source, such as ATIS, AWOS, or ASOS.
The altimeter pointers should indicate the surveyed field
elevation of the airport. If the indication is off more than
75 feet from the surveyed field elevation, the instrument
should be referred to a certificated instrument repair station
for recalibration.

UPTHOUSAND FT PER MIN

2

4

3

Direct static pressure
Calibrated leak

Figure 8-5. Vertical speed indicator (VSI).

8-7

The VSI displays two different types of information:
 	 Trend information shows an immediate indication of
an increase or decrease in the aircraft's rate of climb
or descent.
 	 Rate information shows a stabilized rate of change in
altitude.
The trend information is the direction of movement of the
VSI needle. For example, if an aircraft is maintaining level
flight and the pilot pulls back on the control yoke causing the
nose of the aircraft to pitch up, the VSI needle moves upward
to indicate a climb. If the pitch attitude is held constant,
the needle stabilizes after a short period (6–9 seconds) and
indicates the rate of climb in hundreds of fpm. The time
period from the initial change in the rate of climb, until the
VSI displays an accurate indication of the new rate, is called
the lag. Rough control technique and turbulence can extend
the lag period and cause erratic and unstable rate indications.
Some aircraft are equipped with an instantaneous vertical
speed indicator (IVSI), which incorporates accelerometers
to compensate for the lag in the typical VSI. [Figure 8-6]

Instrument Check
As part of a preflight check, proper operation of the VSI
must be established. Make sure the VSI indicates a near zero
reading prior to leaving the ramp area and again just before
takeoff. If the VSI indicates anything other than zero, that
indication can be referenced as the zero mark. Normally, if the
needle is not exactly zero, it is only slightly above or below
the zero line. After takeoff, the VSI should trend upward to
indicate a positive rate of climb and then, once a stabilized
climb is established, a rate of climb can be referenced.

Airspeed Indicator (ASI)
The ASI is a sensitive, differential pressure gauge that
measures and promptly indicates the difference between pitot
(impact/dynamic pressure) and static pressure. These two
pressures are equal when the aircraft is parked on the ground
in calm air. When the aircraft moves through the air, the
pressure on the pitot line becomes greater than the pressure
in the static lines. This difference in pressure is registered by
the airspeed pointer on the face of the instrument, which is
calibrated in miles per hour, knots (nautical miles per hour),
or both. [Figure 8-7]
The ASI is the one instrument that utilizes both the pitot,
as well as the static system. The ASI introduces the static
pressure into the airspeed case while the pitot pressure
(dynamic) is introduced into the diaphragm. The dynamic
pressure expands or contracts one side of the diaphragm,
which is attached to an indicating system. The system drives
the mechanical linkage and the airspeed needle.
Just as in altitudes, there are multiple types of airspeeds.
Pilots need to be very familiar with each type.
 	 Indicated airspeed (IAS)—the direct instrument
reading obtained from the ASI, uncorrected for
variations in atmospheric density, installation error,
or instrument error. Manufacturers use this airspeed
as the basis for determining aircraft performance.
Takeoff, landing, and stall speeds listed in the AFM/
POH are IAS and do not normally vary with altitude
or temperature.

Diaphragm
Long lever

Sector

Accelerometer
Pitot connection

50

Pitot tube

100

150

I

2

.5 UP

0

.5 D

3

OW

I

200

4

N

2 3
Inlet from static port

Ram air

Static air line

Calibrated leak
Figure 8-7. Airspeed indicator (ASI).
Figure 8-6. An IVSI incorporates accelerometers to help the

instrument immediately indicate changes in vertical speed.

8-8

Handstaff pinion

 	 Calibrated airspeed (CAS)—IAS corrected for
installation error and instrument error. Although
manufacturers attempt to keep airspeed errors to a
minimum, it is not possible to eliminate all errors
throughout the airspeed operating range. At certain
airspeeds and with certain flap settings, the installation
and instrument errors may total several knots. This
error is generally greatest at low airspeeds. In the
cruising and higher airspeed ranges, IAS and CAS
are approximately the same. Refer to the airspeed
calibration chart to correct for possible airspeed errors.

F° 120 90 60 30
PRESS 0
ALT



White arc—commonly referred to as the flap operating
range since its lower limit represents the full flap stall
speed and its upper limit provides the maximum flap

10

60

140
MP

MP
H

H

White arc

120 100

120
100
100

20

As shown in Figure 8-8, ASIs on single-engine small aircraft
include the following standard color-coded markings:

40

KNOTS

15

VN0

80
80

.S.
T.A TS
K

Green arc

VFE

Figure 8-8. Single engine airspeed indicator (ASI).

speed. Approaches and landings are usually flown at
speeds within the white arc.


Lower limit of white arc (VS0)—the stalling speed
or the minimum steady flight speed in the landing
configuration. In small aircraft, this is the power-off
stall speed at the maximum landing weight in the
landing configuration (gear and flaps down).

 	 Upper limit of the white arc (VFE)—the maximum
speed with the flaps extended.


Green arc—the normal operating range of the aircraft.
Most flying occurs within this range.



Lower limit of green arc (VS1)—the stalling speed
or the minimum steady flight speed obtained in a
specified configuration. For most aircraft, this is the
power-off stall speed at the maximum takeoff weight
in the clean configuration (gear up, if retractable, and
flaps up).



Upper limit of green arc (V N0)—the maximum
structural cruising speed. Do not exceed this speed
except in smooth air.



Yellow arc—caution range. Fly within this range only
in smooth air and then only with caution.



Red line (VNE)—never exceed speed. Operating above
this speed is prohibited since it may result in damage
or structural failure.

Airspeed Indicator Markings
Aircraft weighing 12,500 pounds or less, manufactured after
1945, and certificated by the FAA are required to have ASIs
marked in accordance with a standard color-coded marking
system. This system of color-coded markings enables a pilot
to determine at a glance certain airspeed limitations that are
important to the safe operation of the aircraft. For example, if
during the execution of a maneuver, it is noted that the airspeed
needle is in the yellow arc and rapidly approaching the red
line, the immediate reaction should be to reduce airspeed.

VS1

0 -3
0

5

AIRSPEED

160

120

 	 Groundspeed (GS)—the actual speed of the airplane
over the ground. It is TAS adjusted for wind. GS
decreases with a headwind and increases with a
tailwind.

VS0

Yellow arc

140

 	 True airspeed (TAS)—CAS corrected for altitude
and nonstandard temperature. Because air density
decreases with an increase in altitude, an aircraft has
to be flown faster at higher altitudes to cause the same
pressure difference between pitot impact pressure
and static pressure. Therefore, for a given CAS, TAS
increases as altitude increases; or for a given TAS,
CAS decreases as altitude increases. A pilot can find
TAS by two methods. The most accurate method is
to use a flight computer. With this method, the CAS
is corrected for temperature and pressure variation by
using the airspeed correction scale on the computer.
Extremely accurate electronic flight computers are
also available. Just enter the CAS, pressure altitude,
and temperature, and the computer calculates the TAS.
A second method, which is a rule of thumb, provides
the approximate TAS. Simply add 2 percent to the
CAS for each 1,000 feet of altitude. The TAS is the
speed that is used for flight planning and is used when
filing a flight plan.

VNE (red line)

Other Airspeed Limitations
Some important airspeed limitations are not marked on the
face of the ASI, but are found on placards and in the AFM/
POH. These airspeeds include:

8-9

 	 Design maneuvering speed (V A)—the maximum
speed at which the structural design's limit load can
be imposed (either by gusts or full deflection of the
control surfaces) without causing structural damage.
It is important to consider weight when referencing
this speed. For example, VA may be 100 knots when
an airplane is heavily loaded, but only 90 knots when
the load is light.
 	 Landing gear operating speed (VLO)—the maximum
speed for extending or retracting the landing gear if
flying an aircraft with retractable landing gear.
 	 Landing gear extended speed (VLE)—the maximum
speed at which an aircraft can be safely flown with
the landing gear extended.
 	 Best angle-of-climb speed (VX)—the airspeed at
which an aircraft gains the greatest amount of altitude
in a given distance. It is used during a short-field
takeoff to clear an obstacle.
 	 Best rate-of-climb speed (VY)—the airspeed that
provides the most altitude gain in a given period of time.
 	 Single-engine best rate-of-climb (VYSE)—the best
rate-of-climb or minimum rate-of-sink in a light
twin-engine aircraft with one engine inoperative. It is
marked on the ASI with a blue line. VYSE is commonly
referred to as "Blue Line."
 	 Minimum control speed (VMC)—the minimum flight
speed at which a light, twin-engine aircraft can be
satisfactorily controlled when an engine suddenly
becomes inoperative and the remaining engine is at
takeoff power.

already in the system vents through the drain hole, and the
remaining pressure drops to ambient (outside) air pressure.
Under these circumstances, the ASI reading decreases to
zero because the ASI senses no difference between ram and
static air pressure. The ASI no longer operates since dynamic
pressure cannot enter the pitot tube opening. Static pressure
is able to equalize on both sides since the pitot drain hole
is still open. The apparent loss of airspeed is not usually
instantaneous but happens very quickly. [Figure 8-9]
If both the pitot tube opening and the drain hole should
become clogged simultaneously, then the pressure in the pitot
tube is trapped. No change is noted on the airspeed indication
should the airspeed increase or decrease. If the static port
is unblocked and the aircraft should change altitude, then a
change is noted on the ASI. The change is not related to a
change in airspeed but a change in static pressure. The total
pressure in the pitot tube does not change due to the blockage;
however, the static pressure will change.
Because airspeed indications rely upon both static and
dynamic pressure together, the blockage of either of these
systems affects the ASI reading. Remember that the ASI has
a diaphragm in which dynamic air pressure is entered. Behind
this diaphragm is a reference pressure called static pressure
that comes from the static ports. The diaphragm pressurizes
against this static pressure and as a result changes the airspeed
indication via levers and indicators. [Figure 8-10]
For example, take an aircraft and slow it down to zero knots
at a given altitude. If the static port (providing static pressure)
and the pitot tube (providing dynamic pressure) are both
unobstructed, the following claims can be made:

Instrument Check

1. 	 The ASI would be zero.

Prior to takeoff, the ASI should read zero. However, if there
is a strong wind blowing directly into the pitot tube, the ASI
may read higher than zero. When beginning the takeoff,
make sure the airspeed is increasing at an appropriate rate.

2. 	 Dynamic pressure and static pressure are equal.

Blockage of the Pitot-Static System
Errors almost always indicate blockage of the pitot tube, the
static port(s), or both. Blockage may be caused by moisture
(including ice), dirt, or even insects. During preflight, make
sure the pitot tube cover is removed. Then, check the pitot and
static port openings. A blocked pitot tube affects the accuracy
of the ASI, but a blockage of the static port not only affects
the ASI, but also causes errors in the altimeter and VSI.

3.	 Because both dynamic and static air pressure are equal
at zero speed with increased speed, dynamic pressure

Pitot tube

Static port

Blockage

Blocked Pitot System
The pitot system can become blocked completely or only
partially if the pitot tube drain hole remains open. If the pitot
tube becomes blocked and its associated drain hole remains
clear, ram air is no longer able to enter the pitot system. Air
8-10

Drain hole

Figure 8-9. A blocked pitot tube, but clear drain hole.

diaphragm causing it to compress, thereby resulting in an
indication of decreased airspeed. Conversely, if the aircraft
were to climb, the static pressure would decrease allowing
the diaphragm to expand, thereby showing an indication of
greater airspeed. [Figure 8-10]
Blockage

Static port

Pitot tube
Drain hole

Clim

b

nt
sce

De

Figure 8-10. Blocked pitot system with clear static system.

must include two components: static pressure and
dynamic pressure.
It can be inferred that airspeed indication must be based upon
a relationship between these two pressures, and indeed it is.
An ASI uses the static pressure as a reference pressure and
as a result, the ASI's case is kept at this pressure behind the
diaphragm. On the other hand, the dynamic pressure through
the pitot tube is connected to a highly sensitive diaphragm
within the ASI case. Because an aircraft in zero motion
(regardless of altitude) results in a zero airspeed, the pitot tube
always provides static pressure in addition to dynamic pressure.
Therefore, the airspeed indication is the result of two
pressures: the pitot tube static and dynamic pressure within
the diaphragm as measured against the static pressure in the
ASI's case.

The pitot tube may become blocked during flight due to
visible moisture. Some aircraft may be equipped with pitot
heat for flight in visible moisture. Consult the AFM/POH for
specific procedures regarding the use of pitot heat.

Blocked Static System
If the static system becomes blocked but the pitot tube
remains clear, the ASI continues to operate; however, it
is inaccurate. The airspeed indicates lower than the actual
airspeed when the aircraft is operated above the altitude
where the static ports became blocked because the trapped
static pressure is higher than normal for that altitude. When
operating at a lower altitude, a faster than actual airspeed is
displayed due to the relatively low static pressure trapped
in the system.
Revisiting the ratios that were used to explain a blocked pitot
tube, the same principle applies for a blocked static port. If
the aircraft descends, the static pressure increases on the pitot
side showing an increase on the ASI. This assumes that the
aircraft does not actually increase its speed. The increase in
static pressure on the pitot side is equivalent to an increase
in dynamic pressure since the pressure cannot change on
the static side.
If an aircraft begins to climb after a static port becomes
blocked, the airspeed begins to show a decrease as the
aircraft continues to climb. This is due to the decrease in
static pressure on the pitot side, while the pressure on the
static side is held constant.
A blockage of the static system also affects the altimeter and
VSI. Trapped static pressure causes the altimeter to freeze
at the altitude where the blockage occurred. In the case of
the VSI, a blocked static system produces a continuous zero
indication. [Figure 8-11]
Some aircraft are equipped with an alternate static source in
the flight deck. In the case of a blocked static source, opening
the alternate static source introduces static pressure from the
flight deck into the system. Flight deck static pressure is lower
than outside static pressure. Check the aircraft AOM/POH for
airspeed corrections when utilizing alternate static pressure.

If the aircraft were to descend while the pitot tube is
obstructed, the pressure in the pitot system, including the
diaphragm, would remain constant. But as the descent
is made, the static pressure would increase against the
8-11

available to a pilot, but also how the information is displayed.
In addition to the improvement in system reliability, which
increases overall safety, EFDs have decreased the overall cost
of equipping aircraft with state-of-the-art instrumentation.
Primary electronic instrumentation packages are less prone
to failure than their analogue counterparts. No longer is it
necessary for aircraft designers to create cluttered panel
layouts in order to accommodate all necessary flight
instruments. Instead, multi-panel digital flight displays
combine all flight instruments onto a single screen that is
called a primary flight display (PFD). The traditional "six
pack" of instruments is now displayed on one liquid crystal
display (LCD) screen.

Inaccurate airspeed indications
Constant zero indication on VSI
Frozen altimeter

29.8
29.9
30.0

Pitot tube
Blockage
Static port

Figure 8-11. Blocked static system.

Electronic Flight Display (EFD)
Advances in technology have brought about changes in the
instrumentation found in all types of aircraft; for example,
Electronic Flight Displays (EFDs) commonly referred to
as "glass cockpits." EFDs include flight displays such as
primary flight displays (PFD) and multi-function displays
(MFD). This has changed not only what information is

NAV1
NAV2

108.00
108.00

Airspeed indicator

Airspeed Tape
Configured similarly to traditional panel layouts, the ASI
is located on the left side of the screen and is displayed as
a vertical speed tape. As the aircraft increases in speed, the
larger numbers descend from the top of the tape. The TAS is
displayed at the bottom of the tape through the input to the air
data computer (ADC) from the outside air temperature probe.
Airspeed markings for VX, VY, and rotation speed (VR) are
displayed for pilot reference. An additional pilot-controlled
airspeed bug is available to set at any desired reference speed.
As on traditional analogue ASIs, the electronic airspeed tape
displays the color-coded ranges for the flap operating range,

WPT _ _ _ _ _ _ DIS _ _ ._ NM DTK _ _ _°T
113.00
Slip skid indicator
Attitude indicator
110.60

134.000 118.000
123.800 Altimeter
118.000

TRK 360°

130

4000
4300

120

4200

110

4100

3900

80

3800

270°

70

20

1

2

4300

TAS 106KT

Turn indicator

3600
VOR 1

3500

Horizontal situation indicator

3400

270°

3300
6°C

INSET

Vertical speed indicator (VSI)

4000
3900
80

90

OAT

2

1

1
100
9

COM1
COM2

PFD

OBS

Slip/skid indicator

270°

XPDR 5537 IDNT LCL10:12:34
Turn rate indicator tickVORmarks
1

CDI

DME

XPDR

3200 TMR/REF
IDENT

NRST

ALERTS

Turn3100
rate trend vector

Figure 8-12. Primary flight display (PFD). Note that the actual location of indications vary depending on manufacturers.

8-12

normal range, and caution range. [Figure 8-12] The number
value changes color to red when the airspeed exceeds VNE to
warn the pilot of exceeding the maximum speed limitation.
Attitude Indicator
One improvement over analogue instrumentation is the
larger attitude indicator on EFD. The artificial horizon spans
the entire width of the PFD. [Figure 8-12] This expanded
instrumentation offers better reference through all phases of
flight and all flight maneuvers. The attitude indicator receives
its information from the Attitude Heading and Reference
System (AHRS).
Altimeter
The altimeter is located on the right side of the PFD.
[Figure 8-12] As the altitude increases, the larger numbers
descend from the top of the display tape, with the current
altitude being displayed in the black box in the center of the
display tape. The altitude is displayed in increments of 20 feet.
Vertical Speed Indicator (VSI)
The VSI is displayed to the right of the altimeter tape and can
take the form of an arced indicator or a vertical speed tape.
[Figure 8-12] Both are equipped with a vertical speed bug.
Heading Indicator
The heading indicator is located below the artificial horizon
and is normally modeled after a Horizontal Situation
Indicator (HSI). [Figure 8-12] As in the case of the attitude
indicator, the heading indicator receives its information from
the magnetometer, which feeds information to the AHRS unit
and then out to the PFD.

Turn Indicator
The turn indicator takes a slightly different form than the
traditional instrumentation. A sliding bar moves left and right
below the triangle to indicate deflection from coordinated
flight. [Figure 8-12] Reference for coordinated flight comes
from accelerometers contained in the AHRS unit.
Tachometer
The sixth instrument normally associated with the "six pack"
package is the tachometer. This is the only instrument that is
not located on the PFD. The tachometer is normally located
on the multi-function display (MFD). In the event of a display
screen failure, it is displayed on the remaining screen with
the PFD flight instrumentation. [Figure 8-13]
Slip/Skid Indicator
The slip/skid indicator is the horizontal line below the roll
pointer. [Figure 8-12] Like a ball in a turn-and-slip indicator,
a bar width off center is equal to one ball width displacement.
Turn Rate Indicator
The turn rate indicator, illustrated in Figure 8-12, is typically
found directly above the rotating compass card. Tick marks
to the left and right of the lubber line denote the turn
(standard­rate versus half standard-rate). Typically denoted
by a trend line, if the trend vector is extended to the
second tick mark the aircraft is in a standard-rate turn.
Individual panel displays can be configured for a variety
of aircraft by installing different software packages.
[Figure 8-14] Manufacturers are also able to upgrade existing
instrument displays in a similar manner, eliminating the need
to replace individual gauges in order to upgrade.

Figure 8-13. Multi-function display (MFD).

8-13

200

210

220

230

240

250

260

270
58

140
70

120

UY

1100
02

10
00:03:29

80

UX
VS

60

10

10

2320B

10

40

6 710
30.30
65

S

W

70

N

E
IFR APPR

ANG

239 A

MA239 58 00'
239 2.3NM

Figure 8-15. Teledyne's 90004 TAS/Plus Air Data Computer (ADC)
computes air data information from the pitot-static pneumatic
system, aircraft temperature probe, and barometric correction
device to help create a clear picture of flight characteristics.

does not enter a diaphragm. The ADC computes the received
barometric pressure and sends a digital signal to the PFD to
display the proper altitude readout. EFDs also show trend
vectors, which show the pilot how the altitude and airspeed
are progressing.
Trend Vectors
Trend vectors are magenta lines that move up and down both
the ASI and the altimeter. [Figures 8-16 and 8-17] The ADC
computes the rate of change and displays the 6-second projection
of where the aircraft will be. Pilots can utilize the trend vectors
to better control the aircraft's attitude. By including the trend
vectors in the instrument scan, pilots are able to precisely control
airspeed and altitude. Additional information can be obtained
by referencing the Instrument Flying Handbook or specific
avionics manufacturer's training material.

150
Figure 8-14. Chelton's FlightLogic (top) and Avidyne's Entegra

140

(bottom) are examples of panel displays that are configurable.

Air Data Computer (ADC)
EFDs utilize the same type of instrument inputs as traditional
analogue gauges; however, the processing system is different.
The pitot static inputs are received by an ADC. The ADC
computes the difference between the total pressure and the
static pressure and generates the information necessary to
display the airspeed on the PFD. Outside air temperatures
are also monitored and introduced into various components
within the system, as well as being displayed on the PFD
screen. [Figure 8-15]

130

1
120
9

The ADC is a separate solid state device that, in addition to
providing data to the PFD, is capable of providing data to the
autopilot control system. In the event of system malfunction, the
ADC can quickly be removed and replaced in order to decrease
downtime and decrease maintenance turn-around times.

110

Altitude information is derived from the static pressure port
just as an analogue system does; however, the static pressure

150
140
130

1
120
9
110
100
90

100

8-14

Airspeed trend vector

TAS 120KT

Figure 8-16. Airspeed trend vector.

the bicycle wheels increase speed, they become more stable in
their plane of rotation. This is why a bicycle is unstable and
maneuverable at low speeds and stable and less maneuverable
at higher speeds.

Airspeed trend (increasing)
4000
4200

130

2

4100

120

Altitude trend vector
1

4000

110

1
100
9

4000 45

40

33900
900

90

3800

80

3700

270°

80

3925

-500
1

2

70

3700

TAS 100KT

270°
VOR 1

3600
3500
3400

Turn rate trend vector
3300
3200

3000

05
3100
3800

4000 70
3950
3900 30

-375

-250

3100

4000 80
3975
3900 70
4100

-125

20

4000

3000

3900

80

3100

Figure 4-27. Supporting Instruments
Figure 8-17. Altimeter trend vector.

Gyroscopic Flight Instruments
Several flight instruments utilize the properties of a gyroscope
for their operation. The most common instruments containing
gyroscopes are the turn coordinator, heading indicator, and
the attitude indicator. To understand how these instruments
operate requires knowledge of the instrument power systems,
gyroscopic principles, and the operating principles of each
instrument.
Gyroscopic Principles
Any spinning object exhibits gyroscopic properties. A wheel
or rotor designed and mounted to utilize these properties is
called a gyroscope. Two important design characteristics
of an instrument gyro are great weight for its size, or high
density, and rotation at high speed with low friction bearings.

By mounting this wheel, or gyroscope, on a set of gimbal
rings, the gyro is able to rotate freely in any direction. Thus,
if the gimbal rings are tilted, twisted, or otherwise moved,
the gyro remains in the plane in which it was originally
spinning. [Figure 8-18]

Precession
Precession is the tilting or turning of a gyro in response to a
deflective force. The reaction to this force does not occur at
the point at which it was applied; rather, it occurs at a point
that is 90° later in the direction of rotation. This principle
allows the gyro to determine a rate of turn by sensing the
amount of pressure created by a change in direction. The rate
at which the gyro precesses is inversely proportional to the
speed of the rotor and proportional to the deflective force.
Using the example of the bicycle, precession acts on the
wheels in order to allow the bicycle to turn. While riding
at normal speed, it is not necessary to turn the handle bars
in the direction of the desired turn. A rider simply leans in
the direction that he or she wishes to go. Since the wheels
are rotating in a clockwise direction when viewed from the
right side of the bicycle, if a rider leans to the left, a force is
applied to the top of the wheel to the left. The force actually
acts 90° in the direction of rotation, which has the effect of
applying a force to the front of the tire, causing the bicycle

There are two general types of mountings; the type used
depends upon which property of the gyro is utilized. A freely
or universally mounted gyroscope is free to rotate in any
direction about its center of gravity. Such a wheel is said to
have three planes of freedom. The wheel or rotor is free to
rotate in any plane in relation to the base and is balanced so
that, with the gyro wheel at rest, it remains in the position
in which it is placed. Restricted or semi-rigidly mounted
gyroscopes are those mounted so that one of the planes of
freedom is held fixed in relation to the base.
There are two fundamental properties of gyroscopic action:
rigidity in space and precession.

Rigidity in Space
Rigidity in space refers to the principle that a gyroscope
remains in a fixed position in the plane in which it is spinning.
An example of rigidity in space is that of a bicycle wheel. As

Figure 8-18. Regardless of the position of its base, a gyro tends to

remain rigid in space, with its axis of rotation pointed in a constant
direction.

8-15

to move to the left. There is a need to turn the handlebars at
low speeds because of the instability of the slowly turning
gyros and also to increase the rate of turn.
Precession can also create some minor errors in some
instruments. [Figure 8-19] Precession can cause a freely
spinning gyro to become displaced from its intended plane
of rotation through bearing friction, etc. Certain instruments
may require corrective realignment during flight, such as the
heading indicator.
Sources of Power
In some aircraft, all the gyros are vacuum, pressure, or
electrically operated. In other aircraft, vacuum or pressure
systems provide the power for the heading and attitude
indicators, while the electrical system provides the power for
the turn coordinator. Most aircraft have at least two sources
of power to ensure at least one source of bank information is
available if one power source fails. The vacuum or pressure
system spins the gyro by drawing a stream of air against the
rotor vanes to spin the rotor at high speed, much like the
operation of a waterwheel or turbine. The amount of vacuum
or pressure required for instrument operation varies, but is
usually between 4.5 "Hg and 5.5 "Hg.
One source of vacuum for the gyros is a vane-type enginedriven pump that is mounted on the accessory case of the
engine. Pump capacity varies in different aircraft, depending
on the number of gyros.
A typical vacuum system consists of an engine-driven
vacuum pump, relief valve, air filter, gauge, and tubing
necessary to complete the connections. The gauge is mounted
in the aircraft's instrument panel and indicates the amount
Pla

ne

of

Ro

tat

ion

Plane

ce

of For

CE

n

la

P
e
f

o
n

io

ss

ce

re

P
Figure 8-19. Precession of a gyroscope resulting from an applied

8-16

As shown in Figure 8-20, air is drawn into the vacuum
system by the engine-driven vacuum pump. It first goes
through a filter, which prevents foreign matter from entering
the vacuum or pressure system. The air then moves through
the attitude and heading indicators where it causes the gyros
to spin. A relief valve prevents the vacuum pressure, or
suction, from exceeding prescribed limits. After that, the air
is expelled overboard or used in other systems, such as for
inflating pneumatic deicing boots.
It is important to monitor vacuum pressure during flight,
because the attitude and heading indicators may not provide
reliable information when suction pressure is low. The
vacuum, or suction, gauge is generally marked to indicate
the normal range. Some aircraft are equipped with a warning
light that illuminates when the vacuum pressure drops below
the acceptable level.
When the vacuum pressure drops below the normal operating
range, the gyroscopic instruments may become unstable and
inaccurate. Cross-checking the instruments routinely is a
good habit to develop.
Turn Indicators
Aircraft use two types of turn indicators: turn-and-slip
indicators and turn coordinators. Because of the way the gyro
is mounted, the turn-and-slip indicator shows only the rate of
turn in degrees per second. The turn coordinator is mounted
at an angle, or canted, so it can initially show roll rate. When
the roll stabilizes, it indicates rate of turn. Both instruments
indicate turn direction and quality (coordination), and also
serve as a backup source of bank information in the event an
attitude indicator fails. Coordination is achieved by referring
to the inclinometer, which consists of a liquid-filled curved
tube with a ball inside. [Figure 8-21]

Turn-and-Slip Indicator

FOR

deflective force.

of pressure in the system (vacuum is measured in inches of
mercury less than ambient pressure).

The gyro in the turn-and-slip indicator rotates in the vertical
plane corresponding to the aircraft's longitudinal axis. A
single gimbal limits the planes in which the gyro can tilt,
and a spring works to maintain a center position. Because of
precession, a yawing force causes the gyro to tilt left or right,
as viewed from the pilot seat. The turn-and-slip indicator
uses a pointer, called the turn needle, to show the direction
and rate of turn. The turn-and-slip indicator is incapable of
"tumbling" off its rotational axis because of the restraining
springs. When extreme forces are applied to a gyro, the gyro
is displaced from its normal plane of rotation, rendering its
indications invalid. Certain instruments have specific pitch
and bank limits that induce a tumble of the gyro.

Vacuum relief valve

Heading indicator

3

6

30

33

I2

24

Overboard vent line
I5

Vacuum pump

2I

20

20
I0

I0

I0

I0

20

STBY PWR

SUCTION
INCHES

0

TEST

6

4
2

20

8

MERCURT

I0

Suction
gauge

Attitude indicator

Vacuum air filter

Figure 8-20. Typical vacuum system.

Turn Coordinator
The gimbal in the turn coordinator is canted; therefore, its
gyro can sense both rate of roll and rate of turn. Since turn
coordinators are more prevalent in training aircraft, this
discussion concentrates on that instrument. When rolling into
or out of a turn, the miniature aircraft banks in the direction
the aircraft is rolled. A rapid roll rate causes the miniature
aircraft to bank more steeply than a slow roll rate.

aircraft with the turn index. Figure 8-22 shows a picture of a
turn coordinator. There are two marks on each side (left and
right) of the face of the instrument. The first mark is used to
reference a wings level zero rate of turn. The second mark
on the left and right side of the instrument serve to indicate
a standard rate of turn. A standard-rate turn is defined as a
turn rate of 3° per second. The turn coordinator indicates only
the rate and direction of turn; it does not display a specific
angle of bank.

The turn coordinator can be used to establish and maintain
a standard-rate turn by aligning the wing of the miniature
Horizontal gyro

Gimbal rotation
Gimbal

Gyro rotation
Gimbal rotation

Gyro rotation

Canted gyro

Standard rate turn index

Standard rate turn index
Inclinometer

Inclinometer
Turn coordinator

Turn-and-slip indicator

Figure 8-21. Turn indicators rely on controlled precession for their operation.

8-17

L

D.C.
ELEC.

D.C.
ELEC.

TURN COORDINATOR

TURN COORDINATOR

2 MIN.

R

L

NO PITCH
INFORMATION

2 MIN.

to the center of the wind screen. When in coordinated flight,
the string trails straight back over the top of the wind screen.
When the aircraft is either slipping or skidding, the yaw
string moves to the right or left depending on the direction
of slip or skid.
R

NO PITCH
INFORMATION

Slipping turn

Skidding turn

D.C.
ELEC.

TURN COORDINATOR

L

2 MIN.

R

NO PITCH
INFORMATION

Coordinated turn

Figure 8-22. If inadequate right rudder is applied in a right turn,
a slip results. Too much right rudder causes the aircraft to skid
through the turn. Centering the ball results in a coordinated turn.

Inclinometer
The inclinometer is used to depict aircraft yaw, which is
the side-to-side movement of the aircraft's nose. During
coordinated, straight-and-level flight, the force of gravity
causes the ball to rest in the lowest part of the tube, centered
between the reference lines. Coordinated flight is maintained
by keeping the ball centered. If the ball is not centered, it can
be centered by using the rudder.
To center the ball, apply rudder pressure on the side to
which the ball is deflected. Use the simple rule, "step on the
ball," to remember which rudder pedal to press. If aileron
and rudder are coordinated during a turn, the ball remains
centered in the tube. If aerodynamic forces are unbalanced,
the ball moves away from the center of the tube. As shown
in Figure 8-22, in a slip, the rate of turn is too slow for the
angle of bank, and the ball moves to the inside of the turn. In
a skid, the rate of turn is too great for the angle of bank, and
the ball moves to the outside of the turn. To correct for these
conditions, and improve the quality of the turn, remember to
"step on the ball." Varying the angle of bank can also help
restore coordinated flight from a slip or skid. To correct for a
slip, decrease bank and/or increase the rate of turn. To correct
for a skid, increase the bank and/or decrease the rate of turn.

Yaw String
One additional tool that can be added to the aircraft is a yaw
string. A yaw string is simply a string or piece of yarn attached
8-18

Instrument Check
During preflight, ensure that the inclinometer is full of fluid
and has no air bubbles. The ball should also be resting at
its lowest point. When taxiing, the turn coordinator should
indicate a turn in the correct direction while the ball moves
opposite the direction of the turn.
Attitude Indicator
The attitude indicator, with its miniature aircraft and horizon
bar, displays a picture of the attitude of the aircraft. The
relationship of the miniature aircraft to the horizon bar is
the same as the relationship of the real aircraft to the actual
horizon. The instrument gives an instantaneous indication of
even the smallest changes in attitude.
The gyro in the attitude indicator is mounted in a horizontal
plane and depends upon rigidity in space for its operation.
The horizon bar represents the true horizon. This bar is
fixed to the gyro and remains in a horizontal plane as the
aircraft is pitched or banked about its lateral or longitudinal
axis, indicating the attitude of the aircraft relative to the true
horizon. [Figure 8-23]
The gyro spins in the horizontal plane and resists deflection
of the rotational path. Since the gyro relies on rigidity in
space, the aircraft actually rotates around the spinning gyro.
Bank index

Gimbal rotation

20
I0
I0
20

STBY

Roll gimbal

20

20
I020
I0

I0

I0

I0

20I0
20

PWR

20

TEST

Horizon reference arm
Gyro

Figure 8-23. Attitude indicator.

Pitch gimbal

An adjustment knob is provided with which the pilot may
move the miniature aircraft up or down to align the miniature
aircraft with the horizon bar to suit the pilot's line of vision.
Normally, the miniature aircraft is adjusted so that the wings
overlap the horizon bar when the aircraft is in straight-and­
level cruising flight.
The pitch and bank limits depend upon the make and model
of the instrument. Limits in the banking plane are usually
from 100° to 110°, and the pitch limits are usually from 60°
to 70°. If either limit is exceeded, the instrument will tumble
or spill and will give incorrect indications until realigned. A
number of modern attitude indicators do not tumble.
Every pilot should be able to interpret the banking scale
illustrated in Figure 8-24. Most banking scale indicators on

20

the top of the instrument move in the same direction from
that in which the aircraft is actually banked. Some other
models move in the opposite direction from that in which
the aircraft is actually banked. This may confuse the pilot if
the indicator is used to determine the direction of bank. This
scale should be used only to control the degree of desired
bank. The relationship of the miniature aircraft to the horizon
bar should be used for an indication of the direction of bank.
The attitude indicator is reliable and the most realistic flight
instrument on the instrument panel. Its indications are very
close approximations of the actual attitude of the aircraft.
Heading Indicator
The heading indicator is fundamentally a mechanical
instrument designed to facilitate the use of the magnetic
compass. Errors in the magnetic compass are numerous,

20

I0

20

I0

20

20
I0

I0

I0

20

I0

I0

20

I0

I0

20

20
STBY PWR

20
TEST

STBY PWR

Straight climb
Pointer

10°

20°

a
sc

20

le

45°

I0

20

I0

I0
I0

20

20

STBY PWR

I0

I0

I0

I0

90°

TEST

20
I0

Adjustment knob

Artificial horizon

I0

I0

I0

20

20

STBY PWR

TEST

Descending left bank

STBY PWR

TEST

Level right bank

20

20
I0

20

20

TEST

20

I0

I0
20

20

STBY PWR

Level left bank

I0

60°

20

20

20

k

20

Ba
n

30°
20

I0

TEST

Climbing right bank

20
I0

20

I0

20

STBY PWR

TEST

Climbing left bank

I0

I0

STBY PWR

I0

20

I0
I0
I0

I0

I0

20

20

20

TEST

Straight descent

STBY PWR

TEST

Descending right bank

Figure 8-24. Attitude representation by the attitude indicator corresponds to the relation of the aircraft to the real horizon.

8-19

making straight flight and precision turns to headings difficult
to accomplish, particularly in turbulent air. A heading
indicator, however, is not affected by the forces that make
the magnetic compass difficult to interpret. [Figure 8-25]
The operation of the heading indicator depends upon the
principle of rigidity in space. The rotor turns in a vertical
plane and fixed to the rotor is a compass card. Since the rotor
remains rigid in space, the points on the card hold the same
position in space relative to the vertical plane of the gyro. The
aircraft actually rotates around the rotating gyro, not the other
way around. As the instrument case and the aircraft revolve
around the vertical axis of the gyro, the card provides clear
and accurate heading information.
Because of precession caused by friction, the heading
indicator creeps or drifts from its set position. Among
other factors, the amount of drift depends largely upon the
condition of the instrument. If the bearings are worn, dirty,
or improperly lubricated, the drift may be excessive. Another
error in the heading indicator is caused by the fact that the
gyro is oriented in space, and the Earth rotates in space at a
rate of 15° in 1 hour. Thus, discounting precession caused
by friction, the heading indicator may indicate as much as
15° error per every hour of operation.
Some heading indicators referred to as horizontal situation
indicators (HSI) receive a magnetic north reference from
a magnetic slaving transmitter and generally need no
adjustment. The magnetic slaving transmitter is called
a magnetometer.
Main drive gear

Attitude and Heading Reference System (AHRS)
Electronic flight displays have replaced free-spinning gyros
with solid-state laser systems that are capable of flight at
any attitude without tumbling. This capability is the result
of the development of the Attitude and Heading Reference
System (AHRS).
The AHRS sends attitude information to the PFD in order
to generate the pitch and bank information of the attitude
indicator. The heading information is derived from a
magnetometer that senses the earth's lines of magnetic flux.
This information is then processed and sent out to the PFD
to generate the heading display. [Figure 8-26]
The Flux Gate Compass System
As mentioned earlier, the lines of flux in the Earth's magnetic
field have two basic characteristics: a magnet aligns with
them, and an electrical current is induced, or generated, in
any wire crossed by them.
The flux gate compass that drives slaved gyros uses the
characteristic of current induction. The flux valve is a small,
segmented ring, like the one in Figure 8-27, made of soft iron
that readily accepts lines of magnetic flux. An electrical coil
is wound around each of the three legs to accept the current
induced in this ring by the Earth's magnetic field. A coil
wound around the iron spacer in the center of the frame has
400 Hz alternating current (AC) flowing through it. During
the times when this current reaches its peak, twice during
each cycle, there is so much magnetism produced by this
coil that the frame cannot accept the lines of flux from the
Earth's field.

Compass card gear

Gimbal rotation

30

3
6

24

33

I2

I5

2I
Adjustment gears

Gimbal
Gyro

Adjustment knob

Figure 8-25. A heading indicator displays headings based on a 360°

azimuth, with the final zero omitted. For example, "6" represents
060°, while "21" indicates 210°. The adjustment knob is used to
align the heading indicator with the magnetic compass.

8-20

Figure 8-26. Attitude and heading reference system (AHRS).

Remote Indicating Compass
Remote indicating compasses were developed to compensate
for the errors and limitations of the older type of heading
indicators. The two panel-mounted components of a typical
system are the pictorial navigation indicator and the slaving
control and compensator unit. [Figure 8-29] The pictorial
navigation indicator is commonly referred to as an HSI.
The slaving control and compensator unit has a push button
that provides a means of selecting either the "slaved gyro"
or "free gyro" mode. This unit also has a slaving meter
and two manual heading-drive buttons. The slaving meter
indicates the difference between the displayed heading and
the magnetic heading. A right deflection indicates a clockwise
error of the compass card; a left deflection indicates a
counterclockwise error. Whenever the aircraft is in a turn
and the card rotates, the slaving meter shows a full deflection
to one side or the other. When the system is in "free gyro"
mode, the compass card may be adjusted by depressing the
appropriate heading-drive button.
Figure 8-27. The soft iron frame of the flux valve accepts the flux from

the Earth's magnetic field each time the current in the center coil
Figure 3-23. The soft iron frame of the flux valve accepts the
reverses.
This flux
causesmagnetic
current to
flow
in the
pickup in
coils.
flux from
the Earth's
field
each
timethree
the current
the
center coil reverse.This flux causes current to flow in the three
picked coils.
As the
current reverses between the peaks, it demagnetizes

the frame so it can accept the flux from the Earth's field. As
this flux cuts across the windings in the three coils, it causes
current to flow in them. These three coils are connected in
such a way that the current flowing in them changes as the
heading of the aircraft changes. [Figure 8-28]

A separate unit, the magnetic slaving transmitter, is mounted
remotely, usually in a wingtip to eliminate the possibility of
magnetic interference. It contains the flux valve, which is
the direction-sensing device of the system. A concentration
of lines of magnetic force, after being amplified, becomes

The three coils are connected to three similar but smaller
coils in a synchro inside the instrument case. The synchro
rotates the dial of a radio magnetic indicator (RMI) or a HSI.

Figure 8-29. Pictorial navigation indicator (HSI, top), slaving meter
Figure 8-28. The current in each of the three pickup coils changes

(lower right), and slaving control compensator unit (lower left).

with the heading of the aircraft.

Figure 3-24. The current in each of the three pickup coils
changes with the heading of the aircraft.

8-21

a signal relayed to the heading indicator unit, which is also
remotely mounted. This signal operates a torque motor in
the heading indicator unit that processes the gyro unit until
it is aligned with the transmitter signal. The magnetic slaving
transmitter is connected electrically to the HSI.
There are a number of designs of the remote indicating
compass; therefore, only the basic features of the system are
covered here. Instrument pilots must become familiar with
the characteristics of the equipment in their aircraft.
As instrument panels become more crowded and the pilot's
available scan time is reduced by a heavier flight deck
workload, instrument manufacturers have worked toward
combining instruments. One good example of this is the
RMI in Figure 8-30. The compass card is driven by signals
from the flux valve, and the two pointers are driven by an
automatic direction finder (ADF) and a very high frequency
(VHF) omni-directional radio range (VOR).
Heading indicators that do not have this automatic
northseeking capability are called "free" gyros and require
periodic adjustment. It is important to check the indications
frequently (approximately every 15 minutes) and reset the
heading indicator to align it with the magnetic compass
when required. Adjust the heading indicator to the magnetic
compass heading when the aircraft is straight and level at a
constant speed to avoid compass errors.
The bank and pitch limits of the heading indicator vary
with the particular design and make of instrument. On some
heading indicators found in light aircraft, the limits are
approximately 55° of pitch and 55° of bank. When either of
these attitude limits is exceeded, the instrument "tumbles"

Figure 8-30. Driven by signals from a flux valve, the compass card

in this RMI indicates the heading of the aircraft opposite the upper
center index mark. The green pointer is driven by the ADF.

8-22

or "spills" and no longer gives the correct indication until
reset. After spilling, it may be reset with the caging knob.
Many of the modern instruments used are designed in such
a manner so that they do not tumble.
An additional precession error may occur due to a gyro not
spinning fast enough to maintain its alignment. When the
vacuum system stops producing adequate suction to maintain
the gyro speed, the heading indicator and the attitude indicator
gyros begin to slow down. As they slow, they become more
susceptible to deflection from the plane of rotation. Some
aircraft have warning lights to indicate that a low vacuum
situation has occurred. Other aircraft may have only a vacuum
gauge that indicates the suction.

Instrument Check
As the gyro spools up, make sure there are no abnormal
sounds. While taxiing, the instrument should indicate turns in
the correct direction, and precession should be normal. At idle
power settings, the gyroscopic instruments using the vacuum
system might not be up to operating speeds and precession
might occur more rapidly than during flight.

Angle of Attack Indicators
The purpose of an AOA indicator is to give the pilot better
situational awareness pertaining to the aerodynamic health
of the airfoil. This can also be referred to as stall margin
awareness. More simply explained, it is the margin that exists
between the current AOA that the airfoil is operating at, and
the AOA at which the airfoil will stall (critical AOA).
Speed by itself is not a reliable parameter to avoid a stall.
An airplane can stall at any speed. Angle of attack is a better
parameter to use to avoid a stall. For a given configuration,
the airplane always stalls at the same AOA, referred to as
the critical AOA. This critical AOA does not change with:


Weight



Bank Angle



Temperature



Density Altitude



Center of Gravity

An AOA indicator can have several benefits when installed in
General Aviation aircraft, not the least of which is increased
situational awareness. Without an AOA indicator, the AOA
is "invisible" to pilots. These devices measure several
parameters simultaneously and determine the current AOA
providing a visual image to the pilot of the current AOA along
with representations of the proximity to the critical AOA.
[Figure 8-31] These devices can give a visual representation
of the energy management state of the airplane. The energy

Figure 8-31. Angle of attack indicators.

state of an airplane is the balance between airspeed, altitude,
drag, and thrust and represents how efficiently the airfoil is
operating.

There are long and short graduation marks between the letters
and numbers, each long mark representing 10° and each short
mark representing 5°.

Compass Systems

The float and card assembly has a hardened steel pivot in its
center that rides inside a special, spring-loaded, hard glass
jewel cup. The buoyancy of the float takes most of the weight
off of the pivot, and the fluid damps the oscillation of the
float and card. This jewel-and-pivot type mounting allows the
float freedom to rotate and tilt up to approximately 18° angle
of bank. At steeper bank angles, the compass indications are
erratic and unpredictable.

The Earth is a huge magnet, spinning in space, surrounded
by a magnetic field made up of invisible lines of flux. These
lines leave the surface at the magnetic North Pole and reenter
at the magnetic South Pole.
Lines of magnetic flux have two important characteristics:
any magnet that is free to rotate will align with them, and
an electrical current is induced into any conductor that cuts
across them. Most direction indicators installed in aircraft
make use of one of these two characteristics.
Magnetic Compass
One of the oldest and simplest instruments for indicating
direction is the magnetic compass. It is also one of the basic
instruments required by Title 14 of the Code of Federal
Regulations (14 CFR) part 91 for both VFR and IFR flight.

The compass housing is entirely full of compass fluid. To
prevent damage or leakage when the fluid expands and
contracts with temperature changes, the rear of the compass
case is sealed with a flexible diaphragm, or with a metal
bellows in some compasses.
The magnets align with the Earth's magnetic field and the
pilot reads the direction on the scale opposite the lubber
line. Note that in Figure 8-32, the pilot views the compass

A magnet is a piece of material, usually a metal containing
iron, that attracts and holds lines of magnetic flux. Regardless
of size, every magnet has two poles: north and south. When
one magnet is placed in the field of another, the unlike poles
attract each other, and like poles repel.
An aircraft magnetic compass, such as the one in Figure 8-32,
has two small magnets attached to a metal float sealed inside a
bowl of clear compass fluid similar to kerosene. A graduated
scale, called a card, is wrapped around the float and viewed
through a glass window with a lubber line across it. The card
is marked with letters representing the cardinal directions,
north, east, south, and west, and a number for each 30°
between these letters. The final "0" is omitted from these
directions. For example, 3 = 30°, 6 = 60°, and 33 = 330°.

N-S

E-W

Figure 8-32. A magnetic compass. The vertical line is called the

lubber line.

8-23

card from its backside. When the pilot is flying north, as the
compass indicates, east is to the pilot's right. On the card,
"33," which represents 330° (west of north), is to the right of
north. The reason for this apparent backward graduation is
that the card remains stationary, and the compass housing and
the pilot rotate around it. Because of this setup, the magnetic
compass can be confusing to read.

the two poles are aligned, and there is no variation. East
of this line, the magnetic North Pole is to the west of the
geographic North Pole and a correction must be applied to
a compass indication to get a true direction.
Flying in the Washington, D.C., area, for example, the
variation is 10° west. If a pilot wants to fly a true course of
south (180°), the variation must be added to this, resulting in
a magnetic course of 190° to fly. Flying in the Los Angeles,
California area, the variation is 14° east. To fly a true course
of 180° there, the pilot would have to subtract the variation
and fly a magnetic course of 166°. The variation error does
not change with the heading of the aircraft; it is the same
anywhere along the isogonic line.

Magnetic Compass Induced Errors
The magnetic compass is the simplest instrument in
the panel, but it is subject to a number of errors that must
be considered.
Variation
The Earth rotates about its geographic axis; maps and charts
are drawn using meridians of longitude that pass through the
geographic poles. Directions measured from the geographic
poles are called true directions. The magnetic North Pole to
which the magnetic compass points is not collocated with
the geographic North Pole, but is some 1,300 miles away;
directions measured from the magnetic poles are called
magnetic directions. In aerial navigation, the difference
between true and magnetic directions is called variation. This
same angular difference in surveying and land navigation is
called declination.

Deviation
The magnets in a compass align with any magnetic field.
Some causes for magnetic fields in aircraft include flowing
electrical current, magnetized parts, and conflict with the
Earth's magnetic field. These aircraft magnetic fields create
a compass error called deviation.
Deviation, unlike variation, depends on the aircraft heading.
Also unlike variation, the aircraft's geographic location
does not affect deviation. While no one can reduce or
change variation error, an aviation maintenance technician
(AMT) can provide the means to minimize deviation error
by performing the maintenance task known as "swinging
the compass."

Figure 8-33 shows the isogonic lines that identify the number
of degrees of variation in their area. The line that passes near
Chicago is called the agonic line. Anywhere along this line

150˚W

135˚W

120˚W

105˚W

90˚W

75˚W

60˚W

45˚W

30˚W

15˚W

0˚

15˚E

30˚E

45˚E

60˚E

75˚E

90˚E

105˚E

120˚E

135˚E

150˚E

165˚E

180˚W
70˚N

10

-40

-10

165˚W

-10

180˚W
70˚N

20

0

10

0
-3

60˚N

-2
0

-10

60˚N

20

45˚N

45˚N

0

30˚N

30˚N

0
10

15˚N

15˚N

0

-10

10

0˚

-20

-1
0
0˚

0

-10

15˚S

15˚S

10

-20
-30
20

165˚W

150˚W

135˚W

120˚W

105˚W

90˚W

0
-7

75˚W

60˚W

45˚W

Figure 8-33. Isogonic lines are lines of equal variation.

8-24

50

30˚W

15˚W

0˚

15˚E

30˚E

-100

10
-1

45˚E

60˚E

75˚E

90˚E

105˚E

-90

20
-1

120˚E

-1
30

0
-8

-50

80

180˚W

45˚S

40

-40

70˚N

70

60

10

50

60˚N

30

0
-6

20

40

30˚S

20

-30

30

45˚S

Main field declination (D)
Contour interval:
2 degrees
red contours positive (east)
blue negative (west)
pink (agonic) zero line.
Mercator Projection.
Position of dip poles

-20

30˚S

135˚E

70
100 9
0
13
11
0
0
150˚E

60

60˚N

80

165˚E

70˚N
180˚W

To swing the compass, an AMT positions the aircraft
on a series of known headings, usually at a compass
rose. [Figure 8-34] A compass rose consists of a series
of lines marked every 30° on an airport ramp, oriented to
magnetic north. There is minimal magnetic interference at
the compass rose. The pilot or the AMT, if authorized, can
taxi the aircraft to the compass rose and maneuver the aircraft
to the headings prescribed by the AMT.
Figure 8-35. A compass correction card shows the deviation

As the aircraft is "swung" or aligned to each compass rose
heading, the AMT adjusts the compensator assembly located
on the top or bottom of the compass. The compensator
assembly has two shafts whose ends have screwdriver slots
accessible from the front of the compass. Each shaft rotates
one or two small compensating magnets. The end of one shaft
is marked E-W, and its magnets affect the compass when the
aircraft is pointed east or west. The other shaft is marked
N-S and its magnets affect the compass when the aircraft is
pointed north or south.
The adjustments position the compensating magnets to
minimize the difference between the compass indication and
the actual aircraft magnetic heading. The AMT records any
remaining error on a compass correction card like the one
in Figure 8-35 and places it in a holder near the compass.
Only AMTs can adjust the compass or complete the compass
correction card. Pilots determine and fly compass headings
using the deviation errors noted on the card. Pilots must also
note the use of any equipment causing operational magnetic
interference such as radios, deicing equipment, pitot heat,
radar, or magnetic cargo.
The corrections for variation and deviation must be applied
in the correct sequence as shown below, starting from the
true course desired.
True north
330

N

030

300
W

060

240

E
120

210
S

150

Figure 8-34. Utilization of a compass rose aids compensation for

deviation errors.

correction for any heading.

Step 1: Determine the Magnetic Course

True Course (180°) ± Variation (+10°) = Magnetic Course

(190°)

The magnetic course (190°) is steered if there is no deviation

error to be applied. The compass card must now be considered

for the compass course of 190°.

Step 2: Determine the Compass Course

Magnetic Course (190°, from step 1) ± Deviation (–2°, from

correction card) = Compass Course (188°)

NOTE: Intermediate magnetic courses between those listed

on the compass card need to be interpreted. Therefore, to

steer a true course of 180°, the pilot would follow a compass

course of 188°.

To find the true course that is being flown when the compass

course is known:

Compass Course ± Deviation = Magnetic Course ±

Variation= True Course

Dip Errors
The Earth's magnetic field runs parallel to its surface only at
the Magnetic Equator, which is the point halfway between
the Magnetic North and South Poles. As you move away
from the Magnetic Equator towards the magnetic poles, the
angle created by the vertical pull of the Earth's magnetic field
in relation to the Earth's surface increases gradually. This
angle is known as the dip angle. The dip angle increases in
a downward direction as you move towards the Magnetic
North Pole and increases in an upward direction as you move
towards the Magnetic South Pole.
If the compass needle were mounted so that it could pivot
freely in three dimensions, it would align itself with the
magnetic field, pointing up or down at the dip angle in the
direction of local Magnetic North. Because the dip angle is
of no navigational interest, the compass is made so that it can

8-25

rotate only in the horizontal plane. This is done by lowering
the center of gravity below the pivot point and making the
assembly heavy enough that the vertical component of the
magnetic force is too weak to tilt it significantly out of the
horizontal plane. The compass can then work effectively at
all latitudes without specific compensation for dip. However,
close to the magnetic poles, the horizontal component of
the Earth's field is too small to align the compass which
makes the compass unuseable for navigation. Because of
this constraint, the compass only indicates correctly if the
card is horizontal. Once tilted out of the horizontal plane,
it will be affected by the vertical component of the Earth's
field which leads to the following discussions on northerly
and southerly turning errors.
Northerly Turning Errors
The center of gravity of the float assembly is located lower
than the pivotal point. As the aircraft turns, the force that
results from the magnetic dip causes the float assembly to
swing in the same direction that the float turns. The result is
a false northerly turn indication. Because of this lead of the
compass card, or float assembly, a northerly turn should be
stopped prior to arrival at the desired heading. This compass
error is amplified with the proximity to either magnetic pole.
One rule of thumb to correct for this leading error is to stop
the turn 15 degrees plus half of the latitude (i.e., if the aircraft
is being operated in a position near 40 degrees latitude, the
turn should be stopped 15+20=35 degrees prior to the desired
heading). [Figure 8-36A]
Left turn

A

Right turn

DIP

R
CA

D

ct

30
p
Di

e

ff e

3

N 33 30

DIP

3
p

Di

3

Acceleration Error
The magnetic dip and the forces of inertia cause magnetic
compass errors when accelerating and decelerating on
easterly and westerly headings. Because of the penduloustype mounting, the aft end of the compass card is tilted
upward when accelerating and downward when decelerating
during changes of airspeed. When accelerating on either
an easterly or westerly heading, the error appears as a
turn indication toward north. When decelerating on either
of these headings, the compass indicates a turn toward
south. A mnemonic, or memory jogger, for the effect of
acceleration error is the word "ANDS" (AccelerationNorth/Deceleration-South) may help you to remember the
acceleration error. [Figure 8-37] Acceleration causes an

No error

DIP

33
N

Southerly Turning Errors
When turning in a southerly direction, the forces are such that
the compass float assembly lags rather than leads. The result
is a false southerly turn indication. The compass card, or float
assembly, should be allowed to pass the desired heading prior
to stopping the turn. As with the northerly error, this error
is amplified with the proximity to either magnetic pole. To
correct this lagging error, the aircraft should be allowed to
pass the desired heading prior to stopping the turn. The same
rule of 15 degrees plus half of the latitude applies here (i.e.,
if the aircraft is being operated in a position near 30 degrees
latitude, the turn should be stopped 15+15+30 degrees after
passing the desired heading). [Figure 8-36B]

ef

fe

ct

CA

N

RD

33 30

B
Left turn
DIP

ct

12
RD e
p
Di

21 S 15 12

fe

ef

CA

ff e

p

15

Figure 8-36. Northerly and southerly turning errors.

8-26

DIP

DIP

Di

21 S

Right turn

No error

ct

C

21
AR

D

S

15

12

th
Sou
NOR

TH

N

33

3

30

6

12

NAV

15

S

21

24

W

GS

E

OBS

View is from the pilot's
perspective, and the
movable card is reset
after each turn.

Figure 8-37. The effects of acceleration error.
Figure 3-21. The effects of acceleration error.

indication toward north; deceleration causes an indication
toward south.

W 30

24
S 21

When starting a turn from a northerly heading, the compass
lags behind the turn. When starting a turn from a southerly
heading, the compass leads the turn.

15

Lags or Leads

3

E 12

The Vertical Card Magnetic Compass
The vertical card magnetic compass eliminates some of the
errors and confusion encountered with the magnetic compass.
The dial of this compass is graduated with letters representing
the cardinal directions, numbers every 30°, and tick marks
every 5°. The dial is rotated by a set of gears from the shaftmounted magnet, and the nose of the symbolic aircraft on
the instrument glass represents the lubber line for reading the
heading of the aircraft from the dial. [Figure 8-38]

N

6

Oscillation Error
Oscillation is a combination of all of the errors previously
mentioned and results in fluctuation of the compass card in
relation to the actual heading direction of the aircraft. When
setting the gyroscopic heading indicator to agree with the
magnetic compass, use the average indication between the
swings.

33

Figure 8-38. Vertical card magnetic compass.

Eddy Current Damping
In the case of a vertical card magnetic compass, flux from
the oscillating permanent magnet produces eddy currents in
a damping disk or cup. The magnetic flux produced by the
eddy currents opposes the flux from the permanent magnet
and decreases the oscillations.

8-27

Outside Air Temperature (OAT) Gauge
The outside air temperature (OAT) gauge is a simple and
effective device mounted so that the sensing element is
exposed to the outside air. The sensing element consists
of a bimetallic-type thermometer in which two dissimilar
materials are welded together in a single strip and twisted
into a helix. One end is anchored into protective tube and the
other end is affixed to the pointer, which reads against the
calibration on a circular face. OAT gauges are calibrated in
degrees °C, °F, or both. An accurate air temperature provides
the pilot with useful information about temperature lapse rate
with altitude change. [Figure 8-39]

Chapter Summary
Flight instruments enable an aircraft to be operated with
maximum performance and enhanced safety, especially when
flying long distances. Manufacturers provide the necessary
flight instruments, but to use them effectively, pilots need
to understand how they operate. As a pilot, it is important to
become very familiar with the operational aspects of the pitot­
static system and associated instruments, the vacuum system
and associated instruments, the gyroscopic instruments, and
the magnetic compass.

8-28

40

20

60

0
0

80

20
-20

-20

40

100

-40
-40
-60

C
F

120

60
140

Figure 8-39. Outside air temperature (OAT) gauge.

Chapter 9

Flight Manuals and
Other Documents
Introduction
Each aircraft comes with documentation and a set of manuals
with which a pilot must be familiar in order to fly that aircraft.
This chapter covers airplane flight manuals (AFM), the
pilot's operating handbook (POH), and aircraft documents
pertaining to ownership, airworthiness, maintenance, and
operations with inoperative equipment. Knowledge of these
required documents and manuals is essential for a pilot to
conduct a safe flight.

Airplane Flight Manuals (AFM)

Flight manuals and operating handbooks are concise reference
books that provide specific information about a particular
aircraft or subject. They contain basic facts, information,
and/or instructions for the pilot about the operation of an
aircraft, flying techniques, etc., and are intended to be kept
on hand for ready reference.

9-1

The aircraft owner/information manual is a document
developed by the aircraft manufacturer and contains general
information about the make and model of the aircraft.
The manual is not approved by the Federal Aviation
Administration (FAA) and is not specific to an individual
aircraft. The manual provides general information about the
operation of an aircraft, is not kept current, and cannot be
substituted for the AFM/POH.
An AFM is a document developed by the aircraft manufacturer
and approved by the FAA. This book contains the information
and instructions required to operate an aircraft safely. A
pilot must comply with this information which is specific
to a particular make and model of aircraft, usually by serial
number. An AFM contains the operating procedures and
limitations of that aircraft. Title 14 of the Code of Federal
Regulations (14 CFR) part 91 requires that pilots comply
with the operating limitations specified in the approved flight
manuals, markings, and placards.
Originally, flight manuals followed whatever format and
content the manufacturer felt was appropriate, but this
changed with the acceptance of Specification No. 1 prepared
by the General Aviation Manufacturers Association (GAMA).
Specification No. 1 established a standardized format for all
general aviation airplane and helicopter flight manuals.
The POH is a document developed by the aircraft
manufacturer and contains FAA-approved AFM information.
If "POH" is used in the main title, a statement must be
included on the title page indicating that sections of the
document are FAA approved as the AFM.
The POH for most light aircraft built after 1975 is also
designated as the FAA-approved flight manual. The typical
AFM/POH contains the following nine sections: General;
Limitations; Emergency Procedures; Normal Procedures;
Performance; Weight and Balance/Equipment List; Systems
Description; Handling, Service, and Maintenance; and
Supplements. Manufacturers also have the option of including
additional sections, such as one on Safety and Operational
Tips or an alphabetical index at the end of the POH.
Preliminary Pages
While the AFM/POH may appear similar for the same make
and model of aircraft, each manual is unique and contains
specific information about a particular aircraft, such as the
equipment installed and weight and balance information.
Manufacturers are required to include the serial number and
registration on the title page to identify the aircraft to which
the manual belongs. If a manual does not indicate a specific
aircraft registration and serial number, it is limited to general
study purposes only.
9-2

Most manufacturers include a table of contents that identifies
the order of the entire manual by section number and title.
Usually, each section also contains a table of contents for that
section. Page numbers reflect the section and page within that
section (1-1, 1-2, 2-1, 3-1, etc.). If the manual is published
in loose-leaf form, each section is usually marked with a
divider tab indicating the section number, title, or both. The
Emergency Procedures section may have a red tab for quick
identification and reference.
General (Section 1)
The General section provides the basic descriptive
information on the airframe and powerplant(s). Some
manuals include a three-dimensional drawing of the aircraft
that provides dimensions of various components. Included
are such items as wingspan, maximum height, overall length,
wheelbase length, main landing gear track width, diameter
of the rotor system, maximum propeller diameter, propeller
ground clearance, minimum turning radius, and wing area.
This section serves as a quick reference and helps a pilot
become familiar with the aircraft.
The last segment of the General section contains definitions,
abbreviations, explanations of symbology, and some of
the terminology used in the POH. At the discretion of the
manufacturer, metric and other conversion tables may also
be included.
Limitations (Section 2)
The Limitations section contains only those limitations required
by regulation or that are necessary for the safe operation of
the aircraft, powerplant, systems, and equipment. It includes
operating limitations, instrument markings, color-coding,
and basic placards. Some of the limitation areas are airspeed,
powerplant, weight and loading distribution, and flight.

Airspeed
Airspeed limitations are shown on the airspeed indicator
(ASI) by color coding and on placards or graphs in the
aircraft. [Figure 9-1] A red line on the ASI shows the airspeed
limit beyond which structural damage could occur. This is
called the never-exceed speed (VNE). A yellow arc indicates
the speed range between maximum structural cruising speed
(VN0) and VNE. Operation of an aircraft in the yellow airspeed
arc is for smooth air only and then only with caution. A green
arc depicts the normal operating speed range, with the upper
end at VN0 and the lower end at stalling speed at maximum
weight with the landing gear and flaps retracted (VS1). For
airplanes, the flap operating range is depicted by the white
arc, with the upper end at the maximum flap extended speed
(VFE), and the lower end at the stalling speed with the landing
gear and flaps in the landing configuration (VS0).

Maximum

245

II5
I00 P
P R
60 S E

T 200
E °F
M I50
P I00

75

I
20 S

Normal operating range

S

OIL

0
Minimum

Figure 9-3. Minimum, maximum, and normal operating range

markings on oil gauge.
Figure 9-1. Single-engine airspeed indicator.

In addition to the markings listed above, small multi-engine
airplanes have a red radial line to indicate single-engine
minimum controllable airspeed (VMC). A blue radial line
is used to indicate single-engine best rate of climb speed at
maximum weight at sea level (VYSE). [Figure 9-2]

Powerplant
The Powerplant Limitations portion describes operating
limitations on an aircraft's reciprocating or turbine engine(s).
These include limitations for takeoff power, maximum
continuous power, and maximum normal operating power,
which is the maximum power the engine can produce without
any restrictions and is depicted by a green arc. Other items
that can be included in this area are the minimum and
maximum oil and fuel pressures, oil and fuel grades, and
propeller operating limits. [Figure 9-3]
All reciprocating-engine powered aircraft must have a
revolutions per minute (rpm) indicator for each engine.
Aircraft equipped with a constant-speed propeller or rotor
system use a manifold pressure gauge to monitor power
output and a tachometer to monitor propeller or rotor speed.
Both instruments depict the maximum operating limit with

a red radial line and the normal operating range with a green
arc. [Figure 9-4] Some instruments may have a yellow arc
to indicate a caution area.

Weight and Loading Distribution
Weight and Loading Distribution contains the maximum
certificated weights, as well as the center of gravity (CG)
range. The location of the reference datum used in balance
computations is included in this section. Weight and balance
computations are not provided in this area, but rather in the
weight and balance section of the AFM/POH.

25
20

180
160

25

0

60
80

200

160
140

200

AIRSPEED
KNOTS

10

140

TAS

100

120

Figure 9-2. Multi-engine airspeed indicator.

I0
5
3

40

IN Hg
AL g .

I5

40

35

MANIFOLD
PRESSURE

15

IAS

260

30

45

50

20
RPM

HUNDREDS

25
I
I0

HOURS
AVOID
CONTINUOUS
OPERATION
BETWEEN 2250
AND 2350 RPM

30

35

Figure 9-4. Manifold pressure gauge (top) and tachometer (bottom).

9-3

Flight Limits
Flight Limits list authorized maneuvers with appropriate
entry speeds, flight load factor limits, and types of operation
limits. It also indicates those maneuvers that are prohibited,
such as spins or acrobatic flight, as well as operational
limitations such as flight into known icing conditions.

Placards
Most aircraft display one or more placards that contain
information having a direct bearing on the safe operation of
the aircraft. These placards are located in conspicuous places
and are reproduced in the Limitations section or as directed by
an Airworthiness Directive (AD). [Figure 9-5] Airworthiness
Directives are explained in detail later in this chapter.
Emergency Procedures (Section 3)
Checklists describing the recommended procedures and
airspeeds for coping with various types of emergencies or
critical situations are located in the Emergency Procedures
section. Some of the emergencies covered include:
engine failure, fire, and system failure. The procedures
for inflight engine restarting and ditching may also be
included. Manufacturers may first show an emergency
checklist in an abbreviated form with the order of items
reflecting the sequence of action. Amplified checklists that
provide additional information on the procedures follow
the abbreviated checklist. To be prepared for emergency
situations, memorize the immediate action items and, after
completion, refer to the appropriate checklist.

Manufacturers may include an optional subsection
entitled Abnormal Procedures. This subsection describes
recommended procedures for handling malfunctions that are
not considered emergencies.
Normal Procedures (Section 4)
This section begins with a list of the airspeeds for normal
operations. The next area consists of several checklists that
may include preflight inspection, before starting procedures,
starting engine, before taxiing, taxiing, before takeoff, climb,
cruise, descent, before landing, balked landing, after landing,
and post flight procedures. An Amplified Procedures area
follows the checklists to provide more detailed information
about the various previously mentioned procedures.
To avoid missing important steps, always use the appropriate
checklists when available. Consistent adherence to approved
checklists is a sign of a disciplined and competent pilot.
Performance (Section 5)
The Performance section contains all the information required
by the aircraft certification regulations and any additional
performance information the manufacturer deems important
to pilot ability to safely operate the aircraft. Performance
charts, tables, and graphs vary in style, but all contain the
same basic information. Examples of the performance
information found in most flight manuals include a graph or
table for converting calibrated airspeed to true airspeed; stall
speeds in various configurations; and data for determining
takeoff and climb performance, cruise performance, and
landing performance. Figure 9-6 is an example of a typical
performance graph. For more information on use of the
charts, graphs, and tables, refer to Chapter 10, Aircraft
Performance.
Weight and Balance/Equipment List (Section 6)
The Weight and Balance/Equipment List section contains
all the information required by the FAA to calculate the
weight and balance of an aircraft. Manufacturers include
sample weight and balance problems. Weight and balance is
discussed in greater detail in Chapter 10, Weight and Balance.
Systems Description (Section 7)
This section describes the aircraft systems in a manner
appropriate to the pilot most likely to operate the aircraft.
For example, a manufacturer might assume an experienced
pilot will be reading the information for an advanced aircraft.
For more information on aircraft systems, refer to Chapter
7, Aircraft Systems.

Figure 9-5. Placards are used to depict aircraft limitations.

9-4

EXAMPLE

Calibrated stall speed

Gross weight: 2,170 pounds
Angle of bank: 20°
Flap position: 40°
Stall speed, indicated: 44 knots

70

Indicated stall speed

0° Flaps

0° Flaps

50

40

40° Flaps

40° Flaps

2,400

2,200

2,000

1,800

1,600

Gross weight (pounds)

Stall speed (knots)

Maximum weight - 2,440 (pounds)

60

30

0°

20°

40°

60°

Angle of bank (degrees)

Figure 9-6. Stall speed chart.

Handling, Service, and Maintenance (Section 8)
The Handling, Service, and Maintenance section describes
the maintenance and inspections recommended by the
manufacturer (and the regulations). Additional maintenance
or inspections may be required by the issuance of ADs
applicable to the airframe, engine, propeller, or components.
This section also describes preventive maintenance that
may be accomplished by certificated pilots, as well as the
manufacturer's recommended ground handling procedures. It
includes considerations for hangaring, tie-down, and general
storage procedures for the aircraft.
Supplements (Section 9)
The Supplements section contains information necessary
to safely and efficiently operate the aircraft when equipped
with optional systems and equipment (not provided with the
standard aircraft). Some of this information may be supplied
by the aircraft manufacturer or by the manufacturer of the
optional equipment. The appropriate information is inserted
into the flight manual at the time the equipment is installed.
Autopilots, navigation systems, and air-conditioning
systems are examples of equipment described in this section.
[Figure 9-7]

Figure 9-7. Supplements provide information on optional equipment.

9-5

Safety Tips (Section 10)
The Safety Tips section is an optional section containing a
review of information that enhances the safe operation of the
aircraft. For example, physiological factors, general weather
information, fuel conservation procedures, high altitude
operations, or cold weather operations might be discussed.

Aircraft Documents
Certificate of Aircraft Registration
Before an aircraft can be flown legally, it must be registered
with the FAA Aircraft Registry. The Certificate of Aircraft
Registration, which is issued to the owner as evidence of
the registration, must be carried in the aircraft at all times.
[Figure 9-8]
The Certificate of Aircraft Registration cannot be used for
operations when:


The aircraft is registered under the laws of a foreign
country

Figure 9-8. AC Form 8050-3, Certificate of Aircraft Registration.

9-6



The aircraft's registration is canceled upon written
request of the certificate holder



The aircraft is totally destroyed or scrapped



The ownership of the aircraft is transferred



The certificate holder loses United States citizenship

For additional information, see 14 CFR part 47, section 47.41.
When one of the events listed in 14 CFR part 47, section
47.41 occurs, the previous owner must notify the FAA by
filling in the back of the Certificate of Aircraft Registration,
and mailing it to:
FAA Aircraft Registration Branch, AFS-750
P.O. Box 25504
Oklahoma City, OK 73125-0504
A dealer's aircraft registration certificate is another form of
registration certificate, but is valid only for required flight
tests by the manufacturer or in flights that are necessary for

the sale of the aircraft by the manufacturer or a dealer. The
dealer must remove the certificate when the aircraft is sold.
Upon complying with 14 CFR part 47, section 47.31, the
pink copy of the application for an Aircraft Registration
Application, Aeronautical Center (AC) Form 8050-1,
provides authorization to operate an unregistered aircraft
for a period not to exceed 90 days. Since the aircraft is
unregistered, it cannot be operated outside of the United
States until a permanent Certificate of Aircraft Registration
is received and placed in the aircraft.
The FAA does not issue any certificate of ownership or
endorse any information with respect to ownership on a
Certificate of Aircraft Registration.
NOTE: For additional information concerning the Aircraft
Registration Application or the Aircraft Bill of Sale, contact
the nearest FAA Flight Standards District Office (FSDO).
Airworthiness Certificate
An Airworthiness Certificate is issued by a representative of
the FAA after the aircraft has been inspected, is found to meet
the requirements of 14 CFR part 21, and is in condition for safe
operation. The Airworthiness Certificate must be displayed in
the aircraft so it is legible to the passengers and crew whenever
it is operated. The Airworthiness Certificate must remain with
the aircraft unless it is sold to a foreign purchaser.

A Standard Airworthiness Certificate is issued for aircraft
type certificated in the normal, utility, acrobatic, commuter,
transport categories, and manned free balloons. Figure 9-9
illustrates a Standard Airworthiness Certificate, and an
explanation of each item in the certificate follows.
1. 	 Nationality and Registration Marks. The "N"
indicates the aircraft is registered in the United States.
Registration marks consist of a series of up to five
numbers or numbers and letters. In this case, N2631A
is the registration number assigned to this aircraft.
2.	 Manufacturer and Model. Indicates the manufacturer,
make, and model of the aircraft.
3.	 Aircraft Serial Number. Indicates the manufacturer's
serial number assigned to the aircraft, as noted on the
aircraft data plate.
4.	 Category. Indicates the category in which the aircraft
must be operated. In this case, it must be operated
in accordance with the limitations specified for the
"NORMAL" category.
5.	 Authority and Basis for Issuance. Indicates the aircraft
conforms to its type certificate and is considered in
condition for safe operation at the time of inspection
and issuance of the certificate. Any exemptions from
the applicable airworthiness standards are briefly
noted here and the exemption number given. The word
"NONE" is entered if no exemption exists.

UNITED STATES OF AMERICA

DEPARTMENT OF TRANSPORTATION-FEDERAL AVIATION ADMINISTRATION
1

STANDARD AIRWORTHINESS CERTIFICATE

NATIONALITY AND
REGISTRATION MARKS

N12345
5

2

MANUFACTURER AND MODEL

Douglas DC-6A

3

AIRCRAFT SERIAL
NUMBER

43219

4

CATEGORY

Transport

AUTHORITY AND BASIS FOR ISSUANCE
This airworthiness certificate is issued pursuant to the Federal Aviation Act of 1958 and certifies that, as of the date of issuance, the

aircraft to which issued has been inspected and found to conform to the type certificate therefor, to be in condition for safe operation,

and has been shown to meet the requirements of the applicable comprehensive and detailed airworthiness code as provided by Annex 8

to the Convention on International Civil Aviation, except as noted herein.

Exceptions:


None
6

TERMS AND CONDITIONS
Unless sooner surrendered, suspended, revoked, or a termination date is otherwise established by the Administrator, this airworthiness
certificate is effective as long as the maintenance, preventative maintenance, and alterations are performed in accordance with
Parts 21, 43, and 91 of the Federal Aviation Regulations, as appropriate, and the aircraft is registered in the United States.
FAA REPRESENTATIVE
DESIGNATION NUMBER
DATE OF ISSUANCE

01/20/00

E.R. White

E.R. White

NE-XX

Any iteration, reproduction, or misuse of this certificate may be punishable by a fine not exceeding \$1,000 or imprisonment not exceeding 3 years or both.
THIS CERTIFICATE MUST BE DISPLAYED IN THE AIRCRAFT IN ACCORDANCE WITH APPLICABLE FEDERAL AVIATION REGULATIONS.
FAA Form 8100-2 (04-11) Supersedes Previous Edition

Figure 9-9. FAA Form 8100-2, Standard Airworthiness Certificate.

9-7

6.	 Terms and Conditions. Indicates the Airworthiness
Certificate is in effect indefinitely if the aircraft is
maintained in accordance with 14 CFR parts 21, 43, and
91, and the aircraft is registered in the United States.
Also included are the date the certificate was issued and the
signature and office identification of the FAA representative.
A Standard Airworthiness Certificate remains in effect
if the aircraft receives the required maintenance and is
properly registered in the United States. Flight safety relies
in part on the condition of the aircraft, which is determined
by inspections performed by mechanics, approved repair
stations, or manufacturers that meet specific requirements
of 14 CFR part 43.
A Special Airworthiness Certificate is issued for all aircraft
certificated in other than the Standard classifications, such
as Experimental, Restricted, Limited, Provisional, and
Light-Sport Aircraft (LSA). LSA receive a pink special
airworthiness certificate; however, there are exceptions.
For example, the Piper Cub is in the LSA category, but it
was certificated as a normal aircraft during its manufacture.
When purchasing an aircraft classified as other than Standard,
it is recommended that the local FSDO be contacted for an
explanation of the pertinent airworthiness requirements and
the limitations of such a certificate.
Aircraft Maintenance
Maintenance is defined as the preservation, inspection,
overhaul, and repair of an aircraft, including the replacement
of parts. Regular and proper maintenance ensures that
an aircraft meets an acceptable standard of airworthiness
throughout its operational life.
Although maintenance requirements vary for different
types of aircraft, experience shows that aircraft need some
type of preventive maintenance every 25 hours of flying
time or less and minor maintenance at least every 100
hours. This is influenced by the kind of operation, climatic
conditions, storage facilities, age, and construction of the
aircraft. Manufacturers supply maintenance manuals, parts
catalogs, and other service information that should be used
in maintaining the aircraft.

Aircraft Inspections
Under 14 CFR part 91, the primary responsibility for
maintaining an aircraft in an airworthy condition falls on the
owner or operator of the aircraft. Certain inspections must be
performed on the aircraft, and the owner must maintain the
airworthiness of the aircraft during the time between required
inspections by having any defects corrected.

9-8

Under 14 CFR, part 91, subpart E, all civil aircraft are
required to be inspected at specific intervals to determine
the overall condition. The interval depends upon the type
of operations in which the aircraft is engaged. All aircraft
need to be inspected at least once every 12 calendar months,
while inspection is required for others after every 100 hours
of operation. Some aircraft are inspected in accordance with
an inspection system set up to provide for total inspection
of the aircraft on the basis of calendar time, time in service,
number of system operations, or any combination of these.
All inspections should follow the current manufacturer's
maintenance manual, including the Instructions for
Continued Airworthiness concerning inspection intervals,
parts replacement, and life-limited items as applicable to
the aircraft.
Annual Inspection
Any reciprocating engine or single-engine turbojet/
turbopropeller-powered small aircraft (weighing 12,500
pounds or less) flown for business or pleasure and not
flown for compensation or hire is required to be inspected
at least annually. The inspection shall be performed by a
certificated airframe and powerplant (A\&P) mechanic who
holds an inspection authorization (IA) by the manufacturer
of the aircraft or by a certificated and appropriately rated
repair station. The aircraft may not be operated unless the
annual inspection has been performed within the preceding
12 calendar months. A period of 12 calendar months extends
from any day of a month to the last day of the same month the
following year. An aircraft overdue for an annual inspection
may be operated under a Special Flight Permit issued by
the FAA for the purpose of flying the aircraft to a location
where the annual inspection can be performed. However, all
applicable ADs that are due must be complied with before
the flight.
100-Hour Inspection
All aircraft under 12,500 pounds (except turbojet/
turbopropeller-powered multi-engine airplanes and turbine
powered rotorcraft), used to carry passengers for hire, must
receive a 100-hour inspection within the preceding 100 hours
of time in service and must be approved for return to service.
Additionally, an aircraft used for flight instruction for hire,
when provided by the person giving the flight instruction,
must also have received a 100-hour inspection. This inspection
must be performed by an FAA-certificated A\&P mechanic,
an appropriately rated FAA-certificated repair station, or
by the aircraft manufacturer. An annual inspection, or an
inspection for the issuance of an Airworthiness Certificate,
may be substituted for a required 100-hour inspection. The
100-hour limitation may be exceeded by no more than 10

hours for the purpose of traveling to a location at which the
required inspection can be performed. Any excess time used
for this purpose must be included in computing the next 100
hours of time in service.
Other Inspection Programs
The annual and 100-hour inspection requirements do not
apply to large (over 12,500 pounds) airplanes, turbojets, or
turbopropeller-powered multi-engine airplanes or to aircraft
for which the owner complies with a progressive inspection
program. Details of these requirements may be determined
by referencing 14 CFR, part 43, section 43.11 and 14 CFR
part 91, subpart E, or by inquiring at a local FSDO.

Altimeter System Inspection
Under 14 CFR, part 91, section 91.411, requires that the
altimeter, encoding altimeter, and related system must be
tested and inspected within the 24 months prior to operating in
controlled airspace under instrument flight rules (IFR). This
applies to all aircraft being operated in controlled airspace.

Transponder Inspection
Title 14 CFR, part 91, section 91.413, requires that before
a transponder can be used under 14 CFR, part 91, section
91.215(a), it shall be tested and inspected within the 24
months prior to operation of the aircraft regardless of airspace
restrictions.

Emergency Locator Transmitter
An emergency locator transmitter (ELT) is required by 14
CFR, part 91, section 91.207, and must be inspected within
12 calendar months after the last inspection for the following:


Proper installation



Battery corrosion



Operation of the controls and crash sensor



The presence of a sufficient signal radiated from its
antenna

The ELT must be attached to the airplane in such a manner
that the probability of damage to the transmitter in the event
of crash impact is minimized. Fixed and deployable automatic
type transmitters must be attached to the airplane as far aft
as practicable. Batteries used in the ELTs must be replaced
(or recharged, if the batteries are rechargeable):


When the transmitter has been in use for more than 1
cumulative hour



When 50 percent of the battery useful life or, for
rechargeable batteries, 50 percent of useful life of the
charge has expired

An expiration date for replacing (or recharging) the battery
must be legibly marked on the outside of the transmitter
and entered in the aircraft maintenance record. This does
not apply to batteries that are essentially unaffected during
storage intervals, such as water-activated batteries.

Preflight Inspections
The preflight inspection is a thorough and systematic means
by which a pilot determines if an aircraft is airworthy and in
condition for safe operation. POHs and owner/information
manuals contain a section devoted to a systematic method
of performing a preflight inspection.

Minimum Equipment Lists (MEL) and
Operations With Inoperative Equipment
Under 14 CFR, all aircraft instruments and installed equipment
are required to be operative prior to each departure. When the
FAA adopted the minimum equipment list (MEL) concept for
14 CFR part 91 operations, it allowed aircraft to be operated
with inoperative equipment determined to be nonessential for
safe flight. At the same time, it allowed part 91 operators,
without an MEL, to defer repairs on nonessential equipment
within the guidelines of part 91.
The FAA has two acceptable methods of deferring
maintenance on small rotorcraft, non-turbine powered
airplanes, gliders, or lighter-than-air aircraft operated under
part 91. They are the deferral provision of 14 CFR, part 91,
section 91.213(d) and an FAA-approved MEL.
The deferral provision of 14 CFR, part 91, section 91.213(d)
is widely used by most pilot/operators. Its popularity is due
to simplicity and minimal paperwork. When inoperative
equipment is found during a preflight inspection or prior to
departure, the decision should be to cancel the flight, obtain
maintenance prior to flight, or to defer the item or equipment.
Maintenance deferrals are not used for inflight discrepancies.
The manufacturer's AFM/POH procedures are to be used in
those situations. The discussion that follows assumes that
the pilot wishes to defer maintenance that would ordinarily
be required prior to flight.
Using the deferral provision of 14 CFR, part 91, section
91.213(d), the pilot determines whether the inoperative
equipment is required by type design, 14 CFR, or ADs. If
the inoperative item is not required, and the aircraft can be
safely operated without it, the deferral may be made. The
inoperative item shall be deactivated or removed and an
INOPERATIVE placard placed near the appropriate switch,
control, or indicator. If deactivation or removal involves

9-9

maintenance (removal always will), it must be accomplished
by certificated maintenance personnel and recorded in
accordance with 14 CFR part 43.
For example, if the position lights (installed equipment)
were discovered to be inoperative prior to a daytime flight,
the pilot would follow the requirements of 14 CFR, part 91,
section 91.213(d).
The deactivation may be a process as simple as the pilot
positioning a circuit breaker to the OFF position or as
complex as rendering instruments or equipment totally
inoperable. Complex maintenance tasks require a certificated
and appropriately rated maintenance person to perform the
deactivation. In all cases, the item or equipment must be
placarded INOPERATIVE.
All small rotorcraft, non-turbine powered airplanes, gliders,
or lighter-than-air aircraft operated under 14 CFR part 91
are eligible to use the maintenance deferral provisions of 14
CFR, part 91, section 91.213(d). However, once an operator
requests an MEL, and a Letter of Authorization (LOA)
is issued by the FAA, then the use of the MEL becomes
mandatory for that aircraft. All maintenance deferrals must
be accomplished in accordance with the terms and conditions
of the MEL and the operator-generated procedures document.
The use of an MEL for an aircraft operated under 14
CFR part 91 also allows for the deferral of inoperative
items or equipment. The primary guidance becomes the
FAA-approved MEL issued to that specific operator and
N-numbered aircraft.
The FAA has developed master minimum equipment lists
(MMELs) for aircraft in current use. Upon written request by
an operator, the local FSDO may issue the appropriate make
and model MMEL, along with an LOA, and the preamble.
The operator then develops operations and maintenance
(O\&M) procedures from the MMEL. This MMEL with O\&M
procedures now becomes the operator's MEL. The MEL,
LOA, preamble, and procedures document developed by the
operator must be on board the aircraft during each operation.
The FAA considers an approved MEL to be a supplemental
type certificate (STC) issued to an aircraft by serial number
and registration number. It, therefore, becomes the authority
to operate that aircraft in a condition other than originally
type certificated.
With an approved MEL, if the position lights were discovered
inoperative prior to a daytime flight, the pilot would make
an entry in the maintenance record or discrepancy record
provided for that purpose. The item would then either be
repaired or deferred in accordance with the MEL. Upon
9-10

confirming that daytime flight with inoperative position
lights is acceptable in accordance with the provisions of the
MEL, the pilot would leave the position lights switch OFF,
open the circuit breaker (or whatever action is called for in
the procedures document), and placard the position light
switch as INOPERATIVE.
There are exceptions to the use of the MEL for deferral. For
example, should a component fail that is not listed in the MEL
as deferrable (the tachometer, flaps, or stall warning device,
for example), then repairs are required to be performed prior
to departure. If maintenance or parts are not readily available
at that location, a special flight permit can be obtained from
the nearest FSDO. This permit allows the aircraft to be
flown to another location for maintenance. This allows an
aircraft that may not currently meet applicable airworthiness
requirements, but is capable of safe flight, to be operated
under the restrictive special terms and conditions attached
to the special flight permit.
Deferral of maintenance is not to be taken lightly, and due
consideration should be given to the effect an inoperative
component may have on the operation of an aircraft,
particularly if other items are inoperative. Further information
regarding MELs and operations with inoperative equipment
can be found in AC 91-67, Minimum Equipment Requirements
for General Aviation Operations Under FAR Part 91.

Preventive Maintenance
Preventive maintenance is regarded as simple or minor
preservation operations and the replacement of small standard
parts, not involving complex assembly operations. Allowed
items of preventative maintenance are listed and limited to
the items of 14 CFR part 43, appendix A(c).

Maintenance Entries
All pilots who perform preventive maintenance must make
an entry in the maintenance record of the aircraft. The entry
must include the following information:
1.	 A description of the work, such as "changed oil (Shell
Aero-50) at 2,345 hours"
2. 	 The date of completion of the work performed
3. 	 The pilot's name, signature, certificate number, and
type of certificate held

Examples of Preventive Maintenance
The following examples of preventive maintenance are taken
from 14 CFR, part 43, Maintenance, Preventive Maintenance,
Rebuilding, and Alternation, which should be consulted for a
more in-depth look at the preventive maintenance a pilot can
perform on an aircraft. Remember, preventive maintenance

is limited to work that does not involve complex assembly
operations including the following:








Removal, installation, and repair of landing gear
tires and shock cords; servicing landing gear shock
struts by adding oil, air, or both; servicing gear wheel
bearings; replacing defective safety wiring or cotter
keys; lubrication not requiring disassembly other than
removal of nonstructural items, such as cover plates,
cowlings, and fairings; making simple fabric patches
not requiring rib stitching or the removal of structural
parts or control surfaces. In the case of balloons,
the making of small fabric repairs to envelopes
(as defined in, and in accordance with, the balloon
manufacturer's instructions) not requiring load tape
repair or replacement.
Replenishing hydraulic fluid in the hydraulic
reservoir; refinishing decorative coating of fuselage,
balloon baskets, wings, tail group surfaces (excluding
balanced control surfaces), fairings, cowlings,
landing gear, cabin, or flight deck interior when
removal or disassembly of any primary structure
or operating system is not required; applying
preservative or protective material to components
where no disassembly of any primary structure or
operating system is involved and where such coating
is not prohibited or is not contrary to good practices;
repairing upholstery and decorative furnishings of the
cabin, flight deck, or balloon basket interior when the
repair does not require disassembly of any primary
structure or operating system or interfere with an
operating system or affect the primary structure of
the aircraft; making small, simple repairs to fairings,
nonstructural cover plates, cowlings, and small
patches and reinforcements not changing the contour
to interfere with proper air flow; replacing side
windows where that work does not interfere with the
structure or any operating system, such as controls,
electrical equipment, etc.
Replacing safety belts, seats or seat parts with
replacement parts approved for the aircraft, not
involving disassembly of any primary structure or
operating system, bulbs, reflectors, and lenses of
position and landing lights.
Replacing wheels and skis where no weight-and­
balance computation is involved; replacing any
cowling not requiring removal of the propeller or
disconnection of flight controls; replacing or cleaning
spark plugs and setting of spark plug gap clearance;
replacing any hose connection, except hydraulic
connections; however, prefabricated fuel lines may
be replaced.



Cleaning or replacing fuel and oil strainers or filter
elements; servicing batteries, cleaning balloon burner
pilot and main nozzles in accordance with the balloon
manufacturer's instructions.



The interchange of balloon baskets and burners on
envelopes when the basket or burner is designated as
interchangeable in the balloon type certificate data
and the baskets and burners are specifically designed
for quick removal and installation; adjustment
of nonstructural standard fasteners incidental to
operations.



The installations of anti-misfueling devices to reduce
the diameter of fuel tank filler openings only if the
specific device has been made a part of the aircraft type
certificate data by the aircraft manufacturer, the aircraft
manufacturer has provided FAA-approved instructions
for installation of the specific device, and installation
does not involve the disassembly of the existing tank
filler opening; troubleshooting and repairing broken
circuits in landing light wiring circuits.



Removing and replacing self-contained, front
instrument panel-mounted navigation and
communication devices that employ tray-mounted
connectors which connect the unit when the unit
is installed into the instrument panel (excluding
automatic flight control systems, transponders, and
microwave frequency distance measuring equipment
(DME)). The approved unit must be designed to be
readily and repeatedly removed and replaced, and
pertinent instructions must be provided. Prior to the
unit's intended use, an operational check must be
performed in accordance with the applicable sections
of 14 CFR part 91 on checking, removing, and
replacing magnetic chip detectors.



Inspection and maintenance tasks prescribed and
specifically identified as preventive maintenance in
a primary category aircraft type certificate or STC
holder's approved special inspection and preventive
maintenance program when accomplished on a
primary category aircraft.



Updating self-contained, front instrument panelmounted air traffic control (ATC) navigational
software databases (excluding those of automatic
flight control systems, transponders, and microwave
frequency DME), only if no disassembly of the unit is
required and pertinent instructions are provided; prior
to the unit's intended use, an operational check must
be performed in accordance with applicable sections
of 14 CFR part 91.

9-11

Certificated pilots, excluding student pilots, sport pilots, and
recreational pilots, may perform preventive maintenance on
any aircraft that is owned or operated by them provided that
the aircraft is not used in air carrier service and does not
qualify under 14 CFR parts 121, 129, or 135. A pilot holding
a sport pilot certificate may perform preventive maintenance
on an aircraft owned or operated by that pilot if that aircraft is
issued a special airworthiness certificate in the LSA category.
(Sport pilots operating LSA should refer to 14 CFR part 65
for maintenance privileges.) 14 CFR part 43, appendix A,
contains a list of the operations that are considered to be
preventive maintenance.
Repairs and Alterations
Repairs and alterations are classified as either major or minor.
14 CFR part 43, appendix A, describes the alterations and
repairs considered major. Major repairs or alterations shall
be approved for return to service on FAA Form 337, Major
Repair and Alteration, by an appropriately rated certificated
repair station, an FAA-certificated A\&P mechanic holding an
IA, or a representative of the Administrator. Minor repairs and
minor alterations may be approved for return to service with a
proper entry in the maintenance records by an FAA-certificated
A\&P mechanic or an appropriately certificated repair station.
For modifications of experimental aircraft, refer to the
operating limitations issued to that aircraft. Modifications
in accordance with FAA Order 8130.2, Airworthiness
Certification of Aircraft and Related Products, may require
the notification of the issuing authority.
Special Flight Permits
A special flight permit is a Special Airworthiness Certificate
authorizing operation of an aircraft that does not currently
meet applicable airworthiness requirements but is safe
for a specific flight. Before the permit is issued, an FAA
inspector may personally inspect the aircraft or require it to
be inspected by an FAA-certificated A\&P mechanic or an
appropriately certificated repair station to determine its safety
for the intended flight. The inspection shall be recorded in
the aircraft records.
The special flight permit is issued to allow the aircraft to be
flown to a base where repairs, alterations, or maintenance
can be performed; for delivering or exporting the aircraft; or
for evacuating an aircraft from an area of impending danger.
A special flight permit may be issued to allow the operation
of an overweight aircraft for flight beyond its normal range
over water or land areas where adequate landing facilities
or fuel is not available.
If a special flight permit is needed, assistance and the
necessary forms may be obtained from the local FSDO
9-12

or Designated Airworthiness Representative (DAR).
[Figure 9-10]

Airworthiness Directives (ADs)
A primary safety function of the FAA is to require correction
of unsafe conditions found in an aircraft, aircraft engine,
propeller, or appliance when such conditions exist and are
likely to exist or develop in other products of the same design.
The unsafe condition may exist because of a design defect,
maintenance, or other causes. Airworthiness Directives
(ADs), under 14 CFR, part 39, define the authority and
responsibility of the Administrator for requiring the necessary
corrective action. ADs are used to notify aircraft owners and
other interested persons of unsafe conditions and to specify
the conditions under which the product may continue to be
operated. ADs are divided into two categories:
1.	 Those of an emergency nature requiring immediate
compliance prior to further flight
2.	 Those of a less urgent nature requiring compliance
within a specified period of time
ADs are regulatory and shall be complied with unless a
specific exemption is granted. It is the responsibility of the
aircraft owner or operator to ensure compliance with all
pertinent ADs, including those ADs that require recurrent
or continuing action. For example, an AD may require a
repetitive inspection each 50 hours of operation, meaning
the particular inspection shall be accomplished and recorded
every 50 hours of time in service. Owners/operators are
reminded that there is no provision to overfly the maximum
hour requirement of an AD unless it is specifically written
into the AD. To help determine if an AD applies to an
amateur-built aircraft, contact the local FSDO.
14 CFR, part 91, section 91.417 requires a record to be
maintained that shows the current status of applicable
ADs, including the method of compliance; the AD number
and revision date, if recurring; next due date and time; the
signature; type of certificate; and certificate number of the
repair station or mechanic who performed the work. For ready
reference, many aircraft owners have a chronological listing
of the pertinent ADs in the back of their aircraft, engine, and
propeller maintenance records.
All ADs and the AD Biweekly are free on the Internet
at http://rgl.faa.gov and are available through e-mail.
Individuals can enroll for the e-mail service at the website
above. Paper copies of the Summary of Airworthiness
Directives and the AD Biweekly may be purchased from
the Superintendent of Documents. The Summary contains
all the valid ADs previously published and is divided into

UNITED STATES OF AMERICA
DEPARTMENT OF TRANSPORTATION - FEDERAL AVIATION ADMINISTRATION

SPECIAL AIRWORTHINESS CERTIFICATE

A
B
C
D
E

CATEGORY/DESIGNATION
Special Flight Permit
PURPOSE
Production Flight Testing or Customer Demonstration
MANUNAME
The Boeing Company
FACTURER
ADDRESS
P.O. Box 767, Renton WA 13567
FROM
N/A
FLIGHT
TO
N/A
SERIAL NO.
N/A
N- N/A
BUILDER
MODEL
N/A
N/A
EXPIRY
01/31/2001
DATE OF ISSUANCE 01/31/2001
OPERATING LIMITATIONS DATED 01/31/2001 ARE PART OF THIS CERTIFICATE
SIGNATURE OF FAA REPRESENTATIVE
DESIGNATION OR OFFICE NO.

Sam T. Smith
Sam T. Smith

NM-XX

Any alteration, reproduction or misuse of this certificate may be punishable by a fine not exceeding \$1,000 or imprisonment not exceeding 3 years, or
both. THIS CERTIFICATE MUST BE DISPLAYED IN THE AIRCRAFT IN ACCORDANCE WITH APPLICABLE TITLE 14, CODE OF FEDERAL
REGULATIONS (CFR).
FAA Form 8130-7 (07/04)
SEE REVERSE SIDE

Figure 9-10. FAA Form 8130-7, Special Airworthiness Certificate.

two areas. The small aircraft and helicopter books contain
all ADs applicable to small aircraft (12,500 pounds or less
maximum certificated takeoff weight) and ADs applicable
to all helicopters. The large aircraft books contain all ADs
applicable to large aircraft.
For current information on how to order paper copies of AD
books and the AD Biweekly, visit the FAA online regulatory
and guidance library at: http://rgl.faa.gov.

Aircraft Owner/Operator Responsibilities
The registered owner/operator of an aircraft is responsible for:


Having a current Airworthiness Certificate and a
Certificate of Aircraft Registration in the aircraft.



Maintaining the aircraft in an airworthy condition,
including compliance with all applicable ADs and
assuring that maintenance is properly recorded.



Keeping abreast of current regulations concerning the
operation and maintenance of the aircraft.



Notifying the FAA Aircraft Registry immediately of
any change of permanent mailing address, of the sale
or export of the aircraft, or of the loss of the eligibility
to register an aircraft. (Refer to 14 CFR, part 47,
section 47.41.)



Having a current Federal Communications Commission
(FCC) radio station license if equipped with radios,
including emergency locator transmitter (ELT), if
operated outside of the United States.

Chapter Summary
Knowledge of an aircraft's AFM/POH and documents,
such as ADs, provide pilots with ready access to pertinent
information needed to safely fly a particular aircraft. By
understanding the operations, limitations, and performance
characteristics of the aircraft, the pilot can make educated
flight decisions. By learning what preventive maintenance
is allowed on the aircraft, a pilot can maintain his or her
aircraft in an airworthy condition. The goal of every pilot is
a safe flight. Flight manuals and aircraft documentation are
essential tools used to reach that goal.

9-13

9-14


Chapter 10

Weight and Balance
Introduction
Compliance with the weight and balance limits of any aircraft
is critical to flight safety. Operating above the maximum
weight limitation compromises the structural integrity of
an aircraft and adversely affects its performance. Operation
with the center of gravity (CG) outside the approved limits
results in control difficulty.

Weight Control
As discussed in Chapter 5, Aerodynamics of Flight, weight
is the force with which gravity attracts a body toward the
center of the Earth. It is a product of the mass of a body and
the acceleration acting on the body. Weight is a major factor
in aircraft construction and operation and demands respect
from all pilots.
The force of gravity continuously attempts to pull an aircraft
down toward Earth. The force of lift is the only force that
counteracts weight and sustains an aircraft in flight. The
amount of lift produced by an airfoil is limited by the airfoil
design, angle of attack (AOA), airspeed, and air density. To
assure that the lift generated is sufficient to counteract weight,
loading an aircraft beyond the manufacturer's recommended
weight must be avoided. If the weight is greater than the lift
generated, the aircraft may be incapable of flight.

10-1

Effects of Weight
Any item aboard an aircraft that increases the total weight is
undesirable for performance. Manufacturers attempt to make
an aircraft as light as possible without sacrificing strength
or safety.
The pilot should always be aware of the consequences of
overloading. An overloaded aircraft may not be able to leave
the ground, or if it does become airborne, it may exhibit
unexpected and unusually poor flight characteristics. If not
properly loaded, the initial indication of poor performance
usually takes place during takeoff.
Excessive weight reduces the flight performance in almost
every respect. For example, the most important performance
deficiencies of an overloaded aircraft are:


Higher takeoff speed



Longer takeoff run



Reduced rate and angle of climb



Lower maximum altitude



Shorter range



Reduced cruising speed



Reduced maneuverability



Higher stalling speed



Higher approach and landing speed



Longer landing roll



Excessive weight on the nose wheel or tail wheel

The pilot must be knowledgeable about the effect of weight
on the performance of the particular aircraft being flown.
Preflight planning should include a check of performance
charts to determine if the aircraft's weight may contribute
to hazardous flight operations. Excessive weight in itself
reduces the safety margins available to the pilot and becomes
even more hazardous when other performance-reducing
factors are combined with excess weight. The pilot must
also consider the consequences of an overweight aircraft if
an emergency condition arises. If an engine fails on takeoff
or airframe ice forms at low altitude, it is usually too late to
reduce an aircraft's weight to keep it in the air.
Weight Changes
The operating weight of an aircraft can be changed by
simply altering the fuel load. Gasoline has considerable
weight—6 pounds per gallon. Thirty gallons of fuel may
weigh more than one passenger. If a pilot lowers airplane
weight by reducing fuel, the resulting decrease in the range
of the airplane must be taken into consideration during flight
planning. During flight, fuel burn is normally the only weight
10-2

change that takes place. As fuel is used, an aircraft becomes
lighter and performance is improved.
Changes of fixed equipment have a major effect upon the
weight of an aircraft. The installation of extra radios or
instruments, as well as repairs or modifications, may also
affect the weight of an aircraft.

Balance, Stability, and Center of Gravity
Balance refers to the location of the CG of an aircraft, and is
important to stability and safety in flight. The CG is a point
at which the aircraft would balance if it were suspended at
that point.
The primary concern in balancing an aircraft is the fore
and aft location of the CG along the longitudinal axis. The
CG is not necessarily a fixed point; its location depends on
the distribution of weight in the aircraft. As variable load
items are shifted or expended, there is a resultant shift in
CG location. The distance between the forward and back
limits for the position of the center for gravity or CG range
is certified for an aircraft by the manufacturer. The pilot
should realize that if the CG is displaced too far forward
on the longitudinal axis, a nose-heavy condition will
result. Conversely, if the CG is displaced too far aft on the
longitudinal axis, a tail heavy condition results. It is possible
that the pilot could not control the aircraft if the CG location
produced an unstable condition. [Figure 10-1]
Location of the CG with reference to the lateral axis is also
important. For each item of weight existing to the left of
Empty

Full

Lateral unbalance will cause wing heaviness.
Excess baggage

Longitudinal unbalance will cause
either nose or tail heaviness.
Figure 10-1. Lateral and longitudinal unbalance.

the fuselage centerline, there is an equal weight existing at
a corresponding location on the right. This may be upset
by unbalanced lateral loading. The position of the lateral
CG is not computed in all aircraft, but the pilot must be
aware that adverse effects arise as a result of a laterally
unbalanced condition. In an airplane, lateral unbalance occurs
if the fuel load is mismanaged by supplying the engine(s)
unevenly from tanks on one side of the airplane. The pilot
can compensate for the resulting wing-heavy condition by
adjusting the trim or by holding a constant control pressure.
This action places the aircraft controls in an out-of-streamline
condition, increases drag, and results in decreased operating
efficiency. Since lateral balance is addressed when needed in
the aircraft flight manual (AFM) and longitudinal balance is
more critical, further reference to balance in this handbook
means longitudinal location of the CG.
Flying an aircraft that is out of balance can produce increased
pilot fatigue with obvious effects on the safety and efficiency
of flight. The pilot's natural correction for longitudinal
unbalance is a change of trim to remove the excessive control
pressure. Excessive trim, however, has the effect of reducing
not only aerodynamic efficiency but also primary control
travel distance in the direction the trim is applied.
Effects of Adverse Balance
Adverse balance conditions affect flight characteristics in
much the same manner as those mentioned for an excess
weight condition. It is vital to comply with weight and
balance limits established for all aircraft. Operating above
the maximum weight limitation compromises the structural
integrity of the aircraft and can adversely affect performance.
Stability and control are also affected by improper balance.

Stability
Loading in a nose-heavy condition causes problems in
controlling and raising the nose, especially during takeoff
and landing. Loading in a tail heavy condition has a serious
effect upon longitudinal stability, and reduces the capability
to recover from stalls and spins. Tail heavy loading also
produces very light control forces, another undesirable
characteristic. This makes it easy for the pilot to inadvertently
overstress an aircraft.
Stability and Center of Gravity
Limits for the location of the CG are established by the
manufacturer. These are the fore and aft limits beyond
which the CG should not be located for flight. These limits
are published for each aircraft in the Type Certificate Data
Sheet (TCDS), or aircraft specification and the AFM or
pilot's operating handbook (POH). If the CG is not within the
allowable limits after loading, it will be necessary to relocate
some items before flight is attempted.

The forward CG limit is often established at a location that
is determined by the landing characteristics of an aircraft.
During landing, one of the most critical phases of flight,
exceeding the forward CG limit may result in excessive loads
on the nosewheel, a tendency to nose over on tailwheel type
airplanes, decreased performance, higher stalling speeds, and
higher control forces.

Control
In extreme cases, a CG location that is beyond the forward
limit may result in nose heaviness, making it difficult or
impossible to flare for landing. Manufacturers purposely
place the forward CG limit as far rearward as possible to
aid pilots in avoiding damage when landing. In addition to
decreased static and dynamic longitudinal stability, other
undesirable effects caused by a CG location aft of the
allowable range may include extreme control difficulty,
violent stall characteristics, and very light control forces
which make it easy to overstress an aircraft inadvertently.
A restricted forward CG limit is also specified to assure
that sufficient elevator/control deflection is available at
minimum airspeed. When structural limitations do not limit
the forward CG position, it is located at the position where
full-up elevator/control deflection is required to obtain a high
AOA for landing.
The aft CG limit is the most rearward position at which the
CG can be located for the most critical maneuver or operation.
As the CG moves aft, a less stable condition occurs, which
decreases the ability of the aircraft to right itself after
maneuvering or turbulence.
For some aircraft, both fore and aft CG limits may be
specified to vary as gross weight changes. They may also
be changed for certain operations, such as acrobatic flight,
retraction of the landing gear, or the installation of special
loads and devices that change the flight characteristics.
The actual location of the CG can be altered by many variable
factors and is usually controlled by the pilot. Placement of
baggage and cargo items determines the CG location. The
assignment of seats to passengers can also be used as a means
of obtaining a favorable balance. If an aircraft is tail heavy,
it is only logical to place heavy passengers in forward seats.
Fuel burn can also affect the CG based on the location of the
fuel tanks. For example, most small aircraft carry fuel in the
wings very near the CG and burning off fuel has little effect
on the loaded CG.

10-3

Management of Weight and Balance Control
Title 14 of the Code of Federal Regulations (14 CFR) part 23,
section 23.23 requires establishment of the ranges of weights
and CGs within which an aircraft may be operated safely. The
manufacturer provides this information, which is included in
the approved AFM, TCDS, or aircraft specifications.
While there are no specified requirements for a pilot operating
under 14 CFR part 91 to conduct weight and balance
calculations prior to each flight, 14 CFR part 91, section
91.9 requires the pilot in command (PIC) to comply with the
operating limits in the approved AFM. These limits include
the weight and balance of the aircraft. To enable pilots to
make weight and balance computations, charts and graphs
are provided in the approved AFM.
Weight and balance control should be a matter of concern to
all pilots. The pilot controls loading and fuel management
(the two variable factors that can change both total weight
and CG location) of a particular aircraft. The aircraft owner
or operator should make certain that up-to-date information
is available for pilot use, and should ensure that appropriate
entries are made in the records when repairs or modifications
have been accomplished. The removal or addition of
equipment results in changes to the CG.
Weight changes must be accounted for and the proper
notations made in weight and balance records. The
equipment list must be updated, if appropriate. Without such
information, the pilot has no foundation upon which to base
the necessary calculations and decisions.
Standard parts with negligible weight or the addition of minor
items of equipment such as nuts, bolts, washers, rivets, and
similar standard parts of negligible weight on fixed-wing
aircraft do not require a weight and balance check. The
following criteria for negligible weight change is outlined
in Advisory Circular (AC) 43.13-1 (as revised), Methods
Techniques and Practices—Aircraft Inspection and Repair:


One pound or less for an aircraft whose weight empty
is less than 5,000 pounds



Two pounds or less for aircraft with an empty weight
of more than 5,000 pounds to 50,000 pounds



Five pounds or less for aircraft with an empty weight
of more than 50,000 pounds

Negligible CG change is any change of less than 0.05 percent
Mean Aerodynamic Chord (MAC) for fixed-wing aircraft
or 0.2 percent for rotary wing aircraft. MAC is the average
distance from the leading edge to the trailing edge of the
wing. Exceeding these limits would require a weight and
balance check.
10-4

Before any flight, the pilot should determine the weight
and balance condition of the aircraft. Simple and orderly
procedures based on sound principles have been devised
by the manufacturer for the determination of loading
conditions. The pilot uses these procedures and exercises
good judgment when determining weight and balance. In
many modern aircraft, it is not possible to fill all seats,
baggage compartments, and fuel tanks, and still remain within
the approved weight and balance limits. If the maximum
passenger load is carried, the pilot must often reduce the fuel
load or reduce the amount of baggage.
14 CFR part 125 requires aircraft with 20 or more seats
or maximum payload capacity of 6,000 pounds or more
to be weighed every 36 calendar months. Multi-engine
aircraft operated under 14 CFR part 135 are also required
to be weighed every 36 months. Aircraft operated under 14
CFR part 135 are exempt from the 36 month requirement
if operated under a weight and balance system approved in
the operations specifications of the certificate holder. For
additional information on approved weight and balance
control programs for operations under parts 121 and 135,
reference the current edition of AC 120-27, Aircraft Weight
and Balance Control. AC 43.13-l, Acceptable Methods,
Techniques and Practices—Aircraft Inspection and Repair
also requires that the aircraft mechanic ensure that the weight
and balance data in the aircraft records is current and accurate
after a 100-hour or annual inspection.
Terms and Definitions
The pilot should be familiar with the appropriate terms
regarding weight and balance. The following list of terms
and their definitions is standardized, and knowledge of these
terms aids the pilot to better understand weight and balance
calculations of any aircraft. Terms defined by the General
Aviation Manufacturers Association (GAMA) as industry
standard are marked in the titles with GAMA.
 	 Arm (moment arm)—the horizontal distance in inches
from the reference datum line to the CG of an item.
The algebraic sign is plus (+) if measured aft of the
datum and minus (–) if measured forward of the datum.
 	 Basic empty weight (GAMA)—the standard empty
weight plus the weight of optional and special
equipment that have been installed.
 	 Center of gravity (CG)—the point about which an
aircraft would balance if it were possible to suspend it
at that point. It is the mass center of the aircraft or the
theoretical point at which the entire weight of the aircraft
is assumed to be concentrated. It may be expressed in
inches from the reference datum or in percent of MAC.
The CG is a three-dimensional point with longitudinal,
lateral, and vertical positioning in the aircraft.

 	 CG limits—the specified forward and aft points within
which the CG must be located during flight. These
limits are indicated on pertinent aircraft specifications.
 	 CG range—the distance between the forward and aft
CG limits indicated on pertinent aircraft specifications.
 	 Datum (reference datum)—an imaginary vertical
plane or line from which all measurements of arm are
taken. The datum is established by the manufacturer.
Once the datum has been selected, all moment arms
and the location of CG range are measured from this
point.
 	 Delta—a Greek letter expressed by the symbol r to
indicate a change of values. As an example, rCG
indicates a change (or movement) of the CG.
 	 Floor load limit—the maximum weight the floor
can sustain per square inch/foot as provided by the
manufacturer.
 	 Fuel load—the expendable part of the load of the
aircraft. It includes only usable fuel, not fuel required
to fill the lines or that which remains trapped in the
tank sumps.
 	 Licensed empty weight—the empty weight that
consists of the airframe, engine(s), unusable fuel, and
undrainable oil plus standard and optional equipment
as specified in the equipment list. Some manufacturers
used this term prior to GAMA standardization.

 	 Moment index (or index)—a moment divided by a
constant such as 100, 1,000, or 10,000. The purpose
of using a moment index is to simplify weight and
balance computations of aircraft where heavy items
and long arms result in large, unmanageable numbers.
 	 Payload (GAMA)—the weight of occupants, cargo,
and baggage.
 	 Standard empty weight (GAMA)—aircraft weight
that consists of the airframe, engines, and all items of
operating equipment that have fixed locations and are
permanently installed in the aircraft, including fixed
ballast, hydraulic fluid, unusable fuel, and full engine
oil.
 	 Standard weights—established weights for numerous
items involved in weight and balance computations.
These weights should not be used if actual weights
are available. Some of the standard weights are:
Gasoline.................................................. 6 lb/US gal

Jet A, Jet A-1....................................... 6.8 lb/US gal

Jet B......................................................6.5 lb/US gal

Oil.........................................................7.5 lb/US gal

Water ................................................. 8.35 lb/US gal


 	 Maximum landing weight—the greatest weight that
an aircraft is normally allowed to have at landing.

 	 Station—a location in the aircraft that is identified by
a number designating its distance in inches from the
datum. The datum is, therefore, identified as station
zero. An item located at station +50 would have an
arm of 50 inches.

 	 Maximum ramp weight—the total weight of a loaded
aircraft including all fuel. It is greater than the takeoff
weight due to the fuel that will be burned during the
taxi and run-up operations. Ramp weight may also be
referred to as taxi weight.

 	 Useful load—the weight of the pilot, copilot,
passengers, baggage, usable fuel, and drainable oil.
It is the basic empty weight subtracted from the
maximum allowable gross weight. This term applies
to general aviation (GA) aircraft only.

 	 Maximum takeoff weight—the maximum allowable
weight for takeoff.

Principles of Weight and Balance Computations
It is imperative that all pilots understand the basic principles
of weight and balance determination. The following methods
of computation can be applied to any object or vehicle for
which weight and balance information is essential.

 	 Maximum weight—the maximum authorized weight
of the aircraft and all of its equipment as specified in
the TCDS for the aircraft.
 	 Maximum zero fuel weight (GAMA)—the maximum
weight, exclusive of usable fuel.
 	 Mean aerodynamic chord (MAC)—the average
distance from the leading edge to the trailing edge of
the wing.
 	 Moment—the product of the weight of an item
multiplied by its arm. Moments are expressed in
pound-inches (in-lb). Total moment is the weight of
the airplane multiplied by the distance between the
datum and the CG.

By determining the weight of the empty aircraft and adding
the weight of everything loaded on the aircraft, a total weight
can be determined—a simple concept. A greater problem,
particularly if the basic principles of weight and balance are
not understood, is distributing this weight in such a manner
that the entire mass of the loaded aircraft is balanced around
a point (CG) that must be located within specified limits.
The point at which an aircraft balances can be determined by
locating the CG, which is, as stated in the definitions of terms,

10-5

the imaginary point at which all the weight is concentrated.
To provide the necessary balance between longitudinal
stability and elevator control, the CG is usually located
slightly forward of the center of lift. This loading condition
causes a nose-down tendency in flight, which is desirable
during flight at a high AOA and slow speeds.
As mentioned earlier, a safe zone within which the balance
point (CG) must fall is called the CG range. The extremities
of the range are called the forward CG limits and aft CG
limits. These limits are usually specified in inches, along the
longitudinal axis of the airplane, measured from a reference
point called a datum reference. The datum is an arbitrary
point, established by aircraft designers that may vary in
location between different aircraft. [Figure 10-2]
The distance from the datum to any component part or any
object loaded on the aircraft is called the arm. When the
object or component is located aft of the datum, it is measured
in positive inches; if located forward of the datum, it is
measured as negative inches or minus inches. The location
of the object or part is often referred to as the station. If
the weight of any object or component is multiplied by the
distance from the datum (arm), the product is the moment.
The moment is the measurement of the gravitational force
that causes a tendency of the weight to rotate about a point
or axis and is expressed in inch-pounds (in-lb).
To illustrate, assume a weight of 50 pounds is placed on
the board at a station or point 100 inches from the datum.
The downward force of the weight can be determined by
multiplying 50 pounds by 100 inches, which produces a
moment of 5,000 in-lb. [Figure 10-3]
To establish a balance, a total of 5,000 in-lb must be applied
to the other end of the board. Any combination of weight
and distance which, when multiplied, produces a 5,000 inlb moment will balance the board. For example (illustrated

Datum

100"

50
lb
Fulcrum
Moment = 5,000 in-lb
Wt x Arm = Moment
(lb) x (in) = (in-lb)
50 x 100 = 5,000

Note: The datum is assumed to be
located at the fulcrum.

Figure 10-3. Determining moment.

in Figure 10-4), if a 100-pound weight is placed at a point
(station) 25 inches from the datum, and another 50-pound
weight is placed at a point (station) 50 inches from the datum,
the sum of the product of the two weights and their distances
total a moment of 5,000 in-lb, which will balance the board.
Weight and Balance Restrictions
An aircraft's weight and balance restrictions should be
closely followed. The loading conditions and empty weight
of a particular aircraft may differ from that found in the
AFM/POH because modifications or equipment changes
may have been made. Sample loading problems in the
AFM/POH are intended for guidance only; therefore, each
aircraft must be treated separately. Although an aircraft is
certified for a specified maximum gross takeoff weight, it
may not safely take off at this weight under all conditions.
Conditions that affect takeoff and climb performance, such as
high elevations, high temperatures, and high humidity (high
density altitudes), may require a reduction in weight before
flight is attempted. Other factors to consider when computing

Datum

Fwd limit
Datum

CG
range

50"
25"

Aft limit

(–)
Arm

100"

(+)
Arm

50
lb

100
lb

50
lb
Fulcrum

Moment = 700 in-lb

2,500
in-lb

2,500
in-lb

5,000
in-lb

( + ) Arm 70"

10 lb
Sta 0

Sta 70

Figure 10-2. Weight and balance.

10-6

Wt x Arm = Moment
(lb) x (in) = (in-lb)

100 x 25 = 2,500
50 x 50 = 2,500
Total = 5,000

Figure 10-4. Establishing a balance.

weight and balance distribution prior to takeoff are runway
length, runway surface, runway slope, surface wind, and the
presence of obstacles. These factors may require a reduction
in or redistribution of weight prior to flight.
Some aircraft are designed so that it is difficult to load them
in a manner that places the CG out of limits. These are
usually small aircraft with the seats, fuel, and baggage areas
located near the CG limit. Pilots must be aware that while
within CG limits these aircraft can be overloaded in weight.
Other aircraft can be loaded in such a manner that they will
be out of CG limits even though the useful load has not been
exceeded. Because of the effects of an out-of-balance or
overweight condition, a pilot should always be sure that an
aircraft is properly loaded.

Determining Loaded Weight and CG
There are various methods for determining the loaded weight
and CG of an aircraft. There is the computational method as
well as methods that utilize graphs and tables provided by
the aircraft manufacturer.
Computational Method
The following is an example of the computational method
involving the application of basic math functions.
Aircraft Allowances:
Maximum gross weight......................3,400 pounds
CG range.............................................78–86 inches
Given:
Weight of front seat occupants.............340 pounds
Weight of rear seat occupants..............350 pounds
Fuel...........................................................75 gallons
Weight of baggage in area 1....................80 pounds
1.	 List the weight of the aircraft, occupants, fuel, and
baggage. Remember that aviation gas (AVGAS)
weighs 6 pounds per gallon and is used in this
example.
2.	 Enter the moment for each item listed. Remember
"weight x arm = moment."
3.	 Find the total weight and total moment.
4.	 To determine the CG, divide the total moment by the
total weight.
NOTE: The weight and balance records for a particular
aircraft provide the empty weight and moment, as well as the
information on the arm distance. [Figure 10-5]
The total loaded weight of 3,320 pounds does not exceed
the maximum gross weight of 3,400 pounds, and the CG of

Weight

Arm

Moment

Aircraft Empty Weight
Front Seat Occupants
Rear Seat Occupants
Fuel
Baggage Area 1

Item

2,100
340
350
450
80

78.3
85.0
121.0
75.0
150.0

164,430
28,900
42,350
33,750
12,000

Total

3,320

281,430
281,430 ÷ 3,320 = 84.8

Figure 10-5. Example of weight and balance computations.

84.8 is within the 78–86 inch range; therefore, the aircraft is
loaded within limits.
Graph Method
Another method for determining the loaded weight and CG is
the use of graphs provided by the manufacturers. To simplify
calculations, the moment may sometimes be divided by 100,
1,000, or 10,000. [Figures 10-6, 10-7, and 10-8]
Front seat occupants....................................340 pounds

Rear seat occupants ......................................300 pounds

Fuel.................................................................40 gallons

Baggage area 1 ...............................................20 pounds

The same steps should be followed in the graph method as
were used in the computational method except the graphs
provided will calculate the moments and allow the pilot to
determine if the aircraft is loaded within limits. To determine
the moment using the loading graph, find the weight and draw
a line straight across until it intercepts the item for which the
moment is to be calculated. Then draw a line straight down
to determine the moment. (The red line on the loading graph
in Figure 10-7 represents the moment for the pilot and front
passenger. All other moments were determined the same
Sample Loading Problem

Moment
Weight (lb) (in-lb/1,000)

1. Basic empty weight (Use data pertaining
to aircraft as it is presently equipped)
includes unusable fuel and full oil
1,467
2. Usable fuel (At 6 lb/gal)
240
Standard tanks (40 gal maximum)
Long range tanks (50 gal maximum)
Integral tanks (62 gal maximum)
Integral reduced fuel (42 gal)
3. Pilot and front passenger (Station 34
to 46)
340
4. Rear passengers
300
5. Baggage area 1 or passenger on child's
seat (Station 82 to 108, 120 lb maximum)
20
6. Baggage area 2
(Station 108 to 142, 50 lb maximum)
7. Weight and moment
2,367

57.3
11.5

12.7
21.8
1.9

105.2

Figure 10-6. Weight and balance data.

10-7

Load Moment/1,000 (kilogram-millimeters)
0

50

100

150

200

250

300

350

400
200

400
62 gal***(234.7 liters)

ar

Re

p

175
150

lb

100

lot
Pi

30 gal (113.6 liters)

75

150
20 gal (75.7 liters)

100

Maximum Usable Fuel

Baggage area 1 or
passenger on child's seat

* Standard tanks
** Long range tanks
*** Internal tanks

10 gal (37.9 liters)

50

Load Weight (kilograms)

/g

en
ss

l(
6

42 gal reduced***(159 liters)

\&

200

125

40 gal*(189.3 liters)

Fu
e

fro

nt

pa

250

50 gal**(189.3 liters)

al

;0

ge

r

.7

2

300
Load Weight (pounds)

60 gal (227.1 liters)

kg

/li

te

r)

350
340

as

rs

ge

n
se

50
25

Baggage area 2
0

5

12.7

10

15

20

25

30

0

Load Moment/1,000 (inch-pounds)
Figure 10-7. Loading graph.
Loaded Aircraft Moment/1,000 (kilogram-millimeters)
600

700

900

800

1,000

1,100

1,200

1,100

2,400
2,367

Normal
category

2,200

1,000

2,100

950

2,000

te
go
ry

900
850

ilit

y

ca

1,900

Ut

1,800

800

1,700

750
1,600

700
1,500
45

50

55

60

65

70

75

80

85

90

95

Loaded Aircraft Moment/1,000 (inch-pounds)
Figure 10-8. CG moment envelope.

10-8

100

105
110
105.2

Loaded Airplane Weight (kilograms)

1,050

2,300
Loaded Aircraft Weight (pounds)

1,300

Table Method
The table method applies the same principles as the
computational and graph methods. The information
and limitations are contained in tables provided by the
manufacturer. Figure 10-9 is an example of a table and a

way.) Once this has been done for each item, total the weight
and moments and draw a line for both weight and moment
on the CG envelope graph. If the lines intersect within the
envelope, the aircraft is loaded within limits. In this sample
loading problem, the aircraft is loaded within limits.

Occupants

Usable Fuel

Front Seat
Arm 85

Main Wing Tanks Arm 75
Weight

5
10
15
20
25
30
35
40
44

30
60
90
120
150
180
210
240
264

22
45
68
90
112
135
158
180
198

Moment
100

Weight

Moment
100

120
130
140
150
160
170
180
190
200

102
110
119
128
136
144
153
162
170

120
130
140
150
160
170
180
190
200

145
157
169
182
194
206
218
230
242

FUEL TANK

OIL

BAGGAGE AREA

*Oil
Quarts

Weight

Moment
100

10

19

5

Baggage or 5th
Seat Occupant
Arm 140
FUEL TANK

*Included in basic empty weight
Empty Weight \~ 2,015
MOM/ 100 \~ 1,554

Moment Limits vs Weight
Moment limits are based on the following weight and
center of gravity limit data (landing gear down).
Weight
Condition
2,950 lb (takeoff
or landing)
2,525 lb
2,475 lb or less

Forward
CG Limit
82.1
77.5
77.0

Sample Loading Problem

AFT
CG Limit
84.7
85.7
85.7

Weight

Basic empty weight
Fuel main tanks (44 gal)
*Front seat passengers
*Rear seat passengers
Baggage

2,015
264
300
190
30

Total

2,799

Rear Seats
Arm 121

Weight

SEATING AREA

Gallons

Moment
100

Minimum Maximum
Weight Moment Moment
100
100

Weight

10
20
30
40
50
60
70
80
90
100
110
120
130
140
150
160
170
Moment
180
1,554
190
200
198
210
254
220
230
230
240 42
250
260
2,278/100
270

* Interpolate or, as in this case, add appropriate numbers.

Moment
100
14
28
42
56
70
84
98
112
126
140
154
168
182
196
210
224
238
252
266
280
294
308
322
336
350
364
378

Minimum Maximum
Weight Moment Moment
100
100
2,100
2,110
2,120
2,130
2,140
2,150
2,160
2,170
2,180
2,190

1,617
1,625
1,632
1,640
1,648
1,656
1,663
1,671
1,679
1,686

1,800
1,808
1,817
1,825
1,834
1,843
1,851
1,860
1,868
1,877

2,200
2,210
2,220
2,230
2,240
2,250
2,260
2,270
2,280
2,290
2,300
2,310
2,320
2,330
2,340
2,350
2,360
2,370
2,380
2,390

1,694
1,702
1,709
1,717
1,725
1,733
1,740
1,748
1,756
1,763
1,771
1,779
1,786
1,794
1,802
1,810
1,817
1,825
1,833
1,840

1,885
1,894
1,903
1,911
1,920
1,928
1,937
1,945
1,954
1,963
1,971
1,980
1,988
1,997
2,005
2,014
2,023
2,031
2,040
2,048

2,400
2,410
2,420
2,430
2,440
2,450
2,460
2,470
2,480
2,490

1,848
1,856
1,863
1,871
1,879
1,887
1,894
1,092
1,911
1,921

2,057
2,065
2,074
2,083
2,091
2,100
2,108
2,117
2,125
2,134

2,500
2,510
2,520
2,530
2,540
2,550
2,560
2,570
2,580
2,590

1,932
1,942
1,953
1,963
1,974
1,984
1,995
2,005
2,016
2,026

2,143
2,151
2,160
2,168
2,176
2,184
2,192
2,200
2,208
2,216

2,600
2,610
2,620
2,630
2,640
2,650
2,660
2,670
2,680
2,690

2,037
2,048
2,058
2,069
2,080
2,090
2,101
2,112
2,123
2,133

2,224
2,232
2,239
2,247
2,255
2,263
2,271
2,279
2,287
2,295

2,700
2,710
2,720
2,730
2,740
2,750
2,760
2,770
2,780
2,790

2,144
2,155
2,166
2,177
2,188
2,199
2,210
2,221
2,232
2,243

2,303
2,311
2,319
2,326
2,334
2,342
2,350
2,358
2,366
2,374

2,800
2,810
2,820
2,830
2,840
2,850
2,860
2,870
2,880
2,890
2,900
2,910
2,920
2,930
2,940
2,950

2,254
2,265
2,276
2,287
2,298
2,309
2,320
2,332
2,343
2,354
2,365
2,377
2,388
2,399
2,411
2,422

2,381
2,389
2,397
2,405
2,413
2,421
2,426
2,436
2,444
2,452
3,460
2,468
2,475
2,483
2,491
2,499

Figure 10-9. Loading schedule placard.

10-9

weight and balance calculation based on that table. In this
problem, the total weight of 2,799 pounds and moment of
2,278/100 are within the limits of the table.
Computations With a Negative Arm
Figure 10-10 is a sample of weight and balance computation
using an aircraft with a negative arm. It is important to
remember that a positive times a negative equals a negative,
and a negative would be subtracted from the total moments.
Computations With Zero Fuel Weight
Figure 10-11 is a sample of weight and balance computation
using an aircraft with a zero fuel weight. In this example,
the total weight of the aircraft less fuel is 4,240 pounds,
which is under the zero fuel weight of 4,400 pounds. If the
total weight of the aircraft without fuel had exceeded 4,400
pounds, passengers or cargo would have needed to be reduced
to bring the weight at or below the max zero fuel weight.
Shifting, Adding, and Removing Weight
A pilot must be able to solve any problems accurately that
involve the shift, addition, or removal of weight. For example,
the pilot may load the aircraft within the allowable takeoff
weight limit, then find that the CG limit has been exceeded.
The most satisfactory solution to this problem is to shift
baggage, passengers, or both. The pilot should be able to
determine the minimum load shift needed to make the aircraft
safe for flight. Pilots should be able to determine if shifting a
load to a new location will correct an out-of-limit condition.
There are some standardized calculations that can help make
these determinations.

Weight Shifting
When weight is shifted from one location to another, the
total weight of the aircraft is unchanged. The total moments,
however, do change in relation and proportion to the
direction and distance the weight is moved. When weight is
moved forward, the total moments decrease; when weight
is moved aft, total moments increase. The moment change
is proportional to the amount of weight moved. Since many
Item

Weight

Arm

Moment

Licensed empty weight
Oil (6 quarts)
Fuel (18 gallons)
Fuel, auxiliary (18 gallons)
Pilot
Passenger
Baggage

1,011.9
11.0
108.0
108.0
170.0
170.0
70.0

68.6
−31.0
84.0
84.0
81.0
81.0
105.0

69,393.0
−341.0
9,072.0
9,072.0
13,770.0
13,770.0
7,350.0

Total
CG

1,648.9

122,086.0
74.0

Figure 10-10. Sample weight and balance using a negative.

10-10

Item

Weight

Basic empty weight
Front seat occupants
3rd \& 4th seat occupants
forward facing
5th \& 6th seat occupants
Nose baggage
Aft baggage
Zero fuel weight max 4,400 pounds
Subtotal
Fuel
Ramp weight max 5,224 pounds
Subtotal ramp weight
* Less fuel for start,
taxi, and takeoff
Subtotal
takeoff weight
Less fuel to destination
Max landing weight 4,940 pounds
Actual landing weight

Arm

Moment

3,230 CG 90.5 292,315.0
335
89.0 29,815.0
350

126.0

44,100.0

200
100
25

157.0
10.0
183.0

31,400.0
1,000.0
4,575.0

4,240 CG 95.1 403,205.0
822

113.0

92,886.0

5,062 CG 98.0 496,091.0
−24

113.0

−2,712.0

5,038 CG 97.9 493,379.0
−450

113.0 −50,850.0

4,588 CG 96.5 442,529.0

*Fuel for start, taxi, and takeoff is normally 24 pounds.

Figure 10-11. Sample weight and balance using an aircraft with a

published zero fuel weight.

aircraft have forward and aft baggage compartments, weight
may be shifted from one to the other to change the CG. If
starting with a known aircraft weight, CG, and total moments,
calculate the new CG (after the weight shift) by dividing the
new total moments by the total aircraft weight.
To determine the new total moments, find out how many
moments are gained or lost when the weight is shifted.
Assume that 100 pounds has been shifted from station 30 to
station 150. This movement increases the total moments of
the aircraft by 12,000 in-lb.
Moment when
at station 150

= 100 lb x 150 in = 15,000 in-lb

Moment when
at station 30

= 100 lb x 30 in = 3,000 in-lb

Moment change = [15,000 – 3,000] = 12,000 in-lb
By adding the moment change to the original moment (or
subtracting if the weight has been moved forward instead of
aft), the new total moments are obtained. Then determine the
new CG by dividing the new moments by the total weight:
Total moments =

616,000 in-lb + 12,000 in-lb = 628,000 in-lb

CG = 628,000 in-lb = 78.5 in
8,000 lb
The shift has caused the CG to shift to station 78.5.

Example

Example
Weight shifted
∆CG (change of CG)
=
Total weight
Distance weight is shifted
100
∆CG
=
8,000
120
∆CG

Given:
Aircraft total weight . . . . . . . . . . . . . . . . . . . . . . . . . . . . . . . . . 6,860 lb
CG station. . . . . . . . . . . . . . . . . . . . . . . . . . . . . . . . . . . . . . . . . . 80.0 in
Determine the location of the CG if 140 pounds of baggage is added to
station 150.

= 1.5 in

The change of CG is added to (or subtracted from when appropriate)
the original CG to determine the new CG:
77 + 1.5 = 78.5 inches aft of datum
The shifting weight proportion formula can also be used to determine
how much weight must be shifted to achieve a particular shift of the CG.
The following problem illustrates a solution of this type.

Solution:
Added weight
New total weight

=

∆CG
Distance between weight and old CG

140 lb
6,860 lb + 140 lb

=

∆CG
150 in − 80 in

140 lb
=
7,000 lb

Example

CG =

Given:
Aircraft total weight . . . . . . . . . . . . . . . . . . . . . . . . . . . . . . . . . . 7,800 lb
CG station. . . . . . . . . . . . . . . . . . . . . . . . . . . . . . . . . . . . . . . . . . 81.5 in
Aft CG limit . . . . . . . . . . . . . . . . . . . . . . . . . . . . . . . . . . . . . . . . . 80.5 in
Determine how much cargo must be shifted from the aft cargo compartment at station 150 to the forward cargo compartment at station 30 to
move the CG to exactly the aft limit.
Solution:
Weight to be shifted
CG
=
Total weight
Distance weight is shifted
Weight to be shifted
1.0 in
=
7,800 lb
120 in
Weight to be shifted =

65 lb

A simpler solution may be obtained by using a computer
or calculator and a proportional formula. This can be done
because the CG will shift a distance that is proportional to
the distance the weight is shifted.

Weight Addition or Removal
In many instances, the weight and balance of the aircraft will
be changed by the addition or removal of weight. When this
happens, a new CG must be calculated and checked against
the limitations to see if the location is acceptable. This type
of weight and balance problem is commonly encountered
when the aircraft burns fuel in flight, thereby reducing the
weight located at the fuel tanks. Most small aircraft are
designed with the fuel tanks positioned close to the CG;
therefore, the consumption of fuel does not affect the CG to
any great extent.
The addition or removal of cargo presents a CG change
problem that must be calculated before flight. The problem
may always be solved by calculations involving total
moments. A typical problem may involve the calculation
of a new CG for an aircraft which, when loaded and ready
for flight, receives some additional cargo or passengers just
before departure time.

∆CG
70 in
1.4 in aft

Add ∆CG to old CG
New CG = 80 in + 1.4 in = 81.4 in

Example
Given:
Aircraft total weight . . . . . . . . . . . . . . . . . . . . . . . . . . . . . . . . . 6,100 lb
CG station. . . . . . . . . . . . . . . . . . . . . . . . . . . . . . . . . . . . . . . . . 80.0 in
Determine the location of the CG if 100 pounds is removed from
station 150.
Solution:
Weight removed
New total weight

=

∆CG
Distance between weight and old CG

100 lb
6,100 lb − 100 lb

=

∆CG
150 in − 80 in

100 lb
6,000 lb

=

∆CG
70 in

CG =

1.2 in forward

Subtract ∆CG from old CG
New CG = 80 in − 1.2 in = 78.8 in

In the previous examples, the rCG is either added or
subtracted from the old CG. Deciding which to accomplish is
best handled by mentally calculating which way the CG will
shift for the particular weight change. If the CG is shifting
aft, the rCG is added to the old CG; if the CG is shifting
forward, the rCG is subtracted from the old CG.

Chapter Summary
Operating an aircraft within the weight and balance limits
is critical to flight safety. Pilots must ensure that the CG is
and remains within approved limits throughout all phases of
a flight. For additional information on weight, balance, CG,
and aircraft stability refer to the FAA handbook appropriate
to the specific aircraft category.

10-11

10-12


Chapter 11

Aircraft
Performance
Introduction
This chapter discusses the factors that affect aircraft
performance, which include the aircraft weight, atmospheric
conditions, runway environment, and the fundamental
physical laws governing the forces acting on an aircraft.

Importance of Performance Data
The performance or operational information section of the
Aircraft Flight Manual/Pilot's Operating Handbook (AFM/
POH) contains the operating data for the aircraft; that is, the
data pertaining to takeoff, climb, range, endurance, descent,
and landing. The use of this data in flying operations is
mandatory for safe and efficient operation. Considerable
knowledge and familiarity of the aircraft can be gained by
studying this material.

11-1

It must be emphasized that the manufacturers' information
and data furnished in the AFM/POH is not standardized.
Some provide the data in tabular form, while others use
graphs. In addition, the performance data may be presented
on the basis of standard atmospheric conditions, pressure
altitude, or density altitude. The performance information in
the AFM/POH has little or no value unless the user recognizes
those variations and makes the necessary adjustments.

exerted by the weight of the atmosphere is approximately 14.7
pounds per square inch (psi). The density of air has significant
effects on the aircraft's performance. As air becomes less
dense, it reduces:

To be able to make practical use of the aircraft's capabilities
and limitations, it is essential to understand the significance
of the operational data. The pilot must be cognizant of the
basis for the performance data, as well as the meanings of
the various terms used in expressing performance capabilities
and limitations.

The pressure of the atmosphere may vary with time but more
importantly, it varies with altitude and temperature. Due to
the changing atmospheric pressure, a standard reference
was developed. The standard atmosphere at sea level has
a surface temperature of 59 degrees Fahrenheit (°F) or 15
degrees Celsius (°C) and a surface pressure of 29.92 inches
of mercury ("Hg) or 1013.2 millibars (mb). [Figure 11-1]

Since the characteristics of the atmosphere have a major
effect on performance, it is necessary to review two dominant
factors—pressure and temperature.

Structure of the Atmosphere
The atmosphere is an envelope of air that surrounds the
Earth and rests upon its surface. It is as much a part of the
Earth as is land and water. However, air differs from land
and water in that it is a mixture of gases. It has mass, weight,
and indefinite shape.
Air, like any other fluid, is able to flow and change its shape
when subjected to even minute pressures because of the
lack of strong molecular cohesion. For example, gas will
completely fill any container into which it is placed, expanding
or contracting to adjust its shape to the limits of the container.
The atmosphere is composed of 78 percent nitrogen, 21
percent oxygen, and 1 percent other gases, such as argon
or helium. Most of the oxygen is contained below 35,000
feet altitude.

Atmospheric Pressure
Though there are various kinds of pressure, pilots are mainly
concerned with atmospheric pressure. It is one of the basic
factors in weather changes, helps to lift the aircraft, and
actuates some of the most important flight instruments in the
aircraft. These instruments often include the altimeter, the
airspeed indicator (ASI), the vertical speed indicator (VSI),
and the manifold pressure gauge.
Though air is very light, it has mass and is affected by the
attraction of gravity. Therefore, like any other substance, it
has weight; because it has weight, it has force. Since it is a
fluid substance, this force is exerted equally in all directions,
and its effect on bodies within the air is called pressure.
Under standard conditions at sea level, the average pressure
11-2



Power, because the engine takes in less air



Thrust, because the propeller is less efficient in thin air



Lift, because the thin air exerts less force on the airfoils

A standard temperature lapse rate is one in which the
temperature decreases at the rate of approximately 3.5 °F or
2 °C per thousand feet up to 36,000 feet. Above this point,
the temperature is considered constant up to 80,000 feet. A
standard pressure lapse rate is one in which pressure decreases
at a rate of approximately 1 "Hg per 1,000 feet of altitude gain
to 10,000 feet. [Figure 11-2] The International Civil Aviation
Organization (ICAO) has established this as a worldwide
standard, and it is often referred to as International Standard
Atmosphere (ISA) or ICAO Standard Atmosphere. Any
temperature or pressure that differs from the standard lapse
rates is considered nonstandard temperature and pressure.
Adjustments for nonstandard temperatures and pressures are
provided on the manufacturer's performance charts.

Standard
Sea Level
Pressure

29.92"Hg

Inches of
Mercury

Millibars

30

1016

25

847

20

677

15

508

10

339

5
170
Atmospheric Pressure
0
0

Figure 11-1. Standard sea level pressure.

Standard
Sea Level
Pressure

1013 mb

29.92
28.86
27.82
26.82
25.84
24.89
23.98
23.09
22.22
21.38
20.57
19.79
19.02
18.29
17.57
16.88
16.21
15.56
14.94
14.33
13.74

(°C)

(°F)

15.0
13.0
11.0
9.1
7.1
5.1
3.1
1.1
−0.9
−2.8
−4.8
−6.8
−8.8
−10.8
−12.7
−14.7
−16.7
−18.7
−20.7
−22.6
−24.6

59.0
55.4
51.9
48.3
44.7
41.2
37.6
34.0
30.5
26.9
23.3
19.8
16.2
12.6
9.1
5.5
1.9
−1.6
−5.2
−8.8
−12.3

Figure 11-2. Properties of standard atmosphere.

Since all aircraft performance is compared and evaluated
using the standard atmosphere, all aircraft instruments
are calibrated for the standard atmosphere. Thus, certain
corrections must apply to the instrumentation, as well as the
aircraft performance, if the actual operating conditions do
not fit the standard atmosphere. In order to account properly
for the nonstandard atmosphere, certain related terms must
be defined.

Pressure Altitude
Pressure altitude is the height above the standard datum
plane (SDP). The aircraft altimeter is essentially a sensitive
barometer calibrated to indicate altitude in the standard
atmosphere. If the altimeter is set for 29.92 "Hg SDP, the
altitude indicated is the pressure altitude—the altitude in the
standard atmosphere corresponding to the sensed pressure.
The SDP is a theoretical level at which the pressure of the
atmosphere is 29.92 "Hg and the weight of air is 14.7 psi. As
atmospheric pressure changes, the SDP may be below, at, or
above sea level. Pressure altitude is important as a basis for
determining aircraft performance, as well as for assigning
flight levels to aircraft operating at above 18,000 feet.
The pressure altitude can be determined by any of the three
following methods:
1.	 By setting the barometric scale of the altimeter to
29.92 "Hg and reading the indicated altitude,

2.	 By applying a correction factor to the indicated
altitude according to the reported "altimeter setting,"
[Figure 11-3]
3. 	 By using a flight computer

Density Altitude
The more appropriate term for correlating aerodynamic
performance in the nonstandard atmosphere is density
altitude—the altitude in the standard atmosphere
corresponding to a particular value of air density.
Density altitude is pressure altitude corrected for nonstandard
temperature. As the density of the air increases (lower
density altitude), aircraft performance increases. Conversely,
Method for Determining
Pressure Altitude
Altimeter
Altitude
setting
correction
1,824
28.0
1,727
28.1
1,630
28.2
1,533
28.3
1,436
28.4
1,340
28.5
1,244
28.6
1,148
28.7
1,053
28.8
957
28.9
863
29.0
768
29.1
673
29.2
579
29.3
485
29.4
392
29.5
298
29.6
205
29.7
112
29.8
20
29.9
0
29.92
−73
30.0
−165
30.1
−257
30.2
−348
30.3
−440
30.4
−531
30.5
−622
30.6
−712
30.7
−803
30.8
−893
30.9
−983
31.0
Add

0
1,000
2,000
3,000
4,000
5,000
6,000
7,000
8,000
9,000
10,000
11,000
12,000
13,000
14,000
15,000
16,000
17,000
18,000
19,000
20,000

Temperature

Subtract

Pressure
("Hg)

Field elevation is sea level

Altitude (ft)

Alternate Method for Determining
Pressure Altitude

To field
elevation

To get
pressure altitude

From field
elevation

Figure 11-3. Field elevation versus pressure. The aircraft is located

on a field that happens to be at sea level. Set the altimeter to the
current altimeter setting (29.7). The difference of 205 feet is added
to the elevation or a PA of 205 feet.

11-3

flight level. Density altitude can also be determined by referring
to the table and chart in Figures 11-3 and 11-4 respectively.
Effects of Pressure on Density
Since air is a gas, it can be compressed or expanded. When
air is compressed, a greater amount of air can occupy a
given volume. Conversely, when pressure on a given volume
of air is decreased, the air expands and occupies a greater
space. That is, the original column of air at a lower pressure

Density altitude is computed using pressure altitude and
temperature. Since aircraft performance data at any level is
based upon air density under standard day conditions, such
performance data apply to air density levels that may not be
identical to altimeter indications. Under conditions higher
or lower than standard, these levels cannot be determined
directly from the altimeter.

14
,00
0

15,000
14,000

13
,00
0

11-4

et
(fe
e
ud

,00
0

re

al

tit

11

su

,00

00

Pr

es

10
00

9,0
00

7,0
6,0

00

7,000

00

6,000
5,0

5,000
4,0

00

Density altitude (feet)

8,000

ture

Using a flight computer, density altitude can be computed by
inputting the pressure altitude and outside air temperature at

9,000

mpera

3,0

00

4,000

2,0

00

3,000

00

2,000

ve
-1,

Sea level

C -20°
F

-2,
0

00

0

00

Se

a

le

1,000

l

1,0

Air density is affected by changes in altitude, temperature,
and humidity. High density altitude refers to thin air while
low density altitude refers to dense air. The conditions that
result in a high density altitude are high elevations, low
atmospheric pressures, high temperatures, high humidity, or
some combination of these factors. Lower elevations, high
atmospheric pressure, low temperatures, and low humidity
are more indicative of low density altitude.

10,000

ard te

For example, when set at 29.92 "Hg, the altimeter may
indicate a pressure altitude of 5,000 feet. According to the
AFM/POH, the ground run on takeoff may require a distance
of 790 feet under standard temperature conditions. However,
if the temperature is 20 °C above standard, the expansion of
air raises the density level. Using temperature correction data
from tables or graphs, or by deriving the density altitude with
a computer, it may be found that the density level is above
7,000 feet, and the ground run may be closer to 1,000 feet.

11,000

Stand

Density altitude is determined by first finding pressure
altitude and then correcting this altitude for nonstandard
temperature variations. Since density varies directly with
pressure, and inversely with temperature, a given pressure
altitude may exist for a wide range of temperature by allowing
the density to vary. However, a known density occurs for
any one temperature and pressure altitude. The density of the
air, of course, has a pronounced effect on aircraft and engine
performance. Regardless of the actual altitude at which
the aircraft is operating, it will perform as though it were
operating at an altitude equal to the existing density altitude.

0

12,000

)

12
,00
0

13,000

8,0

as air density decreases (higher density altitude), aircraft
performance decreases. A decrease in air density means a
high density altitude; an increase in air density means a lower
density altitude. Density altitude is used in calculating aircraft
performance. Under standard atmospheric condition, air at
each level in the atmosphere has a specific density; under
standard conditions, pressure altitude and density altitude
identify the same level. Density altitude, then, is the vertical
distance above sea level in the standard atmosphere at which
a given density is to be found.

-10°

0°

10°

20°

30°

40°

0° 10° 20° 30° 40° 50° 60° 70° 80° 90° 100°

Outside air temperature (OAT)
Figure 11-4. Density altitude chart.

contains a smaller mass of air. In other words, the density is
decreased. In fact, density is directly proportional to pressure.
If the pressure is doubled, the density is doubled, and if the
pressure is lowered, so is the density. This statement is true
only at a constant temperature.
Effects of Temperature on Density
Increasing the temperature of a substance decreases its
density. Conversely, decreasing the temperature increases
the density. Thus, the density of air varies inversely with
temperature. This statement is true only at a constant pressure.
In the atmosphere, both temperature and pressure decrease
with altitude and have conflicting effects upon density.
However, the fairly rapid drop in pressure as altitude is
increased usually has the dominant effect. Hence, pilots can
expect the density to decrease with altitude.
Effects of Humidity (Moisture) on Density
The preceding paragraphs are based on the presupposition of
perfectly dry air. In reality, it is never completely dry. The
small amount of water vapor suspended in the atmosphere
may be negligible under certain conditions, but in other
conditions humidity may become an important factor in the
performance of an aircraft. Water vapor is lighter than air;
consequently, moist air is lighter than dry air. Therefore, as the
water content of the air increases, the air becomes less dense,
increasing density altitude and decreasing performance. It is
lightest or least dense when, in a given set of conditions, it
contains the maximum amount of water vapor.
Humidity, also called relative humidity, refers to the amount
of water vapor contained in the atmosphere and is expressed
as a percentage of the maximum amount of water vapor
the air can hold. This amount varies with the temperature;
warm air can hold more water vapor, while colder air can
hold less. Perfectly dry air that contains no water vapor has
a relative humidity of zero percent, while saturated air that
cannot hold any more water vapor has a relative humidity
of 100 percent. Humidity alone is usually not considered an
essential factor in calculating density altitude and aircraft
performance; however, it does contribute.
The higher the temperature, the greater amount of water
vapor that the air can hold. When comparing two separate air
masses, the first warm and moist (both qualities making air
lighter) and the second cold and dry (both qualities making
it heavier), the first must be less dense than the second.
Pressure, temperature, and humidity have a great influence
on aircraft performance because of their effect upon density.
There is no rule-of-thumb or chart used to compute the effects
of humidity on density altitude, but it must be taken into

consideration. Expect a decrease in overall performance in
high humidity conditions.

Performance
Performance is a term used to describe the ability of an
aircraft to accomplish certain things that make it useful for
certain purposes. For example, the ability of an aircraft to land
and take off in a very short distance is an important factor
to the pilot who operates in and out of short, unimproved
airfields. The ability to carry heavy loads, fly at high altitudes
at fast speeds, and/or travel long distances is essential for the
performance of airline and executive type aircraft.
The primary factors most affected by performance are the
takeoff and landing distance, rate of climb, ceiling, payload,
range, speed, maneuverability, stability, and fuel economy.
Some of these factors are often directly opposed: for example,
high speed versus short landing distance, long range versus
great payload, and high rate of climb versus fuel economy.
It is the preeminence of one or more of these factors that
dictates differences between aircraft and explains the high
degree of specialization found in modern aircraft.
The various items of aircraft performance result from the
combination of aircraft and powerplant characteristics. The
aerodynamic characteristics of the aircraft generally define
the power and thrust requirements at various conditions of
flight, while powerplant characteristics generally define the
power and thrust available at various conditions of flight.
The matching of the aerodynamic configuration with the
powerplant is accomplished by the manufacturer to provide
maximum performance at the specific design condition (e.g.,
range, endurance, and climb).
Straight-and-Level Flight
All of the principal components of flight performance involve
steady-state flight conditions and equilibrium of the aircraft.
For the aircraft to remain in steady, level flight, equilibrium
must be obtained by a lift equal to the aircraft weight and a
powerplant thrust equal to the aircraft drag. Thus, the aircraft
drag defines the thrust required to maintain steady, level
flight. As presented in Chapter 4, Aerodynamics of Flight,
all parts of an aircraft contribute to the drag, either induced
(from lifting surfaces) or parasite drag.
While parasite drag predominates at high speed, induced drag
predominates at low speed. [Figure 11-5] For example, if
an aircraft in a steady flight condition at 100 knots is then
accelerated to 200 knots, the parasite drag becomes four
times as great, but the power required to overcome that
drag is eight times the original value. Conversely, when the

11-5

energy comes in two forms: (1) Kinetic Energy (KE), the
energy of speed; (2) Potential Energy (PE), the stored energy
of position.

Drag

Total drag

L/DMAX

Parasite drag

Stall
Induced drag
Speed

Aircraft motion (KE) is described by its velocity (airspeed).
Aircraft position (PE) is described by its height (altitude).
Both KE and PE are directly proportional to the object's
mass. KE is directly proportional to the square of the object's
velocity (airspeed). PE is directly proportional to the object's
height (altitude). The formulas below summarize these
energy relationships:

Figure 11-5. Drag versus speed.

KE = ½ × m × v2

m = object mass
v = object velocity

aircraft is operated in steady, level flight at twice as great a
speed, the induced drag is one-fourth the original value, and
the power required to overcome that drag is only one-half
the original value.

PE = m × g × h

m = object mass
g = gravity field strength
h = object height

When an aircraft is in steady, level flight, the condition of
equilibrium must prevail. The unaccelerated condition of
flight is achieved with the aircraft trimmed for lift equal
to weight and the powerplant set for a thrust to equal the
aircraft drag.
The maximum level flight speed for the aircraft is obtained
when the power or thrust required equals the maximum power
or thrust available from the powerplant. [Figure 11-6] The
minimum level flight airspeed is not usually defined by thrust
or power requirement since conditions of stall or stability and
control problems generally predominate.
Climb Performance
If an aircraft is to move, fly, and perform, work must act
upon it. Work involves force moving the aircraft. The aircraft
acquires mechanical energy when it moves. Mechanical

High cruise speed

Low cruise speed
Min. speed

Speed
Figure 11-6. Power versus speed.

11-6

Maximum level flight speed

Power required

m available po
Maximu
wer

We sometimes use the terms "power" and "thrust"
interchangeably when discussing climb performance. This
erroneously implies the terms are synonymous. It is important
to distinguish between these terms. Thrust is a force or
pressure exerted on an object. Thrust is measured in pounds
(lb) or newtons (N). Power, however, is a measurement of
the rate of performing work or transferring energy (KE and
PE). Power is typically measured in horsepower (hp) or
kilowatts (kw). We can think of power as the motion (KE
and PE) a force (thrust) creates when exerted on an object
over a period of time.
Positive climb performance occurs when an aircraft gains PE
by increasing altitude. Two basic factors, or a combination
of the two factors, contribute to positive climb performance
in most aircraft:
1.	 The aircraft climbs (gains PE) using excess power
above that required to maintain level flight, or
2. 	 The aircraft climbs by converting airspeed (KE) to
altitude (PE).
As an example of factor 1 above, an aircraft with an engine
capable of producing 200 horsepower (at a given altitude)
is using only 130 horsepower to maintain level flight at that
altitude. This leaves 70 horsepower available to climb. The
pilot holds airspeed constant and increases power to perform
the climb.
As an example of factor 2, an aircraft is flying level at 120
knots. The pilot leaves the engine power setting constant but
applies other control inputs to perform a climb. The climb,
sometimes called a zoom climb, converts the airspeed (KE)

to altitude (PE); the airspeed decreases to something less
than 120 knots as the altitude increases.
There are two primary reasons to evaluate climb performance.
First, aircraft must climb over obstacles to avoid hitting
them. Second, climbing to higher altitudes can provide
better weather, fuel economy, and other benefits. Maximum
Angle of Climb (AOC), obtained at VX, may provide climb
performance to ensure an aircraft will clear obstacles.
Maximum Rate of Climb (ROC), obtained at VY, provides
climb performance to achieve the greatest altitude gain over
time. Maximum ROC may not be sufficient to avoid obstacles
in some situations, while maximum AOC may be sufficient
to avoid the same obstacles. [Figure 11-7]

Angle of Climb (AOC)
AOC is a comparison of altitude gained relative to distance
traveled. AOC is the inclination (angle) of the flight path. For
maximum AOC performance, a pilot flies the aircraft at VX

so as to achieve maximum altitude increase with minimum
horizontal travel over the ground. A good use of maximum
AOC is when taking off from a short airfield surrounded by
high obstacles, such as trees or power lines. The objective is
to gain sufficient altitude to clear the obstacle while traveling
the least horizontal distance over the surface.
One method to climb (have positive AOC performance) is
to have excess thrust available. Essentially, the greater the
force that pushes the aircraft upward, the steeper it can climb.
Maximum AOC occurs at the airspeed and angle of attack
(AOA) combination which allows the maximum excess
thrust. The airspeed and AOA combination where excess
thrust exists varies amongst aircraft types. As an example,
Figure 11-8 provides a comparison between jet and propeller
airplanes as to where maximum excess thrust (for maximum
AOC) occurs. In a jet, maximum excess thrust normally
occurs at the airspeed where the thrust required is at a
minimum (approximately L/DMAX). In a propeller airplane,
maximum excess thrust normally occurs at an airspeed below
L/DMAX and frequently just above stall speed.

Altitude

Rate of Climb (ROC)
ROC is a comparison of altitude gained relative to the time
needed to reach that altitude. ROC is simply the vertical
component of the aircraft's flight path velocity vector. For
maximum ROC performance, a pilot flies the aircraft at VY
so as to achieve a maximum gain in altitude over a given
period of time.

RO

M

ax

AO

C

C

x
Ma

Distance
Figure 11-7. Maximum angle of climb (AOC) versus maximum rate
of climb (ROC).

Maximum ROC expedites a climb to an assigned altitude.
This gains the greatest vertical distance over a period of
time. For example, in a maximum AOC profile, a certain
aircraft takes 30 seconds to reach 1,000 feet AGL, but
covers only 3,000 feet over the ground. By comparison,
using its maximum ROC profile, the same aircraft climbs

LEGEND

TR
AOC
TAS
L/D MAX

PCL

(Full PCL)

TA

(Full Throttle)

TE
Max AOC (jet)

TR

Thrust

TA

thrust
excess
thrust
available
thrust
required
angle of
climb
true
airspeed
lift to drag
ratio
maximum
power
control lever

Thrust

TE

L/D MAX

TA
TE

TR

L/D MAX
Max AOC (prop)

Velocity (TAS)

Velocity (TAS)

Figure 11-8. Comparison of maximum AOC between jet and propeller airplanes.

11-7

to 1,500 feet in 30 seconds but covers 6,000 feet across the
ground. Note that both ROC and AOC maximum climb
profiles use the aircraft's maximum throttle setting. Any
differences between max ROC and max AOC lie primarily
in the velocity (airspeed) and AOA combination the aircraft
manual specifies. [Figure 11-7]
ROC performance depends upon excess power. Since
climbing is work and power is the rate of performing work,
a pilot can increase the climb rate by using any power not
used to maintain level flight. Maximum ROC occurs at an
airspeed and AOA combination that produces the maximum
excess power. Therefore, maximum ROC for a typical jet
airplane occurs at an airspeed greater than L/DMAX and at an
AOA less than L/DMAX AOA. In contrast, maximum ROC for
a typical propeller airplane occurs at an airspeed and AOA
combination closer to L/DMAX. [Figure 11-9]

Climb Performance Factors
Since weight, altitude and configuration changes affect
excess thrust and power, they also affect climb performance.
Climb performance is directly dependent upon the ability to
produce either excess thrust or excess power. Earlier in the
book it was shown that an increase in weight, an increase in
altitude, lowering the landing gear, or lowering the flaps all
decrease both excess thrust and excess power for all aircraft.
Therefore, maximum AOC and maximum ROC performance
decreases under any of these conditions.
Weight has a very pronounced effect on aircraft performance.
If weight is added to an aircraft, it must fly at a higher AOA
to maintain a given altitude and speed. This increases the
induced drag of the wings, as well as the parasite drag of the
aircraft. Increased drag means that additional thrust is needed
to overcome it, which in turn means that less reserve thrust is
available for climbing. Aircraft designers go to great lengths

to minimize the weight, since it has such a marked effect on
the factors pertaining to performance.
A change in an aircraft's weight produces a twofold effect
on climb performance. First, a change in weight changes the
drag and the power required. This alters the reserve power
available, which in turn, affects both the climb angle and
the climb rate. Secondly, an increase in weight reduces the
maximum ROC, but the aircraft must be operated at a higher
climb speed to achieve the smaller peak climb rate.
An increase in altitude also increases the power required
and decreases the power available. Therefore, the climb
performance of an aircraft diminishes with altitude. The
speeds for maximum ROC, maximum AOC, and maximum
and minimum level flight airspeeds vary with altitude. As
altitude is increased, these various speeds finally converge
at the absolute ceiling of the aircraft. At the absolute ceiling,
there is no excess of power and only one speed allows steady,
level flight. Consequently, the absolute ceiling of an aircraft
produces zero ROC. The service ceiling is the altitude at
which the aircraft is unable to climb at a rate greater than 100
feet per minute (fpm). Usually, these specific performance
reference points are provided for the aircraft at a specific
design configuration. [Figure 11-10]
The terms "power loading," "wing loading," "blade loading,"
and "disk loading" are commonly used in reference to
performance. Power loading is expressed in pounds per
horsepower and is obtained by dividing the total weight
of the aircraft by the rated horsepower of the engine. It
is a significant factor in an aircraft's takeoff and climb
capabilities. Wing loading is expressed in pounds per square
foot and is obtained by dividing the total weight of an airplane
in pounds by the wing area (including ailerons) in square feet.
It is the airplane's wing loading that determines the landing

LEGEND

PR
ROC
TAS
L/D MAX

PCL

(Full PCL)

(Full Throttle)

P
A

L/D MAX

PE

P
R

Power

PA

power
excess
power
available
power
required
rate of
climb
true
airspeed
lift to drag
ratio
maximum
power
control lever

Power

PE

Max ROC (jet)

Velocity (TAS)

Figure 11-9. Comparison of maximum ROC between jet and propeller airplanes.

11-8

P
A
PE

L/D MAX

PR
Max ROC (prop)

Velocity (TAS)

A common element for each of these operating problems
is the specific range; that is, nautical miles (NM) of flying
distance versus the amount of fuel consumed. Range must
be clearly distinguished from the item of endurance. Range
involves consideration of flying distance, while endurance
involves consideration of flying time. Thus, it is appropriate
to define a separate term, specific endurance.

24,000
22,000
Standard altitude (feet)

20,000
18,000

Absolute ceiling

16,000

Service ceiling

14,000

specific endurance =

12,000
10,000
8,000

Best angle
of climb (Vx)

Best rate
of climb (Vy)

or

6,000

specific endurance =

4,000
2,000
Sea level

70

80

90

100

110

flight hours/hour
pounds of fuel/hour
or

120

Indicated airspeed (knots)

specific endurance =

Figure 11-10. Absolute and service ceiling.

1
fuel flow

Fuel flow can be defined in either pounds or gallons. If
maximum endurance is desired, the flight condition must
provide a minimum fuel flow. In Figure 11-11 at point A,
the airspeed is low and fuel flow is high. This would occur
during ground operations or when taking off and climbing.
As airspeed is increased, power requirements decrease due
to aerodynamic factors, and fuel flow decreases to point B.
This is the point of maximum endurance. Beyond this point,
increases in airspeed come at a cost. Airspeed increases
require additional power and fuel flow increases with
additional power.

speed. Blade loading is expressed in pounds per square foot
and is obtained by dividing the total weight of a helicopter by
the area of the rotor blades. Blade loading is not to be confused
with disk loading, which is the total weight of a helicopter
divided by the area of the disk swept by the rotor blades.
Range Performance
The ability of an aircraft to convert fuel energy into flying
distance is one of the most important items of aircraft
performance. In flying operations, the problem of efficient
range operation of an aircraft appears in two general forms:
1. 	 To extract the maximum flying distance from a given
fuel load

Cruise flight operations for maximum range should be
conducted so that the aircraft obtains maximum specific range
throughout the flight. The specific range can be defined by
the following relationship.

At a

ltitu
de

2.	 To fly a specified distance with a minimum
expenditure of fuel

Fuel flow/power required (HP)

flight hours
pounds of fuel

nce

Maximum endurance at
minimum power required

ere
Ref

line

Applicable for a particular
Weight
Altitude
Configuration

A
B

Maximum range at L/DMAX
Speed

Figure 11-11. Airspeed for maximum endurance.

11-9

specific range =

NM/hour
pounds of fuel/hour
or

specific range =

knots
fuel flow

If maximum specific range is desired, the flight condition
must provide a maximum of speed per fuel flow. While
the peak value of specific range would provide maximum
range operation, long-range cruise operation is generally
recommended at a slightly higher airspeed. Most long-range
cruise operations are conducted at the flight condition that
provides 99 percent of the absolute maximum specific range.
The advantage of such operation is that one percent of range
is traded for three to five percent higher cruise speed. Since
the higher cruise speed has a great number of advantages, the
small sacrifice of range is a fair bargain. The values of specific
range versus speed are affected by three principal variables:
1.

Aircraft gross weight

2.

Altitude

3.

The external aerodynamic configuration of the aircraft.

These are the source of range and endurance operating data
included in the performance section of the AFM/POH.
Cruise control of an aircraft implies that the aircraft is
operated to maintain the recommended long-range cruise
condition throughout the flight. Since fuel is consumed during
cruise, the gross weight of the aircraft varies and optimum
airspeed, altitude, and power setting can also vary. Cruise
control means the control of the optimum airspeed, altitude,
and power setting to maintain the 99 percent maximum
specific range condition. At the beginning of cruise flight, the
relatively high initial weight of the aircraft requires specific
values of airspeed, altitude, and power setting to produce the
recommended cruise condition. As fuel is consumed and the
aircraft's gross weight decreases, the optimum airspeed and
power setting may decrease, or the optimum altitude may
increase. In addition, the optimum specific range increases.
Therefore, the pilot must provide the proper cruise control
procedure to ensure that optimum conditions are maintained.
Total range is dependent on both fuel available and specific
range. When range and economy of operation are the principal
goals, the pilot must ensure that the aircraft is operated at the
11-10

A propeller-driven aircraft combines the propeller with the
reciprocating engine for propulsive power. Fuel flow is
determined mainly by the shaft power put into the propeller
rather than thrust. Thus, the fuel flow can be related directly
to the power required to maintain the aircraft in steady, level
flight, and on performance charts power can be substituted
for fuel flow. This fact allows for the determination of range
through analysis of power required versus speed.
The maximum endurance condition would be obtained at the
point of minimum power required since this would require the
lowest fuel flow to keep the airplane in steady, level flight.
Maximum range condition would occur where the ratio of
speed to power required is greatest. [Figure 11-11]
The maximum range condition is obtained at maximum lift/
drag ratio (L/DMAX), and it is important to note that for a given
aircraft configuration, the L/DMAX occurs at a particular AOA
and lift coefficient and is unaffected by weight or altitude. A
variation in weight alters the values of airspeed and power
required to obtain the L/DMAX. [Figure 11-12] Different
theories exist on how to achieve max range when there is a
headwind or tailwind present. Many say that speeding up in
a headwind or slowing down in a tail wind helps to achieve
max range. While this theory may be true in a lot of cases,
it is not always true as there are different variables to every
situation. Each aircraft configuration is different, and there
is not a rule of thumb that encompasses all of them as to how
to achieve the max range.

t

or

recommended long-range cruise condition. By this procedure,
the aircraft is capable of its maximum design-operating radius
or can achieve flight distances less than the maximum with
a maximum of fuel reserve at the destination.

L/DMAX

Hi
Lo B
g
we as
r w i c w h er
w
eig ei
ht gh eigh
t

NM
pounds of fuel

Power required

specific range =

Constant altitude

Speed
Figure 11-12. Effect of weight.

The variations of speed and power required must be
monitored by the pilot as part of the cruise control procedure
to maintain the L/DMAX. When the aircraft's fuel weight is a
small part of the gross weight and the aircraft's range is small,
the cruise control procedure can be simplified to essentially
maintaining a constant speed and power setting throughout
the time of cruise flight. However, a long-range aircraft has a
fuel weight that is a considerable part of the gross weight, and
cruise control procedures must employ scheduled airspeed
and power changes to maintain optimum range conditions.
The effect of altitude on the range of a propeller-driven
aircraft is illustrated in Figure 11-13. A flight conducted at
high altitude has a greater true airspeed (TAS), and the power
required is proportionately greater than when conducted at
sea level. The drag of the aircraft at altitude is the same as the
drag at sea level, but the higher TAS causes a proportionately
greater power required.
NOTE: The straight line that is tangent to the sea level power
curve is also tangent to the altitude power curve.
The effect of altitude on specific range can also be appreciated
from the previous relationships. If a change in altitude causes
identical changes in speed and power required, the proportion
of speed to power required would be unchanged. The fact
implies that the specific range of a propeller-driven aircraft
would be unaffected by altitude. Actually, this is true to the
extent that specific fuel consumption and propeller efficiency
are the principal factors that could cause a variation of
specific range with altitude. If compressibility effects are
negligible, any variation of specific range with altitude is
strictly a function of engine/propeller performance.

Power required

L/DMAX
Speed
Figure 11-13. Effect of altitude on range.

ltit
ud
e
At
a

Se
a le

vel

An aircraft equipped with a reciprocating engine experiences
very little, if any, variation of specific range up to its
absolute altitude. There is negligible variation of brake

Constant weight

specific fuel consumption for values of brake horsepower
below the maximum cruise power rating of the engine that
is the lean range of engine operation. Thus, an increase in
altitude produces a decrease in specific range only when the
increased power requirement exceeds the maximum cruise
power rating of the engine. One advantage of supercharging
is that the cruise power may be maintained at high altitude,
and the aircraft may achieve the range at high altitude with
the corresponding increase in TAS. The principal differences
in the high altitude cruise and low altitude cruise are the TAS
and climb fuel requirements.
Region of Reversed Command
The aerodynamic properties of an aircraft generally determine
the power requirements at various conditions of flight, while
the powerplant capabilities generally determine the power
available at various conditions of flight. When an aircraft
is in steady, level flight, a condition of equilibrium must
prevail. An unaccelerated condition of flight is achieved
when lift equals weight, and the powerplant is set for thrust
equal to drag. The power required to achieve equilibrium in
constant-altitude flight at various airspeeds is depicted on a
power required curve. The power required curve illustrates
the fact that at low airspeeds near the stall or minimum
controllable airspeed, the power setting required for steady,
level flight is quite high.
Flight in the region of normal command means that while
holding a constant altitude, a higher airspeed requires a higher
power setting and a lower airspeed requires a lower power
setting. The majority of aircraft flying (climb, cruise, and
maneuvers) is conducted in the region of normal command.
Flight in the region of reversed command means flight in
which a higher airspeed requires a lower power setting
and a lower airspeed requires a higher power setting to
hold altitude. It does not imply that a decrease in power
produces lower airspeed. The region of reversed command is
encountered in the low speed phases of flight. Flight speeds
below the speed for maximum endurance (lowest point
on the power curve) require higher power settings with a
decrease in airspeed. Since the need to increase the required
power setting with decreased speed is contrary to the normal
command of flight, the regime of flight speeds between the
speed for minimum required power setting and the stall speed
(or minimum control speed) is termed the region of reversed
command. In the region of reversed command, a decrease in
airspeed must be accompanied by an increased power setting
in order to maintain steady flight.
Figure 11-14 shows the maximum power available as a
curved line. Lower power settings, such as cruise power,
would also appear in a similar curve. The lowest point on
11-11

Region of
reversed
command

able

Excess power

re
qu
ire
d

Power setting

Maximum powe
r avail

w
Po

er

Takeoff and landing performance is a condition of
accelerated and decelerated motion. For instance, during
takeoff an aircraft starts at zero speed and accelerates to
the takeoff speed to become airborne. During landing, the
aircraft touches down at the landing speed and decelerates
to zero speed. The important factors of takeoff or landing
performance are:


The takeoff or landing speed is generally a function
of the stall speed or minimum flying speed.



The rate of acceleration/deceleration during the
takeoff or landing roll. The speed (acceleration and
deceleration) experienced by any object varies directly
with the imbalance of force and inversely with the
mass of the object. An airplane on the runway moving
at 75 knots has four times the energy it has traveling
at 37 knots. Thus, an airplane requires four times as
much distance to stop as required at half the speed.



The takeoff or landing roll distance is a function of
both acceleration/deceleration and speed.

Best endurance speed
Speed
Figure 11-14. Power required curve.

the power required curve represents the speed at which the
lowest brake horsepower sustains level flight. This is termed
the best endurance airspeed.
An airplane performing a low airspeed, high pitch attitude
power approach for a short-field landing is an example
of operating in the region of reversed command. If an
unacceptably high sink rate should develop, it may be
possible for the pilot to reduce or stop the descent by applying
power. But without further use of power, the airplane would
probably stall or be incapable of flaring for the landing.
Merely lowering the nose of the airplane to regain flying
speed in this situation, without the use of power, would
result in a rapid sink rate and corresponding loss of altitude.

Runway Surface and Gradient
Runway conditions affect takeoff and landing performance.
Typically, performance chart information assumes paved,
level, smooth, and dry runway surfaces. Since no two
runways are alike, the runway surface differs from one
runway to another, as does the runway gradient or slope.
[Figure 11-15]

Takeoff and Landing Performance

Runway surfaces vary widely from one airport to another.
The runway surface encountered may be concrete, asphalt,
gravel, dirt, or grass. The runway surface for a specific
airport is noted in the Chart Supplement U.S. (formerly
Airport/Facility Directory). Any surface that is not hard
and smooth increases the ground roll during takeoff. This
is due to the inability of the tires to roll smoothly along the
runway. Tires can sink into soft, grassy, or muddy runways.
Potholes or other ruts in the pavement can be the cause of
poor tire movement along the runway. Obstructions such
as mud, snow, or standing water reduce the airplane's
acceleration down the runway. Although muddy and wet
surface conditions can reduce friction between the runway
and the tires, they can also act as obstructions and reduce
the landing distance. [Figure 11-16] Braking effectiveness
is another consideration when dealing with various runway
types. The condition of the surface affects the braking ability
of the aircraft.

The majority of pilot-caused aircraft accidents occur during
the takeoff and landing phase of flight. Because of this fact,
the pilot must be familiar with all the variables that influence
the takeoff and landing performance of an aircraft and must
strive for exacting, professional procedures of operation
during these phases of flight.

The amount of power that is applied to the brakes without
skidding the tires is referred to as braking effectiveness.
Ensure that runways are adequate in length for takeoff
acceleration and landing deceleration when less than ideal
surface conditions are being reported.

If during a soft-field takeoff and climb, for example, the pilot
attempts to climb out of ground effect without first attaining
normal climb pitch attitude and airspeed, the airplane may
inadvertently enter the region of reversed command at a
dangerously low altitude. Even with full power, the airplane
may be incapable of climbing or even maintaining altitude.
The pilot's only recourse in this situation is to lower the pitch
attitude in order to increase airspeed, which inevitably results
in a loss of altitude.
Airplane pilots must give particular attention to precise
control of airspeed when operating in the low flight speeds
of the region of reversed command.

11-12

Figure 11-15. Takeoff distance chart.

The gradient or slope of the runway is the amount of change
in runway height over the length of the runway. The gradient
is expressed as a percentage, such as a 3 percent gradient. This
means that for every 100 feet of runway length, the runway
height changes by 3 feet. A positive gradient indicates the
runway height increases, and a negative gradient indicates the
runway decreases in height. An upsloping runway impedes
acceleration and results in a longer ground run during takeoff.
However, landing on an upsloping runway typically reduces
the landing roll. A downsloping runway aids in acceleration
on takeoff resulting in shorter takeoff distances. The opposite
is true when landing, as landing on a downsloping runway
increases landing distances. Runway slope information is
contained in the Chart Supplement U.S. (formerly Airport/
Facility Directory). [Figure 11-17]

Water on the Runway and Dynamic Hydroplaning
Water on the runways reduces the friction between the tires
and the ground and can reduce braking effectiveness. The
ability to brake can be completely lost when the tires are
hydroplaning because a layer of water separates the tires from
the runway surface. This is also true of braking effectiveness
when runways are covered in ice.
When the runway is wet, the pilot may be confronted with
dynamic hydroplaning. Dynamic hydroplaning is a condition
in which the aircraft tires ride on a thin sheet of water rather
than on the runway's surface. Because hydroplaning wheels
are not touching the runway, braking and directional control
are almost nil. To help minimize dynamic hydroplaning,
some runways are grooved to help drain off water; most
runways are not.

Figure 11-16. An aircraft's performance during takeoff depends greatly on the runway surface.

11-13

Runway surface
Airport name
Runway slope and direction of slope

Runway

Figure 11-17. Chart Supplement U.S. (formerly Airport/Facility Directory) information.

Tire pressure is a factor in dynamic hydroplaning. Using
the simple formula in Figure 11-18, a pilot can calculate
the minimum speed, in knots, at which hydroplaning begins.
In plain language, the minimum hydroplaning speed is
determined by multiplying the square root of the main gear
tire pressure in psi by nine. For example, if the main gear tire
pressure is at 36 psi, the aircraft would begin hydroplaning
at 54 knots.
Landing at higher than recommended touchdown speeds
exposes the aircraft to a greater potential for hydroplaning.
And once hydroplaning starts, it can continue well below the
minimum initial hydroplaning speed.
On wet runways, directional control can be maximized
by landing into the wind. Abrupt control inputs should be
avoided. When the runway is wet, anticipate braking problems

well before landing and be prepared for hydroplaning. Opt for
a suitable runway most aligned with the wind. Mechanical
braking may be ineffective, so aerodynamic braking should
be used to its fullest advantage.
Takeoff Performance
The minimum takeoff distance is of primary interest in
the operation of any aircraft because it defines the runway
requirements. The minimum takeoff distance is obtained by
taking off at some minimum safe speed that allows sufficient
margin above stall and provides satisfactory control and
initial ROC. Generally, the lift-off speed is some fixed
percentage of the stall speed or minimum control speed for
the aircraft in the takeoff configuration. As such, the lift-off is
accomplished at some particular value of lift coefficient and
AOA. Depending on the aircraft characteristics, the lift-off
speed is anywhere from 1.05 to 1.25 times the stall speed or
minimum control speed.

Minimum dynamic hydroplaning speed (rounded off) =

9x

Tire pressure (in psi)

36 = 6
9 x 6 = 54 knots
Figure 11-18. Tire pressure.

11-14

To obtain minimum takeoff distance at the specific lift-off
speed, the forces that act on the aircraft must provide the
maximum acceleration during the takeoff roll. The various
forces acting on the aircraft may or may not be under the
control of the pilot, and various procedures may be necessary
in certain aircraft to maintain takeoff acceleration at the
highest value.
The powerplant thrust is the principal force to provide the
acceleration and, for minimum takeoff distance, the output
thrust should be at a maximum. Lift and drag are produced
as soon as the aircraft has speed, and the values of lift and
drag depend on the AOA and dynamic pressure.

Greater mass to accelerate

3.

Increased retarding force (drag and ground friction)

If the gross weight increases, a greater speed is necessary to
produce the greater lift necessary to get the aircraft airborne
at the takeoff lift coefficient. As an example of the effect of
a change in gross weight, a 21 percent increase in takeoff
weight requires a 10 percent increase in lift-off speed to
support the greater weight.

The effect of proper takeoff speed is especially important
when runway lengths and takeoff distances are critical. The
takeoff speeds specified in the AFM/POH are generally
the minimum safe speeds at which the aircraft can become
airborne. Any attempt to take off below the recommended
speed means that the aircraft could stall, be difficult to
control, or have a very low initial ROC. In some cases, an

A change in gross weight changes the net accelerating force
and changes the mass that is being accelerated. If the aircraft
has a relatively high thrust-to-weight ratio, the change in the
net accelerating force is slight and the principal effect on
acceleration is due to the change in mass.
For example, a 10 percent increase in takeoff gross weight
would cause:


A 5 percent increase in takeoff velocity



At least a 9 percent decrease in rate of acceleration



At least a 21 percent increase in takeoff distance

With ISA conditions, increasing the takeoff weight of the
average Cessna 182 from 2,400 pounds to 2,700 pounds (11
percent increase) results in an increased takeoff distance from
440 feet to 575 feet (23 percent increase).

e

80

lin

2.

70
Percent increase
in takeoff or
landing distance

ce

Higher lift-off speed

The effect of wind on landing distance is identical to its
effect on takeoff distance. Figure 11-19 illustrates the general
effect of wind by the percent change in takeoff or landing
distance as a function of the ratio of wind velocity to takeoff
or landing speed.

en

1.

A headwind that is 10 percent of the takeoff airspeed reduces
the takeoff distance approximately 19 percent. However, a
tailwind that is 10 percent of the takeoff airspeed increases
the takeoff distance approximately 21 percent. In the case
where the headwind speed is 50 percent of the takeoff speed,
the takeoff distance would be approximately 25 percent of
the zero wind takeoff distance (75 percent reduction).

60

er

For example, the effect of gross weight on takeoff distance
is significant, and proper consideration of this item must be
made in predicting the aircraft's takeoff distance. Increased
gross weight can be considered to produce a threefold effect
on takeoff performance:

The effect of wind on takeoff distance is large, and proper
consideration must also be provided when predicting takeoff
distance. The effect of a headwind is to allow the aircraft to
reach the lift-off speed at a lower groundspeed, while the
effect of a tailwind is to require the aircraft to achieve a
greater groundspeed to attain the lift-off speed.

ef

In addition to the important factors of proper procedures,
many other variables affect the takeoff performance of an
aircraft. Any item that alters the takeoff speed or acceleration
rate during the takeoff roll affects the takeoff distance.

ratio, the increase in takeoff distance would be approximately
25 to 30 percent. Such a powerful effect requires proper
consideration of gross weight in predicting takeoff distance.

R

As discussed in Chapter 6, engine pressure ratio (EPR) is the
ratio between exhaust pressure (jet blast) and inlet (static)
pressure on a turbo jet or turbo fan engine. An EPR gauge
tells the pilot how much power the engines are generating.
The higher the EPR, the higher the engine thrust. EPR is
used to avoid over-boosting an engine and to set takeoff and
go around power if needed. This information is important to
know before taking off as it helps determine the performance
of the aircraft.

50
40
30

Ratio of wind
velocity to takeoff
or landing speed
30\%

20\%

Headwind

20
10\%

10

Tailwind
10
20
30

20\%

30\%

Ratio of wind
velocity to takeoff
or landing speed

40
50

For the aircraft with a high thrust-to-weight ratio, the increase
in takeoff distance might be approximately 21 to 22 percent,
but for the aircraft with a relatively low thrust-to-weight

10\%

60

Percent decrease
in takeoff or
landing distance

Figure 11-19. Effect of wind on takeoff and landing.

11-15

excessive AOA may not allow the aircraft to climb out of
ground effect. On the other hand, an excessive airspeed at
takeoff may improve the initial ROC and "feel" of the aircraft
but produces an undesirable increase in takeoff distance.
Assuming that the acceleration is essentially unaffected, the
takeoff distance varies with the square of the takeoff velocity.
Thus, ten percent excess airspeed would increase the takeoff
distance 21 percent. In most critical takeoff conditions, such
an increase in takeoff distance would be prohibitive, and the
pilot must adhere to the recommended takeoff speeds.
The effect of pressure altitude and ambient temperature
is to define the density altitude and its effect on takeoff
performance. While subsequent corrections are appropriate
for the effect of temperature on certain items of powerplant
performance, density altitude defines specific effects on
takeoff performance. An increase in density altitude can
produce a twofold effect on takeoff performance:
1. 	 Greater takeoff speed
2. 	 Decreased thrust and reduced net accelerating force
If an aircraft of given weight and configuration is operated at
greater heights above standard sea level, the aircraft requires
the same dynamic pressure to become airborne at the takeoff
lift coefficient. Thus, the aircraft at altitude takes off at the
same indicated airspeed (IAS) as at sea level, but because of
the reduced air density, the TAS is greater.
The effect of density altitude on powerplant thrust depends
much on the type of powerplant. An increase in altitude
above standard sea level brings an immediate decrease in
power output for the unsupercharged reciprocating engine.
However, an increase in altitude above standard sea level does
not cause a decrease in power output for the supercharged
reciprocating engine until the altitude exceeds the critical
operating altitude. For those powerplants that experience
a decay in thrust with an increase in altitude, the effect
on the net accelerating force and acceleration rate can be
approximated by assuming a direct variation with density.
Actually, this assumed variation would closely approximate
the effect on aircraft with high thrust-to-weight ratios.
Proper accounting of pressure altitude and temperature is
mandatory for accurate prediction of takeoff roll distance.
The most critical conditions of takeoff performance are the
result of some combination of high gross weight, altitude,
temperature, and unfavorable wind. In all cases, the pilot
must make an accurate prediction of takeoff distance from
the performance data of the AFM/POH, regardless of the
runway available, and strive for a polished, professional
takeoff procedure.

11-16

In the prediction of takeoff distance from the AFM/POH
data, the following primary considerations must be given:


Pressure altitude and temperature—to define the effect
of density altitude on distance



Gross weight—a large effect on distance



Wind—a large effect due to the wind or wind
component along the runway



Runway slope and condition—the effect of an incline
and retarding effect of factors such as snow or ice

Landing Performance
In many cases, the landing distance of an aircraft defines the
runway requirements for flight operations. The minimum
landing distance is obtained by landing at some minimum safe
speed, that allows sufficient margin above stall and provides
satisfactory control and capability for a go-around. Generally,
the landing speed is some fixed percentage of the stall speed
or minimum control speed for the aircraft in the landing
configuration. As such, the landing is accomplished at some
particular value of lift coefficient and AOA. The exact values
depend on the aircraft characteristics but, once defined, the
values are independent of weight, altitude, and wind.
To obtain minimum landing distance at the specified landing
speed, the forces that act on the aircraft must provide
maximum deceleration during the landing roll. The forces
acting on the aircraft during the landing roll may require
various procedures to maintain landing deceleration at the
peak value.
A distinction should be made between the procedures for
minimum landing distance and an ordinary landing roll
with considerable excess runway available. Minimum
landing distance is obtained by creating a continuous peak
deceleration of the aircraft; that is, extensive use of the brakes
for maximum deceleration. On the other hand, an ordinary
landing roll with considerable excess runway may allow
extensive use of aerodynamic drag to minimize wear and tear
on the tires and brakes. If aerodynamic drag is sufficient to
cause deceleration, it can be used in deference to the brakes
in the early stages of the landing roll (i.e., brakes and tires
suffer from continuous hard use, but aircraft aerodynamic
drag is free and does not wear out with use). The use of
aerodynamic drag is applicable only for deceleration to 60
or 70 percent of the touchdown speed. At speeds less than
60 to 70 percent of the touchdown speed, aerodynamic drag
is so slight as to be of little use, and braking must be utilized
to produce continued deceleration. Since the objective during
the landing roll is to decelerate, the powerplant thrust should
be the smallest possible positive value (or largest possible
negative value in the case of thrust reversers).

In addition to the important factors of proper procedures,
many other variables affect the landing performance. Any
item that alters the landing speed or deceleration rate during
the landing roll affects the landing distance.
The effect of gross weight on landing distance is one of the
principal items determining the landing distance. One effect
of an increased gross weight is that a greater speed is required
to support the aircraft at the landing AOA and lift coefficient.
For an example of the effect of a change in gross weight, a
21 percent increase in landing weight requires a ten percent
increase in landing speed to support the greater weight.
When minimum landing distances are considered, braking
friction forces predominate during the landing roll and, for
the majority of aircraft configurations, braking friction is the
main source of deceleration.
The minimum landing distance varies in direct proportion
to the gross weight. For example, a ten percent increase in
gross weight at landing would cause a:


Five percent increase in landing velocity



Ten percent increase in landing distance

A contingency of this is the relationship between weight and
braking friction force.
The effect of wind on landing distance is large and deserves
proper consideration when predicting landing distance. Since
the aircraft lands at a particular airspeed independent of the
wind, the principal effect of wind on landing distance is
the change in the groundspeed at which the aircraft touches
down. The effect of wind on deceleration during the landing
is identical to the effect on acceleration during the takeoff.
The effect of pressure altitude and ambient temperature is to
define density altitude and its effect on landing performance.
An increase in density altitude increases the landing speed
but does not alter the net retarding force. Thus, the aircraft
at altitude lands at the same IAS as at sea level but, because
of the reduced density, the TAS is greater. Since the aircraft
lands at altitude with the same weight and dynamic pressure,
the drag and braking friction throughout the landing roll have
the same values as at sea level. As long as the condition is
within the capability of the brakes, the net retarding force
is unchanged, and the deceleration is the same as with the
landing at sea level. Since an increase in altitude does not
alter deceleration, the effect of density altitude on landing
distance is due to the greater TAS.
The minimum landing distance at 5,000 feet is 16 percent
greater than the minimum landing distance at sea level. The
approximate increase in landing distance with altitude is

approximately three and one-half percent for each 1,000 feet
of altitude. Proper accounting of density altitude is necessary
to accurately predict landing distance.
The effect of proper landing speed is important when runway
lengths and landing distances are critical. The landing speeds
specified in the AFM/POH are generally the minimum safe
speeds at which the aircraft can be landed. Any attempt to
land at below the specified speed may mean that the aircraft
may stall, be difficult to control, or develop high rates of
descent. On the other hand, an excessive speed at landing may
improve the controllability slightly (especially in crosswinds)
but causes an undesirable increase in landing distance.
A ten percent excess landing speed causes at least a 21
percent increase in landing distance. The excess speed
places a greater working load on the brakes because of the
additional kinetic energy to be dissipated. Also, the additional
speed causes increased drag and lift in the normal ground
attitude, and the increased lift reduces the normal force on
the braking surfaces. The deceleration during this range of
speed immediately after touchdown may suffer, and it is more
probable for a tire to be blown out from braking at this point.
The most critical conditions of landing performance are
combinations of high gross weight, high density altitude,
and unfavorable wind. These conditions produce the greatest
required landing distances and critical levels of energy
dissipation on the brakes. In all cases, it is necessary to
make an accurate prediction of minimum landing distance to
compare with the available runway. A polished, professional
landing procedure is necessary because the landing phase of
flight accounts for more pilot-caused aircraft accidents than
any other single phase of flight.
In the prediction of minimum landing distance from the
AFM/POH data, the following considerations must be given:


Pressure altitude and temperature—to define the effect
of density altitude



Gross weight—which defines the CAS for landing



Wind—a large effect due to wind or wind component
along the runway



Runway slope and condition—relatively small
correction for ordinary values of runway slope, but a
significant effect of snow, ice, or soft ground

A tail wind of ten knots increases the landing distance by
about 21 percent. An increase of landing speed by ten percent
increases the landing distance by 20 percent. Hydroplaning
makes braking ineffective until a decrease of speed that can
be determined by using Figure 11-18.

11-17

For instance, a pilot is downwind for runway 18, and the
tower asks if runway 27 could be accepted. There is a light
rain and the winds are out of the east at ten knots. The pilot
accepts because he or she is approaching the extended
centerline of runway 27. The turn is tight and the pilot must
descend (dive) to get to runway 27. After becoming aligned
with the runway and at 50 feet AGL, the pilot is already 1,000
feet down the 3,500 feet runway. The airspeed is still high
by about ten percent (should be at 70 knots and is at about
80 knots). The wind of ten knots is blowing from behind.
First, the airspeed being high by about ten percent (80 knots
versus 70 knots), as presented in the performance chapter,
results in a 20 percent increase in the landing distance.
In performance planning, the pilot determined that at 70
knots the distance would be 1,600 feet. However, now it
is increased by 20 percent and the required distance is now
1,920 feet.
The newly revised landing distance of 1,920 feet is also
affected by the wind. In looking at Figure 11-19, the affect
of the wind is an additional 20 percent for every ten miles
per hour (mph) in wind. This is computed not on the original
estimate but on the estimate based upon the increased
airspeed. Now the landing distance is increased by another
320 feet for a total requirement of 2,240 feet to land the
airplane after reaching 50 feet AGL.
That is the original estimate of 1,600 under planned conditions
plus the additional 640 feet for excess speed and the tailwind.
Given the pilot overshot the threshhold by 1,000 feet, the
total length required is 3,240 on a 3,500 foot runway; 260
feet to spare. But this is in a perfect environment. Most pilots
become fearful as the end of the runway is facing them just
ahead. A typical pilot reaction is to brake—and brake hard.
Because the aircraft does not have antilock braking features
like a car, the brakes lock, and the aircraft hydroplanes on
the wet surface of the runway until decreasing to a speed of
about 54 knots (the square root of the tire pressure (√36) ×
9). Braking is ineffective when hydroplaning.
The 260 feet that a pilot might feel is left over has long since
evaporated as the aircraft hydroplaned the first 300–500 feet
when the brakes locked. This is an example of a true story,
but one which only changes from year to year because of new
participants and aircraft with different N-numbers.
In this example, the pilot actually made many bad decisions.
Bad decisions, when combined, have a synergy greater
than the individual errors. Therefore, the corrective
actions become larger and larger until correction is almost
impossible. Aeronautical decision-making is discussed more
fully in Chapter 2, Aeronautical Decision-Making (ADM).
11-18

Performance Speeds
True airspeed (TAS)—the speed of the aircraft in relation to
the air mass in which it is flying.
Indicated airspeed (IAS)—the speed of the aircraft as
observed on the ASI. It is the airspeed without correction for
indicator, position (or installation), or compressibility errors.
Calibrated airspeed (CAS)—the ASI reading corrected for
position (or installation) and instrument errors. (CAS is
equal to TAS at sea level in standard atmosphere.) The color
coding for various design speeds marked on ASIs may be
IAS or CAS.
Equivalent airspeed (EAS)—the ASI reading corrected
for position (or installation), for instrument error, and for
adiabatic compressible flow for the particular altitude. (EAS
is equal to CAS at sea level in standard atmosphere.)
VS0—the calibrated power-off stalling speed or the minimum
steady flight speed at which the aircraft is controllable in the
landing configuration.
VS1—the calibrated power-off stalling speed or the minimum
steady flight speed at which the aircraft is controllable in a
specified configuration.
VY—the speed at which the aircraft obtains the maximum
increase in altitude per unit of time. This best ROC speed
normally decreases slightly with altitude.
VX—the speed at which the aircraft obtains the highest
altitude in a given horizontal distance. This best AOC speed
normally increases slightly with altitude.
VLE—the maximum speed at which the aircraft can be safely
flown with the landing gear extended. This is a problem
involving stability and controllability.
VLO—the maximum speed at which the landing gear can
be safely extended or retracted. This is a problem involving
the air loads imposed on the operating mechanism during
extension or retraction of the gear.
VFE—the highest speed permissible with the wing flaps in a
prescribed extended position. This is because of the air loads
imposed on the structure of the flaps.
VA—the calibrated design maneuvering airspeed. This is
the maximum speed at which the limit load can be imposed
(either by gusts or full deflection of the control surfaces)
without causing structural damage. Operating at or below

Each aircraft performs differently and, therefore, has different
performance numbers. Compute the performance of the
aircraft prior to every flight, as every flight is different. (See
appendix for examples of performance charts for a Cessna
Model 172R and Challenger 605.)

maneuvering speed does not provide structural protection
against multiple full control inputs in one axis or full control
inputs in more than one axis at the same time.
VN0—the maximum speed for normal operation or the
maximum structural cruising speed. This is the speed at
which exceeding the limit load factor may cause permanent
deformation of the aircraft structure.

Every chart is based on certain conditions and contains
notes on how to adapt the information for flight conditions.
It is important to read every chart and understand how to
use it. Read the instructions provided by the manufacturer.
For an explanation on how to use the charts, refer to the
example provided by the manufacturer for that specific chart.
[Figure 11-20]

VNE—the speed that should never be exceeded. If flight is
attempted above this speed, structural damage or structural
failure may result.

Performance Charts

The information manufacturers furnish is not standardized.
Information may be contained in a table format and
other information may be contained in a graph format.
Sometimes combined graphs incorporate two or more graphs
into one chart to compensate for multiple conditions of
flight. Combined graphs allow the pilot to predict aircraft
performance for variations in density altitude, weight,
and winds all on one chart. Because of the vast amount of
information that can be extracted from this type of chart, it
is important to be very accurate in reading the chart. A small
error in the beginning can lead to a large error at the end.

Performance charts allow a pilot to predict the takeoff, climb,
cruise, and landing performance of an aircraft. These charts,
provided by the manufacturer, are included in the AFM/POH.
Information the manufacturer provides on these charts has
been gathered from test flights conducted in a new aircraft,
under normal operating conditions while using average
piloting skills, and with the aircraft and engine in good
working order. Engineers record the flight data and create
performance charts based on the behavior of the aircraft
during the test flights. By using these performance charts,
a pilot can determine the runway length needed to take off
and land, the amount of fuel to be used during flight, and the
time required to arrive at the destination. It is important to
remember that the data from the charts will not be accurate
if the aircraft is not in good working order or when operating
under adverse conditions. Always consider the necessity to
compensate for the performance numbers if the aircraft is not
in good working order or piloting skills are below average.

,

The remainder of this section covers performance information
for aircraft in general and discusses what information the
charts contain and how to extract information from the charts
by direct reading and interpolation methods. Every chart
contains a wealth of information that should be used when
flight planning. Examples of the table, graph, and combined
graph formats for all aspects of flight are discussed.

,

,
Figure 11-20. Conditions notes chart.

11-19

Interpolation
Not all of the information on the charts is easily extracted.
Some charts require interpolation to find the information for
specific flight conditions. Interpolating information means
that by taking the known information, a pilot can compute
intermediate information. However, pilots sometimes round
off values from charts to a more conservative figure.

and read the approximate density altitude. The approximate
density altitude in thousands of feet is 7,700 feet.
Takeoff Charts
Takeoff charts are typically provided in several forms and
allow a pilot to compute the takeoff distance of the aircraft
with no flaps or with a specific flap configuration. A pilot can
also compute distances for a no flap takeoff over a 50 foot
obstacle scenario, as well as with flaps over a 50 foot obstacle.
The takeoff distance chart provides for various aircraft
weights, altitudes, temperatures, winds, and obstacle heights.

Using values that reflect slightly more adverse conditions
provides a reasonable estimate of performance information
and gives a slight margin of safety. The following illustration
is an example of interpolating information from a takeoff
distance chart. [Figure 11-21]

Sample Problem 2
Pressure Altitude...............................................2,000 feet


Density Altitude Charts
Use a density altitude chart to figure the density altitude at the
departing airport. Using Figure 11-22, determine the density
altitude based on the given information.

OAT..........................................................................22 °C

Takeoff Weight.............................................2,600 pounds

Headwind...............................................................6 knots

Obstacle Height.......................................50 foot obstacle


Sample Problem 1

Conditions

Airport Elevation...............................................5,883 feet
 Refer to Figure 11-23. This chart is an example of a combined
OAT...........................................................................70 °F
 takeoff distance graph. It takes into consideration pressure
Altimeter...........................................................30.10 "Hg
 altitude, temperature, weight, wind, and obstacles all on one
chart. First, find the correct temperature on the bottom left
First, compute the pressure altitude conversion. Find 30.10 side of the graph. Follow the line from 22 °C straight up until
under the altimeter heading. Read across to the second it intersects the 2,000 foot altitude line. From that point, draw
column. It reads "–165." Therefore, it is necessary to subtract a line straight across to the first dark reference line. Continue
165 from the airport elevation giving a pressure altitude of to draw the line from the reference point in a diagonal
5,718 feet. Next, locate the outside air temperature on the direction following the surrounding lines until it intersects
scale along the bottom of the graph. From 70°, draw a line up the corresponding weight line. From the intersection of 2,600
to the 5,718 feet pressure altitude line, which is about two- pounds, draw a line straight across until it reaches the second
thirds of the way up between the 5,000 and 6,000 foot lines. reference line. Once again, follow the lines in a diagonal
Draw a line straight across to the far left side of the graph manner until it reaches the six knot headwind mark. Follow
TAKEOFF DISTANCE
MAXIMUM WEIGHT 2,400 LB

Flaps 10°
Full throttle prior to brake release
Paved level runway
Zero wind

Weight
(lb)
2,400

Takeoff
speed KIAS
Lift
off

AT
50 ft

51

56

0 °C
Press
ALT
(ft)
S.L.
1,000
2,000
3,000
4,000
5,000
6,000
7,000
8,000

Grnd
roll
(ft)
795
875
960
1,055
1,165
1,285
1,425
1,580
1,755

10 °C

2

Figure 11-21. Interpolating charts.

11-20

30 °C

40 °C

Total feet
to clear
50 ft OBS

Grnd
roll
(ft)

Total feet
to clear
50 ft OBS

Grnd
roll
(ft)

Total feet
to clear
50 ft OBS

Grnd
roll
(ft)

Total feet
to clear
50 ft OBS

Grnd
roll
(ft)

Total feet
to clear
50 ft OBS

1,460
1,605
1,770
1,960
2,185
2,445
2,755
3,140
3,615

860
940
1,035
1,140
1,260
1,390
1,540
1,710
1,905

1,570
1,725
1,910
2,120
2,365
2,660
3,015
3,450
4,015

925
1,015
1,115
1,230
1,355
1,500
1,665
1,850
2,060

1,685
1,860
2,060
2,295
2,570
2,895
3,300
3,805
4,480

995
1,090
1,200
1,325
1,465
1,620
1,800
2,000
---

1,810
2,000
2,220
2,480
2,790
3,160
3,620
4,220
--­

1,065
1,170
1,290
1,425
1,575
1,745
1,940
-----

1,945
2,155
2,395
2,685
3,030
3,455
3,990
-----

To find the takeoff distance for a pressure altitude of 2,500 feet
at 20 °C, average the ground roll for 2,000 feet and 3,000 feet.
1,115 + 1,230

20 °C

= 1,173 feet

Sample Problem 3

13

(fe
et
)
e

al
tit

ud
e

11
,00
0

Pr

es
su
r

10
,00
0

9,0
00

10

a
Stand

6,0

00

7

re

7,0

peratu

8

00

8,0
00

9

rd tem

5,0

00

6
5
4,0

00

Approximate density altitude (thousand feet)

11

3,0

00

4

00

3

28.0

1,824

28.1

1,727

28.2

1,630

28.3

1,533

28.4

1,436

28.5

1,340

28.6

1,244

28.7

1,148

28.8

1,053

28.9

957

29.0

863

29.1

768

29.2

673

29.3

579

29.4

485

29.5

392

29.6

298

29.7

205

29.8

112

29.9

20

29.92

0
−73

30.1

−165

30.2

−257

30.3

−348

30.4

−440

30.5

−531

C -18 -12° -7° -1° 4° 10° 16° 21° 27° 32° 38° 30.6

−622

30.7

−712

30.8

−803

2,0

30.0

le
a
–1

,00

0

Se

1,0

00

ve

l

2
1

Pressure Altitude...............................................3,000 feet

OAT.........................................................................30 °C

Takeoff Weight............................................2,400 pounds

Headwind............................................................18 knots


12
,00
0

12

Pressure altitude
conversion factor

13
,00
0

14

Altimeter setting
("Hg)

14
,00
0

15

S.L.

F 0° 10° 20° 30° 40° 50° 60° 70° 80° 90° 100°

Outside air temperature
Figure 11-22. Density altitude chart.

straight across to the third reference line and from here, draw
a line in two directions. First, draw a line straight across to
figure the ground roll distance. Next, follow the diagonal lines
again until they reach the corresponding obstacle height. In
this case, it is a 50 foot obstacle. Therefore, draw the diagonal
line to the far edge of the chart. This results in a 700 foot
ground roll distance and a total distance of 1,400 feet over a
50 foot obstacle. To find the corresponding takeoff speeds
at lift-off and over the 50 foot obstacle, refer to the table on
the top of the chart. In this case, the lift-off speed at 2,600
pounds would be 63 knots and over the 50 foot obstacle
would be 68 knots.

Refer to Figure 11-24. This chart is an example of a takeoff
distance table for short-field takeoffs. For this table, first find
the takeoff weight. Once at 2,400 pounds, begin reading from
left to right across the table. The takeoff speed is in the second
column and, in the third column under pressure altitude, find
the pressure altitude of 3,000 feet. Carefully follow that line
to the right until it is under the correct temperature column
of 30 °C. The ground roll total reads 1,325 feet and the total
required to clear a 50 foot obstacle is 2,480 feet. At this point,
there is an 18 knot headwind. According to the notes section
under point number two, decrease the distances by ten percent
for each 9 knots of headwind. With an 18 knot headwind, it
is necessary to decrease the distance by 20 percent. Multiply
1,325 feet by 20 percent (1,325 × .20 = 265), subtract the
product from the total distance (1,325 – 265 = 1,060). Repeat
this process for the total distance over a 50 foot obstacle. The
ground roll distance is 1,060 feet and the total distance over
a 50 foot obstacle is 1,984 feet.
Climb and Cruise Charts
Climb and cruise chart information is based on actual flight
tests conducted in an aircraft of the same type. This information
is extremely useful when planning a cross-country flight to
predict the performance and fuel consumption of the aircraft.
Manufacturers produce several different charts for climb and
cruise performance. These charts include everything from
fuel, time, and distance to climb to best power setting during
cruise to cruise range performance.
The first chart to check for climb performance is a fuel,
time, and distance-to-climb chart. This chart gives the fuel
amount used during the climb, the time it takes to accomplish
the climb, and the ground distance that is covered during
the climb. To use this chart, obtain the information for
the departing airport and for the cruise altitude. Using
Figure 11-25, calculate the fuel, time, and distance to climb
based on the information provided.
Sample Problem 4
Departing Airport Pressure Altitude.................6,000 feet

Departing Airport OAT............................................25 °C

Cruise Pressure Altitude..................................10,000 feet

Cruise OAT..............................................................10 °C


11-21

s

Pre

de -

ltitu

ea
sur

83
81
78
76
73

ap Gui
pl de
ic
ab line
te
rm
le s
fo no
ed
r
t
ia
te

72
70
68
66
63

Power Full throttle 2,600 rpm
Mixture Lean to appropriate fuel
pressure
Flaps
Up
Landing Retract after positive
gear
climb established
Cowl
Open
flaps

e

cl

a
st

he

5,000
s

ht

ig

In

76
74
72
70
67

66
64
63
61
58

MPH

Reference line

2,950
2,800
2,600
2,400
2,200

50 ft
kts

Ta
ilw
in
d

Lift-off
kts
MPH

Reference line

Weight
pounds

Associated conditions

Reference line

6,000

Takeoff speed

4,000

b

O

3,000

He

adw

A
et IS

fe

ind

2,000

000
10, 00
8,0
00
6,0 00
4,0 000
2, .L.
S

1,000

C -40° -30° -20° -10° 0° 10° 20° 30° 40° 50°
Outside air temperature
F

-40° -20°

0°

20° 40°

60°

2,800

80° 100° 120°

2,600
2,400
Weight
(pounds)

2,200 0

10 20 30
0
50
Wind component
Obstacle
(knots)
height (feet)

0

Conditions

Flaps 10°
Full throttle prior to brake release
Paved level runway
Zero wind

Notes

Figure 11-23. Takeoff distance graph.

1. Prior to takeoff from fields above 3,000 feet elevation, the mixture should be leaned to give maximum rpm in a full throttle, static runup.
2. Decrease distances 10\% for each 9 knots headwind. For operation with tailwind up to 10 knots, increase distances by 10\% for each 2 knots.
3. For operation on a dry, grass runway, increase distances by 15\% of the "ground roll" figure.

Takeoff
speed KIAS
Weight
(lb)

TAKEOFF DISTANCE
MAXIMUM WEIGHT 2,400 LB

SHORT FIELD

0 °C
Press
ALT
(ft)

20 °C

30 °C

40 °C

Grnd
roll
(ft)

Total feet
to clear
50 ft OBS

Grnd
roll
(ft)

Total feet
to clear
50 ft OBS

Grnd
roll
(ft)

Total feet
to clear
50 ft OBS

Grnd
roll
(ft)

Total feet
to clear
50 ft OBS

Grnd
roll
(ft)

Total feet
to clear
50 ft OBS

S.L.
1,000
2,000
3,000
4,000
5,000
6,000
7,000
8,000

795
875
960
1,055
1,165
1,285
1,425
1,580
1,755

1,460
1,605
1,770
1,960
2,185
2,445
2,755
3,140
3,615

860
940
1,035
1,140
1,260
1,390
1,540
1,710
1,905

1,570
1,725
1,910
2,120
2,365
2,660
3,015
3,450
4,015

925
1,015
1,115
1,230
1,355
1,500
1,665
1,850
2,060

1,685
1,860
2,060
2,295
2,570
2,895
3,300
3,805
4,480

995
1,090
1,200
1,325
1,465
1,620
1,800
2,000
---

1,810
2,000
2,220
2,480
2,790
3,160
3,620
4,220
--­

1,065
1,170
1,290
1,425
1,575
1,745
1,940
-----

1,945
2,155
2,395
2,685
3,030
3,455
3,990
-----

54

S.L.
1,000
2,000
3,000
4,000
5,000
6,000
7,000
8,000

650
710
780
855
945
1,040
1,150
1,270
1,410

1,195
1,310
1,440
1,585
1,750
1,945
2,170
2,440
2,760

700
765
840
925
1,020
1,125
1,240
1,375
1,525

1,280
1,405
1,545
1,705
1,890
2,105
2,355
2,655
3,015

750
825
905
995
1,100
1,210
1,340
1,485
1,650

1,375
1,510
1,660
1,835
2,040
2,275
2,555
2,890
3,305

805
885
975
1,070
1,180
1,305
1,445
1,605
1,785

1,470
1,615
1,785
1,975
2,200
2,465
2,775
3,155
3,630

865
950
1,045
1,150
1,270
1,405
1,555
1,730
1,925

1,575
1,735
1,915
2,130
2,375
2,665
3,020
3,450
4,005

51

S.L.
1,000
2,000
3,000
4,000
5,000
6,000
7,000
8,000

525
570
625
690
755
830
920
1,015
1,125

970
1,060
1,160
1,270
1,400
1,545
1,710
1,900
2,125

565
615
675
740
815
900
990
1,095
1,215

1,035
1,135
1,240
1,365
1,500
1,660
1,845
2,055
2,305

605
665
725
800
880
970
1,070
1,180
1,310

1,110
1,215
1,330
1,465
1,615
1,790
1,990
2,225
2,500

650
710
780
860
945
2,145
2,405
2,715
1,410

1,185
1,295
1,425
1,570
1,735
1,925
2,145
2,405
2,715

695
765
840
920
1,015
1,120
1,235
1,370
1,520

1,265
1,385
1,525
1,685
1,865
2,070
2,315
2,605
2,950

Lift
off

AT
50 ft

2,400

51

56

2,200

49

2,000

46

Figure 11-24. Takeoff distance short field charts.

11-22

10 °C

- fee

t

F ue

16,000
14,000

-m

AL T

D

Cruise

12,000

es
ut
in

Ti
m
e

re
Pressu
18,000

l-g
allo
ns

0

20,00

enc
ta
s
i

ica
ut
na

lm

s
ile

Associated conditions

10,000

Maximum continuous power*
3,600 lb gross weight
Flaps up
90 KIAS
No wind
* 2,700 rpm \& 36 in M.P. (3-blade prop)
2,575 rpm \& 36 in M.P. (2-blade prop)

8,000
6,000
4,000
2,000

Sea level

Departure

-40° -30° -20° -10° 0° 10° 20° 30° 40°C
Outside air temperature

0

10
20
30
Fuel, time and distance to climb

40

50

Figure 11-25. Fuel, time, and distance climb chart.

The next example is a fuel, time, and distance-to-climb table.
For this table, use the same basic criteria as for the previous
chart. However, it is necessary to figure the information in a
different manner. Refer to Figure 11-26 to work the following
sample problem.

Conditions

To begin, find the given weight of 3,400 in the first column of
the chart. Move across to the pressure altitude column to find
the sea level altitude numbers. At sea level, the numbers read
zero. Next, read the line that corresponds with the cruising
altitude of 8,000 feet. Normally, a pilot would subtract these
Flaps up
Gear up
2,500 rpm
30 "Hg
120 PPH fuel flow
Cowl flaps open
Standard temperature

Notes

First, find the information for the departing airport. Find the
OAT for the departing airport along the bottom, left side of the
graph. Follow the line from 25 °C straight up until it intersects
the line corresponding to the pressure altitude of 6,000 feet.
Continue this line straight across until it intersects all three
lines for fuel, time, and distance. Draw a line straight down
from the intersection of altitude and fuel, altitude and time, and
a third line at altitude and distance. It should read three and
one-half gallons of fuel, 6 minutes of time, and nine NM. Next,
repeat the steps to find the information for the cruise altitude.
It should read six gallons of fuel, 10.5 minutes of time, and
15 NM. Take each set of numbers for fuel, time, and distance
and subtract them from one another (6.0 – 3.5 = 2.5 gallons of
fuel). It takes two and one-half gallons of fuel and 4 minutes
of time to climb to 10,000 feet. During that climb, the distance
covered is six NM. Remember, according to the notes at the
top of the chart, these numbers do not take into account wind,
and it is assumed maximum continuous power is being used.

1. Add 16 pounds of fuel for engine start, taxi, and takeoff allowance.
2. Increase time, fuel, and distance by 10\% for each 7 °C above standard
temperature.
3. Distances shown are based on zero wind.

Press
ALT
(feet)

Rate of
climb
fpm

4,000

S.L.
4,000
8,000
12,000
16,000
20,000

3,700

Weight
(pounds)

Sample Problem 5
Departing Airport Pressure Altitude..................Sea level

Departing Airport OAT............................................22 °C


NORMAL CLIMB
110 KIAS

3,400

From sea level
Time
(minutes)

Fuel used
(pounds)

Distance
(nautical
miles)

605
570
530
485
430
365

0
7
14
22
31
41

0
14
28
44
62
82

0
13
27
43
63
87

S.L.
4,000
8,000
12,000
16,000
20,000

700
665
625
580
525
460

0
6
12
19
26
34

0
12
24
37
52
68

0
11
23
37
53
72

S.L.
4,000
8,000
12,000
16,000
20,000

810
775
735
690
635
565

0
5
10
16
22
29

0
10
21
32
44
57

0
9
20
31
45
61

Cruise Pressure Altitude....................................8,000 feet

Takeoff Weight.............................................3,400 pounds


Figure 11-26. Fuel time distance climb.

11-23

The next example is a cruise and range performance chart.
This type of table is designed to give TAS, fuel consumption,
endurance in hours, and range in miles at specific cruise
configurations. Use Figure 11-27 to determine the cruise and
range performance under the given conditions.
Sample Problem 6
Pressure Altitude...............................................5,000 feet

RPM..................................................................2,400 rpm


Conditions

Gross weight—2,300 lb.
Standard conditions
Zero wind
Lean mixture

Notes

two sets of numbers from one another, but given the fact that
the numbers read zero at sea level, it is known that the time
to climb from sea level to 8,000 feet is 10 minutes. It is also
known that 21 pounds of fuel is used and 20 NM is covered
during the climb. However, the temperature is 22 °C, which is
7° above the standard temperature of 15 °C. The notes section
of this chart indicate that the findings must be increased by ten
percent for each 7° above standard. Multiply the findings by
ten percent or .10 (10 × .10 = 1, 1 + 10 = 11 minutes). After
accounting for the additional ten percent, the findings should
read 11 minutes, 23.1 pounds of fuel, and 22 NM. Notice that
the fuel is reported in pounds of fuel, not gallons. Aviation
fuel weighs six pounds per gallon, so 23.1 pounds of fuel is
equal to 3.85 gallons of fuel (23.1 ÷ 6 = 3.85).

Maximum cruise is normally limited to 75\% power.

\%
BHP

TAS
MPH

GAL/
Hour

2,500 2,700
2,600
2,500
2,400
2,300
2,200

86
79
72
65
58
52

134
129
123
117
111
103

5,000 2,700
2,600
2,500
2,400
2,300
2,200

82
75
68
61
55
49

7,500 2,700
2,600
2,500
2,400
2,300
10,000 2,650
2,600
2,500
2,400
2,300

ALT

RPM

38 gal
(no reserve)

48 gal
(no reserve)

9.7
8.6
7.8
7.2
6.7
6.3

Endr.
hours
3.9
4.4
4.9
5.3
5.7
6.1

Range
miles
525
570
600
620
630
625

Endr.
hours
4.9
5.6
6.2
6.7
7.2
7.7

Range
miles
660
720
760
780
795
790

134
128
122
116
108
100

9.0
8.1
7.4
6.9
6.5
6.0

4.2
4.7
5.1
5.5
5.9
6.3

565
600
625
635
635
630

5.3
5.9
6.4
6.9
7.4
7.9

710
760
790
805
805
795

78
71
64
58
52

133
127
121
113
105

8.4
7.7
7.1
6.7
6.2

4.5
4.9
5.3
5.7
6.1

600
625
645
645
640

5.7
6.2
6.7
7.2
7.7

755
790
810
820
810

70
67
61
55
49

129
125
118
110
100

7.6
7.3
6.9
6.4
6.0

5.0
5.2
5.5
5.9
6.3

640
650
655
650
635

6.3
6.5
7.0
7.5
8.0

810
820
830
825
800

Figure 11-27. Cruise and range performance.

Fuel Carrying Capacity..................38 gallons, no reserve
 Another type of cruise chart is a best power mixture range
graph. This graph gives the best range based on power
Find 5,000 feet pressure altitude in the first column on the setting and altitude. Using Figure 11-29, find the range at
left side of the table. Next, find the correct rpm of 2,400 65 percent power with and without a reserve based on the
in the second column. Follow that line straight across and provided conditions.
read the TAS of 116 mph and a fuel burn rate of 6.9 gallons
per hour. As per the example, the aircraft is equipped with Sample Problem 8
a fuel carrying capacity of 38 gallons. Under this column,
OAT....................................................................Standard

read that the endurance in hours is 5.5 hours and the range
in miles is 635 miles.
Pressure Altitude...............................................5,000 feet

Cruise power setting tables are useful when planning crosscountry flights. The table gives the correct cruise power
settings, as well as the fuel flow and airspeed performance
numbers at that altitude and airspeed.

First, move up the left side of the graph to 5,000 feet and
standard temperature. Follow the line straight across the
graph until it intersects the 65 percent line under both the
reserve and no reserve categories. Draw a line straight down
from both intersections to the bottom of the graph. At 65
Sample Problem 7
percent power with a reserve, the range is approximately
Pressure Altitude at Cruise................................6,000 feet
 522 miles. At 65 percent power with no reserve, the range
should be 581 miles.
OAT..................................................36 °F above standard

Refer to Figure 11-28 for this sample problem. First, locate
the pressure altitude of 6,000 feet on the far left side of the
table. Follow that line across to the far right side of the table
under the 20 °C (or 36 °F) column. At 6,000 feet, the rpm
setting of 2,450 will maintain 65 percent continuous power
at 21.0 "Hg with a fuel flow rate of 11.5 gallons per hour and
airspeed of 161 knots.

11-24

The last cruise chart referenced is a cruise performance graph.
This graph is designed to tell the TAS performance of the
airplane depending on the altitude, temperature, and power
setting. Using Figure 11-30, find the TAS performance based
on the given information.

CRUISE POWER SETTING
65\% MAXIMUM CONTINUOUS POWER (OR FULL THROTTLE)
2,800 POUNDS
ISA –20° (–36 °F)
IOAT

Press
ALT

Notes

S.L.
2,000
4,000
6,000
8,000
10,000
12,000
14,000
16,000

Standard day (ISA)

Fuel
flow per
engine

Engine Man.
speed press

TAS

°F

°C

RPM

"HG PSI GPH kts MPH °F

27
19
12
5
–2
–8
–15
–22
–29

–3
–7
–11
–15
–19
–22
–26
–30
–34

2,450
2,450
2,450
2,450
2,450
2,450
2,450
2,450
2,450

20.7
20.4
20.1
19.8
19.5
19.2
18.8
17.4
16.1

6.6
6.6
6.6
6.6
6.6
6.6
6.4
5.8
5.3

11.5
11.5
11.5
11.5
11.5
11.5
11.3
10.5
9.7

147
149
152
155
157
160
162
159
156

169
171
175
178
181
184
186
183
180

Engine Man.
speed press

IOAT
°C

RPM

63 17 2,450
55 13 2,450
48
9 2,450
41
5 2,450
36
2 2,450
28 –2 2,450
21 –6 2,450
14 –10 2,450
7 –14 2,450

Fuel
flow per
engine

ISA +20° (+36 °F)
TAS

IOAT

Engine Man.
speed press

Fuel
flow per
engine

TAS

"HG PSI GPH kts MPH °F

°C

RPM

"HG PSI GPH kts MPH

21.2
21.0
20.7
20.4
20.2
19.9
18.8
17.4
16.1

37
33
29
26
22
18
14
10
6

2,450
2,450
2,450
2,450
2,450
2,450
2,450
2,450
2,450

21.8
21.5
21.3
21.0
20.8
20.3
18.8
17.4
16.1

6.6
6.6
6.6
6.6
6.6
6.6
6.1
5.6
5.1

11.5
11.5
11.5
11.5
11.5
11.5
10.9
10.1
9.4

150
153
156
158
161
163
163
160
156

173
176
180
182
185
188
188
184
180

99
91
84
79
72
64
57
50
43

6.6
6.6
6.6
6.6
6.6
6.5
5.9
5.4
4.9

11.5
11.5
11.5
11.5
11.5
11.4
10.6
9.8
9.1

153
156
159
161
164
166
163
160
155

176
180
183
185
189
191
188
184
178

1. Full throttle manifold pressure settings are approximate.
2.
Shaded area represents operation with full throttle.

Figure 11-28. Cruise power setting.

Sample Problem 9
OAT.........................................................................16 °C

Pressure Altitude...............................................6,000 feet

Power Setting................................65 percent, best power

Wheel Fairings..............................................Not installed

Begin by finding the correct OAT on the bottom left side of
the graph. Move up that line until it intersects the pressure
altitude of 6,000 feet. Draw a line straight across to the
14 -13°

\%
Power 55

7°

Power 65\%

4

75\%

3°

Power

6

\%

-1°

Power 65

8

Power 55

-5°

Notes

10

No reserve

75\%

-9°

Power

12

\%

45 minutes reserve at 55\%
power best economy mixture

Range may be reduced
by up to 7\% if wheel
fairings are not installed

Runway..........................................................................17

450

Notes

Standard Temperature °C

Pressure ALT (1,000 feet)

Crosswind and Headwind Component Chart
Every aircraft is tested according to Federal Aviation
Administration (FAA) regulations prior to certification. The
aircraft is tested by a pilot with average piloting skills in
90° crosswinds with a velocity up to 0.2 VS0 or two-tenths
of the aircraft's stalling speed with power off, gear down,
and flaps down. This means that if the stalling speed of the
aircraft is 45 knots, it must be capable of landing in a 9-knot,
90° crosswind. The maximum demonstrated crosswind
component is published in the AFM/POH. The crosswind and
headwind component chart allows for figuring the headwind
and crosswind component for any given wind direction and
velocity.
Sample Problem 10

2 11°
S.L. 15°

65 percent, best power line. This is the solid line, that
represents best economy. Draw a line straight down from
this intersection to the bottom of the graph. The TAS at 65
percent best power is 140 knots. However, it is necessary
to subtract 8 knots from the speed since there are no wheel
fairings. This note is listed under the title and conditions.
The TAS is 132 knots.

500

550
600
500
550
600
Range (nautical miles)
(Includes distance to climb and descend)
Add 0.6 NM for each
degree Celsius above
standard temperature
and subtract 1 NM for
each degree Celsius
below standard
temperature.

650

Associated conditions
Mixture
Weight
Wings
Fuel
Wheel
Cruise

Figure 11-29. Best power mixture range.

Leaned per section 4
2,300 lb.
No
48 gal usable
Fairings installed
Mid cruise

Wind........................................................140° at 25 knots

Refer to Figure 11-31 to solve this problem. First, determine
how many degrees difference there is between the runway
and the wind direction. It is known that runway 17 means
a direction of 170°; from that subtract the wind direction
of 140°. This gives a 30° angular difference or wind angle.
Next, locate the 30° mark and draw a line from there until
it intersects the correct wind velocity of 25 knots. From

11-25

r
ssu

L
eA

t 36

at

0r

l

leve

2,7
0

Sea

2,5
7

00

2,0

5r

00

4,0

pm

pm
a

0

6,00

36

IN.

00

8,0

IN.

M.P
.

000

10,

–40° –30° –20° –10° 0° 10° 20° 30° 40°
Outside air temperature (°C)

100

120

140

Subtract 8 knots if wheel
fairings are not installed.

75\%
Powe
r

\%

Notes

Pre

00
12,0

–2
-bl
ade
M.P
.–
pro
3-b
p
lad
ep
ro p

ure
rat
pe
te m

000

14,

)
eet
T (f

Power
65

rd

000

16,

\%

00

18,0

Power
55

20,

a
nd
Sta

000

Best power
Best economy

Associated conditions
Weight
Flaps
Best power

3,600 lb. gross weight
Up
Mixture leaned to 100°
rich of peak EGT
Best economy Mixture leaned to peak EGT
1,650° Max allowable EGT
Wheel
Fairings installed

160
180
200
True airspeed (knots)

Figure 11-30. Cruise performance graph.

there, draw a line straight down and a line straight across.
The headwind component is 22 knots and the crosswind
component is 13 knots. This information is important when
taking off and landing so that, first of all, the appropriate
runway can be picked if more than one exists at a particular
airport, but also so that the aircraft is not pushed beyond its
tested limits.
Landing Charts
Landing performance is affected by variables similar to those
affecting takeoff performance. It is necessary to compensate
for differences in density altitude, weight of the airplane, and
headwinds. Like takeoff performance charts, landing distance
information is available as normal landing information,
as well as landing distance over a 50 foot obstacle. As
70

0°

10°

20°
30°

Headwind component

60

Win
d

50

40°

ve
loc
i ty

50°

40

60°

30

Sample Problem 11
Pressure Altitude...............................................1,250 feet

Temperature.........................................................Standard

Refer to Figure 10-32. This example makes use of a landing
distance table. Notice that the altitude of 1,250 feet is not
on this table. It is, therefore, necessary to interpolate to find
the correct landing distance. The pressure altitude of 1,250
is halfway between sea level and 2,500 feet. First, find the
column for sea level and the column for 2,500 feet. Take the
total distance of 1,075 for sea level and the total distance of
1,135 for 2,500 and add them together. Divide the total by
two to obtain the distance for 1,250 feet. The distance is 1,105
feet total landing distance to clear a 50 foot obstacle. Repeat
this process to obtain the ground roll distance for the pressure
altitude. The ground roll should be 457.5 feet.
Sample Problem 12
OAT.......................................................................... 57 °F


70°

Pressure Altitude.............................................. 4,000 feet


20

80°
10

Landing Weight...........................................2,400 pounds

Headwind.............................................................. 6 knots


90°
0

10

20

30

40

Crosswind component

Figure 10-31. Crosswind component chart.

11-26

usual, read the associated conditions and notes in order to
ascertain the basis of the chart information. Remember, when
calculating landing distance that the landing weight is not the
same as the takeoff weight. The weight must be recalculated
to compensate for the fuel that was used during the flight.

50

60

Obstacle Height..................................................... 50 feet


70

Using the given conditions and Figure 11-33, determine the
landing distance for the aircraft. This graph is an example of

LANDING DISTANCE

Conditions

Flaps lowered to 40°
Power off
Hard surface runway
Zero wind

Gross
weight
lb

Approach speed
IAS, MPH

1,600

60

At sea level \& 59 °F

At 5,000 ft \& 41 °F

At 2,500 ft \& 50 °F

At 7,500 ft \& 32 °F

Ground roll

Total to clear
50 ft OBS

Ground roll

Total to clear
50 ft OBS

Ground roll

Total to clear
50 ft OBS

Ground roll

Total to clear
50 ft OBS

445

1,075

470

1,135

495

1,195

520

1,255

Note

1. Decrease the distances shown by 10\% for each 4 knots of headwind.
2. Increase the distance by 10\% for each 60 °F temperature increase above standard.
3. For operation on a dry, grass runway, increase distances (both "ground roll" and "total to clear 50 ft obstacle") by 20\% of the "total to clear 50 ft obstacle" figure.

Figure 11-32. Landing distance table.

a combined landing distance graph and allows compensation
for temperature, weight, headwinds, tailwinds, and varying
obstacle height. Begin by finding the correct OAT on the
scale on the left side of the chart. Move up in a straight
line to the correct pressure altitude of 4,000 feet. From this
intersection, move straight across to the first dark reference
line. Follow the lines in the same diagonal fashion until the
correct landing weight is reached. At 2,400 pounds, continue
in a straight line across to the second dark reference line.
Once again, draw a line in a diagonal manner to the correct
wind component and then straight across to the third dark
reference line. From this point, draw a line in two separate
directions: one straight across to figure the ground roll and
one in a diagonal manner to the correct obstacle height. This
should be 975 feet for the total ground roll and 1,500 feet for
the total distance over a 50 foot obstacle.

Stall Speed Performance Charts
Stall speed performance charts are designed to give an
understanding of the speed at which the aircraft stalls in
a given configuration. This type of chart typically takes
into account the angle of bank, the position of the gear and
flaps, and the throttle position. Use Figure 11-34 and the
accompanying conditions to find the speed at which the
airplane stalls.
Sample Problem 13
Power........................................................................ OFF

Flaps....................................................................... Down

Gear........................................................................ Down

Angle of Bank............................................................. 45°

First, locate the correct flap and gear configuration. The
bottom half of the chart should be used since the gear and

sure
Pres

pli Guid
ca e
bl lin
Ob e for es n
sta In ot
cle term
he ed
igh ia
ts te
ap

Reference line

80
78
75
72
69

2,500

2,000

He

1,500

ad

wi

nd

1,000

ISA

20° 40°

3,000

et)

C –40° –30° –20° –10° 0° 10° 20° 30° 40° 50°
Outside air temperature
–40° –20° 0°

70
68
65
63
60

de (fe

altitu

00
10,0 00
8,0
0
6,00 00
4,0 000
2, S.L.

F

2,950
2,800
2,600
2,400
2,200

Reference line

Flaps
Landing gear
Runway
Approach speed
Braking

Retarded to maintain
900 feet/on final approach
Down
Down
Paved, level, dry surface
IAS as tabulated
Maximum

Speed
at 50 feet
kts
MPH

ind

Power

Weight
(pounds)

Tail
w

Associated conditions

Reference line

3,500

60°

80° 100° 120°

2,800

2,600
2,400
Weight
(pounds)

2,200 0

10 20 30
0
50
Wind component
Obstacle
(knots)
height (feet)

500

Figure 11-33. Landing distance graph.

11-27

Power

Power

Gross weight
2,750 lb
On
Off

On
Off

60°

Air Carrier Obstacle Clearance
Requirements

88
76
106
92

For information on air carrier obstacle clearance
requirements consult the Instrument Procedures Handbook,
FAA-H-8083-16 (as revised).

Angle of bank
Level

MPH
knots
MPH
knots

62
54
75
65

MPH
knots
MPH
knots

54
47
66
57

30°
45°
Gear and flaps up
67
74
58
64
81
89
70
77
Gear and flaps down
58
64
50
56
71
78
62
68

76
66
93
81

Figure 11-34. Stall speed table.

flaps are down. Next, choose the row corresponding to a
power-off situation. Now, find the correct angle of bank
column, which is 45°. The stall speed is 78 mph, and the
stall speed in knots would be 68 knots.
Performance charts provide valuable information to the pilot.
By using these charts, a pilot can predict the performance of
the aircraft under most flying conditions, providing a better
plan for every flight. The Code of Federal Regulations (CFR)
requires that a pilot be familiar with all information available
prior to any flight. Pilots should use the information to their
advantage as it can only contribute to safety in flight.

Transport Category Aircraft Performance
Transport category aircraft are certificated under Title 14
of the CFR (14 CFR) part 25. For additional information
concerning transport category airplanes, consult the Airplane
Flying Handbook, FAA-H-8083-3 (as revised).
Transport category helicopters are certificated under 14
CFR part 29.

11-28

Chapter Summary
Performance characteristics and capabilities vary greatly
among aircraft. As transport aircraft become more capable
and more complex, most operators find themselves having
to rely increasingly on computerized flight mission planning
systems. These systems may be on board or used during
the planning phase of the flight. Moreover, aircraft weight,
atmospheric conditions, and external environmental factors
can significantly affect aircraft performance. It is essential
that a pilot become intimately familiar with the mission
planning programs, performance characteristics, and
capabilities of the aircraft being flown, as well as all of the
onboard computerized systems in today's complex aircraft.
The primary source of this information is the AFM/POH.

Chapter 12

Weather Theory
Introduction
Weather is an important factor that influences aircraft
performance and flying safety. It is the state of the atmosphere
at a given time and place with respect to variables, such as
temperature (heat or cold), moisture (wetness or dryness),
wind velocity (calm or storm), visibility (clearness or
cloudiness), and barometric pressure (high or low). The
term "weather" can also apply to adverse or destructive
atmospheric conditions, such as high winds.
This chapter explains basic weather theory and offers pilots
background knowledge of weather principles. It is designed
to help them gain a good understanding of how weather
affects daily flying activities. Understanding the theories
behind weather helps a pilot make sound weather decisions
based on the reports and forecasts obtained from a Flight
Service Station (FSS) weather specialist and other aviation
weather services.
Be it a local flight or a long cross-country flight, decisions
based on weather can dramatically affect the safety of the flight.

12-1

Atmosphere
The atmosphere is a blanket of air made up of a mixture of
gases that surrounds the Earth and reaches almost 350 miles
from the surface of the Earth. This mixture is in constant
motion. If the atmosphere were visible, it might look like
an ocean with swirls and eddies, rising and falling air, and
waves that travel for great distances.
Life on Earth is supported by the atmosphere, solar energy,
and the planet's magnetic fields. The atmosphere absorbs
energy from the sun, recycles water and other chemicals, and
works with the electrical and magnetic forces to provide a
moderate climate. The atmosphere also protects life on Earth
from high energy radiation and the frigid vacuum of space.
Composition of the Atmosphere
In any given volume of air, nitrogen accounts for 78 percent
of the gases that comprise the atmosphere, while oxygen
makes up 21 percent. Argon, carbon dioxide, and traces
of other gases make up the remaining one percent. This
volume of air also contains some water vapor, varying from
zero to about five percent by volume. This small amount of
water vapor is responsible for major changes in the weather.
[Figure 12-1]
The envelope of gases surrounding the Earth changes
from the ground up. Four distinct layers or spheres of the

o
erm
Th

ere
sph

p
sos
Me

her

et
280,000 fe

e

ere
sph
o
t
a
Str
re
phe
os
p
o
Tr

Figure 12-2. Layers of the atmosphere.

12-2

160,000

feet

fe
20,000

et

1\%

21\% Oxygen

n
e
g
78\% itro
N
Figure 12-1. Composition of the atmosphere.

atmosphere have been identified using thermal characteristics
(temperature changes), chemical composition, movement,
and density. [Figure 12-2]
The first layer, known as the troposphere, extends from 6
to 20 kilometers (km) (4 to 12 miles) over the northern and
southern poles and up to 48,000 feet (14.5 km) over the
equatorial regions. The vast majority of weather, clouds,
storms, and temperature variances occur within this first
layer of the atmosphere. Inside the troposphere, the average
temperature decreases at a rate of about 2 °Celsius (C) every

1,000 feet of altitude gain, and the pressure decreases at a rate
of about one inch per 1,000 feet of altitude gain.
At the top of the troposphere is a boundary known as the
tropopause, which traps moisture and the associated weather
in the troposphere. The altitude of the tropopause varies with
latitude and with the season of the year; therefore, it takes
on an elliptical shape as opposed to round. Location of the
tropopause is important because it is commonly associated with
the location of the jet stream and possible clear air turbulence.
Above the tropopause are three more atmospheric levels. The
first is the stratosphere, which extends from the tropopause to
a height of about 160,000 feet (50 km). Little weather exists
in this layer and the air remains stable, although certain types
of clouds occasionally extend in it. Above the stratosphere
are the mesosphere and thermosphere, which have little
influence over weather.
Atmospheric Circulation
As noted earlier, the atmosphere is in constant motion.
Certain factors combine to set the atmosphere in motion, but a
major factor is the uneven heating of the Earth's surface. This
heating upsets the equilibrium of the atmosphere, creating
changes in air movement and atmospheric pressure. The
movement of air around the surface of the Earth is called
atmospheric circulation.
Heating of the Earth's surface is accomplished by several
processes, but in the simple convection-only model used for
this discussion, the Earth is warmed by energy radiating from
the sun. The process causes a circular motion that results
when warm air rises and is replaced by cooler air.
Warm air rises because heat causes air molecules to spread
apart. As the air expands, it becomes less dense and lighter
than the surrounding air. As air cools, the molecules pack
together more closely, becoming denser and heavier than
warm air. As a result, cool, heavy air tends to sink and replace
warmer, rising air.
Because the Earth has a curved surface that rotates on a tilted
axis while orbiting the sun, the equatorial regions of the Earth
receive a greater amount of heat from the sun than the polar
regions. The amount of solar energy that heats the Earth
depends on the time of year and the latitude of the specific
region. All of these factors affect the length of time and the
angle at which sunlight strikes the surface.
Solar heating causes higher temperatures in equatorial areas,
which causes the air to be less dense and rise. As the warm
air flows toward the poles, it cools, becoming denser and
sinks back toward the surface. [Figure 12-3]

Figure 12-3. Circulation pattern in a static environment.

Atmospheric Pressure
The unequal heating of the Earth's surface not only modifies
air density and creates circulation patterns; it also causes
changes in air pressure or the force exerted by the weight
of air molecules. Although air molecules are invisible, they
still have weight and take up space.
Imagine a sealed column of air that has a footprint of one
square inch and is 350 miles high. It would take 14.7 pounds
of effort to lift that column. This represents the air's weight;
if the column is shortened, the pressure exerted at the bottom
(and its weight) would be less.
The weight of the shortened column of air at 18,000 feet is
approximately 7.4 pounds; almost 50 percent that at sea level.
For instance, if a bathroom scale (calibrated for sea level)
were raised to 18,000 feet, the column of air weighing 14.7
pounds at sea level would be 18,000 feet shorter and would
weigh approximately 7.3 pounds (50 percent) less than at
sea level. [Figure 12-4]
The actual pressure at a given place and time differs with
altitude, temperature, and density of the air. These conditions
also affect aircraft performance, especially with regard to
takeoff, rate of climb, and landings.

Coriolis Force
In general atmospheric circulation theory, areas of low
pressure exist over the equatorial regions and areas of high
pressure exist over the polar regions due to a difference in
temperature. The resulting low pressure allows the highpressure air at the poles to flow along the planet's surface
toward the equator. While this pattern of air circulation is

12-3

1 Square Inch

1 Square Inch

1 Square Inch

1 Square Inch

7.4

1

1188,,00 lb
0000 f
eet

1
lb

.7
14

Sea

leve

l

Figure 12-5. Three-cell circulation pattern due to the rotation of

Figure 12-4. Atmosphere weights.

the surface. Then, it flows southward along the surface back
toward the equator. Coriolis force bends the flow to the right,
thus creating the northeasterly trade winds that prevail from
30° latitude to the equator. Similar forces create circulation
cells that encircle the Earth between 30° and 60° latitude and
between 60° and the poles. This circulation pattern results in
the prevailing upper level westerly winds in the conterminous
United States.

correct in theory, the circulation of air is modified by several
forces, the most important of which is the rotation of the Earth.
The force created by the rotation of the Earth is known as
the Coriolis force. This force is not perceptible to humans as
they walk around because humans move slowly and travel
relatively short distances compared to the size and rotation
rate of the Earth. However, the Coriolis force significantly
affects motion over large distances, such as an air mass or
body of water.
The Coriolis force deflects air to the right in the Northern
Hemisphere, causing it to follow a curved path instead of a
straight line. The amount of deflection differs depending on
the latitude. It is greatest at the poles and diminishes to zero
at the equator. The magnitude of Coriolis force also differs
with the speed of the moving body—the greater the speed,
the greater the deviation. In the Northern Hemisphere, the
rotation of the Earth deflects moving air to the right and
changes the general circulation pattern of the air.
The Coriolis force causes the general flow to break up into
three distinct cells in each hemisphere. [Figure 12-5] In
the Northern Hemisphere, the warm air at the equator rises
upward from the surface, travels northward, and is deflected
eastward by the rotation of the Earth. By the time it has
traveled one-third of the distance from the equator to the
North Pole, it is no longer moving northward, but eastward.
This air cools and sinks in a belt-like area at about 30°
latitude, creating an area of high pressure as it sinks toward

12-4

the Earth.

Circulation patterns are further complicated by seasonal
changes, differences between the surfaces of continents and
oceans, and other factors such as frictional forces caused
by the topography of the Earth's surface that modify the
movement of the air in the atmosphere. For example, within
2,000 feet of the ground, the friction between the surface and
the atmosphere slows the moving air. The wind is diverted
from its path because of the frictional force. Thus, the wind
direction at the surface varies somewhat from the wind
direction just a few thousand feet above the Earth.

Measurement of Atmosphere Pressure
Atmospheric pressure historically was measured in inches of
mercury ("Hg) by a mercurial barometer. [Figure 12-6] The
barometer measures the height of a column of mercury inside a
glass tube. A section of the mercury is exposed to the pressure
of the atmosphere, which exerts a force on the mercury. An
increase in pressure forces the mercury to rise inside the tube.
When the pressure drops, mercury drains out of the tube
decreasing the height of the column. This type of barometer is
typically used in a laboratory or weather observation station,
is not easily transported, and difficult to read.

Atmosphe

29.92" (760 mm)

Height of mercury

ric pressu

re

At sea level in a standard
atmosphere, the weight
of the atmosphere
(14.7 lb/in2) supports
a column of mercury
29.92 inches high.

Sea level

29.92 "Hg = 1,013.2 mb (hPa) = 14.7 lb/in2
Figure 12-6. Although mercurial barometers are no longer used
in the U. S., they are still a good historical reference for where the
altimeter setting came from (inches of mercury).

An aneroid barometer is the standard instrument used
to measure pressure; it is easier to read and transport.
[Figure 12-7] The aneroid barometer contains a closed vessel
called an aneroid cell that contracts or expands with changes
in pressure. The aneroid cell attaches to a pressure indicator
with a mechanical linkage to provide pressure readings. The
pressure sensing part of an aircraft altimeter is essentially
an aneroid barometer. It is important to note that due to

h
Atmosp

eric pre

ssure

er
gh
Hi

Lower

Sealed aneroid cell
Sealed aneroid cell
Sealed aneroid cell

the linkage mechanism of an aneroid barometer, it is not as
accurate as a mercurial barometer.
To provide a common reference, the International Standard
Atmosphere (ISA) has been established. These standard
conditions are the basis for certain flight instruments and
most aircraft performance data. Standard sea level pressure
is defined as 29.92 "Hg and a standard temperature of 59 °F
(15 °C). Atmospheric pressure is also reported in millibars
(mb), with 1 "Hg equal to approximately 34 mb. Standard sea
level pressure is 1,013.2 mb. Typical mb pressure readings
range from 950.0 to 1,040.0 mb. Surface charts, high and low
pressure centers, and hurricane data are reported using mb.
Since weather stations are located around the globe, all local
barometric pressure readings are converted to a sea level
pressure to provide a standard for records and reports. To
achieve this, each station converts its barometric pressure by
adding approximately 1 "Hg for every 1,000 feet of elevation.
For example, a station at 5,000 feet above sea level, with a
reading of 24.92 "Hg, reports a sea level pressure reading of
29.92 "Hg. [Figure 12-8] Using common sea level pressure
readings helps ensure aircraft altimeters are set correctly,
based on the current pressure readings.
By tracking barometric pressure trends across a large area,
weather forecasters can more accurately predict movement
of pressure systems and the associated weather. For example,
tracking a pattern of rising pressure at a single weather station
generally indicates the approach of fair weather. Conversely,
decreasing or rapidly falling pressure usually indicates
approaching bad weather and, possibly, severe storms.

Altitude and Atmospheric Pressure
As altitude increases, atmospheric pressure decreases. On
average, with every 1,000 feet of increase in altitude, the
atmospheric pressure decreases 1 "Hg. As pressure decreases,
the air becomes less dense or thinner. This is the equivalent of
being at a higher altitude and is referred to as density altitude.
As pressure decreases, density altitude increases and has a
pronounced effect on aircraft performance.
Differences in air density caused by changes in temperature
result in a change in pressure. This, in turn, creates motion in
the atmosphere, both vertically and horizontally, in the form
of currents and wind. The atmosphere is almost constantly in
motion as it strives to reach equilibrium. These never-ending
air movements set up chain reactions that cause a continuing
variety in the weather.

Figure 12-7. Aneroid barometer.

12-5

Station Pressure
Denver

24.92 "Hg

Standard Atmosphere
Station Pressure
New Orleans

29.92 "Hg

New Orleans 29.92 "Hg
Denver 29.92 "Hg

Figure 12-8. Station pressure is converted to and reported in sea level pressure.

Altitude and Flight
Altitude affects every aspect of flight from aircraft
performance to human performance. At higher altitudes,
with a decreased atmospheric pressure, takeoff and landing
distances are increased, while climb rates decrease.
When an aircraft takes off, lift is created by the flow of air
around the wings. If the air is thin, more speed is required
to obtain enough lift for takeoff; therefore, the ground run

Pressure Altitude: Sea level

is longer. An aircraft that requires 745 feet of ground run at
sea level requires more than double that at a pressure altitude
of 8,000 feet. [Figure 12-9]. It is also true that at higher
altitudes, due to the decreased density of the air, aircraft
engines and propellers are less efficient. This leads to reduced
rates of climb and a greater ground run for obstacle clearance.

Altitude and the Human Body
As discussed earlier, nitrogen and other trace gases make
up 79 percent of the atmosphere, while the remaining 21

TAKEOFF DISTANCE
MAXIMUM WEIGHT 2,400 LB
0 °C
Pressure
altitude
(feet)

745 feet

Pressure Altitude: 8,000 feet

1,590 feet

Figure 12-9. Takeoff distances increase with increased altitude.

12-6

S.L.
1,000
2,000
3,000
4,000
5,000
6,000
7,000
8,000

Ground
roll
(feet)

Total feet
to clear
50 foot obstacle

745
815
895
980
1,075
1,185
1,305
1,440
1,590

1,320
1,445
1,585
1,740
1,920
2,125
2,360
2,635
2,960

percent is life sustaining atmospheric oxygen. At sea level,
atmospheric pressure is great enough to support normal
growth, activity, and life. By 18,000 feet, the partial pressure
of oxygen is reduced and adversely affects the normal
activities and functions of the human body.
The reactions of the average person become impaired at an
altitude of about 10,000 feet, but for some people impairment
can occur at an altitude as low as 5,000 feet. The physiological
reactions to hypoxia or oxygen deprivation are insidious and
affect people in different ways. These symptoms range from
mild disorientation to total incapacitation, depending on
body tolerance and altitude. Supplemental oxygen or cabin
pressurization systems help pilots fly at higher altitudes and
overcome the effects of oxygen deprivation.

Wind and Currents
Air flows from areas of high pressure into areas of low
pressure because air always seeks out lower pressure. The
combination of atmospheric pressure differences, Coriolis
force, friction, and temperature differences of the air near
the earth cause two kinds of atmospheric motion: convective
currents (upward and downward motion) and wind
(horizontal motion). Currents and winds are important as
they affect takeoff, landing, and cruise flight operations. Most
importantly, currents and winds or atmospheric circulation
cause weather changes.
Wind Patterns
In the Northern Hemisphere, the flow of air from areas of
high to low pressure is deflected to the right and produces
a clockwise circulation around an area of high pressure.
This is known as anticyclonic circulation. The opposite
is true of low-pressure areas; the air flows toward a low
and is deflected to create a counterclockwise or cyclonic
circulation. [Figure 12-10]
High-pressure systems are generally areas of dry, descending
air. Good weather is typically associated with high-pressure
systems for this reason. Conversely, air flows into a lowpressure area to replace rising air. This air usually brings
increasing cloudiness and precipitation. Thus, bad weather
is commonly associated with areas of low pressure.
A good understanding of high- and low-pressure wind patterns
can be of great help when planning a flight because a pilot can
take advantage of beneficial tailwinds. [Figure 12-11] When
planning a flight from west to east, favorable winds would
be encountered along the northern side of a high-pressure
system or the southern side of a low-pressure system. On
the return flight, the most favorable winds would be along
the southern side of the same high-pressure system or the
northern side of a low-pressure system. An added advantage

Figure 12-10. Circulation pattern about areas of high and low

pressure.

is a better understanding of what type of weather to expect
in a given area along a route of flight based on the prevailing
areas of highs and lows.
While the theory of circulation and wind patterns is accurate for
large scale atmospheric circulation, it does not take into account
changes to the circulation on a local scale. Local conditions,
geological features, and other anomalies can change the wind
direction and speed close to the Earth's surface.
Convective Currents
Plowed ground, rocks, sand, and barren land absorb solar
energy quickly and can therefore give off a large amount
of heat; whereas, water, trees, and other areas of vegetation
tend to more slowly absorb heat and give off heat. The
resulting uneven heating of the air creates small areas of
local circulation called convective currents.
Convective currents cause the bumpy, turbulent air sometimes
experienced when flying at lower altitudes during warmer
weather. On a low-altitude flight over varying surfaces,
updrafts are likely to occur over pavement or barren places,
and downdrafts often occur over water or expansive areas
of vegetation like a group of trees. Typically, these turbulent
conditions can be avoided by flying at higher altitudes, even
above cumulus cloud layers. [Figure 12-12]
Convective currents are particularly noticeable in areas with a
land mass directly adjacent to a large body of water, such as an
ocean, large lake, or other appreciable area of water. During
the day, land heats faster than water, so the air over the land
becomes warmer and less dense. It rises and is replaced by
12-7

Figure 12-11. Favorable winds near a high pressure system.

Figure 12-12. Convective turbulence avoidance.

cooler, denser air flowing in from over the water. This causes
an onshore wind called a sea breeze. Conversely, at night land
cools faster than water, as does the corresponding air. In this
case, the warmer air over the water rises and is replaced by
the cooler, denser air from the land, creating an offshore wind
called a land breeze. This reverses the local wind circulation
pattern. Convective currents can occur anywhere there is an
uneven heating of the Earth's surface. [Figure 12-13]
Convective currents close to the ground can affect a pilot's
ability to control the aircraft. For example, on final approach,
the rising air from terrain devoid of vegetation sometimes
produces a ballooning effect that can cause a pilot to overshoot
the intended landing spot. On the other hand, an approach over
a large body of water or an area of thick vegetation tends to
create a sinking effect that can cause an unwary pilot to land
short of the intended landing spot. [Figure 12-14]

12-8

Effect of Obstructions on Wind
Another atmospheric hazard exists that can create problems
for pilots. Obstructions on the ground affect the flow of
wind and can be an unseen danger. Ground topography and
large buildings can break up the flow of the wind and create
wind gusts that change rapidly in direction and speed. These
obstructions range from man-made structures, like hangars,
to large natural obstructions, such as mountains, bluffs, or
canyons. It is especially important to be vigilant when flying
in or out of airports that have large buildings or natural
obstructions located near the runway. [Figure 12-15]
The intensity of the turbulence associated with ground
obstructions depends on the size of the obstacle and the
primary velocity of the wind. This can affect the takeoff and
landing performance of any aircraft and can present a very
serious hazard. During the landing phase of flight, an aircraft

Cool

Warm

Return flow

Sea breeze

Return flow

Cool

Warm

Land breeze

Figure 12-13. Sea breeze and land breeze wind circulation patterns.

Cool
sinking
air

Warm
rising
air
Inte

nde

d Fl

ight

path

Figure 12-14. Currents generated by varying surface conditions.

12-9

WI N

D

Figure 12-15. Turbulence caused by manmade obstructions.

may "drop in" due to the turbulent air and be too low to clear
obstacles during the approach.
This same condition is even more noticeable when flying in
mountainous regions. [Figure 12-16] While the wind flows
smoothly up the windward side of the mountain and the
upward currents help to carry an aircraft over the peak of
the mountain, the wind on the leeward side does not act in

WIND

Figure 12-16. Turbulence in mountainous regions.

12-10

a similar manner. As the air flows down the leeward side of
the mountain, the air follows the contour of the terrain and
is increasingly turbulent. This tends to push an aircraft into
the side of a mountain. The stronger the wind, the greater the
downward pressure and turbulence become.
Due to the effect terrain has on the wind in valleys or canyons,
downdrafts can be severe. Before conducting a flight in or

near mountainous terrain, it is helpful for a pilot unfamiliar
with a mountainous area to get a checkout with a mountain
qualified flight instructor.

and headwind losses of 30–90 knots, seriously degrading
performance. It can also produce strong turbulence and
hazardous wind direction changes. Consider Figure 12-17:
During an inadvertent takeoff into a microburst, the plane
may first experience a performance-increasing headwind
(1), followed by performance-decreasing downdrafts (2),
followed by a rapidly increasing tailwind (3). This can result
in terrain impact or flight dangerously close to the ground (4).
An encounter during approach involves the same sequence
of wind changes and could force the plane to the ground
short of the runway.

Low-Level Wind Shear
Wind shear is a sudden, drastic change in wind speed and/or
direction over a very small area. Wind shear can subject an
aircraft to violent updrafts and downdrafts, as well as abrupt
changes to the horizontal movement of the aircraft. While
wind shear can occur at any altitude, low-level wind shear is
especially hazardous due to the proximity of an aircraft to the
ground. Low-level wind shear is commonly associated with
passing frontal systems, thunderstorms, temperature inversions,
and strong upper level winds (greater than 25 knots).

The FAA has made a substantial investment in microburst
accident prevention. The totally redesigned LLWAS-NE, the
TDWR, and the ASR-9 WSP are skillful microburst alerting
systems installed at major airports. These three systems were
extensively evaluated over a 3-year period. Each was seen
to issue very few false alerts and to detect microbursts well
above the 90 percent detection requirement established by
Congress. Many flights involve airports that lack microburst
alert equipment, so the FAA has also prepared wind shear
training material: Advisory Circular (AC) 00-54, FAA
Pilot Wind Shear Guide. Included is information on how to
recognize the risk of a microburst encounter, how to avoid an
encounter, and the best flight strategy for successful escape
should an encounter occur.

Wind shear is dangerous to an aircraft. It can rapidly change
the performance of the aircraft and disrupt the normal flight
attitude. For example, a tailwind quickly changing to a
headwind causes an increase in airspeed and performance.
Conversely, a headwind changing to a tailwind causes a
decrease in airspeed and performance. In either case, a pilot
must be prepared to react immediately to these changes to
maintain control of the aircraft.
The most severe type of low-level wind shear, a microburst,
is associated with convective precipitation into dry air at
cloud base. Microburst activity may be indicated by an
intense rain shaft at the surface but virga at cloud base
and a ring of blowing dust is often the only visible clue.
A typical microburst has a horizontal diameter of 1–2
miles and a nominal depth of 1,000 feet. The lifespan of a
microburst is about 5–15 minutes during which time it can
produce downdrafts of up to 6,000 feet per minute (fpm)

It is important to remember that wind shear can affect any
flight and any pilot at any altitude. While wind shear may be
reported, it often remains undetected and is a silent danger
to aviation. Always be alert to the possibility of wind shear,
especially when flying in and around thunderstorms and
frontal systems.

Strong downdraft

ded

n
Inte

Inc

rea
sing
headwind

Outflow

1

i
Increasing ta
2

Path

d
lwin

Outflow
3

4

Figure 12-17. Effects of a microburst wind.

12-11

Wind and Pressure Representation on Surface
Weather Maps
Surface weather maps provide information about fronts, areas
of high and low pressure, and surface winds and pressures
for each station. This type of weather map allows pilots to
see the locations of fronts and pressure systems, but more
importantly, it depicts the wind and pressure at the surface
for each location. For more information on surface analysis
and weather depiction charts, see Chapter 13, Aviation
Weather Services.
Wind conditions are reported by an arrow attached to the
station location circle. [Figure 12-18] The station circle
represents the head of the arrow, with the arrow pointing
in the direction from which the wind is blowing. Winds
are described by the direction from which they blow, thus
a northwest wind means that the wind is blowing from the
northwest toward the southeast. The speed of the wind is
depicted by barbs or pennants placed on the wind line. Each
barb represents a speed of ten knots, while half a barb is equal
to five knots, and a pennant is equal to 50 knots.

1028

p
tee
a s ds.
n
a
n
me g wi
ars tron
sob nd s
i
a
d
ce nt
pa adie
s
r
ely re g
os
Cl ssu
e
pr

s
Isobar

1024

s
Isobar

1020

1016

rs
oba re
d is ressu
e
c
a
p
sp
Widely hallow tively
s
la
a
e
r
n
a
me
and
ient
grad nds.
i
w
light

1012

1008

L

Figure 12-19. Isobars reveal the pressure gradient of an area of

high- or low-pressure areas.

The pressure for each station is recorded on the weather chart
and is shown in mb. Isobars are lines drawn on the chart to
depict lines of equal pressure. These lines result in a pattern
that reveals the pressure gradient or change in pressure over
distance. [Figure 12-19] Isobars are similar to contour lines
on a topographic map that indicate terrain altitudes and
slope steepness. For example, isobars that are closely spaced
indicate a steep pressure gradient and strong winds prevail.
Shallow gradients, on the other hand, are represented by
isobars that are spaced far apart and are indicative of light
winds. Isobars help identify low- and high-pressure systems,
as well as the location of ridges and troughs. A high is an
area of high pressure surrounded by lower pressure; a low
is an area of low pressure surrounded by higher pressure. A
ridge is an elongated area of high pressure, and a trough is
an elongated area of low pressure.
Isobars furnish valuable information about winds in the first
few thousand feet above the surface. Close to the ground,
Calm

E /35 kts

NW / 5 kts

N / 50 kts

SW / 20 kts

W / 105 kts

Figure 12-18. Depiction of winds on a surface weather chart.

12-12

wind direction is modified by the friction and wind speed
decreases due to friction with the surface. At levels 2,000 to
3,000 feet above the surface, however, the speed is greater
and the direction becomes more parallel to the isobars.
Generally, the wind 2,000 feet above ground level (AGL) is
20° to 40° to the right of surface winds, and the wind speed is
greater. The change of wind direction is greatest over rough
terrain and least over flat surfaces, such as open water. In the
absence of winds aloft information, this rule of thumb allows
for a rough estimate of the wind conditions a few thousand
feet above the surface.

Atmospheric Stability
The stability of the atmosphere depends on its ability to
resist vertical motion. A stable atmosphere makes vertical
movement difficult, and small vertical disturbances dampen
out and disappear. In an unstable atmosphere, small vertical air
movements tend to become larger, resulting in turbulent airflow
and convective activity. Instability can lead to significant
turbulence, extensive vertical clouds, and severe weather.
Rising air expands and cools due to the decrease in air
pressure as altitude increases. The opposite is true of
descending air; as atmospheric pressure increases, the
temperature of descending air increases as it is compressed.
Adiabatic heating and adiabatic cooling are terms used to
describe this temperature change.

The adiabatic process takes place in all upward and
downward moving air. When air rises into an area of lower
pressure, it expands to a larger volume. As the molecules
of air expand, the temperature of the air lowers. As a result,
when a parcel of air rises, pressure decreases, volume
increases, and temperature decreases. When air descends,
the opposite is true. The rate at which temperature decreases
with an increase in altitude is referred to as its lapse rate.
As air ascends through the atmosphere, the average rate of
temperature change is 2 °C (3.5 °F) per 1,000 feet.
Since water vapor is lighter than air, moisture decreases air
density, causing it to rise. Conversely, as moisture decreases,
air becomes denser and tends to sink. Since moist air cools
at a slower rate, it is generally less stable than dry air since
the moist air must rise higher before its temperature cools
to that of the surrounding air. The dry adiabatic lapse rate
(unsaturated air) is 3 °C (5.4 °F) per 1,000 feet. The moist
adiabatic lapse rate varies from 1.1 °C to 2.8 °C (2 °F to
5 °F) per 1,000 feet.
The combination of moisture and temperature determine the
stability of the air and the resulting weather. Cool, dry air
is very stable and resists vertical movement, which leads to
good and generally clear weather. The greatest instability
occurs when the air is moist and warm, as it is in the tropical
regions in the summer. Typically, thunderstorms appear on
a daily basis in these regions due to the instability of the
surrounding air.
Inversion
As air rises and expands in the atmosphere, the temperature
decreases. There is an atmospheric anomaly that can occur;
however, that changes this typical pattern of atmospheric
behavior. When the temperature of the air rises with altitude, a
temperature inversion exists. Inversion layers are commonly
shallow layers of smooth, stable air close to the ground. The
temperature of the air increases with altitude to a certain
point, which is the top of the inversion. The air at the top
of the layer acts as a lid, keeping weather and pollutants
trapped below. If the relative humidity of the air is high, it
can contribute to the formation of clouds, fog, haze, or smoke
resulting in diminished visibility in the inversion layer.
Surface-based temperature inversions occur on clear, cool
nights when the air close to the ground is cooled by the
lowering temperature of the ground. The air within a few
hundred feet of the surface becomes cooler than the air above
it. Frontal inversions occur when warm air spreads over a
layer of cooler air, or cooler air is forced under a layer of
warmer air.

Moisture and Temperature
The atmosphere, by nature, contains moisture in the form
of water vapor. The amount of moisture present in the
atmosphere is dependent upon the temperature of the air.
Every 20 °F increase in temperature doubles the amount of
moisture the air can hold. Conversely, a decrease of 20 °F
cuts the capacity in half.
Water is present in the atmosphere in three states: liquid,
solid, and gaseous. All three forms can readily change to
another, and all are present within the temperature ranges of
the atmosphere. As water changes from one state to another,
an exchange of heat takes place. These changes occur through
the processes of evaporation, sublimation, condensation,
deposition, melting, or freezing. However, water vapor
is added into the atmosphere only by the processes of
evaporation and sublimation.
Evaporation is the changing of liquid water to water vapor.
As water vapor forms, it absorbs heat from the nearest
available source. This heat exchange is known as the latent
heat of evaporation. A good example is the evaporation of
human perspiration. The net effect is a cooling sensation
as heat is extracted from the body. Similarly, sublimation
is the changing of ice directly to water vapor, completely
bypassing the liquid stage. Though dry ice is not made of
water, but rather carbon dioxide, it demonstrates the principle
of sublimation when a solid turns directly into vapor.
Relative Humidity
Humidity refers to the amount of water vapor present in the
atmosphere at a given time. Relative humidity is the actual
amount of moisture in the air compared to the total amount of
moisture the air could hold at that temperature. For example,
if the current relative humidity is 65 percent, the air is
holding 65 percent of the total amount of moisture that it is
capable of holding at that temperature and pressure. While
much of the western United States rarely sees days of high
humidity, relative humidity readings of 75 to 90 percent are
not uncommon in the southern United States during warmer
months. [Figure 12-20]
Temperature/Dew Point Relationship
The relationship between dew point and temperature defines
the concept of relative humidity. The dew point, given in
degrees, is the temperature at which the air can hold no
more moisture. When the temperature of the air is reduced
to the dew point, the air is completely saturated and moisture
begins to condense out of the air in the form of fog, dew,
frost, clouds, rain, or snow.

12-13

At sea level pressure, air can hold
9 g H2O/cubic meter of air at 10 °C
17 g H2O/cubic meter of air at 20 °C
30 g H2O/cubic meter of air at 30 °C

If the temperature is lowered to 10 °C,
the air can hold only 9 g of water
vapor, and 8 g of water will condense
as water droplets. The relative
humidity will still be at 100\%.

If the same cubic meter of air warms
to 30 °C, the 17 g of water vapor will
produce a relative humidity of 56\%.
(17 g is 56\% of the 30 g the air could
hold at this temperature.)
A cubic meter of air with 17g of water
vapor at 20 °C is at saturation
or 100\% relative humidity. Any further
cooling will cause condensation (fog,
clouds, dew) to form. Thus, 20 °C is
the dew point for this situation.

Figure 12-20. Relationship between relative humidity, temperature, and dewpoint.

As moist, unstable air rises, clouds often form at the altitude
where temperature and dew point reach the same value. When
lifted, unsaturated air cools at a rate of 5.4 °F per 1,000 feet
and the dew point temperature decreases at a rate of 1 °F per
1,000 feet. This results in a convergence of temperature and
dew point at a rate of 4.4 °F. Apply the convergence rate
to the reported temperature and dew point to determine the
height of the cloud base.
Given:
Temperature (T) = 85 °F
Dew point (DP) = 71 °F
Convergence Rate (CR) = 4.4°
T – DP = Temperature Dew Point Spread (TDS)
TDS ÷ CR = X
X × 1,000 feet = height of cloud base AGL
Example:
85 °F – 71 °F = 14 °F
14 °F ÷ 4.4 °F = 3.18
3.18 × 1,000 = 3,180 feet AGL

The height of the cloud base is 3,180 feet AGL.


12-14

Explanation:
With an outside air temperature (OAT) of 85 °F at the surface
and dew point at the surface of 71 °F, the spread is 14°. Divide
the temperature dew point spread by the convergence rate of
4.4 °F, and multiply by 1,000 to determine the approximate
height of the cloud base.
Methods by Which Air Reaches the Saturation Point
If air reaches the saturation point while temperature and
dew point are close together, it is highly likely that fog, low
clouds, and precipitation will form. There are four methods
by which air can reach the saturation point. First, when warm
air moves over a cold surface, the air temperature drops and
reaches the saturation point. Second, the saturation point may
be reached when cold air and warm air mix. Third, when air
cools at night through contact with the cooler ground, air
reaches its saturation point. The fourth method occurs when
air is lifted or is forced upward in the atmosphere.
As air rises, it uses heat energy to expand. As a result, the rising
air loses heat rapidly. Unsaturated air loses heat at a rate of
3.0 °C (5.4 °F) for every 1,000 feet of altitude gain. No matter
what causes the air to reach its saturation point, saturated air
brings clouds, rain, and other critical weather situations.

Dew and Frost
On cool, clear, calm nights, the temperature of the ground
and objects on the surface can cause temperatures of the
surrounding air to drop below the dew point. When this
occurs, the moisture in the air condenses and deposits itself on
the ground, buildings, and other objects like cars and aircraft.
This moisture is known as dew and sometimes can be seen
on grass and other objects in the morning. If the temperature
is below freezing, the moisture is deposited in the form of
frost. While dew poses no threat to an aircraft, frost poses a
definite flight safety hazard. Frost disrupts the flow of air over
the wing and can drastically reduce the production of lift. It
also increases drag, which when combined with lowered lift
production, can adversely affect the ability to take off. An
aircraft must be thoroughly cleaned and free of frost prior
to beginning a flight.
Fog
Fog is a cloud that is on the surface. It typically occurs when
the temperature of air near the ground is cooled to the air's
dew point. At this point, water vapor in the air condenses and
becomes visible in the form of fog. Fog is classified according
to the manner in which it forms and is dependent upon the
current temperature and the amount of water vapor in the air.
On clear nights, with relatively little to no wind present,
radiation fog may develop. [Figure 12-21] Usually, it forms
in low-lying areas like mountain valleys. This type of fog
occurs when the ground cools rapidly due to terrestrial
radiation, and the surrounding air temperature reaches its
dew point. As the sun rises and the temperature increases,
radiation fog lifts and eventually burns off. Any increase in
wind also speeds the dissipation of radiation fog. If radiation
fog is less than 20 feet thick, it is known as ground fog.
When a layer of warm, moist air moves over a cold surface,
advection fog is likely to occur. Unlike radiation fog, wind
is required to form advection fog. Winds of up to 15 knots

allow the fog to form and intensify; above a speed of 15 knots,
the fog usually lifts and forms low stratus clouds. Advection
fog is common in coastal areas where sea breezes can blow
the air over cooler landmasses.
Upslope fog occurs when moist, stable air is forced up sloping
land features like a mountain range. This type of fog also
requires wind for formation and continued existence. Upslope
and advection fog, unlike radiation fog, may not burn off with
the morning sun but instead can persist for days. They can
also extend to greater heights than radiation fog.
Steam fog, or sea smoke, forms when cold, dry air moves over
warm water. As the water evaporates, it rises and resembles
smoke. This type of fog is common over bodies of water
during the coldest times of the year. Low-level turbulence
and icing are commonly associated with steam fog.
Ice fog occurs in cold weather when the temperature is
much below freezing and water vapor forms directly into ice
crystals. Conditions favorable for its formation are the same
as for radiation fog except for cold temperature, usually –25
°F or colder. It occurs mostly in the arctic regions but is not
unknown in middle latitudes during the cold season.
Clouds
Clouds are visible indicators and are often indicative of
future weather. For clouds to form, there must be adequate
water vapor and condensation nuclei, as well as a method by
which the air can be cooled. When the air cools and reaches
its saturation point, the invisible water vapor changes into
a visible state. Through the processes of deposition (also
referred to as sublimation) and condensation, moisture
condenses or sublimates onto miniscule particles of matter
like dust, salt, and smoke known as condensation nuclei. The
nuclei are important because they provide a means for the
moisture to change from one state to another.
Cloud type is determined by its height, shape, and
characteristics. They are classified according to the height of
their bases as low, middle, or high clouds, as well as clouds
with vertical development. [Figure 12-22]
Low clouds are those that form near the Earth's surface and
extend up to about 6,500 feet AGL. They are made primarily
of water droplets but can include supercooled water droplets
that induce hazardous aircraft icing. Typical low clouds
are stratus, stratocumulus, and nimbostratus. Fog is also
classified as a type of low cloud formation. Clouds in this
family create low ceilings, hamper visibility, and can change
rapidly. Because of this, they influence flight planning and
can make visual flight rules (VFR) flight impossible.

Figure 12-21. Radiation fog.

12-15

Cirrocumulus

Cirrus

High clouds

Cirrostratus

Cumulonimbus

Altostratus

Altocumulus

Middle clouds

20,000 AGL

Clouds with vertical development

Stratocumulus
Stratus

Nimbostratus

Low clouds

6,500 AGL

Towering Cumulus

Figure 12-22. Basic cloud types.

Middle clouds form around 6,500 feet AGL and extend up to
20,000 feet AGL. They are composed of water, ice crystals,
and supercooled water droplets. Typical middle-level clouds
include altostratus and altocumulus. These types of clouds
may be encountered on cross-country flights at higher
altitudes. Altostratus clouds can produce turbulence and may
contain moderate icing. Altocumulus clouds, which usually
form when altostratus clouds are breaking apart, also may
contain light turbulence and icing.
High clouds form above 20,000 feet AGL and usually form
only in stable air. They are made up of ice crystals and pose
no real threat of turbulence or aircraft icing. Typical high
level clouds are cirrus, cirrostratus, and cirrocumulus.
Clouds with extensive vertical development are cumulus
clouds that build vertically into towering cumulus or
cumulonimbus clouds. The bases of these clouds form in
the low to middle cloud base region but can extend into high
altitude cloud levels. Towering cumulus clouds indicate areas
of instability in the atmosphere, and the air around and inside
them is turbulent. These types of clouds often develop into
cumulonimbus clouds or thunderstorms. Cumulonimbus
clouds contain large amounts of moisture and unstable air

12-16

and usually produce hazardous weather phenomena, such
as lightning, hail, tornadoes, gusty winds, and wind shear.
These extensive vertical clouds can be obscured by other
cloud formations and are not always visible from the ground
or while in flight. When this happens, these clouds are said
to be embedded, hence the term, embedded thunderstorms.
To pilots, the cumulonimbus cloud is perhaps the most
dangerous cloud type. It appears individually or in groups and
is known as either an air mass or orographic thunderstorm.
Heating of the air near the Earth's surface creates an air mass
thunderstorm; the upslope motion of air in the mountainous
regions causes orographic thunderstorms. Cumulonimbus
clouds that form in a continuous line are nonfrontal bands
of thunderstorms or squall lines.
Since rising air currents cause cumulonimbus clouds, they
are extremely turbulent and pose a significant hazard to flight
safety. For example, if an aircraft enters a thunderstorm,
the aircraft could experience updrafts and downdrafts that
exceed 3,000 fpm. In addition, thunderstorms can produce
large hailstones, damaging lightning, tornadoes, and large
quantities of water, all of which are potentially hazardous
to aircraft.

Cloud classification can be further broken down into specific
cloud types according to the outward appearance and cloud
composition. Knowing these terms can help a pilot identify
visible clouds.
The following is a list of cloud classifications:


Cumulus—heaped or piled clouds



Stratus—formed in layers



Cirrus—ringlets, fibrous clouds, also high level clouds
above 20,000 feet



Castellanus—common base with separate vertical
development, castle-like



Lenticularus—lens-shaped, formed over mountains
in strong winds



Nimbus—rain-bearing clouds



Fracto—ragged or broken



Alto—middle level clouds existing at 5,000 to 20,000
feet

Ceiling
For aviation purposes, a ceiling is the lowest layer of clouds
reported as being broken or overcast, or the vertical visibility
into an obscuration like fog or haze. Clouds are reported
as broken when five-eighths to seven-eighths of the sky is
covered with clouds. Overcast means the entire sky is covered
with clouds. Current ceiling information is reported by the
aviation routine weather report (METAR) and automated
weather stations of various types.
Visibility
Closely related to cloud cover and reported ceilings is
visibility information. Visibility refers to the greatest
horizontal distance at which prominent objects can be
viewed with the naked eye. Current visibility is also reported
in METAR and other aviation weather reports, as well as
by automated weather systems. Visibility information, as
predicted by meteorologists, is available for a pilot during a
preflight weather briefing.
Precipitation
Precipitation refers to any type of water particles that
form in the atmosphere and fall to the ground. It has a
profound impact on flight safety. Depending on the form of
precipitation, it can reduce visibility, create icing situations,
and affect landing and takeoff performance of an aircraft.
Precipitation occurs because water or ice particles in clouds
grow in size until the atmosphere can no longer support
them. It can occur in several forms as it falls toward the
Earth, including drizzle, rain, ice pellets, hail, snow, and ice.

Drizzle is classified as very small water droplets, smaller
than 0.02 inches in diameter. Drizzle usually accompanies
fog or low stratus clouds. Water droplets of larger size are
referred to as rain. Rain that falls through the atmosphere but
evaporates prior to striking the ground is known as virga.
Freezing rain and freezing drizzle occur when the temperature
of the surface is below freezing; the rain freezes on contact
with the cooler surface.
If rain falls through a temperature inversion, it may freeze
as it passes through the underlying cold air and fall to the
ground in the form of ice pellets. Ice pellets are an indication
of a temperature inversion and that freezing rain exists at a
higher altitude. In the case of hail, freezing water droplets are
carried up and down by drafts inside cumulonimbus clouds,
growing larger in size as they come in contact with more
moisture. Once the updrafts can no longer hold the freezing
water, it falls to the Earth in the form of hail. Hail can be
pea sized, or it can grow as large as five inches in diameter,
larger than a softball.
Snow is precipitation in the form of ice crystals that falls
at a steady rate or in snow showers that begin, change in
intensity, and end rapidly. Snow also varies in size, from very
small grains to large flakes. Snow grains are the equivalent
of drizzle in size.
Precipitation in any form poses a threat to safety of flight.
Often, precipitation is accompanied by low ceilings and
reduced visibility. Aircraft that have ice, snow, or frost on
their surfaces must be carefully cleaned prior to beginning
a flight because of the possible airflow disruption and
loss of lift. Rain can contribute to water in the fuel tanks.
Precipitation can create hazards on the runway surface itself,
making takeoffs and landings difficult, if not impossible,
due to snow, ice, or pooling water and very slick surfaces.

Air Masses
Air masses are classified according to the regions where
they originate. They are large bodies of air that take on the
characteristics of the surrounding area or source region. A
source region is typically an area in which the air remains
relatively stagnant for a period of days or longer. During
this time of stagnation, the air mass takes on the temperature
and moisture characteristics of the source region. Areas of
stagnation can be found in polar regions, tropical oceans, and
dry deserts. Air masses are generally identified as polar or
tropical based on temperature characteristics and maritime
or continental based on moisture content.
A continental polar air mass forms over a polar region and
brings cool, dry air with it. Maritime tropical air masses form

12-17

over warm tropical waters like the Caribbean Sea and bring
warm, moist air. As the air mass moves from its source region
and passes over land or water, the air mass is subjected to
the varying conditions of the land or water which modify the
nature of the air mass. [Figure 12-23]
An air mass passing over a warmer surface is warmed from
below, and convective currents form, causing the air to rise.
This creates an unstable air mass with good surface visibility.
Moist, unstable air causes cumulus clouds, showers, and
turbulence to form.
Conversely, an air mass passing over a colder surface does not
form convective currents but instead creates a stable air mass
with poor surface visibility. The poor surface visibility is due
to the fact that smoke, dust, and other particles cannot rise
out of the air mass and are instead trapped near the surface.
A stable air mass can produce low stratus clouds and fog.

Fronts
As an air mass moves across bodies of water and land, it
eventually comes in contact with another air mass with
different characteristics. The boundary layer between two
types of air masses is known as a front. An approaching
front of any type always means changes to the weather
are imminent.

A
cP

There are four types of fronts that are named according to the
temperature of the advancing air relative to the temperature
of the air it is replacing: [Figure 12-24]


Warm



Cold



Stationary



Occluded

Any discussion of frontal systems must be tempered with
the knowledge that no two fronts are the same. However,
generalized weather conditions are associated with a specific
type of front that helps identify the front.
Warm Front
A warm front occurs when a warm mass of air advances and
replaces a body of colder air. Warm fronts move slowly,
typically 10 to 25 miles per hour (mph). The slope of the
advancing front slides over the top of the cooler air and
gradually pushes it out of the area. Warm fronts contain
warm air that often has very high humidity. As the warm
air is lifted, the temperature drops and condensation occurs.
Generally, prior to the passage of a warm front, cirriform
or stratiform clouds, along with fog, can be expected to
form along the frontal boundary. In the summer months,
cumulonimbus clouds (thunderstorms) are likely to develop.

Standard air mass abbreviations: arctic (A), continental polar (cP),
maritime polar (mP), continental tropical (cT), and maritime tropical
(mT).

mP

mP
mT
mT
cT

Figure 12-23. North American air mass source regions.

12-18

mT

Symbols for surface fronts and other significant lines
shown on the surface analysis chart
Warm front (red)*
Cold front (blue)*
Stationary front (red/blue)*
Occluded front (purple)*
* Note: Fronts may be black and white or color depending on their
source. Also, fronts shown in color code do not necessarily show
frontal symbols.

Figure 12-24. Common chart symbology to depict weather front

location.

Light to moderate precipitation is probable, usually in the
form of rain, sleet, snow, or drizzle, accentuated by poor
visibility. The wind blows from the south-southeast, and the
outside temperature is cool or cold with an increasing dew
point. Finally, as the warm front approaches, the barometric
pressure continues to fall until the front passes completely.
During the passage of a warm front, stratiform clouds are
visible and drizzle may be falling. The visibility is generally
poor, but improves with variable winds. The temperature rises

steadily from the inflow of relatively warmer air. For the most
part, the dew point remains steady and the pressure levels off.
After the passage of a warm front, stratocumulus clouds
predominate and rain showers are possible. The visibility
eventually improves, but hazy conditions may exist for a
short period after passage. The wind blows from the southsouthwest. With warming temperatures, the dew point
rises and then levels off. There is generally a slight rise in
barometric pressure, followed by a decrease of barometric
pressure.

Flight Toward an Approaching Warm Front
By studying a typical warm front, much can be learned
about the general patterns and atmospheric conditions
that exist when a warm front is encountered in flight.
Figure 12-25 depicts a warm front advancing eastward from
St. Louis, Missouri, toward Pittsburgh, Pennsylvania during
a flight from Pittsburgh to St. Louis.
At the time of departure from Pittsburgh, the weather is good
VFR with a scattered layer of cirrus clouds at 15,000 feet.
As the flight progresses westward to Columbus and closer
to the oncoming warm front, the clouds deepen and become
increasingly stratiform in appearance with a ceiling of 6,000
feet. The visibility decreases to six miles in haze with a falling
CIRRUS
CIRROSTRATUS

IR
MA

ALTOSTRATUS

WAR

COLD AIR

NIMBOSTRATUS

999

1002

1005

Columbus

Indianapolis

St. Louis

200 miles

1008

1011

METAR

KSTL
0VC010

1950Z

21018KT
18/18

1SM
A2960

–RA

METAR

KIND
BKN020

1950Z

16012KT
15/15

3SM
A2973

RA

METAR

KCMH
0VC060

1950Z

13018KT
14/10

6SM
A2990

HZ

METAR

KPIT
SCT150

1950Z

13012KT
12/01

10SM
A3002

999

1002

59
3
59

020
10
10 St. Louis

068
40
20

56
6
50

60

125
26

53
10
34

Columbus

Indianapolis

1011

1014

166
18

Pittsburgh

1005
1008

600 miles

1014
1017

65
1
65

Pittsburgh

400 miles

1017

Figure 12-25. Warm front cross-section with surface weather chart depiction and associated METAR.

12-19

barometric pressure. Approaching Indianapolis, the weather
deteriorates to broken clouds at 2,000 feet with three miles
visibility and rain. With the temperature and dew point the
same, fog is likely to develop. At St. Louis, the sky is overcast
with low clouds and drizzle and the visibility is one mile.
Beyond Indianapolis, the ceiling and visibility are too low
to continue VFR. Therefore, it would be wise to remain in
Indianapolis until the warm front passes, which may take
up to two days.
Cold Front
A cold front occurs when a mass of cold, dense, and stable
air advances and replaces a body of warmer air.
Cold fronts move more rapidly than warm fronts, progressing
at a rate of 25 to 30 mph. However, extreme cold fronts
have been recorded moving at speeds of up to 60 mph.
A typical cold front moves in a manner opposite that of a
warm front. It is so dense, it stays close to the ground and
acts like a snowplow, sliding under the warmer air and
forcing the less dense air aloft. The rapidly ascending air
causes the temperature to decrease suddenly, forcing the
creation of clouds. The type of clouds that form depends
on the stability of the warmer air mass. A cold front in the
Northern Hemisphere is normally oriented in a northeast to
southwest manner and can be several hundred miles long,
encompassing a large area of land.
Prior to the passage of a typical cold front, cirriform or
towering cumulus clouds are present, and cumulonimbus
clouds may develop. Rain showers may also develop due
to the rapid development of clouds. A high dew point and
falling barometric pressure are indicative of imminent cold
front passage.
As the cold front passes, towering cumulus or cumulonimbus
clouds continue to dominate the sky. Depending on the
intensity of the cold front, heavy rain showers form and may
be accompanied by lightning, thunder, and/or hail. More
severe cold fronts can also produce tornadoes. During cold
front passage, the visibility is poor with winds variable and
gusty, and the temperature and dew point drop rapidly. A
quickly falling barometric pressure bottoms out during frontal
passage, then begins a gradual increase.
After frontal passage, the towering cumulus and
cumulonimbus clouds begin to dissipate to cumulus clouds
with a corresponding decrease in the precipitation. Good
visibility eventually prevails with the winds from the westnorthwest. Temperatures remain cooler and the barometric
pressure continues to rise.

12-20

Fast-Moving Cold Front
Fast-moving cold fronts are pushed by intense pressure
systems far behind the actual front. The friction between
the ground and the cold front retards the movement of the
front and creates a steeper frontal surface. This results in a
very narrow band of weather, concentrated along the leading
edge of the front. If the warm air being overtaken by the
cold front is relatively stable, overcast skies and rain may
occur for some distance behind the front. If the warm air
is unstable, scattered thunderstorms and rain showers may
form. A continuous line of thunderstorms, or squall line,
may form along or ahead of the front. Squall lines present
a serious hazard to pilots as squall-type thunderstorms are
intense and move quickly. Behind a fast-moving cold front,
the skies usually clear rapidly, and the front leaves behind
gusty, turbulent winds and colder temperatures.

Flight Toward an Approaching Cold Front
Like warm fronts, not all cold fronts are the same. Examining
a flight toward an approaching cold front, pilots can get a
better understanding of the type of conditions that can be
encountered in flight. Figure 12-26 depicts a flight from
Pittsburgh, Pennsylvania, toward St. Louis, Missouri.
At the time of departure from Pittsburgh, the weather is VFR
with three miles visibility in smoke and a scattered layer of
clouds at 3,500 feet. As the flight progresses westward to
Columbus and closer to the oncoming cold front, the clouds
show signs of vertical development with a broken layer at
2,500 feet. The visibility is six miles in haze with a falling
barometric pressure. Approaching Indianapolis, the weather
has deteriorated to overcast clouds at 1,000 feet and three
miles visibility with thunderstorms and heavy rain showers.
At St. Louis, the weather gets better with scattered clouds at
1,000 feet and a ten mile visibility.
A pilot using sound judgment based on the knowledge of
frontal conditions will likely remain in Indianapolis until the
front has passed. Trying to fly below a line of thunderstorms
or a squall line is hazardous, and flight over the top of or
around the storm is not an option. Thunderstorms can extend
up to well over the capability of small airplanes and can
extend in a line for 300 to 500 miles.
Comparison of Cold and Warm Fronts
Warm fronts and cold fronts are very different in nature as are
the hazards associated with each front. They vary in speed,
composition, weather phenomenon, and prediction. Cold fronts,
which move at 20 to 35 mph, travel faster than warm fronts,
which move at only 10 to 25 mph. Cold fronts also possess a

CO

LD

A
I

WARM AIR

R
CUMULONIMBUS

1008

1005

1005

Columbus

Indianapolis

St. Louis

200 miles

46
10
33

071

74
3
71

066
42

4
10

77
6
73

102
8

75
3
70

25
10

Indianapolis

St. Louis

122
12
35

Columbus

Pittsburgh
1014

1011

1011

600 miles

1008
1011

1011

Pittsburgh

400 miles

METAR

KSTL
SCT010

1950Z

30018KT
08/02

10SM
A2979

METAR

KIND
OVC010

1950Z

20024KT
24/23

3SM
A2974

+TSRA

METAR

KCMH
BKN025

1950Z

20012KT
25/24

6SM
A2983

HZ

METAR

KPIT
SCT035

1950Z

20012KT
24/22

3SM
A2989

FU

1014

Figure 12-26. Cold front cross-section with surface weather chart depiction and associated METAR.

steeper frontal slope. Violent weather activity is associated with
cold fronts, and the weather usually occurs along the frontal
boundary, not in advance. However, squall lines can form
during the summer months as far as 200 miles in advance of
a strong cold front. Whereas warm fronts bring low ceilings,
poor visibility, and rain, cold fronts bring sudden storms, gusty
winds, turbulence, and sometimes hail or tornadoes.

Stationary Front
When the forces of two air masses are relatively equal, the
boundary or front that separates them remains stationary and
influences the local weather for days. This front is called a
stationary front. The weather associated with a stationary
front is typically a mixture that can be found in both warm
and cold fronts.

Cold fronts are fast approaching with little or no warning,
and they bring about a complete weather change in just a
few hours. The weather clears rapidly after passage and drier
air with unlimited visibilities prevail. Warm fronts, on the
other hand, provide advance warning of their approach and
can take days to pass through a region.

Occluded Front
An occluded front occurs when a fast-moving cold front
catches up with a slow-moving warm front. As the occluded
front approaches, warm front weather prevails but is
immediately followed by cold front weather. There are two
types of occluded fronts that can occur, and the temperatures
of the colliding frontal systems play a large part in defining
the type of front and the resulting weather. A cold front
occlusion occurs when a fast moving cold front is colder
than the air ahead of the slow moving warm front. When
this occurs, the cold air replaces the cool air and forces the
warm front aloft into the atmosphere. Typically, the cold
front occlusion creates a mixture of weather found in both
warm and cold fronts, providing the air is relatively stable.
A warm front occlusion occurs when the air ahead of the

Wind Shifts
Wind around a high-pressure system rotates clockwise, while
low-pressure winds rotate counter-clockwise. When two
high pressure systems are adjacent, the winds are almost in
direct opposition to each other at the point of contact. Fronts
are the boundaries between two areas of high pressure, and
therefore, wind shifts are continually occurring within a front.
Shifting wind direction is most pronounced in conjunction
with cold fronts.

12-21

warm front is colder than the air of the cold front. When this
is the case, the cold front rides up and over the warm front. If
the air forced aloft by the warm front occlusion is unstable,
the weather is more severe than the weather found in a cold
front occlusion. Embedded thunderstorms, rain, and fog are
likely to occur.
Figure 12-27 depicts a cross-section of a typical cold
front occlusion. The warm front slopes over the prevailing
cooler air and produces the warm front type weather. Prior
to the passage of the typical occluded front, cirriform and
stratiform clouds prevail, light to heavy precipitation falls,
visibility is poor, dew point is steady, and barometric pressure
drops. During the passage of the front, nimbostratus and
cumulonimbus clouds predominate, and towering cumulus
clouds may also form. Light to heavy precipitation falls,
visibility is poor, winds are variable, and the barometric
pressure levels off. After the passage of the front, nimbostratus
and altostratus clouds are visible, precipitation decreases, and
visibility improves.
Thunderstorms
A thunderstorm makes its way through three distinct stages
before dissipating. It begins with the cumulus stage, in
which lifting action of the air begins. If sufficient moisture

and instability are present, the clouds continue to increase
in vertical height. Continuous, strong updrafts prohibit
moisture from falling. Within approximately 15 minutes,
the thunderstorm reaches the mature stage, which is the most
violent time period of the thunderstorm's life cycle. At this
point, drops of moisture, whether rain or ice, are too heavy
for the cloud to support and begin falling in the form of rain
or hail. This creates a downward motion of the air. Warm,
rising air; cool, precipitation-induced descending air; and
violent turbulence all exist within and near the cloud. Below
the cloud, the down-rushing air increases surface winds and
decreases the temperature. Once the vertical motion near the
top of the cloud slows down, the top of the cloud spreads
out and takes on an anvil-like shape. At this point, the storm
enters the dissipating stage. This is when the downdrafts
spread out and replace the updrafts needed to sustain the
storm. [Figure 12-28]
It is impossible to fly over thunderstorms in light aircraft.
Severe thunderstorms can punch through the tropopause and
reach staggering heights of 50,000 to 60,000 feet depending
on latitude. Flying under thunderstorms can subject aircraft
to rain, hail, damaging lightning, and violent turbulence.
A good rule of thumb is to circumnavigate thunderstorms
identified as severe or giving an extreme radar echo by at
CIRRUS

WARM AIR

CUMULONIMBUS

CIRROSTRATUS

ALTOSTRATUS

COL
D

NIMBOSTRATUS

COLD AIR

AIR

1006 1005 1002 999

8
26

42

076
32

200 miles

999 1002 1005 1006

1

1011

52
2
51

2

62

2

Indianapolis

St. Louis

1011

1014

1017

1014

142
34

Columbus

1017

Pittsburgh

400 miles

600 miles

1020

1023

058
8

66

Columbus

Indianapolis

St. Louis

METAR

KSTL
SCT035

1950Z

31023G40KT 8SM
05/M03
A2976

METAR

KIND
VV005

1950Z

29028G45KT 1/2SM
18/16
A2970

TSRAGR

METAR

KCMH
OVC080

1950Z

16017KT
11/10

2SM
A2970

BR

METAR

KPIT
BKN130

1950Z

13012KT
08/04

75SM
A3012

200
20

47
7
40

Pittsburgh

1020

1023

Figure 12-27. Occluded front cross-section with a weather chart depiction and associated METAR.

12-22

Cumulus Stage (3–5 mile height)

Mature Stage (5–10 mile height)

Dissipating Stage (5–7 mile height)

40,000 ft.

Equilibrium level
30,000 ft.

20,000 ft.
0 °C

32 °F
10,000 ft.
5,000 ft.

Figure 12-28. Life cycle of a thunderstorm.

least 20 nautical miles (NM) since hail may fall for miles
outside of the clouds. If flying around a thunderstorm is not
an option, stay on the ground until it passes.
For a thunderstorm to form, the air must have sufficient water
vapor, an unstable lapse rate, and an initial lifting action to
start the storm process. Some storms occur at random in
unstable air, last for only an hour or two, and produce only
moderate wind gusts and rainfall. These are known as air
mass thunderstorms and are generally a result of surface
heating. Steady-state thunderstorms are associated with
weather systems. Fronts, converging winds, and troughs
aloft force upward motion spawning these storms that often
form into squall lines. In the mature stage, updrafts become
stronger and last much longer than in air mass storms, hence
the name steady state. [Figure 12-29]
Knowledge of thunderstorms and the hazards associated with
them is critical to the safety of flight.

Hazards
All thunderstorms have conditions that are a hazard to aviation.
These hazards occur in numerous combinations. While not
every thunderstorm contains all hazards, it is not possible to
visually determine which hazards a thunderstorm contains.

Squall Line
A squall line is a narrow band of active thunderstorms. Often
it develops on or ahead of a cold front in moist, unstable
air, but it may develop in unstable air far removed from
any front. The line may be too long to detour easily and too
wide and severe to penetrate. It often contains steady-state
thunderstorms and presents the single most intense weather
hazard to aircraft. It usually forms rapidly, generally reaching
maximum intensity during the late afternoon and the first
few hours of darkness.

Tornadoes
The most violent thunderstorms draw air into their cloud
bases with great vigor. If the incoming air has any initial
rotating motion, it often forms an extremely concentrated
vortex from the surface well into the cloud. Meteorologists
have estimated that wind in such a vortex can exceed 200
knots with pressure inside the vortex quite low. The strong
winds gather dust and debris and the low pressure generates
a funnel-shaped cloud extending downward from the
cumulonimbus base. If the cloud does not reach the surface,
it is a funnel cloud; if it touches a land surface, it is a tornado;
and if it touches water, it is a "waterspout."

12-23

Turbulence

Anvil

Storm movement

Roll cloud
Wind shear turbulence

Wind shear turbulance

First gust
Dust

Figure 12-29. Movement and turbulence of a maturing thunderstorm.

Tornadoes occur with both isolated and squall line
thunderstorms. Reports for forecasts of tornadoes indicate
that atmospheric conditions are favorable for violent
turbulence. An aircraft entering a tornado vortex is almost
certain to suffer loss of control and structural damage. Since
the vortex extends well into the cloud, any pilot inadvertently
caught on instruments in a severe thunderstorm could
encounter a hidden vortex.
Families of tornadoes have been observed as appendages of
the main cloud extending several miles outward from the area
of lightning and precipitation. Thus, any cloud connected to
a severe thunderstorm carries a threat of violence.

Turbulence
Potentially hazardous turbulence is present in all
thunderstorms, and a severe thunderstorm can destroy an
aircraft. Strongest turbulence within the cloud occurs with
shear between updrafts and downdrafts. Outside the cloud,
shear turbulence has been encountered several thousand feet
above and 20 miles laterally from a severe storm. A low-level
turbulent area is the shear zone associated with the gust front.
Often, a "roll cloud" on the leading edge of a storm marks the
top of the eddies in this shear, and it signifies an extremely
turbulent zone. Gust fronts often move far ahead (up to 15
miles) of associated precipitation. The gust front causes a
rapid, and sometimes drastic, change in surface wind ahead
12-24

of an approaching storm. Advisory Circular (AC) 00-54, Pilot
Windshear Guide, explains gust front hazards associated with
thunderstorms. Figure 2 in the AC shows a cross section of a
mature stage thunderstorm with a gust front area where very
serious turbulence may be encountered.

Icing
Updrafts in a thunderstorm support abundant liquid water
with relatively large droplet sizes. When carried above
the freezing level, the water becomes supercooled. When
temperature in the upward current cools to about –15 °C,
much of the remaining water vapor sublimates as ice crystals.
Above this level, at lower temperatures, the amount of
supercooled water decreases.
Supercooled water freezes on impact with an aircraft. Clear
icing can occur at any altitude above the freezing level, but at
high levels, icing from smaller droplets may be rime or mixed
rime and clear ice. The abundance of large, supercooled
water droplets makes clear icing very rapid between 0 °C and
–15 °C and encounters can be frequent in a cluster of cells.
Thunderstorm icing can be extremely hazardous.
Thunderstorms are not the only area where pilots could
encounter icing conditions. Pilots should be alert for icing
anytime the temperature approaches 0 °C and visible moisture
is present.

Hail

Engine Water Ingestion

Hail competes with turbulence as the greatest thunderstorm
hazard to aircraft. Supercooled drops above the freezing level
begin to freeze. Once a drop has frozen, other drops latch on
and freeze to it, so the hailstone grows—sometimes into a
huge ice ball. Large hail occurs with severe thunderstorms
with strong updrafts that have built to great heights.
Eventually, the hailstones fall, possibly some distance from
the storm core. Hail may be encountered in clear air several
miles from thunderstorm clouds.

Turbine engines have a limit on the amount of water they
can ingest. Updrafts are present in many thunderstorms,
particularly those in the developing stages. If the updraft
velocity in the thunderstorm approaches or exceeds the
terminal velocity of the falling raindrops, very high
concentrations of water may occur. It is possible that these
concentrations can be in excess of the quantity of water
turbine engines are designed to ingest. Therefore, severe
thunderstorms may contain areas of high water concentration,
which could result in flameout and/or structural failure of
one or more engines.

As hailstones fall through air whose temperature is above 0
°C, they begin to melt and precipitation may reach the ground
as either hail or rain. Rain at the surface does not mean the
absence of hail aloft. Possible hail should be anticipated
with any thunderstorm, especially beneath the anvil of a
large cumulonimbus. Hailstones larger than one-half inch
in diameter can significantly damage an aircraft in a few
seconds.

Ceiling and Visibility
Generally, visibility is near zero within a thunderstorm
cloud. Ceiling and visibility also may be restricted in
precipitation and dust between the cloud base and the ground.
The restrictions create the same problem as all ceiling and
visibility restrictions; but the hazards are multiplied when
associated with the other thunderstorm hazards of turbulence,
hail, and lightning.

Chapter Summary
Knowledge of the atmosphere and the forces acting within
it to create weather is essential to understand how weather
affects a flight. By understanding basic weather theories, a
pilot can make sound decisions during flight planning after
receiving weather briefings. For additional information on the
topics discussed in this chapter, see the following publications
as amended: AC 00-6, Aviation Weather For Pilots and Flight
Operations Personnel; AC 00-24, Thunderstorms; AC 00-45,
Aviation Weather Services; AC 91-74, Pilot Guide: Flight in
Icing Conditions; and chapter 7, section 2 of the Aeronautical
Information Manual (AIM).

Effect on Altimeters
Pressure usually falls rapidly with the approach of a
thunderstorm, rises sharply with the onset of the first gust
and arrival of the cold downdraft and heavy rain showers,
and then falls back to normal as the storm moves on. This
cycle of pressure change may occur in 15 minutes. If the pilot
does not receive a corrected altimeter setting, the altimeter
may be more than 100 feet in error.

Lightning
A lightning strike can puncture the skin of an aircraft
and damage communications and electronic navigational
equipment. Although lightning has been suspected of igniting
fuel vapors and causing an explosion, serious accidents due
to lightning strikes are rare. Nearby lightning can blind the
pilot, rendering him or her momentarily unable to navigate
either by instrument or by visual reference. Nearby lightning
can also induce permanent errors in the magnetic compass.
Lightning discharges, even distant ones, can disrupt radio
communications on low and medium frequencies. Though
lightning intensity and frequency have no simple relationship
to other storm parameters, severe storms, as a rule, have a
high frequency of lightning.

12-25

12-26


Chapter 13

Aviation Weather Services
Introduction
In aviation, weather service is a combined effort of the
National Weather Service (NWS), Federal Aviation
Administration (FAA), Department of Defense (DOD), other
aviation groups, and individuals. Because of the increasing
need for worldwide weather services, foreign weather
organizations also provide vital input.
While weather forecasts are not 100 percent accurate,
meteorologists, through careful scientific study and computer
modeling, have the ability to predict weather patterns, trends,
and characteristics with increasing accuracy. Through a
complex system of weather services, government agencies,
and independent weather observers, pilots and other aviation
professionals receive the benefit of this vast knowledge base
in the form of up-to-date weather reports and forecasts.
These reports and forecasts enable pilots to make informed
decisions regarding weather and flight safety before and
during a flight.

13-1

Observations
The data gathered from surface and upper altitude
observations form the basis of all weather forecasts,
advisories, and briefings. There are four types of weather
observations: surface, upper air, radar, and satellite.
Surface Aviation Weather Observations
Surface aviation weather observations (METARs) are a
compilation of elements of the current weather at individual
ground stations across the United States. The network is
made up of government and privately contracted facilities
that provide continuous up-to-date weather information.
Automated weather sources, such as the Automated Weather
Observing Systems (AWOS), Automated Surface Observing
Systems (ASOS), as well as other automated facilities, also
play a major role in the gathering of surface observations.
Surface observations provide local weather conditions
and other relevant information for a specific airport. This
information includes the type of report, station identifier,
date and time, modifier (as required), wind, visibility,
runway visual range (RVR), weather phenomena, sky
condition, temperature/dew point, altimeter reading, and
applicable remarks. The information gathered for the surface
observation may be from a person, an automated station, or
an automated station that is updated or enhanced by a weather
observer. In any form, the surface observation provides
valuable information about individual airports around the
country. Although the reports cover only a small radius, the
pilot can generate a good picture of the weather over a wide
area when many reporting stations are viewed together.

Air Route Traffic Control Center (ARTCC)
The Air Route Traffic Control Center (ARTCC) facilities
are responsible for maintaining separation between flights
conducted under instrument flight rules (IFR) in the en
route structure. Center radars (Air Route Surveillance Radar
(ARSR)) acquire and track transponder returns using the same
basic technology as terminal radars. Earlier center radars
displayed weather as an area of slashes (light precipitation)
and Hs (moderate rainfall). Because the controller could not
detect higher levels of precipitation, pilots had to be wary
of areas showing moderate rainfall. Newer radar displays
show weather as three shades of blue. Controllers can select
the level of weather to be displayed. Weather displays of
higher levels of intensity make it difficult for controllers to
see aircraft data blocks, so pilots should not expect air traffic
control (ATC) to keep weather displayed continuously.
Upper Air Observations
Observations of upper air weather are more challenging
than surface observations. There are several methods by
which upper air weather phenomena can be observed:
13-2

radiosonde observations, pilot weather reports (PIREPs),
Aircraft Meteorological Data Relay (AMDAR) and the
Meteorological Data Collection and Reporting System
(MDCRS). A radiosonde is a small cubic instrumentation
package that is suspended below a six foot hydrogen- or
helium-filled balloon. Once released, the balloon rises at a rate
of approximately 1,000 feet per minute (fpm). As it ascends,
the instrumentation gathers various pieces of data, such as air
temperature, moisture, and pressure, as well as wind speed
and direction. Once the information is gathered, it is relayed
to ground stations via a 300 milliwatt radio transmitter.
The balloon flight can last as long as 2 hours or more and
can ascend to altitudes as high as 115,000 feet and drift as
far as 125 miles. The temperatures and pressures experienced
during the flight can be as low as -130 °F and pressures as
low as a few thousandths of what is experienced at sea level.
Since the pressure decreases as the balloon rises in the
atmosphere, the balloon expands until it reaches the limits
of its elasticity. This point is reached when the diameter has
increased to over 20 feet. At this point, the balloon pops and
the radiosonde falls back to Earth. The descent is slowed by
means of a parachute. The parachute aids in protecting people
and objects on the ground. Each year over 75,000 balloons
are launched. Of that number, 20 percent are recovered and
returned for reconditioning. Return instructions are printed
on the side of each radiosonde.
Pilots also provide vital information regarding upper air
weather observations and remain the only real-time source
of information regarding turbulence, icing, and cloud
heights. This information is gathered and filed by pilots
in flight. Together, PIREPs and radiosonde observations
provide information on upper air conditions important for
flight planning. Many domestic and international airlines
have equipped their aircraft with instrumentation that
automatically transmits in flight weather observations
through the DataLink system.
The Aircraft Meteorological Data Relay (AMDAR) is
an international program utilizing commercial aircraft to
provide automated weather observations. The AMDAR
program provides approximately 220,000-230,000 aircraft
observations per day on a worldwide basis utilizing aircraft
onboard sensors and probes that measure wind, temperature,
humidity/water vapor, turbulence and icing data. AMDAR
vertical profiles and en route observations provide significant
benefits to the aviation community by enhancing aircraft
safety and operating efficiency through improved weather
analysis and forecasting. The AMDAR program also
contributes to improved short and medium term numerical
weather forecasts for a wide range of services including

severe weather, defense, marine, public weather and
environmental monitoring. The information is down linked
either via Very High Frequency (VHF) communications
through the Aircraft Communications Addressing and
Reporting System (ACARS) or via satellite link through the
Aircraft to Satellite Data Acquisition and Relay (ASDAR).
The Meteorological Data Collection and Reporting System
(MDCRS) is an automated airborne weather observation
program that is used in the U.S. This program collects and
disseminates real-time upper-air weather observations from
participating airlines. The weather elements are down linked
via ACARS and are managed by Aeronautical Radio, Inc.
(ARINC) who then forwards them in Binary Universal Form
for the Representation of Meteorological Data (BUFR)
format to the NWS and in raw data form to the Earth Science
Research Laboratory (ESRL) and the participating airline.
More than 1,500 aircraft report wind and temperature data
with some of these same aircraft also providing turbulence
and humidity/water vapor information. In conjunction with
avionics manufacturers, each participating airline programs
their equipment to provide certain levels of meteorological
data. The monitoring and collection of climb, en route, and
descent data is accomplished through the aircraft's Flight Data
Acquisition and Monitoring System (FDAMS) and is then
transmitted via ACARS. When aircraft are out of ACARS
range, reports can be relayed through ASDAR. However, in
most cases, the reports are buffered until the aircraft comes
within ACARS range, at which point they are downloaded.

Figure 13-1. Example of a weather radar scope.

Radar Observations
There are four types of radars which provide information
about precipitation and wind.
1.	 The WSR-88D NEXRAD radar, commonly called
Doppler radar, provides in-depth observations that
inform surrounding communities of impending
weather. Doppler radar has two operational modes:
clear air and precipitation. In clear air mode, the radar
is in its most sensitive operational mode because a
slow antenna rotation allows the radar to sample the
atmosphere longer. Images are updated about every
10 minutes in this mode.
Precipitation targets provide stronger return signals;
therefore, the radar is operated in the Precipitation
mode when precipitation is present. A faster antenna
rotation in this mode allows images to update at
a faster rate, approximately every 4 to 6 minutes.
Intensity values in both modes are measured in
dBZ (decibels of Z) and are depicted in color on the
radar image. [Figure 13-1] Intensities are correlated
to intensity terminology (phraseology) for ATC
purposes. [Figures 13-2 and 13-3]

Figure 13-2. WSR-88D Weather Radar Echo Intensity Legend.
Reflectivity (dBZ) Ranges

Weather Radar Echo Intensity

<30 dBZ
30–40 dBZ
>40–50
50+ dBZ

Light
Moderate
Heavy
Extreme

Figure 13-3. WSR-88D Weather Radar Precipitation Intensity

Terminology.

13-3

2.	 FAA terminal Doppler weather radar (TDWR),
installed at some major airports around the country,
also aids in providing severe weather alerts and
warnings to ATC. Terminal radar ensures pilots
are aware of wind shear, gust fronts, and heavy
precipitation, all of which are dangerous to arriving
and departing aircraft.
3.	 The third type of radar commonly used in the detection
of precipitation is the FAA airport surveillance radar.
This radar is used primarily to detect aircraft, but it
also detects the location and intensity of precipitation,
which is used to route aircraft traffic around severe
weather in an airport environment.
4. 	 Airborne radar is equipment carried by aircraft to
locate weather disturbances. The airborne radars
generally operate in the C or X bands (around 6
GHz or around 10 GHz, respectively) permitting
both penetration of heavy precipitation, required for
determining the extent of thunderstorms, and sufficient
reflection from less intense precipitation.
Satellite
Advancement in satellite technologies has recently allowed
for commercial use to include weather uplinks. Through the
use of satellite subscription services, individuals are now able
to receive satellite transmitted signals that provide near realtime weather information for the North American continent.

Service Outlets
Service outlets are government, government contract, or
private facilities that provide aviation weather services. Several
different government agencies, including the FAA, National
Oceanic and Atmospheric Administration (NOAA), and the
NWS work in conjunction with private aviation companies
to provide different means of accessing weather information.

recordings of meteorological and aeronautical information.
TIBS provides area and route briefings, airspace procedures,
and special announcements. The recordings are automatically
updated as changes occur. It is designed to be a preliminary
briefing tool and is not intended to replace a standard briefing
from a FSS specialist. The TIBS service can only be accessed
by a touchtone phone. The phone numbers for the TIBS
service are listed in the Chart Supplement U.S. (formerly
Airport/Facility Directory).
Hazardous Inflight Weather Advisory Service
(HIWAS)
Hazardous Inflight Weather Advisory Service (HIWAS),
available in the 48 conterminous states, is an automated
continuous broadcast of hazardous weather information
over selected VOR navigational aids (NAVAIDs). The
broadcasts include advisories such as AIRMETS, SIGMETS,
convective SIGMETS, and urgent PIREPs. The broadcasts
are automatically updated as changes occur. Pilots should
contact a FSS or EFAS for additional information. VORs that
have HIWAS capability are depicted on aeronautical charts
with an "H" in the upper right corner of the identification
box. [Figure 13-4]
Transcribed Weather Broadcast (TWEB) (Alaska
Only)
A continuous automated broadcast of meteorological and
aeronautical data over selected low or medium frequency (L/
MF) and very high frequency (VHF) omnidirectional range
(VOR) NAVAID facilities. The broadcasts are automatically
updated as changes occur. The broadcast contains adverse
conditions, surface weather observations, PIREPS, and
a density altitude statement (if applicable). Recordings
may also include a synopsis, winds aloft forecast, en route
and terminal forecast data, and radar reports. At selected
locations, telephone access to the TWEB has been provided
(TEL-TWEB). Telephone numbers for this service are found

Flight Service Station (FSS)
The FSS is the primary source for preflight weather
information. A preflight weather briefing from an FSS can be
obtained 24 hours a day by calling 1-800-WX BRIEF from
anywhere in the United States and Puerto Rico. Telephone
numbers for FSS can be found in the Chart Supplement U.S.
(formerly Airport/Facility Directory) or in the United States
Government section of the telephone book.
The FSS also provides inflight weather briefing services
and weather advisories to flights within the FSS area of
responsibility.
Telephone Information Briefing Service (TIBS)
The Telephone Information Briefing Service (TIBS),
provided by FSS, is a system of automated telephone
13-4

Symbol indicates HIWAS
Figure 13-4. HIWAS availability is shown on sectional chart.

in the Alaska Chart Supplement U.S. (formerly Airport/
Facility Directory). These broadcasts are made available
primarily for preflight and inflight planning, and as such,
should not be considered as a substitute for specialistprovided preflight briefings.

5.	 En route forecast—a summary of the weather forecast
for the proposed route of flight.

Weather Briefings

7.	 Forecast winds and temperatures aloft—a forecast of
the winds at specific altitudes for the route of flight.
The forecast temperature information aloft is provided
only upon request.

Prior to every flight, pilots should gather all information
vital to the nature of the flight. This includes an appropriate
weather briefing obtained from a specialist at a FSS.
For weather specialists to provide an appropriate weather
briefing, they need to know which of the three types of
briefings is needed—standard, abbreviated, or outlook. Other
helpful information is whether the flight is visual flight rules
(VFR) or IFR, aircraft identification and type, departure
point, estimated time of departure (ETD), flight altitude, route
of flight, destination, and estimated time en route (ETE).
This information is recorded in the flight plan system and a
note is made regarding the type of weather briefing provided.
If necessary, it can be referenced later to file or amend a
flight plan. It is also used when an aircraft is overdue or is
reported missing.
Standard Briefing
A standard briefing provides the most complete information
and a more complete weather picture. This type of briefing
should be obtained prior to the departure of any flight and
should be used during flight planning. A standard briefing
provides the following information in sequential order if it
is applicable to the route of flight.
1.	 Adverse conditions—this includes information about
adverse conditions that may influence a decision to
cancel or alter the route of flight. Adverse conditions
include significant weather, such as thunderstorms or
aircraft icing, or other important items such as airport
closings.
2.	 VFR flight not recommended—if the weather for
the route of flight is below VFR minimums, or if
it is doubtful the flight could be made under VFR
conditions due to the forecast weather, the briefer may
state "VFR flight not recommended." It is the pilot's
decision whether or not to continue the flight under
VFR, but this advisory should be weighed carefully.
3.	 Synopsis—an overview of the larger weather picture.
Fronts and major weather systems that affect the
general area are provided.
4.	 Current conditions—the current ceilings, visibility,
winds, and temperatures. If the departure time is more
than 2 hours away, current conditions are not included
in the briefing.

6.	 Destination forecast—a summary of the expected
weather for the destination airport at the estimated
time of arrival (ETA).

8.	 Notices to Airmen (NOTAM)—information pertinent
to the route of flight that has not been published in the
NOTAM publication. Published NOTAM information
is provided during the briefing only when requested.
9.	 ATC delays—an advisory of any known ATC delays
that may affect the flight.
10.	 Other information—at the end of the standard briefing,
the FSS specialist provides the radio frequencies
needed to open a flight plan and to contact EFAS. Any
additional information requested is also provided at
this time.
Abbreviated Briefing
An abbreviated briefing is a shortened version of the standard
briefing. It should be requested when a departure has been
delayed or when weather information is needed to update
the previous briefing. When this is the case, the weather
specialist needs to know the time and source of the previous
briefing so the necessary weather information is not omitted
inadvertently. It is always a good idea for the pilot to update
the weather information whenever he/she has additional time.
Outlook Briefing
An outlook briefing should be requested when a planned
departure is 6 hours or more away. It provides initial forecast
information that is limited in scope due to the time frame
of the planned flight. This type of briefing is a good source
of flight planning information that can influence decisions
regarding route of flight, altitude, and ultimately the go/no-go
decision. A prudent pilot requests a follow-up briefing prior
to departure since an outlook briefing generally only contains
information based on weather trends and existing weather in
geographical areas at or near the departure airport. A standard
briefing near the time of departure ensures that the pilot has
the latest information available prior to his/her flight.

Aviation Weather Reports
Aviation weather reports are designed to give accurate
depictions of current weather conditions. Each report
provides current information that is updated at different times.
Some typical reports are METARs and PIREPs.

13-5

Aviation Routine Weather Report (METAR)
A METAR is an observation of current surface weather
reported in a standard international format. While the
METAR code has been adopted worldwide, each country is
allowed to make modifications to the code. Normally, these
differences are minor but necessary to accommodate local
procedures or particular units of measure. This discussion of
METAR covers elements used in the United States.
METARs are issued on a regularly scheduled basis unless
significant weather changes have occurred. A special
METAR (SPECI) can be issued at any time between routine
METAR reports.
Example:
METAR KGGG 161753Z AUTO 14021G26KT 3/4SM
+TSRA BR BKN008 OVC012CB 18/17 A2970 RMK
PRESFR
A typical METAR report contains the following information
in sequential order:
1.	 Type of report—there are two types of METAR
reports. The first is the routine METAR report that is
transmitted on a regular time interval. The second is
the aviation selected SPECI. This is a special report
that can be given at any time to update the METAR for
rapidly changing weather conditions, aircraft mishaps,
or other critical information.
2.	 Station identifier—a four-letter code as established by
the International Civil Aviation Organization (ICAO).
In the 48 contiguous states, a unique three-letter
identifier is preceded by the letter "K." For example,
Gregg County Airport in Longview, Texas, is
identified by the letters "KGGG," K being the country
designation and GGG being the airport identifier.
In other regions of the world, including Alaska and
Hawaii, the first two letters of the four-letter ICAO
identifier indicate the region, country, or state. Alaska
identifiers always begin with the letters "PA" and
Hawaii identifiers always begin with the letters "PH."
Station identifiers can be found by calling the FSS, a
NWS office, or by searching various websites such
as DUATS and NOAA's Aviation Weather Aviation
Digital Data Services (ADDS).
3.	 Date and time of report—depicted in a six-digit group
(161753Z). The first two digits are the date. The last
four digits are the time of the METAR/SPECI, which
is always given in coordinated universal time (UTC).
A "Z" is appended to the end of the time to denote
the time is given in Zulu time (UTC) as opposed to
local time.

13-6

4.	 Modifier—denotes that the METAR/SPECI came from
an automated source or that the report was corrected. If
the notation "AUTO" is listed in the METAR/SPECI,
the report came from an automated source. It also lists
"AO1" (for no precipitation discriminator) or "AO2"
(with precipitation discriminator) in the "Remarks"
section to indicate the type of precipitation sensors
employed at the automated station.
When the modifier "COR" is used, it identifies a
corrected report sent out to replace an earlier report
that contained an error (for example: METAR KGGG
161753Z COR).
5.	 Wind—reported with five digits (14021KT) unless
the speed is greater than 99 knots, in which case the
wind is reported with six digits. The first three digits
indicate the direction the true wind is blowing from in
tens of degrees. If the wind is variable, it is reported
as "VRB." The last two digits indicate the speed of
the wind in knots unless the wind is greater than 99
knots, in which case it is indicated by three digits. If
the winds are gusting, the letter "G" follows the wind
speed (G26KT). After the letter "G," the peak gust
recorded is provided. If the wind direction varies more
than 60° and the wind speed is greater than six knots,
a separate group of numbers, separated by a "V," will
indicate the extremes of the wind directions.
6.	 Visibility—the prevailing visibility (¾ SM) is reported
in statute miles as denoted by the letters "SM." It is
reported in both miles and fractions of miles. At times,
runway visual range (RVR) is reported following the
prevailing visibility. RVR is the distance a pilot can
see down the runway in a moving aircraft. When RVR
is reported, it is shown with an R, then the runway
number followed by a slant, then the visual range
in feet. For example, when the RVR is reported as
R17L/1400FT, it translates to a visual range of 1,400
feet on runway 17 left.
7.	 Weather—can be broken down into two different
categories: qualifiers and weather phenomenon
(+TSRA BR). First, the qualifiers of intensity,
proximity, and the descriptor of the weather are given.
The intensity may be light (–), moderate ( ), or heavy
(+). Proximity only depicts weather phenomena that
are in the airport vicinity. The notation "VC" indicates
a specific weather phenomenon is in the vicinity
of five to ten miles from the airport. Descriptors
are used to describe certain types of precipitation
and obscurations. Weather phenomena may be
reported as being precipitation, obscurations, and
other phenomena, such as squalls or funnel clouds.

Descriptions of weather phenomena as they begin or
end and hailstone size are also listed in the "Remarks"
sections of the report. [Figure 13-5]
8.

9.

Sky condition—always reported in the sequence of
amount, height, and type or indefinite ceiling/height
(vertical visibility) (BKN008 OVC012CB, VV003).
The heights of the cloud bases are reported with a
three-digit number in hundreds of feet AGL. Clouds
above 12,000 feet are not detected or reported by an
automated station. The types of clouds, specifically
towering cumulus (TCU) or cumulonimbus (CB)
clouds, are reported with their height. Contractions
are used to describe the amount of cloud coverage and
obscuring phenomena. The amount of sky coverage is
reported in eighths of the sky from horizon to horizon.
[Figure 13-6]
Temperature and dew point—the air temperature and
dew point are always given in degrees Celsius (C) or
(18/17). Temperatures below 0 °C are preceded by
the letter "M" to indicate minus.

10. Altimeter setting—reported as inches of mercury
("Hg) in a four-digit number group (A2970). It is
always preceded by the letter "A." Rising or falling
pressure may also be denoted in the "Remarks"
sections as "PRESRR" or "PRESFR," respectively.
11. Zulu time—a term used in aviation for UTC, which
places the entire world on one time standard.

Sky Cover

Contraction

Less than 1∕8
(Clear) 1∕8–2∕8 (Few)
³∕8–4∕8 (Scattered)
5∕8–7∕8 (Broken)
8∕8 or (Overcast)

SKC, CLR, FEW
FEW
SCT
BKN
OVC

Figure 13-6. Reportable contractions for sky condition.

12. Remarks—the remarks section always begins with the
letters "RMK." Comments may or may not appear in
this section of the METAR. The information contained
in this section may include wind data, variable
visibility, beginning and ending times of particular
phenomenon, pressure information, and various other
information deemed necessary. An example of a
remark regarding weather phenomenon that does not
fit in any other category would be: OCNL LTGICCG.
This translates as occasional lightning in the clouds
and from cloud to ground. Automated stations also use
the remarks section to indicate the equipment needs
maintenance.
Example:
METAR KGGG 161753Z AUTO 14021G26KT 3/4SM
+TSRA BR BKN008 OVC012CB 18/17 A2970 RMK
PRESFR

Qualifier
Intensity or Proximity 1

Weather Phenomena
Descriptor 2

Precipitation 3

Obscuration 4

Other 5

MI Shallow

DZ Drizzle

BR Mist

PO Dust/sand whirls

BC Patches

RA Rain

FG Fog

SQ Squalls

+ Heavy

DR Low drifting

SN Snow

FU Smoke

FC Funnel cloud

VC in the vicinity

BL Blowing

SG Snow grains

DU Dust

+FC Tornado or waterspout

SH Showers

IC Ice crystals (diamond dust)

SA Sand

SS Sandstorm

TS Thunderstorms

PL Ice pellets

HZ Haze

DS Dust storm

FZ Freezing

GR Hail

PY Spray

PR Partial

GS Small hail or snow pellets

VA Volcanic ash

– Light
Moderate (no qualifier)

UP *Unknown precipitation
The weather groups are constructed by considering columns 1–5 in this table in sequence:
intensity, followed by descriptor, followed by weather phenomena (e.g., heavy rain showers(s) is coded as +SHRA).
* Automated stations only
Figure 13-5. Descriptors and weather phenomena used in a typical METAR.

13-7

Explanation:

Routine METAR for Gregg County Airport for the 16th

day of the month at 1753Z automated source. Winds are

140 at 21 knots gusting to 26. Visibility is ¾ statute mile.

Thunderstorms with heavy rain and mist. Ceiling is broken

at 800 feet, overcast at 1,200 feet with cumulonimbus clouds.

Temperature 18 °C and dew point 17 °C. Barometric pressure

is 29.70 "Hg and falling rapidly.

Pilot Weather Reports (PIREPs)
PIREPs provide valuable information regarding the
conditions as they actually exist in the air, which cannot be
gathered from any other source. Pilots can confirm the height
of bases and tops of clouds, locations of wind shear and
turbulence, and the location of inflight icing. If the ceiling is
below 5,000 feet, or visibility is at or below five miles, ATC
facilities are required to solicit PIREPs from pilots in the area.
When unexpected weather conditions are encountered, pilots
are encouraged to make a report to a FSS or ATC. When a
pilot weather report is filed, the ATC facility or FSS adds it
to the distribution system to brief other pilots and provide
inflight advisories.
PIREPs are easy to file and a standard reporting form outlines
the manner in which they should be filed. Figure 13-7 shows
the elements of a PIREP form. Item numbers 1 through 5 are
required information when making a report, as well as at least

one weather phenomenon encountered. A PIREP is normally

transmitted as an individual report but may be appended to

a surface report. Pilot reports are easily decoded, and most

contractions used in the reports are self-explanatory.

Example:

UA/OV GGG 090025/TM 1450/FL 060/TP C182/SK

080 OVC/WX FV04SM RA/TA 05/WV 270030KT/TB

LGT/RM HVY RAIN

Explanation:

Type: .................................Routine pilot report

Location: .......................... 25 NM out on the 090° radial,

Gregg County VOR

Time: ................................ 1450 Zulu

Altitude or Flight Level: 6,000 feet

Aircraft Type: ...................Cessna 182

Sky Cover: ........................8,000 overcast

Visibility/Weather: ...........4 miles in rain

Temperature: .....................5 °Celsius

Wind: ................................270° at 30 knots

Turbulence: .......................Light

Icing: ................................ None reported

Remarks: .......................... Rain is heavy


Encoding Pilot Weather Reports (PIREPS)
1
2
3

XXX
UA
/OV

3-letter station identifier
Routine PIREP, UUA-Urgent PIREP.
Location

4
5
6
7

/TM
/FL
/TP
/SK

Time
Altitude/flight level
Type aircraft
Sky cover/cloud layers

8

/WX

Weather

9
10

/TA
/WV

Air temperature in celsius (C)
Wind

11

/TB

Turbulence

12

/IC

Icing

13

/RM

Remarks

Figure 13-7. PIREP encoding and decoding.

13-8

Nearest weather reporting location to the reported phenomenon
Use 3-letter NAVAID idents only.
a. Fix: /OV ABC, /OV ABC 090025.
b. Fix: /OV ABC 045020-DEF, /OV ABC-DEF-GHI
4 digits in UTC: /TM 0915.
3 digits for hundreds of feet. If not known, use UNKN: /FL095, /FL310, /FLUNKN.
4 digits maximum. If not known, use UNKN: /TP L329, /TP B727, /TP UNKN.
Describe as follows:
a. Height of cloud base in hundreds of feet. If unknown, use UNKN.
b. Cloud cover symbol.
c. Height of cloud tops in hundreds of feet.
Flight visibility reported first:
Use standard weather symbols:
/WX FV02SM RA HZ, /WX FV01SM TSRA.
If below zero, prefix with a hyphen: /TA 15, /TA M06.
Direction in degrees magnetic north and speed in six digits:
/WV270045KT, WV 280110KT.
Use standard contractions for intensity and type (use CAT or CHOP when
appropriate). Include altitude only if different from /FL, /TB EXTRM, /TB
LGT-MOD BLO 090.
Describe using standard intensity and type contractions. Include altitude only if
different than /FL: /IC LGT-MOD RIME, /IC SEV CLR 028-045.
Use free form to clarify the report and type hazardous elements first:
/RM LLWS -15KT SFC-030 DURC RY22 JFK.

Aviation Forecasts
Observed weather condition reports are often used in the
creation of forecasts for the same area. A variety of different
forecast products are produced and designed to be used in the
preflight planning stage. The printed forecasts that pilots need
to be familiar with are the terminal aerodrome forecast (TAF),
aviation area forecast (FA), inflight weather advisories
(SIGMET, AIRMET), and the winds and temperatures aloft
forecast (FB).
Terminal Aerodrome Forecasts (TAF)
A TAF is a report established for the five statute mile
radius around an airport. TAF reports are usually given for
larger airports. Each TAF is valid for a 24 or 30-hour time
period and is updated four times a day at 0000Z, 0600Z,
1200Z, and 1800Z. The TAF utilizes the same descriptors
and abbreviations as used in the METAR report. The TAF
includes the following information in sequential order:
1.	 Type of report—a TAF can be either a routine forecast
(TAF) or an amended forecast (TAF AMD).
2.	 ICAO station identifier—the station identifier is the
same as that used in a METAR.
3.	 Date and time of origin—time and date (081125Z)
of TAF origination is given in the six-number code
with the first two being the date, the last four being
the time. Time is always given in UTC as denoted by
the Z following the time block.
4.	 Valid period dates and times—The TAF valid period
(0812/0912) follows the date/time of forecast origin
group. Scheduled 24 and 30 hour TAFs are issued
four times per day, at 0000, 0600, 1200, and 1800Z.
The first two digits (08) are the day of the month for
the start of the TAF. The next two digits (12) are the
starting hour (UTC). 09 is the day of the month for
the end of the TAF, and the last two digits (12) are
the ending hour (UTC) of the valid period. A forecast
period that begins at midnight UTC is annotated as 00.
If the end time of a valid period is at midnight UTC,
it is annotated as 24. For example, a 00Z TAF issued
on the 9th of the month and valid for 24 hours would
have a valid period of 0900/0924.
5.	 Forecast wind—the wind direction and speed forecast
are coded in a five-digit number group. An example
would be 15011KT. The first three digits indicate the
direction of the wind in reference to true north. The
last two digits state the windspeed in knots appended
with "KT." Like the METAR, winds greater than 99
knots are given in three digits.

7.	 Forecast significant weather—weather phenomena
are coded in the TAF reports in the same format as
the METAR.
8.	 Forecast sky condition—given in the same format as
the METAR. Only cumulonimbus (CB) clouds are
forecast in this portion of the TAF report as opposed
to CBs and towering cumulus in the METAR.
9.	 Forecast change group—for any significant weather
change forecast to occur during the TAF time period,
the expected conditions and time period are included
in this group. This information may be shown as from
(FM), and temporary (TEMPO). "FM" is used when a
rapid and significant change, usually within an hour, is
expected. "TEMPO" is used for temporary fluctuations
of weather, expected to last less than 1 hour.
10. PROB30—a given percentage that describes the
probability of thunderstorms and precipitation
occurring in the coming hours. This forecast is not
used for the first 6 hours of the 24-hour forecast.
Example:
TAF
KPIR 111130Z 1112/1212
TEMPO 1112/1114 5SM BR
FM1500 16015G25KT P6SM SCT040 BKN250
FM120000 14012KT P6SM BKN080 OVC150 PROB30
1200/1204 3SM TSRA BKN030CB
FM120400 1408KT P6SM SCT040 OVC080
TEMPO 1204/1208 3SM TSRA OVC030CB
Explanation:
Routine TAF for Pierre, South Dakota…on the 11th day of
the month, at 1130Z…valid for 24 hours from 1200Z on the
11th to 1200Z on the 12th…wind from 150° at 12 knots…
visibility greater than 6 SM…broken clouds at 9,000 feet…
temporarily, between 1200Z and 1400Z, visibility 5 SM in
mist…from 1500Z winds from 160° at 15 knots, gusting
to 25 knots visibility greater than 6 SM…clouds scattered
at 4,000 feet and broken at 25,000 feet…from 0000Z wind
from 140° at 12 knots…visibility greater than 6 SM…clouds
broken at 8,000 feet, overcast at 15,000 feet…between 0000Z
and 0400Z, there is 30 percent probability of visibility 3
SM…thunderstorm with moderate rain showers…clouds
broken at 3,000 feet with cumulonimbus clouds…from
0400Z…winds from 140° at 8 knots…visibility greater than
6 miles…clouds at 4,000 scattered and overcast at 8,000…
temporarily between 0400Z and 0800Z…visibility 3 miles…
thunderstorms with moderate rain showers…clouds overcast
at 3,000 feet with cumulonimbus clouds…end of report (=).

6.	 Forecast visibility—given in statute miles and may
be in whole numbers or fractions. If the forecast is
greater than six miles, it is coded as "P6SM."
13-9

Area Forecasts (FA)
The FA gives a picture of clouds, general weather conditions,
and visual meteorological conditions (VMC) expected over
a large area encompassing several states. There are six areas
for which area forecasts are published in the contiguous 48
states. Area forecasts are issued three times a day and are
valid for 18 hours. This type of forecast gives information
vital to en route operations, as well as forecast information
for smaller airports that do not have terminal forecasts.
Area forecasts are typically disseminated in four sections and
include the following information:
1.	 Header—gives the location identifier of the source of
the FA, the date and time of issuance, the valid forecast
time, and the area of coverage.
Example:
DFWC FA 120945
SYNOPSIS AND VFR CLDS/WX
SYNOPSIS VALID UNTIL 130400
CLDS/WX VALID UNTIL 122200…OTLK VALID
122200-130400
OK TX AR LA MS AL AND CSTL WTRS
Explanation:
The area forecast shows information given by Dallas Fort
Worth, for the region of Oklahoma, Texas, Arkansas,
Louisiana, Mississippi, and Alabama, as well as a portion
of the Gulf coastal waters. It was issued on the 12th day
of the month at 0945. The synopsis is valid from the time
of issuance until 0400 hours on the 13th. VFR clouds and
weather information on this area forecast are valid until 2200
hours on the 12th and the outlook is valid from 2200Z on the
12th to 0400Z on the 13th.
2.	 Precautionary statements—IFR conditions, mountain
obscurations, and thunderstorm hazards are described
in this section. Statements made here regarding height
are given in MSL, and if given otherwise, AGL or
ceiling (CIG) is noted.

and implies there may be occurrences of severe or greater
turbulence, severe icing, low-level wind shear, and IFR
conditions. The final line of the precautionary statement alerts
the user that heights, for the most part, are MSL. Those that
are not MSL will state AGL or CIG.
3.	 Synopsis—gives a brief summary identifying the
location and movement of pressure systems, fronts,
and circulation patterns.
Example:

SYNOPSIS…LOW PRES TROF 10Z OK/TX PNHDL AREA

FCST MOV EWD INTO CNTRL-SWRN OK BY 04Z.

WRMFNT 10Z CNTRL OK-SRN AR-NRN MS FCST LIFT

NWD INTO NERN OK-NRN AR EXTRM NRN MS BY 04Z.

Explanation:

As of 1000Z, there is a low pressure trough over the Oklahoma

and Texas panhandle area, which is forecast to move eastward

into central to southwestern Oklahoma by 0400Z. A warm

front located over central Oklahoma, southern Arkansas, and

northern Mississippi at 1000Z is forecast to lift northwestward

into northeastern Oklahoma, northern Arkansas, and extreme

northern Mississippi by 0400Z.

4.	 VFR Clouds and Weather—This section lists expected
sky conditions, visibility, and weather for the next 12
hours and an outlook for the following 6 hours.
Example:

S CNTRL AND SERN TX

AGL SCT-BKN010. TOPS 030. VIS 3-5SM BR. 14-16Z

BECMG AGL SCT030. 19Z AGL SCT050.

OTLK…VFR

OK

PNDLAND NW…AGL SCT030 SCT-BKN100.

TOPS FL200.

15Z AGL SCT040 SCT100. AFT 20Z SCT TSRA DVLPG..

FEW POSS SEV. CB TOPS FL450.

OTLK…VFR


Explanation:

In south central and southeastern Texas, there is a scattered

Example:

SEE AIRMET SIERRA FOR IFR CONDS AND MTN
 to broken layer of clouds from 1,000 feet AGL with tops at

3,000 feet, visibility is 3 to 5 SM in mist. Between 1400Z and

OBSCN.

TS IMPLY SEV OR GTR TURB SEV ICE LLWS AND
 1600Z, the cloud bases are expected to increase to 3,000 feet

AGL. After 1900Z, the cloud bases are expected to continue

IFR CONDS.

to increase to 5,000 feet AGL and the outlook is VFR.

NON MSL HGTS DENOTED BYAGL OR CIG.

Explanation:

The area forecast covers VFR clouds and weather, so the

precautionary statement warns that AIRMET Sierra should

be referenced for IFR conditions and mountain obscuration.

The code TS indicates the possibility of thunderstorms

13-10

In northwestern Oklahoma and panhandle, the clouds are

scattered at 3,000 feet with another scattered to broken layer

at 10,000 feet AGL, with the tops at 20,000 feet. At 1500

Z, the lowest cloud base is expected to increase to 4,000

feet AGL with a scattered layer at 10,000 feet AGL. After


2000Z, the forecast calls for scattered thunderstorms with rain
developing and a few becoming severe; the CB clouds have
tops at flight level (FL) 450 or 45,000 feet MSL.

phenomena considered potentially hazardous to light aircraft
and aircraft with limited operational capabilities.
An AIRMET includes forecast of moderate icing, moderate
turbulence, sustained surface winds of 30 knots or
greater, widespread areas of ceilings less than 1,000 feet
and/or visibilities less than three miles, and extensive
mountain obscurement.

It should be noted that when information is given in the area
forecast, locations may be given by states, regions, or specific
geological features such as mountain ranges. Figure 13-8
shows an area forecast chart with six regions of forecast,
states, regional areas, and common geographical features.

Each AIRMET bulletin has a fixed alphanumeric designator,
numbered sequentially for easy identification, beginning with
the first issuance of the day. Sierra is the AIRMET code used
to denote IFR and mountain obscuration; Tango is used to
denote turbulence, strong surface winds, and low-level wind
shear; and Zulu is used to denote icing and freezing levels.

Inflight Weather Advisories
Inflight weather advisories, which are provided to en route
aircraft, are forecasts that detail potentially hazardous
weather. These advisories are also available to pilots prior
to departure for flight planning purposes. An inflight
weather advisory is issued in the form of either an AIRMET,
SIGMET, or convective SIGMET.

Example:
BOSS WA 211945
AIRMET SIERRA UPDT 3 FOR IFR AND MTN OBSCN
VALID UNTIL 220200
AIRMET IFT…ME NH VT MA CT RI NY NJ AND CSTL
WTRS FROM CAR TO YSJ TO 150E ACK TO EWR TO
YOW TO CAR OCNL CIG BLW 010/VIS BLW 3SM
PCPN/BR. CONDS CONT BYD 02Z THRU 08Z

AIRMET
AIRMETs (WAs) are examples of inflight weather advisories
that are issued every 6 hours with intermediate updates
issued as needed for a particular area forecast region. The
information contained in an AIRMET is of operational
interest to all aircraft, but the weather section concerns

MT

NW

WRN Upper MI

ND

Upper
Mississippi
Valley

Ar

ro

d
ea

r
rio
pe
Su line
ke re
La Sho

wh

Champlain
Valley

ERN Upper MI
Cntrl
Upper
MI

NE

ME
S

W

ta

Riv Sou
er ris
Va
lle
y

BOS Boston

Gree
Mtnsn
White Mtns

Fort Peck
Reservoir

Valley

e
Blu tns a
M llow
a s
W Mtn
Centra
of
l
Mtns R
Mtn
O
E
N
(Sawto s
oth)

ge

East
Slopes of
Cont Dvd

Red River

WA

lle y
ad Va

of

C
Wil ostal R
lam
a
ette nge
sca
Va
de
lley
Mtn
s
Eas
t Slo
pes

Columbia Gor

Columbia
Basin

Fl at he

Interior
Valley

Mtns
NE WA

t Ra ng e

Olympic Mtns
Coast and
Costal Valley

CHI Chicago

SLC Salt Lake City

ca

Bit ter roo

e Fu

P
S ug
sc oun et
ad d
e
M
tn
ad
s
es

nD

sc

Jua

Ca

it of

Ca

Stra

S E t io n
ec
lS

Hudson Valley

as

Co

NJ

t

on

Pie

s ta l P l a in

dm

Coa

la

pa

Ap

NC

DC

al Pla
in

GA

Co

as t

SC

MIA Miami

r

bile
Mo rea
A

ive

LA

AL

al

Allegheny
Plateau

ins

nta

ou

M

VA

ch

TX

MS

RI

r

R

o

hi

lley

ippi Va

Mississ

NM

Lower

Sa

Ne Easte
w M rn
ex
ico

e

ng

in

nt

rem
Mtn ento
s

Co

AR

P

ua
ah
iric ns
Ch Mt

Sac

in

Valley

Pla

Rio Grande

B
H oo
ee t
l

to

re
Mt De C
ns ris

lle

Va

River

tal

ado

as

Color

Co

SFO
San Francisco

AZ

TN

OK

CT

MA

st

L ak

e

Le

s

ill

H

sh

Iri

WV

ia

La

ke

KS

NH

MD

Cntrl
East

ll

we

Po

IN

PA

OH

MO

CO

NY

n

Wasatch

Do
or
Pe
n
ch Shore insu
line la
N
R
N
H
ig
hl
an
ds

IA

NE

IL

y

Ca

MI

WI

Lake Mi

e

nn

a ns
Pl M t
ie
m

ra

N

La

CA

ye

Teton
Mtns

Green River
Basin

s

UT

tn

Mtns

rM

NV

SD

VT

ill
tsk
Ca tns
M

ve

WY

MN

a l
s C
wi e
L e as
Lake cis C
Fran

Ri

Coastal Mtns and Valleys
lleys
a
d Va
Shast us
s an
Sacramento
l Mtn
Valley
aquin
Siskiyo
Jo
San alley
V

d

sta
Coa

in

ID

W

OR

s
ke
La ion
NE Reg

SWRN
a
Missouri
Mtns
Dame
Are
ac
Slope
Adirondac
Reservoir
k
ckin
Northeast Lakes
Ma
Mtns
E of Cont Dvd
Yellows
Region
North of
N of Catskills
tone
at
Park
e
The
Black
Mohawk
Grasin
NW
Big Horn
NE
r
Thumb
Big Horn Mtns
Hills v er B
trl
we B
Hig
Cape Cod
E Cn
Basin
Ri Re ig B
io Valley
Loakes
hP
e On ta r
se en
late
L
NE WY
Sn
rv d
au
inaw
Extrm
Wind River
le y
James
ak
g
l
o
of
a
t
S
a
es
i
e River V
y
r
W
Black
SE
Basin
River
he Pine Rosebud
xtrm
ills
Valle
SE
s
SW
sk
E
C
at
n
C
Hills
Lee
ai
NW
Grand
Pl
CA
Ridge Area Country
NE
Lake
NE
Valley
N
Nebraska
Erie
Sand Hills
NRN
Weste
nd
NE
a
Cntrl
NW
tt e
pan
SW
Nevadrn
Co
ke
Ri v
Easter
ar
a
a
e
Handle
N
WY
i
l
Great
NW
L
e
NRN
r
Central Nebraska
ear
k
rV
Extrm
SE
ra
Nevad n
ar
So Central
NE
Lake E
NW
nt a
Salt
Sierra
S
a
Ne Mi
N Extrm
SE alley
NE
CeIow
Mtns
N Third
NW
SW
Lake
Cntrl
Mtns
Pl a
SRN
WY
E
NE
tte R
SE
SW
W
Lake Tahoe
North Park
NE
NW
NW
i v er
Cntrl
y
rl
n
i
Cnt
e
s
U
ll
B
a
inta B
rn
a
Great
asin
eV
ste e Coastal
Southern e
Cntrl
SW
l
Platt
SE
Basin
Blue River Valley
Ea hor Waters
iv NW NE
ra
So
S
R
t
L
Colorado
S
N
Re ow
SR
A loNn Third
en
NE
West
Extrm
p
NW
Mtns
a
e
SE
C
rr
g
u
SW
Cen
r
Sie s
Va blic r
NE
O
SW
Valle tral
ve N
S Cntrl
Kansas River Valley
Palmer Lake Ridge
Mtn
lle an
ys
Ri Cntr
y
Mtns
l
SW SE
io
or Divide
Marais des Cygnes Basin SW SE
NE
Oh
San Ju
ar
SE
an
High Plains
Eastern Plains
e
Lake
M
Sa
Death
tns
Northern
Extreme N
Flint
SW
Mead
NE
r
Valley
Lake of the
Va n Lu
Coastal
Lower Arkansas
SRN
Fou ers
Hills
Valley
lle is
Southern
Ozarks Area
IL
Areas
SE
e
y
Grand
An
Cornrea
telo Mojavrt
SW
Canyo
Missouri
SE
West
A
n
Va pe Dese
Lakes
Area
Lake
Co lley
e
Santa
Panhandle
as
East
vid
Middle
SRN
Northwest
Mojave
West
tal
Barbara
Di
Little
Ra
al
Deserts
SE
Northeast
Ozarks
Channel
Colorado
ng
nt
e
West
M
Valley
Panhan
o
dle
NW
Santa Monica
Mtns
go
Coacne
NE
E
lla
llo
N
Bay
of
and
n
NW s Mtns
White
SC
R
Northwest
Imperia
Southwest
Mtn
im
Hills
l
Southeast
Mtns
Northern
Valleys
Central
Texas
of a
t
Gulf of
am
Eas l
Alab
Santa Catalina
ley
Cntr
st
er Val
We trl
SE
Gilariv
SW
Cn
Eastern
m
ExtrE
Plains
NW
NE
North Central
S
North
SE
Coastal
Waters
thern
West of
Sou
m
SW
Northeast
Extr
ec
East of
Pecos
Georgia
os R
Central
Pecos
FL W 85
Extrm
rth
No
South
Southwest
Ap
Southeast l
De
Big
al
South Central
ta
lta
B
as
Are
Baach
Are end
a
Co
y ee
North
a
er lain
p
Up P
l
Lower
al
Centra
st
Rio
oa n
i
C
Grande d la
th
Valley Mi P
Sou
Extrm
Coastal Bend
Coastal Waters–from
South
coast outward to the
Lower
Upper
Coastal
Keys
information
flight
Plain
S tr

f

ul

region border

G

DFW
Dallas/Fort Worth

e am

FL

Lower Keys

Florida Straits

Figure 13-8. Area forecast region map.

13-11

AIRMET MTN OBSCN…ME NH VT MA NY PA
FROM CAR TO MLT TO CON TO SLT TO SYR TO CAR
MTNS OCNLY OBSCD BY CLDS/PCPN/BR. CONDS
CONT BYD 02Z THRU 08Z
Explanation:
AIRMET SIERRA was issued for the Boston area at 1945Z
on the 21st day of the month. SIERRA contains information
on IFR and/or mountain obscurations. This is the third
updated issuance of this Boston AIRMET series as indicated
by "SIERRA UPDT 3" and is valid until 0200Z on the
22nd. The affected states within the BOS area are: Maine,
New Hampshire, Vermont, Massachusetts, New York, and
Pennsylvania. Within an area bounded by: Caribou, ME;
to Saint Johns, New Brunswick; to 150 nautical miles east
of Nantucket, MA; to Newark, NJ; to Ottawa, Ontario;
to Caribou, ME. The effected states within Caribou, ME
to Millinocket, ME to Concord, NH to Slate Run, PA to
Syracuse, NY to Caribou, ME will experience ceilings
below 1,000 feet/visibility below 3 SM, precipitation/mist.
Conditions will continue beyond 0200Z through 0800Z.

SIGMET
SIGMETs (WSs) are inflight advisories concerning nonconvective weather that is potentially hazardous to all
aircraft. They report weather forecasts that include severe
icing not associated with thunderstorms, severe or extreme
turbulence or clear air turbulence (CAT) not associated with
thunderstorms, dust storms or sandstorms that lower surface
or inflight visibilities to below three miles, and volcanic ash.
SIGMETs are unscheduled forecasts that are valid for 4 hours
unless the SIGMET relates to a hurricane, in which case it
is valid for 6 hours.
A SIGMET is issued under an alphabetic identifier, from
November through Yankee. The first issuance of a SIGMET
is designated as an Urgent Weather SIGMET (UWS).
Reissued SIGMETs for the same weather phenomenon are
sequentially numbered until the weather phenomenon ends.
Example:
SFOR WS 100130
SIGMET ROME02 VALID UNTIL 100530
OR WA
FROM SEA TO PDT TO EUG TO SEA
OCNL SEV CAT BTN FL280 AND FL350 EXPCD
DUE TO JTSTR.
CONDS BGNG AFT 0200Z CONTG BYD 0530Z .
Explanation:
This is SIGMET Romeo 2, the second issuance for this
weather phenomenon. It is valid until the 10th day of the
month at 0530Z time. This SIGMET is for Oregon and
13-12

Washington, for a defined area from Seattle to Portland to
Eugene to Seattle. It calls for occasional severe clear air
turbulence between FL280 and FL350 due to the location of
the jet stream. These conditions will begin after 0200Z and
continue beyond the forecast scope of this SIGMET of 0530Z.

Convective Significant Meteorological Information
(WST)
A Convective SIGMET (WST) is an inflight weather
advisory issued for hazardous convective weather that affects
the safety of every flight. Convective SIGMETs are issued
for severe thunderstorms with surface winds greater than 50
knots, hail at the surface greater than or equal to ¾ inch in
diameter, or tornadoes. They are also issued to advise pilots
of embedded thunderstorms, lines of thunderstorms, or
thunderstorms with heavy or greater precipitation that affect
40 percent or more of a 3,000 square mile or greater region.
Convective SIGMETs are issued for each area of the
contiguous 48 states but not Alaska or Hawaii. Convective
SIGMETs are issued for the eastern (E), western (W), and
central (C) United States. Each report is issued at 55 minutes
past the hour, but special Convective SIGMETs can be
issued during the interim for any reason. Each forecast is
valid for 2 hours. They are numbered sequentially each day
from 1–99, beginning at 00Z time. If no hazardous weather
exists, the convective SIGMET is still issued; however, it
states "CONVECTIVE SIGMET…NONE."
Example:
MKCC WST 221855
CONVECTIVE SIGMET 20C
VALID UNTIL 2055Z
ND SD
FROM 90W MOT-GFK-ABR-90W MOT
INTSFYG AREA SEV TS MOVG FROM 24045KT. TOPS
ABV FL450. WIND GUSTS TO 60KTS RPRTD.
TORNADOES…HAIL TO 2 IN… WIND GUSTS TO
65KTS
POSS ND PTN
Explanation:
Convective SIGMET was issued for the central portion of
the United States on the 22nd at 1855Z. This is the 20th
Convective SIGMET issued on the 22nd for the central
United States as indicated by "20C" and is valid until 2055Z.
The affected states are North and South Dakota, from 90
nautical miles west of Minot, ND; to Grand Forks, ND; to
Aberdeen, SD; to 90 nautical miles west of Minot, ND. An
intensifying area of severe thunderstorms moving from 240
degrees at 45 knots (to the northeast). Thunderstorm tops will
be above FL 450. Wind gusts up to 60 knots were reported.

Also reported were tornadoes, hail to 2 inches in diameter, and
wind gusts to 65 knots possible in the North Dakota portion.
Winds and Temperature Aloft Forecast (FB)
Winds and temperatures aloft forecasts (FB) provide wind
and temperature forecasts for specific locations throughout
the United States, including network locations in Hawaii
and Alaska. The forecasts are made twice a day based on the
radiosonde upper air observations taken at 0000Z and 1200Z.
Altitudes through 12,000 feet are classified as true altitudes,
while altitudes 18,000 feet and above are classified as
altitudes and are termed flight levels. Wind direction is
always in reference to true north, and wind speed is given in
knots. The temperature is given in degrees Celsius. No winds
are forecast when a given level is within 1,500 feet of the
station elevation. Similarly, temperatures are not forecast for
any station within 2,500 feet of the station elevation.
If the wind speed is forecast to be greater than 99 knots but
less than 199 knots, the computer adds 50 to the direction
and subtracts 100 from the speed. To decode this type of data
group, the reverse must be accomplished. For example, when
the data appears as "731960," subtract 50 from the 73 and
add 100 to the 19, and the wind would be 230° at 119 knots
with a temperature of –60 °C. If the wind speed is forecast to
be 200 knots or greater, the wind group is coded as 99 knots.
For example, when the data appears as "7799," subtract 50
from 77 and add 100 to 99, and the wind is 270° at 199 knots
or greater. When the forecast wind speed is calm, or less than
5 knots, the data group is coded "9900," which means light
and variable. [Figure 13-9]
Explanation of Figure 13-9:
The heading indicates that this FB was transmitted on the
15th of the month at 1640Z and is based on the 1200Z upper
air data. The valid time is 1800Z on the same day and should
be used for the period between 1400Z and 2100Z. The
heading also indicates that the temperatures above FL240
are negative. Therefore, the minus sign will be omitted for
all forecast temperatures above FL240.
A four-digit data group shows the wind direction in reference
to true north and the wind speed in knots. The elevation at
Amarillo, Texas (AMA) is 3,605 feet, so the lowest reportable
FB KWBC 151640
DATA BASED ON 151200Z
VALID 151800Z FOR USE 1400-2100Z
TEMPS NEGATIVE ABV 24000
FB
AMA
DEN

3000

6000
2714

9000

12000

18000

24000

30000

2725+00 2625-04 2531-15 2542-27 265842
2321-04 2532-08 2434-19 2441-31 235347

Figure 13-9. Winds and temperature aloft forecast.

altitude is 6,000 feet for the forecast winds. In this case,
"2714" means the wind is forecast to be from 270° at a speed
of 14 knots.
A six-digit group includes the forecast temperature aloft.
The elevation at Denver (DEN) is 5,431 feet, so the lowest
reportable altitude is 9,000 feet for the winds and temperature
forecast. In this case, "2321-04" indicates the wind is forecast to
be from 230° at a speed of 21 knots with a temperature of –4 °C.

Weather Charts
Weather charts are graphic charts that depict current or
forecast weather. They provide an overall picture of the
United States and should be used in the beginning stages of
flight planning. Typically, weather charts show the movement
of major weather systems and fronts. Surface analysis,
weather depiction, and significant weather prognostic charts
are sources of current weather information. Significant
weather prognostic charts provide an overall forecast weather
picture.
Surface Analysis Chart
The surface analysis chart depicts an analysis of the current
surface weather. [Figure 13-10] This chart is transmitted
every 3 hours and covers the contiguous 48 states and
adjacent areas. A surface analysis chart shows the areas of
high and low pressure, fronts, temperatures, dew points, wind
directions and speeds, local weather, and visual obstructions.
Surface weather observations for reporting points across
the United States are also depicted on this chart. Each of
these reporting points is illustrated by a station model.
[Figure 13-11] A station model includes:


Sky cover—the station model depicts total sky cover
and is shown as clear, scattered, broken, overcast, or
obscured/partially obscured.



Sea level pressure—given in three digits to the nearest
tenth of a millibar (mb). For 1,000 mbs or greater,
prefix a 10 to the three digits. For less than 1,000 mbs,
prefix a 9 to the three digits.



Pressure change/tendency—pressure change in tenths
of mb over the past 3 hours. This is depicted directly
below the sea level pressure.



Dew point—given in degrees Fahrenheit.



Present weather—over 100 different standard weather
symbols are used to describe the current weather.



Temperature—given in degrees Fahrenheit.



Wind—true direction of wind is given by the wind
pointer line, indicating the direction from which the
wind is blowing. A short barb is equal to 5 knots of
wind, a long barb is equal to 10 knots of wind, and a
pennant is equal to 50 knots.
13-13

Figure 13-10. Surface analysis chart.

Wind speed

Total sky cover

Wind direction
Temperature
Present weather
Dew point

1. Total sky cover:
2. Temperature/dew point:
3. Wind:

Sea level pressure
34

147

**
32

28 /

6. Pressure change in past 3 hours:

Examples of Wind Speed and Direction Plots
NE/ 5 kts
SW/ 10 kts
W/ 50 kts
N / 15 kts

S / 60 kts

Continuous light snow
1014.7 millibars (mb)
Note: Pressure is always shown in 3 digits to the nearest tenth of a millibar.
For 1,000 mb or greater, prefix a "10" to the 3 digits
For less than 1,000 mb, prefix a "9" to the 3 digits
Increased steadily or unsteadily by 2.8 mb

Figure 13-11. Sample station model and weather chart symbols.

13-14

Pressure tendency

Overcast
34 °F/32 °F
From the northwest at 20 knots (relative to true north)
Calm

4. Present weather:
5. Sea level pressure:

Pressure change in past 3 hours

Weather Depiction Chart
A weather depiction chart details surface conditions as
derived from METAR and other surface observations. The
weather depiction chart is prepared and transmitted by
computer every 3 hours beginning at 0100Z time and is valid
data for the forecast period. It is designed to be used for flight
planning by giving an overall picture of the weather across
the United States. [Figure 13-12]
The weather depiction chart also provides a graphic display
of IFR, VFR, and marginal VFR (MVFR) weather. Areas of
IFR conditions (ceilings less than 1,000 feet and visibility
less than three miles) are shown by a hatched area outlined
by a smooth line. MVFR regions (ceilings 1,000 to 3,000
feet, visibility 3 to 5 miles) are shown by a nonhatched area
outlined by a smooth line. Areas of VFR (no ceiling or ceiling
greater than 3,000 feet and visibility greater than five miles)
are not outlined. Also plotted are fronts, troughs, and squall
lines from the previous hours surface analysis chart.
Weather depiction charts show a modified station model
that provides sky conditions in the form of total sky cover,

ceiling height, weather, and obstructions to visibility, but
does not include winds or pressure readings like the surface
analysis chart. A bracket ( ] ) symbol to the right of the station
indicates the observation was made by an automated station.
Significant Weather Prognostic Charts
Significant weather prognostic charts are available for lowlevel significant weather from the surface to FL 240 (24,000
feet), also referred to as the 400 mb level and high-level
significant weather from FL 250 to FL 630 (25,000 to 63,000
feet). The primary concern of this discussion is the low-level
significant weather prognostic chart.
The low-level chart is is a forecast of aviation weather
hazards, primarily intended to be used as a guidance product
for briefing the VFR pilot. The forecast domain covers the 48
contiguous states, southern Canada and the coastal waters for
altitudes below 24,000 ft. Low altitude Significant Weather
charts are issued four times daily and are valid at fixed times:
0000, 0600, 1200, and 1800 UTC. Each chart is divided on
the left and right into 12 and 24 hour forecast intervals (based
on the current NAM model available).

Figure 13-12. Weather depiction chart.

13-15

Effective September 1, 2015, the four-panel Low Level
SFC-240 chart was replaced with a two-panel chart. The new
two-panel chart will be the same as the top two panels in the
former four-panel chart, depicting the freezing level and areas
of IFR, MVFR, and moderate or greater turbulence. The bottom
two panels of the chart have been removed. In lieu of these
bottom two panels, an enhanced surface chart that includes
fronts, pressure, precipitation type, precipitation intensity, and
weather type, is displayed. The green precipitation polygons
will be replaced by shaded precipitation areas using the
National Digital Forecast Database (NDFD) weather grid.
Figure 13-13 depicts the new two-panel significant weather
prognostic chart, as well as the symbols typically used to
depict precipitation. The two panels depict freezing levels,
turbulence, and low cloud ceilings and/or restrictions to
visibility (shown as contoured areas of MVFR and IFR
conditions). These charts enable the pilot to pictorially
evaluate existing and potential weather hazards they may
encounter. Pilots can balance weather phenomena with
their aircraft capability and skill set resulting in aeronautical
decision-making appropriate to the flight. Prognostic charts
are an excellent source of information for preflight planning;
however, this chart should be viewed in light of current
conditions and specific local area forecasts.
The 36- and 48-hour significant weather prognostic chart is
an extension of the 12- and 24-hour forecast. This chart is
issued twice a day. It typically contains forecast positions and
characteristics of pressure patterns, fronts, and precipitation.
An example of a 36- and 48-hour surface prognostic chart is
shown in Figure 13-14.

ATC Radar Weather Displays
Although ATC systems cannot always detect the presence
or absence of clouds, they can often determine the intensity

Figure 13-13. Significant weather prognostic chart.

13-16

of a precipitation area, but the specific character of that area
(snow, rain, hail, VIRGA, etc.) cannot be determined. For
this reason, ATC refers to all weather areas displayed on ATC
radar scopes as "precipitation."
ARTCC facilities normally use a Weather and Radar
Processor (WARP) to display a mosaic of data obtained
from multiple NEXRAD sites. There is a time delay between
actual conditions and those displayed to the controller.
The precipitation data on the ARTCC controller's display
could be up to 6 minutes old. The WARP processor is only
used in ARTCC facilities. All ATC facilities using radar
weather processors with the ability to determine precipitation
intensity, describe the intensity to pilots as:


Light



Moderate



Heavy



Extreme

When the WARP is not available, a second system, the
narrowband Air Route Surveillance Radar (ARSR) can
display two distinct levels of precipitation intensity that
is described to pilots as "MODERATE and "HEAVY TO
EXTREME."
ATC facilities that cannot display the intensity levels of
precipitation due to equipment limitations describe the
location of the precipitation area by geographic position or
position relative to the aircraft. Since the intensity level is not
available, the controller states "INTENSITY UNKNOWN."
ATC radar is not able to detect turbulence. Generally,
turbulence can be expected to occur as the rate of rainfall or
intensity of precipitation increases. Turbulence associated
with greater rates of precipitation is normally more severe than

Figure 13-14. 36- (top) and 48-hour (bottom) surface prognostic chart.

13-17

any associated with lesser rates of precipitation. Turbulence
should be expected to occur near convective activity, even
in clear air. Thunderstorms are a form of convective activity
that imply severe or greater turbulence. Operation within
20 miles of thunderstorms should be approached with great
caution, as the severity of turbulence can be much greater than
the precipitation intensity might indicate.



Graphical cloud tops (CLD TOPS)



Graphical lightning strikes (LTNG)



Graphical storm cell movement (CELL MOV)



NEXRAD radar coverage (information displayed with
the NEXRAD data)



SIGMETs/AIRMETs (SIG/AIR)

Weather Avoidance Assistance
To the extent possible, controllers will issue pertinent
information on weather and assist pilots in avoiding such
areas when requested. Pilots should respond to a weather
advisory by either acknowledging the advisory or by
acknowledging the advisory and requesting an alternative
course of action as follows:



Surface analysis to include city forecasts (SFC)



County warnings (COUNTY)



Freezing levels (FRZ LVL)



Hurricane track (CYCLONE)



Temporary flight restrictions (TFR)



Request to deviate off course by stating the number
of miles and the direction of the requested deviation.

	 Request a new route to avoid the affected area.
	 Request a change of altitude.
	 Request radar vectors around the affected areas.
The controller's primary function is to provide safe separation
between aircraft. Any additional service, such as weather
avoidance assistance, can only be provided to the extent
that it does not detract from the primary function. It's also
worth noting that the separation workload is generally greater
than normal when weather disrupts the usual flow of traffic.
ATC radar limitations and frequency congestion may also
be a factor in limiting the controller's capability to provide
additional service.

Electronic Flight Displays (EFD) /MultiFunction Display (MFD) Weather
Many aircraft manufacturers now include data link weather
services with new electronic flight display (EFD) systems.
EFDs give a pilot access to many of the data link weather
services available.
Products available to a pilot on the display pictured in
Figure 13-15 are listed as follows. The letters in parentheses
indicate the soft key to press in order to access the data.


Graphical NEXRAD data (NEXRAD)



Graphical METAR data (METAR)



Textual METAR data



Textual terminal aerodrome forecasts (TAF)



City forecast data



Graphical wind data (WIND)



Graphical echo tops (ECHO T,,,OPS)

13-18

Pilots must be familiar with any EFD or MFD used and the
data link weather products available on the display.
Weather Products Age and Expiration
The information displayed using a data link weather link is
near real time but should not be thought of as instantaneous,
up-to-date information. Each type of weather display is
stamped with the age information on the MFD. The time is
referenced from Zulu when the information was assembled at
the ground station. The age should not be assumed to be the
time when the FIS received the information from the data link.
Two types of weather are displayed on the screen: "current"
weather and forecast data. Current information is displayed
by an age while the forecast data has a data stamp in the form
of "__ / __ __ : __." [Figure 13-16]
The Next Generation Weather Radar System
(NEXRAD)
The NEXRAD system is comprised of a series of 159 Weather
Surveillance Radar–1988 Doppler (WSR-88D) sites situated
throughout the United States, as well as selected overseas
sites. The NEXRAD system is a joint venture between the
United States Department of Commerce (DOC), the United
States DOD, as well as the United States Department of
Transportation (DOT). The individual agencies that have
control over the system are the NWS, Air Force Weather
Agency (AFWA) and the FAA. [Figure 13-17]
NEXRAD data for up to a 2,000 mile range can be displayed.
It is important to realize that the radar image is not real time
and can be up to 5 minutes old. The NTSB has reported on 2
fatal accidents where in-cockpit NEXRAD mosaic imagery
was available to pilots operating near quickly-developing and
fast-moving convective weather. In one of these accidents,
the images were from 6 to 8 minutes old. In some cases,
NEXRAD data can age significantly by the time the mosaic
image is created. In some extreme latency cases, the actual

Figure 13-15. Information page.

age of the oldest NEXRAD data in the mosaic can exceed
the age indication in the cockpit by 15 to 20 minutes. Even
small-time differences between the age indicator and actual
conditions can be important for safety of flight, especially
when considering fast-moving weather hazards, quickly
developing weather scenarios, and/or fast-moving aircraft.
At no time should the images be used as storm penetrating
radar nor to navigate through a line of storms. The images
display should only be used as a reference.

does not show the age of the actual weather conditions,
but rather the age of the mosaic image. The actual weather
conditions could be up to 15 to 20 minutes OLDER than
the age indicated on the display. You should consider this

NEXRAD radar is mutually exclusive of Topographic
(TOPO), TERRAIN and STORMSCOPE. When NEXRAD
is turned on, TOPO, TERRAIN, and STORMSCOPE are
turned off because the colors used to display intensities are
very similar.

RAIN

Lightning information is available to assist when NEXRAD
is enabled. This presents a more comprehensive picture of
the weather in the surrounding area.
In addition to utilizing the soft keys to activate the NEXRAD
display, the pilot also has the option of setting the desired
range. It is possible to zoom in on a specific area of the
display in order to gain a more detailed picture of the radar
display. [Figure 13-18]

What Can Pilots Do?
Remember that the in-cockpit NEXRAD display depicts
where the weather WAS, not where it IS. The age indicator

NORTH UP
NEXRAD
L
I
MIX
G
H
T SNOW

H
E
A
V
Y

SIGMET
__/__ __:__
AIRMET
__/__ __:__
Figure 13-16. List of weather products and the expiration times of each.

13-19

Figure 13-17. NEXRAD radar display.

Figure 13-18. NEXRAD radar display (500 mile range). The individual color gradients can be easily discerned and interpreted via the

legend in the upper right corner of the screen. Additional information can be gained by pressing the LEGEND soft key, which displays
the legend page.

13-20

potential delay when using in-cockpit NEXRAD capabilities,
as the movement and/or intensification of weather could
adversely affect safety of flight.


Understand that the common perception of a "5-minute
latency" with radar data is not always correct.



Get your preflight weather briefing! Having in-cockpit
weather capabilities does not circumvent the need for
a complete weather briefing before takeoff.



Use all appropriate sources of weather information to
make in-flight decisions.

 	 Let your fellow pilots know about the limitations of
in-cockpit NEXRAD.

NEXRAD Abnormalities
Although NEXRAD is a compilation of stations across
the country, there can be abnormalities associated with the
system. Some of the abnormalities are listed below.


Ground clutter



Strobes and spurious radar data



Sun strobes, when the radar antenna points directly at
the sun



Interference from buildings or mountains that may
cause shadows



Military aircraft that deploy metallic dust and may
reflect the radar signature

NEXRAD Limitations
In addition to the abnormalities listed, the NEXRAD system
does have some specific limitations.
Base Reflectivity
The NEXRAD base reflectivity does not provide adequate
information from which to determine cloud layers or type
of precipitation with respect to hail versus rain. Therefore, a
pilot may mistake rain for hail.
In addition, the base reflectivity is sampled at the minimum
antenna elevation angle. With this minimum angle, an
individual site cannot depict high altitude storms directly
over the station. This leaves an area of null coverage if an
adjacent site does not also cover the affected area.
Resolution Display
The resolution of the displayed data poses additional concerns
when the range is decreased. The minimum resolution for
NEXRAD returns is 1.24 miles. This means that when the
display range is zoomed in to approximately ten miles, the
individual square return boxes are more prevalent. Each

square indicates the strongest display return within that 1.24
mile square area.
AIRMET/SIGMET Display
AIRMET/SIGMET information is available for the displayed
viewing range on the MFD. Some displays are capable of
displaying weather information for a 2,000 mile range.
AIRMETS/SIGMETS are displayed by dashed lines on the
map. [Figure 13-19]
The legend box denotes the various colors used to depict the
AIRMETs, such as icing, turbulence, IFR weather, mountain
obscuration, and surface winds. [Figure 13-20] The great
advantage of the graphically displayed AIRMET/SIGMET
boundary box is the pilot can see the extent of the area that
the advisory covers. The pilot does not need to manually plot
the points to determine the full extent of the coverage area.
Graphical METARs
METARs can be displayed on the MFD. Each reporting
station that has a METAR/TAF available is depicted by a flag
from the center of the airport symbol. Each flag is color coded
to depict the type of weather that is currently reported at that
station. A legend is available to assist users in determining
what each flag color represents. [Figure 13-21]
The graphical METAR display shows all available reporting
stations within the set viewing range. By setting the range
knob up to a 2,000 mile range, pilots can pan around the
display map to check the current conditions of various
airports along the route of flight.
By understanding what each colored flag indicates, a pilot can
quickly determine where weather patterns display marginal
weather, IFR, or areas of VFR. These flags make it easy to
determine weather at a specific airport should the need arise
to divert from the intended airport of landing.
Data Link Weather
Pilots now have the capability of receiving continuously
updated weather across the entire country at any altitude.
No longer are pilots restricted by radio range or geographic
isolations, such as mountains or valleys.
In addition, pilots no longer have to request specific
information from weather briefing personnel directly. When
the weather becomes questionable, radio congestion often
increases, delaying the timely exchange of valuable inflight
weather updates for a pilot's specific route of flight. Flight
Service Station (FSS) personnel can communicate with only
one pilot at a time, which leaves other pilots waiting and
flying in uncertain weather conditions. Data link weather

13-21

Figure 13-19. The AIRMET information box instructs the pilot to press the ENTER button soft key (ENT) to gain additional information
on the selected area of weather. Once the ENTER soft key (ENT) is depressed, the specific textual information is displayed on the right
side of the screen.

Figure 13-20. SIGMET/AIRMET legend display.

13-22

Figure 13-21. Graphical METAR legend display.

provides the pilot with a powerful resource for enhanced
situational awareness at any time. Due to continuous data link
broadcasts, pilots can obtain a weather briefing by looking at
a display screen. Pilots have a choice between FAA-certified
devices or portable receivers as a source of weather data.
Data Link Weather Products

Flight Information Service- Broadcast (FIS-B)
Flight Information Service–Broadcast (FIS-B) is a ground
broadcast service provided through the Automatic Dependent
Surveillance–Broadcast (ADS-B) Services network over
the 978 MHz UAT data link. The FAA FIS-B system
provides pilots and flight crews of properly-equipped aircraft
with a flightdeck display of certain aviation weather and
aeronautical information which are listed below.


Aviation Routine Weather Reports (METARs)



Special Aviation Reports (SPECIs)



Terminal Area Forecasts (TAFs) and their amendments



NEXRAD (regional and CONUS) precipitation maps



Notice to Airmen (NOTAM) Distant and Flight Data
Center



Airmen's Meteorological Conditions (AIRMET)



Significant Meteorological Conditions (SIGMET) and
Convective SIGMET



Status of Special Use Airspace (SUA)



Temporary Flight Restrictions (TFRs)



Winds and Temperatures Aloft.



Pilot Reports (PIREPS)



TIS-B service status

The weather products provided by FIS-B are for information
only. Therefore, these products do not meet the safety and
regulatory requirements of official weather products. The
weather products displayed on FIS-B should not be used
as primary weather products (i.e., aviation weather to meet
operational and safety requirements). Each aircraft system is
different and some of the data that is rendered can be up to
20 or 30 minutes old and not current. Pilots should consult
the individual equipment manuals for specific delay times.

13-23

Pilot Responsibility
It is important for pilots to understand the realization that the
derived safety benefits of data link depends heavily upon the
pilot's understanding of the specific system's capabilities and
limitations which are listed below.


Product latency—be aware of the time stamp or "valid
until" time on the particular data link information
displayed in the flightdeck. For example, since initial
processing and transmission of NEXRAD data can
take several minutes, pilots must assume that data
link weather information will always be a minimum
of seven to eight minutes older than shown on the time
stamp and only use data link weather radar images for
broad strategic avoidance of adverse weather.

 	 Product update cycles—be aware of when and how
often a product is updated as well as the Data Link
Service Providers (DLSP) update rate for particular
products.
 	 Indication of system failure—be aware of partial or
total system failure indications.


Coverage areas/service volume—coverage limitations
are associated with the type of data link network
being used. For example, ground-based systems that
require a line-of-sight may have relatively limited
coverage below 5,000 feet AGL. Satellite-based data
link weather systems can have limitations stemming
from whether the network is in geosynchronous orbit
or low earth orbit. Also, NWS NEXRAD coverage
has gaps, especially in the western states.

 	 Content/format—since service providers often refine
or enhance data link products for flightdeck display,
pilots must be familiar with the content, format, and
meaning of symbols and displays (i.e., the legend) in
the specific system.


Data integrity/limitations to use—reliability of
information depicted. Be aware of any applicable
disclaimer provided by the service provider.



Use of equipment/avionics display—pilots remain
responsible for the proper use of an electronic flight
bag (EFB) or installed avionics. Pilots should be
cognizant that, per the FAA Practical Test Standards,
they may be evaluated on the use and interpretation
of an EFB or installed avionics on the aircraft.



Overload of Information—most DLSPs offer
numerous products with information that can be
layered on top of each other. Pilots need to be aware
that too much information can have a negative effect
on their cognitive work load. Pilots need to manage the
amount of information to a level that offers the most
pertinent information to that specific flight without

13-24

creating a distraction. Pilots may need to adjust the
amount of information based on numerous factors
including, but not limited to, the phase of flight, single
pilot operation, autopilot availability, class of airspace,
and the weather conditions encountered.

Chapter Summary
While no weather forecast is guaranteed to be 100 percent
accurate, pilots have access to a myriad of weather information
on which to base flight decisions. Weather products available
for preflight planning to en route information received over
the radio or via data link provide the pilot with the most
accurate and up-to-date information available. Each report
provides a piece of the weather puzzle. Pilots must use several
reports to get an overall picture and gain an understanding
of the weather that affects the safe completion of a flight.

Chapter 14

Airport
Operations
Introduction
Each time a pilot operates an aircraft, the flight normally
begins and ends at an airport. An airport may be a small sod
field or a large complex utilized by air carriers. This chapter
examines airport operations, identifies features of an airport
complex, and provides information on operating on or in the
vicinity of an airport.

Airport Categories
The definition for airports refers to any area of land or water
used or intended for landing or takeoff of aircraft. This
includes, within the five categories of airports listed below,
special types of facilities including seaplane bases, heliports,
and facilities to accommodate tilt rotor aircraft. An airport
includes an area used or intended for airport buildings,
facilities, as well as rights of way together with the buildings
and facilities.

14-1

The law defines airports by categories of airport activities,
including commercial service, primary, cargo service,
reliever, and general aviation airports, as shown below:


Commercial Service Airports—publicly owned
airports that have at least 2,500 passenger boardings
each calendar year and receive scheduled passenger
service. Passenger boardings refer to revenue passenger
boardings on an aircraft in service in air commerce
whether or not in scheduled service. The definition
also includes passengers who continue on an aircraft
in international flight that stops at an airport in any of
the 50 States for a non-traffic purpose, such as refueling
or aircraft maintenance rather than passenger activity.
Passenger boardings at airports that receive scheduled
passenger service are also referred to as Enplanements.



Cargo Service Airports—airports that, in addition to any
other air transportation services that may be available,
are served by aircraft providing air transportation of
only cargo with a total annual landed weight of more
than 100 million pounds. "Landed weight" means the
weight of aircraft transporting only cargo in intrastate,
interstate, and foreign air transportation. An airport
may be both a commercial service and a cargo service
airport.



Reliever Airports—airports designated by the FAA
to relieve congestion at Commercial Service Airports
and to provide improved general aviation access to the
overall community. These may be publicly or privatelyowned.



General Aviation Airports — the remaining airports
are commonly described as General Aviation Airports.
This airport type is the largest single group of airports
in the U.S. system. The category also includes privately
owned, public use airports that enplane 2500 or more
passengers annually and receive scheduled airline
service.

Types of Airports
There are two types of airports—towered and nontowered.
These types can be further subdivided to:


Civil Airports—airports that are open to the general
public.



Military/Federal Government airports—airports
operated by the military, National Aeronautics and
Space Administration (NASA), or other agencies of
the Federal Government.



14-2

Private Airports—airports designated for private or
restricted use only, not open to the general public.

Towered Airport
A towered airport has an operating control tower. Air traffic
control (ATC) is responsible for providing the safe, orderly,
and expeditious flow of air traffic at airports where the
type of operations and/or volume of traffic requires such a
service. Pilots operating from a towered airport are required
to maintain two-way radio communication with ATC and to
acknowledge and comply with their instructions. Pilots must
advise ATC if they cannot comply with the instructions issued
and request amended instructions. A pilot may deviate from
an air traffic instruction in an emergency, but must advise
ATC of the deviation as soon as possible.

Nontowered Airport
A nontowered airport does not have an operating control
tower. Two-way radio communications are not required,
although it is a good operating practice for pilots to transmit
their intentions on the specified frequency for the benefit
of other traffic in the area. The key to communicating at an
airport without an operating control tower is selection of the
correct common frequency. The acronym CTAF, which stands
for Common Traffic Advisory Frequency, is synonymous
with this program. A CTAF is a frequency designated for
the purpose of carrying out airport advisory practices while
operating to or from an airport without an operating control
tower. The CTAF may be a Universal Integrated Community
(UNICOM), MULTICOM, Flight Service Station (FSS), or
tower frequency and is identified in appropriate aeronautical
publications. UNICOM is a nongovernment air/ground radio
communication station that may provide airport information
at public use airports where there is no tower or FSS. On pilot
request, UNICOM stations may provide pilots with weather
information, wind direction, the recommended runway, or
other necessary information. If the UNICOM frequency
is designated as the CTAF, it is identified in appropriate
aeronautical publications. Figure 14-1 lists recommended
communication procedures. More information regarding
radio communications is provided later in this chapter.
Nontowered airport traffic patterns are always entered at
pattern altitude. How you enter the pattern depends upon the
direction of arrival. The preferred method for entering from
the downwind side of the pattern is to approach the pattern
on a course 45 degrees to the downwind leg and join the
pattern at midfield.
There are several ways to enter the pattern if you're coming
from the upwind leg side of the airport. One method of entry
from the opposite side of the pattern is to announce your
intentions and cross over midfield at least 500 feet above

Communication/Broadcast Procedures
Facility at Airport

Frequency Use

Outbound

Inbound

Practice Instrument
Approach

UNICOM
(no tower or FSS)

Communicate with UNICOM
station on published CTAF
frequency (122.7, 122.8, 122.725,
122.975, or 123.0). If unable to
contact UNICOM station, use selfannounce procedures on CTAF.

Before taxiing and
before taxiing on the
runway for departure.

10 miles out.
Entering downwind,
base, and final.
Leaving the runway.

No tower, FSS,
or UNICOM

Self-announce on MULTICOM
frequency 122.9.

Before taxiing and
before taxiing on the
runway for departure.

10 miles out.
Entering downwind,
base, and final.
Leaving the runway.

Departing final
approach fix (name)
or on final approach
segment inbound.

No tower in
operation, FSS open

Communicate with FSS on CTAF
frequency.

Before taxiing and
before taxiing on the
runway for departure.

10 miles out.
Entering downwind,
base, and final.
Leaving the runway.

Approach
completed/terminated.

FSS closed
(no tower)

Self-announce on CTAF.

Before taxiing and
before taxiing on the
runway for departure.

10 miles out.
Entering downwind,
base, and final.
Leaving the runway.

Tower or FSS
not in operation

Self-announce on CTAF.

Before taxiing and
before taxiing on the
runway for departure.

10 miles out.
Entering downwind,
base, and final.
Leaving the runway.

Figure 14-1. Recommended communication procedures.

pattern altitude (normally 1,500 feet AGL.) However, if large or
turbine aircraft operate at your airport, it is best to remain 2,000
feet AGL so you are not in conflict with their traffic pattern.
When well clear of the pattern—approximately 2 miles–scan
carefully for traffic, descend to pattern altitude, then turn right
to enter at 45° to the downwind leg at midfield. [Figure 14-2]
An alternate method is to enter on a midfield crosswind at
pattern altitude, carefully scan for traffic, announce your
intentions, and then turn downwind. [Figure 14-3] This
technique should not be used if the pattern is busy. Always
remember to give way to aircraft on the preferred 45° entry
and to aircraft already established on downwind.
In either case, it is vital to announce your intentions, and
remember to scan outside. Before joining the downwind
leg, adjust your course or speed to blend into the traffic.
Adjust power on the downwind leg, or sooner, to fit into
the flow of traffic. Avoid flying too fast or too slow. Speeds
recommended by the airplane manufacturer should be used.
They will generally fall between 70 to 80 knots for fixed-gear
singles and 80 to 90 knots for high-performance retractable.

Sources for Airport Data
When a pilot flies into a different airport, it is important to
review the current data for that airport. This data provides the

pilot with information, such as communication frequencies,
services available, closed runways, or airport construction.
Three common sources of information are:


Aeronautical Charts



Chart Supplement U.S. (formerly Airport/Facility
Directory)



Notices to Airmen (NOTAMs)



Automated Terminal Information Service (ATIS)

Aeronautical Charts
Aeronautical charts provide specific information on airports.
Chapter 16, "Navigation," contains an excerpt from an
aeronautical chart and an aeronautical chart legend, which
provides guidance on interpreting the information on the chart.
Chart Supplement U.S. (formerly Airport/Facility
Directory)
The Chart Supplement U.S. (formerly Airport/Facility
Directory) provides the most comprehensive information on
a given airport. It contains information on airports, heliports,
and seaplane bases that are open to the public. The Chart
Supplement U.S. is published in seven books, which are
organized by regions and are revised every 56 days. The
Chart Supplement U.S. is also available digitally.

14-3

Pattern altitude

1

Pattern altitude
+500 feet
2

4

Yield to
downwind
traffic and
enter
midfield
downwind
at 45°

3

Fly clear of
traffic pattern
(approx. 2 mi.)

Descend to
pattern altitude,
then turn

Figure 14-2. Preferred Entry-Crossing Midfield.

Yield to the preferred
45° and downwind
traffic, then turn
downwind

Figure 14-3. Alternate Midfield Entry.

faa.gov/air_traffic/flight_info/aeronav. Figure 14-4 contains
an excerpt from a directory. For a complete listing
of information provided in a Chart Supplement U.S.
and how the information may be decoded, refer to the
"Legend Sample" located in the front of each Chart
Supplement U.S.
In addition to airport information, each Chart Supplement
U.S. contains information such as special notices, Federal
Aviation Administration (FAA) and National Weather
Service (NWS) telephone numbers, preferred instrument
flight rules (IFR) routing, visual flight rules (VFR) waypoints,
a listing of very high frequency (VHF) omnidirectional range
(VOR) receiver checkpoints, aeronautical chart bulletins, land
and hold short operations (LAHSO) for selected airports,
airport diagrams for selected towered airports, en route
flight advisory service (EFAS) outlets, parachute jumping
areas, and facility telephone numbers. It is beneficial
to review a Chart Supplement U.S. to become familiar
with the information it contains.
Notices to Airmen (NOTAM)
Time-critical aeronautical information, which is of a temporary
nature or not sufficiently known in advance to permit
publication, on aeronautical charts or in other operational
14-4

Figure 14-4. Chart Supplement U.S. (formerly Airport/Facility
Directory excerpt.

publications receives immediate dissemination by the
NOTAM system. The NOTAM information could affect your
decision to make the flight. It includes such information as
taxiway and runway closures, construction, communications,
changes in status of navigational aids, and other information
essential to planned en route, terminal, or landing operations.
Exercise good judgment and common sense by carefully
regarding the information readily available in NOTAMs.
Prior to any flight, pilots should check for any NOTAMs that
could affect their intended flight. For more information on
NOTAMs, refer back to Chapter 1, "Pilot and Aeronautical
Information" section.
Automated Terminal Information Service (ATIS)
The Automated Terminal Information Service (ATIS) is a
recording of the local weather conditions and other pertinent
non-control information broadcast on a local frequency in a
looped format. It is normally updated once per hour but is
updated more often when changing local conditions warrant.
Important information is broadcast on ATIS including
weather, runways in use, specific ATC procedures, and any
airport construction activity that could affect taxi planning.
When the ATIS is recorded, it is given a code. This code is
changed with every ATIS update. For example, ATIS Alpha
is replaced by ATIS Bravo. The next hour, ATIS Charlie is
recorded, followed by ATIS Delta and progresses down the
alphabet.
Prior to calling ATC, tune to the ATIS frequency and listen to
the recorded broadcast. The broadcast ends with a statement
containing the ATIS code. For example, "Advise on initial
contact, you have information Bravo." Upon contacting the
tower controller, state information Bravo was received. This
allows the tower controller to verify the pilot has the current
local weather and airport information without having to
state it all to each pilot who calls. This also clears the tower
frequency from being overtaken by the constant relay of
the same information, which would result without an ATIS
broadcast. The use of ATIS broadcasts at departure and arrival
airports is not only a sound practice but a wise decision.

Airport Markings and Signs
There are markings and signs used at airports that provide
directions and assist pilots in airport operations. It is important
for you to know the meanings of the signs, markings, and lights
that are used on airports as surface navigational aids. All airport
markings are painted on the surface, whereas some signs are
vertical and some are painted on the surface. An overview of the
most common signs and markings are described on the following
pages. Additional information may be found in Chapter 2,

"Aeronautical Lighting and Other Airport Visual Aids," of
the Aeronautical Information Manual (AIM).
Runway Markings and Signs
Runway markings vary depending on the type of operations
conducted at the airport. A basic VFR runway may only
have centerline markings and runway numbers. Refer to
Appendix C of this publication for an example of the most
common runway markings that are found at airports.
Since aircraft are affected by the wind during takeoffs and
landings, runways are laid out according to the local prevailing
winds. Runway numbers are in reference to magnetic north.
Certain airports have two or even three runways laid out in the
same direction. These are referred to as parallel runways and
are distinguished by a letter added to the runway number (e.g.,
runway 36L (left), 36C (center), and 36R (right)).

Relocated Runway Threshold
It is sometimes necessary, due to construction or runway
maintenance, to close only a portion of a runway. When
a portion of a runway is closed, the runway threshold is
relocated as necessary. It is referred to as a relocated threshold
and methods for identifying the relocated threshold vary. A
common way for the relocated threshold to be marked is
a ten foot wide white bar across the width of the runway.
[Figure 14-5A and B]
When the threshold is relocated, the closed portion of the
runway is not available for use by aircraft for takeoff or
landing, but it is available for taxi. When a threshold is
relocated, it closes not only a set portion of the approach
end of a runway, but also shortens the length of the opposite
direction runway. Yellow arrow heads are placed across the
width of the runway just prior to the threshold bar.

Displaced Threshold
A displaced threshold is a threshold located at a point on
the runway other than the designated beginning of the
runway. Displacement of a threshold reduces the length
of runway available for landings. The portion of runway
behind a displaced threshold is available for takeoffs in either
direction, or landings from the opposite direction. A ten feet
wide white threshold bar is located across the width of the
runway at the displaced threshold, and white arrows are
located along the centerline in the area between the beginning
of the runway and displaced threshold. White arrow heads
are located across the width of the runway just prior to the
threshold bar. [Figure 14-6A and B]

14-5

A

B

36

Figure 14-5. (A) Relocated runway threshold drawing. (B) Relocated threshold for Runway 36 at Joplin Regional Airport (JLN).

Runway Safety Area

Runway Safety Area Boundary Sign

The runway safety area (RSA) is a defined surface
surrounding the runway prepared, or suitable, for reducing
the risk of damage to airplanes in the event of an undershoot,
overshoot, or excursion from the runway. The dimensions
of the RSA vary and can be determined by using the
criteria contained within AC 150/5300-13, Airport Design,
Chapter 3. Figure 3-1 in AC 150/5300-13 depicts the RSA.
Additionally, it provides greater accessibility for firefighting
and rescue equipment in emergency situations.

Some taxiway stubs also have a runway safety area boundary
sign that faces the runway and is visible to you only when
exiting the runway. This sign has a yellow background with
black markings and is typically used at towered airports
where a controller commonly requests you to report clear of
a runway. This sign is intended to provide you with another
visual cue that is used as a guide to determine when you are
clear of the runway safety boundary area. The sign shown in
Figure 14-8 is what you would see when exiting the runway at
Taxiway Kilo. You are out of the runway safety area boundary
when the entire aircraft passes the sign and the accompanying
surface painted marking.

The RSA is typically graded and mowed. The lateral
boundaries are usually identified by the presence of the
runway holding position signs and markings on the adjoining
taxiway stubs. Aircraft should not enter the RSA without
making sure of adequate separation from other aircraft during
operations at uncontrolled airports. [Figure 14-7]

14-6

Runway Holding Position Sign
Noncompliance with a runway holding position sign may
result in the FAA filing a Pilot Deviation against you. A

A

B

×

×

17
Figure 14-6. (A) Displaced runway threshold drawing. (B) Displaced threshold for Runway 17 at Albuquerque International Airport (ABQ).

runway holding position sign is an airport version of a
stop sign. [Figure 14-9] It may be seen as a sign and/or its
characters painted on the airport pavement. The sign has
white characters outlined in black on a red background. It is
always collocated with the surface painted holding position
markings and is located where taxiways intersect runways.
On taxiways that intersect the threshold of the takeoff runway,
only the designation of the runway may appear on the sign.
If a taxiway intersects a runway somewhere other than at
the threshold, the sign has the designation of the intersecting
runway. The runway numbers on the sign are arranged to
correspond to the relative location of the respective runway
thresholds. Figure 14-10 shows "18-36" to indicate the
threshold for Runway 18 is to the left and the threshold for

Figure 14-8. Runway safety area boundary sign and marking located
on Taxiway Kilo.

C 9

9
A 9

Figure 14-7. Runway Safety Area.

27

Typical Runway Safety Area

B 27

Figure 14-9. Runway holding position sign at takeoff end of Runway
14 with collocated Taxiway Alpha location sign.

14-7

Runway Holding Position Marking
Noncompliance with a runway holding position marking
may result in the FAA filing a Pilot Deviation against you.
Runway holding position markings consist of four yellow
lines, two solid and two dashed, that are painted on the surface
and extend across the width of the taxiway to indicate where
the aircraft should stop when approaching a runway. These
markings are painted across the entire taxiway pavement, are
in alignment, and are collocated with the holding position
sign as described above.
Figure 14-10. Runway holding position sign at a location other

than the takeoff end of Runway 18-36 with collocated Taxiway
Alpha location sign.

Runway 36 is to the right. The sign also indicates that you
are located on Taxiway Alpha.
If the runway holding position sign is located on a taxiway
at the intersection of two runways, the designations for
both runways are shown on the sign along with arrows
showing the approximate alignment of each runway.
[Figure 14-11A and B] In addition to showing the
approximate runway alignment, the arrows indicate the
direction(s) to the threshold of the runway whose designation
is immediately next to each corresponding arrow.
This type of taxiway and runway/runway intersection
geometry can be very confusing and create navigational
challenges. Extreme caution must be exercised when taxiing
onto or crossing this type of intersection. Figure 14-11A and
B shows a depiction of a taxiway, runway/runway intersection
and is also designated as a "hot spot" on the airport diagram.
In the example, Taxiway Bravo intersects with two runways,
31-13 and 35-17, which cross each other.
Surface painted runway holding position signs may also be
used to aid you in determining the holding position. These
markings consist of white characters on a red background
and are painted on the left side of the taxiway centerline.
Figure 14-12 shows a surface painted runway holding
position sign that is the holding point for Runway 32R-14L.
You should never allow any part of your aircraft to cross the
runway holding position sign (either a vertical or surface
painted sign) without a clearance from ATC. Doing so poses
a hazard to yourself and others.
When the tower is closed or you are operating at a nontowered
airport, you may taxi past a runway holding position sign only
when the runway is clear of aircraft, and there are no aircraft on
final approach. You may then proceed with extreme caution.

14-8

As you approach the runway, two solid yellow lines and two
dashed lines will be visible. Prior to reaching the solid lines, it
is imperative to stop and ensure that no portion of the aircraft
intersects the first solid yellow line. Do not cross the double
solid lines until a clearance from ATC has been received.
[Figure 14-13] When the tower is closed or when operating at
a nontowered airport, you may taxi onto or across the runway
only when the runway is clear and there are no aircraft on final
approach. You should use extreme caution when crossing or
taxiing onto the runway and always look both ways.
When exiting the runway, the same markings will be seen
except the aircraft will be approaching the double dashed
lines. [Figure 14-14] In order to be clear of the runway, the
entire aircraft must cross both the dashed and solid lines.
An ATC clearance is not needed to cross this marking when
exiting the runway.

Runway Distance Remaining Signs
Runway distance remaining signs have a black background
with a white number and may be installed along one or both
sides of the runway. [Figure 14-15] The number on the
signs indicates the distance, in thousands of feet, of landing
runway remaining. The last sign, which has the numeral "1,"
is located at least 950 feet from the runway end.

Runway Designation Marking
Runway numbers and letters are determined from the
approach direction. The runway number is the whole number
nearest one-tenth the magnetic azimuth of the centerline of the
runway, measured clockwise from the magnetic north. In the
case where there are parallel runways, the letters differentiate
between left (L), right (R), or center (C). [Figure 14-16] For
example, if there are two parallel runways, they would show
the designation number and then either L or R beneath it.
For three parallel runways, the designation number would
be presented with L, C, or R beneath it.

n
35-17

rn

31-13

35-17
31-13

No

tt

o

be

us

ed

fo

NC-3, 08 MAR 2012 to 05 APR 2012

NC-3, 08 MAR 2012 to 05 APR 2012

3

B

5

Runway 31

B

av
ig

Runway 13

at
io

A

Figure 14-11. (A) Taxiway Bravo location sign collocated with runway/runway intersection holding signs at Sioux Gateway Airport

(SUX) (B) Airport diagram of Sioux Gateway Airport (SUX), Sioux City, Iowa. The area outlined in red is a designated "hot spot" (HS1).

14-9

Figure 14-12. Surface painted runway holding position signs for
Runway 32R-14L along with the enhanced taxiway centerline marking.
Figure 14-15. Runway distance remaining sign indicating that there
is 2,000 feet of runway remaining.

compliance. As pilot in command (PIC), you have the final
authority to accept or decline any LAHSO clearance.
If issued a land and hold short clearance, you must be aware
of the reduced runway distances and whether or not you can
comply before accepting the clearance. You do not have
to accept a LAHSO clearance. Pilots should only receive a
LAHSO clearance when there is a minimum ceiling of 1,000
feet and 3 statute miles of visibility.

Figure 14-13. Surface painted holding position marking along with

enhanced taxiway centerline.

Runway holding position signs and markings are installed
on those runways used for LAHSO. The signs and markings
are placed at the LAHSO point to aid you in determining
where to stop and hold the aircraft and are located prior to
the runway/runway intersection. [Figure 14-17]
The holding position sign has a white inscription with black
border around the numbers on a red background and is installed
adjacent to the holding position markings. If you accept a land
and hold short clearance, you must comply so that no portion
of the aircraft extends beyond these hold markings.

Figure 14-14. Runway holding position markings as seen when
exiting the runway. When exiting the runway, no ATC clearance
is required to cross.

Land and Hold Short Operations (LAHSO)
When simultaneous operations (takeoffs and landings) are
being conducted on intersecting runways, Land and Hold
Short Operations (LAHSO) may also be in effect. LAHSO
is an ATC procedure that may require your participation and
14-10

If receiving "cleared to land" instructions from ATC, you
are authorized to use the entire landing length of the runway
and should disregard any LAHSO holding position markings
located on the runway. If you receive and accept LAHSO
instructions, you must stop short of the intersecting runway
prior to the LAHSO signs and markings.
Below is a list of items which, if thoroughly understood
and complied with, will ensure that LAHSO operations are
conducted properly.


Know landing distance available.



Be advised by ATC as to why LAHSO are being
conducted.

35
L

35
C

Figure 14-16. Two of three parallel runways.



Advise ATC if you cannot comply with LAHSO.



Know what signs and markings are at the LAHSO point.



LAHSO are not authorized for student pilots who are
performing a solo flight.



At many airports air carrier aircraft are not authorized
to participate in LAHSO if the other aircraft is a
general aviation aircraft.



Generally, LAHSO are not authorized at night.



LAHSO are not authorized on wet runways.

If you accept the following clearance from ATC: "Cleared to
land Runway 36 hold short of Runway 23," you must either exit
Runway 36 or stop at the holding position prior to Runway 23.
Taxiway Markings and Signs
Taxiway direction signs have a yellow background and
black characters, which identifies the designation or
intersecting taxiways. Arrows indicate the direction of turn
that would place the aircraft on the designated taxiway.
[Figure 14-18] Direction signs are normally located on
the left side of the taxiway and prior to the intersection.
These signs and markings (with a yellow background and
black characters) indicate the direction toward a different
taxiway, leading off a runway, or out of an intersection.
Figure 14-18 shows Taxiway Delta and how Taxiway Bravo
intersects ahead at 90° both left and right.
Taxiway direction signs can also be displayed as surface
painted markings. Figure 14-19 shows Taxiway Bravo as
proceeding straight ahead while Taxiway Alpha turns to the
right at approximately 45°.

Figure 14-17. Runway holding position sign and marking for LAHSO.

14-11

Figure 14-18. Taxiway Bravo direction sign with a collocated
Taxiway Delta location sign. When the arrow on the direction
sign indicates a turn, the sign is located prior to the intersection.

Figure 14-20A and B shows an example of a direction sign at
a complex taxiway intersection. Figure 14-20A and B shows
Taxiway Bravo intersects with Taxiway Sierra at 90°, but at
45° with Taxiway Foxtrot. This type of array can be displayed
with or without the taxiway location sign, which in this case
would be Taxiway Bravo.

Enhanced Taxiway Centerline Markings
At most towered airports, the enhanced taxiway centerline
marking is used to warn you of an upcoming runway. It consists
of yellow dashed lines on either side of the normal solid taxiway
centerline and the dashes extend up to 150 feet prior to a
runway holding position marking. [Figure 14-21A and B] They
are used to aid you in maintaining awareness during surface
movement to reduce runway incursions.

Destination Signs
Destination signs have black characters on a yellow
background indicating a destination at the airport. These

signs always have an arrow showing the direction of the taxi
route to that destination. [Figure 14-22] When the arrow on
the destination sign indicates a turn, the sign is located prior
to the intersection. Destinations commonly shown on these
types of signs include runways, aprons, terminals, military
areas, civil aviation areas, cargo areas, international areas,
and fixed-base operators. When the inscription for two or
more destinations having a common taxi route are placed
on a sign, the destinations are separated by a "dot" () and
one arrow would be used as shown in Figure 14-22. When
the inscription on a sign contains two or more destinations
having different taxi routes, each destination is accompanied
by an arrow and separated from the other destination(s) on
the sign with a vertical black message divider as shown in
Figure 14-23. The example shown in Figure 14-23 shows
two signs. The sign in the foreground explains that Runway
20 threshold is to the left, and Runways 32, 2, and 14 are to
the right. The sign in the background indicates that you are
located on Taxiway Bravo and Taxiway November will take
you to those runways.

Holding Position Signs and Markings for an
Instrument Landing System (ILS) Critical Area
The instrument landing system (ILS) broadcasts signals to
arriving instrument aircraft to guide them to the runway. Each
of these ILSs have critical areas that must be kept clear of all
obstacles in order to ensure quality of the broadcast signal. At
many airports, taxiways extend into the ILS critical area. Most
of the time, this is of no concern; however, during times of
poor weather, an aircraft on approach may depend on a good
signal quality. When necessary, ATC will protect the ILS
critical area for arrival instrument traffic by instructing taxiing
aircraft to "hold short" of Runway (XX) ILS critical area.
The ILS critical area hold sign has white characters, outlined
in black, on a red background and is installed adjacent to the
ILS holding position markings. [Figure 14-24] The holding
position markings for the ILS critical area appear on the
pavement as a horizontal yellow ladder extending across the
width of the taxiway.
When instructed to "hold short of Runway (XX) ILS critical
area," you must ensure no portion of the aircraft extends
beyond these markings. [Figure 14-25] If ATC does not
instruct you to hold at this point, then you may bypass the ILS
critical area hold position markings and continue with your
taxi. Figure 14-24 shows that the ILS hold sign is located
on Taxiway Golf and the ILS ladder hold position marking
is adjacent to the hold sign.

Figure 14-19. Surface painted taxiway direction signs.

14-12

A

F

S

S
A

S

F

B

S

F

F
B

Figure 14-20. Orientation of signs is from left to right in a clockwise manner.
Left turn
signs
are on the
left and
right turn
right.
Enhanced
taxiway
centerline
marking
extends
150 on
feetthe
prior
to a runway holding position marking. Prepare to STOP.
In this view, the pilot is on Taxiway Bravo.

B

A

B

Enhanced taxiway centerline marking extends 150 feet prior
to a runway holding position marking. Prepare to STOP.

Prepare to STOP unless you have been cleared onto
or across the runway by ATC.

B
Figure 14-21. (A) Enhanced taxiway centerline marking. (B) Enhanced taxiway centerline marking and runway holding position marking.

14-13

19
The yellow surface
painted "ladder"
marking and red ILS
sign are located
on taxiways where
the taxiways intersect
the ILS critical area.

I LS

Crit

Figure 14-22. Destination sign to the fixed-base operator (FBO).

ical
Are
a bo
und

ary

ILS

Hold only
when
specifically
instructed by ATC

Figure 14-25. Holding position sign and marking for instrument

landing system (ILS) critical area boundary.

Holding Position Markings for Taxiway/Taxiway
Intersections

Figure 14-23. Runway destination sign with different taxi routes.

Holding position markings for taxiway/taxiway intersections
consist of a single dashed yellow line extending across the
width of the taxiway. [Figure 14-26] They are painted on
taxiways where ATC normally holds aircraft short of a
taxiway intersection. When instructed by ATC "hold short
of Taxiway X," you should stop so that no part of your
aircraft extends beyond the holding position marking. When
the marking is not present, you should stop your aircraft at
a point that provides adequate clearance from an aircraft on
the intersecting taxiway.

Marking and Lighting of Permanently Closed
Runways and Taxiways
For runways and taxiways that are permanently closed, the
lighting circuits are disconnected. The runway threshold,
runway designation, and touchdown markings are obliterated
and yellow "Xs" are placed at each end of the runway and
at 1,000-foot intervals.
Figure 14-24. Instrument landing system (ILS) holding position sign
and marking on Taxiway Golf.

14-14

G

B

A

B

G

B

B

G

G

G

Figure 14-26. Holding position marking on a taxiway.

Temporarily Closed Runways and Taxiways

B

For temporarily closed runways and taxiways, a visual
indication is often provided with yellow "Xs" or raised
lighted yellow "Xs" placed at each end of the runway.
Depending on the reason for the closure, duration of closure,
airfield configuration, and the existence and the hours of
operation of an ATC tower, a visual indication may not be
present. As discussed previously in the chapter, you must
always check NOTAMs and ATIS for runway and taxiway
closure information.
Figure 14-27A shows an example of a yellow "X" laid flat
with an adequate number of heavy sand bags to keep the
wind from getting under and displacing the vinyl material.

C

A very effective and preferable visual aid to depict temporary
closure is the lighted "X" placed on or near the runway
designation numbers. [Figure 14-27B and C] This device is
much more discernible to approaching aircraft than the other
materials described above.
Other Markings
Some other markings found on the airport include vehicle
roadway markings, VOR receiver checkpoint markings, and
non-movement area boundary markings.
Airport Signs
There are six types of signs that may be found at airports. The
more complex the layout of an airport, the more important
the signs become to pilots. Appendix C of this publication
shows examples of some signs that are found at most airports,
their purpose, and appropriate pilot action. The six types of
signs are:


Mandatory instruction signs—red background with
white inscription. These signs denote an entrance to a
runway, critical area, or prohibited area.

Figure 14-27. (A) Yellow "X" placed on surface of temporarily closed
runways. (B) Lighted "X" placed on temporarily closed runways.
(C) Lighted "X" at night showing a temporarily closed runway.



Location signs—black with yellow inscription and a
yellow border, no arrows. They are used to identify a
taxiway or runway location, to identify the boundary
of the runway, or identify an instrument landing
system (ILS) critical area.

14-15



Direction signs—yellow background with black
inscription. The inscription identifies the designation
of the intersecting taxiway(s) leading out of an
intersection.



Destination signs—yellow background with black
inscription and arrows. These signs provide information
on locating areas, such as runways, terminals, cargo
areas, and civil aviation areas.



Information signs—yellow background with black
inscription. These signs are used to provide the pilot
with information on areas that cannot be seen from the
control tower, applicable radio frequencies, and noise
abatement procedures. The airport operator determines
the need, size, and location of these signs.



Runway distance remaining signs—black background
with white numbers. The numbers indicate the
distance of the remaining runway in thousands of feet.

Airport Lighting
The majority of airports have some type of lighting for night
operations. The variety and type of lighting systems depends
on the volume and complexity of operations at a given airport.
Airport lighting is standardized so that airports use the same
light colors for runways and taxiways.
Airport Beacon
Airport beacons help a pilot identify an airport at night.
The beacons are normally operated from dusk until dawn.
Sometimes they are turned on if the ceiling is less than 1,000
feet and/or the ground visibility is less than 3 statute miles (VFR
minimums). However, there is no requirement for this, so a
pilot has the responsibility of determining if the weather meets
VFR requirements. The beacon has a vertical light distribution
to make it most effective from 1–10° above the horizon,
although it can be seen well above or below this spread. The
beacon may be an omnidirectional capacitor-discharge device,
or it may rotate at a constant speed, that produces the visual
effect of flashes at regular intervals. The combination of light
colors from an airport beacon indicates the type of airport.
[Figure 14-28] Some of the most common beacons are:


Flashing white and green for civilian land airports



Flashing white and yellow for a water airport



Flashing white, yellow, and green for a heliport



Two quick white flashes alternating with a green flash
identifying a military airport

Approach Light Systems
Approach light systems are primarily intended to provide a
means to transition from instrument flight to visual flight for
landing. The system configuration depends on whether the
14-16

White

Green

White

Yellow

Figure 14-28. Airport rotating beacons.

runway is a precision or nonprecision instrument runway.
Some systems include sequenced flashing lights that appear
to the pilot as a ball of light traveling toward the runway at
high speed. Approach lights can also aid pilots operating
under VFR at night.
Visual Glideslope Indicators
Visual glideslope indicators provide the pilot with glidepath
information that can be used for day or night approaches. By
maintaining the proper glidepath as provided by the system,
a pilot should have adequate obstacle clearance and should
touch down within a specified portion of the runway.

Visual Approach Slope Indicator (VASI)
VASI installations are the most common visual glidepath
systems in use. The VASI provides obstruction clearance
within 10° of the extended runway centerline and up to four
nautical miles (NM) from the runway threshold.
The VASI consists of light units arranged in bars. There are
2-bar and 3-bar VASIs. The 2-bar VASI has near and far light
bars and the 3-bar VASI has near, middle, and far light bars.
Two-bar VASI installations provide one visual glidepath
that is normally set at 3°. The 3-bar system provides two
glidepaths, the lower glidepath normally set at 3° and the
upper glidepath ¼ degree above the lower glidepath.
The basic principle of the VASI is that of color differentiation
between red and white. Each light unit projects a beam of
light, a white segment in the upper part of the beam and a
red segment in the lower part of the beam. The lights are
arranged so the pilot sees the combination of lights shown in
Figure 14-29 to indicate below, on, or above the glidepath.

Other Glidepath Systems
A precision approach path indicator (PAPI) uses lights similar
to the VASI system, except they are installed in a single row,
normally on the left side of the runway. [Figure 14-30]

Below Glidepath

On Glidepath

Above Glidepath

Far Bar

Far Bar

Far Bar

Near Bar

Near Bar

Near Bar

light. The "slightly below glidepath" indication is a steady red
light. If the aircraft descends further below the glidepath, the
red light starts to pulsate. The "above glidepath" indication
is a pulsating white light. The pulsating rate increases as the
aircraft gets further above or below the desired glideslope.
The useful range of the system is about four miles during the
day and up to ten miles at night. [Figure 14-32]
Runway Lighting
There are various lights that identify parts of the runway
complex. These assist a pilot in safely making a takeoff or
landing during night operations.

Runway End Identifier Lights (REIL)

Figure 14-29. Two-bar VASI system.

A tri-color system consists of a single-light unit projecting
a three-color visual approach path. Below the glidepath is
indicated by red, on the glidepath is indicated by green, and
above the glidepath is indicated by amber. When descending
below the glidepath, there is a small area of dark amber. Pilots
should not mistake this area for an "above the glidepath"
indication. [Figure 14-31]
Pulsating VASIs normally consist of a single-light unit
projecting a two-color visual approach path into the final
approach area of the runway upon which the indicator is
installed. The "on glidepath" indication is a steady white

High
more than 3.5°

Slightly High
3.2°

Runway end identifier lights (REIL) are installed at many
airfields to provide rapid and positive identification of the
approach end of a particular runway. The system consists
of a pair of synchronized flashing lights located laterally
on each side of the runway threshold. REILs may be either
omnidirectional or unidirectional facing the approach area.

Runway Edge Lights
Runway edge lights are used to outline the edges of
runways at night or during low visibility conditions.
[Figure 14-33] These lights are classified according to the
intensity they are capable of producing: high intensity runway
lights (HIRL), medium intensity runway lights (MIRL), and

On Glidepath
3°

Slightly Low
2.8°

Low
less than 2.5°

Figure 14-30. Precision approach path indicator for a typical 3° glide slope.

depath
Above gli
On glidepath
Below glidepath

r
be
m
A n
ee
Gr
d
Re

Amber

Figure 14-31. Tri-color visual approach slope indicator.

14-17

ting
lu saite
P wh
Above

dy
tS eate
i
wh
red
nti g
a
uls

th

glidepa

On glidepath
Slightly below glidepath
Below glidepath

Steady
red

P

Threshold
Figure 14-32. Pulsating visual approach slope indicator.

low intensity runway lights (LIRL). The HIRL and MIRL
have variable intensity settings. These lights are white, except
on instrument runways where amber lights are used on the
last 2,000 feet or half the length of the runway, whichever
is less. The lights marking the end of the runway are red.

In-Runway Lighting
Runway centerline lighting system (RCLS)—installed on some
precision approach runways to facilitate landing under adverse
visibility conditions. They are located along the runway
centerline and are spaced at 50-foot intervals. When viewed
from the landing threshold, the runway centerline lights are
white until the last 3,000 feet of the runway. The white lights
begin to alternate with red for the next 2,000 feet. For the
remaining 1,000 feet of the runway, all centerline lights are red.
Touchdown zone lights (TDZL)—installed on some precision
approach runways to indicate the touchdown zone when
landing under adverse visibility conditions. They consist of
two rows of transverse light bars disposed symmetrically
about the runway centerline. The system consists of steadyburning white lights that start 100 feet beyond the landing
threshold and extend to 3,000 feet beyond the landing
threshold or to the midpoint of the runway, whichever is less.

Figure 14-33. Runway lights.

14-18

Taxiway centerline lead-off lights—provide visual guidance
to persons exiting the runway. They are color-coded to warn
pilots and vehicle drivers that they are within the runway
environment or ILS critical area, whichever is more restrictive.
Alternate green and yellow lights are installed, beginning
with green, from the runway centerline to one centerline light
position beyond the runway holding position or ILS critical
area holding position.
Taxiway centerline lead-on lights—provide visual guidance
to persons entering the runway. These "lead-on" lights are
also color-coded with the same color pattern as lead-off
lights to warn pilots and vehicle drivers that they are within
the runway environment or ILS critical area, whichever is
more conservative. The fixtures used for lead-on lights are
bidirectional (i.e., one side emits light for the lead-on function
while the other side emits light for the lead-off function). Any
fixture that emits yellow light for the lead-off function also
emits yellow light for the lead-on function.
Land and hold short lights—used to indicate the hold short
point on certain runways which are approved for LAHSO.
Land and hold short lights consist of a row of pulsing white
lights installed across the runway at the hold short point.
Where installed, the lights are on anytime LAHSO is in effect.
These lights are off when LAHSO is not in effect.
Control of Airport Lighting
Airport lighting is controlled by ATC at towered airports. At
nontowered airports, the lights may be on a timer, or where an
FSS is located at an airport, the FSS personnel may control the
lighting. A pilot may request various light systems be turned
on or off and also request a specified intensity, if available,
from ATC or FSS personnel. At selected nontowered airports,
the pilot may control the lighting by using the radio. This
is done by selecting a specified frequency and clicking the
radio microphone. [Figure 14-34] For information on pilot
controlled lighting at various airports, refer to the Chart
Supplement U.S. (formerly Airport/Facility Directory).

Key Mike

Function

7 times within 5 seconds

Highest intensity available

5 times within 5 seconds

Medium or lower intensity
(Lower REIL or REIL off)

3 times within 5 seconds

Lowest intensity available
(Lower REIL or REIL off)

Figure 14-34. Radio controlled runway lighting.

Taxiway Lights
Similar to runway lighting, taxiways also have various lights
which help pilots identify areas of the taxiway and any
surrounding runways.

red lights on each side. A controlled stop bar is operated in
conjunction with the taxiway centerline lead-on lights which
extend from the stop bar toward the runway. Following the
ATC clearance to proceed, the stop bar is turned off and the
lead-on lights are turned on. The stop bar and lead-on lights
are automatically reset by a sensor or backup timer.
Obstruction Lights
Obstructions are marked or lighted to warn pilots of
their presence during daytime and nighttime conditions.
Obstruction lighting can be found both on and off an airport
to identify obstructions. They may be marked or lighted in
any of the following conditions.


Red obstruction lights—flash or emit a steady red
color during nighttime operations, and the obstructions
are painted orange and white for daytime operations.



High intensity white obstruction lights—flash high
intensity white lights during the daytime with the
intensity reduced for nighttime.



Dual lighting—a combination of flashing red beacons
and steady red lights for nighttime operation and high
intensity white lights for daytime operations.

Omnidirectional
Omnidirectional taxiway lights outline the edges of the
taxiway and are blue in color. At many airports, these
edge lights may have variable intensity settings that may
be adjusted by an ATC when deemed necessary or when
requested by the pilot. Some airports also have taxiway
centerline lights that are green in color.

Clearance Bar Lights
Clearance bar lights are installed at holding positions on
taxiways in order to increase the conspicuity of the holding
position in low visibility conditions. They may also be
installed to indicate the location of an intersecting taxiway
during periods of darkness. Clearance bars consist of three
in-pavement steady-burning yellow lights.

Runway Guard Lights
Runway guard lights are installed at taxiway/runway
intersections. They are primarily used to enhance the
conspicuity of taxiway/runway intersections during low
visibility conditions, but may be used in all weather conditions.
Runway guard lights consist of either a pair of elevated flashing
yellow lights installed on either side of the taxiway, or a row of
in-pavement yellow lights installed across the entire taxiway,
at the runway holding position marking.
Note: Some airports may have a row of three or five
in-pavement yellow lights installed at taxiway/runway
intersections. They should not be confused with clearance
bar lights described previously in this section.

Stop Bar Lights
Stop bar lights, when installed, are used to confirm the ATC
clearance to enter or cross the active runway in low visibility
conditions (below 1,200 ft Runway Visual Range (RVR)).
A stop bar consists of a row of red, unidirectional, steadyburning in-pavement lights installed across the entire taxiway
at the runway holding position, and elevated steady-burning

New Lighting Technologies
A top priority of the FAA is to continue to enhance airport
safety while maintaining airport capacity. Reducing runway
incursions is a major component of this effort. Runway
incursions develop quickly and without warning during routine
traffic situations on the airport surface, leaving little time for
corrective action. The Runway Status Lights (RWSL) System
is designed to provide a direct indication to you that it is unsafe
to enter a runway, cross a runway, or takeoff from or land on
a runway when the system is activated.
Runway status lights are red in color and indicate runway
status only; they do not indicate clearance to enter a runway
or clearance to takeoff. The RWSL system provides warning
lights on runways and taxiways, illuminating when it is unsafe
to enter, cross, or begin takeoff on a runway. Currently, there
are two types: Runway Entrance Lights (REL) and Takeoff
Hold Lights (THL). [Figures 14-35 and 14-36]
REL provide a warning to aircraft crossing or entering a
runway from intersecting taxiways that there is conflicting
traffic on the runway. THL provide a warning signal to
aircraft in position for takeoff that the runway is occupied
and it is unsafe to take off. As of 2016, the RWSL system is
operational at 14 of the nation's busiest airports with 3 more
airports scheduled to receive the system by 2017.

14-19

out straighter in strong winds and tends to move back and
forth when the wind is gusting. Wind tees and tetrahedrons
can swing freely and align themselves with the wind direction.
Since a wind tee or tetrahedron can also be manually set to
align with the runway in use, a pilot should also look at the
wind sock for wind information, if one is available.

Traffic Patterns

Figure 14-35. Runway Entrance Lights (REL).

Figure 14-36. Takeoff Hold Lights (THL).

Wind Direction Indicators
It is important for a pilot to know the direction of the wind. At
facilities with an operating control tower, this information is
provided by ATC. Information may also be provided by FSS
personnel either located at a particular airport or remotely
available through a remote communication outlet (RCO), or
by requesting information on a CTAF at airports that have the
capacity to receive and broadcast on this frequency.
When none of these services is available, it is possible
to determine wind direction and runway in use by visual
wind indicators. A pilot should check these wind indicators
even when information is provided on the CTAF at a given
airport because there is no assurance that the information
provided is accurate.
The wind direction indicator can be a wind cone, wind sock,
tetrahedron, or wind tee. These are usually located in a central
location near the runway and may be placed in the center
of a segmented circle, which identifies the traffic pattern
direction if it is other than the standard left-hand pattern.
[Figures 14-37 and 14-38]
The wind sock is a good source of information since it not
only indicates wind direction but allows the pilot to estimate
the wind velocity and/or gust factor. The wind sock extends
14-20

At airports without an operating control tower, a segmented
circle visual indicator system, if installed, is designed to
provide traffic pattern information. [Figure 14-38] Usually
located in a position affording maximum visibility to pilots in
the air and on the ground and providing a centralized location
for other elements of the system, the segmented circle consists
of the following components: wind direction indicators,
landing direction indicators, landing strip indicators, and
traffic pattern indicators.
A tetrahedron is installed to indicate the direction of landings
and takeoffs when conditions at the airport warrant its use.
It may be located at the center of a segmented circle and
may be lighted for night operations. The small end of the
tetrahedron points in the direction of landing. Pilots are
cautioned against using a tetrahedron for any purpose other
than as an indicator of landing direction. At airports with
control towers, the tetrahedron should only be referenced
when the control tower is not in operation. Tower instructions
supersede tetrahedron indications.
Landing strip indicators are installed in pairs and are used to
show the alignment of landing strips. [Figure 14-38] Traffic
pattern indicators are arranged in pairs in conjunction with
landing strip indicators and used to indicate the direction of
turns when there is a variation from the normal left traffic
pattern. (If there is no segmented circle installed at the airport,
traffic pattern indicators may be installed on or near the end
of the runway.)
At most airports and military air bases, traffic pattern altitudes
for propeller-driven aircraft generally extend from 600 feet
to as high as 1,500 feet above ground level (AGL). Pilots
can obtain the traffic pattern altitude for an airport from the
Chart Supplement U.S. (formerly Airport/Facility Directory).
Also, traffic pattern altitudes for military turbojet aircraft
sometimes extend up to 2,500 feet AGL. Therefore, pilots of
en route aircraft should be constantly on alert for other aircraft
in traffic patterns and avoid these areas whenever possible.
When operating at an airport, traffic pattern altitudes should
be maintained unless otherwise required by the applicable
distance from cloud criteria according to Title 14 of the Code
of Federal Regulations (14 CFR) part 91, section 91.155.
Additional information on airport traffic pattern operations

Tetrahedron

WIND
Wind tee

Wind sock or cone

Figure 14-37. Wind direction indicators.

Traffic pattern
indicators
Landing direction
indicator

2.	 Maintain pattern altitude until abeam approach end of
the landing runway on downwind leg. [Figure 14-39]
3.	 Complete turn to final at least ¼ mile from the runway.
[Figure 14-39]
4. 	 After takeoff or go-around, continue straight ahead
until beyond departure end of runway. [Figure 14-39]
5. 	 If remaining in the traffic pattern, commence turn to
crosswind leg beyond the departure end of the runway
within 300 feet of pattern altitude. [Figure 14-39]

Landing runway
or landing strip
indicators

Wind cone

Figure 14-38. Segmented circle.

can be found in Chapter 4, "Air Traffic Control," of the AIM.
Pilots can find traffic pattern information and restrictions, such
as noise abatement in the Chart Supplement U.S. (formerly
Airport/Facility Directory).
Example: Key to Traffic Pattern Operations—
Single Runway
1.	 Enter pattern in level flight, abeam the midpoint
of the runway, at pattern altitude. (1,000' AGL is
recommended pattern altitude unless otherwise
established.) [Figure 14-39]

6. 	 If departing the traffic pattern, continue straight out,
or exit with a 45° turn (to the left when in a left-hand
traffic pattern; to the right when in a right-hand traffic
pattern) beyond the departure end of the runway, after
reaching pattern altitude. [Figure 14-39]
Example: Key to Traffic Pattern Operations—
Parallel Runways
1.	 Enter pattern in level flight, abeam the midpoint
of the runway, at pattern altitude. (1,000' AGL is
recommended pattern altitude unless otherwise
established.) [Figure 14-40]
2.	 Maintain pattern altitude until abeam approach end of
the landing runway on downwind leg. [Figure 14-40]
3.	 Complete turn to final at least ¼ mile from the runway.
[Figure 14-40]
4.	 Do not overshoot final or continue on a track that
penetrates the final approach of the parallel runway
5. 	 After takeoff or go-around, continue straight ahead
until beyond departure end of runway. [Figure 14-40]

14-21

LEGEND

En
try

Application of traffic
pattern indicators

Recommended standard left-hand traffic
pattern (depicted)
(standard right-hand
traffic pattern would be mirror image)

1

Downwind

2

Base

Crosswind

Segmented circle

3

Final

Departure

4

e

ur

rt

a
ep

D

5
6

6

Departure

RUNWAY

Figure 14-39. Traffic pattern operations—single runway.

6. 	 If remaining in the traffic pattern, commence turn to
crosswind leg beyond the departure end of the runway
within 300 feet of pattern altitude. [Figure 14-40]
7. 	 If departing the traffic pattern, continue straight out,
or exit with a 45° turn (to the left when in a left-hand
traffic pattern; to the right when in a right-hand traffic
pattern) beyond the departure end of the runway, after
reaching pattern altitude. [Figure 14-40]
8.	 Do not continue on a track that penetrates the departure
path of the parallel runway. [Figure 14-40]

Radio Communications
Operating in and out of a towered airport, as well as in a good
portion of the airspace system, requires that an aircraft have twoway radio communication capability. For this reason, a pilot
should be knowledgeable of radio station license requirements
and radio communications equipment and procedures.
Radio License
There is no license requirement for a pilot operating in the
United States; however, a pilot who operates internationally
is required to hold a restricted radiotelephone permit issued
by the Federal Communications Commission (FCC). There
is also no station license requirement for most general
aviation aircraft operating in the United States. A station
license is required, however, for an aircraft that is operating
internationally, that uses other than a VHF radio, and that
meets other criteria.

14-22

Radio Equipment
In general aviation, the most common types of radios are
VHF. A VHF radio operates on frequencies between 118.0
megahertz (MHz) and 136.975 MHz and is classified as
720 or 760 depending on the number of channels it can
accommodate. The 720 and 760 use .025 MHz (25 kilohertz
(KHz) spacing (118.025, 118.050) with the 720 having a
frequency range up to 135.975 MHz and the 760 reaching
up to 136.975 MHz. VHF radios are limited to line of sight
transmissions; therefore, aircraft at higher altitudes are able
to transmit and receive at greater distances.
In March of 1997, the International Civil Aviation Organization
(ICAO) amended its International Standards and Recommended
Practices to incorporate a channel plan specifying 8.33 kHz
channel spacings in the Aeronautical Mobile Service. The
8.33 kHz channel plan was adopted to alleviate the shortage of
VHF ATC channels experienced in western Europe and in the
United Kingdom. Seven western European countries and the
United Kingdom implemented the 8.33 kHz channel plan on
January 1, 1999. Accordingly, aircraft operating in the airspace
of these countries must have the capability of transmitting and
receiving on the 8.33 kHz spaced channels.
Using Proper Radio Procedures
Using proper radio phraseology and procedures contribute to
a pilot's ability to operate safely and efficiently in the airspace
system. A review of the Pilot/Controller Glossary contained
in the AIM assists a pilot in the use and understanding of

LEGEND
Standard left-hand
traffic pattern (depicted)

1

Right-hand traffic
pattern (depicted)

Base

Crosswind

2

Final

Departure

No transgression zone

Final

Departure

2

4

6

No transgression zone

Segmented circle

Base

3

6

4

6

Crosswind

3

5

6
5

Downwind

1

y

tr

En

Figure 14-40. Traffic pattern operation—parallel runways.

standard terminology. The AIM also contains many examples
of radio communications.
ICAO has adopted a phonetic alphabet that should be used in
radio communications. When communicating with ATC, pilots
should use this alphabet to identify their aircraft. [Figure 14-41]
Lost Communication Procedures
It is possible that a pilot might experience a malfunction of
the radio. This might cause the transmitter, receiver, or both
to become inoperative. If a receiver becomes inoperative and a
pilot needs to land at a towered airport, it is advisable to remain
outside or above Class D airspace until the direction and flow
of traffic is determined. A pilot should then advise the tower of
the aircraft type, position, altitude, and intention to land. The
pilot should continue, enter the pattern, report a position as
appropriate, and watch for light signals from the tower. Light
signal colors and their meanings are contained in Figure 14-42.

If the transmitter becomes inoperative, a pilot should follow
the previously stated procedures and also monitor the
appropriate ATC frequency. During daylight hours, ATC
transmissions may be acknowledged by rocking the wings
and at night by blinking the landing light.
When both receiver and transmitter are inoperative, the pilot
should remain outside of Class D airspace until the flow of
traffic has been determined and then enter the pattern and
watch for light signals.
Radio malfunctions should be repaired before further
flight. If this is not possible, ATC may be contacted by
telephone requesting a VFR departure without two-way radio
communications. No radio (NORDO) procedure arrivals
are not accepted at busy airports. If authorization is given
to depart, the pilot is advised to monitor the appropriate
frequency and/or watch for light signals as appropriate.

14-23

Character

Morse Code

Telephony

Phonic Pronunciation

AA
a
BB
b
CC
c
DDd
EE
e
FF
f
GGg
hHh
II i
JJ
j
KKk
LL
L
MMm
NNn
OO
o
PPp
QQq
RR
r
SS
s
TTt
UU
u
VVv
WWw
XXx
YYy
ZZ
z
11
1
22
2
33
3
444
555
N6n
O7
o
P8p
Q9q
R0
r
Figure 14-41. Phonetic alphabet.

If radio communication is lost, it may be a prudent decision
to land at a non-towered airport with lower traffic volume, if
practical. When operating at a non-towered airport, no radio
communication is necessary. However, pilots should be extra
vigilant when not using the radio. Other traffic may not as
14-24

easily be aware of your presence when they are expecting
the standard radio calls.

Air Traffic Control (ATC) Services
Besides the services provided by an FSS as discussed in
Chapter 12, "Aviation Weather Services," numerous other
services are provided by ATC. In many instances a pilot
is required to have contact with ATC, but even when not
required, a pilot may find their services helpful.
Primary Radar
Radar is a device that provides information on range, azimuth,
and/or elevation of objects in the path of the transmitted
pulses. It measures the time interval between transmission and
reception of radio pulses and correlates the angular orientation
of the radiated antenna beam or beams in azimuth and/or
elevation. Range is determined by measuring the time it takes
for the radio wave to go out to the object and then return to the
receiving antenna. The direction of a detected object from a
radar site is determined by the position of the rotating antenna
when the reflected portion of the radio wave is received.
Modern radar is very reliable and there are seldom outages.
This is due to reliable maintenance and improved equipment.
There are, however, some limitations that may affect ATC
services and prevent a controller from issuing advisories
concerning aircraft that are not under his or her control and
cannot be seen on radar.
The characteristics of radio waves are such that they normally
travel in a continuous straight line unless they are "bent" by
atmospheric phenomena, such as temperature inversions,
reflected or attenuated by dense objects such as heavy clouds
and precipitation, or screened by high terrain features. Radar
signals degrade over distance, cannot penetrate through solid
objects such as mountains, and the fastest radar updates every
4.7 seconds. By contrast, the satellite signals used with
Automatic Dependent Surveillance−Broadcast (ADS−B) do
not degrade over distance, provide better visibility around
mountainous terrain and allows equipped aircraft to update
their own position once a second with better accuracy.
ATC Radar Beacon System (ATCRBS)
The ATC radar beacon system (ATCRBS) is often referred to
as "secondary surveillance radar." This system consists of three
components and helps in alleviating some of the limitations
associated with primary radar. The three components are an
interrogator, transponder, and radarscope. The advantages of
ATCRBS are the reinforcement of radar targets, rapid target
identification, and a unique display of selected codes.
Growing air traffic in the National Airspace System (NAS)
will be addressed through the use of ADS-B, which not only

Movement of Vehicles,
Equipment and Personnel

Aircraft on the Ground

Aircraft in Flight

Cleared to cross,
proceed or go

Cleared for takeoff

Cleared to land

Not applicable

Cleared for taxi

Return for landing (to be followed
by steady green at the proper time)

Stop

Stop

Give way to other aircraft and
continue circling

Flashing red

Clear the taxiway/runway

Taxi clear of the runway
in use

Airport unsafe, do not land

Flashing white

Return to starting point
on airport

Return to starting point
on airport

Not applicable

Exercise extreme caution!!!!

Exercise extreme caution!!!!

Exercise extreme caution!!!!

Color and Type of Signal
Steady green
Flashing green
Steady red

Alternating red and green
Figure 14-42. Light gun signals.

provides all the same information the ATCRBS, but will do
so more rapidly and with significantly more accuracy. By
broadcasting aircraft position information to a ground station,
ADS–B can also provide coverage in areas that do not have
radar coverage. In addition, ADS–B provides trajectory
information that includes speed and direction of motion.
Transponder
The transponder is the airborne portion of the secondary
surveillance radar system and a system with which a pilot
should be familiar. The ATCRBS cannot display the secondary

information unless an aircraft is equipped with a transponder.
A transponder is also required to operate in certain controlled
airspace as discussed in Chapter 15, "Airspace."
A transponder code consists of four numbers from 0 to 7
(4,096 possible codes). There are some standard codes or ATC
may issue a four-digit code to an aircraft. When a controller
requests a code or function on the transponder, the word
"squawk" may be used. Figure 14-43 lists some standard
transponder phraseology. Additional information concerning
transponder operation can be found in the AIM, Chapter 4.

Radar Beacon Phraseology
SQUAWK (number)

Operate radar beacon transponder on designated code in MODE A/3.

IDENT

Engage the "IDENT" feature (military I/P) of the transponder.

SQUAWK (number) and IDENT

Operate transponder on specified code in MODE A/3 and engage the "IDENT"
(military I/P) feature.

SQUAWK Standby

Switch transponder to standby position.

SQUAWK Low/Normal

Operate transponder on low or normal sensitivity as specified. Transponder is
operated in "NORMAL" position unless ATC specifies "LOW" ("ON" is used instead of
"NORMAL" as a master control label on some types of transponders).

SQUAWK Altitude

Activate MODE C with automatic altitude reporting.

STOP Altitude SQUAWK

Turn off altitude reporting switch and continue transmitting MODE C framing pulses.
If your equipment does not have this capability, turn off MODE C.

STOP SQUAWK (mode in use)

Switch off specified mode. (Used for military aircraft when the controller is unaware of
military service requirements for the aircraft to continue operation on another MODE.)

STOP SQUAWK

Switch off transponder.

SQUAWK Mayday

Operate transponder in the emergency position (MODE A Code 7700 for civil
transponder, MODE 3 Code 7700 and emergency feature for military transponder).

SQUAWK VFR

Operate radar beacon transponder on Code 1200 in MODE A/3, or other
appropriate VFR code.

Figure 14-43. Transponder phraseology.

14-25

Automatic Dependent Surveillance–Broadcast
(ADS-B)
Automatic Dependent Surveillance−Broadcast (ADS−B) is a
surveillance technology being deployed throughout the NAS
to facilitate improvements needed to increase the capacity
and efficiency of the NAS, while maintaining safety. ADS-B
supports these improvements by providing a higher update rate
and enhanced accuracy of surveillance information over the
current radar-based surveillance system. In addition, ADS-B
enables the expansion of air traffic control (ATC) surveillance
services into areas where none existed previously. The ADS-B
ground system also provides Traffic Information ServicesBroadcast (TIS-B) and Flight Information Services-Broadcast
(FIS-B) for use on appropriately equipped aircraft, enhancing
the user's situational awareness (SA) and improving the
overall safety of the NAS.
The ADS−B system is composed of aircraft avionics and a
ground infrastructure. Onboard avionics determine the position
of the aircraft by using the GPS and transmit its position,
along with additional information about the aircraft, to ground
stations for use by ATC and nearby ADS-B equipped aircraft.
In the United States, ADS−B equipped aircraft exchange
information on one of two frequencies: 978 or 1090 MHz.
The 1090 MHz frequency is associated with Mode A, C, and S
transponder operations. 1090 MHz transponders with integrated
ADS−B functionality extend the transponder message sets with
additional ADS−B information. This additional information
is known as an "extended squitter" message and referred to as
1090ES. ADS−B equipment operating on 978 MHz is known
as the Universal Access Transceiver (UAT).
Radar Traffic Advisories
Radar equipped ATC facilities provide radar assistance
to aircraft on instrument flight plans and VFR aircraft
provided the aircraft can communicate with the facility and
are within radar coverage. This basic service includes safety
alerts, traffic advisories, limited vectoring when requested,
and sequencing at locations where this procedure has been
established. ATC issues traffic advisories based on observed
radar targets. The traffic is referenced by azimuth from the
aircraft in terms of the 12-hour clock. Also, distance in
nautical miles, direction in which the target is moving, and
type and altitude of the aircraft, if known, are given.
An example would be: "Traffic 10 o'clock 5 miles east
bound, Cessna 152, 3,000 feet." The pilot should note that
traffic position is based on the aircraft track and that wind
correction can affect the clock position at which a pilot locates
traffic. This service is not intended to relieve the pilot of the
responsibility to see and avoid other aircraft. [Figure 14-44]
In addition to basic radar service, terminal radar service
14-26

A

B
Wind

TRACK

TRACK

Traffic information would be issued to the pilot of aircraft "A"
as 12 o'clock. The actual position of the traffic as seen by
the pilot of aircraft "A" would be 1 o'clock. Traffic information
issued to aircraft "B" would also be given as 12 o'clock, but
in this case, the pilot of "B" would see traffic at 10 o'clock.
Figure 14-44. Traffic advisories.

area (TRSA) has been implemented at certain terminal
locations. TRSAs are depicted on sectional aeronautical
charts and listed in the Chart Supplement U.S. (formerly
Airport/Facility Directory). The purpose of this service is to
provide separation between all participating VFR aircraft and
all IFR aircraft operating within the TRSA. Class C service
provides approved separation between IFR and VFR aircraft
and sequencing of VFR aircraft to the primary airport. Class
B service provides approved separation of aircraft based on
IFR, VFR, and/or weight and sequencing of VFR arrivals to
the primary airport(s).

Wake Turbulence
All aircraft generate wake turbulence during flight. This
disturbance is caused by a pair of counter-rotating vortices
trailing from the wingtips. The vortices from larger aircraft
pose problems to encountering aircraft. The wake of these
aircraft can impose rolling moments exceeding the rollcontrol authority of the encountering aircraft. Also, the
turbulence generated within the vortices can damage aircraft
components and equipment if encountered at close range. For
this reason, a pilot must envision the location of the vortex
wake and adjust the flight path accordingly.
Vortex Generation
Lift is generated by the creation of a pressure differential over
the wing surface. The lowest pressure occurs over the upper
wing surface and the highest pressure under the wing. This
pressure differential triggers the rollup of the airflow aft of
the wing resulting in swirling air masses trailing downstream
of the wingtips. After the rollup is completed, the wake
consists of two counter rotating cylindrical vortices. Most of
the energy lies within a few feet of the center of each vortex.
[Figure 14-45]

Figure 14-45. Vortex generation.

Vortex Strength

Terminal Area

Wake turbulence has historically been thought of as only
a function of aircraft weight, but recent research considers
additional parameters, such as speed, aspects of the wing, wake
decay rates, and aircraft resistance to wake, just to name a few.
The vortex characteristics of any aircraft will be changed with
the extension of flaps or other wing configuration devices, as
well as changing speed. However, as the basic factors are weight
and speed, the vortex strength increases proportionately with
an increase in aircraft operating weight or decrease in aircraft
speed. The greatest vortex strength occurs when the generating
aircraft is heavy, slow, and clean, since the turbulence from a
"dirty" aircraft configuration hastens wake decay.

En Route
En route wake turbulence events have been influenced by
changes to the aircraft fleet mix that have more "Super"
(A380) and "Heavy" (B-747, B-777, A340, etc.) aircraft

operating in the NAS. There have been wake turbulence
events in excess of 30NM and 2000 feet lower than the wake
generating aircraft. Air density is also a factor in wake strength.
Even though the speeds are higher in cruise at high altitude,
the reduced air density may result in wake strength comparable
to that in the terminal area. In addition, for a given separation
distance, the higher speeds in cruise result in less time for the
wake to decay before being encountered by a trailing aircraft.
Vortex Behavior
Trailing vortices have certain behavioral characteristics
that can help a pilot visualize the wake location and take
avoidance precautions.
Vortices are generated from the moment an aircraft leaves the
ground (until it touches down), since trailing vortices are the
byproduct of wing lift. [Figure 14-46] The vortex circulation
is outward, upward, and around the wingtips when viewed
from either ahead or behind the aircraft. Tests with large

25

Wake ends

Wake begins
Rotation

Touchdown

Figure 14-46. Vortex behavior.

14-27

aircraft have shown that vortices remain spaced a bit less than
a wingspan apart, drifting with the wind, at altitudes greater
than a wingspan from the ground. Tests have also shown that
the vortices sink at a rate of several hundred feet per minute,
slowing their descent and diminishing in strength with time
and distance behind the generating aircraft.
When the vortices of larger aircraft sink close to the ground
(within 100 to 200 feet), they tend to move laterally over
the ground at a speed of 2–3 knots. A crosswind decreases
the lateral movement of the upwind vortex and increases
the movement of the downwind vortex. A light quartering
tailwind presents the worst case scenario as the wake
vortices could be all present along a significant portion of
the final approach and extended centerline and not just in the
touchdown zone as typically expected.
Vortex Avoidance Procedures
The following procedures are in place to assist pilots in vortex
avoidance in the given scenario.


Landing behind a larger aircraft on the same runway—
stay at or above the larger aircraft's approach
flight path and land beyond its touchdown point.
[Figure 14-47A]



Landing behind a larger aircraft on a parallel runway
closer than 2,500 feet—consider the possibility of drift
and stay at or above the larger aircraft's final approach
flight path and note its touchdown point. [Figure 14-47B]



Landing behind a larger aircraft on crossing runway—
cross above the larger aircraft's flight path.



Landing behind a departing aircraft on the same
runway—land prior to the departing aircraft's
rotating point.



Landing behind a larger aircraft on a crossing
runway—note the aircraft's rotation point and, if that
point is past the intersection, continue and land prior
to the intersection. If the larger aircraft rotates prior
to the intersection, avoid flight below its flight path.
Abandon the approach unless a landing is ensured well
before reaching the intersection. [Figure 14-47C]

landing (since vortices settle and move laterally
near the ground, the vortex hazard may exist along
the runway and in the flight path, particularly in a
quartering tailwind), it is prudent to wait at least 2
minutes prior to a takeoff or landing.


En route, it is advisable to avoid a path below and
behind a large aircraft, and if a large aircraft is
observed above on the same track, change the aircraft
position laterally and preferably upwind.

Collision Avoidance
Title 14 of the CFR part 91 has established right-of-way
rules, minimum safe altitudes, and VFR cruising altitudes
to enhance flight safety. The pilot can contribute to collision
avoidance by being alert and scanning for other aircraft. This
is particularly important in the vicinity of an airport.
Effective scanning is accomplished with a series of short,
regularly spaced eye movements that bring successive areas of
the sky into the central visual field. Each movement should not
exceed 10°, and each should be observed for at least 1 second
to enable detection. Although back and forth eye movements
seem preferred by most pilots, each pilot should develop a
scanning pattern that is most comfortable and then adhere to
it to assure optimum scanning. Even if entitled to the right-of­
way, a pilot should yield if another aircraft seems too close.
Clearing Procedures
The following procedures and considerations are in place to
assist pilots in collision avoidance under various situations:


Before takeoff—prior to taxiing onto a runway or
landing area in preparation for takeoff, pilots should
scan the approach area for possible landing traffic,
executing appropriate maneuvers to provide a clear
view of the approach areas.



Climbs and descents—during climbs and descents in
flight conditions that permit visual detection of other
traffic, pilots should execute gentle banks left and right
at a frequency that permits continuous visual scanning
of the airspace.



Departing behind a large aircraft—rotate prior to the
large aircraft's rotation point and climb above its climb
path until turning clear of the wake.



Straight and level—during sustained periods of
straight-and-level flight, a pilot should execute
appropriate clearing procedures at periodic intervals.



For intersection takeoffs on the same runway—
be alert to adjacent larger aircraft operations,
particularly upwind of the runway of intended use.
If an intersection takeoff clearance is received, avoid
headings that cross below the larger aircraft's path.



Traffic patterns—entries into traffic patterns while
descending should be avoided.



Traffic at VOR sites—due to converging traffic,
sustained vigilance should be maintained in the
vicinity of VORs and intersections.

If departing or landing after a large aircraft executing
a low approach, missed approach, or touch-and-go



Training operations—vigilance should be maintained
and clearing turns should be made prior to a practice



14-28

W

IN
D

A

Touchdown point of larger aircraft
Side view

B

Aircraft altitude is above wake
Less than 2500 feet

Touchdown point

W

IN

D

Parallel Runway Situation

C

Aircraft altitude is above wake

Aircraft crossing over
wake turbulence

Figure 14-47. Vortex avoidance procedures.

14-29

maneuver. During instruction, the pilot should be
asked to verbalize the clearing procedures (call out
"clear left, right, above, and below").
High-wing and low-wing aircraft have their respective blind
spots. The pilot of a high-wing aircraft should momentarily
raise the wing in the direction of the intended turn and look
for traffic prior to commencing the turn. The pilot of a lowwing aircraft should momentarily lower the wing and look
for traffic prior to commencing the turn.
Pilot Deviations (PDs)
A pilot deviation (PD) is an action of a pilot that violates any
Federal Aviation Regulation. While PDs should be avoided,
the regulations do authorize deviations from a clearance in
response to a traffic alert and collision avoidance system
resolution advisory. You must notify ATC as soon as possible
following a deviation.
Pilot deviations can occur in several different ways.
Airborne deviations result when a pilot strays from
an assigned heading or altitude or from an instrument
procedure, or if the pilot penetrates controlled or restricted
airspace without ATC clearance.

with aircraft operations by entering or moving on the runway
movement area without authorization from air traffic control.
In serious instances, any ground deviation (PD or VPD) can
result in a runway incursion. Best practices in preventing
ground deviations can be found in the following section
under runway incursion avoidance.
Runway Incursion Avoidance
A runway incursion is "any occurrence in the airport runway
environment involving an aircraft, vehicle, person, or object
on the ground that creates a collision hazard or results in a loss
of required separation with an aircraft taking off, intending
to take off, landing, or intending to land." It is important
to give the same attention to operating on the surface as in
other phases of flights. Proper planning can prevent runway
incursions and the possibility of a ground collision. A pilot
should always be aware of the aircraft's position on the
surface at all times and be aware of other aircraft and vehicle
operations on the airport. At times, towered airports can be
busy and taxi instructions complex. In this situation, it may
be advisable to write down taxi instructions. The following
are some practices to help prevent a runway incursion:


Read back all runway crossing and/or hold instructions.



Review airport layouts as part of preflight planning,
before descending to land and while taxiing, as
needed.



Know airport signage.



Review NOTAM for information on runway/taxiway
closures and construction areas.

To prevent airborne deviations, follow these steps:






Plan each flight—you may have flown the flight many
times before but conditions and situations can change
rapidly, such as in the case of a pop-up temporary
flight restriction (TFR). Take a few minutes prior to
each flight to plan accordingly.



Talk and squawk—Proper communication with ATC
has its benefits. Flight following often makes the
controller's job easier because they can better integrate
VFR and IFR traffic.

Request progressive taxi instructions from ATC when
unsure of the taxi route.



Check for traffic before crossing any runway hold line
and before entering a taxiway.

Give yourself some room—GPS is usually more
precise than ATC radar. Using your GPS to fly up
to and along the line of the airspace you are trying to
avoid could result in a pilot deviation because ATC
radar may show you within the restricted airspace.



Turn on aircraft lights and the rotating beacon or strobe
lights while taxing.



When landing, clear the active runway as soon as
possible, then wait for taxi instructions before further
movement.



Study and use proper phraseology in order to
understand and respond to ground control instructions.



Write down complex taxi instructions at unfamiliar
airports.

Ground deviations (also called surface deviations) include
taxiing, taking off, or landing without clearance, deviating
from an assigned taxi route, or failing to hold short of an
assigned clearance limit. To prevent ground deviations, stay
alert during ground operations. Pilot deviations can and
frequently do occur on the ground. Many strategies and tactics
pilots use to avoid airborne deviations also work on the ground.
Pilots should also remain vigilant about vehicle/pedestrian
deviations (V/PDs). A vehicle or pedestrian deviation
includes pedestrians, vehicles or other objects interfering
14-30

Approximately three runway incursions occur each day at
towered airports within the United States. The potential
that these numbers present for a catastrophic accident is
unacceptable. The following are examples of pilot deviations,
operational incidents (OI), and vehicle (driver) deviations
that may lead to runway incursions.

Pilot Deviations:


Crossing a runway hold marking without clearance
from ATC



Taking off without clearance



Landing without clearance

Operational Incidents (OI):


Clearing an aircraft onto a runway while another
aircraft is landing on the same runway



Issuing a takeoff clearance while the runway is
occupied by another aircraft or vehicle

Vehicle (Driver) Deviations:


Crossing a runway hold marking without ATC
clearance

According to FAA data, approximately 65 percent of all
runway incursions are caused by pilots. Of the pilot runway
incursions, FAA data shows almost half of those incursions
are caused by GA pilots.
Causal Factors of Runway Incursions
Detailed investigations of runway incursions over the past
10 years have identified three major areas contributing to
these events:


Failure to comply with ATC instructions



Lack of airport familiarity



Nonconformance with standard operating procedures

Clear, concise, and effective pilot/controller communication is
paramount to safe airport surface operations. You must fully
understand and comply with all ATC instructions. It is mandatory
to read back all runway "hold short" instructions verbatim.
Taxiing on an unfamiliar airport can be very challenging,
especially during hours of darkness or low visibility. A
request may be made for progressive taxi instructions which
include step by step taxi routing instructions. Ensure you
have a current airport diagram, remain "heads-up" with eyes
outside, and devote your entire attention to surface navigation
per ATC clearance. All checklists should be completed while
the aircraft is stopped. There is no place for non-essential
chatter or other activities while maintaining vigilance during
taxi. [Figure 14-48]
Runway Confusion
Runway confusion is a subset of runway incursions and
often results in you unintentionally taking off or landing on
a taxiway or wrong runway. Generally, you are unaware of
the mistake until after it has occurred.

Figure 14-48. Heads-up, eyes outside.

In August 2006, the flight crew of a commercial regional jet
was cleared for takeoff on Runway 22 but mistakenly lined
up and departed on Runway 26, a much shorter runway. As
a result, the aircraft crashed off the end of the runway.

Causal Factors of Runway Confusion
There are three major factors that increase the risk of runway
confusion and can lead to a wrong runway departure:


Airport complexity



Close proximity of runway thresholds



Joint use of a runway as a taxiway

Not only can airport complexity contribute to a runway
incursion; it can also play a significant role in runway
confusion. If you are operating at an unfamiliar airport and
need assistance in executing the taxi clearance, do not hesitate
to ask ATC for help. Always carry a current airport diagram
and trace or highlight your taxi route to the departure runway
prior to leaving the ramp.
If you are operating from an airport with runway thresholds
in close proximity to one another, exercise extreme caution
when taxiing onto the runway. Figure 14-49 shows a perfect
example of a taxiway leading to multiple runways that may
cause confusion. If departing on Runway 36, ensure that you
set your aircraft heading "bug" to 360°, and align your aircraft
to the runway heading to avoid departing from the wrong
runway. Before adding power, make one last instrument scan
to ensure the aircraft heading and runway heading are aligned.
Under certain circumstances, it may be necessary to
use a runway as a taxiway. For example, during airport
construction some taxiways may be closed requiring re­
routing of traffic onto runways. In other cases, departing
traffic may be required to back taxi on the runway in order
to utilize the full runway length.

14-31

36

Another way to mitigate the risk of runway incursions is to
write down all taxi instructions as soon as they are received
from ATC. [Figure 14-50] It is also helpful to monitor ATC
clearances and instructions that are issued to other aircraft.
You should be especially vigilant if another aircraft has a
similar sounding call sign so there is no mistake about who
ATC is contacting or to whom they are giving instructions
and clearances.
Read back your complete ATC clearance with your aircraft
call sign. This gives ATC the opportunity to clarify any
misunderstandings and ensure that instructions were given to
the correct aircraft. If, at any time, there is uncertainty about
any ATC instructions or clearances, ask ATC to "say again"
or ask for progressive taxi instructions.

ATC Instructions—"Hold Short"
The most important sign and marking on the airport is the
hold sign and hold marking. These are located on a stub
taxiway leading directly to a runway. They depict the holding
position or the location where the aircraft is to stop so as not to
enter the runway environment. [Figure 14-51] For example,
Figure 14-52 shows the holding position sign and marking
for Runway 13 and Runway 31.

Figure 14-49. Confusing runway/runway intersection.

Since inattention and confusion often are factors contributing to
runway incursion, it is important to remain extremely cautious
and maintain situational awareness (SA). When instructed to
use a runway as a taxiway, do not become confused and take
off on the runway you are using as a taxiway.

When ATC issues a "hold short" clearance, you are
expected to taxi up to, but not cross any part of the runway
holding marking. At a towered airport, runway hold
markings should never be crossed without explicit ATC
instructions. Do not enter a runway at a towered airport
unless instructions are given from ATC to cross, takeoff
from, or "line up and wait" on that specific runway.
ATC is required to obtain a read-back from the pilot of
all runway "hold short" instructions. Therefore, you
must read back the entire clearance and "hold short"
instruction, to include runway identifier and your call sign.

ATC Instructions
Title 14 of the Code of Federal Regulations (14 CFR) part
91, section 91.123 requires you to follow all ATC clearances
and instructions. Request clarification if you are unsure of the
clearance or instruction to be followed. If you are unfamiliar
with the airport or unsure of a taxi route, ask ATC for a
"progressive taxi." Progressive taxi requires the controller
to provide step-by-step taxi instructions.
The final decision to act on ATC's instruction rests with you.
If you cannot safely comply with any of ATC's instructions,
inform them immediately by using the word "UNABLE."
There is nothing wrong with telling a controller that you are
unable to safely comply with the clearance.
14-32

Figure 14-50. A sound practice is to write down taxi instructions

from ATC.

Controller
November 477ZA,
Runway four, taxi via
Echo, hold short of
Runway two five at
Taxiway Delta.
Pilot
November 477ZA,
Runway four via Echo,
hold short of Runway
two five at Delta.

Figure 14-53. Example of taxi and "hold short" instructions from
ATC to a pilot.
Figure 14-51. Do NOT cross a runway holding position marking

without ATC clearance. If the tower is closed or you are operating
from a non-towered airport, check both directions for conflicting
traffic before crossing the hold position marking.

Figure 14-53 shows an example of a controller's taxi and "hold
short" instructions and the reply from the pilot.

ATC Instructions—Explicit Runway Crossing
As of June 30, 2010, ATC is required to issue explicit
instructions to "cross" or "hold short" of each runway.
Instructions to "cross" a runway are normally issued one at a
time, and an aircraft must have crossed the previous runway
before another runway crossing is issued. Exceptions may
apply for closely spaced runways that have less than 1,000
feet between centerlines. This applies to all runways to include
active, inactive, or closed. Figure 14-54 shows communication
between ATC and a pilot who is requesting a taxi clearance.
Extra caution should be used when directed by ATC to
taxi onto or across a runway, especially at night and during
reduced visibility conditions. Always comply with "hold

short" or crossing instructions when approaching an entrance
to a runway. Scan the full length of the runway and the final
approaches before entering or crossing any runway, even if
ATC has issued a clearance.

ATC Instructions—"Line Up and Wait" (LUAW)
ATC now uses the "line up and wait" (LUAW) instruction
when a takeoff clearance cannot be issued immediately due
to traffic or other reasons. The words "line up and wait" have
replaced "position and hold" in directing you to taxi onto a
runway and await takeoff clearance.
An ATC instruction to "line up and wait" is not a clearance
for takeoff. It is only a clearance to enter the runway and
hold in position for takeoff. Under LUAW phraseology, the
controller states the aircraft call sign, departure runway, and
"line up and wait." Be aware that "traffic holding in position"
will continue to be used to advise other aircraft that traffic
has been authorized to line up and wait on an active runway.
Pay close attention when instructed to "line up and wait,"
especially at night or during periods of low visibility. Before
Pilot
"Ground, November 1234
ready to taxi from the GA
ramp with Bravo."

ATC

"November 1234,
Runway two seven, taxi
via Alpha, hold short
of Runway three one."
Pilot
"November 1234,
Runway two seven, taxi
via Alpha, hold short
Runway three one."
ATC
When able, tower will
issue crossing clearance:
"November 1234, cross
Runway three one."

Figure 14-52. Runway 13-31 holding position sign and marking
located on Taxiway Charlie.

Figure 14-54. Communication between ATC and a pilot who is
requesting taxi procedures.

14-33

entering the runway, remember to scan the full length of the
runway and its approach end for other aircraft.
There have been collisions and incidents involving aircraft
instructed to "line up and wait" while ATC waits for the
necessary conditions to issue a takeoff clearance. An OI
caused a 737 to land on a runway occupied by a twin-engine
turboprop. The turboprop was holding in position awaiting
takeoff clearance. Upon landing, the 737 collided with the
twin-engine turboprop.
When ATC instructs you to "line up and wait," they should
advise you of any anticipated delay in receiving your takeoff
clearance. Possible reasons for ATC takeoff clearance delays
may include other aircraft landing and/or departing, wake
turbulence, or traffic crossing an intersecting runway.


If advised of a reason for the delay, or the reason is
clearly visible, expect an imminent takeoff clearance
once the reason is no longer an issue.



If a takeoff clearance is not received within 90 seconds
after receiving the "line up and wait" instruction,
contact ATC immediately.



When ATC issues "line up and wait" instructions
and takeoff clearances from taxiway intersection, the
taxiway designator is included.
Example – "N123AG Runway One-Eight, at Charlie
Three, line up and wait."
Example – "N123AG Runway One-Eight, at Charlie
Three, cleared for takeoff."

If LUAW procedures are being used and landing traffic is a
factor, ATC is required to:


Inform the aircraft in the LUAW position of the closest
aircraft that is requesting a full-stop, touch-and-go,
stop-and-go, option, or unrestricted low approach.
Example – "N123AG, Runway One-Eight, line up
and wait, traffic a Cessna 210 on a six-mile final."



In some cases, where safety logic is being used, ATC
is permitted to issue landing clearances with traffic in
the LUAW position. Traffic information is issued to
the landing traffic.
Example – "N456HK, Runway One-Eight, cleared to
land, traffic a DeHavilland Otter holding in position."
NOTE: ATC will/must issue a takeoff clearance to the
traffic holding in position in sufficient time to ensure
no conflict exists with landing aircraft. Prescribed
runway separation must exist no later than when the
landing aircraft crosses the threshold.

14-34



In cases where ATC is not permitted to issue landing
clearances with traffic in the LUAW position, traffic
information is issued to the closest aircraft that is
requesting a full-stop, touch-and-go, stop-and-go,
option, or unrestricted low approach.
Example – "N456HK, Runway One-Eight, continue,
traffic holding in position."

ATC Instructions—"Runway Shortened"
You should review NOTAMs in your preflight planning to
determine any airport changes that will affect your departure
or arrival. When the available runway length has been
temporarily or permanently shortened due to construction,
the ATIS includes the words "warning" and "shortened" in
the text of the message. For the duration of the construction
when the runway is temporarily shortened, ATC will
include the word "shortened" in their clearance instructions.
Furthermore, the use of the term "full length" will not be used
by ATC during this period of the construction.
Some examples of ATC instructions are:


"Runway three six shortened, line up and wait."



"Runway three six shortened, cleared for takeoff."



"Runway three six shortened, cleared to land."

When an intersection departure is requested on a temporarily
or permanently shortened runway during the construction,
the remaining length of runway is included in the clearance.
For example, "Runway three six at Echo, intersection
departure, 5,600 feet available." If following the construction,
the runway is permanently shortened, ATC will include
the word "shortened" until the Chart Supplement U.S.
(formerly Airport/Facility Directory) is updated to include
the permanent changes to the runway length.
Pre-Landing, Landing, and After-Landing
While en route and after receiving the destination airport
ATIS/landing information, review the airport diagram and
brief yourself as to your exit taxiway. Determine the following:


Are there any runway hold markings in close proximity
to the exit taxiway?



Do not cross any hold markings or exit onto any
runways without ATC clearance.

After landing, use the utmost caution where the exit taxiways
intersect another runway, and do not exit onto another runway
without ATC authorization. Do not accept last minute
turnoff instructions from the control tower unless you clearly
understand the instructions and are at a speed that ensures you

can safely comply. Finally, after landing and upon exiting
the runway, ensure your aircraft has completely crossed over
the runway hold markings. Once all parts of the aircraft have
crossed the runway holding position markings, you must hold
unless further instructions have been issued by ATC. Do not
initiate non-essential communications or actions until the
aircraft has stopped and the brakes set.

Engineered Materials Arresting Systems
(EMAS)
Aircraft can and do overrun the ends of runways and
sometimes with devastating results. An overrun occurs
when an aircraft passes beyond the end of a runway during
an aborted takeoff or on landing rollout. To minimize the
hazards of overruns, the FAA incorporated the concept of
a runway safety area (RSA) beyond the runway end into
airport design standards. At most commercial airports, the
RSA is 500 feet wide and extends 1,000 feet beyond each
end of the runway. The FAA implemented this requirement
in the event that an aircraft overruns, undershoots, or veers
off the side of the runway.
The most dangerous of these incidents are overruns, but
since many airports were built before the 1,000-foot RSA
length was adopted some 20 years ago, the area beyond the
end of the runway is where many airports cannot achieve the
full standard RSA. This is due to obstacles, such as bodies
of water, highways, railroads, populated areas, or severe
drop-off of terrain. Under these specific circumstances, the
installation of an Engineered Materials Arresting System
(EMAS) is an acceptable alternative to a RSA beyond the
runway end. It provides a level of safety that is generally
equivalent to a full RSA. [Figure 14-55]
An EMAS uses materials of closely controlled strength and
density placed at the end of a runway to stop or greatly slow
an aircraft that overruns the runway. The best material found
to date is a lightweight, crushable concrete. When an aircraft
rolls into an EMAS arrestor bed, the tires of the aircraft sink
into the lightweight concrete and the aircraft is decelerated
by having to roll through the material. [Figure 14-56]
Incidents
To date, there have been several incidents listed below where
the EMAS technology has worked successfully to arrest
aircraft that overrun the runway. All cases have resulted in
minimal to do damage to the aircraft. The only known injury
was an ankle injury to a passenger during egress following
the arrestment. [Figure 14-57]


Figure 14-55. Engineered material arresting system (EMAS)

located at Yeager Airport, Charleston, West Virginia.

 	 May 2003—A Cargo McDonnell Douglas (MD)-11
overran the runway at JFK.
 	 January 2005—A Boeing 747 overran the runway at
JFK.


July 2006—A Mystere Falcon 900 overran the
runway at Greenville Downtown Airport (KGMU)
in Greenville, South Carolina.

 	 July 2008—An Airbus A320 overran the runway at
O'Hare International Airport (ORD).


January 2010—A Bombardier CRJ-200 regional jet
overran the runway at Yeager Airport (KCRW) in
Charleston, West Virginia (WV). [Figure 14-58]



October 2010—A G-4 Gulfstream overran the
runway at Teterboro Airport (KTEB) in Teterboro,
New Jersey (NJ).

 	 November 2011—A Cessna Citation 550 overran the
runway at Key West International Airport (KEYW)
in Key West, Florida.
EMAS Installations and Information
Currently, EMAS is installed at 63 runway ends at 42 airports
in the United States with plans to install more throughout the
next few years.
EMAS information is available in the Chart Supplement
U.S. (formerly Airport/Facility Directory) under the specific
airport information. Figure 14-59 shows airport information
for Boston Logan International Airport. At the bottom of the
page, it shows which runways are equipped with arresting
systems and the type that they have. It is important for pilots
to study airport information, become familiar with the details
and limitations of the arresting system, and the runways that
are equipped with them. [Figure 14-60]

May 1999—A Saab 340 commuter aircraft overran
the runway at John F. Kennedy International Airport
(JFK).

14-35

Typical Plan View
Runway safety area length
Set back

ARRESTOR BED

Runway width

Runway

Side steps
An EMASMAX bed is typically the full width of the runway and the arrestor bed is set-back from the end of the runway.

Typical Profile View
Debris deflector
over concrete beam

Arrestor bed

Lead in ramp

Side steps

Base surface
The front of an EMASMAX bed includes a lead-in ramp to transition the aircraft into the material.

Typical Section View
Arrestor bed
Stepped sides provide arff
access and passenger egress

Base surface
 Beyond the runway width, the sides of an EMASMAX bed are stepped to provide emergency vehicle access and passenger
egress.
 The length of the EMAS bed is dependent upon the space available in the existing RSA and the design aircraft for the EMAS.
 As stated in FAA Advisory Circular 150/5220-22A, Engineered Materials Arresting Systems (EMAS) for Aircraft Overruns, the
EMAS is designed to arrest aircraft exiting the runway at speeds between 40 and 70 knots. 70 knots is the preferred EMAS
design runway exit speed but in limited spaces, the EMAS may have design runway exit speeds as low as 40 knots.
Figure 14-56. Diagram of an EMASMAX system.

Pilot Considerations
Although engaging an EMAS should not be a desired
outcome for the end of a flight, pilots need to know what
EMAS is, how to identify it on the airfield diagram and
on the airfield, as well as knowing what to do should they
find themselves approaching an installation in an overrun
situation. [Figure 14-59 and Figure 14-60] Pilots also need
to know that an EMAS may not stop lightweight general
aviation aircraft that are not heavy enough to sink into the
crushable concrete. The time to discuss whether or not a
runway has an EMAS at the end is during the pre-departure
briefing prior to takeoff or during the approach briefing prior
14-36

to commencing the approach. Following the guidance below
ensures that the aircraft engages the EMAS according to the
design entry parameters.
During the takeoff or landing phase, if a pilot determines that
the aircraft will exit the runway end and enter the EMAS, the
following guidance should be adhered to:
1.	 Continue deceleration - Regardless of aircraft speed
upon exiting the runway, continue to follow Rejected/
Aborted Takeoff procedures, or if landing, Maximum
Braking procedures outlined in the Flight Manual.

Figure 14-57. There have been several incidents where the EMAS
has successfully arrested the aircraft.

Figure 14-58. A Bombardier CRJ-200 regional jet overran the

2.	 Maintain runway centerline - Not veering left or right
of the bed and continuing straight ahead will maximize
stopping capability of the EMAS bed. The quality of
deceleration will be best within the confines of the bed.

The chapter identifies best practices to help you avoid errors
that may potentially lead to runway incursions. Although the
chapter pertains mostly to surface movements for single-pilot
operations, all of the information is relevant for flight crew
operations as well.

3.	 Maintain deceleration efforts - The arrestor bed is a
passive system, so this is the only action required by
the pilot.
4.	 Once stopped, do not attempt to taxi or otherwise move
the aircraft.

Chapter Summary
This chapter focused on airport operations both in the air and
on the surface. For specific information about an unfamiliar
airport, consult the Chart Supplement U.S. (formerly
Airport/Facility Directory) and NOTAMS before flying. For
further information regarding procedures discussed in this
chapter, refer to 14 CFR part 91 and the AIM. By adhering
to established procedures, both airport operations and safety
are enhanced.
This chapter is also designed to help you attain an
understanding of the risks associated with surface navigation
and is intended to provide you with basic information
regarding the safe operation of aircraft at towered and
nontowered airports. This chapter focuses on the following
major areas:


Runway incursion overview



Taxi route planning



Taxi procedures



Communications



Airport signs, markings and lighting

runway at Yeager Airport (KCRW) in Charleston, West Virginia.

Additional information about surface operations is available
through the following sources:


Federal Aviation Administration (FAA) Runway
Safety website—\url{faa.gov/go/runwaysafety}



FAA National Aeronautical Navigation Services
(AeroNav), formerly known as the National
Aeronautical Charting Office (NACO)—\url{faa.
gov/air_traffic/flight_info/aeronav}



Chart Supplement U.S. (formerly Airport/Facility
Directory)—\url{faa.gov/air_traffic/flight_info/
aeronav/digital_products/dafd/search/}



Automatic Terminal Information Service (ATIS)



Notice to Airmen (NOTAMs)—\url{http://www.faa.gov/
pilots/flt_plan/notams}



Advisory Circular (AC) 91-73, part 91 and part 135,
Single-Pilot and Flight School Procedures During Taxi
Operations



Aeronautical Information Manual (AIM)—\url{faa.
gov/air_traffic/publications/atpubs/aim/}



AC 120-74, parts 91, 121, 125, and 135, Flight Crew
Procedures During Taxi Operations

14-37

rn
av
ig
at
io
n
fo
ed
us
be
to
No
t
Figure 14-59. EMAS information for Boston Logan International Airport located in the Chart Supplement U.S. (formerly Airport/
Facility Directory).

14-38

ga
tio
n
av
i
rn
NE-1, 28 JUL 2011 to 25 AUG 2011

fo
ed
us

N

ot

to

be

NE-1, 28 JUL 2011 to 25 AUG 2011

Figure 14-60. An airport diagram with EMAS information.

14-39

14-40


Chapter 15

Airspace
Introduction
The two categories of airspace are: regulatory and
nonregulatory. Within these two categories, there are four
types: controlled, uncontrolled, special use, and other
airspace. The categories and types of airspace are dictated
by the complexity or density of aircraft movements, nature
of the operations conducted within the airspace, the level of
safety required, and national and public interest. Figure 15-1
presents a profile view of the dimensions of various classes
of airspace. Also, there are excerpts from sectional charts
that are discussed in Chapter 16, Navigation, that are used
to illustrate how airspace is depicted.

15-1

FL 600

Class

A

18,000' MSL

Class

B

Class

14,500' MSL

Class

G

Nontowered
airport with
instrument
approach

1,200'
AGL
700'
AGL
Class

G

1,200'
AGL
700'
AGL
Class

G

Class

C

E

1,200'
AGL
700'
AGL
Class

Class

D

Nontowered
airport with
no instrument
approach

G

Figure 15-1. Airspace profile.

Controlled Airspace
Controlled airspace is a generic term that covers the
different classifications of airspace and defined dimensions
within which air traffic control (ATC) service is provided
in accordance with the airspace classification. Controlled
airspace consists of:


Class A



Class B



Class C



Class D



Class E

Class A Airspace
Class A airspace is generally the airspace from 18,000 feet
mean sea level (MSL) up to and including flight level (FL)
600, including the airspace overlying the waters within 12
nautical miles (NM) of the coast of the 48 contiguous states
and Alaska. Unless otherwise authorized, all operation in Class
A airspace is conducted under instrument flight rules (IFR).
Class B Airspace
Class B airspace is generally airspace from the surface to
10,000 feet MSL surrounding the nation's busiest airports in
terms of airport operations or passenger enplanements. The
configuration of each Class B airspace area is individually
tailored, consists of a surface area and two or more layers
(some Class B airspace areas resemble upside-down wedding
cakes), and is designed to contain all published instrument
procedures once an aircraft enters the airspace. ATC
clearance is required for all aircraft to operate in the area,
and all aircraft that are so cleared receive separation services
within the airspace.
15-2

Class C Airspace
Class C airspace is generally airspace from the surface to
4,000 feet above the airport elevation (charted in MSL)
surrounding those airports that have an operational control
tower, are serviced by a radar approach control, and have a
certain number of IFR operations or passenger enplanements.
Although the configuration of each Class C area is
individually tailored, the airspace usually consists of a surface
area with a five NM radius, an outer circle with a ten NM
radius that extends from 1,200 feet to 4,000 feet above the
airport elevation. Each aircraft must establish two-way radio
communications with the ATC facility providing air traffic
services prior to entering the airspace and thereafter must
maintain those communications while within the airspace.
Class D Airspace
Class D airspace is generally airspace from the surface to
2,500 feet above the airport elevation (charted in MSL)
surrounding those airports that have an operational control
tower. The configuration of each Class D airspace area is
individually tailored and, when instrument procedures are
published, the airspace is normally designed to contain the
procedures. Arrival extensions for instrument approach
procedures (IAPs) may be Class D or Class E airspace. Unless
otherwise authorized, each aircraft must establish two-way
radio communications with the ATC facility providing air
traffic services prior to entering the airspace and thereafter
maintain those communications while in the airspace.
Class E Airspace
Class E airspace is the controlled airspace not classified as
Class A, B, C, or D airspace. A large amount of the airspace
over the United States is designated as Class E airspace.

This provides sufficient airspace for the safe control and
separation of aircraft during IFR operations. Chapter 3 of
the Aeronautical Information Manual (AIM) explains the
various types of Class E airspace.
Sectional and other charts depict all locations of Class E
airspace with bases below 14,500 feet MSL. In areas where
charts do not depict a class E base, class E begins at 14,500
feet MSL.
In most areas, the Class E airspace base is 1,200 feet AGL. In
many other areas, the Class E airspace base is either the surface
or 700 feet AGL. Some Class E airspace begins at an MSL
altitude depicted on the charts, instead of an AGL altitude.
Class E airspace typically extends up to, but not including,
18,000 feet MSL (the lower limit of Class A airspace). All
airspace above FL 600 is Class E airspace.

Uncontrolled Airspace
Class G Airspace
Uncontrolled airspace or Class G airspace is the portion of
the airspace that has not been designated as Class A, B, C,
D, or E. It is therefore designated uncontrolled airspace.
Class G airspace extends from the surface to the base of the
overlying Class E airspace. Although ATC has no authority
or responsibility to control air traffic, pilots should remember
there are visual flight rules (VFR) minimums that apply to
Class G airspace.

Special Use Airspace
Special use airspace or special area of operation (SAO)
is the designation for airspace in which certain activities
must be confined, or where limitations may be imposed
on aircraft operations that are not part of those activities.
Certain special use airspace areas can create limitations on
the mixed use of airspace. The special use airspace depicted
on instrument charts includes the area name or number,
effective altitude, time and weather conditions of operation,
the controlling agency, and the chart panel location. On
National Aeronautical Charting Group (NACG) en route
charts, this information is available on one of the end panels.
Special use airspace usually consists of:

Prohibited Areas
Prohibited areas contain airspace of defined dimensions
within which the flight of aircraft is prohibited. Such areas
are established for security or other reasons associated with
the national welfare. These areas are published in the Federal
Register and are depicted on aeronautical charts. The area is
charted as a "P" followed by a number (e.g., P-40). Examples
of prohibited areas include Camp David and the National
Mall in Washington, D.C., where the White House and the
Congressional buildings are located. [Figure 15-2]
Restricted Areas
Restricted areas are areas where operations are hazardous to
nonparticipating aircraft and contain airspace within which
the flight of aircraft, while not wholly prohibited, is subject
to restrictions. Activities within these areas must be confined
because of their nature, or limitations may be imposed upon
aircraft operations that are not a part of those activities, or
both. Restricted areas denote the existence of unusual, often
invisible, hazards to aircraft (e.g., artillery firing, aerial
gunnery, or guided missiles). IFR flights may be authorized
to transit the airspace and are routed accordingly. Penetration
of restricted areas without authorization from the using
or controlling agency may be extremely hazardous to the
aircraft and its occupants. ATC facilities apply the following
procedures when aircraft are operating on an IFR clearance
(including those cleared by ATC to maintain VFR on top) via
a route that lies within joint-use restricted airspace:
1.	 If the restricted area is not active and has been released
to the Federal Aviation Administration (FAA), the
ATC facility allows the aircraft to operate in the
restricted airspace without issuing specific clearance
for it to do so.
2.	 If the restricted area is active and has not been released
to the FAA, the ATC facility issues a clearance that
ensures the aircraft avoids the restricted airspace.

 	 Prohibited areas
 	 Restricted areas
 	 Warning areas
 	 Military operation areas (MOAs)
 	 Alert areas
 	 Controlled firing areas (CFAs)

Figure 15-2. An example of a prohibited area, P-40 around Camp

David.

15-3

Restricted areas are charted with an "R" followed by a
number (e.g., R-4401) and are depicted on the en route
chart appropriate for use at the altitude or FL being flown.
[Figure 15-3] Restricted area information can be obtained
on the back of the chart.
Warning Areas
Warning areas are similar in nature to restricted areas;
however, the United States government does not have sole
jurisdiction over the airspace. A warning area is airspace of
defined dimensions, extending from 3 NM outward from
the coast of the United States, containing activity that may
be hazardous to nonparticipating aircraft. The purpose of
such areas is to warn nonparticipating pilots of the potential
danger. A warning area may be located over domestic or
international waters or both. The airspace is designated with
a "W" followed by a number (e.g., W-237). [Figure 15-4]
Military Operation Areas (MOAs)
MOAs consist of airspace with defined vertical and lateral
limits established for the purpose of separating certain
military training activities from IFR traffic. Whenever an
MOA is being used, nonparticipating IFR traffic may be
cleared through an MOA if IFR separation can be provided by
ATC. Otherwise, ATC reroutes or restricts nonparticipating
IFR traffic. MOAs are depicted on sectional, VFR terminal
area, and en route low altitude charts and are not numbered
(e.g., "Camden Ridge MOA"). [Figure 15-5] However, the
MOA is also further defined on the back of the sectional
charts with times of operation, altitudes affected, and the
controlling agency.
Alert Areas
Alert areas are depicted on aeronautical charts with an "A"
followed by a number (e.g., A-211) to inform nonparticipating

Figure 15-4. Requirements for airspace operations.

pilots of areas that may contain a high volume of pilot training
or an unusual type of aerial activity. Pilots should exercise
caution in alert areas. All activity within an alert area shall
be conducted in accordance with regulations, without waiver,
and pilots of participating aircraft, as well as pilots transiting
the area, shall be equally responsible for collision avoidance.
[Figure 15-6]
Controlled Firing Areas (CFAs)
CFAs contain activities that, if not conducted in a controlled
environment, could be hazardous to nonparticipating aircraft.
The difference between CFAs and other special use airspace
is that activities must be suspended when a spotter aircraft,
radar, or ground lookout position indicates an aircraft might
be approaching the area. There is no need to chart CFAs
since they do not cause a nonparticipating aircraft to change
its flight path.

Other Airspace Areas
"Other airspace areas" is a general term referring to the
majority of the remaining airspace. It includes:
 	 Local airport advisory (LAA)
 	 Military training route (MTR)
 	 Temporary flight restriction (TFR)
 	 Parachute jump aircraft operations
 	 Published VFR routes
 	 Terminal radar service area (TRSA)
 	 National security area (NSA)

Figure 15-3. Restricted areas on a sectional chart.

15-4



Air Defense Identification Zones (ADIZ) land and
water based and need for Defense VFR (DVFR) flight
plan to operate VFR in this airspace



Intercept Procedures and use of 121.5 for
communication if not on ATC already

Figure 15-5. Camden Ridge MOA is an example of a military operations area.

Figure 15-6. Alert area (A-211).

15-5

 	 Flight Restricted Zones (FRZ) in vicinity of Capitol
and White House


Special Awareness Training required by
14 CFR 91.161 for pilots to operate VFR within 60
NM of the Washington, DC VOR/DME

 	 Wildlife Areas/Wilderness Areas/National Parks and
request to operate above 2,000 AGL


National Oceanic and Atmospheric Administration
(NOAA) Marine Areas off the coast with requirement
to operate above 2,000 AGL



Tethered Balloons for observation and weather
recordings that extend on cables up to 60,000

Local Airport Advisory (LAA)
An advisory service provided by Flight Service Station
(FSS) facilities, which are located on the landing airport,
using a discrete ground-to-air frequency or the tower
frequency when the tower is closed. LAA services include
local airport advisories, automated weather reporting with
voice broadcasting, and a continuous Automated Surface
Observing System (ASOS)/Automated Weather Observing
Station (AWOS) data display, other continuous direct reading
instruments, or manual observations available to the specialist.
Military Training Routes (MTRs)
MTRs are routes used by military aircraft to maintain
proficiency in tactical flying. These routes are usually
established below 10,000 feet MSL for operations at speeds
in excess of 250 knots. Some route segments may be defined
at higher altitudes for purposes of route continuity. Routes
are identified as IFR (IR), and VFR (VR), followed by
a number. [Figure 15-7] MTRs with no segment above
1,500 feet AGL are identified by four number characters
(e.g., IR1206, VR1207). MTRs that include one or more
segments above 1,500 feet AGL are identified by three
number characters (e.g., IR206, VR207). IFR low altitude
en route charts depict all IR routes and all VR routes that

accommodate operations above 1,500 feet AGL. IR routes
are conducted in accordance with IFR regardless of weather
conditions. VFR sectional charts depict military training
activities, such as IR, VR, MOA, restricted area, warning
area, and alert area information.
Temporary Flight Restrictions (TFR)
A flight data center (FDC) Notice to Airmen (NOTAM)
is issued to designate a TFR. The NOTAM begins with
the phrase "FLIGHT RESTRICTIONS" followed by the
location of the temporary restriction, effective time period,
area defined in statute miles, and altitudes affected. The
NOTAM also contains the FAA coordination facility and
telephone number, the reason for the restriction, and any other
information deemed appropriate. The pilot should check the
NOTAMs as part of flight planning.
Some of the purposes for establishing a TFR are:


Protect persons and property in the air or on the surface
from an existing or imminent hazard.



Provide a safe environment for the operation of
disaster relief aircraft.

 	 Prevent an unsafe congestion of sightseeing aircraft
above an incident or event, that may generate a high
degree of public interest.
 	 Protect declared national disasters for humanitarian
reasons in the State of Hawaii.
 	 Protect the President, Vice President, or other public
figures.


Provide a safe environment for space agency
operations.

Since the events of September 11, 2001, the use of TFRs has
become much more common. There have been a number of
incidents of aircraft incursions into TFRs that have resulted
in pilots undergoing security investigations and certificate
suspensions. It is a pilot's responsibility to be aware of TFRs
in their proposed area of flight. One way to check is to visit
the FAA website, \url{tfr.faa.gov}, and verify that there is
not a TFR in the area.
Parachute Jump Aircraft Operations
Parachute jump aircraft operations are published in the Chart
Supplement U.S. (formerly Airport/Facility Directory). Sites
that are used frequently are depicted on sectional charts.

Military Training
Route (MTR)

Figure 15-7. Military training route (MTR) chart symbols.

15-6

Published VFR Routes
Published VFR routes are for transitioning around, under, or
through some complex airspace. Terms such as VFR flyway,
VFR corridor, Class B airspace VFR transition route, and
terminal area VFR route have been applied to such routes.

These routes are generally found on VFR terminal area
planning charts.
Terminal Radar Service Areas (TRSAs)
TRSAs are areas where participating pilots can receive
additional radar services. The purpose of the service is
to provide separation between all IFR operations and
participating VFR aircraft.
The primary airport(s) within the TRSA become(s) Class D
airspace. The remaining portion of the TRSA overlies other
controlled airspace, which is normally Class E airspace
beginning at 700 or 1,200 feet and established to transition to/
from the en route/terminal environment. TRSAs are depicted
on VFR sectional charts and terminal area charts with a solid
black line and altitudes for each segment. The Class D portion
is charted with a blue segmented line. Participation in TRSA
services is voluntary; however, pilots operating under VFR
are encouraged to contact the radar approach control and take
advantage of TRSA service.
National Security Areas (NSAs)
NSAs consist of airspace of defined vertical and lateral
dimensions established at locations where there is a
requirement for increased security and safety of ground
facilities. Flight in NSAs may be temporarily prohibited by
regulation under the provisions of Title 14 of the Code of
Federal Regulations (14 CFR) part 99, and prohibitions are
disseminated via NOTAM. Pilots are requested to voluntarily
avoid flying through these depicted areas.

Air Traffic Control and the National
Airspace System
The primary purpose of the ATC system is to prevent a
collision between aircraft operating in the system and to
organize and expedite the flow of traffic. In addition to
its primary function, the ATC system has the capability to
provide (with certain limitations) additional services. The
ability to provide additional services is limited by many
factors, such as the volume of traffic, frequency congestion,
quality of radar, controller workload, higher priority duties,
and the pure physical inability to scan and detect those
situations that fall in this category. It is recognized that these
services cannot be provided in cases in which the provision
of services is precluded by the above factors.
Consistent with the aforementioned conditions, controllers
shall provide additional service procedures to the extent
permitted by higher priority duties and other circumstances.
The provision of additional services is not optional on the
part of the controller, but rather is required when the work
situation permits. Provide ATC service in accordance with
the procedures and minima in this order except when:

1.	 A deviation is necessary to conform to ICAO
Documents, National Rules of the Air, or special
agreements where the United States provides ATC
service in airspace outside the country and its
possessions
2. 	 Other procedures/minima are prescribed in a letter of
agreement, FAA directive, or a military document
3. 	 A deviation is necessary to assist an aircraft when an
emergency has been declared
Coordinating the Use of Airspace
ATC is responsible for ensuring that the necessary
coordination has been accomplished before allowing an
aircraft under their control to enter another controller's area
of jurisdiction.
Before issuing control instructions directly or relaying
through another source to an aircraft that is within another
controller's area of jurisdiction that will change that
aircraft's heading, route, speed, or altitude, ATC ensures
that coordination has been accomplished with each of the
controllers listed below whose area of jurisdiction is affected
by those instructions unless otherwise specified by a letter
of agreement or a facility directive:
1. 	 The controller within whose area of jurisdiction the
control instructions are issued
2. 	 The controller receiving the transfer of control
3. 	 Any intervening controller(s) through whose area of
jurisdiction the aircraft will pass
If ATC issues control instructions to an aircraft through a
source other than another controller (e.g., Aeronautical Radio,
Incorporated (ARINC), FSS, another pilot), they ensure that
the necessary coordination has been accomplished with any
controllers listed above, whose area of jurisdiction is affected
by those instructions unless otherwise specified by a letter
of agreement or a facility directive.
Operating in the Various Types of Airspace
It is important that pilots be familiar with the operational
requirements for each of the various types or classes of
airspace. Subsequent sections cover each class in sufficient
detail to facilitate understanding regarding weather, type of
pilot certificate held, and equipment required.

Basic VFR Weather Minimums
No pilot may operate an aircraft under basic VFR when the
flight visibility is less, or at a distance from clouds that is
less, than that prescribed for the corresponding altitude and
class of airspace. [Figure 15-8] Except as provided in 14 CFR
part 91, section 91.157, "Special VFR Weather Minimums,"
15-7

Basic VFR Weather Minimums
Airspace

Flight Visibility

Distance from Clouds

Class

A

Not applicable

Not applicable

Class

B

3 statute miles

Clear of clouds

C

3 statute miles

Class

1,000 feet above
500 feet below
2,000 feet horizontal

Class

D

3 statute miles

1,000 feet above
500 feet below
2,000 feet horizontal

5 statute miles

1,000 feet above
1,000 feet below
1 statute mile horizontal

3 statute miles

1,000 feet above
500 feet below
2,000 feet horizontal

Day, except as provided in section 91.155(b)

1 statute mile

Clear of clouds

Night, except as provided in section 91.155(b)

3 statute miles

1,000 feet above
500 feet below
2,000 feet horizontal

Day

1 statute mile

1,000 feet above
500 feet below
2,000 feet horizontal

Night

3 statute miles

1,000 feet above
500 feet below
2,000 feet horizontal

5 statute miles

1,000 feet above
1,000 feet below
1 statute mile horizontal

At or above
10,000 feet MSL
Class

E
Less than
10,000 feet MSL

1,200 feet or less
above the surface
(regardless of
MSL altitude).

Class

G

More than 1,200
feet above the
surface but less
than 10,000 feet
MSL.
More than 1,200
feet above the
surface and at or
above 10,000 feet
MSL.

Figure 15-8. Visual flight rule weather minimums.

no person may operate an aircraft beneath the ceiling under
VFR within the lateral boundaries of controlled airspace
designated to the surface for an airport when the ceiling is
less than 1,000 feet. Additional information can be found in
14 CFR part 91, section 91.155(c).

Operating Rules and Pilot/Equipment Requirements
The safety of flight is a top priority of all pilots and the
responsibilities associated with operating an aircraft
should always be taken seriously. The air traffic system
maintains a high degree of safety and efficiency with strict
regulatory oversight of the FAA. Pilots fly in accordance
with regulations that have served the United States well, as
evidenced by the fact that the country has the safest aviation
system in the world.

15-8

All aircraft operating in today's National Airspace System
(NAS) has complied with the CFR governing its certification
and maintenance; all pilots operating today have completed
rigorous pilot certification training and testing. Of equal
importance is the proper execution of preflight planning,
aeronautical decision-making (ADM) and risk management.
ADM involves a systematic approach to risk assessment
and stress management in aviation, illustrates how personal
attitudes can influence decision-making, and how those
attitudes can be modified to enhance safety in the flight
deck. More detailed information regarding ADM and
risk mitigation can be found in Chapter 2, "Aeronautical
Decision-Making."

Pilots also comply with very strict FAA general operating
and flight rules as outlined in the CFR, including the FAA's
important "see and avoid" mandate. These regulations provide
the historical foundation of the FAA regulations governing
the aviation system and the individual classes of airspace.
Figure 15-9 lists the operational and equipment requirements
for these various classes of airspace. It is helpful to refer to this
figure as the specific classes are discussed in greater detail.
Class A
Pilots operating an aircraft in Class A airspace must conduct
that operation under IFR and only under an ATC clearance
received prior to entering the airspace. Unless otherwise
authorized by ATC, each aircraft operating in Class A
airspace must be equipped with a two-way radio capable of
communicating with ATC on a frequency assigned by ATC.
Unless otherwise authorized by ATC, all aircraft within
Class A airspace must be equipped with the appropriate
transponder equipment meeting all applicable specifications
found in 14 CFR part 91, section 91.215. Additionally,
beginning January 1, 2020, aircraft operating in the Class
A airspace described in 14 CFR part 91, section 91.225,
must have ADS-B Out equipment installed, which meets the
performance requirements of 14 CFR part 91, section 91.227.
Class B
All pilots operating an aircraft within a Class B airspace area
must receive an ATC clearance from the ATC facility having
jurisdiction for that area. The pilot in command (PIC) may
not take off or land an aircraft at an airport within a Class
B airspace unless he or she has met one of the following
requirements:
1. 	 A private pilot certificate

Class
Airspace

Entry Requirements

2.	 A recreational pilot certificate and all requirements
contained within 14 CFR part 61, section 61.101(d), or
the requirements for a student pilot seeking a recreational
pilot certificate in 14 CFR part 61, section 61.94.
3.	 A sport pilot certificate and all requirements contained
within 14 CFR part 61, section 61.325, or the
requirements for a student pilot seeking a recreational
pilot certificate in 14 CFR part 61, section 61.94, or
the aircraft is operated by a student pilot who has met
the requirements of 14 CFR part 61, sections 61.94
and 61.95, as applicable.
Unless otherwise authorized by ATC, all aircraft within Class
B airspace must be equipped with the applicable operating
transponder and automatic altitude reporting equipment
specified in 14 CFR part 91, section 91.215(a) and an
operable two-way radio capable of communications with
ATC on appropriate frequencies for that Class B airspace
area. Additionally, beginning January 1, 2020, aircraft
operating in the Class B airspace described in 14 CFR part 91,
section 91.225, must have ADS-B Out equipment installed,
which meets the performance requirements of 14 CFR part
91, section 91.227.
Class C
For the purpose of this section, the primary airport is the
airport for which the Class C airspace area is designated. A
satellite airport is any other airport within the Class C airspace
area. No pilot may take off or land an aircraft at a satellite
airport within a Class C airspace area except in compliance
with FAA arrival and departure traffic patterns.
Two-way radio communications must be established and
maintained with the ATC facility providing air traffic services
Equipment*

Minimum Pilot Certificate

Class

A

ATC clearance

IFR equipped

Instrument rating

Class

B

ATC clearance

Two-way radio, transponder
with altitude reporting capability

Private—(However, a student or
recreational pilot may operate at
other than the primary airport if
seeking private pilot certification and
if regulatory requirements are met.)

Class

C

Two-way radio communications
prior to entry

Two-way radio, transponder
with altitude reporting capability

No specific requirement

Class

D

Two-way radio communications
prior to entry

Two-way radio

No specific requirement

Class

E

None for VFR

No specific requirement

No specific requirement

Class

G

None

No specific requirement

No specific requirement

*Beginning January 1, 2020, ADS-B Out equipment may be required in accordance with 14 CFR part 91, section 91.225.

Figure 15-9. Requirements for airspace operations.

15-9

prior to entering the airspace and thereafter maintained while
within the airspace.
A pilot departing from the primary airport or satellite airport
with an operating control tower must establish and maintain
two-way radio communications with the control tower,
and thereafter as instructed by ATC while operating in the
Class C airspace area. If departing from a satellite airport
without an operating control tower, the pilot must establish
and maintain two-way radio communications with the ATC
facility having jurisdiction over the Class C airspace area as
soon as practicable after departing.
Unless otherwise authorized by the ATC having jurisdiction
over the Class C airspace area, all aircraft within Class C
airspace must be equipped with the appropriate transponder
equipment meeting all applicable specifications found in 14
CFR part 91, section 91.215. Additionally, beginning January
1, 2020, aircraft operating in the Class C airspace described
in 14 CFR part 91, section 91.225, must have ADS-B
Out equipment installed, which meets the performance
requirements of 14 CFR part 91, section 91.227.
Class D
No pilot may take off or land an aircraft at a satellite airport
within a Class D airspace area except in compliance with
FAA arrival and departure traffic patterns. A pilot departing
from the primary airport or satellite airport with an operating
control tower must establish and maintain two-way radio
communications with the control tower, and thereafter as
instructed by ATC while operating in the Class D airspace
area. If departing from a satellite airport without an operating
control tower, the pilot must establish and maintain twoway radio communications with the ATC facility having
jurisdiction over the Class D airspace area as soon as
practicable after departing.
Two-way radio communications must be established and
maintained with the ATC facility providing air traffic services
prior to entering the airspace and thereafter maintained while
within the airspace.
If the aircraft radio fails in flight under IFR, the pilot should
continue the flight by the route assigned in the last ATC
clearance received; or, if being radar vectored, by the direct
route from the point of radio failure to the fix, route, or
airway specified in the vector clearance. In the absence of
an assigned route, the pilot should continue by the route
that ATC advised may be expected in a further clearance;
or, if a route had not been advised, by the route filed in the
flight plan.

15-10

If the aircraft radio fails in flight under VFR, the PIC may
operate that aircraft and land if weather conditions are at or
above basic VFR weather minimums, visual contact with
the tower is maintained, and a clearance to land is received.
Class E
Unless otherwise required by 14 CFR part 93 or unless
otherwise authorized or required by the ATC facility having
jurisdiction over the Class E airspace area, each pilot
operating an aircraft on or in the vicinity of an airport in a
Class E airspace area must comply with the requirements
of Class G airspace. Each pilot must also comply with any
traffic patterns established for that airport in 14 CFR part 93.
Unless otherwise authorized or required by ATC, no person
may operate an aircraft to, from, through, or on an airport
having an operational control tower unless two-way radio
communications are maintained between that aircraft and the
control tower. Communications must be established within
four nautical miles from the airport, up to and including 2,500
feet AGL. However, if the aircraft radio fails in flight, the PIC
may operate that aircraft and land if weather conditions are at
or above basic VFR weather minimums, visual contact with
the tower is maintained, and a clearance to land is received.
If the aircraft radio fails in flight under IFR, the pilot should
continue the flight by the route assigned in the last ATC
clearance received; or, if being radar vectored, by the direct
route from the point of radio failure to the fix, route, or
airway specified in the vector clearance. In the absence of
an assigned route, the pilot should continue by the route
that ATC advised may be expected in a further clearance;
or, if a route had not been advised, by the route filed in the
flight plan. Additionally, beginning January 1, 2020, aircraft
operating in the Class E airspace described in 14 CFR part 91,
section 91.225, must have ADS-B Out equipment installed,
which meets the performance requirements of 14 CFR part
91, section 91.227.
Class G
When approaching to land at an airport without an operating
control tower in Class G airspace:
1. 	 Each pilot of an airplane must make all turns of that
airplane to the left unless the airport displays approved
light signals or visual markings indicating that turns
should be made to the right, in which case the pilot
must make all turns to the right.
2.	 Each pilot of a helicopter or a powered parachute must
avoid the flow of fixed-wing aircraft.

Unless otherwise authorized or required by ATC, no person
may operate an aircraft to, from, through, or on an airport
having an operational control tower unless two-way radio
communications are maintained between that aircraft and the
control tower. Communications must be established within
four nautical miles from the airport, up to and including 2,500
feet AGL. However, if the aircraft radio fails in flight, the PIC
may operate that aircraft and land if weather conditions are at
or above basic VFR weather minimums, visual contact with
the tower is maintained, and a clearance to land is received.
If the aircraft radio fails in flight under IFR, the pilot should
continue the flight by the route assigned in the last ATC
clearance received; or, if being radar vectored, by the direct
route from the point of radio failure to the fix, route, or airway
specified in the vector clearance. In the absence of an assigned
route, the pilot should continue by the route that ATC advised
may be expected in a further clearance; or, if a route had not
been advised, by the route filed in the flight plan.
Uncontrolled Airspace
It is possible for some airports within Class G airspace to
have a control tower (Lake City, FL, for example). Be sure to
check the Chart Supplement U.S. (formerly Airport/Facility
Directory) to be familiar with the airport and associated
airspace prior to flight.

Unmanned Free Balloons
Unless otherwise authorized by ATC, no person may operate
an unmanned free balloon below 2,000 feet above the surface
within the lateral boundaries of Class B, Class C, Class D,
or Class E airspace designated for an airport. (See 14 CFR
part 101.)

Unmanned Aircraft Systems
Regulations regarding unmanned aircraft systems (UAS) are
currently being developed and are expected to be published
by summer 2016 as 14 CFR part 107.

Parachute Jumps
No person may make a parachute jump, and no PIC may
allow a parachute jump to be made from an aircraft, in or
into Class A, Class B, Class C, or Class D airspace without,
or in violation of, the terms of an ATC authorization issued
by the ATC facility having jurisdiction over the airspace.
(See 14 CFR part 105.)

Chapter Summary
This chapter introduces the various classifications of airspace
and provides information on the requirements to operate in
such airspace. For further information, consult the AIM and
14 CFR parts 71, 73, and 91.

Ultralight Vehicles
No person may operate an ultralight vehicle within Class A,
Class B, Class C, or Class D airspace or within the lateral
boundaries of the surface area of Class E airspace designated
for an airport unless that person has prior authorization from
the ATC facility having jurisdiction over that airspace. (See
14 CFR part 103.)

15-11

15-12


Chapter 16

Navigation
Introduction
This chapter provides an introduction to cross-country
flying under visual flight rules (VFR). It contains practical
information for planning and executing cross-country flights
for the beginning pilot.
Air navigation is the process of piloting an aircraft from
one geographic position to another while monitoring one's
position as the flight progresses. It introduces the need for
planning, which includes plotting the course on an aeronautical
chart, selecting checkpoints, measuring distances, obtaining
pertinent weather information, and computing flight time,
headings, and fuel requirements. The methods used in this
chapter include pilotage—navigating by reference to visible
landmarks, dead reckoning—computations of direction and
distance from a known position, and radio navigation—by
use of radio aids.

16-1

Aeronautical Charts
An aeronautical chart is the road map for a pilot flying under
VFR. The chart provides information that allows pilots to track
their position and provides available information that enhances
safety. The three aeronautical charts used by VFR pilots are:


Sectional



VFR Terminal Area



World Aeronautical

A free catalog listing aeronautical charts and related
publications including prices and instructions for ordering is
available at the Aeronautical Navigation Products website:
\url{aeronav.faa.gov}.
Sectional Charts
Sectional charts are the most common charts used by pilots
today. The charts have a scale of 1:500,000 (1 inch = 6.86
nautical miles (NM) or approximately 8 statute miles (SM)),
which allows for more detailed information to be included
on the chart.

Figure 16-1. Sectional chart and legend.

16-2

The charts provide an abundance of information, including
airport data, navigational aids, airspace, and topography.
Figure 16-1 is an excerpt from the legend of a sectional
chart. By referring to the chart legend, a pilot can interpret
most of the information on the chart. A pilot should also
check the chart for other legend information, which includes
air traffic control (ATC) frequencies and information on
airspace. These charts are revised semiannually except for
some areas outside the conterminous United States where
they are revised annually.
VFR Terminal Area Charts
VFR terminal area charts are helpful when flying in or near
Class B airspace. They have a scale of 1:250,000 (1 inch
= 3.43 NM or approximately 4 SM). These charts provide
a more detailed display of topographical information and
are revised semiannually, except for several Alaskan and
Caribbean charts. [Figure 16-2]
World Aeronautical Charts
World aeronautical charts are designed to provide a standard
series of aeronautical charts, covering land areas of the world,

Figure 16-2. VFR Terminal Area Chart and legend.

at a size and scale convenient for navigation by moderate speed
aircraft. They are produced at a scale of 1:1,000,000 (1 inch =
13.7 NM or approximately 16 SM). These charts are similar to
sectional charts, and the symbols are the same except there is
less detail due to the smaller scale. [Figure 16-3] These charts
are revised annually except several Alaskan charts and the
Mexican/Caribbean charts, which are revised every 2 years.

Latitude and Longitude (Meridians and
Parallels)
The equator is an imaginary circle equidistant from the poles
of the Earth. Circles parallel to the equator (lines running east
and west) are parallels of latitude. They are used to measure
degrees of latitude north (N) or south (S) of the equator. The
angular distance from the equator to the pole is one-fourth
of a circle or 90°. The 48 conterminous states of the United
States are located between 25° and 49° N latitude. The arrows
in Figure 16-4 labeled "Latitude" point to lines of latitude.
Meridians of longitude are drawn from the North Pole to the
South Pole and are at right angles to the Equator. The "Prime

Meridian," which passes through Greenwich, England, is
used as the zero line from which measurements are made in
degrees east (E) and west (W) to 180°. The 48 conterminous
states of the United States are between 67° and 125° W
longitude. The arrows in Figure 16-4 labeled "Longitude"
point to lines of longitude.
Any specific geographical point can be located by reference
to its longitude and latitude. Washington, D.C., for example,
is approximately 39° N latitude, 77° W longitude. Chicago
is approximately 42° N latitude, 88° W longitude.
Time Zones
The meridians are also useful for designating time zones. A
day is defined as the time required for the Earth to make one
complete rotation of 360°. Since the day is divided into 24
hours, the Earth revolves at the rate of 15° an hour. Noon is
the time when the sun is directly above a meridian; to the
west of that meridian is morning, to the east is afternoon.

16-3

Figure 16-3. World aeronautical chart.
90°N

75°N

60°N

5
1
15

n
r idia
me
W
°
°W

30°W

45°W

60°W

7 5° W

90°W

105°W

120 °W

13 5°
W

1
15
50
0°
°W
W

15°N

e
ri m
P

30°N

Lo
ngi
tude

de
titu
La

45°N

E quator

15°S
30°S
45°S

The standard practice is to establish a time zone for each
15° of longitude. This makes a difference of exactly 1 hour
between each zone. In the conterminous United States,
there are four time zones. The time zones are Eastern (75°),
Central (90°), Mountain (105°), and Pacific (120°). The
dividing lines are somewhat irregular because communities
near the boundaries often find it more convenient to use time
designations of neighboring communities or trade centers.
Figure 16-5 shows the time zones in the conterminous United
States. When the sun is directly above the 90th meridian, it
is noon Central Standard Time. At the same time, it is 1 p.m.
Eastern Standard Time, 11 a.m. Mountain Standard Time,
and 10 a.m. Pacific Standard Time. When Daylight Saving
Time is in effect, generally between the second Sunday in
March and the first Sunday in November, the sun is directly
above the 75th meridian at noon, Central Daylight Time.

60°S

Figure 16-4. Meridians and parallels—the basis of measuring time,
distance, and direction.

16-4

These time zone differences must be taken into account
during long flights eastward—especially if the flight must
be completed before dark. Remember, an hour is lost when

Pacific standard time

Mountain standard time

Central standard time

Eastern standard time

75°
120°

105°

90°

Figure 16-5. Time zones in the conterminous United States.

flying eastward from one time zone to another, or perhaps
even when flying from the western edge to the eastern edge
of the same time zone. Determine the time of sunset at the
destination by consulting the flight service station (FSS) and
take this into account when planning an eastbound flight.

For Daylight Saving Time, 1 hour should be subtracted from
the calculated times.

In most aviation operations, time is expressed in terms of
the 24-hour clock. ATC instructions, weather reports and
broadcasts, and estimated times of arrival are all based on
this system. For example: 9 a.m. is expressed as 0900, 1 p.m.
is 1300, and 10 p.m. is 2200.

Measurement of Direction
By using the meridians, direction from one point to another
can be measured in degrees, in a clockwise direction from
true north. To indicate a course to be followed in flight,
draw a line on the chart from the point of departure to the
destination and measure the angle that this line forms with
a meridian. Direction is expressed in degrees, as shown by
the compass rose in Figure 16-6.

Because a pilot may cross several time zones during a flight, a
standard time system has been adopted. It is called Universal
Coordinated Time (UTC) and is often referred to as Zulu
time. UTC is the time at the 0° line of longitude which passes
through Greenwich, England. All of the time zones around
the world are based on this reference. To convert to this time,
a pilot should do the following:

Because meridians converge toward the poles, course
measurement should be taken at a meridian near the midpoint
of the course rather than at the point of departure. The course
measured on the chart is known as the true course (TC). This
is the direction measured by reference to a meridian or true
north (TN). It is the direction of intended flight as measured
in degrees clockwise from TN.

Eastern Standard Time ..........Add 5 hours

Central Standard Time ..........Add 6 hours

Mountain Standard Time.......Add 7 hours


As shown in Figure 16-7, the direction from A to B would be
a TC of 065°, whereas the return trip (called the reciprocal)
would be a TC of 245°.

Pacific Standard Time ...........Add 8 hours

16-5

36

33

N

NN

E

W

N

6

9

ES

W
S

E
S

W

W

E

E
S
S

E

S

SS

W

24

S

2
1

1
2

27

W

N
E

W
N

N

W

E

N

W
N

3

N

3
0

The north magnetic pole is located close to 71° N latitude, 96°
W longitude and is about 1,300 miles from the geographic
or true north pole, as indicated in Figure 16-8. If the Earth
were uniformly magnetized, the compass needle would point
toward the magnetic pole, in which case the variation between
TN (as shown by the geographical meridians) and MN (as
shown by the magnetic meridians) could be measured at any
intersection of the meridians.

15

18

Figure 16-6. Compass rose.

The true heading (TH) is the direction in which the nose of
the aircraft points during a flight when measured in degrees
clockwise from TN. Usually, it is necessary to head the
aircraft in a direction slightly different from the TC to offset
the effect of wind. Consequently, numerical value of the TH
may not correspond with that of the TC. This is discussed
more fully in subsequent sections in this chapter. For the
purpose of this discussion, assume a no-wind condition exists
under which heading and course would coincide. Thus, for
a TC of 065°, the TH would be 065°. To use the compass
accurately, however, corrections must be made for magnetic
variation and compass deviation.
Variation
Variation is the angle between TN and magnetic north (MN).
It is expressed as east variation or west variation depending
upon whether MN is to the east or west of TN.
Course A to B 065°
0 65
°

Actually, the Earth is not uniformly magnetized. In the United
States, the needle usually points in the general direction of
the magnetic pole, but it may vary in certain geographical
localities by many degrees. Consequently, the exact amount
of variation at thousands of selected locations in the United
States has been carefully determined. The amount and the
direction of variation, which change slightly from time
to time, are shown on most aeronautical charts as broken
magenta lines called isogonic lines that connect points of
equal magnetic variation. (The line connecting points at
which there is no variation between TN and MN is the agonic
line.) An isogonic chart is shown in Figure 16-9. Minor
bends and turns in the isogonic and agonic lines are caused
by unusual geological conditions affecting magnetic forces
in these areas.
On the west coast of the United States, the compass needle
points to the east of TN; on the east coast, the compass needle
points to the west of TN.

TN

MN

B

°

24 5

A

Course B to A 245°

Figure 16-7. Courses are determined by reference to meridians on

aeronautical charts.

16-6

Figure 16-8. Magnetic meridians are in red while the lines of
longitude and latitude are in blue. From these lines of variation
(magnetic meridians), one can determine the effect of local magnetic
variations on a magnetic compass.

Easterly variation

this course off the magnetic compass would not provide an
accurate course between the two points due to three elements
that must be considered. The first is magnetic variation, the
second is compass deviation, and the third is wind correction.
All three must be considered for accurate navigation.

Westerly variation

Magnetic Variation
As mentioned in the paragraph discussing variation, the
appropriate variation for the geographical location of
the flight must be considered and added or subtracted as
appropriate. If flying across an area where the variation
changes, then the values must be applied along the route of
flight appropriately. Once applied, this new course is called
the magnetic course.
Agonic line

Magnetic Deviation
Because each aircraft has its own internal effect upon the
onboard compass systems from its own localized magnetic
influencers, the pilot must add or subtract these influencers
based upon the direction he or she is flying. The application of
deviation (taken from a compass deviation card) compensates
the magnetic course unique to that aircraft's compass system
(as affected by localized magnetic influencers) and it now
becomes the compass course. Therefore, the compass course,
when followed (in a no wind condition), takes the aircraft
from point A to point B even though the aircraft heading
may not match the original course line drawn on the chart.

Figure 16-9. Note the agonic line where magnetic variation is zero.

Zero degree variation exists on the agonic line where
MN and TN coincide. This line runs roughly west of the
Great Lakes, south through Wisconsin, Illinois, western
Tennessee, and along the border of Mississippi and Alabama.
Compare Figures 16-9 and 16-10.
Because courses are measured in reference to geographical
meridians that point toward TN, and these courses are
maintained by reference to the compass that points along a
magnetic meridian in the general direction of MN, the true
direction must be converted into magnetic direction for the
purpose of flight. This conversion is made by adding or
subtracting the variation indicated by the nearest isogonic
line on the chart.

If the variation is shown as "9° E," this means that MN is
9° east of TN. If a TC of 360° is to be flown, 9° must be
subtracted from 360°, which results in a magnetic heading
of 351°. To fly east, a magnetic course of 081° (090° – 9°)
would be flown. To fly south, the magnetic course would be
171° (180° – 9°). To fly west, it would be 261° (270° – 9°).
To fly a TH of 060°, a magnetic course of 051° (060° – 9°)
would be flown.

For example, a line drawn between two points on a chart
is called a TC as it is measured from TN. However, flying
NP
E
varia as
tio t
n

NP

NP
MP

MP

MP

t
n
esiatio
W ar
v

Zero
variation

6
E

6
E
1
2

1
2

N

33

3

33

6

3

E

N

33

1
2

N
3

SP

24

SP

S

W

24

24

2
1

15

3
0

W

W

S

3
0

3
0

2
1

2
1

15

S

15

SP

Figure 16-10. Effect of variation on the compass.

16-7

Remember, if variation is west, add; if east, subtract. One
method for remembering whether to add or subtract variation
is the phrase "east is least (subtract) and west is best (add)."
Deviation
Determining the magnetic heading is an intermediate step
necessary to obtain the correct compass heading for the flight.
To determine compass heading, a correction for deviation
must be made. Because of magnetic influences within an
aircraft, such as electrical circuits, radio, lights, tools, engine,
and magnetized metal parts, the compass needle is frequently
deflected from its normal reading. This deflection is called
deviation. The deviation is different for each aircraft, and
it also may vary for different headings in the same aircraft.
For instance, if magnetism in the engine attracts the north
end of the compass, there would be no effect when the plane
is on a heading of MN. On easterly or westerly headings,
however, the compass indications would be in error, as shown
in Figure 16-11. Magnetic attraction can come from many
other parts of the aircraft; the assumption of attraction in the
engine is merely used for the purpose of illustration.
Some adjustment of the compass, referred to as compensation,
can be made to reduce this error, but the remaining correction
must be applied by the pilot.
Proper compensation of the compass is best performed by
a competent technician. Since the magnetic forces within
the aircraft change because of landing shocks, vibration,
mechanical work, or changes in equipment, the pilot should
occasionally have the deviation of the compass checked. The
procedure used to check the deviation is called "swinging the
compass" and is briefly outlined as follows.

Magnetic North

Magnetic North

Magnetic North
N
3

33

6

3
0

E

W

1
2

24

6
E

33

E

W

W

1
2

24

24
S

1
2

N
3
6

3
0

3
0

2
1

33

15

N

S

3

2
1

2
1

15

S

15

Magnetized engine

Figure 16-11. Magnetized portions of the airplane cause the

compass to deviate from its normal indications.

16-8

The aircraft is placed on a magnetic compass rose, the engine
started, and electrical devices normally used (such as radio)
are turned on. Tailwheel-type aircraft should be jacked up into
flying position. The aircraft is aligned with MN indicated on
the compass rose and the reading shown on the compass is
recorded on a deviation card. The aircraft is then aligned at
30° intervals and each reading is recorded. If the aircraft is to
be flown at night, the lights are turned on and any significant
changes in the readings are noted. If so, additional entries
are made for use at night. The accuracy of the compass can
also be checked by comparing the compass reading with the
known runway headings.
A deviation card, similar to Figure 16-12, is mounted near
the compass showing the addition or subtraction required to
correct for deviation on various headings, usually at intervals
of 30°. For intermediate readings, the pilot should be able to
interpolate mentally with sufficient accuracy. For example,
if the pilot needed the correction for 195° and noted the
correction for 180° to be 0° and for 210° to be +2°, it could
be assumed that the correction for 195° would be +1°. The
magnetic heading, when corrected for deviation, is known
as compass heading.

Effect of Wind
The preceding discussion explained how to measure a TC
on the aeronautical chart and how to make corrections for
variation and deviation, but one important factor has not
been considered—wind. As discussed in the study of the
atmosphere, wind is a mass of air moving over the surface of
the Earth in a definite direction. When the wind is blowing
from the north at 25 knots, it simply means that air is moving
southward over the Earth's surface at the rate of 25 NM in
1 hour.
Under these conditions, any inert object free from contact
with the Earth is carried 25 NM southward in 1 hour. This
effect becomes apparent when such things as clouds, dust,
and toy balloons are observed being blown along by the wind.
Obviously, an aircraft flying within the moving mass of air
is similarly affected. Even though the aircraft does not float
freely with the wind, it moves through the air at the same
time the air is moving over the ground, and thus is affected
by wind. Consequently, at the end of 1 hour of flight, the
aircraft is in a position that results from a combination of
the following two motions:
For (Magnetic)
Steer (Compass)

N
0

30
28

60
57

E
86

120
117

150
148

For (Magnetic
Steer (Compass)

S
180

210
212

240
243

W
274

300
303

330
332

Figure 16-12. Compass deviation card.



Movement of the air mass in reference to the ground



Forward movement of the aircraft through the air mass

Actually, these two motions are independent. It makes no
difference whether the mass of air through which the aircraft
is flying is moving or is stationary. A pilot flying in a 70knot gale would be totally unaware of any wind (except for
possible turbulence) unless the ground were observed. In
reference to the ground, however, the aircraft would appear
to fly faster with a tailwind or slower with a headwind, or to
drift right or left with a crosswind.
As shown in Figure 16-13, an aircraft flying eastward at
an airspeed of 120 knots in still air has a groundspeed (GS)
exactly the same—120 knots. If the mass of air is moving
eastward at 20 knots, the airspeed of the aircraft is not
affected, but the progress of the aircraft over the ground is
120 plus 20 or a GS of 140 knots. On the other hand, if the

WINDS ARE CALM

090°

Groundspeed 120 knots

WINDS 270° AT 20 KNOTS

090°

Groundspeed 140 knots

WINDS 090° AT 20 KNOTS

090°

mass of air is moving westward at 20 knots, the airspeed of
the aircraft remains the same, but GS becomes 120 minus
20 or 100 knots.
Assuming no correction is made for wind effect, if an aircraft
is heading eastward at 120 knots and the air mass moving
southward at 20 knots, the aircraft at the end of 1 hour is
almost 120 miles east of its point of departure because of its
progress through the air. It is 20 miles south because of the
motion of the air. Under these circumstances, the airspeed
remains 120 knots, but the GS is determined by combining
the movement of the aircraft with that of the air mass. GS can
be measured as the distance from the point of departure to
the position of the aircraft at the end of 1 hour. The GS can
be computed by the time required to fly between two points a
known distance apart. It also can be determined before flight
by constructing a wind triangle, which is explained later in
this chapter. [Figure 16-14]
The direction in which the aircraft is pointing as it flies is
called heading. Its actual path over the ground, which is a
combination of the motion of the aircraft and the motion of
the air, is called track. The angle between the heading and
the track is called drift angle. If the aircraft heading coincides
with the TC and the wind is blowing from the left, the track
does not coincide with the TC. The wind causes the aircraft
to drift to the right, so the track falls to the right of the desired
course or TC. [Figure 16-15]
The following method is used by many pilots to determine
compass heading: after the TC is measured, and wind
correction applied resulting in a TH, the sequence TH ±
variation (V) = magnetic heading (MH) ± deviation (D)
= compass heading (CH) is followed to arrive at compass
heading. [Figure 16-16]
By determining the amount of drift, the pilot can counteract
the effect of the wind and make the track of the aircraft
coincide with the desired course. If the mass of air is moving
across the course from the left, the aircraft drifts to the
right, and a correction must be made by heading the aircraft
sufficiently to the left to offset this drift. In other words, if
the wind is from the left, the correction is made by pointing
the aircraft to the left a certain number of degrees, therefore
correcting for wind drift. This is the wind correction angle
(WCA) and is expressed in terms of degrees right or left of
the TC. [Figure 16-17]

Groundspeed 100 knots
Figure 16-13. Motion of the air affects the speed with which aircraft

move over the Earth's surface. Airspeed, the rate at which an
aircraft moves through the air, is not affected by air motion.

16-9

Dista

nce

cove

red o

ver g

roun

20 knots

Airspeed effect (1 hour)

d (1

hour

)

Wind

Figure 16-14. Aircraft flight path resulting from its airspeed and direction and the wind speed and direction.

Heading
Desired course

Drift angle
Trac

k

Figure 16-15. Effects of wind drift on maintaining desired course.

To summarize:

MN C

N



Course—intended path of an aircraft over the ground
or the direction of a line drawn on a chart representing
the intended aircraft path, expressed as the angle
measured from a specific reference datum clockwise
from 0° through 360° to the line.



Heading—direction in which the nose of the aircraft
points during flight.



Track—actual path made over the ground in flight. (If
proper correction has been made for the wind, track
and course are identical.)



Drift angle—angle between heading and track.



WCA—correction applied to the course to establish
a heading so that track coincides with course.



Airspeed—rate of the aircraft's progress through
the air.



GS—rate of the aircraft's inflight progress over
the ground.

DEV 4

°

TN

CH
-0

°
88

VAR 10° E

7

8°

TH
-0

74
°

MH
­0

Heading

Figure 16-16. Relationship between true, magnetic, and compass

headings for a particular instance.

16-10

°

Wind

075

ding

Hea

Wind
correction
angle

Track

Desired course

090°

Figure 16-17. Establishing a wind correction angle that counteracts wind drift and maintains the desired course.

Basic Calculations
Before a cross-country flight, a pilot should make common
calculations for time, speed, and distance, and the amount
of fuel required.
Converting Minutes to Equivalent Hours
Frequently, it is necessary to convert minutes into equivalent
hours when solving speed, time, and distance problems. To
convert minutes to hours, divide by 60 (60 minutes = 1 hour).
Thus, 30 minutes is 30/60 = 0.5 hour. To convert hours to
minutes, multiply by 60. Thus, 0.75 hour equals 0.75 × 60
= 45 minutes.

Time T = D/GS
To find the time (T) in flight, divide the distance (D) by the
GS. The time to fly 210 NM at a GS of 140 knots is 210 ÷
140 or 1.5 hours. (The 0.5 hour multiplied by 60 minutes
equals 30 minutes.) Answer: 1:30.

Distance D = GS X T
To find the distance flown in a given time, multiply GS by
time. The distance flown in 1 hour 45 minutes at a GS of 120
knots is 120 × 1.75 or 210 NM.

GS GS = D/T
To find the GS, divide the distance flown by the time
required. If an aircraft flies 270 NM in 3 hours, the GS is
270 ÷ 3 = 90 knots.
Converting Knots to Miles Per Hour
Another conversion is that of changing knots to miles per hour
(mph). The aviation industry is using knots more frequently
than mph, but is important to understand the conversion for
those that use mph when working with speed problems. The
NWS reports both surface winds and winds aloft in knots.
However, airspeed indicators in some aircraft are calibrated
in mph (although many are now calibrated in both mph and

knots). Pilots, therefore, should learn to convert wind speeds
that are reported in knots to mph.
A knot is 1 nautical mile per hour (NMPH). Because there are
6,076.1 feet in 1 NM and 5,280 feet in 1 SM, the conversion
factor is 1.15. To convert knots to mph, multiply speed in
knots by 1.15. For example: a wind speed of 20 knots is
equivalent to 23 mph.
Most flight computers or electronic calculators have a
means of making this conversion. Another quick method of
conversion is to use the scales of NM and SM at the bottom
of aeronautical charts.
Fuel Consumption
To ensure that sufficient fuel is available for your intended
flight, you must be able to accurately compute aircraft fuel
consumption during preflight planning. Typically, fuel
consumption in gasoline-fueled aircraft is measured in
gallons per hour. Since turbine engines consume much more
fuel than reciprocating engines, turbine-powered aircraft
require much more fuel, and thus much larger fuel tanks.
When determining these large fuel quantities, using a volume
measurement such as gallons presents a problem because
the volume of fuel varies greatly in relation to temperature.
In contrast, density (weight) is less affected by temperature
and therefore, provides a more uniform and repeatable
measurement. For this reason, jet fuel is generally quantified
by its density and volume.
This standard industry convention yields a pounds-of-fuel­
per-hour value which, when divided into the nautical miles
(NM) per hour of travel (TAS ± winds) value, results in a
specific range value. The typical label for specific range is
NM per pound of fuel, or often NM per 1,000 pounds of fuel.
Preflight planning should be supported by proper monitoring
of past fuel consumption as well as use of specified fuel
management and mixture adjustment procedures in flight.

16-11

For simple aircraft with reciprocating engines, the Aircraft
Flight Manual/Pilot's Operating Handbook (AFM/POH)
supplied by the aircraft manufacturer provides gallons-per­
hour values to assist with preflight planning.
When planning a flight, you must determine how much
fuel is needed to reach your destination by calculating the
distance the aircraft can travel (with winds considered) at
a known rate of fuel consumption (gal/hr or lbs/hr) for the
expected groundspeed (GS) and ensure this amount, plus an
adequate reserve, is available on board. GS determines the
time the flight will take. The amount of fuel needed for a
given flight can be calculated by multiplying the estimated
flight time by the rate of consumption. For example, a flight
of 400 NM at 100 knots GS takes 4 hours to complete. If an
aircraft consumes 5 gallons of fuel per hour, the total fuel
consumption is 20 gallons (4 hours times 5 gallons). In this
example, there is no wind; therefore, true airspeed (TAS)
is also 100 knots, the same as GS. Since the rate of fuel
consumption remains relatively constant at a given TAS,
you must use GS to calculate fuel consumption when wind
is present. Specific range (NM/lb or NM/gal) is also useful
in calculating fuel consumption when wind is a factor.
You should always plan to be on the surface before any of
the following occur:


Your flight time exceeds the amount of flight time
you calculated for the consumption of your preflight
fuel amount



Your fuel gauge indicates low fuel level

The rate of fuel consumption depends on many factors:
condition of the engine, propeller/rotor pitch, propeller/
rotor revolutions per minute (rpm), richness of the mixture,
and the percentage of horsepower used for flight at cruising
speed. The pilot should know the approximate consumption
rate from cruise performance charts or from experience.
In addition to the amount of fuel required for the flight,
there should be sufficient fuel for reserve. When estimating
consumption you must plan for cruise flight as well as startup
and taxi, and higher fuel burn during climb. Remember that
ground speed during climb is less than during cruise flight
at the same airspeed. Additional fuel for adequate reserve
should also be added as a safety measure.
Flight Computers
Up to this point, only mathematical formulas have been used
to determine such items as time, distance, speed, and fuel
consumption. In reality, most pilots use a mechanical flight
computer called an E6B or electronic flight calculator. These
devices can compute numerous problems associated with
flight planning and navigation. The mechanical or electronic

16-12

computer has an instruction book that probably includes
sample problems so the pilot can become familiar with its
functions and operation. [Figure 16-18]
Plotter
Another aid in flight planning is a plotter, which is a protractor
and ruler. The pilot can use this when determining TC and
measuring distance. Most plotters have a ruler that measures
in both NM and SM and has a scale for a sectional chart on one
side and a world aeronautical chart on the other. [Figure 16-18]

Pilotage
Pilotage is navigation by reference to landmarks or
checkpoints. It is a method of navigation that can be used
on any course that has adequate checkpoints, but it is more
commonly used in conjunction with dead reckoning and
VFR radio navigation.
The checkpoints selected should be prominent features
common to the area of the flight. Choose checkpoints that can
be readily identified by other features, such as roads, rivers,
railroad tracks, lakes, and power lines. If possible, select
features that make useful boundaries or brackets on each
side of the course, such as highways, rivers, railroads, and
mountains. A pilot can keep from drifting too far off course
by referring to and not crossing the selected brackets. Never
place complete reliance on any single checkpoint. Choose
ample checkpoints. If one is missed, look for the next one while
maintaining the heading. When determining position from
checkpoints, remember that the scale of a sectional chart is 1
inch = 8 SM or 6.86 NM. For example, if a checkpoint selected
was approximately one-half inch from the course line on the
chart, it is 4 SM or 3.43 NM from the course on the ground.
In the more congested areas, some of the smaller features are
not included on the chart. If confused, hold the heading. If a
turn is made away from the heading, it is easy to become lost.
Roads shown on the chart are primarily the well-traveled
roads or those most apparent when viewed from the air.
New roads and structures are constantly being built and
may not be shown on the chart until the next chart is issued.
Some structures, such as antennas, may be difficult to see.
Sometimes TV antennas are grouped together in an area near
a town. They are supported by almost invisible guy wires.
Never approach an area of antennas less than 500 feet above
the tallest one. Most of the taller structures are marked with
strobe lights to make them more visible to pilots. However,
some weather conditions or background lighting may make
them difficult to see. Aeronautical charts display the best
information available at the time of printing, but a pilot should
be cautious for new structures or changes that have occurred
since the chart was printed.

0

1

0
14
32

0

30

350 3
40

33

0

60

50
23

0

0
21

15

20

25

340
350

170

10

190

NAUTICAL 5 MILES

30

10

Plotter

200

A

DEGREES

20

160

33

0

0

24
0
15
0

30

0

10
20

70

260
250
170
160

0
22

15

80

280 270
290
190 180
200

40

0
31

0
30
0
21

90

100

110

12
30

35

40

45

50

55

60

65

70

75

80 NAUTICAL 85 MILES

INSTRUCTIONS FOR USE
1.
2.
3.
4.

SECTIONAL CHART SIDE - 1:500,000
0 STATUTE 5 MILES

10

15

20

25

30

35

Place hole over intersection of true course and true north line.
Without changing position rotate plotter until edge is over true course line.
From hole follow true north line to curved scale with arrow pointing in direction of flight.
Read true course in degrees, on proper scale, over true north line. read scales counter-clockwise.

40

45

50

55

60

NAVIGATIONAL FLIGHT PLOTTER
65

70

75

80

85

90

95

100

TSD Alt: As Wind Wt. Bal Timer
Conv: Dist Vol Wt Wx

C
M
P

Mode

C
B

On/Off

Clr

Dist

Vol

Wt

Wx

÷

:

7

8

9

x

Sto

4

5

6

−

Rcl

1

2

3

+

Bksp

0

.

+/­

=

Electronic flight computer

Mechanical flight computer

Figure 16-18. A plotter (A), the computational and wind side of a mechanical flight computer (E6B) (B), and an electronic flight computer (C).

Dead Reckoning
Dead reckoning is navigation solely by means of computations
based on time, airspeed, distance, and direction. The products
derived from these variables, when adjusted by wind speed
and velocity, are heading and GS. The predicted heading
takes the aircraft along the intended path and the GS
establishes the time to arrive at each checkpoint and the
destination. Except for flights over water, dead reckoning
is usually used with pilotage for cross-country flying. The
heading and GS, as calculated, is constantly monitored and
corrected by pilotage as observed from checkpoints.

Wind Triangle or Vector Analysis
If there is no wind, the aircraft's ground track is the same as
the heading and the GS is the same as the true airspeed. This
condition rarely exists. A wind triangle, the pilot's version
of vector analysis, is the basis of dead reckoning.
The wind triangle is a graphic explanation of the effect of
wind upon flight. GS, heading, and time for any flight can be
determined by using the wind triangle. It can be applied to
the simplest kind of cross-country flight, as well as the most
complicated instrument flight. The experienced pilot becomes

16-13

so familiar with the fundamental principles that estimates can
be made that are adequate for visual flight without actually
drawing the diagrams. The beginning student, however, needs
to develop skill in constructing these diagrams as an aid to the
complete understanding of wind effect. Either consciously or
unconsciously, every good pilot thinks of the flight in terms
of wind triangle.

the blue, yellow, and black lines in Figure 16-20, which is
explained in the following example.
Suppose a flight is to be flown from E to P. Draw a line on
the aeronautical chart connecting these two points; measure
its direction with a protractor, or plotter, in reference to a
meridian. This is the TC, which in this example is assumed
to be 090° (east). From the NWS, it is learned that the wind
at the altitude of the intended flight is 40 knots from the
northeast (045°). Since the NWS reports the wind speed in
knots, if the true airspeed of the aircraft is 120 knots, there is
no need to convert speeds from knots to mph or vice versa.

If flight is to be made on a course to the east, with a wind
blowing from the northeast, the aircraft must be headed
somewhat to the north of east to counteract drift. This can
be represented by a diagram as shown in Figure 16-19. Each
line represents direction and speed. The long blue and white
hashed line shows the direction the aircraft is heading, and
its length represents the distance traveled at the indicated
airspeed for 1 hour. The short blue arrow at the right shows
the wind direction, and its length represents the wind velocity
for 1 hour. The solid yellow line shows the direction of the
track or the path of the aircraft as measured over the earth, and
its length represents the distance traveled in 1 hour or the GS.

Now, on a plain sheet of paper draw a vertical line representing
north to south. (The various steps are shown in Figure 16-21.)

Step 1
Place the protractor with the base resting on the vertical line
and the curved edge facing east. At the center point of the
base, make a dot labeled "E" (point of departure) and at the
curved edge, make a dot at 90° (indicating the direction of the
true course) and another at 45° (indicating wind direction).

In actual practice, the triangle illustrated in Figure 16-19 is
not drawn; instead, construct a similar triangle as shown by
N

eadin

080° h

peed

ots airs

20 kn
g and 1

N

33

6

10°

3

3
0

W

E

Drift Angle

Wind at 20°
direction and
35 knots velocity

090° course and 110 knots groundspeed

1
2

24
2
1

S

15

8° left correction

S

Figure 16-19. Principle of the wind triangle.

N

3

Wind direction and velocity

N

33

6

3
0

P

E

W

E
1
2

24

2
1

S

15

d

pee

ding

Hea
W

irs
nd a

a

S

Figure 16-20. The wind triangle as is drawn in navigation practice.

16-14

Course and groundspeed

10

20

10

STEP 1

N

30

40

20

60

20

50

Mid point

STEP 2 and 3

50

40
10

N

45 °

30

70

30

40

60 80

50

60

90 °

70

70

80

90

90

TC 090°

E

80

100
12
0

13

90

100

110

0

d

in

100

14

0

110

W

15

0

160

170

0

12

110

13
0

12

14

0

15

0

0

3

0

N

160

6

14

15

0

160

170

170

TC 090° GS 88
knots
d 120

1
2
S

P

E

33

24

W

13

W

2
1

S

0

3
0

E

W

15

ee

Airsp

S

STEP 4

Figure 16-21. Steps in drawing the wind triangle.

Step 2



By placing the straight side of the protractor along
the north-south line, with its center point at the
intersection of the airspeed line and north-south line,
read the TH directly in degrees (076°). [Figure 16-22]



By placing the straight side of the protractor along the
TC line, with its center at P, read the angle between
the TC and the airspeed line. This is the WCA, which
must be applied to the TC to obtain the TH. If the wind
blows from the right of TC, the angle is added; if from
the left, it is subtracted. In the example given, the
WCA is 14° and the wind is from the left; therefore,
subtract 14° from TC of 090°, making the TH 076°.
[Figure 16-23]

With the ruler, draw the true course line from E, extending it
somewhat beyond the dot by 90°, and labeling it "TC 090°."

Step 3
Next, align the ruler with E and the dot at 45°, and draw
the wind arrow from E, not toward 045°, but downwind in
the direction the wind is blowing making it 40 units long to
correspond with the wind velocity of 40 knots. Identify this
line as the wind line by placing the letter "W" at the end to
show the wind direction.

Step 4
Finally, measure 120 units on the ruler to represent the
airspeed, making a dot on the ruler at this point. The units
used may be of any convenient scale or value (such as ¼
inch = 10 knots), but once selected, the same scale must
be used for each of the linear movements involved. Then
place the ruler so that the end is on the arrowhead (W) and
the 120-knot dot intercepts the TC line. Draw the line and
label it "AS 120." The point "P" placed at the intersection
represents the position of the aircraft at the end of 1 hour.
The diagram is now complete.
The distance flown in 1 hour (GS) is measured as the numbers
of units on the TC line (88 NMPH or 88 knots). The TH
necessary to offset drift is indicated by the direction of the
airspeed line, which can be determined in one of two ways:

After obtaining the TH, apply the correction for magnetic
variation to obtain magnetic heading and the correction
for compass deviation to obtain a compass heading. The
compass heading can be used to fly to the destination by
dead reckoning.
To determine the time and fuel required for the flight, first
find the distance to your destination by measuring the length
of the course line drawn on the aeronautical chart (using the
appropriate scale at the bottom of the chart). If the distance
measures 220 NM, divide by the GS of 88 knots, which gives
2.5 hours, or 2:30, as the time required. If fuel consumption
is 8 gallons an hour, 8 × 2.5 or about 20 gallons is used.

16-15

N
10

20

30

40

TC 090° GS 88

50

E

P

60

10

20

76°

30

40

70

50

S 120

76° A

60

TH 0

70

80

80
90

90

100
110

0

100

12

W

13
0
14

0

15

0

110
12
0

160

170

S
0

13

Figure 16-22. Finding true heading by the wind correction angle.
N

TC 090° GS 88
170

P
14

40
50

0

13

60

0

100

110

12

0

90

15

80

0
14

70

14°

S 120

76° A

TH 0

0

20

15

30

0

20

160

10

WCA =14° L

10

E

0

13

70

80

90

100

110

W

12
0

S
Figure 16-23. Finding true heading by direct measurement.

Briefly summarized, the steps in obtaining flight information
are as follows:


TC—direction of the line connecting two desired
points, drawn on the chart and measured clockwise
in degrees from TN on the mid-meridian



WCA—determined from the wind triangle. (Added
to TC if the wind is from the right; subtracted if wind
is from the left)



16-16

TH—direction measured in degrees clockwise from
TN, in which the nose of the plane should point to
remain on the desired course



Variation—obtained from the isogonic line on the
chart (added to TH if west; subtracted if east)



MH—an intermediate step in the conversion (obtained
by applying variation to TH)



Deviation—obtained from the deviation card on the
aircraft (added to or subtracted from MH, as indicated)



Compass heading—reading on the compass (found by
applying deviation to MH) that is followed to remain
on the desired course



Total distance—obtained by measuring the length of
the TC line on the chart (using the scale at the bottom
of the chart)



GS—obtained by measuring the length of the TC line
on the wind triangle (using the scale employed for
drawing the diagram)



Estimated time en route (ETE)—total distance divided
by GS



Fuel rate—predetermined gallons per hour used at
cruising speed

NOTE: Additional fuel for adequate reserve should be added
as a safety measure.

Flight Planning
Title 14 of the Code of Federal Regulations (14 CFR) part
91 states, in part, that before beginning a flight, the pilot in
command (PIC) of an aircraft shall become familiar with all
available information concerning that flight. For flights not
in the vicinity of an airport, this must include information
on available current weather reports and forecasts, fuel
requirements, alternatives available if the planned flight
cannot be completed, and any known traffic delays of which
the PIC has been advised by ATC.
Assembling Necessary Material
The pilot should collect the necessary material well before
beginning the flight. An appropriate current sectional chart
and charts for areas adjoining the flight route should be among
this material if the route of flight is near the border of a chart.
Additional equipment should include a flight computer or
electronic calculator, plotter, and any other item appropriate
to the particular flight. For example, if a night flight is to
be undertaken, carry a flashlight; if a flight is over desert
country, carry a supply of water and other necessities.
Weather Check
It is wise to check the weather before continuing with other
aspects of flight planning to see, first of all, if the flight is
feasible and, if it is, which route is best. Chapter 12, "Aviation
Weather Services," discusses obtaining a weather briefing.
Use of Chart Supplement U.S. (formerly Airport/
Facility Directory)
Study available information about each airport at which a
landing is intended. This should include a study of the Notices
to Airmen (NOTAMs) and the Chart Supplement U.S.
(formerly Airport/Facility Directory). [Figure 16-24] This
includes location, elevation, runway and lighting facilities,
available services, availability of aeronautical advisory
station frequency (UNICOM), types of fuel available (use to

Figure 16-24. Chart Supplement U.S. (formerly Airport/Facility

Directory).

decide on refueling stops), FSS located on the airport, control
tower and ground control frequencies, traffic information,
remarks, and other pertinent information. The NOTAMs,
issued every 28 days, should be checked for additional
information on hazardous conditions or changes that have
been made since issuance of the Chart Supplement U.S.
The sectional chart bulletin subsection should be checked for
major changes that have occurred since the last publication date
of each sectional chart being used. Remember, the chart may
be up to 6 months old. The effective date of the chart appears
at the top of the front of the chart. The Chart Supplement U.S.
generally has the latest information pertaining to such matters
and should be used in preference to the information on the
back of the chart, if there are differences.
Airplane Flight Manual or Pilot's Operating
Handbook (AFM/POH)
The Aircraft Flight Manual or Pilot's Operating Handbook
(AFM/POH) should be checked to determine the proper
loading of the aircraft (weight and balance data). The weight
of the usable fuel and drainable oil aboard must be known.
Also, check the weight of the passengers, the weight of all
baggage to be carried, and the empty weight of the aircraft to
be sure that the total weight does not exceed the maximum
allowable weight. The distribution of the load must be known
to tell if the resulting center of gravity (CG) is within limits.
16-17

Be sure to use the latest weight and balance information in
the FAA-approved AFM or other permanent aircraft records,
as appropriate, to obtain empty weight and empty weight
CG information.
Determine the takeoff and landing distances from the
appropriate charts, based on the calculated load, elevation
of the airport, and temperature; then compare these distances
with the amount of runway available. Remember, the
heavier the load and the higher the elevation, temperature,
or humidity, the longer the takeoff roll and landing roll and
the lower the rate of climb.
Check the fuel consumption charts to determine the rate of
fuel consumption at the estimated flight altitude and power
settings. Calculate the rate of fuel consumption, and compare
it with the estimated time for the flight so that refueling points
along the route can be included in the plan.

Charting the Course
Once the weather has been checked and some preliminary
planning completed, it is time to chart the course and
determine the data needed to accomplish the flight. The
following sections provide a logical sequence to follow in
charting the course, complete a flight log, and filing a flight
plan. In the following example, a trip is planned based on the
following data and the sectional chart excerpt in Figure 16-25.
Route of flight: Chickasha Airport direct to Guthrie Airport
True airspeed (TAS)........................................115 knots

Winds aloft...........................................360° at 10 knots

Usable fuel.....................................................38 gallons

Fuel rate...............................................................8 GPH

Deviation..................................................................+2°

Steps in Charting the Course
The following is a suggested sequence for arriving at the
pertinent information for the trip. As information is determined,
it may be noted as illustrated in the example of a flight log in
Figure 16-26. Where calculations are required, the pilot may
use a mathematical formula or a manual or electronic flight
computer. If unfamiliar with the use of a manual or electronic
computer, it would be advantageous to read the operation
manual and work several practice problems at this point.
First, draw a line from Chickasha Airport (point A) directly
to Guthrie Airport (point F). The course line should begin at
the center of the airport of departure and end at the center of
the destination airport. If the route is direct, the course line
consists of a single straight line. If the route is not direct, it
consists of two or more straight line segments. For example, a
16-18

VOR station that is off the direct route, but makes navigating
easier, may be chosen (radio navigation is discussed later in
this chapter).
Appropriate checkpoints should be selected along the route
and noted in some way. These should be easy-to-locate
points, such as large towns, large lakes and rivers, or
combinations of recognizable points, such as towns with
an airport, towns with a network of highways, and railroads
entering and departing.
Normally, choose only towns indicated by splashes of yellow
on the chart. Do not choose towns represented by a small
circle—these may turn out to be only a half-dozen houses. (In
isolated areas, however, towns represented by a small circle
can be prominent checkpoints.) For this trip, four checkpoints
have been selected. Checkpoint 1 consists of a tower located
east of the course and can be further identified by the highway
and railroad track, which almost parallels the course at this
point. Checkpoint 2 is the obstruction just to the west of the
course and can be further identified by Will Rogers World
Airport, which is directly to the east. Checkpoint 3 is Wiley
Post Airport, which the aircraft should fly directly over.
Checkpoint 4 is a private, non-surfaced airport to the west of
the course and can be further identified by the railroad track
and highway to the east of the course.
The course and areas on either side of the planned route
should be checked to determine if there is any type of airspace
with which the pilot should be concerned or which has
special operational requirements. For this trip, it should be
noted that the course passes through a segment of the Class
C airspace surrounding Will Rogers World Airport where the
floor of the airspace is 2,500 feet mean sea level (MSL) and
the ceiling is 5,300 feet MSL (point B). Also, there is Class
D airspace from the surface to 3,800 feet MSL surrounding
Wiley Post Airport (point C) during the time the control
tower is in operation.
Study the terrain and obstructions along the route. This is
necessary to determine the highest and lowest elevations,
as well as the highest obstruction to be encountered so
an appropriate altitude that conforms to 14 CFR part 91
regulations can be selected. If the flight is to be flown
at an altitude of more than 3,000 feet above the terrain,
conformance to the cruising altitude appropriate to the
direction of flight is required. Check the route for particularly
rugged terrain so it can be avoided. Areas where a takeoff
or landing is made should be carefully checked for tall
obstructions. Television transmitting towers may extend to
altitudes over 1,500 feet above the surrounding terrain. It is
essential that pilots be aware of their presence and location.
For this trip, it should be noted that the tallest obstruction is

F
4

Checkpoint R

Highest elevation

D
3

Class D Airspace

Checkpoint

B
1

Tallest obstruction

Checkpoint

C
2

E

Class C Airspace

Checkpoint

Course line

A

Route of flight: Chickasha Airport direct to Guthrie Airport
True airspeed (TAS) . . . . . . . . . . . . . . . . . . . . 115 knots
Winds aloft . . . . . . . . . . . . . . . . . . . . . . . 360° at 10 knots
Usable fuel . . . . . . . . . . . . . . . . . . . . . . . . . . . 38 gallons
Fuel rate . . . . . . . . . . . . . . . . . . . . . . . . . . . . . . . .8 GPH
Deviation . . . . . . . . . . . . . . . . . . . . . . . . . . . . . . . . . . +2°

Figure 16-25. Sectional chart excerpt.

16-19

PILOT'S PLANNING SHEET
PLANE IDENTIFICATION N123DB
COURSE

TC

From Chickasha
To Guthrie

031°

WIND
Knots From

10 360°

DATE

ALTITUDE WCA TH MAG VAR MH DEV CH TOTAL
R+ L­
W+ E­
MILES
8000

3° L 28

7° E

21° +2°

23

53

GS

TOTAL
TIME

FUEL
RATE

TOTAL
FUEL

106 kts

35 min

8 GPH

38 gal

From
To

VISUAL FLIGHT LOG
TIME OF
DEPARTURE

NAVIGATION
AIDS

POINT OF
DEPARTURE
Chickasha Airport

NAVAID
IDENT.
FREQ.

COURSE ALTITUDE
TO

TO
FROM

FROM

CHECKPOINT
\#1

8000

CHECKPOINT
\#2

8000

CHECKPOINT
\#3

8000

CHECKPOINT
\#4

8000

DISTANCE
NT

POI

NT
POI
E
TO
TIV
ULA
M
CU

11 NM
10000
10 NM
10000

TED
IMA
UAL
ACT

EST

GS

CH

ED
ED
AT AL
AT AL
TIM TU
TIM TU
S
S
C
C
E A
E A

6 min
+5

106 kts

023°

6 min

106 kts

023°

6 min

106 kts

023°

7 min

106 kts

023°

REMARKS
WEATHER
AIRSPACE ETC.

21 NM
10.5 NM

10000

31.5 NM
13 NM

10000

DESTINATION

ELAPSED TIME

44.5 NM

8.5 NM

Guthrie Airport

5 min
53 NM

Figure 16-26. Pilot's planning sheet and visual flight log.

part of a series of antennas with a height of 2,749 feet MSL
(point D). The highest elevation should be located in the
northeast quadrant and is 2,900 feet MSL (point E).

The formula is:

Since the wind is no factor and it is desirable and within the
aircraft's capability to fly above the Class C and D airspace
to be encountered, an altitude of 5,500 feet MSL is chosen.
This altitude also gives adequate clearance of all obstructions,
as well as conforms to the 14 CFR part 91 requirement to
fly at an altitude of odd thousand plus 500 feet when on a
magnetic course between 0 and 179°.

The WCA can be determined by using a manual or electronic
flight computer. Using a wind of 360° at 10 knots, it is
determined the WCA is 3° left. This is subtracted from the
TC making the TH 28°. Next, the pilot should locate the
isogonic line closest to the route of the flight to determine
variation. Figure 16-25 shows the variation to be 6.30° E
(rounded to 7° E), which means it should be subtracted from
the TH, giving an MH of 21°. Next, add 2° to the MH for
the deviation correction. This gives the pilot the compass
heading of 23°.

Next, the pilot should measure the total distance of the
course, as well as the distance between checkpoints. The total
distance is 53 NM, and the distance between checkpoints is
as noted on the flight log in Figure 16-26.
After determining the distance, the TC should be measured.
If using a plotter, follow the directions on the plotter. The TC
is 031°. Once the TH is established, the pilot can determine
the compass heading. This is done by following the formula
given earlier in this chapter.

16-20

TC ± WCA = TH ± V = MH ± D = CH

Now, the GS can be determined. This is done using a manual
or electronic calculator. The GS is determined to be 106
knots. Based on this information, the total trip time, as well
as time between checkpoints, and the fuel burned can be
determined. These numbers can be calculated by using a
manual or electronic calculator.

met, but actual time cannot be given because of inadequate
communication. The FSS specialist who accepts the flight
plan does not inform the pilot of this procedure, however.

For this trip, the GS is 106 knots and the total time is 35
minutes (30 minutes plus 5 minutes for climb) with a fuel
burn of 4.7 gallons. Refer to the flight log in Figure 16-26
for the time between checkpoints.
As the trip progresses, the pilot can note headings and time
and make adjustments in heading, GS, and time.

Filing a VFR Flight Plan
Filing a flight plan is not required by regulations; however, it
is a good operating practice since the information contained
in the flight plan can be used in search and rescue in the
event of an emergency.
Flight plans can be filed in the air by radio, but it is best to
file a flight plan by phone just before departing. After takeoff,
contact the FSS by radio and give them the takeoff time so
the flight plan can be activated.
When a VFR flight plan is filed, it is held by the FSS until
1 hour after the proposed departure time and then canceled
unless: the actual departure time is received; a revised
proposed departure time is received; or at the time of filing,
the FSS is informed that the proposed departure time is

X

N123DB

C150/X

115

Figure 16-27 shows the flight plan form a pilot files with the
FSS. When filing a flight plan by telephone or radio, give
the information in the order of the numbered spaces. This
enables the FSS specialist to copy the information more
efficiently. Most of the fields are either self-explanatory
or non-applicable to the VFR flight plan (such as item 13).
However, some fields may need explanation.


Item 3 is the aircraft type and special equipment. An
example would be C-150/X, which means the aircraft
has no transponder. A listing of special equipment
codes is found in the Aeronautical Information Manual
(AIM).



Item 6 is the proposed departure time in UTC
(indicated by the "Z").



Item 7 is the cruising altitude. Normally, "VFR" can be
entered in this block since the pilot chooses a cruising
altitude to conform to FAA regulations.

CHK, CHICKASHA
AIRPORT

1400

5500

Chickasha direct Guthrie

GOK, Guthrie Airport
Guthrie, OK

4

45

35

Jane Smith
Aero Air, Oklahoma City, OK (405) 555-4149

1

Red/White
McAlester

Figure 16-27. Domestic flight plan form.

16-21



Item 8 is the route of flight. If the flight is to be direct,
enter the word "direct;" if not, enter the actual route to
be followed, such as via certain towns or navigation
aids.



Item 10 is the estimated time en route. In the sample
flight plan, 5 minutes was added to the total time to
allow for the climb.



Item 12 is the fuel on board in hours and minutes. This
is determined by dividing the total usable fuel aboard
in gallons by the estimated rate of fuel consumption
in gallons.

Remember, there is every advantage in filing a flight plan;
but do not forget to close the flight plan upon arrival. This
should be done via telephone to avoid radio congestion.

Ground-Based Navigation
Advances in navigational radio receivers installed in aircraft,
the development of aeronautical charts that show the exact
location of ground transmitting stations and their frequencies,
along with refined flight deck instrumentation make it
possible for pilots to navigate with precision to almost any
point desired. Although precision in navigation is obtainable
through the proper use of this equipment, beginning pilots
should use this equipment to supplement navigation by visual
reference to the ground (pilotage). This method provides the
pilot with an effective safeguard against disorientation in the
event of radio malfunction.

The prefix "omni-" means all, and an omnidirectional range
is a VHF radio transmitting ground station that projects
straight line courses (radials) from the station in all directions.
From a top view, it can be visualized as being similar to the
spokes from the hub of a wheel. The distance VOR radials are
projected depends upon the power output of the transmitter.
The course or radials projected from the station are referenced
to MN. Therefore, a radial is defined as a line of magnetic
bearing extending outward from the VOR station. Radials are
identified by numbers beginning with 001, which is 1° east
of MN and progress in sequence through all the degrees of a
circle until reaching 360. To aid in orientation, a compass rose
reference to magnetic north is superimposed on aeronautical
charts at the station location.
VOR ground stations transmit within a VHF frequency band
of 108.0–117.95 MHz. Because the equipment is VHF, the
signals transmitted are subject to line-of-sight restrictions.
Therefore, its range varies in direct proportion to the altitude
of receiving equipment. Generally, the reception range of
the signals at an altitude of 1,000 feet above ground level
(AGL) is about 40 to 45 miles. This distance increases with
altitude. [Figure 16-28]
"A" and "B" signal
received
Only "A" signal received

There are three radio navigation systems available for use
for VFR navigation. These are:


VHF Omnidirectional Range (VOR)



Nondirectional Radio Beacon (NDB)



Global Positioning System (GPS)

Very High Frequency (VHF) Omnidirectional
Range (VOR)
The VOR system is present in three slightly different
navigation aids (NAVAIDs): VOR, VOR/distance measuring
equipment (DME)(discussed in a later section), and
VORTAC. By itself it is known as a VOR, and it provides
magnetic bearing information to and from the station. When
DME is also installed with a VOR, the NAVAID is referred
to as a VOR/DME. When military tactical air navigation
(TACAN) equipment is installed with a VOR, the NAVAID
is known as a VORTAC. DME is always an integral part of
a VORTAC. Regardless of the type of NAVAID utilized
(VOR, VOR/DME, or VORTAC), the VOR indicator
behaves the same. Unless otherwise noted in this section,
VOR, VOR/DME, and VORTAC NAVAIDs are all referred
to hereafter as VORs.
16-22

Only "B" signal received

Neither "A" nor "B"
signal received

VOR
station
"A"

VOR
station
"B"

Figure 16-28. VHF transmissions follow a line-of-sight course.

VORs and VORTACs are classed according to operational
use. There are three classes:


T (Terminal)



L (Low altitude)



H (High altitude)

The normal useful range for the various classes is shown in
the following table:
VOR/VORTAC NAVAIDS
Normal Usable Altitudes and Radius Distances

Class
T
L
H
H
H
H

Distance
Altitudes
(Miles)
12,000' and below
25
Below 18,000'
40
Below 14,500'
40
Within the conterminous 48 states
only, between 14,500 and 17,999' 100
18,000'—FL 450
130
FL 450—60,000'
100

The useful range of certain facilities may be less than 50
miles. For further information concerning these restrictions,
refer to the Communication/NAVAID Remarks in the Chart
Supplement U.S.
The accuracy of course alignment of VOR radials is
considered to be excellent. It is generally within plus or minus
1°. However, certain parts of the VOR receiver equipment
deteriorate, affecting its accuracy. This is particularly true
at great distances from the VOR station. The best assurance
of maintaining an accurate VOR receiver is periodic checks
and calibrations. VOR accuracy checks are not a regulatory
requirement for VFR flight. However, to assure accuracy of
the equipment, these checks should be accomplished quite
frequently and a complete calibration should be performed
each year. The following means are provided for pilots to
check VOR accuracy:


FAA VOR test facility (VOT)



Certified airborne checkpoints



Certified ground checkpoints located on airport
surfaces

If an aircraft has two VOR receivers installed, a dual VOR
receiver check can be made. To accomplish the dual receiver
check, a pilot must tune both VOR receivers to the same VOR
ground facility. The maximum permissible variation between
the two indicated bearings is 4°. A list of the airborne and
ground checkpoints is published in the Chart Supplement U.S.

Basically, these checks consist of verifying that the VOR
radials the aircraft equipment receives are aligned with the
radials the station transmits. There are not specific tolerances
in VOR checks required for VFR flight. But as a guide to
assure acceptable accuracy, the required IFR tolerances can
be used—±4° for ground checks and ±6° for airborne checks.
These checks can be performed by the pilot.
The VOR transmitting station can be positively identified
by its Morse code identification or by a recorded voice
identification that states the name of the station followed by
"VOR." Many FSSs transmit voice messages on the same
frequency that the VOR operates. Voice transmissions should
not be relied upon to identify stations because many FSSs
remotely transmit over several omniranges that have names
different from that of the transmitting FSS. If the VOR is
out of service for maintenance, the coded identification is
removed and not transmitted. This serves to alert pilots that
this station should not be used for navigation. VOR receivers
are designed with an alarm flag to indicate when signal
strength is inadequate to operate the navigational equipment.
This happens if the aircraft is too far from the VOR or the
aircraft is too low and, therefore, is out of the line of sight
of the transmitting signals.

Using the VOR
In review, for VOR radio navigation, there are two
components required: ground transmitter and aircraft
receiving equipment. The ground transmitter is located at a
specific position on the ground and transmits on an assigned
frequency. The aircraft equipment includes a receiver with
a tuning device and a VOR or omninavigation instrument.
The navigation instrument could be a course deviation
indicator (CDI), horizontal situation indicator (HSI), or a
radio magnetic indicator (RMI). Each of these instruments
indicates the course to the tuned VOR.
Course Deviation Indicator (CDI)
The CDI is found in most training aircraft. It consists of an
omnibearing selector (OBS) sometimes referred to as the
course selector, a CDI needle (left-right needle), and a TO/
FROM indicator.
The course selector is an azimuth dial that can be rotated to
select a desired radial or to determine the radial over which
the aircraft is flying. In addition, the magnetic course "TO"
or "FROM" the station can be determined.
When the course selector is rotated, it moves the CDI or
needle to indicate the position of the radial relative to the
aircraft. If the course selector is rotated until the deviation

16-23

The desired course is selected by rotating the course select
pointer, in relation to the compass card, by means of the

Course index

Unreliable signal flag

3

TO

E

W

CDI needle

12

Approximately 2 degrees
in the VOR mode

24

TO/FROM indicator

21

Figure 16-29. VOR indicator.

16-24

S

OBS knob

15

OBS

6

N
A
V

30

33

N

Compass card

Compass warning flag
Lubber line

NAV warning flag

DC

I5

GS

HDG

I2

NAV

2I

GS

6

30

In Figure 16-30, the aircraft magnetic heading displayed
on the compass card under the lubber line is 184°. The
course select pointer shown is set to 295°; the tail of the
pointer indicates the reciprocal, 115°. The course deviation
bar operates with a VOR/Localizer (VOR/LOC) or GPS
navigation receiver to indicate left or right deviations from
the course selected with the course select pointer; operating
in the same manner, the angular movement of a conventional
VOR/LOC needle indicates deviation from course.

Course deviation scale

24

Horizontal Situation Indicator
The HSI is a direction indicator that uses the output
from a flux valve to drive the compass card. The HSI
[Figure 16-30] combines the magnetic compass with
navigation signals and a glideslope. The HSI gives the pilot
an indication of the location of the aircraft in relation to the
chosen course or radial.

To/From indicator

3

By centering the needle, the course selector indicates either
the course "FROM" the station or the course "TO" the station.
If the flag displays a "TO," the course shown on the course
selector must be flown to the station. [Figure 16-29] If
"FROM" is displayed and the course shown is followed, the
aircraft is flown away from the station.

Glideslope deviation scale

33

needle is centered, the radial (magnetic course "FROM" the
station) or its reciprocal (magnetic course "TO" the station)
can be determined. The course deviation needle also moves
to the right or left if the aircraft is flown or drifting away
from the radial which is set in the course selector.

Course select knob
Symbolic aircraft

Course deviation bar

Heading select bug

Course select pointer
Heading select knob

Figure 16-30. Horizontal situation indicator.

course select knob. The HSI has a fixed aircraft symbol
and the course deviation bar displays the aircraft's position
relative to the selected course. The TO/FROM indicator is a
triangular pointer. When the indicator points to the head of the
course select pointer, the arrow shows the course selected. If
properly intercepted and flown, the course takes the aircraft
to the chosen facility. When the indicator points to the tail
of the course, the arrow shows that the course selected, if
properly intercepted and flown, takes the aircraft directly
away from the chosen facility.
When the NAV warning flag appears, it indicates no reliable
signal is being received. The appearance of the HDG flag
indicates the compass card is not functioning properly.
Radio Magnetic Indicator (RMI)
The RMI is a navigational aid providing aircraft magnetic
or directional gyro heading and very high frequency
omnidirectional range (VOR), GPS, and automatic direction

7

N

3

33

30

W
24

30
24
30

30 33

24

15

33

30

6

27

30

W
24

21

S

24

FROM

15

I2
OBS

E

33

N

3

2I

I5
N

12

6

18

21

12

W

2I 24
E

3

6

N

S

2I

I5

I2 I5

3

0
33

6

12

6

21

6


E
3

30

TO

6

W

5

24

E
12
15

21

S

OBS

N

33

3

W

E

2I

I5

24

24

I2

33

3

12
15

21

S
N

33

3

30

TO

W

4

6

6

30

OBS

WIND

6

30

30

3

33

TO

6

E

24

24
2I

15

21

S

3

I5

OBS

I2

12

3

30 3
3

6

I2

I5

2I 2
4
N

33

2

3
TO

30

BRAVO

BRA
115.0


W

6

24

E
12
15

N

33

3

33

30

TO

3

30

21

S

OBS

6

I5

2I

24

I2

12
15

S

21

OBS

1

24

W

6

E

First, tune the VOR receiver to the frequency of the selected
VOR station. For example, 115.0 to receive Bravo VOR.
Next, check the identifiers to verify that the desired VOR
is being received. As soon as the VOR is properly tuned,
the course deviation needle deflects either left or right.
Then, rotate the azimuth dial to the course selector until the
course deviation needle centers and the TO-FROM indicator
indicates "TO." If the needle centers with a "FROM"
indication, the azimuth should be rotated 180° because, in
this case, it is desired to fly "TO" the station. Now, turn the
aircraft to the heading indicated on the VOR azimuth dial or
course selector, 350° in this example.

S

3

6

Tracking With VOR
The following describes a step-by-step procedure for tracking
to and from a VOR station using a CDI. Figure 16-32
illustrates the procedure.

15

8

3

FROM

OBS

33

finder (ADF) bearing information. [Figure 16-31] Remote
indicating compasses were developed to compensate for
errors in and limitations of older types of heading indicators.
The remote compass transmitter is a separate unit usually
mounted in a wingtip to eliminate the possibility of magnetic
interference. The RMI consists of a compass card, a heading
index, two bearing pointers, and pointer function switches.
The two pointers are driven by any two combinations of a
GPS, an ADF, and/or a VOR. The pilot has the ability to select
the navigation aid to be indicated. The pointer indicates the
course to the selected NAVAID or waypoint. In Figure 16-31,
the green pointer is indicating the station tuned on the ADF.
The yellow pointer is indicating the course to a VOR or GPS
waypoint. Note that there is no requirement for a pilot to
select a course with the RMI. Only the selected navigation
source is pointed to by the needle(s).

2I

I2
N

33

NAV


Figure 16-31. Radio magnetic indicator.

FROM

I5

6

30

A

D

F


OBS

3

12

33

W

NAV

I2

E

21

24

A

D

F


6

6

S

HDG 15

3

24

30

33

Figure 16-32. Tracking a radial in a crosswind.

If a heading of 350° is maintained with a wind from the right
as shown, the aircraft drifts to the left of the intended track.
As the aircraft drifts off course, the VOR course deviation
needle gradually moves to the right of center or indicates the
direction of the desired radial or track.
16-25

To return to the desired radial, the aircraft heading must be
altered to the right. As the aircraft returns to the desired track,
the deviation needle slowly returns to center. When centered,
the aircraft is on the desired radial and a left turn must be
made toward, but not to the original heading of 350° because
a wind drift correction must be established. The amount of
correction depends upon the strength of the wind. If the wind
velocity is unknown, a trial-and-error method can be used
to find the correct heading. Assume, for this example, a 10°
correction for a heading of 360° is maintained.

maintain the course. If the aircraft drifts, fly a heading
to re-intercept the course then apply a correction to
compensate for wind drift.


If minor needle fluctuations occur, avoid changing
headings immediately. Wait a moment to see if the
needle recenters; if it does not, then you must correctly
recenter the course to the needle.



When flying "TO" a station, always fly the selected
course with a "TO" indication. When flying "FROM" a
station, always fly the selected course with a "FROM"
indication. If this is not done, the action of the course
deviation needle is reversed. To further explain this
reverse action, if the aircraft is flown toward a station
with a "FROM" indication or away from a station
with a "TO" indication, the course deviation needle
indicates in a direction opposite to that which it should
indicate. For example, if the aircraft drifts to the right
of a radial being flown, the needle moves to the right
or points away from the radial. If the aircraft drifts to
the left of the radial being flown, the needle moves
left or in the direction opposite of the radial.



When navigating using the VOR, it is important to
fly headings that maintain or re-intercept the course.
Just turning toward the needle will cause overshooting
the radial and flying an S turn to the left and right of
course.

While maintaining a heading of 360°, assume that the course
deviation begins to move to the left. This means that the wind
correction of 10° is too great and the aircraft is flying to the
right of course. A slight turn to the left should be made to
permit the aircraft to return to the desired radial.
When the deviation needle centers, a small wind drift
correction of 5° or a heading correction of 355° should be
flown. If this correction is adequate, the aircraft remains
on the radial. If not, small variations in heading should be
made to keep the needle centered and consequently keep the
aircraft on the radial.
As the VOR station is passed, the course deviation needle
fluctuates, then settles down, and the "TO" indication
changes to "FROM." If the aircraft passes to one side of the
station, the needle deflects in the direction of the station as
the indicator changes to "FROM."
Generally, the same techniques apply when tracking
outbound as those used for tracking inbound. If the intent is
to fly over the station and track outbound on the reciprocal of
the inbound radial, the course selector should not be changed.
Corrections are made in the same manner to keep the needle
centered. The only difference is that the omnidirectional
range indicator indicates "FROM."
If tracking outbound on a course other than the reciprocal of
the inbound radial, this new course or radial must be set in
the course selector and a turn made to intercept this course.
After this course is reached, tracking procedures are the same
as previously discussed.
Tips on Using the VOR


Positively identify the station by its code or voice
identification.



Remember that VOR signals are "line-of-sight." A
weak signal or no signal at all is received if the aircraft
is too low or too far from the station.



16-26

When navigating to a station, determine the inbound
radial and use this radial. Fly a heading that will

Time and Distance Check From a Station Using a
RMI
To compute time and distance from a station, first turn the
aircraft to place the RMI bearing pointer on the nearest 90°
index. Note the time and maintain the heading. When the
RMI bearing pointer has moved 10°, note the elapsed time
in seconds and apply the formulas in the following example
to determine the approximate time and distance from a given
station. [Figure 16-33]
The time from station may also be calculated by using a short
method based on the above formula, if a 10° bearing change
is flown. If the elapsed time for the bearing change is noted
Time-Distance Check Example
Time in seconds between bearings
Degrees of bearing change

=

Minutes to station

For example, if 2 minutes (120 seconds) is required to fly a
bearing change of 10 degrees, the aircraft is—
120
10

= 12 minutes to the station

Figure 16-33. Time-distance check example.

in seconds and a 10° bearing change is made, the time from
the station, in minutes, is determined by counting off one
decimal point. Thus, if 75 seconds are required to fly a 10°
bearing change, the aircraft is 7.5 minutes from the station.
When the RMI bearing pointer is moving rapidly or when
several corrections are required to place the pointer on the
wingtip position, the aircraft is at station passage.
The distance from the station is computed by multiplying TAS
or GS (in miles per minute) by the previously determined time
in minutes. For example, if the aircraft is 7.5 minutes from
station, flying at a TAS of 120 knots or 2 NM per minute,
the distance from station is 15 NM (7.5 × 2 = 15).
The accuracy of time and distance checks is governed by
existing wind, degree of bearing change, and accuracy of
timing. The number of variables involved causes the result
to be only an approximation. However, by flying an accurate
heading and checking the time and bearing closely, the pilot
can make a reasonable estimate of time and distance from
the station.
Time and Distance Check From a Station Using a
CDI
To compute time and distance from a station using a CDI,
first tune and identify the VOR station and determine the
radial on which you are located. Then turn inbound and
re-center the needle if necessary. Turn 90° left or right, of
the inbound course, rotating the OBS to the nearest 10°
increment opposite the direction of turn. Maintain heading
and when the CDI centers, note the time. Maintaining the
same heading, rotate the OBS 10° in the same direction as
was done previously and note the elapsed time when the
CDI again centers. Time and distance from the station is
determined from the formula shown in Figure 16-34.
Course Intercept
Course interceptions are performed in most phases of
instrument navigation. The equipment used varies, but an
intercept heading must be flown that results in an angle or
rate of intercept sufficient for solving a particular problem.
Time-Distance Check Formula
60 x minutes flown between bearing change
A Time to =
station

degrees of bearing change

B Distance to station =

TAS x minutes flown
degrees of bearing change

Figure 16-34. Time-distance check formula using a CDI.

Rate of Intercept
Rate of intercept, seen by the aviator as bearing pointer or
HSI movement, is a result of the following factors:


The angle at which the aircraft is flown toward a
desired course (angle of intercept)



True airspeed and wind (GS)



Distance from the station

Angle of Intercept
The angle of intercept is the angle between the heading
of the aircraft (intercept heading) and the desired course.
Controlling this angle by selection/adjustment of the intercept
heading is the easiest and most effective way to control
course interceptions. Angle of intercept must be greater than
the degrees from course, but should not exceed 90°. Within
this limit, make adjustments as needed, to achieve the most
desirable rate of intercept.
When selecting an intercept heading, the key factor is the
relationship between distance from the station and degrees
from the course. Each degree, or radial, is 1 NM wide at
a distance of 60 NM from the station. Width increases or
decreases in proportion to the 60 NM distance. For example,
1 degree is 2 NM wide at 120 NM—and ½ NM wide at 30
NM. For a given GS and angle of intercept, the resultant rate
of intercept varies according to the distance from the station.
When selecting an intercept heading to form an angle of
intercept, consider the following factors:


Degrees from course



Distance from the station



True airspeed and wind (GS)

Distance Measuring Equipment (DME)
Distance measuring equipment (DME) consists of an ultra
high frequency (UHF) navigational aid with VOR/DMEs and
VORTACs. It measures, in NM, the slant range distance of
an aircraft from a VOR/DME or VORTAC (both hereafter
referred to as a VORTAC). Although DME equipment is
very popular, not all aircraft are DME equipped.
To utilize DME, the pilot should select, tune, and identify
a VORTAC, as previously described. The DME receiver,
utilizing what is called a "paired frequency" concept,
automatically selects and tunes the UHF DME frequency
associated with the VHF VORTAC frequency selected by
the pilot. This process is entirely transparent to the pilot.
After a brief pause, the DME display shows the slant range
distance to or from the VORTAC. Slant range distance is the
direct distance between the aircraft and the VORTAC and
is therefore affected by aircraft altitude. (Station passage
directly over a VORTAC from an altitude of 6,076 feet AGL
16-27

would show approximately 1.0 NM on the DME.) DME is a
very useful adjunct to VOR navigation. A VOR radial alone
merely gives line of position information. With DME, a pilot
may precisely locate the aircraft on a given line (radial).
Most DME receivers also provide GS and time-to-station
modes of operation. The GS is displayed in knots (NMPH).
The time-to-station mode displays the minutes remaining to
VORTAC station passage, predicated upon the present GS.
GS and time-to-station information is only accurate when
tracking directly to or from a VORTAC. DME receivers
typically need a minute or two of stabilized flight directly
to or from a VORTAC before displaying accurate GS or
time-to-station information.
Some DME installations have a hold feature that permits a
DME signal to be retained from one VORTAC while the
course indicator displays course deviation information from
an ILS or another VORTAC.
VOR/DME RNAV
Area navigation (RNAV) permits electronic course guidance
on any direct route between points established by the pilot.
While RNAV is a generic term that applies to a variety
of NAVAIDS, such as GPS and others, this section deals
with VOR/DME-based RNAV. VOR/DME RNAV is not a
separate ground-based NAVAID, but a method of navigation
using VOR/DME and VORTAC signals specially processed
by the aircraft's RNAV computer. [Figure 16-35]
NOTE: In this section, the term "VORTAC" also includes
VOR/DME NAVAIDs.
In its simplest form, VOR/DME RNAV allows the pilot to
electronically move VORTACs around to more convenient
locations. Once electronically relocated, they are referred
to as waypoints. These waypoints are described as a
combination of a selected radial and distance within the
service volume of the VORTAC to be used. These waypoints
allow a straight course to be flown between almost any
origin and destination, without regard to the orientation of
VORTACs or the existence of airways.

While the capabilities and methods of operation of VOR/
DME RNAV units differ, there are basic principles of
operation that are common to all. Pilots are urged to study
the manufacturer's operating guide and receive instruction
prior to the use of VOR/DME RNAV or any unfamiliar
navigational system. Operational information and limitations
should also be sought from placards and the supplement
section of the AFM/POH.
VOR/DME-based RNAV units operate in at least three
modes: VOR, en route, and approach. A fourth mode, VOR
Parallel, may also be found on some models. The units need
both VOR and DME signals to operate in any RNAV mode.
If the NAVAID selected is a VOR without DME, RNAV
mode will not function.
In the VOR (or non-RNAV) mode, the unit simply functions
as a VOR receiver with DME capability. [Figure 16-36] The
unit's display on the VOR indicator is conventional in all
respects. For operation on established airways or any other
ordinary VOR navigation, the VOR mode is used.
To utilize the unit's RNAV capability, the pilot selects and
establishes a waypoint or a series of waypoints to define a
course. A VORTAC (or VOR/DME) needs to be selected
as a NAVAID, since both radial and distance signals are
available from these stations. To establish a waypoint, a
point somewhere within the service range of a VORTAC is
defined on the basis of radial and distance. Once the waypoint
is entered into the unit and the RNAV en route mode is
selected, the CDI displays course guidance to the waypoint,
not the original VORTAC. DME also displays distance to
the waypoint. Many units have the capability to store several
waypoints, allowing them to be programmed prior to flight,
if desired, and called up in flight.
RNAV waypoints are entered into the unit in magnetic
bearings (radials) of degrees and tenths (i.e., 275.5°) and
distances in NM and tenths (i.e., 25.2 NM). When plotting
RNAV waypoints on an aeronautical chart, pilots find it

OFFSET

R
N
A
V

3

BEARING

3 3 4

4
8

Area Navigation Direct Route

16-28

Figure 16-36. RNAV controls.

EN ROUTE

DISTANCE

LOAD

12

Figure 16-35. Flying an RNAV course.

3

4
8
12

2 7 9

difficult to measure to that level of accuracy, and in practical
application, it is rarely necessary. A number of flight planning
publications publish airport coordinates and waypoints with
this precision and the unit accepts those figures. There is a
subtle but important difference in CDI operation and display
in the RNAV modes.
In the RNAV modes, course deviation is displayed in terms
of linear deviation. In the RNAV en route mode, maximum
deflection of the CDI typically represents 5 NM on either
side of the selected course without regard to distance from the
waypoint. In the RNAV approach mode, maximum deflection
of the CDI typically represents 1¼ NM on either side of the
selected course. There is no increase in CDI sensitivity as the
aircraft approaches a waypoint in RNAV mode.
The RNAV approach mode is used for instrument approaches.
Its narrow scale width (¼ of the en route mode) permits very
precise tracking to or from the selected waypoint. In VFR
cross-country navigation, tracking a course in the approach
mode is not desirable because it requires a great deal of
attention and soon becomes tedious.
A fourth, lesser-used mode on some units is the VOR
Parallel mode. This permits the CDI to display linear
(not angular) deviation as the aircraft tracks to and from
VORTACs. It derives its name from permitting the pilot
to offset (or parallel) a selected course or airway at a fixed
distance of the pilot's choosing, if desired. The VOR parallel
mode has the same effect as placing a waypoint directly
over an existing VORTAC. Some pilots select the VOR
parallel mode when utilizing the navigation (NAV) tracking
function of their autopilot for smoother course following
near the VORTAC.

of 200 to 415 kHz. The frequencies are readily available on
aeronautical charts or in the Chart Supplement U.S.
All radio beacons, except compass locators, transmit
a continuous three-letter identification in code, except
during voice transmissions. A compass locator, which is
associated with an instrument landing system, transmits a
two-letter identification.
Standard broadcast stations can also be used in conjunction
with ADF. Positive identification of all radio stations is
extremely important and this is particularly true when using
standard broadcast stations for navigation.
NDBs have one advantage over the VOR in that low or
medium frequencies are not affected by line-of-sight. The
signals follow the curvature of the Earth; therefore, if the
aircraft is within the range of the station, the signals can be
received regardless of altitude.
The following table gives the class of NDB stations, their
power, and their usable range:
NONDIRECTIONAL RADIO BEACON (NDB)
(Usable radius distances for all altitudes)

Class
Compass Locator
MH
H
HH
*Service

Power
Distance
(Watts)
(Miles)
Under 25
15
Under 50
25
*5 0
50–1999
2000 or more
75

range of individual facilities may be less than

50 miles.
Navigating an aircraft with VOR/DME-based RNAV can be
confusing, and it is essential that the pilot become familiar
with the equipment installed. It is not unknown for pilots to
operate inadvertently in one of the RNAV modes when the
operation was not intended, by overlooking switch positions
or annunciators. The reverse has also occurred with a pilot
neglecting to place the unit into one of the RNAV modes by
overlooking switch positions or annunciators. As always, the
prudent pilot is not only familiar with the equipment used,
but never places complete reliance in just one method of
navigation when others are available for cross-check.
Automatic Direction Finder (ADF)
Many general aviation-type aircraft are equipped with ADF
radio receiving equipment. To navigate using the ADF,
the pilot tunes the receiving equipment to a ground station
known as a nondirectional radio beacon (NDB). The NDB
stations normally operate in a low or medium frequency band

One of the disadvantages that should be considered when
using low frequency (LF) for navigation is that LF signals are
very susceptible to electrical disturbances, such as lightning.
These disturbances create excessive static, needle deviations,
and signal fades. There may be interference from distant
stations. Pilots should know the conditions under which these
disturbances can occur so they can be more alert to possible
interference when using the ADF.
Basically, the ADF aircraft equipment consists of a tuner,
which is used to set the desired station frequency, and the
navigational display.
The navigational display consists of a dial upon which the
azimuth is printed and a needle which rotates around the
dial and points to the station to which the receiver is tuned.

16-29

Some of the ADF dials can be rotated to align the
azimuth with the aircraft heading; others are fixed with 0°
representing the nose of the aircraft and 180° representing
the tail. Only the fixed azimuth dial is discussed in this
handbook. [Figure 16-37]
Figure 16-38 illustrates terms that are used with the ADF
and should be understood by the pilot.

6

3

Tracking to the station requires correcting for wind drift and
results in maintaining flight along a straight track or bearing
to the station. When the wind drift correction is established,
the ADF needle indicates the amount of correction to the
right or left. For instance, if the magnetic bearing to the
station is 340°, a correction for a left crosswind would
result in a magnetic heading of 330°, and the ADF needle
would indicate 10° to the right or a relative bearing of 010°.
[Figure 16-39]

E

33

12

30

N-S

N

E-W

To determine the magnetic bearing "FROM" the station,
180° is added to or subtracted from the magnetic bearing to
the station. This is the reciprocal bearing and is used when
plotting position fixes.

15

S
21

24

W

din
ea
ch
eti
gn
Ma

W

24
21

E

S

16-30

15

Figure 16-38. ADF terms.

12

Radio station

tion
to sta
ng
ar i
be
bearing
ive
lat

6

E-W

Re

N

tic

33

3

N-S

Ma
gn
e

When tracking away from the station, wind corrections are
made similar to tracking to the station, but the ADF needle
points toward the tail of the aircraft or the 180° position on
the azimuth dial. Attempting to keep the ADF needle on
the 180° position during winds results in the aircraft flying
a curved flight leading further and further from the desired
track. When tracking outbound, corrections for wind should
be made in the direction opposite of that in which the needle
is pointing.
Although the ADF is not as popular as the VOR for radio
navigation, with proper precautions and intelligent use, the
ADF can be a valuable aid to navigation.

g

Magnetic North

Figure 16-37. ADF with fixed azimuth and magnetic compass.

30

Keep in mind that the needle of fixed azimuth points to the
station in relation to the nose of the aircraft. If the needle
is deflected 30° to the left for a relative bearing of 330°,
this means that the station is located 30° left. If the aircraft
is turned left 30°, the needle moves to the right 30° and
indicates a relative bearing of 0° meaning that the aircraft
is pointing toward the station. If the pilot continues flight
toward the station keeping the needle on 0°, the procedure
is called homing to the station. If a crosswind exists, the
ADF needle continues to drift away from zero. To keep the
needle on zero, the aircraft must be turned slightly resulting
in a curved flight path to the station. Homing to the station
is a common procedure but may result in drifting downwind,
thus lengthening the distance to the station.

Global Positioning System
The GPS is a satellite-based radio navigation system.
Its RNAV guidance is worldwide in scope. There are no
symbols for GPS on aeronautical charts as it is a space-based
system with global coverage. Development of the system is
underway so that GPS is capable of providing the primary
means of electronic navigation. Portable and yoke-mounted
units are proving to be very popular in addition to those
permanently installed in the aircraft. Extensive navigation
databases are common features in aircraft GPS receivers.

Defense (DOD) is responsible for operating the GPS satellite
constellation and monitors the GPS satellites to ensure proper
operation.
33
0°

The status of a GPS satellite is broadcast as part of the data
message transmitted by the satellite. GPS status information
is also available from the U.S. Coast Guard navigation
information service at (703) 313-5907 or online at \url{
navcen.uscg.gov}. Additionally, satellite status is available
through the NOTAM system.

6

E

33

N

3

30

12

W

15
S

21

24

0°

33
3

6

E

33

N

30

12

W

15
S

21

24

g to

arin

° be

340
tion

sta

Figure 16-39. ADF tracking.

The GPS is a satellite radio navigation and time dissemination
system developed and operated by the U.S. Department of
Defense (DOD). Civilian interface and GPS system status
is available from the U.S. Coast Guard.
It is not necessary to understand the technical aspects of
GPS operation to use it in VFR/IFR navigation. It does differ
significantly from conventional, ground-based electronic
navigation and awareness of those differences is important.
Awareness of equipment approvals and limitations is critical
to the safety of flight.
The GPS navigation system broadcasts a signal that is used
by receivers to determine precise position anywhere in the
world. The receiver tracks multiple satellites and determines
a pseudorange measurement to determine the user location.
A minimum of four satellites is necessary to establish an
accurate three-dimensional position. The Department of

The GPS receiver verifies the integrity (usability) of the
signals received from the GPS constellation through receiver
autonomous integrity monitoring (RAIM) to determine if
a satellite is providing corrupted information. At least one
satellite, in addition to those required for navigation, must
be in view for the receiver to perform the RAIM function;
thus, RAIM needs a minimum of five satellites in view or
four satellites and a barometric altimeter (baro-aiding) to
detect an integrity anomaly. For receivers capable of doing
so, RAIM needs six satellites in view (or five satellites with
baro-aiding) to isolate the corrupt satellite signal and remove
it from the navigation solution. Baro-aiding is a method of
augmenting the GPS integrity solution by using a nonsatellite
input source. GPS derived altitude should not be relied upon
to determine aircraft altitude since the vertical error can
be quite large and no integrity is provided. To ensure that
baro-aiding is available, the current altimeter setting must be
entered into the receiver as described in the operating manual.
RAIM messages vary somewhat between receivers; however,
generally there are two types. One type indicates that there
are not enough satellites available to provide RAIM integrity
monitoring and another type indicates that the RAIM integrity
monitor has detected a potential error that exceeds the limit
for the current phase of flight. Without RAIM capability, the
pilot has no assurance of the accuracy of the GPS position.

Selective Availability
Selective Availability (SA) is a method by which the accuracy
of GPS is intentionally degraded. This feature is designed
to deny hostile use of precise GPS positioning data. SA was
discontinued on May 1, 2000, but many GPS receivers are
designed to assume that SA is still active.
The baseline GPS satellite constellation consists of 24
satellites positioned in six earth-centered orbital planes
with four operation satellites and a spare satellite slot in
each orbital plane. The system can support a constellation
of up to thirty satellites in orbit. The orbital period of a GPS
satellite is one-half of a sidereal day or 11 hours 58 minutes.
The orbits are nearly circular and equally spaced about the
equator at a 60-degree separation with an inclination of
16-31

55 degrees relative to the equator. The orbital radius (i.e.
distance from the center of mass of the earth to the satellite)
is approximately 26,600 km.
With the baseline satellite constellation, users with a clear
view of the sky have a minimum of four satellites in view.
It is more likely that a user would see six to eight satellites.
The satellites broadcast ranging signals and navigation data
allowing users to measure their pseudoranges in order to
estimate their position, velocity and time, in a passive, listenonly mode. The receiver uses data from a minimum of four
satellites above the mask angle (the lowest angle above the
horizon at which a receiver can use a satellite). The exact
number of satellites operating at any one particular time
varies depending on the number of satellite outages and
operational spares in orbit. For current status of the GPS
constellation, please visit http://tycho.usno.navy.mil/gpscurr.
html. [Figure 16-40]

VFR Use of GPS
GPS navigation has become a great asset to VFR pilots
providing increased navigation capability and enhanced
situational awareness while reducing operating costs due
to greater ease in flying direct routes. While GPS has many
benefits to the VFR pilot, care must be exercised to ensure
that system capabilities are not exceeded.
Types of receivers used for GPS navigation under VFR are
varied from a full IFR installation being used to support a
VFR flight to a VFR only installation (in either a VFR or IFR
capable aircraft) to a hand-held receiver. The limitations of

each type of receiver installation or use must be understood
by the pilot to avoid misusing navigation information. In all
cases, VFR pilots should never rely solely on one system
of navigation. GPS navigation must be integrated with
other forms of electronic navigation, as well as pilotage
and dead reckoning. Only through the integration of these
techniques can the VFR pilot ensure accuracy in navigation.
Some critical concerns in VFR use of GPS include RAIM
capability, database currency, and antenna location.

RAIM Capability
Many VFR GPS receivers and all hand-held units are not
equipped with RAIM alerting capability. Loss of the required
number of satellites in view, or the detection of a position
error, cannot be displayed to the pilot by such receivers.
In receivers with no RAIM capability, no alert would
be provided to the pilot that the navigation solution had
deteriorated and an undetected navigation error could occur.
A systematic cross-check with other navigation techniques
would identify this failure and prevent a serious deviation.
In many receivers, an updatable database is used for
navigation fixes, airports, and instrument procedures.
These databases must be maintained to the current update
for IFR operation, but no such requirement exists for VFR
use. However, in many cases, the database drives a moving
map display that indicates Special Use Airspace and the
various classes of airspace in addition to other operational
information. Without a current database, the moving map
display may be outdated and offer erroneous information
to VFR pilots wishing to fly around critical airspace areas,
such as a Restricted Area or a Class B airspace segment.
Numerous pilots have ventured into airspace they were trying
to avoid by using an outdated database. If there is not a current
database in the receiver, disregard the moving map display
when making critical navigation decisions.
In addition, waypoints are added, removed, relocated, or re­
named as required to meet operational needs. When using
GPS to navigate relative to a named fix, a current database
must be used to properly locate a named waypoint. Without
the update, it is the pilot's responsibility to verify the
waypoint location referencing to an official current source,
such as the Chart Supplement U.S., sectional chart, or en
route chart.

Figure 16-40. Satellite constellation.

16-32

In many VFR installations of GPS receivers, antenna location
is more a matter of convenience than performance. In IFR
installations, care is exercised to ensure that an adequate
clear view is provided for the antenna to communicate with
satellites. If an alternate location is used, some portion of
the aircraft may block the view of the antenna increasing the
possibility of losing navigation signal.

This is especially true in the case of hand-held receivers. The
use of hand-held receivers for VFR operations is a growing
trend, especially among rental pilots. Typically, suction cups
are used to place the GPS antennas on the inside of aircraft
windows. While this method has great utility, the antenna
location is limited by aircraft structure for optimal reception
of available satellites. Consequently, signal loss may occur
in certain situations where aircraft-satellite geometry causes
a loss of navigation signal. These losses, coupled with a lack
of RAIM capability, could present erroneous position and
navigation information with no warning to the pilot.
While the use of hand-held GPS receivers for VFR operations
is not limited by regulation, modification of the aircraft, such
as installing a panel- or yoke-mounted holder, is governed by
14 CFR part 43. Pilots should consult a mechanic to ensure
compliance with the regulation and a safe installation.
Tips for Using GPS for VFR Operations
Always check to see if the unit has RAIM capability. If no
RAIM capability exists, be suspicious of a GPS displayed
position when any disagreement exists with the position
derived from other radio navigation systems, pilotage, or
dead reckoning.
Check the currency of the database, if any. If expired, update
the database using the current revision. If an update of an
expired database is not possible, disregard any moving map
display of airspace for critical navigation decisions. Be aware
that named waypoints may no longer exist or may have been
relocated since the database expired. At a minimum, the
waypoints to be used should be verified against a current
official source, such as the Chart Supplement U.S. or a
Sectional Aeronautical Chart.
While a hand-held GPS receiver can provide excellent
navigation capability to VFR pilots, be prepared for
intermittent loss of navigation signal, possibly with no RAIM
warning to the pilot. If mounting the receiver in the aircraft,
be sure to comply with 14 CFR part 43.
Plan flights carefully before taking off. If navigating to userdefined waypoints, enter them prior to flight, not on the fly.
Verify the planned flight against a current source, such as a
current sectional chart. There have been cases in which one
pilot used waypoints created by another pilot that were not
where the pilot flying was expecting. This generally resulted
in a navigation error. Minimize head-down time in the aircraft
and maintain a sharp lookout for traffic, terrain, and obstacles.
Just a few minutes of preparation and planning on the ground
makes a great difference in the air.

Another way to minimize head-down time is to become very
familiar with the receiver's operation. Most receivers are not
intuitive. The pilot must take the time to learn the various
keystrokes, knob functions, and displays that are used in
the operation of the receiver. Some manufacturers provide
computer-based tutorials or simulations of their receivers.
Take the time to learn about the particular unit before using
it in flight.
In summary, be careful not to rely on GPS to solve all VFR
navigational problems. Unless an IFR receiver is installed in
accordance with IFR requirements, no standard of accuracy
or integrity can be assured. While the practicality of GPS is
compelling, the fact remains that only the pilot can navigate the
aircraft, and GPS is just one of the pilot's tools to do the job.
VFR Waypoints
VFR waypoints provide VFR pilots with a supplementary
tool to assist with position awareness while navigating
visually in aircraft equipped with area navigation receivers.
VFR waypoints should be used as a tool to supplement current
navigation procedures. The use of VFR waypoints include
providing navigational aids for pilots unfamiliar with an area,
waypoint definition of existing reporting points, enhanced
navigation in and around Class B and Class C airspace, and
enhanced navigation around Special Use Airspace. VFR
pilots should rely on appropriate and current aeronautical
charts published specifically for visual navigation. If
operating in a terminal area, pilots should take advantage of
the Terminal Area Chart available for the area, if published.
The use of VFR waypoints does not relieve the pilot of any
responsibility to comply with the operational requirements
of 14 CFR part 91.
VFR waypoint names (for computer entry and flight plans)
consist of five letters beginning with the letters "VP" and are
retrievable from navigation databases. The VFR waypoint
names are not intended to be pronounceable, and they are not
for use in ATC communications. On VFR charts, a stand­
alone VFR waypoint is portrayed using the same four-point
star symbol used for IFR waypoints. VFR waypoint collocated
with a visual checkpoint on the chart is identified by a small
magenta flag symbol. A VFR waypoint collocated with a
visual checkpoint is pronounceable based on the name of the
visual checkpoint and may be used for ATC communications.
Each VFR waypoint name appears in parentheses adjacent
to the geographic location on the chart. Latitude/longitude
data for all established VFR waypoints may be found in the
appropriate regional Chart Supplement U.S.

16-33

When filing VFR flight plans, use the five-letter identifier as
a waypoint in the route of flight section if there is an intended
course change at that point or if used to describe the planned
route of flight. This VFR filing would be similar to VOR use
in a route of flight. Pilots must use the VFR waypoints only
when operating under VFR conditions.

information to plot the aircraft position and then give vectors
to a suitable landing site. If the situation becomes threatening,
transmit the situation on the emergency frequency 121.5 MHz
and set the transponder to 7700. Most facilities, and even
airliners, monitor the emergency frequency.

Any VFR waypoints intended for use during a flight should
be loaded into the receiver while on the ground and prior to
departure. Once airborne, pilots should avoid programming
routes or VFR waypoint chains into their receivers.

There may come a time when a pilot is not able to make it to
the planned destination. This can be the result of unpredicted
weather conditions, a system malfunction, or poor preflight
planning. In any case, the pilot needs to be able to safely
and efficiently divert to an alternate destination. Risk
management procedures become a priority during any type
of flight diversion and should be used the pilot. For example,
the hazards of inadvertent VFR into IMC involve a risk that
the pilot can identify and assess and then mitigate through a
pre-planned or in-flight diversion around hazardous weather.
Before any cross-country flight, check the charts for airports
or suitable landing areas along or near the route of flight.
Also, check for navigational aids that can be used during a
diversion. Risk management is explained in greater detail in
Chapter 2, Aeronautical Decision-making.

Pilots should be especially vigilant for other traffic while
operating near VFR waypoints. The same effort to see and
avoid other aircraft near VFR waypoints is necessary, as
is the case when operating near VORs and NDBs. In fact,
the increased accuracy of navigation through the use of
GPS demands even greater vigilance as there are fewer
off-course deviations among different pilots and receivers.
When operating near a VFR waypoint, use all available
ATC services, even if outside a class of airspace where
communications are required. Regardless of the class of
airspace, monitor the available ATC frequency closely for
information on other aircraft operating in the vicinity. It is
also a good idea to turn on landing light(s) when operating
near a VFR waypoint to make the aircraft more conspicuous
to other pilots, especially when visibility is reduced.

Lost Procedures
Getting lost in flight is a potentially dangerous situation,
especially when low on fuel. If a pilot becomes lost, there
are some good common sense procedures to follow. If a town
or city cannot be seen, the first thing to do is climb, being
mindful of traffic and weather conditions. An increase in
altitude increases radio and navigation reception range and
also increases radar coverage. If flying near a town or city, it
may be possible to read the name of the town on a water tower.
If the aircraft has a navigational radio, such as a VOR or ADF
receiver, it can be possible to determine position by plotting
an azimuth from two or more navigational facilities. If GPS
is installed, or a pilot has a portable aviation GPS on board,
it can be used to determine the position and the location of
the nearest airport.
Communicate with any available facility using frequencies
shown on the sectional chart. If contact is made with a
controller, radar vectors may be offered. Other facilities may
offer direction finding (DF) assistance. To use this procedure,
the controller requests the pilot to hold down the transmit
button for a few seconds and then release it. The controller
may ask the pilot to change directions a few times and repeat
the transmit procedure. This gives the controller enough
16-34

Flight Diversion

Computing course, time, speed, and distance information in
flight requires the same computations used during preflight
planning. However, because of the limited flight deck space
and because attention must be divided between flying the
aircraft, making calculations, and scanning for other aircraft,
take advantage of all possible shortcuts and rule-of-thumb
computations.
When in flight, it is rarely practical to actually plot a course
on a sectional chart and mark checkpoints and distances.
Furthermore, because an alternate airport is usually not very far
from your original course, actual plotting is seldom necessary.
The course to an alternate destination can be measured
accurately with a protractor or plotter but can also be
measured with reasonable accuracy using a straightedge
and the compass rose depicted around VOR stations. This
approximation can be made on the basis of a radial from a
nearby VOR or an airway that closely parallels the course
to your alternate destination. However, remember that the
magnetic heading associated with a VOR radial or printed
airway is outbound from the station. To find the course to
the station, it may be necessary to determine the reciprocal of
that heading. It is typically easier to navigate to an alternate
airport that has a VOR or NDB facility on the field.
After selecting the most appropriate alternate destination,
approximate the magnetic course to the alternate using a
compass rose or airway on the sectional chart. If time permits,
try to start the diversion over a prominent ground feature.

However, in an emergency, divert promptly toward your
alternate destination. Attempting to complete all plotting,
measuring, and computations involved before diverting to the
alternate destination may only aggravate an actual emergency.
Once established on course, note the time, and then use the
winds aloft nearest to your diversion point to calculate a
heading and GS. Once a GS has been calculated, determine a
new arrival time and fuel consumption. Give priority to flying
the aircraft while dividing attention between navigation and
planning. When determining an altitude to use while diverting,
consider cloud heights, winds, terrain, and radio reception.

Chapter Summary
This chapter has discussed the fundamentals of VFR
navigation. Beginning with an introduction to the charts that
can be used for navigation to the more technically advanced
concept of GPS, there is one aspect of navigation that remains
the same—the pilot is responsible for proper planning and
the execution of that planning to ensure a safe flight.

16-35

16-36


Chapter 17

Aeromedical
Factors
Introduction
It is important for a pilot to be aware of the mental and
physical standards required for the type of flying performed.
This chapter provides information on medical certification and
on a variety of aeromedical factors related to flight activities.

17-1

Obtaining a Medical Certificate
Most pilots must have a valid medical certificate to exercise
the privileges of their airman certificates. Glider and free
balloon pilots are not required to hold a medical certificate.
Sport pilots may hold either a medical certificate or a valid
state driver's license. Regardless of whether a medical
certificate or drivers license is required, 14 CFR 61.53
requires every pilot not to act as a crewmember if they know,
or have reason to know, of any medical condition that would
make them unable to operate the aircraft in a safe manner.
Acquisition of a medical certificate requires an examination
by an aviation medical examiner (AME), a physician
with training in aviation medicine designated by the Civil
Aerospace Medical Institute (CAMI). There are three classes
of medical certificates. The class of certificate needed
depends on the type of flying the pilot plans to perform.
A third-class medical certificate is required for a private or
recreational pilot certificate. It is valid for 5 years for those
individuals who have not reached the age of 40; otherwise it
is valid for 2 years. A commercial pilot certificate requires at
least a second-class medical certificate, which is valid for 1
year. First-class medical certificates are required for airline
transport pilots and are valid for one year if the airman is 40
or younger; 40 and older it is valid for 6 months.

demonstrated ability" (SODA) can be issued. This waiver,
or SODA, is valid as long as the physical impairment does
not worsen. Contact the local Flight Standards District Office
(FSDO) for more information on this subject.
The FAA medical standards, 14 CFR part 67, specify fifteen
medical conditions that are considered disqualifying by
"history or clinical diagnosis." Regardless of when one of
these conditions was diagnosed and treated, an airman may
not be issued a medical certificate except through a process
called a "Special Issuance Authorization," as explained
in 14 CFR part 67, section 67.401. A special issuance is a
discretionary issuance by the FAA Federal Air Surgeon and
requires satisfactory completion of special testing determined
by the FAA to demonstrate that an airman is safe to fly for
the duration of the medical certificate issued. The specific
disqualifying conditions include:


Diabetes mellitus requiring oral hypoglycemic
medication or insulin

 	 Angina pectoris


Coronary heart disease that has been treated or, if
untreated, that has been symptomatic or clinically
significant

 	 Myocardial infarction
 	 Cardiac valve replacement

The standards are more rigorous for the higher classes of
certificates. A pilot with a higher class medical certificate
has met the requirements for the lower classes as well. Since
the required medical class applies only when exercising the
privileges of the pilot certificate for which it is required, a
first-class medical certificate would be valid for 1 year if
exercising the privileges of a commercial certificate and 2 or 5
years, as appropriate, for exercising the privileges of a private
or recreational certificate. The same applies for a second-class
medical certificate. The standards for medical certification
are contained in Title 14 of the Code of Federal Regulations
(14 CFR) part 67 and the requirements for obtaining medical
certificates can be found in 14 CFR part 61.

 	 Permanent cardiac pacemaker



Disturbance of consciousness and without satisfactory
explanation of cause

Students who have physical limitations, such as impaired
vision, loss of a limb, or hearing impairment may be issued
a medical certificate valid for "student pilot privileges only"
while learning to fly. Pilots with disabilities may require
special equipment to be installed in the aircraft, such as
hand controls for pilots with paraplegia. Some disabilities
necessitate a limitation on the individual's certificate; for
example, impaired hearing would require the limitation
"not valid for flight requiring the use of radio." When all the
knowledge, experience, and proficiency requirements have
been met and a student can demonstrate the ability to operate
the aircraft with the normal level of safety, a "statement of



Transient loss of control of nervous system function(s)
without satisfactory explanation of cause

17-2

 	 Heart replacement
 	 Psychosis
 	 Bipolar disorder


Personality disorder that is severe enough to have
repeatedly manifested itself by overt acts

 	 Substance dependence (including alcohol)
 	 Substance abuse
 	 Epilepsy

However, this list includes only the mandatory disqualifying
conditions. There are many other medical conditions that fall
into the General Medical Condition section of the regulations
that are considered by the FAA to be disqualifying even
though they are not stated in the regulations. Conditions
such as cancer, kidney stones, neurologic and neuromuscular
conditions including Parkinson's disease and multiple
sclerosis, certain blood disorders, and other conditions that

may progress over time require review by the FAA before a
medical certificate may be issued.
The important thing to remember is that with very few
exceptions, all disqualifying medical conditions may
be considered for special issuance. If you can present
satisfactory medical documentation to the FAA that your
condition is stable, the chances are good that you will be
able to qualify for an Authorization.

Health and Physiological Factors
Affecting Pilot Performance
A number of health factors and physiological effects can be
linked to flying. Some are minor, while others are important
enough to require special attention to ensure safety of flight.
In some cases, physiological factors can lead to inflight
emergencies. Some important medical factors that a pilot
should be aware of include hypoxia, hyperventilation,
middle ear and sinus problems, spatial disorientation, motion
sickness, carbon monoxide (CO) poisoning, stress and
fatigue, dehydration, and heatstroke. Other subjects include
the effects of alcohol and drugs, anxiety, and excess nitrogen
in the blood after scuba diving.
Hypoxia
Hypoxia means "reduced oxygen" or "not enough oxygen."
Although any tissue will die if deprived of oxygen long
enough, the greatest concern regarding hypoxia during
flight is lack of oxygen to the brain, since it is particularly
vulnerable to oxygen deprivation. Any reduction in mental
function while flying can result in life-threatening errors.
Hypoxia can be caused by several factors, including an
insufficient supply of oxygen, inadequate transportation of
oxygen, or the inability of the body tissues to use oxygen.
The forms of hypoxia are based on their causes:


Hypoxic hypoxia



Hypemic hypoxia



Stagnant hypoxia



Histotoxic hypoxia

Hypoxic Hypoxia
Hypoxic hypoxia is a result of insufficient oxygen available
to the body as a whole. A blocked airway and drowning
are obvious examples of how the lungs can be deprived of
oxygen, but the reduction in partial pressure of oxygen at high
altitude is an appropriate example for pilots. Although the
percentage of oxygen in the atmosphere is constant, its partial
pressure decreases proportionately as atmospheric pressure
decreases. As an aircraft ascends during flight, the percentage
of each gas in the atmosphere remains the same, but there are
fewer molecules available at the pressure required for them

to pass between the membranes in the respiratory system.
This decrease in number of oxygen molecules at sufficient
pressure can lead to hypoxic hypoxia.
Dangers of Transporting Dry Ice
Sublimation is a process in which a substance transitions
from a solid to a gaseous state without passing through
an intermediate liquid state. Dry ice sublimates into large
quantities of CO2 gas, which can rapidly displace oxygencontaining air and potentially cause hypoxia via carbon
dioxide intoxication. Case studies have shown that both illness
and death can be caused by occupational and/or unintentional
exposure when transporting dry ice in small, confined
spaces such as a flightdeck or airplane. Exposure to high
concentration of CO2 gas may lead to increased respiration,
tachycardia, cardiac arrhythmia, and unconsciousness.
Exposure to concentration of CO2 gas in excess of 10 percent
may cause convulsions, coma, and/or death.
The tendency of dry ice to rapidly sublimate also means that
without proper ventilation, it can rapidly pressurize. For
this reason, dry ice should never be placed inside a sealed
transport container (i.e., leak-proof secondary container)
and must be placed within an outer shipping container or
storage container that allows adequate ventilation to release
the CO2 gas and avoid pressurization. Sealing dry ice within a
leak-proof container may result in explosion of the container
potentially leading to serious physical injury or death.

Hypemic Hypoxia
Hypemic hypoxia occurs when the blood is not able to take
up and transport a sufficient amount of oxygen to the cells
in the body. Hypemic means "not enough blood." This type
of hypoxia is a result of oxygen deficiency in the blood,
rather than a lack of inhaled oxygen, and can be caused by
a variety of factors. It may be due to reduced blood volume
(from severe bleeding), or it may result from certain blood
diseases, such as anemia. More often, hypemic hypoxia
occurs because hemoglobin, the actual blood molecule that
transports oxygen, is chemically unable to bind oxygen
molecules. The most common form of hypemic hypoxia is
CO poisoning. This is explained in greater detail later in this
chapter. Hypemic hypoxia can also be caused by the loss
of blood due to blood donation. Blood volume can require
several weeks to return to normal following a donation.
Although the effects of the blood loss are slight at ground
level, there are risks when flying during this time.

Stagnant Hypoxia
Stagnant means "not flowing," and stagnant hypoxia or
ischemia results when the oxygen-rich blood in the lungs
is not moving, for one reason or another, to the tissues that

17-3

need it. An arm or leg "going to sleep" because the blood
flow has accidentally been shut off is one form of stagnant
hypoxia. This kind of hypoxia can also result from shock,
the heart failing to pump blood effectively, or a constricted
artery. During flight, stagnant hypoxia can occur with
excessive acceleration of gravity (Gs). Cold temperatures
can also reduce circulation and decrease the blood supplied
to extremities.

Histotoxic Hypoxia
The inability of the cells to effectively use oxygen is defined
as histotoxic hypoxia. "Histo" refers to tissues or cells, and
"toxic" means poisonous. In this case, enough oxygen is being
transported to the cells that need it, but they are unable to make
use of it. This impairment of cellular respiration can be caused
by alcohol and other drugs, such as narcotics and poisons.
Research has shown that drinking one ounce of alcohol can
equate to an additional 2,000 feet of physiological altitude.
Symptoms of Hypoxia
High-altitude flying can place a pilot in danger of becoming
hypoxic. Oxygen starvation causes the brain and other vital
organs to become impaired. The first symptoms of hypoxia
can include euphoria and a carefree feeling. With increased
oxygen starvation, the extremities become less responsive and
flying becomes less coordinated. The symptoms of hypoxia
vary with the individual, but common symptoms include:


Cyanosis (blue fingernails and lips)



Headache



Decreased response to stimuli and increased reaction
time



Impaired judgment



Euphoria



Visual impairment



Drowsiness



Lightheaded or dizzy sensation



Tingling in fingers and toes



Numbness

endurance or acclimatization. When flying at high altitudes,
it is paramount that oxygen be used to avoid the effects of
hypoxia. The term "time of useful consciousness" describes
the maximum time the pilot has to make rational, life-saving
decisions and carry them out at a given altitude without
supplemental oxygen. As altitude increases above 10,000
feet, the symptoms of hypoxia increase in severity, and the
time of useful consciousness rapidly decreases. [Figure 17-1]
Since symptoms of hypoxia can be different for each
individual, the ability to recognize hypoxia can be greatly
improved by experiencing and witnessing the effects of it
during an altitude chamber "flight." The Federal Aviation
Administration (FAA) provides this opportunity through
aviation physiology training, which is conducted at the FAA
CAMI in Oklahoma City, Oklahoma, and at many military
facilities across the United States. For information about the
FAA's one-day physiological training course with altitude
chamber and vertigo demonstrations, visit the FAA website
at \url{faa.gov}.
Hyperventilation
Hyperventilation is the excessive rate and depth of respiration
leading to abnormal loss of carbon dioxide from the blood.
This condition occurs more often among pilots than is
generally recognized. It seldom incapacitates completely, but
it causes disturbing symptoms that can alarm the uninformed
pilot. In such cases, increased breathing rate and anxiety
further aggravate the problem. Hyperventilation can lead to
unconsciousness due to the respiratory system's overriding
mechanism to regain control of breathing.
Pilots encountering an unexpected stressful situation may
subconsciously increase their breathing rate. If flying at
higher altitudes, either with or without oxygen, a pilot may
have a tendency to breathe more rapidly than normal, which
often leads to hyperventilation.
Since many of the symptoms of hyperventilation are similar
to those of hypoxia, it is important to correctly diagnose and
treat the proper condition. If using supplemental oxygen,
check the equipment and flow rate to ensure the symptoms are
Altitude

Time of useful consciousness

45,000 feet MSL

9 to 15 seconds

40,000 feet MSL

15 to 20 seconds

35,000 feet MSL

30 to 60 seconds

30,000 feet MSL

1 to 2 minutes

28,000 feet MSL

2½ to 3 minutes

Treatment of Hypoxia

25,000 feet MSL

3 to 5 minutes

Treatment for hypoxia includes flying at lower altitudes and/
or using supplemental oxygen. All pilots are susceptible
to the effects of oxygen starvation, regardless of physical

22,000 feet MSL

5 to 10 minutes

20,000 feet MSL

30 minutes or more

As hypoxia worsens, the field of vision begins to narrow and
instrument interpretation can become difficult. Even with all
these symptoms, the effects of hypoxia can cause a pilot to
have a false sense of security and be deceived into believing
everything is normal.

17-4

Figure 17-1. Time of useful consciousness.

not hypoxia related. Common symptoms of hyperventilation
include:


Visual impairment



Unconsciousness



Lightheaded or dizzy sensation



Tingling sensations



Hot and cold sensations



Muscle spasms

The treatment for hyperventilation involves restoring
the proper carbon dioxide level in the body. Breathing
normally is both the best prevention and the best cure
for hyperventilation. In addition to slowing the breathing
rate, breathing into a paper bag or talking aloud helps to
overcome hyperventilation. Recovery is usually rapid once
the breathing rate is returned to normal.
Middle Ear and Sinus Problems
During climbs and descents, the free gas formerly present in
various body cavities expands due to a difference between
the pressure of the air outside the body and that of the air
inside the body. If the escape of the expanded gas is impeded,
pressure builds up within the cavity and pain is experienced.
Trapped gas expansion accounts for ear pain and sinus pain,
as well as a temporary reduction in the ability to hear.
The middle ear is a small cavity located in the bone of the
skull. It is closed off from the external ear canal by the
eardrum. Normally, pressure differences between the middle
ear and the outside world are equalized by a tube leading
from inside each ear to the back of the throat on each side
called the Eustachian tube. These tubes are usually closed but
open during chewing, yawning, or swallowing to equalize
pressure. Even a slight difference between external pressure
and middle ear pressure can cause discomfort. [Figure 17-2]
During a climb, middle ear air pressure may exceed the
pressure of the air in the external ear canal causing the
eardrum to bulge outward. Pilots become aware of this
pressure change when they experience alternate sensations
of "fullness" and "clearing." During descent, the reverse
happens. While the pressure of the air in the external ear
canal increases, the middle ear cavity, which equalized with
the lower pressure at altitude, is at lower pressure than the
external ear canal. This results in the higher outside pressure
causing the eardrum to bulge inward.
This condition can be more difficult to relieve due to the
fact that the partial vacuum tends to constrict the walls of
the Eustachian tube. To remedy this often painful condition,
which also causes a temporary reduction in hearing

Middle ear

Eustachian tube

Eardrum

Auditory canal
Outer ear

Opening to throat

Figure 17-2. The Eustachian tube allows air pressure to equalize

in the middle ear.

sensitivity, pinch the nostrils shut, close the mouth and lips,
and blow slowly and gently into the mouth and nose.
This procedure forces air through the Eustachian tube into the
middle ear. It may not be possible to equalize the pressure in
the ears if a pilot has a cold, an ear infection, or sore throat.
A flight in this condition can be extremely painful, as well as
damaging to the eardrums. If experiencing minor congestion,
nose drops or nasal sprays may reduce the risk of a painful
ear blockage. Before using any medication, check with an
AME to ensure that it will not affect the ability to fly.
In a similar way, air pressure in the sinuses equalizes with
the pressure in the flight deck through small openings
that connect the sinuses to the nasal passages. An upper
respiratory infection, such as a cold or sinusitis, or a nasal
allergic condition can produce enough congestion around an
opening to slow equalization. As the difference in pressure
between the sinuses and the flight deck increases, congestion
may plug the opening. This "sinus block" occurs most
frequently during descent. Slow descent rates can reduce the
associated pain. A sinus block can occur in the frontal sinuses,
located above each eyebrow, or in the maxillary sinuses,
located in each upper cheek. It usually produces excruciating
pain over the sinus area. A maxillary sinus block can also
make the upper teeth ache. Bloody mucus may discharge
from the nasal passages.
Sinus block can be avoided by not flying with an upper
respiratory infection or nasal allergic condition. Adequate
protection is usually not provided by decongestant sprays
or drops to reduce congestion around the sinus openings.
Oral decongestants have side effects that can impair pilot
performance. If a sinus block does not clear shortly after
landing, a physician should be consulted.
17-5

Spatial Disorientation and Illusions
Spatial disorientation specifically refers to the lack of
orientation with regard to the position, attitude, or movement
of the airplane in space. The body uses three integrated
systems that work together to ascertain orientation and
movement in space.


Vestibular system—organs found in the inner ear that
sense position by the way we are balanced

 	 Somatosensory system—nerves in the skin, muscles,
and joints that, along with hearing, sense position
based on gravity, feeling, and sound


Visual system—eyes, which sense position based on
what is seen

All this information comes together in the brain and, most
of the time, the three streams of information agree, giving
a clear idea of where and how the body is moving. Flying
can sometimes cause these systems to supply conflicting
information to the brain, which can lead to disorientation.
During flight in visual meteorological conditions (VMC),
the eyes are the major orientation source and usually prevail
over false sensations from other sensory systems. When
these visual cues are removed, as they are in instrument
meteorological conditions (IMC), false sensations can cause
a pilot to quickly become disoriented.
The vestibular system in the inner ear allows the pilot to
sense movement and determine orientation in the surrounding
environment. In both the left and right inner ear, three
semicircular canals are positioned at approximate right angles
to each other. [Figure 17-3] Each canal is filled with fluid
and has a section full of fine hairs. Acceleration of the inner

ear in any direction causes the tiny hairs to deflect, which
in turn stimulates nerve impulses, sending messages to the
brain. The vestibular nerve transmits the impulses from
the utricle, saccule, and semicircular canals to the brain to
interpret motion.
The somatosensory system sends signals from the skin, joints,
and muscles to the brain that are interpreted in relation to the
Earth's gravitational pull. These signals determine posture.
Inputs from each movement update the body's position to the
brain on a constant basis. "Seat of the pants" flying is largely
dependent upon these signals. Used in conjunction with visual
and vestibular clues, these sensations can be fairly reliable.
However, the body cannot distinguish between acceleration
forces due to gravity and those resulting from maneuvering
the aircraft, which can lead to sensory illusions and false
impressions of an aircraft's orientation and movement.
Under normal flight conditions, when there is a visual
reference to the horizon and ground, the sensory system in the
inner ear helps to identify the pitch, roll, and yaw movements
of the aircraft. When visual contact with the horizon is lost,
the vestibular system becomes unreliable. Without visual
references outside the aircraft, there are many situations in
which combinations of normal motions and forces create
convincing illusions that are difficult to overcome.
Prevention is usually the best remedy for spatial disorientation.
Unless a pilot has many hours of training in instrument flight,
flight should be avoided in reduced visibility or at night when
the horizon is not visible. A pilot can reduce susceptibility
to disorienting illusions through training and awareness and
learning to rely totally on flight instruments.

Ampulla of semicircular canal
YAW

Semicircular canals
Otolith organ

ROLL

PITC

H

PITC

ROLL

H

YAW
The semicircular tubes are arranged
at approximately, right angles to each
other in the roll, pitch, and yaw axes.

Endolymph fluid
Vestibular nerve

Figure 17-3. The semicircular canals lie in three planes and sense motions of roll, pitch, and yaw.

17-6

Cupola
Hair cells

Vestibular Illusions
The Leans
A condition called the leans, is the most common illusion
during flight and is caused by a sudden return to level flight
following a gradual and prolonged turn that went unnoticed by
the pilot. The reason a pilot can be unaware of such a gradual
turn is that human exposure to a rotational acceleration of 2
degrees per second or lower is below the detection threshold
of the semicircular canals. [Figure 17-4] Leveling the wings
after such a turn may cause an illusion that the aircraft is
banking in the opposite direction. In response to such an
illusion, a pilot may lean in the direction of the original turn
in a corrective attempt to regain the perception of a correct
vertical posture.
Coriolis Illusion
The "coriolis illusion" occurs when a pilot has been in a turn
long enough for the fluid in the ear canal to move at the same
speed as the canal. A movement of the head in a different
plane, such as looking at something in a different part of the
flight deck, may set the fluid moving, creating the illusion
of turning or accelerating on an entirely different axis. This
action causes the pilot to think the aircraft is performing a
maneuver it is not. The disoriented pilot may maneuver the
aircraft into a dangerous attitude in an attempt to correct the
aircraft's perceived attitude.

direction causing the disoriented pilot to return the aircraft
to its original turn. Because an aircraft tends to lose altitude
in turns unless the pilot compensates for the loss in lift,
the pilot may notice a loss of altitude. The absence of any
sensation of turning creates the illusion of being in a level
descent. The pilot may pull back on the controls in an attempt
to climb or stop the descent. This action tightens the spiral
and increases the loss of altitude; this illusion is referred to
as a "graveyard spiral." [Figure 17-5] This may lead to a
loss of aircraft control.
Somatogravic Illusion
A rapid acceleration, such as experienced during takeoff,
stimulates the otolith organs in the same way as tilting the
head backwards. This action may create what is known as
the "somatogravic illusion" of being in a nose-up attitude,
especially in conditions with poor visual references. The
disoriented pilot may push the aircraft into a nose-low or
dive attitude. A rapid deceleration by quick reduction of the
throttle(s) can have the opposite effect, with the disoriented
pilot pulling the aircraft into a nose-up or stall attitude.

Graveyard spin

For this reason, it is important that pilots develop an
instrument cross-check or scan that involves minimal head
movement. Take care when retrieving charts and other objects
in the flight deck—if something is dropped, retrieve it with
minimal head movement and be alert for the coriolis illusion.
Graveyard Spiral
As in other illusions, a pilot in a prolonged coordinated,
constant-rate turn may experience the illusion of not
turning. During the recovery to level flight, the pilot will
then experience the sensation of turning in the opposite

Endolymph

Graveyard spiral
Figure 17-5. Graveyard spiral.

Cupola

Tube

No turning
No sensation.

Start of turn
Sensation of turning
as moving fluid deflects
hairs.

Constant rate turn
Turn stopped
No sensation after fluid
Sensation of turning in
accelerates to same opposite direction as moving
speed as tube wall. fluid deflects hairs in opposite
direction.

Figure 17-4. Human sensation of angular acceleration.

17-7

Inversion Illusion
An abrupt change from climb to straight-and-level flight can
stimulate the otolith organs enough to create the illusion of
tumbling backwards, known as "inversion illusion." The
disoriented pilot may push the aircraft abruptly into a noselow attitude, which may intensify this illusion.

Autokinesis
When flying in the dark, a stationary light may appear to
move if it is stared at for a prolonged period of time. As
a result, a pilot may attempt to align the aircraft with the
perceived moving light potentially causing him/her to lose
control of the aircraft. This illusion is known as "autokinesis."

Elevator Illusion
An abrupt upward vertical acceleration, as can occur in an
updraft, can stimulate the otolith organs to create the illusion
of being in a climb. This is known as "elevator illusion."
The disoriented pilot may push the aircraft into a nose-low
attitude. An abrupt downward vertical acceleration, usually
in a downdraft, has the opposite effect with the disoriented
pilot pulling the aircraft into a nose-up attitude.

Postural Considerations
The postural system sends signals from the skin, joints, and
muscles to the brain that are interpreted in relation to the
Earth's gravitational pull. These signals determine posture.
Inputs from each movement update the body's position to
the brain on a constant basis. "Seat of the pants" flying is
largely dependent upon these signals. Used in conjunction
with visual and vestibular clues, these sensations can be
fairly reliable. However, because of the forces acting upon
the body in certain flight situations, many false sensations
can occur due to acceleration forces overpowering gravity.
[Figure 17-6] These situations include uncoordinated turns,
climbing turns, and turbulence.

Visual Illusions
Visual illusions are especially hazardous because pilots rely
on their eyes for correct information. Two illusions that lead
to spatial disorientation, false horizon and autokinesis, affect
the visual system only.
False Horizon
A sloping cloud formation, an obscured horizon, an aurora
borealis, a dark scene spread with ground lights and stars,
and certain geometric patterns of ground lights can provide
inaccurate visual information, or "false horizon," when
attempting to align the aircraft with the actual horizon.
The disoriented pilots as a result may place the aircraft in a
dangerous attitude.

Demonstration of Spatial Disorientation
There are a number of controlled aircraft maneuvers a pilot
can perform to experiment with spatial disorientation. While
each maneuver normally creates a specific illusion, any false
sensation is an effective demonstration of disorientation.
Thus, even if there is no sensation during any of these
maneuvers, the absence of sensation is still an effective
demonstration because it illustrates the inability to detect
bank or roll.

Level

Coordinated turn

Pull out

Level skid

Forward slip

Uncoordinated turn

Skid, slip, and uncoordinated turns feel similar.
Pilots feel they are being forced sideways in their seat.
Figure 17-6. Sensations from centrifugal force.

17-8

There are several objectives in demonstrating these various
maneuvers.
1.	 They teach pilots to understand the susceptibility of
the human system to spatial disorientation.
2.	 They demonstrate that judgments of aircraft attitude
based on bodily sensations are frequently false.
3.	 They help decrease the occurrence and degree
of disorientation through a better understanding
of the relationship between aircraft motion, head
movements, and resulting disorientation.
4.	 They help instill a greater confidence in relying on
flight instruments for assessing true aircraft attitude.
A pilot should not attempt any of these maneuvers at
low altitudes or in the absence of an instructor pilot or an
appropriate safety pilot.

Climbing While Accelerating
With the pilot's eyes closed, the instructor pilot maintains
approach airspeed in a straight-and-level attitude for several
seconds, then accelerates while maintaining straight-and­
level attitude. The usual illusion during this maneuver,
without visual references, is that the aircraft is climbing.

Climbing While Turning
With the pilot's eyes still closed and the aircraft in a straight­
and-level attitude, the instructor pilot now executes, with a
relatively slow entry, a well coordinated turn of about 1.5
positive G (approximately 50° bank) for 90°. While in the
turn, without outside visual references and under the effect of
the slight positive G, the usual illusion produced is that of a
climb. Upon sensing the climb, the pilot should immediately
open the eyes to see that a slowly established, coordinated
turn produces the same sensation as a climb.

Diving While Turning

aircraft to approximately 45° bank attitude while maintaining
heading and pitch attitude. This creates the illusion of a strong
sense of rotation in the opposite direction. After this illusion
is noted, the pilot should open his or her eyes and observe
that the aircraft is in a banked attitude.

Diving or Rolling Beyond the Vertical Plane
This maneuver may produce extreme disorientation. While
in straight-and-level flight, the pilot should sit normally,
either with eyes closed or gaze lowered to the floor. The
instructor pilot starts a positive, coordinated roll toward a
30° or 40° angle of bank. As this is in progress, the pilot
tilts his or her head forward, looks to the right or left, then
immediately returns his or her head to an upright position.
The instructor pilot should time the maneuver so the roll is
stopped as the pilot returns his or her head upright. An intense
disorientation is usually produced by this maneuver, and the
pilot experiences the sensation of falling downward into the
direction of the roll.
In the descriptions of these maneuvers, the instructor pilot is
doing the flying, but having the pilot do the flying can also
be a very effective demonstration. The pilot should close his
or her eyes and tilt the head to one side. The instructor pilot
tells the pilot what control inputs to perform. The pilot then
attempts to establish the correct attitude or control input with
eyes closed and head tilted. While it is clear the pilot has no
idea of the actual attitude, he or she will react to what the
senses are saying. After a short time, the pilot will become
disoriented and the instructor pilot will tell the pilot to look
up and recover. This exercise allows the pilot to experience
the disorientation while flying the aircraft.
Coping with Spatial Disorientation
To prevent illusions and their potentially disastrous
consequences, pilots can:

Repeating the previous procedure, but with the pilot's
eyes should be kept closed until recovery from the turn is
approximately one-half completed, can create the illusion of
diving while turning.

1.	 Understand the causes of these illusions and remain
constantly alert for them. Take the opportunity to
experience spatial disorientation illusions in a device,
such as a Barany chair, a Vertigon, or a Virtual Reality
Spatial Disorientation Demonstrator.

Tilting to Right or Left

2.	 Always obtain and understand preflight weather
briefings.

While in a straight-and-level attitude, with the pilot's eyes
closed, the instructor pilot executes a moderate or slight skid
to the left with wings level. This creates the illusion of the
body being tilted to the right.

Reversal of Motion
This illusion can be demonstrated in any of the three planes
of motion. While straight and level, with the pilot's eyes
closed, the instructor pilot smoothly and positively rolls the

3.	 Before flying in marginal visibility (less than 3 miles)
or where a visible horizon is not evident, such as flight
over open water during the night, obtain training and
maintain proficiency in aircraft control by reference
to instruments.
4.	 Do not fly into adverse weather conditions or into
dusk or darkness unless proficient in the use of flight
instruments. If intending to fly at night, maintain
17-9

night-flight currency and proficiency. Include crosscountry and local operations at various airfields.
5.	 Ensure that when outside visual references are used,
they are reliable, fixed points on the Earth's surface.
6.	 Avoid sudden head movement, particularly during
takeoffs, turns, and approaches to landing.
7.	 Be physically tuned for flight into reduced visibility.
Ensure proper rest, adequate diet, and, if flying at
night, allow for night adaptation. Remember that
illness, medication, alcohol, fatigue, sleep loss, and
mild hypoxia are likely to increase susceptibility to
spatial disorientation.
8.	 Most importantly, become proficient in the use of
flight instruments and rely upon them. Trust the
instruments and disregard your sensory perceptions.
The sensations that lead to illusions during instrument
flight conditions are normal perceptions experienced by
pilots. These undesirable sensations cannot be completely
prevented, but through training and awareness, pilots can
ignore or suppress them by developing absolute reliance
on the flight instruments. As pilots gain proficiency in
instrument flying, they become less susceptible to these
illusions and their effects.
Optical Illusions
Of the senses, vision is the most important for safe flight.
However, various terrain features and atmospheric conditions
can create optical illusions. These illusions are primarily
associated with landing. Since pilots must transition from
reliance on instruments to visual cues outside the flight
deck for landing at the end of an instrument approach, it
is imperative that they be aware of the potential problems
associated with these illusions and take appropriate corrective
action. The major illusions leading to landing errors are
described below.

Runway Width Illusion
A narrower-than-usual runway can create an illusion that the
aircraft is at a higher altitude than it actually is, especially
when runway length-to-width relationships are comparable.
[Figure 17-7] The pilot who does not recognize this illusion
will fly a lower approach, with the risk of striking objects
along the approach path or landing short. A wider-than­
usual runway can have the opposite effect with the risk of
the pilot leveling out the aircraft high and landing hard or
overshooting the runway.

Runway and Terrain Slopes Illusion
An upsloping runway, upsloping terrain, or both can create an
illusion that the aircraft is at a higher altitude than it actually

17-10

is. [Figure 17-7] The pilot who does not recognize this
illusion will fly a lower approach. Downsloping runways and
downsloping approach terrain can have the opposite effect.

Featureless Terrain Illusion
An absence of surrounding ground features, as in an
overwater approach over darkened areas or terrain made
featureless by snow, can create an illusion that the aircraft is
at a higher altitude than it actually is. This illusion, sometimes
referred to as the "black hole approach," causes pilots to fly
a lower approach than is desired.

Water Refraction
Rain on the windscreen can create an illusion of being at a
higher altitude due to the horizon appearing lower than it is.
This can result in the pilot flying a lower approach.

Haze
Atmospheric haze can create an illusion of being at a greater
distance and height from the runway. As a result, the pilot
has a tendency to be low on the approach. Conversely,
extremely clear air (clear bright conditions of a high attitude
airport) can give the pilot the illusion of being closer than
he or she actually is, resulting in a high approach that may
result in an overshoot or go around. The diffusion of light
due to water particles on the windshield can adversely affect
depth perception. The lights and terrain features normally
used to gauge height during landing become less effective
for the pilot.

Fog
Flying into fog can create an illusion of pitching up. Pilots
who do not recognize this illusion often steepen the approach
abruptly.

Ground Lighting Illusions
Lights along a straight path, such as a road or lights on moving
trains, can be mistaken for runway and approach lights. Bright
runway and approach lighting systems, especially where
few lights illuminate the surrounding terrain, may create the
illusion of less distance to the runway. The pilot who does
not recognize this illusion will often fly a higher approach.
How To Prevent Landing Errors Due to Optical
Illusions
To prevent these illusions and their potentially hazardous
consequences, pilots can:
1.	 Anticipate the possibility of visual illusions during
approaches to unfamiliar airports, particularly at night
or in adverse weather conditions. Consult airport

Narrower runway

Runway width illusion

Wider runway

 A narrower-than-usual runway can
create an illusion that the aircraft
is higher than it actually is, leading
to a lower approach.
h
roac
App
mal
r
o
N

25

25

h
proac
al Ap
Norm

Narrower runway

 A wider-than-usual runway can
create an illusion that the aircraft is
lower than it actually is, leading to
a higher approach.

Wider runway

25

Downsloping runway

25

Runway slope illusion

Upsloping runway

ach
pro
l Ap
a
m
Nor

ach
Normal Appro

25

25

 A downsloping runway can create
the illusion that the aircraft is lower
than it actually is, leading to a
higher approach.

Downsloping runway

 An upsloping runway can create
the illusion that the aircraft is higher
than it actually is, leading to a lower
approach.

Upsloping runway

Normal approach
Approach due to illusion

Figure 17-7. Runway illusions.

diagrams and the Chart Supplement U.S. (formerly
Airport/Facility Directory) for information on runway
slope, terrain, and lighting.
2.	 Make frequent reference to the altimeter, especially
during all approaches, day and night.
3.	 If possible, conduct an aerial visual inspection of

unfamiliar airports before landing.


4.

Use Visual Approach Slope Indicator (VASI) or
Precision Approach Path Indicator (PAPI) systems
for a visual reference, or an electronic glideslope,
whenever they are available.

5.

Utilize the visual descent point (VDP) found on many
nonprecision instrument approach procedure charts.

17-11

6.	 Recognize that the chances of being involved in an
approach accident increase when an emergency or
other activity distracts from usual procedures.
7.	 Maintain optimum proficiency in landing procedures.
In addition to the sensory illusions due to misleading inputs to
the vestibular system, a pilot may also encounter various visual
illusions during flight. Illusions rank among the most common
factors cited as contributing to fatal aviation accidents.
Sloping cloud formations, an obscured horizon, a dark scene
spread with ground lights and stars, and certain geometric
patterns of ground light can create illusions of not being
aligned correctly with the actual horizon. Various surface
features and atmospheric conditions encountered in landing
can create illusions of being on the wrong approach path.
Landing errors due to these illusions can be prevented by
anticipating them during approaches, inspecting unfamiliar
airports before landing, using electronic glideslope or VASI
systems when available, and maintaining proficiency in
landing procedures.
Motion Sickness
Motion sickness, or airsickness, is caused by the brain
receiving conflicting messages about the state of the body. A
pilot may experience motion sickness during initial flights, but
it generally goes away within the first few lessons. Anxiety
and stress, which may be experienced at the beginning of
flight training, can contribute to motion sickness. Symptoms
of motion sickness include general discomfort, nausea,
dizziness, paleness, sweating, and vomiting.
It is important to remember that experiencing airsickness is
no reflection on one's ability as a pilot. If prone to motion
sickness, let the flight instructor know, there are techniques
that can be used to overcome this problem. For example,
avoid lessons in turbulent conditions until becoming more
comfortable in the aircraft or start with shorter flights and
graduate to longer instruction periods. If symptoms of motion
sickness are experienced during a lesson, opening fresh air
vents, focusing on objects outside the airplane, and avoiding
unnecessary head movements may help alleviate some of the
discomfort. Although medications like Dramamine can prevent
airsickness in passengers, they are not recommended while
flying since they can cause drowsiness and other problems.
Carbon Monoxide (CO) Poisoning
CO is a colorless and odorless gas produced by all internal
combustion engines. Attaching itself to the hemoglobin in
the blood about 200 times more easily than oxygen, CO
prevents the hemoglobin from carrying oxygen to the cells,
resulting in hypemic hypoxia. The body requires up to 48
hours to dispose of CO. If severe enough, the CO poisoning
17-12

can result in death. Aircraft heater vents and defrost vents
may provide CO a passageway into the cabin, particularly if
the engine exhaust system has a leak or is damaged. If a strong
odor of exhaust gases is detected, assume that CO is present.
However, CO may be present in dangerous amounts even
if no exhaust odor is detected. Disposable, inexpensive CO
detectors are widely available. In the presence of CO, these
detectors change color to alert the pilot of the presence of CO.
Some effects of CO poisoning are headache, blurred vision,
dizziness, drowsiness, and/or loss of muscle power. Any time
a pilot smells exhaust odor, or any time these symptoms are
experienced, immediate corrective action should be taken
including turning off the heater, opening fresh air vents and
windows, and using supplemental oxygen, if available.
Tobacco smoke also causes CO poisoning. Smoking at
sea level can raise the CO concentration in the blood and
result in physiological effects similar to flying at 8,000 feet.
Besides hypoxia, tobacco causes diseases and physiological
debilitation that can be medically disqualifying for pilots.
Stress
Stress is the body's response to physical and psychological
demands placed upon it. The body's reaction to stress includes
releasing chemical hormones (such as adrenaline) into the
blood and increasing metabolism to provide more energy
to the muscles. Blood sugar, heart rate, respiration, blood
pressure, and perspiration all increase. The term "stressor"
is used to describe an element that causes an individual to
experience stress. Examples of stressors include physical
stress (noise or vibration), physiological stress (fatigue), and
psychological stress (difficult work or personal situations).
Stress falls into two broad categories: acute (short term) and
chronic (long term). Acute stress involves an immediate
threat that is perceived as danger. This is the type of stress that
triggers a "fight or flight" response in an individual, whether
the threat is real or imagined. Normally, a healthy person can
cope with acute stress and prevent stress overload. However,
ongoing acute stress can develop into chronic stress.
Chronic stress can be defined as a level of stress that presents
an intolerable burden, exceeds the ability of an individual
to cope, and causes individual performance to fall sharply.
Unrelenting psychological pressures, such as loneliness,
financial worries, and relationship or work problems can
produce a cumulative level of stress that exceeds a person's
ability to cope with the situation. When stress reaches these
levels, performance falls off rapidly. Pilots experiencing
this level of stress are not safe and should not exercise their
airman privileges. Pilots who suspect they are suffering from
chronic stress should consult a physician.

Fatigue
Fatigue is frequently associated with pilot error. Some of
the effects of fatigue include degradation of attention and
concentration, impaired coordination, and decreased ability
to communicate. These factors seriously influence the
ability to make effective decisions. Physical fatigue results
from sleep loss, exercise, or physical work. Factors such as
stress and prolonged performance of cognitive work result
in mental fatigue.
Like stress, fatigue falls into two broad categories: acute
and chronic. Acute fatigue is short term and is a normal
occurrence in everyday living. It is the kind of tiredness
people feel after a period of strenuous effort, excitement, or
lack of sleep. Rest after exertion and 8 hours of sound sleep
ordinarily cures this condition.
A special type of acute fatigue is skill fatigue. This type of
fatigue has two main effects on performance:




Timing disruption—appearing to perform a task as
usual, but the timing of each component is slightly off.
This makes the pattern of the operation less smooth
because the pilot performs each component as though it
were separate, instead of part of an integrated activity.
Disruption of the perceptual field—concentrating
attention upon movements or objects in the center of
vision and neglecting those in the periphery. This is
accompanied by loss of accuracy and smoothness in
control movements.

Acute fatigue has many causes, but the following are among
the most important for the pilot:


Mild hypoxia (oxygen deficiency)



Physical stress



Psychological stress



Depletion of physical energy resulting from
psychological stress



Sustained psychological stress

Sustained psychological stress accelerates the glandular
secretions that prepare the body for quick reactions during
an emergency. These secretions make the circulatory and
respiratory systems work harder, and the liver releases energy
to provide the extra fuel needed for brain and muscle work.
When this reserve energy supply is depleted, the body lapses
into generalized and severe fatigue.
Acute fatigue can be prevented by proper diet and adequate
rest and sleep. A well-balanced diet prevents the body from
needing to consume its own tissues as an energy source.
Adequate rest maintains the body's store of vital energy.

Chronic fatigue, extending over a long period of time, usually
has psychological roots, although an underlying disease is
sometimes responsible. Continuous high-stress levels produce
chronic fatigue. Chronic fatigue is not relieved by proper diet
and adequate rest and sleep and usually requires treatment
by a physician. An individual may experience this condition
in the form of weakness, tiredness, palpitations of the heart,
breathlessness, headaches, or irritability. Sometimes chronic
fatigue even creates stomach or intestinal problems and
generalized aches and pains throughout the body. When the
condition becomes serious enough, it leads to emotional illness.
If suffering from acute fatigue, stay on the ground. If fatigue
occurs in the flight deck, no amount of training or experience
can overcome the detrimental effects. Getting adequate rest
is the only way to prevent fatigue from occurring. Avoid
flying without a full night's rest, after working excessive
hours, or after an especially exhausting or stressful day. Pilots
who suspect they are suffering from chronic fatigue should
consult a physician.
Exposure to Chemicals
When conducting preflight and post-flight inspections, pilots
must verify that the fluid levels in their aircraft meet the
levels specified for safe operations as stated in the Pilot's
Operating Handbook. These fluids include, but are not limited
to hydraulic fluid, engine oil, and fuel.
It is important that every pilot recognize the potential hazards
of working with these fluids as well as the recommended first
aid measures to follow should any of these fluids come in
contact with their eyes, skin, and/or respiratory system. As
the specific first aid measures for dealing with exposure to
these chemicals can vary by chemical type, it is important that
every pilot be familiar with the location and use of the Material
Safety Data Sheet (MSDS) for each chemical they encounter.
The procedures described in the following sections are
minimum guideline for first aid for each of the indicated
scenarios. Ultimately, the pilot should consult the MSDS
for first aid procedures specific to the type of chemical and
exposure scenario.

Hydraulic Fluid


Eye Contact—immediately flush the eyes with clean
water and seek medical attention if irritation occurs.



Skin Contact—remove all contaminated clothing and
thoroughly cleanse the affected areas with mild soap
and water or a waterless hand cleaner. If irritation or
redness develops and persists, seek medical attention.
Should the hydraulic fluid get into or under the skin,
or into any other part of the body, regardless of the
17-13

appearance of the wound or its size, seek medical
attention immediately.




Inhalation—if respiratory symptoms develop, move
away from the source of exposure and into fresh air
in a position comfortable for breathing. If symptoms
persist, seek medical attention.
Ingestion—first aid is not normally required; however,
if swallowed and symptoms develop, seek medical
attention.

Engine Oil


Eye Contact—immediately flush the eyes with clean
water and seek medical attention if irritation occurs.



Skin Contact—remove all contaminated clothing and
thoroughly cleanse the affected areas with soap and
water. Launder contaminated clothing before reuse.



Inhalation—move away from the source of exposure
and into fresh air. If respiratory irritation, dizziness,
nausea, or unconsciousness occurs, seek immediate
medical attention. If breathing stops, assisted
ventilation is required via a bag-valve-mask or
cardiopulmonary resuscitation (CPR).



Ingestion—seek immediate medical attention. If
immediate medical attention is not available, contact
a regional poison control center or emergency medical
professional regarding the induction of vomiting or
use of activated charcoal. Vomiting should never be
induced to a person who is groggy or unconscious.

Fuel


Eye Contact—immediately flush the eyes with
clean water for at least 15 minutes and seek medical
attention immediately.



Skin Contact—remove all contaminated clothing
and thoroughly cleanse the affected areas with mild
soap and water or a waterless hand cleaner. If skin
surface is damaged, apply a clean dressing and seek
medical attention. If irritation or redness develops,
seek medical attention. Launder contaminated clothing
before reuse.





17-14

Inhalation—move away from the source of exposure
and into fresh air. If breathing stops, assisted
ventilation is required via a bag-valve-mask or
cardiopulmonary resuscitation (CPR). Once breathing
is restored, the use of additional oxygen may be
necessary. Seek medical attention immediately.
Ingestion—seek immediate medical attention. Do not
induce vomiting or take anything by mouth as this may
cause the material to enter the lungs and cause severe

lung damage. Should vomiting occur, keep head below
the hips to reduce the risks of aspiration. Monitor for
breathing difficulties. Rinse out any material which
enters the mouth until the taste is dissipated.
Dehydration and Heatstroke
Dehydration is the term given to a critical loss of water from
the body. Causes of dehydration are hot flight decks and
flight lines, wind, humidity, and diuretic drinks—coffee, tea,
alcohol, and caffeinated soft drinks. Some common signs of
dehydration are headache, fatigue, cramps, sleepiness, and
dizziness.
The first noticeable effect of dehydration is fatigue, which
in turn makes top physical and mental performance difficult,
if not impossible. Flying for long periods in hot summer
temperatures or at high altitudes increases the susceptibility
to dehydration because these conditions tend to increase the
rate of water loss from the body.
To help prevent dehydration, drink two to four quarts of
water every 24 hours. Since each person is physiologically
different, this is only a guide. Most people are aware of the
eight-glasses-a-day guide: If each glass of water is eight
ounces, this equates to 64 ounces, which is two quarts. If
this fluid is not replaced, fatigue progresses to dizziness,
weakness, nausea, tingling of hands and feet, abdominal
cramps, and extreme thirst.
The key for pilots is to be continually aware of their condition.
Most people become thirsty with a 1.5 quart deficit or a loss
of 2 percent of total body weight. This level of dehydration
triggers the "thirst mechanism." The problem is that the thirst
mechanism arrives too late and is turned off too easily. A
small amount of fluid in the mouth turns this mechanism off
and the replacement of needed body fluid is delayed.
Other steps to prevent dehydration include:


Carrying a container in order to measure daily water
intake.



Staying ahead—not relying on the thirst sensation as
an alarm. If plain water is not preferred, add some
sport drink flavoring to make it more acceptable.



Limiting daily intake of caffeine and alcohol (both are
diuretics and stimulate increased production of urine).

Heatstroke is a condition caused by any inability of the body
to control its temperature. Onset of this condition may be
recognized by the symptoms of dehydration, but also has
been known to be recognized only upon complete collapse.

To prevent these symptoms, it is recommended that an
ample supply of water be carried and used at frequent
intervals on any long flight, whether thirsty or not. The body
normally absorbs water at a rate of 1.2 to 1.5 quarts per hour.
Individuals should drink one quart per hour for severe heat
stress conditions or one pint per hour for moderate stress
conditions. If the aircraft has a canopy or roof window,
wearing light-colored, porous clothing and a hat will help
provide protection from the sun. Keeping the flight deck well
ventilated aids in dissipating excess heat.
Alcohol
Alcohol impairs the efficiency of the human body.
[Figure 17-8] Studies have shown that consuming alcohol
is closely linked to performance deterioration. Pilots must
make hundreds of decisions, some of them time-critical,
during the course of a flight. The safe outcome of any flight
depends on the ability to make the correct decisions and take
Type Beverage

Typical Serving
(oz)

Table wine
Light beer
Aperitif liquor
Champagne
Vodka
Whiskey

4.0
12.0
1.5
4.0
1.0
1.25

Pure Alcohol
Content (oz)
.48
.48
.38
.48
.50
.50

0.01–0.05\%
(10–50 mg)

average individual appears normal

0.03–0.12\%*
(30–120 mg)

mild euphoria, talkativeness, decreased
inhibitions, decreased attention, impaired
judgment, increased reaction time

0.09–0.25\%
(90–250 mg)

emotional instability, loss of critical
judgment, impairment of memory and
comprehension, decreased sensory
response, mild muscular incoordination

0.18–0.30\%
(180–300 mg)

confusion, dizziness, exaggerated
emotions (anger, fear, grief), impaired
visual perception, decreased pain
sensation, impaired balance, staggering
gait, slurred speech, moderate muscular
incoordination

0.27–0.40\%
(270–400 mg)

apathy, impaired consciousness, stupor,
significantly decreased response to
stimulation, severe muscular
incoordination, inability to stand or walk,
vomiting, incontinence of urine and feces

0.35–0.50\%
(350–500 mg)

unconsciousness, depressed or
abolished reflexes, abnormal body
temperature, coma, possible death from
respiratory paralysis (450 mg or above)

* Legal limit for motor vehicle operation in most states is 0.08
or 0.10\% (80–100 mg of alcohol per dL of blood).
Figure 17-8. Impairment scale with alcohol use.

the appropriate actions during routine occurrences, as well
as abnormal situations. The influence of alcohol drastically
reduces the chances of completing a flight without incident.
Even in small amounts, alcohol can impair judgment,
decrease sense of responsibility, affect coordination, constrict
visual field, diminish memory, reduce reasoning ability, and
lower attention span. As little as one ounce of alcohol can
decrease the speed and strength of muscular reflexes, lessen
the efficiency of eye movements while reading, and increase
the frequency at which errors are committed. Impairments
in vision and hearing can occur from consuming as little as
one drink.
The alcohol consumed in beer and mixed drinks is ethyl
alcohol, a central nervous system depressant. From a medical
point of view, it acts on the body much like a general
anesthetic. The "dose" is generally much lower and more
slowly consumed in the case of alcohol, but the basic effects
on the human body are similar. Alcohol is easily and quickly
absorbed by the digestive tract. The bloodstream absorbs
about 80 to 90 percent of the alcohol in a drink within 30
minutes when ingested on an empty stomach. The body
requires about 3 hours to rid itself of all the alcohol contained
in one mixed drink or one beer.
While experiencing a hangover, a pilot is still under the
influence of alcohol. Although a pilot may think he or she is
functioning normally, motor and mental response impairment
is still present. Considerable amounts of alcohol can remain
in the body for over 16 hours, so pilots should be cautious
about flying too soon after drinking.
Altitude multiplies the effects of alcohol on the brain. When
combined with altitude, the alcohol from two drinks may have
the same effect as three or four drinks. Alcohol interferes
with the brain's ability to utilize oxygen, producing a form
of histotoxic hypoxia. The effects are rapid because alcohol
passes quickly into the bloodstream. In addition, the brain
is a highly vascular organ that is immediately sensitive to
changes in the blood's composition. For a pilot, the lower
oxygen availability at altitude and the lower capability of
the brain to use the oxygen that is available can add up to a
deadly combination.
Intoxication is determined by the amount of alcohol in the
bloodstream. This is usually measured as a percentage by
weight in the blood. 14 CFR part 91 requires that blood
alcohol level be less than .04 percent and that 8 hours pass
between drinking alcohol and piloting an aircraft. A pilot with
a blood alcohol level of .04 percent or greater after 8 hours
cannot fly until the blood alcohol falls below that amount.
Even though blood alcohol may be well below .04 percent,
a pilot cannot fly sooner than 8 hours after drinking alcohol.
17-15

Although the regulations are quite specific, it is a good idea
to be more conservative than the regulations.

as well for cognitive impairment, and either or both could be
found unacceptable for medical certification.

Drugs
The Federal Aviation Regulations include no specific
references to medication usage. Two regulations, though,
are important to keep in mind. Title 14 of the CFR part 61,
section 61.53 prohibits acting as pilot-in-command or in any
other capacity as a required pilot flight crewmember, while
that person:

Some of the most commonly used OTC drugs, antihistamines
and decongestants, have the potential to cause noticeable
adverse side effects, including drowsiness and cognitive
deficits. The symptoms associated with common upper
respiratory infections, including the common cold, often
suppress a pilot's desire to fly, and treating symptoms with
a drug that causes adverse side effects only compounds
the problem. Particularly, medications containing
diphenhydramine (e.g., Benadryl) are known to cause
drowsiness and have a prolonged half-life, meaning the drugs
stay in one's system for an extended time, which lengthens
the time that side effects are present.

1.	 Knows or has reason to know of any medical condition
that would make the person unable to meet the
requirement for the medical certificate necessary for
the pilot operation, or
2. 	 Is taking medication or receiving other treatment for
a medical condition that results in the person being
unable to meet the requirements for the medical
certificate necessary for the pilot operation.
Further, 14 CFR part 91, section 91.17 prohibits the use
of any drug that affects the person's faculties in any way
contrary to safety.
There are several thousand medications currently approved
by the U.S. Food and Drug Administration (FDA), not
including OTC (over the counter) drugs. Virtually all
medications have the potential for adverse side effects in
some people. Additionally, herbal and dietary supplements,
sport and energy boosters, and some other "natural" products
are derived from substances often found in medications that
could also have adverse side effects. While some individuals
experience no side effects with a particular drug or product,
others may be noticeably affected. The FAA regularly
reviews FDA and other data to assure that medications found
acceptable for aviation duties do not pose an adverse safety
risk. Drugs that cause no apparent side effects on the ground
can create serious problems at even relatively low altitudes.
Even at typical general aviation altitudes, the changes in
concentrations of atmospheric gases in the blood can enhance
the effects of seemingly innocuous drugs that can result in
impaired judgment, decision-making, and performance. In
addition, fatigue, stress, dehydration, and inadequate nutrition
can increase an airman's susceptibility to adverse effects from
various drugs, even if they appeared to tolerate them in the
past. If multiple medications are being taken at the same time,
the adverse effects can be even more pronounced.
Another important consideration is that the medical
condition for which a medication is prescribed may itself be
disqualifying. The FAA will consider the condition in the
context of risk for medical incapacitation, and the medication

17-16

Many medications, such as tranquilizers, sedatives, strong
pain relievers, and cough suppressants, have primary
effects that may impair judgment, memory, alertness,
coordination, vision, and the ability to make calculations.
[Figure 17-9] Others, such as antihistamines, blood pressure
drugs, muscle relaxants, and agents to control diarrhea and
motion sickness, have side effects that may impair the same
critical functions. Any medication that depresses the nervous
system, such as a sedative, tranquilizer, or antihistamine, can
make a pilot more susceptible to hypoxia.
Painkillers are grouped into two broad categories: analgesics
and anesthetics. Analgesics are drugs that reduce pain,
while anesthetics are drugs that deaden pain or cause loss
of consciousness.
Over-the-counter analgesics, such as acetylsalicylic acid
(aspirin), acetaminophen (Tylenol), and ibuprofen (Advil),
have few side effects when taken in the correct dosage.
Although some people are allergic to certain analgesics or may
suffer from stomach irritation, flying usually is not restricted
when taking these drugs. However, flying is almost always
precluded while using prescription analgesics, such as drugs
containing propoxyphene (e.g., Darvon), oxycodone (e.g.,
Percodan), meperidine (e.g., Demerol), and codeine, since
these drugs are known to cause side effects, such as mental
confusion, dizziness, headaches, nausea, and vision problems.
Anesthetic drugs are commonly used for dental and surgical
procedures. Most local anesthetics used for minor dental and
outpatient procedures wear off within a relatively short period
of time. The anesthetic itself may not limit flying as much
as the actual procedure and subsequent pain.

Stimulants are drugs that excite the central nervous
system and produce an increase in alertness and activity.
Amphetamines, caffeine, and nicotine are all forms of
stimulants. Common uses of these drugs include appetite
suppression, fatigue reduction, and mood elevation. Some
of these drugs may cause a stimulant reaction, even though
this reaction is not their primary function. In some cases,
stimulants can produce anxiety and mood swings, both of
which are dangerous when flying.

Substance

Generic Or
Brand Name

Treatment
for

Depressants are drugs that reduce the body's functioning in
many areas. These drugs lower blood pressure, reduce mental
processing, and slow motor and reaction responses. There are
several types of drugs that can cause a depressing effect on the
body, including tranquilizers, motion sickness medication,
some types of stomach medication, decongestants, and
antihistamines. The most common depressant is alcohol.

Possible Side Effects

Alcohol

Beer
Liquor
Wine

N/A

Impaired judgment and perception
Impaired coordination and motor control
Reduced reaction time
Impaired sensory perception
Reduced intellectual functions
Reduced tolerance to G-forces
Inner-ear disturbance and spatial disorientation (up to 48 hours)
Central nervous system depression

Nicotine

Cigars
Cigarettes
Pipe tobacco
Chewing tobacco
Snuff

N/A

Sinus and respiratory system infection and irritation
Impaired night vision
Hypertension
Carbon monoxide poisoning (from smoking)

Amphetamines

Ritalin
Obetrol
Eskatrol

Obesity (diet pills)
Tiredness

Prolonged wakefulness
Nervousness
Impaired vision
Suppressed appetite
Shakiness
Excessive sweating
Rapid heart rate
Sleep disturbance
Seriously impaired judgment

Caffeine

Coffee
Tea
Chocolate
No-Doz

N/A

Impaired judgment
Reduced reaction time
Sleep disturbance
Increased motor activity and tremors
Hypertension
Irregular heart rate
Rapid heart rate
Body dehydration (through increased urine output)
Headaches

Antacid

Alka-2
Di-Gel
Maalox

Stomach acids

Liberations of carbon dioxide at altitude (distension may cause
acute abdominal pain and may mask other medical problems)

Antihistamines

Coricidin
Contac
Dristan
Dimetapp
Omade
Chlor-Trimeton
Diphenhydramine

Allergies
Colds

Drowsiness and dizziness (sometimes recurring)
Visual disturbances (when medications also contain antispasmodic drugs)

Aspirin

Bayer
Bufferin
Alka-Seltzer

Headaches
Fevers
Aches
Pains

Irregular body temperature
Variation in rate and depth of respiration
Hypoxia and hyperventilation (two aspirin can contribute to)
Nausea, ringing in ears, deafness, diarrhea, and hallucinations when taken in
excessive dosages
Corrosive action on the stomach lining
Gastrointestinal problems
Decreased clotting ability of the blood (clotting ability could be the difference
between life and death in a survival situation)

Figure 17-9. Adverse affects of various drugs.

17-17

Some drugs that are classified as neither stimulants nor
depressants have adverse effects on flying. For example,
some antibiotics can produce dangerous side effects, such
as balance disorders, hearing loss, nausea, and vomiting.
While many antibiotics are safe for use while flying, the
infection requiring the antibiotic may prohibit flying. In
addition, unless specifically prescribed by a physician, do
not take more than one drug at a time, and never mix drugs
with alcohol because the effects are often unpredictable.
The dangers of illegal drugs also are well documented.
Certain illegal drugs can have hallucinatory effects that occur
days or weeks after the drug is taken. Obviously, these drugs
have no place in the aviation community.
14 CFR prohibits pilots from performing crewmember
duties while using any medication that affects the body in
any way contrary to safety. The safest rule is not to fly as a
crewmember while taking any medication, unless approved to
do so by the FAA. If there is any doubt regarding the effects
of any medication, consult an AME before flying.
Prior to each and every flight, all pilots must do a proper
physical self-assessment to ensure safety. A great mnemonic,
covered in Chapter 2 on Aeronautical Decision-Making,
is IMSAFE, which stands for Illness, Medication, Stress,
Alcohol, Fatigue, and Emotion.
For the medication component of IMSAFE, pilots need to
ask themselves, "Am I taking any medicines that might affect
my judgment or make me drowsy? For any new medication,
OTC or prescribed, you should wait at least 48 hours after
the first dose before flying to determine you do not have any
adverse side effects that would make it unsafe to operate an
aircraft. In addition to medication questions, pilots should
also consider the following –


Do not take any unnecessary or elective medications;



Make sure you eat regular balanced meals;



Bring a snack for both you and your passengers for
the flight;



Maintain good hydration - bring plenty of water;



Ensure adequate sleep the night prior to the flight; and



Stay physically fit.

Additionally, you should wait at least five maximal dosing
intervals, the time between recommended or prescribed
dosing, (e.g., a dosing interval of 5 to 6 hours would require
you to wait 30 hours) before flying after taking any medication
that has potentially adverse side effects (e.g., sedating or
dizziness). Observing the recommended dosing interval
doesn't eliminate the risk for adverse side effects because
17-18

everyone metabolizes medications differently. However,
five times the dosing interval is a reasonable rule of thumb.
Altitude-Induced Decompression Sickness (DCS)
Decompression sickness (DCS) describes a condition
characterized by a variety of symptoms resulting from
exposure to low barometric pressures that cause inert gases
(mainly nitrogen), normally dissolved in body fluids and
tissues, to come out of physical solution and form bubbles.
Nitrogen is an inert gas normally stored throughout the
human body (tissues and fluids) in physical solution. When
the body is exposed to decreased barometric pressures (as in
flying an unpressurized aircraft to altitude or during a rapid
decompression), the nitrogen dissolved in the body comes out
of solution. If the nitrogen is forced to leave the solution too
rapidly, bubbles form in different areas of the body causing a
variety of signs and symptoms. The most common symptom
is joint pain, which is known as "the bends." [Figure 17-10]
What to do when altitude-induced DCS occurs:


Put on oxygen mask immediately and switch the
regulator to 100 percent oxygen.



Begin an emergency descent and land as soon as
possible. Even if the symptoms disappear during
descent, land and seek medical evaluation while
continuing to breathe oxygen.



If one of the symptoms is joint pain, keep the affected
area still; do not try to work pain out by moving the
joint around.



Upon landing, seek medical assistance from an FAA
medical officer, AME, military flight surgeon, or
a hyperbaric medicine specialist. Be aware that a
physician not specialized in aviation or hypobaric
medicine may not be familiar with this type of medical
problem.



Definitive medical treatment may involve the use of
a hyperbaric chamber operated by specially-trained
personnel.



Delayed signs and symptoms of altitude-induced DCS
can occur after return to ground level regardless of
presence during flight.

DCS After Scuba Diving
Scuba diving subjects the body to increased pressure, which
allows more nitrogen to dissolve in body tissues and fluids.
[Figure 17-11] The reduction of atmospheric pressure that
accompanies flying can produce physical problems for scuba
divers. A pilot or passenger who intends to fly after scuba
diving should allow the body sufficient time to rid itself of
excess nitrogen absorbed during diving. If not, DCS due to

DCS Type

Bubble Location

BENDS


Mostly large joints
of the body (elbows,
shoulders, hip, wrists,
knees, ankles)

 	 Localized deep pain, ranging from mild (a "niggle") to excruciating–sometimes a dull
ache, but rarely a sharp pain
 	 Active and passive motion of the joint aggravating the pain
 	 Pain occurring at altitude, during the descent, or many hours later

Brain

 	 Confusion or memory loss
 Headache
 	 Spots in visual field (scotoma), tunnel vision, double vision (diplopia), or blurry vision
 	 Unexplained extreme fatigue or behavior changes
 	 Seizures, dizziness, vertigo, nausea, vomiting, and unconsciousness

Spinal cord

 	 Abnormal sensations, such as burning, stinging, and tingling, around the lower chest
and back
 	 Symptoms spreading from the feet up and possibly accompanied by ascending
weakness or paralysis
 	 Girdling abdominal or chest pain

Peripheral nerves

 	 Urinary and rectal incontinence
 	 Abnormal sensations, such as numbness, burning, stinging and tingling (paresthesia)
 	 Muscle weakness or twitching

Lungs

 	 Burning deep chest pain (under the sternum)
 	 Pain aggravated by breathing
 	 Shortness of breath (dyspnea)
 	 Dry constant cough

Skin

 	 Itching usually around the ears, face, neck, arms, and upper torso
 	 Sensation of tiny insects crawling over the skin
 	 Mottled or marbled skin usually around the shoulders, upper chest, and abdomen
accompanied by itching
 	 Swelling of the skin, accompanied by tiny scar-like skin depressions (pitting edema)

NEUROLOGIC

Manifestations


CHOKES

SKIN BENDS

Signs and Symptoms (Clinical Manifestations)

Figure 17-10. Signs and symptoms of altitude decompression sickness.

evolved gas can occur during exposure to low altitude and
create a serious inflight emergency.
The recommended waiting time before going to flight
altitudes of up to 8,000 feet is at least 12 hours after diving
that does not require controlled ascent (nondecompression
stop diving), and at least 24 hours after diving that does
require controlled ascent (decompression stop diving). The
waiting time before going to flight altitudes above 8,000
feet should be at least 24 hours after any scuba dive. These
recommended altitudes are actual flight altitudes above
mean sea level (MSL) and not pressurized cabin altitudes.
This takes into consideration the risk of decompression of
the aircraft during flight.

The eye functions much like a camera. Its structure includes
an aperture, a lens, a mechanism for focusing, and a surface
for registering images. Light enters through the cornea at the
front of the eyeball, travels through the lens, and falls on the
retina. The retina contains light sensitive cells that convert

Vision in Flight
Of all the senses, vision is the most important for safe flight.
Most of the things perceived while flying are visual or heavily
supplemented by vision. As remarkable and vital as it is,
vision is subject to limitations, such as illusions and blind
spots. The more a pilot understands about the eyes and how
they function, the easier it is to use vision effectively and
compensate for potential problems.

Figure 17-11. To avoid the bends, scuba divers must not fly for

specific time periods following dives.

17-19

The rods and
cones (film) of
the retina are
the receptors
which record
the image and
transmit it
through the
optic nerve to
the brain for
interpretation.

Rods and
cones

Fovea
(all cones)

Fovea centralis

Rod
concentration

Optic disk
(blind spot)

Lens
Iris

Optic nerve
Retina
PUPIL

CORNEA

The pupil (aperture) is the opening at
the center of the iris. The size of the
pupil is adjusted to control the amount
of light entering the eye.

Light passes through the cornea (the
transparent window on the front of the
eye) and then through the lens to
focus on the retina.

Figure 17-12. The human eye.

light energy into electrical impulses that travel through nerves
to the brain. The brain interprets the electrical signals to form
images. There are two kinds of light-sensitive cells in the
eyes: rods and cones. [Figure 17-12]
The cones are responsible for all color vision, from
appreciating a glorious sunset to discerning the subtle shades
in a fine painting. Cones are present throughout the retina, but
are concentrated toward the center of the field of vision at the
back of the retina. There is a small pit called the fovea where
almost all the light sensing cells are cones. This is the area
where most "looking" occurs (the center of the visual field
where detail, color sensitivity, and resolution are highest).
While the cones and their associated nerves are well suited
to detecting fine detail and color in high light levels, the
rods are better able to detect movement and provide vision
in dim light. The rods are unable to discern color but are
very sensitive at low-light levels. The trouble with rods is

that a large amount of light overwhelms them, and they take
longer to "reset" and adapt to the dark again. There are so
many cones in the fovea that are at the very center of the
visual field but virtually has no rods at all. So in low light,
the middle of the visual field is not very sensitive, but farther
from the fovea, the rods are more numerous and provide the
major portion of night vision.
Vision Types
There are three types of vision: photopic, mesopic, and
scotopic. Each type functions under different sensory stimuli
or ambient light conditions. [Figure 17-13]

Photopic Vision
Photopic vision provides the capability for seeing color and
resolving fine detail (20/20 or better), but it functions only
in good illumination. Photopic vision is experienced during
daylight or when a high level of artificial illumination exists.

Types of Vision
Types of vision used

Light level

Technique of viewing

Color perception

Receptors used

Acuity best

Blind spot

Photopic

High

Central

Good

Cones

20/20

Day

Mesopic

Medium/Low

Both

Some

Cones/Rods

Varies

Day/Night

Scotopic

Low

Scanning

None

Rods

20/200

Day/Night

Figure 17-13. Types of vision.

17-20

The cones concentrated in the fovea centralis of the eye are
primarily responsible for vision in bright light. [Figure 17-12]
Because of the high light level, rhodopsin, which is a
biological pigment of the retina that is responsible for both
the formation of the photoreceptor cells and the first events
in the perception of light, is bleached out causing the rod
cells to become less effective.

Center of vision

Blind spot
Pupil

Mesopic Vision
Mesopic vision is achieved by a combination of rods and
cones and is experienced at dawn, dusk, and during full
moonlight. Visual acuity steadily decreases as available light
decreases and color perception changes because the cones
become less effective. Mesopic viewing period is considered
the most dangerous period for viewing. As cone sensitivity
decreases, pilots should use off-center vision and proper
scanning techniques to detect objects during low-light levels.

Scotopic Vision
Scotopic vision is experienced under low-light levels and
the cones become ineffective, resulting in poor resolution of
detail. Visual acuity decreases to 20/200 or less and enables
a person to see only objects the size of or larger than the
big "E" on visual acuity testing charts from 20 feet away.
In other words, a person must stand at 20 feet to see what
can normally be seen at 200 feet under daylight conditions.
When using scotopic vision, color perception is lost and a
night blind spot in the central field of view appears at low
light levels when the cone-cell sensitivity is lost.
Central Blind Spot
The area where the optic nerve connects to the retina in the
back of each eye is known as the optic disk. There is a total
absence of cones and rods in this area, and consequently,
each eye is completely blind in this spot. [Figure 17-14]
As a result, it is referred to as the blind spot that everyone

Retina

Optic nerve

LEFT

Right

Figure 17-14. Central blind spot.

has in each eye. Under normal binocular vision conditions
(both eyes are used together), this is not a problem because
an object cannot be in the blind spot of both eyes at the same
time. On the other hand, where the field of vision of one eye
is obstructed by an object (windshield divider or another
aircraft), a visual target could fall in the blind spot of the
other eye and remain undetected.
Figure 17-15 provides a dramatic example of the eye's
blind spot.
1.	 Hold this page at an arm's length.
2.	 Completely cover your left eye (without closing or
pressing on it) using your hand or other flat object.
3.	 With your right eye, stare directly at the airplane on
the left side of the picture page. In your periphery, you
will notice the black X on the right side of the picture.
4.	 Slowly move the page closer to you while continuing
to stare at the airplane.

Figure 17-15. The eye's blind spot.

17-21

5.	 When the page is about 16–18 inches from you, the
black X should disappear completely because it has
been imaged onto the blind spot of your right eye.
(Resist the temptation to move your right eye while
the black X is gone or else it reappears. Keep staring
at the airplane.)
6.	 As you continue to look at the airplane, keep moving
the page closer to you a few more inches, and the black
X will come back into view.
7.	 There is an interval where you are able to move the
page a few inches backward and forward, and the black
X will be gone. This demonstrates to you the extent
of your blind spot.
8.	 You can try the same thing again, except this time with
your right eye covered stare at the black X with your
left eye. Move the page in closer and the airplane will
disappear.
Another way to check your blind spot is to do a similar test
outside at night when there is a full moon. Cover your left
eye, looking at the full moon with your right eye. Gradually
move your right eye to the left (and maybe slightly up or
down). Before long, all you will be able to see is the large
halo around the full moon; the entire moon itself will seem
to have disappeared.

Figure 17-16. Night vision.

expose the rods to the image. This can be done by looking 5°
to 10° off center of the object to be seen. This can be tried in
a dim light in a darkened room. When looking directly at the
light, it dims or disappears altogether. When looking slightly
off center, it becomes clearer and brighter.
When looking directly at an object, the image is focused
mainly on the fovea, where detail is best seen. At night, the
ability to see an object in the center of the visual field is
reduced as the cones lose much of their sensitivity and the
rods become more sensitive. Looking off center can help
compensate for this night blind spot. Along with the loss of

Empty-Field Myopia
Empty-field myopia is a condition that usually occurs when
flying above the clouds or in a haze layer that provides
nothing specific to focus on outside the aircraft. This causes
the eyes to relax and seek a comfortable focal distance that
may range from 10 to 30 feet. For the pilot, this means
looking without seeing, which is dangerous. Searching out
and focusing on distant light sources, no matter how dim,
helps prevent the onset of empty-field myopia.
Night Vision
There are many good reasons to fly at night, but pilots must
keep in mind that the risks of night flying are different than
during the day and often times higher. [Figure 17-16] Pilots
who are cautious and educated on night-flying techniques
can mitigate those risks and become very comfortable and
proficient in the task.

Cones active

Night blind spot

Night Blind Spot
It is estimated that once fully adapted to darkness, the rods are
10,000 times more sensitive to light than the cones, making
them the primary receptors for night vision. Since the cones
are concentrated near the fovea, the rods are also responsible
for much of the peripheral vision. The concentration of cones
in the fovea can make a night blind spot in the center of the
field of vision. To see an object clearly at night, the pilot must
17-22

Rods active
Pilots must look 5°–10° off center of the
object in order for the object to be seen.

Figure 17-17. Night blind spot.

sharpness (acuity) and color at night, depth perception and
judgment of size may be lost. [Figure 17-17]

moving from one viewing point to the next, pilots should
overlap the previous field of view by 10°. [Figure 17-18]

Dark Adaptation

Off-center viewing is another type of scan that pilots can use
during night flying. It is a technique that requires an object be
viewed by looking 10° above, below, or to either side of the
object. [Figure 17-19] In this manner, the peripheral vision
can maintain contact with an object.

Dark adaptation is the adjustment of the human eye to a dark
environment. That adjustment takes longer depending on the
amount of light in the environment that a person has just left.
Moving from a bright room into a dark one takes longer than
moving from a dim room and going into a dark one.
While the cones adapt rapidly to changes in light intensities,
the rods take much longer. Walking from bright sunlight into
a dark movie theater is an example of this dark adaptation
period experience. The rods can take approximately 30
minutes to fully adapt to darkness. A bright light, however,
can completely destroy night adaptation, leaving night
vision severely compromised while the adaptation process
is repeated.

Scanning Techniques
Scanning techniques are very important in identifying objects
at night. To scan effectively, pilots must look from right to
left or left to right. They should begin scanning at the greatest
distance an object can be perceived (top) and move inward
toward the position of the aircraft (bottom). For each stop, an
area approximately 30° wide should be scanned. The duration
of each stop is based on the degree of detail that is required,
but no stop should last longer than 2 to 3 seconds. When

With off-center vision, the images of an object viewed longer
than 2 to 3 seconds will disappear. This occurs because the
rods reach a photochemical equilibrium that prevents any
further response until the scene changes. This produces
a potentially unsafe operating condition. To overcome
this night vision limitation, pilots must be aware of the
phenomenon and avoid viewing an object for longer than 2
or 3 seconds. The peripheral field of vision will continue to
pick up the object when the eyes are shifted from one offcenter point to another.

Night Vision Protection
Several things can be done to help with the dark adaptation
process and to keep the eyes adapted to darkness. Some of
the steps pilots and flight crews can take to protect their night
vision are described in the following paragraphs.

10°
1

2

4

3

Figure 17-18. Scanning techniques.

17-23

Focal points

X
10 degrees

X

X

10 degrees

X
OBSERVER

10 degrees

Once an object is detected in the peripheral field of darkadapted vision, continued surveillance is maintained by
use of "off-center" vision. Looking 10° right or left and
above and below the object, viewing no longer than 2 to
3 seconds at each position.
X
10 degrees
Figure 17-19. Off-center viewing.

Sunglasses
If a night flight is scheduled, pilots and crew members should
wear neutral density (N-15) sunglasses or equivalent filter
lenses when exposed to bright sunlight. This precaution
increases the rate of dark adaptation at night and improves
night visual sensitivity.

High Intensity Lighting
If, during the flight, any high intensity lighting areas are
encountered, attempt to turn the aircraft away and fly in the
periphery of the lighted area. This will not expose the eyes to
such a large amount of light all at once. If possible, plan your
route to avoid direct over flight of built-up, brightly lit areas.

Oxygen Supply
Unaided night vision depends on optimum function and
sensitivity of the rods of the retina. Lack of oxygen to the rods
(hypoxia) significantly reduces their sensitivity. Sharp clear
vision (with the best being equal to 20–20 vision) requires
significant oxygen especially at night. Without supplemental
oxygen, an individual's night vision declines measurably at
pressure altitudes above 4,000 feet. As altitude increases,
the available oxygen decreases, degrading night vision.
Compounding the problem is fatigue, which minimizes
physiological well being. Adding fatigue to high altitude
exposure is a recipe for disaster. In fact, if flying at night at
an altitude of 12,000 feet, the pilot may actually see elements
of his or her normal vision missing or not in focus. Missing
visual elements resemble the missing pixels in a digital image
while unfocused vision is dim and washed out.

Flightdeck Lighting
Flightdeck lighting should be kept as low as possible so that
the light does not monopolize night vision. After reaching
the desired flight altitude, pilots should allow time to
adjust to the flight conditions. This includes readjustment
of instrument lights and orientation to outside references.
During the adjustment period, night vision should continue
to improve until optimum night adaptation is achieved. When
it is necessary to read maps, charts, and checklists, use a dim
white light flashlight and avoid shining it in your or any other
crewmember's eyes.

For the pilot suffering the effects of hypoxic hypoxia, a simple
descent to a lower altitude may not be sufficient to reestablish
vision. For example, a climb from 8,000 feet to 12,000 feet for
30 minutes does not mean a descent to 8,000 feet will rectify
the problem. Visual acuity may not be regained for over an
hour. Thus, it is important to remember, altitude and fatigue
have a profound effect on a pilot's ability to see.

17-24

Airfield Precautions
Often time, pilots have no say in how airfield operations are
handled, but listed below are some precautions that can be
taken to make night flying safer and help protect night vision.


Airfield lighting should be reduced to the lowest
usable intensity.



Maintenance personnel should practice light discipline
with headlights and flashlights.



Position the aircraft at a part of the airfield where the
least amount of lighting exists.



Self-Imposed Stress
Night flight can be more fatiguing and stressful than day
flight, and many self- imposed stressors can limit night vision.
Pilots can control this type of stress by knowing the factors
that can cause self-imposed stressors. Some of these factors
are listed in the following paragraphs. [Figure 17-20]
Drugs
Drugs can seriously degrade visual acuity during the day and
especially at night. Pilots who become ill should consult an
aviation medical examiner (AME) or flight surgeon as to
which drugs are appropriate to take while flying.
Exhaustion
Pilots who become fatigued during a night flight will not be
mentally alert and will respond more slowly to situations
requiring immediate action. Exhausted pilots tend to
concentrate on one aspect of a situation without considering
the total requirement. Their performance may become a
safety hazard depending on the degree of fatigue and instead
of using proper scanning techniques may get fixated on the
instruments or stare off rather than multitask.
Poor Physical Conditioning
To overcome poor physical conditioning, pilots should
participate in regular exercise programs. People who are
physically fit become less fatigued during flight and have
better night scanning efficiency. However, too much exercise
in a given day may leave crew members too fatigued for
night flying.
Alcohol
Alcohol is a sedative and its use impairs both coordination
and judgment. As a result, pilots who are impaired by alcohol
fail to apply the proper techniques of night vision. They are
likely to stare at objects and to neglect scanning techniques.
The amount of alcohol consumed determines the degree to
which night vision is affected. The effects of alcohol are long
lasting and the residual effects of alcohol can also impair
visual scanning efficiency.
Tobacco
Of all the self-imposed stressors, cigarette smoking most
decreases visual sensitivity at night. Smoking significantly
increases the amount of carbon monoxide carried by the
hemoglobin in red blood cells. This reduces the blood's
capacity to combine with oxygen, so less oxygen is carried
in the blood. Hypoxia caused by carbon monoxide poisoning

SS
UUGG
DDRR

Select approach and departure routes that avoid
highways and residential areas where illumination
can impair night vision.

IIAA
M
M
E
E
CC
LLYY
G
G
PPOO
Y
Y
HH
ALL
NA
ON
RIITTIIO
UTTR
N
NU
Y
C
C
N
EN Y
CIIE
EFFIIC
D
DE

EXHAUSTION
ALL
CA
IIC
S
S
Y
Y
N
H
H
ON
P D
LP
IITTIIO
R
D
OR
N
N
OO
O
PO
O
P
C
CO
CCO
C
C
AA ALCOHOL
ALCOHOL
OBB
TTO
Figure 17-20. Self-imposed stress.

affects peripheral vision and dark adaptation. The results
are the same as those for hypoxia caused by high altitude.
Smoking 3 cigarettes in rapid succession or 20 to 30 cigarettes
within a 24-hour period may saturate from 8 to 10 percent
of the capacity of hemoglobin. Smokers lose 20 percent of
their night vision capability at sea level, which is equal to a
physiological altitude of 5,000 feet.
Hypoglycemia and Nutritional Deficiency
Missing or postponing meals can cause low blood sugar,
which impairs night flight performance. Low blood sugar
levels may result in stomach contractions, distraction,
breakdown in habit pattern, and a shortened attention span.
Likewise, an insufficient consumption of vitamin A may
also impair night vision. Foods high in vitamin A include
eggs, butter, cheese, liver, apricots, peaches, carrots, squash,
spinach, peas, and most types of greens. High quantities of
vitamin A do not increase night vision but a lack of vitamin
A certainly impairs it.

Distance Estimation and Depth Perception
Knowledge of the mechanisms and cues affecting distance
estimation and depth perception assist pilots in judging
distances at night. These cues may be monocular or binocular.
The monocular cues that aid in distance estimation and depth
perception include motion parallax, geometric perspective,
retinal image size, and aerial perspective.
Motion Parallax
Motion parallax refers to the apparent motion of stationary
objects as viewed by an observer moving across the
landscape. When the pilot or crewmember looks outside the
aircraft perpendicular to the direction of travel, near objects
appear to move backward, past, or opposite the path of
motion; far objects seem to move in the direction of motion
or remain fixed. The rate of apparent movement depends on
the distance the observer is from the object.
17-25

Geometric Perspective
An object may appear to have a different shape when viewed
at varying distances and from different angles. Geometric
perspective cues include linear perspective, apparent
foreshortening, and vertical position in the field.


Linear perspective—parallel lines, such as runway
lights, power lines and railroad tracks, tend to
converge as distance from the observer increases.
[Figure 17-21A]



Apparent foreshortening—the true shape of an object
or a terrain feature appears elliptical when viewed
from a distance. [Figure 17-21B]



Vertical position in the field—objects or terrain
features farther away from the observer appear higher
on the horizon than those closer to the observer.
[Figure 17-21C]

Aerial Perspective
The clarity of an object and the shadow cast by it are
perceived by the brain and are cues for estimating distance.
Subtle variations in color or shade are clearer the closer the
observer is to an object. However, as distance increases,
these distinctions may become blurry. The same applies to
an object detail or texture. As a person gets farther from an
object, its discrete details become less apparent. Another
important fact to remember while flying at night is that every
object casts a shadow from a light source. The direction in
which the shadow is cast depends on the position of the light
source. If the shadow of an object is cast toward the observer,
the object is closer than the light source is to the observer.

make an obvious difference in the viewing angle of both eyes.
In the flight environment, most distances outside the cockpit
are so great that binocular cues are of little, if any, value. In
addition, binocular cues operate on a more subconscious
level than monocular cues and are performed automatically.
Night Vision Illusions
There are many different types of visual illusions that
commonly occur at night. Anticipating and maintaining
awareness of them is usually the best way to avoid them.

Autokinesis
Autokinesis is caused by staring at a single point of light
against a dark background for more than a few seconds.
After a few moments, the light appears to move on its own.
Apparent movement of the light source will begin in about
8 to 10 seconds. To prevent this illusion, focus the eyes on
objects at varying distances and avoid fixating on one source
of light. This illusion can be eliminated or reduced by visual
scanning, by increasing the number of lights, or by varying
the light intensity. The most important of the three solutions
is visual scanning. A light or lights should not be stared at
for more than 10 seconds.

False Horizon
A false horizon can occur when the natural horizon is
obscured or not readily apparent. It can be generated by
confusing bright stars and city lights. It can also occur while
flying toward the shore of an ocean or a large lake. Because
of the relative darkness of the water, the lights along the
shoreline can be mistaken for stars in the sky. [Figure 17-22]

Binocular Cues

Reversible Perspective Illusion

Binocular cues of an object are dependent upon the slightly
different viewing angle of each eye of an object. Binocular
perception is useful only when the object is close enough to

At night, an aircraft may appear to be moving away from
a second aircraft when it is, in fact, approaching a second
aircraft. This illusion often occurs when an aircraft is flying

A

Figure 17-21. Geometric perspective.

17-26

B

C

on

al

ctu

A

iz
or

h

Apparent horizon

Figure 17-22. At night, the horizon may be hard to discern due to dark terrain and misleading light patterns on the ground.

parallel to another's course. To determine the direction of
flight, pilots should observe aircraft lights and their relative
position to the horizon. If the intensity of the lights increases,
the aircraft is approaching; if the lights dim, the aircraft is
moving away.

Size-Distance Illusion
This illusion results from viewing a source of light that is
increasing or decreasing in luminance (brightness). Pilots
may interpret the light as approaching or retreating.

Fascination (Fixation)
This illusion occurs when pilots ignore orientation cues and
fix their attention on a goal or an object. Student pilots tend to
have this happen when they are concentrating on the aircraft
instruments or attempting to land. They become fixated on
one task and forget to look at what is going on around them.
At night, this can be especially dangerous because aircraft
ground-closure rates are difficult to determine, and there may
be minimal time to correct the situation.

Flicker Vertigo
A light flickering at a rate between 4 and 20 cycles per
second can produce unpleasant and dangerous reactions. Such
conditions as nausea, vomiting, and vertigo may occur. On
rare occasions, convulsions and unconsciousness may also
occur. Proper scanning techniques at night can prevent pilots
from getting flicker vertigo.
Night Landing Illusions
Landing illusions occur in many forms. Above featureless
terrain at night, there is a natural tendency to fly a lower­
than-normal approach. Elements that cause any type of
visual obscurities, such as rain, haze, or a dark runway
environment, can also cause low approaches. Bright lights,

steep surrounding terrain, and a wide runway can produce the
illusion of being too low with a tendency to fly a higher-than­
normal approach. A set of regularly spaced lights along a road
or highway can appear to be runway lights. Pilots have even
mistaken the lights on moving trains as runway or approach
lights. Bright runway or approach lighting systems can create
the illusion that the aircraft is closer to the runway, especially
where few lights illuminate the surrounding terrain.
Prior to flying at night, it is best to learn and know the
challenges of the area in which you are flying in. Study the
area and know how to navigate your way through areas that
may pose a problem at night. For example, many areas near
water may be obscured by low lying clouds or fog. To help
deal with this type of situation, it is important to have a plan
before you leave the ground. In the daytime, fly the routes
and passes that you will be flying at night and determine the
minimum altitude you are willing to use at night. If weather
prevents you from maintaining the altitude that you planned,
make a decision early to turn 180° and land at an alternate
airport with better weather conditions. Always consider
safer alternatives rather than hope things will work out by
taking a chance.
Pilots who fly at night should strongly consider oxygen
supplementation at altitudes and times not required by the
FAA, especially at night when critical judgment and hand-eye
coordination is necessary (e.g., IFR) or if he/she is a smoker
or not perfectly healthy.
Enhanced Night Vision Systems
Synthetic Vision Systems (SVS) and Enhanced Flight Vision
Systems (EFVS) are two systems that can improve the safety
of flight at night. The technology of both is evolving rapidly
and being used more and more. [Figure 17-23]

17-27

Synthetic vision system

Enhanced vision system

Figure 17-23. Synthetic and enhanced vision systems.

Synthetic Vision System
A Synthetic Vision System (SVS) is an electronic means
to display a synthetic vision image of the external scene
topography to the flight crew. [Figure 17-24] It is not a
real-time image like that produced by an EFVS. Unlike
EFVS, SVS requires a terrain and obstacle database, a
precise navigation solution, and a display. The terrain
image is based on the use of data from a Digital Elevation
Model (DEM) that is stored within the SVS. With SVS, the
synthetic terrain/vision image is intended to enhance pilot
awareness of spatial position relative to important features
in all visibility conditions. This is particularly useful during
critical phases of flight, such as takeoff, approach, and
landing, where important features, such as terrain, obstacles,
runways, and landmarks, may be depicted on the SVS
display. [Figure 17-25] During approach operations, the

obvious advantages of SVS are that the digital terrain image
remains on the pilot's display regardless of how poor the
visibility is outside.
An SVS image can be displayed on either a head-down
display or head-up display (HUD); however, to date, SVS
has only been certified on head-down displays. Development
efforts to display a synthetic image on a HUD are currently
underway as are efforts that would combine SVS with a realtime sensor image produced by an EFVS. These systems are
known as Combined Vision Systems. While SVS is currently
certified as an aid to situation awareness only, the FAA
and aviation industry are working on defining operational
concepts and airworthiness criteria that would enable SVS
to be used for operational credit in certain low visibility
conditions. Other future enhancements to SVS displays could
include integrating ADS-B to display traffic information.

Enhanced Flight Vision System
Enhanced Vision (EV) or Enhanced Flight Vision System
(EFVS) is an electronic means to provide a display of

Figure 17-24. SVS system.

17-28

Figure 17-25. Night time SVS system.

the external scene by use of an imaging sensor, such as
a Forward-Looking InfraRed (FLIR) or millimeter wave
radar (MMWR). In 2004, 14 CFR part 91, section 91.175
was amended to reflect that operators conducting straightin instrument approach procedures (in other than Category
II or Category III operations) may now operate below the
published decision height (DH) or minimum descent altitude
(MDA) when using an approved EFVS shown on the pilot's
HUD. This rule change provides "operational credit" for EV
equipage. No such credit exists for SV.

Chapter Summary
This chapter provides an introduction to aeromedical factors
relating to flight activities. More detailed information on
the subjects discussed in this chapter is available in the
Aeronautical Information Manual (AIM) and online at \url{
faa.gov}.

17-29

17-30


Appendix A

Performance Data for Cessna
Model 172R and Challenger 605
Short Field Takeoff Distance at 2,450 Pounds for a Cessna Model 172R

A-1

Time, Fuel, and Distance to Climb at 2,450 Pounds for a Cessna Model 172R

A-2

Cruise Performance for a Cessna Model 172R

A-3

Short Field Landing Distance at 2,450 Pounds for a Cessna Model 172R

A-4

Challenger 605 Range/Payload Profile
Fuel
Burn (lb)

Takeoff Field Length (feet)
Gross
Takeoff
Weight (lb)

SL
ISA

5,000 ft
ISA +20°C

5,840

9,400

4,940

7,755

46,000

4,219

6,432

40,000

3,190

6,570

10,230

14,200

18,105

Time (hour) 2:00

4:00

6:00

8:00

9:50

50,000

3,600
3,465
3,401

5,234
4,804

35,000

Conditions: 26,985 lb BOW, M 0.74
cruise speed, ISA, zero wind, NBAA
IFR reserves (200 NM)

Max
Payload
3,000 lb
Payload
1,000 lb
Payload

4,535

Note: Fuel burn figures provided on
top of graph are based on 1,000 lb
payload performance computations.

Zero Payload

30,000
0

500

1,000

1,500

2,000

2,500

3,000

3,500

4,000

4,500

Range (NM)

A-5

Challenger 605 Time and Fuel Versus Distance
CHALLENGER 605 TIME AND FUEL VERSUS DISTANCE
4,500
M0.80 Cruise Speed
M0.74 Cruise Speed

4,000

3,512 NM
16,820 lb

3,500

Distance (NM)

3,000

2,424 NM
10,230 lb

1,685 NM
7,570 lb

2,000
1,500

1,577 NM
6,570 lb

771 NM
3,550 lb

1,000

Conditions: 26,985 lb BOW, 1,000
lb payload, ISA, zero wind, NBAA
IFR reserves (200 NM)

730 NM
3,190 lb

500

4,045 NM
18,105 lb

3,272 NM
14,200 lb

2,599 NM
11,980 lb

2,500

3,700 NM
18,105 lb

0
0:00

1:00

2:00

3:00

M0.80 Cruise Speed Time
Distance (NM)
Fuel (lb)
M0.74 Cruise Speed Time
Distance (NM)
Fuel (lb)

4:00

5:00
6:00
Time (hour)
0:00 2:00
0
771
0 3,550
0:00 2:00
0
730
0 3,190

7:00

8:00

9:00

10:00

4:00
6:00
8:00
8:25
1,685 2,599 3,512 3,701
7,570 11,980 16,820 18,105
4:00
6:00
8:00
9:50
1,577 2,424 3,272 4,045
6,570 10,230 14,200 18,105

Conditions: 1,000 lb payload, ISA, zero wind, NBAA IFR reserves (200 NM alternate), 26,985 lb BOW
Note: All Challenger 605 performance data are for discussion purposes only. By this document, Bombardier Inc.,
does not intend to make, and is not making, any offer, commitment, representation or warranty of any kind whatsoever.
All data are subject to change without prior notice.

A-6

11:00

Challenger 605 Time and Fuel Versus Distance
CHALLENGER 605 SPECIFIC RANGE
0.240
0.230

Specific Range (NM/lb)

0.220

FL 390

0.210

FL 370
FL 350

M 0.74
Cruise Speed

0.200

FL 330

0.190

M 0.80 Cruise Speed

FL 310

0.180

M 0.82 Cruise Speed

0.170
0.160

Conditions: 40,000 lb mid-cruise weight, zero wind, ISA

0.150
420

430

440

Plotting of constant FL lines
Flight Level

450
460
Speed (KTAS)


470

M0.82

M0.80 M0.74

310 Speed
Spc Range

481
0.165

469
0.178

434
0.199

330 Speed
Spc Range
350 Speed
Spc Range

477
0.174
473
0.181

465
0.188
461
0.197

430
0.208
427
0.216

370 Speed
Spc Range

470
0.185

459
0.204

424
0.222

459
0.205

424
0.223

480

490


290 Speed
Spc Range

390 Speed
Spc Range

Plotting of Long Range Cruise and High Speed Cruise lines
M0.74 "X"
M0.74 "Y"

FL290

FL310
434
0.199

FL330
430
0.208

FL350
427
0.216

FL370
424
0.222

FL390
424
0.223

M0.80 "X"
M0.80 "Y"

469
0.178

465
0.188

461
0.197

459
0.204

459
0.205

M0.82 "X"
M0.82 "Y"

481
0.165

477
0.174

473
0.181

470
0.185

Note: Based on 40,000 lb mid-cruise weight, ISA Conditions, zero wind
Note: All Challenger 605 performance data are for discussion purposes only. By this document, Bombardier Inc.,
does not intend to make, and is not making, any offer, commitment, representation or warranty of any kind whatsoever.
All data are subject to change without prior notice.

A-7

A-8


Appendix B


Acronyms, Abbreviations, and
NOTAM Contractions
This is a list of common acronyms and abbreviations used in the aviation industry as well as NOTAM contractions. For
a more complete list of contractions used in aviation, see FAA Order JO 7340.2 (as amended). Additional information
regarding NOTAMs can be found at pilotweb.nas.faa.gov/PilotWeb/.

A
A/C—aircraft
A/FD—airport/facility directory
A/G—air to ground
A/HA—altitude/height
AAF—Army Air Field
AAI —arrival aircraft interval
AAP—advanced automation program
AAR—airport acceptance rate
ABDIS—Automated Data Interchange System Service B
ABN—aerodrome beacon
ABV—above
ACAIS—air carrier activity information system
ACAS—aircraft collision avoidance system
ACC—area control center; Airports Consultants Council
ACCT—accounting records
ACCUM—accumulate
ACD—Automatic Call Distributor
ACDO—Air Carrier District Office
ACF—Area Control Facility
ACFO—Aircraft Certification Field Office
ACFT—aircraft
ACID—aircraft identification
ACI-NA—Airports Council International-North America
ACIP—airport capital improvement plan
ACLS—automatic carrier landing system
ACLT—actual landing time calculated
ACO—Office of Airports Compliance and Field Operations;
Aircraft Certification Office
ACR—air carrier
ACRP—Airport Cooperative Research Program
ACS—Airman Certification Standard
ACT—active, activated, or activity
ADA—air defense area
ADAP—Airport Development Aid Program
ADAS—AWOS data acquisition system
ADCCP—advanced data communications control procedure
ADDA—administrative data
ADF—automatic direction finding
ADI—automatic de-ice and inhibitor

ADIN—AUTODIN service
ADIZ—air defense identification zone
ADJ—adjacent
ADL—aeronautical data-link
ADLY—arrival delay
ADO—airline dispatch office
ADP—automated data processing
ADS—automatic dependent surveillance
ADSIM —airfield delay simulation model
ADSY—administrative equipment systems
ADTN—Administrative Data Transmission Network
ADTN2000—Administrative Data Transmission Network
2000
ADVO—administrative voice
ADZD—advised
AEG—Aircraft Evaluation Group
AERA—automated en route air traffic control
AEX—automated execution
AF—airway facilities
AFB—Air Force Base
AFIS—automated flight inspection system
AFP—area flight plan
AFRES—Air Force Reserve Station
AFS—airways facilities sector
AFSFO—AFS field office
AFSFU—AFS field unit
AFSOU—AFS field office unit (standard is AFSFOU)
AFSS—automated flight service station
AFTN—Automated Fixed Telecommunications Network
AGIS—airports geographic information system
AGL—above ground level
AID—airport information desk
AIG—Airbus Industries Group
AIM—Airman's Information Manual
AlP—airport improvement plan
AIRMET—Airmen's Meteorological Information
AIRNET—Airport Network Simulation Model
AIS—aeronautical lnformation service
AlT—automated information transfer
ALP—airport layout plan
B-1

ALS—approach light system
ALSFl—ALS with sequenced flashers I
ALSF2—ALS with sequenced flashers II
ALSIP—Approach Lighting System Improvement Plan
ALSTG—altimeter setting
ALT—altitude
ALTM—altimeter
ALTN—alternate
ALTNLY—alternately
ALTRV—altitude reservation
AMASS—airport movement area safety system
AMCC—ADF/ARTCC Maintenance Control Center
AMDT—amendment
AMGR—Airport Manager
AMOS—Automatic meteorological observing system
AMP—ARINC Message Processor; Airport Master Plan
AMVER—automated mutual assistance vessel rescue system
ANC—alternate network connectivity
ANCA—Airport Noise and Capacity Act
ANG—Air National Guard
ANGB—Air National Guard Base
ANMS—automated network monitoring system
ANSI—American National Standards Group
AOA—air operations area
AP—airport; acquisition plan
APCH—approach
APL—airport lights
APP—approach; approach control; Approach Control Office
APS—airport planning standard
AQAFO—Aeronautical Quality Assurance Field Office
ARAC—Army Radar Approach Control (AAF); Aviation
Rulemaking Advisory Committee
ARCTR—FAA Aeronautical Center or Academy
ARF—airport reservation function
ARFF—aircraft rescue and fire fighting
ARINC—Aeronautical Radio, Inc.
ARLNO—Airline Office
ARO—Airport Reservation Office
ARP—airport reference point
ARR—arrive; arrival
ARRA—American Recovery and Reinvestment Act of 2009
ARSA—airport service radar area
ARSR—air route surveillance radar
ARTCC—air route traffic control center
ARTS—automated radar terminal system
ASAS—aviation safety analysis system
ASC—AUTODIN switching center
ASCP—Aviation System Capacity Plan
ASD—aircraft situation display
ASDA—accelerate-stop distance available
ASLAR—aircraft surge launch and recovery
ASM—available seat mile
ASOS—automated surface observing system
B-2

ASP—arrival sequencing program
ASPH—asphalt
ASQP—airline service quality performance
ASR—airport surveillance radar
ASTA—airport surface traffic automation
ASV—airline schedule vendor
AT—air traffic
ATA—Air Transport Association of America
ATAS—airspace and traffic advisory service
ATC—air traffic control
ATCAA—air traffic control assigned airspace
ATCBI—air traffic control beacon indicator
ATCCC—Air Traffic Control Command Center
ATCO—Air Taxi Commercial Operator
ATCRB—air traffic control radar beacon
ATCRBS—air traffic control radar beacon system
ATCSCC—Air Traffic Control System Command Center
ATCT—airport traffic control tower
ATIS—automatic terminal information service
ATISR—ATIS recorder
ATM—air traffic management; asynchronous transfer mode
ATMS—advanced traffic management system
ATN—Aeronautical Telecommunications Network
ATODN—AUTODIN terminal (FUS)
ATOMS—air traffic operations management system
ATOVN—AUOTVON (facility)
ATS—air traffic service
ATSCCP—ATS contingency command post
AUTH—authority
AUTOB—automatic weather reporting system
AUTODIN—DoD Automatic Digital Network
AUTOVON—DoD Automatic Voice Network
AVBL—available
AVN—Aviation Standards National Field Office, Oklahoma
City
AVON—AUTOVON service
AWlS—airport weather information
AWOS—automatic weather; observing/reporting system
AWP—Aviation Weather Processor
AWPG—aviation weather products generator
AWS—air weather station
AWY—airway
AZM—azimuth

B
BA FAIR—braking action fair
BA NIL—braking action nil
BA POOR—braking action poor
BANS—BRITE alphanumeric system
BART—billing analysis reporting tool (GSA software tool)
BASIC—basic contract observing station
BASOP—military base operations

BC—back course
BCA—benefit/cost analysis
BCN—beacon
BCR—benefit/cost ratio
BDAT—digitized beacon data
BERM—snowbank(s) containing earth/gravel
BLW—below
BMP—best management practices
BND—bound
BOC—Bell Operating Company
bps—bits per second
BRG—bearing
BRI—basic rate interface
BRITE—bright radar indicator terminal equipment
BRL—building restriction line
BUEC—back-up emergency communications
BUECE—back-up emergency communications equipment
BYD—beyond

C
C/S/S/N—capacity/safety/security/noise
CAA—civil aviation authority; Clean Air Act
CAAS—Class A Airspace
CAB—civil aeronautics board
CARF—Central Altitude Reservation Facility
CASFO—Civil Aviation Security Office
CAT—category; clear-air turbulence
CAU—Crypto Ancillary Unit
CBAS—Class B airspace
CBI—computer based instruction
CBSA—Class B surface area
CC\&O—customer cost and obligation
CCAS—Class C Airspace
CCC—Communications Command Center
CCCC—staff communications
CCCH—central computer complex host
CCLKWS—counterclockwise
CCS7-NI—Communication Channel Signal-7-Network
Interconnect
CCSA—Class C surface area
CCSD—Command Communications Service Designator
CCU—Central Control Unit
CD—clearance delivery; common digitizer
CDAS—Class D Airspace
CDR—cost detail report
CDSA—Class D surface area
CDT—controlled departure time
CDTI—cockpit display of traffic information
CEAS—Class E Airspace
CENTX—central telephone exchange
CEP—capacity enhancement program
CEQ—council on environmental quality

CERAP—center radar approach control; combined center
radar approach control
CESA—Class E surface area
CFC—central flow control
CFCF—Central Flow Control Facility
CFCS—central flow control service
CFR—Code of Federal Regulations
CFWP—central flow weather processor
CFWU—central flow weather unit
CGAS—Class G Airspace; Coast Guard Air Station
CHG—change
CIG—ceiling
CK—check
CL—centerline
CLC—course line computer
CLIN—contract line item
CLKWS—clockwise
CLR—clearance, clear(s), cleared to
CLSD—closed
CLT—calculated landing time
CM—commercial service airport
CMB—climb
CMSND—commissioned
CNL—cancel
CNMPS—Canadian Minimum Navigation Performance
Specification Airspace
CNS—consolidated NOTAM system
CNSP—consolidated NOTAM system processor
CO—central office
COE—U.S. Army Corps of Engineers
COM—communications
COMCO—command communications outlet
CONC—concrete
CONUS—Continental United States
CORP—private corporation other than ARINC or MITRE
CPD—coupled
CPE—customer premise equipment
CPMIS—consolidated personnel management information
system
CRA—conflict resolution advisory
CRDA—converging runway display aid
CRS—course
CRT—cathode ray tube
CSA—communications service authorization
CSIS—centralized storm information system
CSO—customer service office
CSR—communications service request
CSS—central site system
CTA—controlled time of arrival; control area
CTA/FIR—control area/flight information region
CTAF—common traffic advisory frequency
CTAS—center-TRACON automation system

B-3

CTC—contact
CTL—control
CTMA—Center Traffic Management Advisor
CUPS—consolidated uniform payroll system
CVFR—controlled visual flight rules
CVTS—compressed video transmission service
CW—continuous wave
CWSU—Central Weather Service Unit
CWY—clearway

D
DA—direct access; decision altitude/decision height;
Descent Advisor
DABBS—DITCO automated bulletin board system
DAIR—direct altitude and identity readout
DALGT—daylight
DAR—Designated Agency Representative
DARC—direct access radar channel
dBA—decibels A-weighted
DBCRC—Defense Base Closure and Realignment
Commission
DBE—disadvantaged business enterprise
DBMS—database management system
DBRITE—digital bright radar indicator tower equipment
DCA—Defense Communications Agency
DCAA—dual call, automatic answer device
DCCU—Data Communications Control Unit
DCE—data communications equipment
DCMSND—decommissioned
DCT—direct
DDA—dedicated digital access
DDD—direct distance dialing
DDM—difference in depth of modulation
DDS—Digital Data Service
DEA—Drug Enforcement Agency
DEDS—data entry and display system
DEGS—degrees
DEIS—Draft Environmental Impact Statement
DEP—depart/departure
DEPPROC—departure procedures
DEWIZ—distance early warning identification zone
DF—direction finder
DFAX—digital facsimile
DFI—direction finding indicator
DGPS—Differential Global Positioning Satellite (System)
DH—decision height
DID—direct inward dial
DIP—drop and insert point
DIRF—direction finding
DISABLD—disabled
DIST—distance
DITCO—Defense Information Technology Contracting
Office Agency
B-4

DLA—delay or delayed
DLT—delete
DLY—daily
DME—distance measuring equipment
DME/P—precision distance measuring equipment
DMN—Data Multiplexing Network
DMSTN—demonstration
DNL—day-night equivalent sound level (also called Ldn)
DOD—direct outward dial
DoD—Department of Defense
DOI—Department of Interior
DOS—Department of State
DOT—Department of Transportation
DOTCC—Department of Transportation Computer Center
DOTS—dynamic ocean tracking system
DP—dew point temperature
DRFT—snowbank(s) caused by wind action
DSCS—digital satellite compression service
DSPLCD—displaced
DSUA—dynamic special use airspace
DTS—dedicated transmission service
DUAT—direct user access terminal
DVFR—defense visual flight rules; day visual flight rules
DVOR—doppler very high frequency omni-directional range
DYSIM—dynamic simulator

E
E—east
EA—environmental assessment
EARTS—en route automated radar tracking system
EB—eastbound
ECOM—en route communications
ECVFP—expanded charted visual flight procedures
EDCT—expedite departure path
EFC—expect further clearance
EFIS—electronic flight information systems
EIAF—expanded inward access features
EIS—environmental impact statement
ELEV—elevation
ELT—emergency locator transmitter
ELWRT—electrowriter
EMAS—engineered materials arresting system
EMPS—en route maintenance processor system
EMS—environmental management system
E-MSAW—en route automated minimum safe altitude
warning
ENAV—en route navigational aids
ENG—engine
ENRT—en route
ENTR—entire
EOF—emergency Operating Facility
EPA—Environmental Protection Agency
EPS—Engineered Performance Standards

EPSS—enhanced packet switched service
ERAD—en route broadband radar
ESEC—en route broadband secondary radar
ESF—extended superframe format
ESP—en route spacing program
ESYS—en route equipment systems
ETA—estimated time of arrival
ETE—estimated time en route
ETG—enhanced target generator
ETMS—enhanced traffic management system
ETN—Electronic Telecommunications Network
EVAS—enhanced vortex advisory system
EVCS—emergency voice communications system
EXC—except

F
F\&E—facility and equipment
FAA—Federal Aviation Administration
FAAAC—FAA aeronautical center
FAACIS—FAA communications information system
FAATC—FAA technical center
FAATSAT—FAA telecommunications satellite
FAC—facility/facilities
FAF—final approach fix
FAN—MKR fan marker
FAP—final approach point
FAPM—FTS2000 associate program manager
FAR—Federal Aviation Regulation
FAST—final approach spacing tool
FAX—facsimile equipment
FBO—fixed base operator
FBS—fall back switch
FCC—Federal Communications Commission
FCLT—freeze calculated landing time
FCOM—FSS radio voice communications
FCPU—Facility Central Processing Unit
FDAT—flight data entry and printout (FDEP) and flight
data service
FDC—flight data center
FDE—flight data entry
FDEP—flight data entry and printout
FDIO—flight data input/output
FDIOC—flight data input/output center
FDIOR—flight data input/output remote
FDM—frequency division multiplexing
FDP—flight data processing
FED—federal
FEIS—Final Environmental Impact Statement
FEP—front end processor
FFAC—from facility
FI/P—flight inspection permanent
FI/T—flight inspection temporary
FIFO—Flight Inspection Field Office

FIG—flight inspection group
FINO—Flight Inspection National Field Office
FIPS—federal information publication standard
FIR—flight information region
FIRE—fire station
FIRMR—Federal Information Resource Management
Regulation
FL—flight level
FLOWSIM—traffic flow planning simulation
FM—from
FMA—final monitor aid
FMF—facility master file
FMIS—FTS2000 management information system
FMS—flight management system
FNA—final approach
FNMS—FTS2000 network management system
FOIA—Freedom Of Information Act
FONSI—finding of no significant impact
FP—flight plan
FPM—feet per minute
FRC—request full route clearance
FREQ—frequency
FRH—fly runway heading
FRI—Friday
FRZN—frozen
FSAS—flight service automation system
FSDO—Flight Standards District Office
FSDPS—flight service data processing system
FSEP—facility/service/equipment profile
FSP—flight strip printer
FSPD—freeze speed parameter
FSS—flight service station
FSSA—flight service station automated service
FSTS—federal secure telephone service
FSYS—flight service station equipment systems
FTS—federal telecommunications system
FT—feet/foot
FTS2000—Federal Telecommunications System 2000
FUS—functional units or systems
FWCS—flight watch control station

G
GA—general aviation
GAA—general aviation activity
GAAA—general aviation activity and avionics
GADO—General Aviation District Office
GC—ground control
GCA—ground control approach
GIS—geographic information system
GNAS—general national airspace system
GNSS—global navigation satellite system
GOES—Geostationary Operational Environmental Satellite
GOESF—GOES feed point
B-5

GOEST—GOES terminal equipment
GOVT—government
GP—glide path
GPRA—Government Performance Results Act
GPS—global positioning system
GPWS—ground proximity warning system
GRADE—graphical airspace design environment
GRVL—gravel
GS—glide slope indicator
GSA—General Services Administration
GSE—ground support equipment

H
H—non-directional radio homing beacon (NDB)
HAA—height above airport
HAL—height above landing
HARS—high altitude route system
HAT—height above touchdown
HAZMAT—hazardous materials
HCAP—high capacity carriers
HDG—heading
HDME—NDB with distance measuring equipment
HDQ—FAA headquarters
HEL—helicopter
HELI—heliport
HF—high frequency
HH—NDB, 2kw or more
HI-EFAS—high altitude EFAS
HIRL—high intensity runway lights
HIWAS—Hazardous lnflight Weather Advisory Service
HLDC—high level data link control
HLDG—holding
HOL—holiday
HOV—high occupancy vehicle
HP—holding pattern
HR—hour
HSI—horizontal situation indicators
HUD—housing and urban development
HWAS—hazardous in-flight weather advisory
Hz—Hertz

I
I/AFSS—international AFSS
IA—indirect access
IAF—initial approach fix
IAP—instrument approach procedures
IAPA—instrument approach procedures automation
IBM—International Business Machines
IBP—international boundary point
IBR—intermediate bit rate
ICAO—International Civil Aviation Organization
ICSS—international communications switching systems
B-6

ID—identification
IDAT—interfacility data
IDENT—identify/identifier/identification
IF—intermediate fix
IFCP—interfacility communications processor
IFDS—interfacility data system
IFEA—in-flight emergency assistance
IFO—International Field Office
IFR—instrument flight rules
IFSS—international flight service station
ILS—instrument landing system
IM—inner marker
IMC—instrument meteorological conditions
IN—inch/inches
INBD—inbound
INDEFLY—indefinitely
INFO—information
INM—integrated noise model
INOP—inoperative
INS—inertial navigation system
INSTR—instrument
INT—intersection
INTL—international
INTST—intensity
IR—ice on runway(s)
IRMP—information resources management plan
ISDN—integrated services digital network
ISMLS—interim standard microwave landing system
ITI—interactive terminal interface
IVRS—interim voice response system
IW—inside wiring

K
Kbps—Kilobits per second
Khz—Kilohertz
KT—knots
KVDT—keyboard video display terminal

L
L—left
LAA—local airport advisory
LAAS—low altitude alert system
LABS—leased A B service
LABSC—LABS GS-200 computer
LABSR—LABS remote equipment
LABSW—LABS switch system
LAHSO—land and hold short operation
LAN—local area network
LAT—latitude
LATA—local access and transport area
LAWRS—limited aviation weather reporting station
LB—pound/pounds

LC—local control
LCF—local control facility
LCN—local communications network
LCTD—located
LDA—localizer-type directional aid; landing directional aid
LDG—landing
LDIN—lead-in lights
LEC—local exchange carrier
LF—low frequency
LGT—light or lighting
LGTD—lighted
LINCS—leased interfacility NAS C
LIRL—low intensity runway lights
LIS—logistics and inventory system
LLWAS—low level wind shear alert system
LLZ—localizer
LM—compass locator at ILS middle marker
LM/MS—low/medium frequency
LMM—locator middle marker
LO—compass locator at ILS outer marker
LOC—local; locally; location; localizer
LOCID—location identifier
LOI—letter of intent
LOM—compass locator at outer marker
LONG—longitude
LPV—lateral precision performance with vertical guidance
LRCO—limited remote communications outlet
LRNAV—long range navigation
LRR—long range radar
LSR—loose snow on runway(s)
LT—left turn

M
MAA—maximum authorized altitude
MAG—magnetic
MAINT—maintain, maintenance
MALS—medium intensity approach light system
MALSF—medium intensity approach light system with
sequenced flashers
MALSR—medium intensity approach light system with
runway alignment indicator lights
MAP—maintenance automation program; military airport
program; missed approach point; modified access pricing
MAPT—missed approach point
Mbps—megabits per second
MCA—minimum crossing altitude
MCAS—Marine Corps air station
MCC—maintenance control center
MCL—middle compass locater
MCS—maintenance and control system
MDA—minimum descent altitude
MDT—maintenance data terminal
MEA—minimum en route altitude

MED—medium
METI—meteorological information
MF—middle frequency
MFJ—modified final judgment
MFT—meter fix crossing time/slot time
MHA—minimum holding altitude
Mhg—Meghertz
MIA—minimum IFR altitudes
MIDO—Manufacturing Inspection District Office
MIN—minute
MIRL—medium intensity runway lights
MIS—Meteorological Impact Statement
MISC—miscellaneous
MISO—Manufacturing Inspection Satellite Office
MIT—miles in trail
MITRE—Mitre Corporation
MLS—microwave landing system
MM—middle marker
MMAC—Mike Monroney Aeronautical Center
MMC—maintenance monitoring console
MMS—maintenance monitoring system
MNM—minimum
MNPS—minimum navigation performance specification
MNPSA—minimum navigation performance specifications
airspace
MNT—monitor; monitoring; monitored
MOA—memorandum of agreement; military operations area
MOC—minimum obstruction clearance
MOCA—minimum obstruction clearance altitude
MODE C—altitude-encoded beacon reply; altitude reporting
mode of secondary radar
MODE S—mode select beacon system
MON—Monday
MOU—memorandum of understanding
MPO—Metropolitan Planning Organization
MPS—maintenance processor subsystem or master plan
supplement
MRA—minimum reception altitude
MRC—monthly recurring charge
MSA—minimum safe altitude; minimum sector altitude
MSAW—minimum safe altitude warning
MSG—message
MSL—mean sea level
MSN—message switching network
MTCS—modular terminal communications system
MTI—moving target indicator
MU—mu meters
MUD—mud
MUNI—municipal
MUX—multiplexor
MVA—minimum vectoring altitude
MVFR—marginal visual flight rules

B-7

N
N—north
NA—not authorized
NAAQS—national ambient air quality standards
NADA—ADIN concentrator
NADIN—National Airspace Data Interchange Network
NADSW—NADIN switches
NAILS—National Airspace Integrated Logistics Support
NAMS—NADIN IA
NAPRS—National Airspace Performance Reporting System
NAS—National Airspace System or Naval Air Station
NASDC—National Aviation Safety Data
NASP—National Airspace System Plan
NASPAC—National Airspace System Performance Analysis
Capability
NATCO—National Communications Switching Center
NAV—navigation
NAVAID—navigation aid
NAVMN—navigation monitor and control
NAWAU—National Aviation Weather Advisory Unit
NAWPF—National Aviation Weather Processing Facility
NB—northbound
NCAR—National Center for Atmospheric Research,
Boulder, CO
NCF—National Control Facility
NCIU—NEXRAD Communications Interface Unit
NCP—noise compatibility program
NCS—national communications system
NDB—non-directional radio beacon
NDNB—NADIN II
NE—northeast
NEM—noise exposure map
NEPA—National Environmental Policy Act
NEXRAD—next generation weather radar
NFAX—National Facsimile Service
NFDC—National Flight Data Center
NFIS—NAS Facilities Information System
NGT—night
NI—network interface
NICS—national interfacility communications system
NM—nautical mile(s)
NMAC—near mid-air collision
NMC—National Meteorological Center
NMCE—network monitoring and control equipment
NMCS—network monitoring and control system
NMR—nautical mile radius
NOAA—National Oceanic and Atmospheric Administration
NOC—notice of completion
NONSTD—nonstandard
NOPT—no procedure turn required
NOTAM—notice to airmen
NPDES—National pollutant discharge elimination system
NPE—non-primary airport entitlement
B-8

NPIAS—national plan of integrated airport systems
NR—number
NRC—non-recurring charge
NRCS—national radio communications systems
NSAP—National Service Assurance Plan
NSRCATN—National Strategy to Reduce Congestion on
America's Transportation Network
NSSFC—National Severe Storms Forecast Center
NSSL—National Severe Storms Laboratory, Norman, OK
NSWRH—NWS Regional Headquarters
NTAP—Notices To Airmen Publication
NTP—National Transportation Policy
NTSB—National Transportation Safety Board
NTZ—no transgression zone
NW—northwest
NWS—National Weather Service
NWSR—NWS weather excluding NXRD
NXRD—advanced weather radar system

O
OAG—official airline guide
OALT—operational acceptable level of traffic
OAW—off-airway weather station
OBSC—obscured
OBST—obstruction
ODAL—omnidirectional approach lighting system
ODAPS—oceanic display and processing station
OEP—operational evolution plan/partnership
OFA—object free area
OFDPS—offshore flight data processing system
OFT—outer fix time
OFZ—obstacle free zone
OM—outer marker
OMB—Office Of Management and Budget
ONER—Oceanic Navigational Error Report
OPLT—operational acceptable level of traffic
OPR—operate
OPS—operation
OPSW—operational switch
OPX—off premises exchange
ORD—operational readiness demonstration
ORIG—original
OTR—oceanic transition route
OTS—out of service; organized track system
OVR—over

P
PABX—private automated branch exchange
PAD—packet assembler/disassembler
PAEW—personnel and equipment working
PAM—peripheral adapter module
PAPI—precision approach path indicator
PAR—precision approach radar; preferential arrival route

PARL—parallel
PAT—pattern
PATWAS—Pilots Automatic Telephone Weather Answering
Service
PAX—passenger
PBCT—proposed boundary crossing time
PBRF—pilot briefing
PBX—private branch exchange
PCA—positive control airspace
PCL—pilot controlled lighting
PCM—pulse code modulation
PD—Pilot Deviation
PDAR—preferential arrival and departure route
PDC—pre-departure clearance; program designator code
PDN—Public Data Network
PDR—preferential departure route
PERM—permanent/permanently
PFC—passenger facility charge
PGP—planning grant program
PIC—principal interexchange carrier
PIDP—programmable indicator data processor
PIREP—pilot weather report
PJE—parachute jumping exercise
PLA—practice low approach
PLW—plow/plowed
PMS—program management system
PNR—prior notice required
POLIC—police station
POP—point of presence
POT—point of termination
PPIMS—personal property information management system
PPR—prior permission required
PR—primary commercial service airport
PREV—previous
PRI—primary rate interface
PRM—precision runway monitor
PRN—pseudo random noise
PROC—procedure
PROP—propeller
PSDN—public switched data network
PSN—packet switched network
PSR —packed snow on runway(s)
PSS—packet switched service
PSTN—public switched telephone network
PTC—presumed-to-conform
PTCHY—patchy
PTN—procedure turn
PUB—publication
PUP—principal user processor
PVC—permanent virtual circuit
PVD—plan view display
PVT—private

R
RAIL—runway alignment indicator lights
RAMOS—remote automatic meteorological observing
system
RAPCO—radar approach control (USAF)
RAPCON—radar approach control (FAA)
RATCC—Radar Air Traffic Control Center
RATCF—Radar Air Traffic Control Facility (USN)
RBC—rotating beam ceilometer
RBDPE—radar beacon data processing equipment
RBSS—Radar Bomb Scoring Squadron
RCAG—remote communications air/ground facility
RCC—Rescue Coordination Center
RCCC—Regional Communications Control Centers
RCF—Remote Communication Facility
RCIU— Remote Control Interface Unit
RCL— runway centerline; radio communications link
RCLL—runway centerline light system
RCLR—RCL repeater
RCLT—RCL terminal
RCO—remote communications outlet
RCU—remote control unit
RDAT—digitized radar data
RDP—radar data processing
RDSIM—runway delay simulation model
REC—receive/receiver
REIL—runway end identifier lights
RELCTD—relocated
REP—report
RF—radio frequency
RL—General Aviation Reliever Airport
RLLS—runway lead-in lights system
RMCC—Remote Monitor Control Center
RMCF—Remote Monitor Control Facility
RML—radio microwave link
RMLR—RML repeater
RMLT—RML terminal
RMM—remote maintenance monitoring
RMMS—remote maintenance monitoring system
RMNDR—remainder
RMS—remote monitoring subsystem
RMSC—remote monitoring subsystem concentrator
RNAV—area navigation
RNP—required navigation performance
ROD—record of decision
ROSA—report of service activity
ROT—runway occupancy time
RP—restoration priority
RPC—restoration priority code
RPG—radar processing group
RPLC—replace

B-9

RPZ—runway protection zone
RQRD—required
RRH—remote reading hygrothermometer
RRHS—remote reading hydrometer
RRL—runway remaining lights
RRWDS—remote radar weather display
RRWSS—RWDS sensor site
RSA—runway safety area
RSAT—runway safety action team
RSR—en route surveillance radar
RSS—remote speaking system
RSVN—reservation
RT—right turn; remote transmitter
RT \& BTL—radar tracking and beacon tracking level
RTAD—remote tower alphanumerics display
RTCA—Radio Technical Commission for Aeronautics
RTE—route
RTP—regional transportation plan
RTR—remote transmitter/receiver
RTRD—remote tower radar display
RTS—return to service
RUF—rough
RVR—runway visual range
RVRM—runway visual range midpoint
RVRR—runway visual range rollout
RVRT—runway visual range touchdown
RW—runway
RWDS—same as RRWDS
RWP—real-time weather processor
RWY—runway

S
S—south
S/S—sector suite
SA—sand, sanded
SAC—Strategic Air Command
SAFI—semi-automatic flight inspection
SALS—short approach lighting system
SAT—Saturday
SATCOM—satellite communications
SAWR—Supplementary Aviation Weather Reporting Station
SAWRS—Supplementary Aviation Weather Reporting
System
SB—southbound
SBGP—state block grant program
SCC—System Command Center
SCVTS—Switched Compressed Video Telecommunications
Service
SDF—simplified directional facility; simplified direction
finding; software defined network
SDIS—switched digital integrated service
SDP—service delivery point

B-10

SD-ROB—radar weather report
SDS—switched data service
SE—southeast
SEL—single event level
SELF—simplified short approach lighting system with
sequenced flashing lights
SFAR-38—Special Federal Aviation Regulation 38
SFL—sequence flashing lights
SHPO—State Historic Preservation Officer
SIC—service initiation charge
SID— standard instrument departure; station identifier
SIGMET—significant meteorological information
SIMMOD—airport and airspace simulation model
SIMUL—simultaneous
SIP—state implementation plan
SIR—packed or compacted snow and ice on runway(s)
SKED—scheduled
SLR—slush on runway(s)
SM—statute miles
SMGC—surface movement guidance and control
SMPS—sector maintenance processor subsystem
SMS—safety management system; simulation modeling
system
SN—snow
SNBNK—snowbank(s) caused by plowing
SNGL—single
SNR—signal-to-noise ratio, also: S/N
SOAR—system of airports reporting
SOC—service oversight center
SOIR—simultaneous operations on intersecting runways
SOIWR—simultaneous operations on intersecting wet
runways
SPD—speed
SRAP—sensor receiver and processor
SSALF—simplified short approach lighting system with
sequenced flashers
SSALR—simplified short approach lighting system with
runway alignment indicator lights
SSALS—simplified short approach lighting system
SSB—single side band
SSR—secondary surveillance radar
STA—straight-in approach
STAR—standard terminal arrival route
STD—standard
STMUX—statistical data multiplexer
STOL—short takeoff and landing
SUN—Sunday
SURPIC—surface picture
SVC—service
SVCA—service A
SVCB—service B
SVCC—service C

SVCO—service O

SVFB—interphone service F (B)

SVFC—interphone service F (C)

SVFD—interphone service F (D)

SVFO—interphone service F (A)

SVFR—special visual flight rules

SW—southwest

SWEPT—swept or broom/broomed


T
T—temperature
T1MUX—T1 multiplexer
TAA—terminal arrival area
TAAS—terminal advance automation system
TACAN—tactical air navigation
TACR—TACAN at VOR, TACAN only
TAF—terminal area forecast
TAR—terminal area surveillance radar
TARS—terminal automated radar service
TAS—true air speed
TATCA—terminal air traffic control automation
TAVT—terminal airspace visualization tool
TCA—traffic control airport or tower control airport;
terminal control area
TCACCIS—Transportation Coordinator Automated
Command And Control Information System
TCAS—Traffic Alert and Collision Avoidance System
TCC—DOT Transportation Computer Center
TCCC—Tower Control Computer Complex
TCE—tone control equipment
TCLT—tentative calculated landing time
TCO—Telecommunications Certification Officer
TCOM—Terminal Communications
TCS—tower communications system
TDLS—Tower Data-Link Services
TDMUX—time division data multiplexer
TDWR—terminal doppler weather radar
TDZ—touchdown zone
TDZ LG—touchdown zone lights
TELCO—telephone company
TELMS—telecommunications management system
TEMPO—temporary
TERPS—terminal instrument procedures
TFAC—to facility
TFC—traffic
TFR—temporary flight restriction
TGL—touch-and-go landings
TH—threshold
THN—thin
THR—threshold
THRU—through
THU—Thursday
TIL—until

TIMS—telecommunications information management system
TIPS—terminal information processing system
TKOF—takeoff
TL—taxilane
TM—traffic management
TM\&O—telecommunications management and operations
TMA—Traffic Management Advisor
TMC—Traffic Management Coordinator
TMC/MC—Traffic Management Coordinator/Military
Coordinator
TMCC—terminal information processing system; Traffic
Management Computer Complex
TMF—Traffic Management Facility
TML—television microwave link
TMLI—television microwave link indicator
TMLR—television microwave link repeater
TMLT—television microwave link terminal
TMP—Traffic Management Processor
TMPA—traffic management program alert
TMS—traffic management system
TMSPS—traffic management specialists
TMU—traffic management unit
TNAV—terminal navigational aids
TODA—takeoff distance available
TOF—time of flight
TOFMS—time of flight mass spectrometer
TOPS—Telecommunications Ordering And Pricing System
(GSA software tool)
TORA—take-off run available
TR—telecommunications request
TRACAB—terminal radar approach control in tower cab
TRACON—Terminal Radar Approach Control Facility
TRAD—terminal radar service
TRB—Transportation Research Board
TRML—terminal
TRNG—training
TRSN—transition
TSA—taxiway safety area; Transportation Security
Administration
TSEC—terminal secondary radar service
TSNT—transient
TSP—telecommunications service priority
TSR—telecommunications service request
TSYS—terminal equipment systems
TTMA—TRACON Traffic Management Advisor
TTY—teletype
TUE—Tuesday
TVOR—terminal VHF omnidirectional range
TW—taxiway
TWEB—transcribed weather broadcast
TWR—tower
TWY—taxiway
TY—type (FAACIS)
B-11

U

W

UAS—unmanned aircraft systems
UFN—until further notice
UHF—ultra high frequency
UNAVBL—unavailable
UNLGTD—unlighted
UNMKD—unmarked
UNMNT—unmonitored
UNREL—unreliable
UNUSBL—unusable
URA—Uniform Relocation Assistance and Real Property
Acquisition Policies Act of 1970
USAF—United States Air Force
USC—United States Code
USOC—Uniform Service Order Code

W—west
WAAS—Wide Area Augmentation System
WAN—wide area network
WB—westbound
WC—work center
WCP—Weather Communications Processor
WECO—Western Electric Company
WED—Wednesday
WEF—with effect from; effective from
WESCOM—Western Electric Satellite Communications
WI—within
WIE—with immediate effect, or effective immediately
WKDAYS—Monday through Friday
WKEND—Saturday and Sunday
WMSC—Weather Message Switching Center
WMSCR—Weather Message Switching Center Replacement
WND—wind
WPT—waypoint
WSCMO—Weather Service Contract Meteorological
Observatory
WSFO—Weather Service Forecast Office
WSMO—Weather Service Meteorological Observatory
WSO—Weather Service Office
WSR—wet snow on runway(s)
WTHR—weather
WTR—water on runway(s)
WX—weather

V
V/PD—Vehicle/pedestrian deviation
VALE—voluntary airport low emission
VASI—visual approach slope indicator
VDME—VOR with distance measuring equipment
VDP—visual descent point
VF—voice frequency
VFR—visual flight rules
VGSI—visual glide slope indicator
VHF—very high frequency
VIA—by way of
VICE—instead/versus
VIS—visibility
VLF—very low frequency
VMC—visual meteorological conditions
VNAV—visual navigational aids
VNTSC—Volpe National Transportation System Center
VOL—volume
VON—virtual on-net
VOR—VHF omnidirectional range
VOR/DME—VHF omnidirectional range/distance
measuring equipment
VORTAC—VOR and TACAN (collocated)
VOT—VOR Test Facility
VP/D—vehicle/pedestrian deviation
VRS—voice recording system
VSCS—voice switching and control system
VTA—vertex time of arrival
VTAC—VOR and TACAN (collocated)
VTOL—vertical takeoff and landing
VTS—voice telecommunications system

B-12

Appendix C


Airport Signs and Markings
Airport Signs
Type of Sign

A

4-22
26-8

B 8-APCH
C

ILS

Action or Purpose

Action or Purpose

Taxiway/Runway Hold Position:
Holding position for RWY 4-22 on TWY A.

Runway Safety Area Boundary:
Identifies exit boundary of runway
safety area.

Runway/Runway Intersection:
Identifies intersecting runways or
holding position for LAHSO operations.

ILS Critical Area Boundary:
Identifies exit boundary of ILS critical area.

Runway Approach Hold Position:
Runway approach holding position for
RWY 8 on TWY B.
ILS Critical Area Hold Position:
Holding position for the ILS critical area
on TWY C.
No Entry:
Identifies paved areas where aircraft entry is
prohibited.

B

Taxiway Location:
Identifies taxiway on which aircraft is located.

22

Runway Location:
Identifies runway on which aircraft is located.

4

Type of Sign

Runway Distance Remaining:
Provides remaining runway length in 1,000­
foot increments.

J
K
22
MIL

Taxiway Direction:
Defines direction and designation of
intersecting taxiway(s).
Runway Exit:
Defines direction and designation of exit
taxiway from runway.
Outbound Destination:
Defines directions to takeoff runway(s).
Inbound Destination:
Defines directions to destination for
arriving aircraft.
Taxiway Ending Marker:
Indicates taxiway does not continue.

A G L

Direction Sign Array:
Identifies location in conjunction with
multiple intersecting taxiways.

Figure C-1. Samples and explanations of standard airport signs.

C-1

7

1

D 15-APCH

3a

D
8

1

14

15
8

A1

9

8

A1 15
2

A

D
D 15-APCH

6

15

3a

1

9

10

12

Figure C-2. A sample runway with various possible markings and signs.

1

15

17

A A1

1

9

ILS

15

7

9

C-2

A2 15-33

4

5

A D

1

1

1

3

A

1

A2

1

13

11

2a

5

3

1

Taxiway location sign

2

Runway holding position sign at takeoff end

18

2a Runway holding position sign at other than takeoff end
3

15

Runway holding position marking

3a Holding position marking for runway approach area
4

Elevated runway guard lights

5

Surface painted runway hold position sign

6

Enhanced centerline marking (located 150' prior to
runway hold position marking)

7

Holding position sign for a runway approach area

8

Runway safety area boundary sign (located on the
backside of holding position sign)

9

Taxiway direction sign

33

-15
33

-15

10 Surface painted destination sign
11 Holding position sign for ILS critical area
12 Surface painted ILS critical area boundary marking
ILS critical area boundary sign (located on backside of

13 ILS hold sign)
14 Blast pad

Runway holding position sign and marking for Land and

15 Hold Short Operations (LAHSO)
16

2a

3

-3
15

4

A 18-36

4

1

6

3

1

A

9

2a

1

A A2

6

9

17 Outbound destination sign

4

3

A 36-18

A2

1

15-33 A2
15-33

16 Runway hold position sign for intersecting runways

6

2a
1

C-3

Airport Markings
Type of Marking

Action or Purpose
Holding Position:
Denotes entrance to a runway from a
taxiway, approach hold position on a
taxiway, or LAHSO holding position
on a runway.

NON-USEABLE

Taxi direction

ILS Critical Area Boundary:
Denotes entrance to an area to be
protected for an ILS signal.
Taxiway/Taxiway Holding Position:
Denotes location on taxiway or apron
where aircraft hold short of another
taxiway.
Non-Movement Area Boundary:
Delineates movement area under
control of ATC, from non-movement
area.

4-22

4-22

Surface Painted Holding Position:
Denotes entrance to a runway from a
taxiway.

Action or Purpose
Taxiway Edge: Solid Double Yellow Lines
Defines edge of usable, full strength taxiway. Adjoining
pavement IS NOT intended for use by aircraft.

Type of Marking
USEABLE

Taxi direction

Enhanced Taxiway Centerline:
Provides visual cue to help identify
location of a runway holding position
on a taxiway. These markings are
installed 150 feet prior to the holding
position markings.
Surface Painted Taxiway Direction:
Defines designation/direction of
intersecting taxiway(s).

T
B

Surface Painted Taxiway Location:
Identifies taxiway on which the
aircraft is located.

Figure C-3. Samples and explanations of standard airport markings.

C-4

Type of Marking

Action or Purpose
Taxiway Edge: Dashed Double Yellow Lines
Defines taxiway edge where adjoining pavement
IS USABLE, such as along an apron or ramp.

Glossary
A
14 CFR. See Title 14 of the Code of Federal Regulations.
100-hour inspection. An inspection identical in scope to
an annual inspection. Conducted every 100 hours of flight
on aircraft of under 12,500 pounds that are used to carry
passengers for hire.
Absolute accuracy. The ability to determine present position
in space independently, and is most often used by pilots.
Absolute altitude. The actual distance between an aircraft
and the terrain over which it is flying.
Absolute pressure. Pressure measured from the reference
of zero pressure, or a vacuum.
A.C. Alternating current.
Acceleration. Force involved in overcoming inertia, and
which may be defined as a change in velocity per unit of time.
Acceleration error. A magnetic compass error apparent when
the aircraft accelerates while flying on an easterly or westerly
heading, causing the compass card to rotate toward North.
Accelerate-go distance. The distance required to accelerate
to V1 with all engines at takeoff power, experience an engine
failure at V1, and continue the takeoff on the remaining
engine(s). The runway required includes the distance required
to climb to 35 feet by which time V2 speed must be attained.
Accelerate-stop distance. The distance required to accelerate
to V1 with all engines at takeoff power, experience an engine
failure at V1, and abort the takeoff and bring the airplane to
a stop using braking action only (use of thrust reversing is
not considered).
Accelerometer. A part of an inertial navigation system
(INS) that accurately measures the force of acceleration in
one direction.

ADC. See air data computer.
ADF. See automatic direction finder.
ADI. See attitude director indicator.
Adiabatic cooling. A process of cooling the air through
expansion. For example, as air moves up slope it expands with
the reduction of atmospheric pressure and cools as it expands.
Adiabatic heating. A process of heating dry air through
compression. For example, as air moves down a slope it is
compressed, which results in an increase in temperature.
Adjustable-pitch propeller. A propeller with blades whose
pitch can be adjusted on the ground with the engine not
running, but which cannot be adjusted in flight. Also referred
to as a ground adjustable propeller. Sometimes also used to
refer to constant-speed propellers that are adjustable in flight.
Adjustable stabilizer. A stabilizer that can be adjusted in
flight to trim the airplane, thereby allowing the airplane to
fly hands-off at any given airspeed.
ADM. See aeronautical decision-making.
ADS-B. See automatic dependent surveillance-broadcast.
Advection fog. Fog resulting from the movement of warm,
humid air over a cold surface.
Adverse yaw. A condition of flight in which the nose of an
airplane tends to yaw toward the outside of the turn. This is
caused by the higher induced drag on the outside wing, which
is also producing more lift. Induced drag is a by-product of
the lift associated with the outside wing.
Aerodynamics. The science of the action of air on an object,
and with the motion of air on other gases. Aerodynamics
deals with the production of lift by the aircraft, the relative
wind, and the atmosphere.

G-1

Aeronautical chart. A map used in air navigation containing
all or part of the following: topographic features, hazards and
obstructions, navigation aids, navigation routes, designated
airspace, and airports.
Aeronautical decision-making (ADM). A systematic
approach to the mental process used by pilots to consistently
determine the best course of action in response to a given
set of circumstances.
Agonic line. An irregular imaginary line across the surface of
the Earth along which the magnetic and geographic poles are
in alignment, and along which there is no magnetic variation.
Ailerons. Primary flight control surfaces mounted on the
trailing edge of an airplane wing, near the tip. Ailerons control
roll about the longitudinal axis.
Aircraft. A device that is used, or intended to be used, for
flight.
Aircraft altitude. The actual height above sea level at which
the aircraft is flying.
Aircraft approach category. A performance grouping of
aircraft based on a speed of 1.3 times the stall speed in the
landing configuration at maximum gross landing weight.
Air data computer (ADC). An aircraft computer that
receives and processes pitot pressure, static pressure, and
temperature to calculate very precise altitude, indicated
airspeed, true airspeed, and air temperature.
Airfoil. Any surface, such as a wing, propeller, rudder, or
even a trim tab, which provides aerodynamic force when it
interacts with a moving stream of air.
Air mass. An extensive body of air having fairly uniform
properties of temperature and moisture.
AIRMET. Inflight weather advisory issued as an amendment
to the area forecast, concerning weather phenomena of
operational interest to all aircraft and that is potentially
hazardous to aircraft with limited capability due to lack of
equipment, instrumentation, or pilot qualifications.
Airplane. An engine-driven, fixed-wing aircraft heavier than
air that is supported in flight by the dynamic reaction of air
against its wings.

Airplane Flight Manual (AFM). A document developed
by the airplane manufacturer and approved by the Federal
Aviation Administration (FAA). It is specific to a particular
make and model airplane by serial number and it contains
operating procedures and limitations.
Airplane Owner/Information Manual. A document
developed by the airplane manufacturer containing general
information about the make and model of an airplane. The
airplane owner's manual is not FAA approved and is not
specific to a particular serial numbered airplane. This manual
is not kept current, and therefore cannot be substituted for
the AFM/POH.
Airport diagram. The section of an instrument approach
procedure chart that shows a detailed diagram of the
airport. This diagram includes surface features and airport
configuration information.
Airport/Facility Directory (A/FD). See Chart Supplement
U.S.
Airport surface detection equipment (ASDE). Radar
equipment specifically designed to detect all principal
features and traffic on the surface of an airport, presenting the
entire image on the control tower console; used to augment
visual observation by tower personnel of aircraft and/or
vehicular movements on runways and taxiways.
Airport surveillance radar (ASR). Approach control
radar used to detect and display an aircraft's position in the
terminal area.
Airport surveillance radar approach. An instrument
approach in which ATC issues instructions for pilot
compliance based on aircraft position in relation to the final
approach course and the distance from the end of the runway
as displayed on the controller's radar scope.
Air route surveillance radar (ARSR). Air route traffic
control center (ARTCC) radar used primarily to detect
and display an aircraft's position while en route between
terminal areas.
Air route traffic control center (ARTCC). Provides ATC
service to aircraft operating on IFR flight plans within
controlled airspace and principally during the en route phase
of flight.
Airspeed. Rate of the aircraft's progress through the air.

G-2

Airspeed indicator. A differential pressure gauge that
measures the dynamic pressure of the air through which the
aircraft is flying. Displays the craft's airspeed, typically in
knots, to the pilot.

Alternate static source valve. A valve in the instrument static
air system that supplies reference air pressure to the altimeter,
airspeed indicator, and vertical speed indicator if the normal
static pickup should become clogged or iced over.

Air traffic control radar beacon system (ATCRBS).
Sometimes called secondary surveillance radar (SSR), which
utilizes a transponder in the aircraft. The ground equipment is
an interrogating unit, in which the beacon antenna is mounted
so it rotates with the surveillance antenna. The interrogating
unit transmits a coded pulse sequence that actuates the aircraft
transponder. The transponder answers the coded sequence
by transmitting a preselected coded sequence back to the
ground equipment, providing a strong return signal and
positive aircraft identification, as well as other special data.

Altimeter. A flight instrument that indicates altitude by
sensing pressure changes.

Airway. An airway is based on a centerline that extends from
one navigation aid or intersection to another navigation aid
(or through several navigation aids or intersections); used
to establish a known route for en route procedures between
terminal areas.

Ambient pressure. The pressure in the area immediately
surrounding the aircraft.

Airworthiness Certificate. A certificate issued by the FAA
to all aircraft that have been proven to meet the minimum
standards set down by the Code of Federal Regulations.

AME. See aviation medical examiner.

Airworthiness Directive. A regulatory notice sent out by
the FAA to the registered owner of an aircraft informing
the owner of a condition that prevents the aircraft from
continuing to meet its conditions for airworthiness.
Airworthiness Directives (AD notes) are to be complied with
within the required time limit, and the fact of compliance,
the date of compliance, and the method of compliance are
recorded in the aircraft's maintenance records.
Alert area. An area in which there is a high volume of pilot
training or an unusual type of aeronautical activity.
Almanac data. Information the global positioning system
(GPS) receiver can obtain from one satellite which describes
the approximate orbital positioning of all satellites in the
constellation. This information is necessary for the GPS
receiver to know what satellites to look for in the sky at a
given time.
ALS. See approach lighting system.
Alternate airport. An airport designated in an IFR flight
plan, providing a suitable destination if a landing at the
intended airport becomes inadvisable.

Altimeter setting. Station pressure (the barometric pressure
at the location the reading is taken) which has been corrected
for the height of the station above sea level.
Altitude engine. A reciprocating aircraft engine having a
rated takeoff power that is producible from sea level to an
established higher altitude.

Ambient temperature. The temperature in the area
immediately surrounding the aircraft.

Amendment status. The circulation date and revision
number of an instrument approach procedure, printed above
the procedure identification.
Ammeter. An instrument installed in series with an electrical
load used to measure the amount of current flowing through
the load.
Aneroid. The sensitive component in an altimeter or
barometer that measures the absolute pressure of the air.
It is a sealed, flat capsule made of thin disks of corrugated
metal soldered together and evacuated by pumping all of
the air out of it.
Aneroid barometer. An instrument that measures the
absolute pressure of the atmosphere by balancing the weight
of the air above it against the spring action of the aneroid.
Angle of attack. The angle of attack is the angle at which
relative wind meets an airfoil. It is the angle that is formed
by the chord of the airfoil and the direction of the relative
wind or between the chord line and the flight path. The
angle of attack changes during a flight as the pilot changes
the direction of the aircraft and is related to the amount of
lift being produced.

G-3

Angle of incidence. The acute angle formed between the
chord line of an airfoil and the longitudinal axis of the aircraft
on which it is mounted.
Anhedral. A downward slant from root to tip of an aircraft's
wing or horizontal tail surface.
Annual inspection. A complete inspection of an aircraft and
engine, required by the Code of Federal Regulations, to be
accomplished every 12 calendar months on all certificated
aircraft. Only an A\&P technician holding an Inspection
Authorization can conduct an annual inspection.

Asymmetric thrust. Also known as P-factor. A tendency for
an aircraft to yaw to the left due to the descending propeller
blade on the right producing more thrust than the ascending
blade on the left. This occurs when the aircraft's longitudinal
axis is in a climbing attitude in relation to the relative wind.
The P-factor would be to the right if the aircraft had a
counterclockwise rotating propeller.
ATC. Air Traffic Control.
ATCRBS. See air traffic control radar beacon system.
ATIS. See automatic terminal information service.

Anti-ice. Preventing the accumulation of ice on an aircraft
structure via a system designed for that purpose.
Antiservo tab. An adjustable tab attached to the trailing edge
of a stabilator that moves in the same direction as the primary
control. It is used to make the stabilator less sensitive.
Approach lighting system (ALS). Provides lights that will
penetrate the atmosphere far enough from touchdown to
give directional, distance, and glidepath information for safe
transition from instrument to visual flight.
Area chart. Part of the low-altitude en route chart series,
this chart furnishes terminal data at a larger scale for
congested areas.
Area forecast (FA). A report that gives a picture of clouds,
general weather conditions, and visual meteorological
conditions (VMC) expected over a large area encompassing
several states.
Area navigation (RNAV). Allows a pilot to fly a selected
course to a predetermined point without the need to overfly
ground-based navigation facilities, by using waypoints.
Arm. See moment arm.
ARSR. See air route surveillance radar.
ARTCC. See air route traffic control center.

Atmospheric propagation delay. A bending of the
electromagnetic (EM) wave from the satellite that creates
an error in the GPS system.
Attitude. A personal motivational predisposition to respond
to persons, situations, or events in a given manner that can,
nevertheless, be changed or modified through training as sort
of a mental shortcut to decision-making.
Attitude and heading reference system (AHRS). A system
composed of three-axis sensors that provide heading, attitude,
and yaw information for aircraft. AHRS are designed to
replace traditional mechanical gyroscopic flight instruments
and provide superior reliability and accuracy.
Attitude director indicator (ADI). An aircraft attitude
indicator that incorporates flight command bars to provide
pitch and roll commands.
Attitude indicator. The foundation for all instrument flight,
this instrument reflects the airplane's attitude in relation to
the horizon.
Attitude instrument flying. Controlling the aircraft by
reference to the instruments rather than by outside visual cues.
Attitude management. The ability to recognize hazardous
attitudes in oneself and the willingness to modify them as
necessary through the application of an appropriate antidote
thought.

ASDE. See airport surface detection equipment.
ASOS. See Automated Surface Observing System.
Aspect ratio. Span of a wing divided by its average chord.
ASR. See airport surveillance radar.

G-4

Autokinesis. Nighttime visual illusion that a stationary light
is moving, which becomes apparent after several seconds of
staring at the light.

Automated Surface Observing System (ASOS). Weather
reporting system which provides surface observations every
minute via digitized voice broadcasts and printed reports.
Automated Weather Observing System (AWOS).
Automated weather reporting system consisting of various
sensors, a processor, a computer-generated voice subsystem,
and a transmitter to broadcast weather data.
Automatic dependent surveillance—broadcast (ADS–B).
A function on an aircraft or vehicle that periodically
broadcasts its state vector (i.e., horizontal and vertical
position, horizontal and vertical velocity) and other
information.
Automatic direction finder (ADF). Electronic navigation
equipment that operates in the low- and medium-frequency
bands. Used in conjunction with the ground-based
nondirectional beacon (NDB), the instrument displays the
number of degrees clockwise from the nose of the aircraft
to the station being received.
Automatic terminal information service (ATIS). The
continuous broadcast of recorded non-control information in
selected terminal areas. Its purpose is to improve controller
effectiveness and relieve frequency congestion by automating
repetitive transmission of essential but routine information.
Autopilot. An automatic flight control system which keeps
an aircraft in level flight or on a set course. Automatic pilots
can be directed by the pilot, or they may be coupled to a radio
navigation signal.
Aviation medical examiner (AME). A physician with
training in aviation medicine designated by the Civil
Aerospace Medical Institute (CAMI).

Axial flow compressor. A type of compressor used in a
turbine engine in which the airflow through the compressor
is essentially linear. An axial-flow compressor is made up of
several stages of alternate rotors and stators. The compressor
ratio is determined by the decrease in area of the succeeding
stages.
Azimuth card. A card that may be set, gyroscopically
controlled, or driven by a remote compass.

B
Back course (BC). The reciprocal of the localizer course
for an ILS. When flying a back-course approach, an aircraft
approaches the instrument runway from the end at which the
localizer antennas are installed.
Balance tab. An auxiliary control mounted on a primary
control surface, which automatically moves in the direction
opposite the primary control to provide an aerodynamic
assist in the movement of the control.
Baro-aiding. A method of augmenting the GPS integrity
solution by using a nonsatellite input source. To ensure that
baro-aiding is available, the current altimeter setting must
be entered as described in the operating manual.
Barometric scale. A scale on the dial of an altimeter to which
the pilot sets the barometric pressure level from which the
altitude shown by the pointers is measured.
Basic empty weight (GAMA). Basic empty weight
includes the standard empty weight plus optional and special
equipment that has been installed.
BC. See back course.

Aviation Routine Weather Report (METAR). Observation
of current surface weather reported in a standard international
format.

Bernoulli's Principle. A principle that explains how the
pressure of a moving fluid varies with its speed of motion.
An increase in the speed of movement causes a decrease in
the fluid's pressure.

AWOS. See Automated Weather Observing System.

Biplanes. Airplanes with two sets of wings.

Axes of an aircraft. Three imaginary lines that pass through
an aircraft's center of gravity. The axes can be considered
as imaginary axles around which the aircraft rotates. The
three axes pass through the center of gravity at 90° angles to
each other. The axis from nose to tail is the longitudinal axis
(pitch), the axis that passes from wingtip to wingtip is the
lateral axis (roll), and the axis that passes vertically through
the center of gravity is the vertical axis (yaw).

Block altitude. A block of altitudes assigned by ATC to
allow altitude deviations; for example, "Maintain block
altitude 9 to 11 thousand."
Bypass ratio. The ratio of the mass airflow in pounds per
second through the fan section of a turbofan engine to the
mass airflow that passes through the gas generator portion
of the engine.

G-5

C
Cabin altitude. Cabin pressure in terms of equivalent altitude
above sea level.
Cage. The black markings on the ball instrument indicating
its neutral position.
Calibrated. The instrument indication compared with a
standard value to determine the accuracy of the instrument.
Calibrated orifice. A hole of specific diameter used to delay
the pressure change in the case of a vertical speed indicator.
Calibrated airspeed. The speed at which the aircraft
is moving through the air, found by correcting IAS for
instrument and position errors.
Camber. The camber of an airfoil is the characteristic curve
of its upper and lower surfaces. The upper camber is more
pronounced, while the lower camber is comparatively flat.
This causes the velocity of the airflow immediately above the
wing to be much higher than that below the wing.
Canard. A horizontal surface mounted ahead of the main
wing to provide longitudinal stability and control. It may
be a fixed, movable, or variable geometry surface, with or
without control surfaces.
Canard configuration. A configuration in which the span
of the forward wings is substantially less than that of the
main wing.
Cantilever. A wing designed to carry loads without external
struts.
CAS. Calibrated airspeed.
CDI. Course deviation indicator.
Ceiling. The height above the earth's surface of the lowest
layer of clouds, which is reported as broken or overcast, or
the vertical visibility into an obscuration.
Center of gravity (CG). The point at which an airplane
would balance if it were possible to suspend it at that point.
It is the mass center of the airplane, or the theoretical point
at which the entire weight of the airplane is assumed to
be concentrated. It may be expressed in inches from the
reference datum, or in percentage of mean aerodynamic
chord (MAC). The location depends on the distribution of
weight in the airplane.

G-6

Center of gravity limits. The specified forward and aft
points within which the CG must be located during flight.
These limits are indicated on pertinent airplane specifications.
Center of gravity range. The distance between the forward
and aft CG limits indicated on pertinent airplane specifications.
Center of pressure. A point along the wing chord line where
lift is considered to be concentrated. For this reason, the center
of pressure is commonly referred to as the center of lift.
Centrifugal flow compressor. An impeller-shaped device
that receives air at its center and slings the air outward at high
velocity into a diffuser for increased pressure. Also referred
to as a radial outflow compressor.
Centrifugal force. An outward force that opposes centripetal
force, resulting from the effect of inertia during a turn.
Centripetal force. A center-seeking force directed inward
toward the center of rotation created by the horizontal
component of lift in turning flight.
CG. See center of gravity.
Changeover point (COP). A point along the route or
airway segment between two adjacent navigation facilities
or waypoints where changeover in navigation guidance
should occur.
Chart Supplement U.S. (formerly Airport/Facility
Directory). An FAA publication containing information on
all airports, communications, and NAVAIDs.
Checklist. A tool that is used as a human factors aid in
aviation safety. It is a systematic and sequential list of all
operations that must be performed to properly accomplish
a task.
Chord line. An imaginary straight line drawn through an
airfoil from the leading edge to the trailing edge.
Circling approach. A maneuver initiated by the pilot to
align the aircraft with a runway for landing when a straightin landing from an instrument approach is not possible or is
not desirable.

Class A airspace. Airspace from 18,000 feet MSL up to and
including FL 600, including the airspace overlying the waters
within 12 NM of the coast of the 48 contiguous states and
Alaska; and designated international airspace beyond 12 NM
of the coast of the 48 contiguous states and Alaska within areas
of domestic radio navigational signal or ATC radar coverage,
and within which domestic procedures are applied.
Class B airspace. Airspace from the surface to 10,000 feet
MSL surrounding the nation's busiest airports in terms of
IFR operations or passenger numbers. The configuration of
each Class B airspace is individually tailored and consists
of a surface area and two or more layers, and is designed to
contain all published instrument procedures once an aircraft
enters the airspace. For all aircraft, an ATC clearance is
required to operate in the area, and aircraft so cleared receive
separation services within the airspace.
Class C airspace. Airspace from the surface to 4,000 feet
above the airport elevation (charted in MSL) surrounding
those airports having an operational control tower, serviced
by radar approach control, and having a certain number of IFR
operations or passenger numbers. Although the configuration
of each Class C airspace area is individually tailored, the
airspace usually consists of a 5 NM radius core surface area
that extends from the surface up to 4,000 feet above the airport
elevation, and a 10 NM radius shelf area that extends from
1,200 feet to 4,000 feet above the airport elevation.
Class D airspace. Airspace from the surface to 2,500 feet
above the airport elevation (charted in MSL) surrounding
those airports that have an operational control tower. The
configuration of each Class D airspace area is individually
tailored, and when instrument procedures are published, the
airspace is normally designed to contain the procedures.

Clearance delivery. Control tower position responsible for
transmitting departure clearances to IFR flights.
Clearance limit. The fix, point, or location to which an
aircraft is cleared when issued an air traffic clearance.
Clearance on request. An IFR clearance not yet received
after filing a flight plan.
Clearance void time. Used by ATC, the time at which the
departure clearance is automatically canceled if takeoff has
not been made. The pilot must obtain a new clearance or
cancel the IFR flight plan if not off by the specified time.
Clear ice. Glossy, clear, or translucent ice formed by the
relatively slow freezing of large, supercooled water droplets.
Coefficient of lift (CL). The ratio between lift pressure and
dynamic pressure.
Cold front. The boundary between two air masses where
cold air is replacing warm air.
Compass course. A true course corrected for variation and
deviation errors.
Compass locator. A low-power, low- or medium-frequency
(L/MF) radio beacon installed at the site of the outer or middle
marker of an ILS.
Compass rose. A small circle graduated in 360° increments,
to show direction expressed in degrees.
Complex aircraft. An aircraft with retractable landing gear,
flaps, and a controllable-pitch propeller.

Class E airspace. Airspace that is not Class A, Class B, Class
C, or Class D, and is controlled airspace.

Compressor pressure ratio. The ratio of compressor
discharge pressure to compressor inlet pressure.

Class G airspace. Airspace that is uncontrolled, except
when associated with a temporary control tower, and has
not been designated as Class A, Class B, Class C, Class D,
or Class E airspace.
Clean configuration. A configuration in which all flight
control surfaces have been placed to create minimum drag.
In most aircraft this means flaps and gear retracted.

Compressor stall. In gas turbine engines, a condition in
an axial-flow compressor in which one or more stages of
rotor blades fail to pass air smoothly to the succeeding
stages. A stall condition is caused by a pressure ratio that is
incompatible with the engine rpm. Compressor stall will be
indicated by a rise in exhaust temperature or rpm fluctuation,
and if allowed to continue, may result in flameout and
physical damage to the engine.

Clearance. ATC permission for an aircraft to proceed under
specified traffic conditions within controlled airspace, for
the purpose of providing separation between known aircraft.

Computer navigation fix. A point used to define a
navigation track for an airborne computer system such as
GPS or FMS.

G-7

Concentric rings. Dashed-line circles depicted in the plan
view of IAP charts, outside of the reference circle, that show
en route and feeder facilities.
Condensation. A change of state of water from a gas (water
vapor) to a liquid.

Control pressures. The amount of physical exertion on the
control column necessary to achieve the desired attitude.
Convective weather. Unstable, rising air found in
cumiliform clouds.

Condensation nuclei. Small particles of solid matter in the
air on which water vapor condenses.

Convective SIGMET. Weather advisory concerning
convective weather significant to the safety of all aircraft,
including thunderstorms, hail, and tornadoes.

Cone of confusion. A cone-shaped volume of airspace
directly above a VOR station where no signal is received,
causing the CDI to fluctuate.

Conventional landing gear. Landing gear employing a third
rear-mounted wheel. These airplanes are also sometimes
referred to as tailwheel airplanes.

Configuration. This is a general term, which normally refers
to the position of the landing gear and flaps.

Coordinated flight. Flight with a minimum disturbance of
the forces maintaining equilibrium, established via effective
control use.

Constant-speed propeller. A controllable-pitch propeller
whose pitch is automatically varied in flight by a governor
to maintain a constant rpm in spite of varying air loads.
Continuous flow oxygen system. System that supplies
a constant supply of pure oxygen to a rebreather bag that
dilutes the pure oxygen with exhaled gases and thus supplies a
healthy mix of oxygen and ambient air to the mask. Primarily
used in passenger cabins of commercial airliners.
Control and performance. A method of attitude instrument
flying in which one instrument is used for making attitude
changes, and the other instruments are used to monitor the
progress of the change.
Control display unit. A display interfaced with the master
computer, providing the pilot with a single control point
for all navigations systems, thereby reducing the number of
required flight deck panels.
Controllability. A measure of the response of an aircraft
relative to the pilot's flight control inputs.
Controllable-pitch propeller (CPP). A type of propeller
with blades that can be rotated around their long axis to
change their pitch. If the pitch can be set to negative values,
the reversible propeller can also create reverse thrust for
braking or reversing without the need of changing the
direction of shaft revolutions.
Controlled airspace. An airspace of defined dimensions
within which ATC service is provided to IFR and VFR flights
in accordance with the airspace classification. It includes
Class A, Class B, Class C, Class D, and Class E airspace.

G-8

COP. See changeover point.
Coriolis illusion. The illusion of rotation or movement in an
entirely different axis, caused by an abrupt head movement,
while in a prolonged constant-rate turn that has ceased to
stimulate the brain's motion sensing system.
Coupled ailerons and rudder. Rudder and ailerons are
connected with interconnected springs in order to counteract
adverse yaw. Can be overridden if it becomes necessary to
slip the aircraft.
Course. The intended direction of flight in the horizontal
plane measured in degrees from north.
Cowl flaps. Shutter-like devices arranged around certain
air-cooled engine cowlings, which may be opened or closed
to regulate the flow of air around the engine.
Crew resource management (CRM). The application of
team management concepts in the flight deck environment.
It was initially known as cockpit resource management,
but as CRM programs evolved to include cabin crews,
maintenance personnel, and others, the phrase "crew
resource management" was adopted. This includes single
pilots, as in most general aviation aircraft. Pilots of small
aircraft, as well as crews of larger aircraft, must make
effective use of all available resources; human resources,
hardware, and information. A current definition includes
all groups routinely working with the flight crew who
are involved in decisions required to operate a flight
safely. These groups include, but are not limited to pilots,
dispatchers, cabin crewmembers, maintenance personnel,
and air traffic controllers. CRM is one way of addressing
the challenge of optimizing the human/machine interface
and accompanying interpersonal activities.

Critical altitude. The maximum altitude under standard
atmospheric conditions at which a turbocharged engine can
produce its rated horsepower.
Critical angle of attack. The angle of attack at which a
wing stalls regardless of airspeed, flight attitude, or weight.
Critical areas. Areas where disturbances to the ILS localizer
and glideslope courses may occur when surface vehicles or
aircraft operate near the localizer or glideslope antennas.

Decision altitude (DA). A specified altitude in the precision
approach, charted in feet MSL, at which a missed approach
must be initiated if the required visual reference to continue
the approach has not been established.
Decision height (DH). A specified altitude in the precision
approach, charted in height above threshold elevation,
at which a decision must be made either to continue the
approach or to execute a missed approach.

CRM. See crew resource management.

Deice. The act of removing ice accumulation from an
aircraft structure.

Cross-check. The first fundamental skill of instrument flight,
also known as "scan," the continuous and logical observation
of instruments for attitude and performance information.

Delta. A Greek letter expressed by the symbol Δ to indicate
a change of values. As an example, ΔCG indicates a change
(or movement) of the CG.

Cruise clearance. An ATC clearance issued to allow a
pilot to conduct flight at any altitude from the minimum
IFR altitude up to and including the altitude specified in the
clearance. Also authorizes a pilot to proceed to and make an
approach at the destination airport.

Density altitude. Pressure altitude corrected for nonstandard
temperature. Density altitude is used in computing the
performance of an aircraft and its engines.

Current induction. An electrical current being induced into,
or generated in, any conductor that is crossed by lines of flux
from any magnet.

D
DA. See decision altitude.
Datum (Reference Datum). An imaginary vertical plane
or line from which all measurements of arm are taken. The
datum is established by the manufacturer. Once the datum
has been selected, all moment arms and the location of CG
range are measured from this point.
D.C. Direct current.

Departure procedure (DP). Preplanned IFR ATC departure,
published for pilot use, in textual and graphic format.
Deposition. The direct transformation of a gas to a solid
state, in which the liquid state is bypassed. Some sources use
sublimation to describe this process instead of deposition.
Detonation. The sudden release of heat energy from fuel in
an aircraft engine caused by the fuel-air mixture reaching
its critical pressure and temperature. Detonation occurs as
a violent explosion rather than a smooth burning process.
Deviation. A magnetic compass error caused by local
magnetic fields within the aircraft. Deviation error is different
on each heading.

Dark adaptation. Physical and chemical adjustments of the
eye that make vision possible in relative darkness.

Dew. Moisture that has condensed from water vapor. Usually
found on cooler objects near the ground, such as grass, as
the near-surface layer of air cools faster than the layers of
air above it.

Dead reckoning. Navigation of an airplane solely by means
of computations based on airspeed, course, heading, wind
direction and speed, groundspeed, and elapsed time.

Dewpoint. The temperature at which air reaches a state where
it can hold no more water.

Deceleration error. A magnetic compass error that occurs
when the aircraft decelerates while flying on an easterly
or westerly heading, causing the compass card to rotate
toward South.

DGPS. Differential global positioning system.
DH. See decision height.

G-9

Differential ailerons. Control surface rigged such that the
aileron moving up moves a greater distance than the aileron
moving down. The up aileron produces extra parasite drag
to compensate for the additional induced drag caused by
the down aileron. This balancing of the drag forces helps
minimize adverse yaw.
Differential Global Positioning System (DGPS). A system
that improves the accuracy of Global Navigation Satellite
Systems (GNSS) by measuring changes in variables to
provide satellite positioning corrections.
Differential pressure. A difference between two pressures.
The measurement of airspeed is an example of the use of
differential pressure.
Dihedral. The positive acute angle between the lateral
axis of an airplane and a line through the center of a wing
or horizontal stabilizer. Dihedral contributes to the lateral
stability of an airplane.
Diluter-demand oxygen system. An oxygen system that
delivers oxygen mixed or diluted with air in order to maintain
a constant oxygen partial pressure as the altitude changes.
Direct indication. The true and instantaneous reflection of
aircraft pitch-and-bank attitude by the miniature aircraft,
relative to the horizon bar of the attitude indicator.
Direct User Access Terminal System (DUATS). A system
that provides current FAA weather and flight plan filing
services to certified civil pilots, via personal computer,
modem, or telephone access to the system. Pilots can request
specific types of weather briefings and other pertinent data
for planned flights.
Directional stability. Stability about the vertical axis of an
aircraft, whereby an aircraft tends to return, on its own, to
flight aligned with the relative wind when disturbed from that
equilibrium state. The vertical tail is the primary contributor
to directional stability, causing an airplane in flight to align
with the relative wind.
Distance circle. See reference circle.
Distance measuring equipment (DME). A pulse-type
electronic navigation system that shows the pilot, by an
instrument-panel indication, the number of nautical miles
between the aircraft and a ground station or waypoint.
DME. See distance measuring equipment.

G-10

DME arc. A flight track that is a constant distance from the
station or waypoint.
DOD. Department of Defense.
Doghouse. A turn-and-slip indicator dial mark in the shape
of a doghouse.
Domestic Reduced Vertical Separation Minimum
(DRVSM). Additional flight levels between FL 290 and FL
410 to provide operational, traffic, and airspace efficiency.
Double gimbal. A type of mount used for the gyro in an
attitude instrument. The axes of the two gimbals are at right
angles to the spin axis of the gyro, allowing free motion in
two planes around the gyro.
DP. See departure procedure.
Drag. The net aerodynamic force parallel to the relative
wind, usually the sum of two components: induced drag
and parasite drag.
Drag curve. The curve created when plotting induced drag
and parasite drag.
Drift angle. Angle between heading and track.
DRVSM. See Domestic Reduced Vertical Separation
Minimum.
DUATS. See direct user access terminal system.
Duplex. Transmitting on one frequency and receiving on a
separate frequency.
Dutch roll. A combination of rolling and yawing oscillations
that normally occurs when the dihedral effects of an aircraft
are more powerful than the directional stability. Usually
dynamically stable but objectionable in an airplane because
of the oscillatory nature.
Dynamic hydroplaning. A condition that exists when
landing on a surface with standing water deeper than the
tread depth of the tires. When the brakes are applied, there is
a possibility that the brake will lock up and the tire will ride
on the surface of the water, much like a water ski. When the
tires are hydroplaning, directional control and braking action
are virtually impossible. An effective anti-skid system can
minimize the effects of hydroplaning.

Dynamic stability. The property of an aircraft that causes
it, when disturbed from straight-and-level flight, to develop
forces or moments that restore the original condition of
straight and level.

E
Eddy currents. Current induced in a metal cup or disc when
it is crossed by lines of flux from a moving magnet.

Encoding altimeter. A special type of pressure altimeter
used to send a signal to the air traffic controller on the ground,
showing the pressure altitude the aircraft is flying.
Engine pressure ratio (EPR). The ratio of turbine discharge
pressure divided by compressor inlet pressure, which is used
as an indication of the amount of thrust being developed by
a turbine engine.

Eddy current damping. The decreased amplitude of
oscillations by the interaction of magnetic fields. In the case
of a vertical card magnetic compass, flux from the oscillating
permanent magnet produces eddy currents in a damping disk
or cup. The magnetic flux produced by the eddy currents
opposes the flux from the permanent magnet and decreases
the oscillations.

En route facilities ring. Depicted in the plan view of IAP
charts, a circle which designates NAVAIDs, fixes, and
intersections that are part of the en route low altitude airway
structure.

EFC. See expect-further-clearance.

En route low-altitude charts. Aeronautical charts for en
route IFR navigation below 18,000 feet MSL.

EFD. See electronic flight display.
EGT. See exhaust gas temperature.
Electronic flight display (EFD). For the purpose of
standardization, any flight instrument display that uses LCD or
other image-producing system (cathode ray tube (CRT), etc.)
Elevator. The horizontal, movable primary control surface in
the tail section, or empennage, of an airplane. The elevator is
hinged to the trailing edge of the fixed horizontal stabilizer.
Elevator illusion. The sensation of being in a climb or
descent, caused by the kind of abrupt vertical accelerations
that result from up- or downdrafts.
Emergency. A distress or urgent condition.
Empennage. The section of the airplane that consists of the
vertical stabilizer, the horizontal stabilizer, and the associated
control surfaces.
Emphasis error. The result of giving too much attention
to a particular instrument during the cross-check, instead of
relying on a combination of instruments necessary for attitude
and performance information.
Empty-field myopia. Induced nearsightedness that is
associated with flying at night, in instrument meteorological
conditions and/or reduced visibility. With nothing to focus
on, the eyes automatically focus on a point just slightly ahead
of the airplane.

En route high-altitude charts. Aeronautical charts for en
route instrument navigation at or above 18,000 feet MSL.

EPR. See engine pressure ratio.
Equilibrium. A condition that exists within a body when the
sum of the moments of all of the forces acting on the body
is equal to zero. In aerodynamics, equilibrium is when all
opposing forces acting on an aircraft are balanced (steady,
unaccelerated flight conditions).
Equivalent airspeed. Airspeed equivalent to CAS in standard
atmosphere at sea level. As the airspeed and pressure altitude
increase, the CAS becomes higher than it should be, and a
correction for compression must be subtracted from the CAS.
Evaporation. The transformation of a liquid to a gaseous
state, such as the change of water to water vapor.
Exhaust gas temperature (EGT). The temperature of the
exhaust gases as they leave the cylinders of a reciprocating
engine or the turbine section of a turbine engine.
Expect-further-clearance (EFC). The time a pilot can
expect to receive clearance beyond a clearance limit.
Explosive decompression. A change in cabin pressure faster
than the lungs can decompress. Lung damage is possible.

F
FA. See area forecast.
FAA. Federal Aviation Administration.
FAF. See final approach fix.

EM wave. Electromagnetic wave.
G-11

False horizon. Inaccurate visual information for aligning the
aircraft, caused by various natural and geometric formations
that disorient the pilot from the actual horizon.

Flight director indicator (FDI). One of the major
components of a flight director system, it provides steering
commands that the pilot (or the autopilot, if coupled) follows.

FDI. See flight director indicator.

Flight level (FL). A measure of altitude (in hundreds of feet)
used by aircraft flying above 18,000 feet with the altimeter
set at 29.92 "Hg.

Federal airways. Class E airspace areas that extend upward
from 1,200 feet to, but not including, 18,000 feet MSL, unless
otherwise specified.
Feeder facilities. Used by ATC to direct aircraft to
intervening fixes between the en route structure and the
initial approach fix.

Flight management system (FMS). Provides pilot and crew
with highly accurate and automatic long-range navigation
capability, blending available inputs from long- and shortrange sensors.

Final approach. Part of an instrument approach procedure in
which alignment and descent for landing are accomplished.

Flight path. The line, course, or track along which an aircraft
is flying or is intended to be flown.

Final approach fix (FAF). The fix from which the IFR
final approach to an airport is executed, and which identifies
the beginning of the final approach segment. An FAF is
designated on government charts by a Maltese cross symbol
for nonprecision approaches, and a lightning bolt symbol for
precision approaches.

Flight patterns. Basic maneuvers, flown by reference to the
instruments rather than outside visual cues, for the purpose
of practicing basic attitude flying. The patterns simulate
maneuvers encountered on instrument flights such as holding
patterns, procedure turns, and approaches.

Fixating. Staring at a single instrument, thereby interrupting
the cross-check process.

Flight strips. Paper strips containing instrument flight
information, used by ATC when processing flight plans.
FMS. See flight management system.

Fixed-pitch propellers. Propellers with fixed blade angles.
Fixed-pitch propellers are designed as climb propellers,
cruise propellers, or standard propellers.
Fixed slot. A fixed, nozzle shaped opening near the leading
edge of a wing that ducts air onto the top surface of the wing.
Its purpose is to increase lift at higher angles of attack.
FL. See flight level.
Flameout. A condition in the operation of a gas turbine
engine in which the fire in the engine goes out due to either
too much or too little fuel sprayed into the combustors.
Flaps. Hinged portion of the trailing edge between the
ailerons and fuselage. In some aircraft ailerons and flaps are
interconnected to produce full-span "flaperons." In either
case, flaps change the lift and drag on the wing.
Floor load limit. The maximum weight the floor can sustain
per square inch/foot as provided by the manufacturer.
Flight configurations. Adjusting the aircraft control surfaces
(including flaps and landing gear) in a manner that will
achieve a specified attitude.

G-12

FOD. See foreign object damage.
Fog. Cloud consisting of numerous minute water droplets
and based at the surface; droplets are small enough to be
suspended in the earth's atmosphere indefinitely. (Unlike
drizzle, it does not fall to the surface. Fog differs from a
cloud only in that a cloud is not based at the surface, and
is distinguished from haze by its wetness and gray color.)
Force (F). The energy applied to an object that attempts to
cause the object to change its direction, speed, or motion.
In aerodynamics, it is expressed as F, T (thrust), L (lift), W
(weight), or D (drag), usually in pounds.
Foreign object damage (FOD). Damage to a gas turbine
engine caused by some object being sucked into the engine
while it is running. Debris from runways or taxiways can
cause foreign object damage during ground operations, and
the ingestion of ice and birds can cause FOD in flight.
Form drag. The drag created because of the shape of a
component or the aircraft.

Frise-type aileron. Aileron having the nose portion
projecting ahead of the hinge line. When the trailing edge
of the aileron moves up, the nose projects below the wing's
lower surface and produces some parasite drag, decreasing
the amount of adverse yaw.

Global positioning system (GPS). Navigation system
that uses satellite rather than ground-based transmitters for
location information.

Front. The boundary between two different air masses.

GNSS. See global navigation satellite system.

Frost. Ice crystal deposits formed by sublimation when
temperature and dewpoint are below freezing.

Goniometer. As used in radio frequency (RF) antenna
systems, a direction-sensing device consisting of two fixed
loops of wire oriented 90° from each other, which separately
sense received signal strength and send those signals to two
rotors (also oriented 90°) in the sealed direction-indicating
instrument. The rotors are attached to the direction-indicating
needle of the instrument and rotated by a small motor until
minimum magnetic field is sensed near the rotors.

Fuel load. The expendable part of the load of the airplane.
It includes only usable fuel, not fuel required to fill the lines
or that which remains trapped in the tank sumps.
Fundamental skills. Pilot skills of instrument cross-check,
instrument interpretation, and aircraft control.

GLS. See global landing system.

GPS. See global positioning system.
Fuselage. The section of the airplane that consists of the
cabin and/or cockpit, containing seats for the occupants and
the controls for the airplane.

G
GAMA. General Aviation Manufacturers Association.
Gimbal ring. A type of support that allows an object, such
as a gyroscope, to remain in an upright condition when its
base is tilted.
Glideslope (GS). Part of the ILS that projects a radio beam
upward at an angle of approximately 3° from the approach
end of an instrument runway. The glideslope provides
vertical guidance to aircraft on the final approach course for
the aircraft to follow when making an ILS approach along
the localizer path.
Glideslope intercept altitude. The minimum altitude of an
intermediate approach segment prescribed for a precision
approach that ensures obstacle clearance.
Global landing system (GLS). An instrument approach with
lateral and vertical guidance with integrity limits (similar to
barometric vertical navigation (BARO VNAV).
Global navigation satellite system (GNSS). Satellite
navigation system that provides autonomous geospatial
positioning with global coverage. It allows small electronic
receivers to determine their location (longitude, latitude, and
altitude) to within a few meters using time signals transmitted
along a line of sight by radio from satellites.

GPS Approach Overlay Program. An authorization for
pilots to use GPS avionics under IFR for flying designated
existing nonprecision instrument approach procedures, with
the exception of LOC, LDA, and SDF procedures.
GPWS. See ground proximity warning system.
Graveyard spiral. The illusion of the cessation of a turn while
still in a prolonged, coordinated, constant rate turn, which
can lead a disoriented pilot to a loss of control of the aircraft.
Great circle route. The shortest distance across the surface
of a sphere (the Earth) between two points on the surface.
Ground adjustable trim tab. Non-movable metal trim tab
on a control surface. Bent in one direction or another while
on the ground to apply trim forces to the control surface.
Ground effect. The condition of slightly increased air pressure
below an airplane wing or helicopter rotor system that increases
the amount of lift produced. It exists within approximately one
wing span or one rotor diameter from the ground. It results
from a reduction in upwash, downwash, and wingtip vortices,
and provides a corresponding decrease in induced drag.
Ground proximity warning system (GPWS). A system
designed to determine an aircraft's clearance above the Earth
and provides limited predictability about aircraft position
relative to rising terrain.

G-13

Groundspeed. Speed over the ground, either closing speed to
the station or waypoint, or speed over the ground in whatever
direction the aircraft is going at the moment, depending upon
the navigation system used.
GS. See glideslope.

Height above landing (HAL). The height above a designated
helicopter landing area used for helicopter instrument
approach procedures.
Height above touchdown elevation (HAT). The DA/DH or
MDA above the highest runway elevation in the touchdown
zone (first 3,000 feet of the runway).

GWPS. See ground proximity warning system.
HF. High frequency.
Gyroscopic precession. An inherent quality of rotating bodies,
which causes an applied force to be manifested 90° in the
direction of rotation from the point where the force is applied.

H
HAA. See height above airport.
HAL. See height above landing.
HAT. See height above touchdown elevation.

Hg. Abbreviation for mercury, from the Latin hydrargyrum.
High performance aircraft. An aircraft with an engine of
more than 200 horsepower.
Histotoxic hypoxia. The inability of cells to effectively use
oxygen. Plenty of oxygen is being transported to the cells
that need it, but they are unable to use it.
HIWAS. See Hazardous Inflight Weather Advisory Service.

Hazardous attitudes. Five aeronautical decision-making
attitudes that may contribute to poor pilot judgment: antiauthority, impulsivity, invulnerability, machismo, and
resignation.

Holding. A predetermined maneuver that keeps aircraft
within a specified airspace while awaiting further clearance
from ATC.

Hazardous Inflight Weather Advisory Service (HIWAS).
An en route FSS service providing continuously updated
automated of hazardous weather within 150 nautical miles of
selected VORs, available only in the conterminous 48 states.

Holding pattern. A racetrack pattern, involving two turns
and two legs, used to keep an aircraft within a prescribed
airspace with respect to a geographic fix. A standard pattern
uses right turns; nonstandard patterns use left turns.

Head-up display (HUD). A special type of flight viewing
screen that allows the pilot to watch the flight instruments
and other data while looking through the windshield of the
aircraft for other traffic, the approach lights, or the runway.

Homing. Flying the aircraft on any heading required to keep
the needle pointing to the 0° relative bearing position.

Heading. The direction in which the nose of the aircraft is
pointing during flight.
Heading indicator. An instrument which senses airplane
movement and displays heading based on a 360° azimuth,
with the final zero omitted. The heading indicator, also called
a directional gyro (DG), is fundamentally a mechanical
instrument designed to facilitate the use of the magnetic
compass. The heading indicator is not affected by the forces
that make the magnetic compass difficult to interpret.
Headwork. Required to accomplish a conscious, rational
thought process when making decisions. Good decisionmaking involves risk identification and assessment,
information processing, and problem solving.
Height above airport (HAA). The height of the MDA above
the published airport elevation.
G-14

Horizontal situation indicator (HSI). A flight navigation
instrument that combines the heading indicator with a CDI,
in order to provide the pilot with better situational awareness
of location with respect to the courseline.
Horsepower. The term, originated by inventor James Watt,
means the amount of work a horse could do in one second.
One horsepower equals 550 foot-pounds per second, or
33,000 foot-pounds per minute.
Hot start. In gas turbine engines, a start which occurs with
normal engine rotation, but exhaust temperature exceeds
prescribed limits. This is usually caused by an excessively
rich mixture in the combustor. The fuel to the engine must
be terminated immediately to prevent engine damage.
HSI. See horizontal situation indicator.
HUD. See head-up display.

Human factors. A multidisciplinary field encompassing the
behavioral and social sciences, engineering, and physiology,
to consider the variables that influence individual and
crew performance for the purpose of optimizing human
performance and reducing errors.

Ident. Air Traffic Control request for a pilot to push
the button on the transponder to identify return on the
controller's scope.

Hung start. In gas turbine engines, a condition of normal
light off but with rpm remaining at some low value rather than
increasing to the normal idle rpm. This is often the result of
insufficient power to the engine from the starter. In the event
of a hung start, the engine should be shut down.

ILS. See instrument landing system.

Hydroplaning. A condition that exists when landing on a
surface with standing water deeper than the tread depth of
the tires. When the brakes are applied, there is a possibility
that the brake will lock up and the tire will ride on the
surface of the water, much like a water ski. When the tires
are hydroplaning, directional control and braking action
are virtually impossible. An effective anti-skid system can
minimize the effects of hydroplaning.
Hypemic hypoxia. A type of hypoxia that is a result of
oxygen deficiency in the blood, rather than a lack of inhaled
oxygen. It can be caused by a variety of factors. Hypemic
means "not enough blood."
Hyperventilation. Occurs when an individual is experiencing
emotional stress, fright, or pain, and the breathing rate and
depth increase, although the carbon dioxide level in the
blood is already at a reduced level. The result is an excessive
loss of carbon dioxide from the body, which can lead to
unconsciousness due to the respiratory system's overriding
mechanism to regain control of breathing.
Hypoxia. A state of oxygen deficiency in the body sufficient
to impair functions of the brain and other organs.
Hypoxic hypoxia. This type of hypoxia is a result of
insufficient oxygen available to the lungs. A decrease of
oxygen molecules at sufficient pressure can lead to hypoxic
hypoxia.

I
IAF. See initial approach fix.
IAP. See instrument approach procedures.
IAS. See indicated airspeed.
ICAO. See International Civil Aviation Organization.

IFR. See instrument flight rules.

ILS categories. Categories of instrument approach
procedures allowed at airports equipped with the following
types of instrument landing systems:
ILS Category I: Provides for approach to a height
above touchdown of not less than 200 feet, and with
runway visual range of not less than 1,800 feet.
ILS Category II: Provides for approach to a height
above touchdown of not less than 100 feet and with
runway visual range of not less than 1,200 feet.
ILS Category IIIA: Provides for approach without
a decision height minimum and with runway visual
range of not less than 700 feet.
ILS Category IIIB: Provides for approach without
a decision height minimum and with runway visual
range of not less than 150 feet.
ILS Category IIIC: Provides for approach without a
decision height minimum and without runway visual
range minimum.
IMC. See instrument meteorological conditions.
Inclinometer. An instrument consisting of a curved glass
tube, housing a glass ball, and damped with a fluid similar
to kerosene. It may be used to indicate inclination, as a level,
or, as used in the turn indicators, to show the relationship
between gravity and centrifugal force in a turn.
Indicated airspeed (IAS). Shown on the dial of the
instrument airspeed indicator on an aircraft. Indicated
airspeed (IAS) is the airspeed indicator reading uncorrected
for instrument, position, and other errors. Indicated airspeed
means the speed of an aircraft as shown on its pitot static
airspeed indicator calibrated to reflect standard atmosphere
adiabatic compressible flow at sea level uncorrected for
airspeed system errors. Calibrated airspeed (CAS) is IAS
corrected for instrument errors, position error (due to
incorrect pressure at the static port) and installation errors.
Indicated altitude. The altitude read directly from the
altimeter (uncorrected) when it is set to the current altimeter
setting.

G-15

Indirect indication. A reflection of aircraft pitch-and-bank
attitude by instruments other than the attitude indicator.
Induced drag. Drag caused by the same factors that produce
lift; its amount varies inversely with airspeed. As airspeed
decreases, the angle of attack must increase, in turn increasing
induced drag.
Induction icing. A type of ice in the induction system that
reduces the amount of air available for combustion. The most
commonly found induction icing is carburetor icing.
Inertial navigation system (INS). A computer-based
navigation system that tracks the movement of an aircraft
via signals produced by onboard accelerometers. The initial
location of the aircraft is entered into the computer, and all
subsequent movement of the aircraft is sensed and used to
keep the position updated. An INS does not require any inputs
from outside signals.
Initial approach fix (IAF). The fix depicted on IAP charts
where the instrument approach procedure (IAP) begins unless
otherwise authorized by ATC.
Inoperative components. Higher minimums are prescribed
when the specified visual aids are not functioning; this
information is listed in the Inoperative Components Table
found in the United States Terminal Procedures Publications.
INS. See inertial navigation system.
Instantaneous vertical speed indicator (IVSI). Assists in
interpretation by instantaneously indicating the rate of climb
or descent at a given moment with little or no lag as displayed
in a vertical speed indicator (VSI).
Instrument approach procedures (IAP). A series of
predetermined maneuvers for the orderly transfer of an
aircraft under IFR from the beginning of the initial approach
to a landing or to a point from which a landing may be
made visually.
Instrument flight rules (IFR). Rules and regulations
established by the Federal Aviation Administration to govern
flight under conditions in which flight by outside visual
reference is not safe. IFR flight depends upon flying by
reference to instruments in the flight deck, and navigation is
accomplished by reference to electronic signals.
Instrument landing system (ILS). An electronic system
that provides both horizontal and vertical guidance to a
specific runway, used to execute a precision instrument
approach procedure.
G-16

Instrument meteorological conditions (IMC).
Meteorological conditions expressed in terms of visibility,
distance from clouds, and ceiling less than the minimums
specified for visual meteorological conditions, requiring
operations to be conducted under IFR.
Instrument takeoff. Using the instruments rather than
outside visual cues to maintain runway heading and execute
a safe takeoff.
Intercooler. A device used to reduce the temperatures of the
compressed air before it enters the fuel metering device. The
resulting cooler air has a higher density, which permits the
engine to be operated with a higher power setting.
Interference drag. Drag generated by the collision of
airstreams creating eddy currents, turbulence, or restrictions
to smooth flow.
International Civil Aviation Organization (ICAO). The
United Nations agency for developing the principles and
techniques of international air navigation, and fostering
planning and development of international civil air transport.
International standard atmosphere (IAS). A model of
standard variation of pressure and temperature.
Interpolation. The estimation of an intermediate value
of a quantity that falls between marked values in a series.
Example: In a measurement of length, with a rule that is
marked in eighths of an inch, the value falls between 3/8
inch and 1/2 inch. The estimated (interpolated) value might
then be said to be 7/16 inch.
Inversion. An increase in temperature with altitude.
Inversion illusion. The feeling that the aircraft is tumbling
backwards, caused by an abrupt change from climb to straight­
and-level flight while in situations lacking visual reference.
Inverter. A solid-state electronic device that converts D.C.
into A.C. current of the proper voltage and frequency to
operate A.C. gyro instruments.
Isobars. Lines which connect points of equal barometric
pressure.
Isogonic lines. Lines drawn across aeronautical charts to
connect points having the same magnetic variation.
IVSI. See instantaneous vertical speed indicator.

J
Jet route. A route designated to serve flight operations from
18,000 feet MSL up to and including FL 450.
Jet stream. A high-velocity narrow stream of winds, usually
found near the upper limit of the troposphere, which flows
generally from west to east.
Judgment. The mental process of recognizing and analyzing
all pertinent information in a particular situation, a rational
evaluation of alternative actions in response to it, and a timely
decision on which action to take.

K
KIAS. Knots indicated airspeed.
Knot. The knot is a unit of speed equal to one nautical mile
(1.852 km) per hour, approximately 1.151 mph.
Kollsman window. A barometric scale window of a
sensitive altimeter used to adjust the altitude for the
altimeter setting.

L
LAAS. See local area augmentation system.
Lag. The delay that occurs before an instrument needle attains
a stable indication.
Land breeze. A coastal breeze flowing from land to sea
caused by temperature differences when the sea surface is
warmer than the adjacent land. The land breeze usually occurs
at night and alternates with the sea breeze that blows in the
opposite direction by day.

Lateral stability (rolling). The stability about the
longitudinal axis of an aircraft. Rolling stability or the ability
of an airplane to return to level flight due to a disturbance
that causes one of the wings to drop.
Latitude. Measurement north or south of the equator in
degrees, minutes, and seconds. Lines of latitude are also
referred to as parallels.
LDA. See localizer-type directional aid.
Lead radial. The radial at which the turn from the DME arc
to the inbound course is started.
Leading edge. The part of an airfoil that meets the airflow first.
Leading edge devices. High lift devices which are found
on the leading edge of the airfoil. The most common types
are fixed slots, movable slats, and leading edge flaps.
Leading-edge flap. A portion of the leading edge of an
airplane wing that folds downward to increase the camber,
lift, and drag of the wing. The leading-edge flaps are
extended for takeoffs and landings to increase the amount
of aerodynamic lift that is produced at any given airspeed.
Leans, the. A physical sensation caused by an abrupt correction
of a banked attitude entered too slowly to stimulate the motion
sensing system in the inner ear. The abrupt correction can
create the illusion of banking in the opposite direction.
Licensed empty weight. The empty weight that consists
of the airframe, engine(s), unusable fuel, and undrainable
oil plus standard and optional equipment as specified in the
equipment list. Some manufacturers used this term prior to
GAMA standardization.

Land as soon as possible. Land without delay at the nearest
suitable area, such as an open field, at which a safe approach
and landing is assured.

Lift. A component of the total aerodynamic force on an airfoil
and acts perpendicular to the relative wind.

Land as soon as practical. The landing site and duration of
flight are at the discretion of the pilot. Extended flight beyond
the nearest approved landing area is not recommended.

Limit load factor. Amount of stress, or load factor, that an
aircraft can withstand before structural damage or failure
occurs.

Land immediately. The urgency of the landing is paramount.
The primary consideration is to ensure the survival of the
occupants. Landing in trees, water, or other unsafe areas
should be considered only as a last resort.

Lines of flux. Invisible lines of magnetic force passing
between the poles of a magnet.

Lateral axis. An imaginary line passing through the center
of gravity of an airplane and extending across the airplane
from wingtip to wingtip.

LMM. See locator middle marker.

L/MF. See low or medium frequency.

G-17

Load factor. The ratio of a specified load to the total weight
of the aircraft. The specified load is expressed in terms of
any of the following: aerodynamic forces, inertial forces, or
ground or water reactions.
Loadmeter. A type of ammeter installed between the generator
output and the main bus in an aircraft electrical system.
LOC. See localizer.
Local area augmentation system (LAAS). A differential
global positioning system (DGPS) that improves the accuracy
of the system by determining position error from the GPS
satellites, then transmitting the error, or corrective factors,
to the airborne GPS receiver.
Localizer (LOC). The portion of an ILS that gives left/right
guidance information down the centerline of the instrument
runway for final approach.
Localizer-type directional aid (LDA). A NAVAID used
for nonprecision instrument approaches with utility and
accuracy comparable to a localizer but which is not a part
of a complete ILS and is not aligned with the runway. Some
LDAs are equipped with a glideslope.
Locator middle marker (LMM). Nondirectional radio
beacon (NDB) compass locator, collocated with a middle
marker (MM).
Locator outer marker (LOM). NDB compass locator,
collocated with an outer marker (OM).
LOM. See locator outer marker.
Longitude. Measurement east or west of the Prime Meridian
in degrees, minutes, and seconds. The Prime Meridian is 0°
longitude and runs through Greenwich, England. Lines of
longitude are also referred to as meridians.
Longitudinal axis. An imaginary line through an aircraft
from nose to tail, passing through its center of gravity. The
longitudinal axis is also called the roll axis of the aircraft.
Movement of the ailerons rotates an airplane about its
longitudinal axis.
Longitudinal stability (pitching). Stability about the lateral
axis. A desirable characteristic of an airplane whereby it tends
to return to its trimmed angle of attack after displacement.

G-18

Low or medium frequency. A frequency range between
190 and 535 kHz with the medium frequency above 300
kHz. Generally associated with nondirectional beacons
transmitting a continuous carrier with either a 400 or 1,020
Hz modulation.
Lubber line. The reference line used in a magnetic compass
or heading indicator.

M
MAA. See maximum authorized altitude.
MAC. See mean aerodynamic chord.
Mach number. The ratio of the true airspeed of the aircraft
to the speed of sound in the same atmospheric conditions,
named in honor of Ernst Mach, late 19th century physicist.
Mach meter. The instrument that displays the ratio of the
speed of sound to the true airspeed an aircraft is flying.
Magnetic bearing (MB). The direction to or from a radio
transmitting station measured relative to magnetic north.
Magnetic compass. A device for determining direction
measured from magnetic north.
Magnetic dip. A vertical attraction between a compass
needle and the magnetic poles. The closer the aircraft is to a
pole, the more severe the effect.
Magnetic heading (MH). The direction an aircraft is pointed
with respect to magnetic north.
Magneto. A self-contained, engine-driven unit that supplies
electrical current to the spark plugs; completely independent
of the airplane's electrical system. Normally there are two
magnetos per engine.
Magnus effect. Lifting force produced when a rotating
cylinder produces a pressure differential. This is the same
effect that makes a baseball curve or a golf ball slice.
Mandatory altitude. An altitude depicted on an instrument
approach chart with the altitude value both underscored and
overscored. Aircraft are required to maintain altitude at the
depicted value.
Mandatory block altitude. An altitude depicted on an
instrument approach chart with two underscored and
overscored altitude values between which aircraft are
required to maintain altitude.

Maneuverability. Ability of an aircraft to change directions
along a flight path and withstand the stresses imposed upon it.

MB. See magnetic bearing.
MCA. See minimum crossing altitude.

Maneuvering speed (VA). The design maneuvering speed.
Operating at or below design maneuvering speed does not
provide structural protection against multiple full control
inputs in one axis or full control inputs in more than one
axis at the same time.
Manifold absolute pressure. The absolute pressure of the
fuel/air mixture within the intake manifold, usually indicated
in inches of mercury.
MAP. See missed approach point.
Margin identification. The top and bottom areas on an
instrument approach chart that depict information about
the procedure, including airport location and procedure
identification.
Marker beacon. A low-powered transmitter that directs its
signal upward in a small, fan-shaped pattern. Used along the
flight path when approaching an airport for landing, marker
beacons indicate both aurally and visually when the aircraft
is directly over the facility.

MDA. See minimum descent altitude.
MEA. See minimum en route altitude.
Mean aerodynamic chord (MAC). The average distance
from the leading edge to the trailing edge of the wing.
Mean sea level. The average height of the surface of the
sea at a particular location for all stages of the tide over a
19-year period.
MEL. See minimum equipment list.
Meridians. Lines of longitude.
Mesophere. A layer of the atmosphere directly above the
stratosphere.
METAR. See Aviation Routine Weather Report.
MFD. See multi-function display.

Mass. The amount of matter in a body.

MH. See magnetic heading.

Maximum altitude. An altitude depicted on an instrument
approach chart with overscored altitude value at which or
below aircraft are required to maintain altitude.

MHz. Megahertz.

Maximum authorized altitude (MAA). A published altitude
representing the maximum usable altitude or flight level for
an airspace structure or route segment.
Maximum landing weight. The greatest weight that an
airplane normally is allowed to have at landing.
Maximum ramp weight. The total weight of a loaded aircraft,
including all fuel. It is greater than the takeoff weight due to the
fuel that will be burned during the taxi and runup operations.
Ramp weight may also be referred to as taxi weight.
Maximum takeoff weight. The maximum allowable weight
for takeoff.
Maximum weight. The maximum authorized weight of
the aircraft and all of its equipment as specified in the Type
Certificate Data Sheets (TCDS) for the aircraft.

Microburts. A strong downdraft which normally occurs
over horizontal distances of 1 NM or less and vertical
distances of less than 1,000 feet. In spite of its small
horizontal scale, an intense microburst could induce
windspeeds greater than 100 knots and downdrafts as strong
as 6,000 feet per minute.
Microwave landing system (MLS). A precision instrument
approach system operating in the microwave spectrum which
normally consists of an azimuth station, elevation station,
and precision distance measuring equipment.
Mileage breakdown. A fix indicating a course change
that appears on the chart as an "x" at a break between two
segments of a federal airway.
Military operations area (MOA). Airspace established for
the purpose of separating certain military training activities
from IFR traffic.

Maximum zero fuel weight (GAMA). The maximum
weight, exclusive of usable fuel.
G-19

Military training route (MTR). Airspace of defined vertical
and lateral dimensions established for the conduct of military
training at airspeeds in excess of 250 knots indicated airspeed
(KIAS).
Minimum altitude. An altitude depicted on an instrument
approach chart with the altitude value underscored. Aircraft
are required to maintain altitude at or above the depicted value.
Minimum crossing altitude (MCA). The lowest allowed
altitude at certain fixes an aircraft must cross when proceeding
in the direction of a higher minimum en route altitude (MEA).
Minimum descent altitude (MDA). The lowest altitude (in
feet MSL) to which descent is authorized on final approach,
or during circle-to-land maneuvering in execution of a
nonprecision approach.

Minimums section. The area on an IAP chart that displays the
lowest altitude and visibility requirements for the approach.
Missed approach. A maneuver conducted by a pilot when
an instrument approach cannot be completed to a landing.
Missed approach point (MAP). A point prescribed in each
instrument approach at which a missed approach procedure
shall be executed if the required visual reference has not
been established.
Mixed ice. A mixture of clear ice and rime ice.
MLS. See microwave landing system.
MM. Middle marker.
MOA. See military operations area.

Minimum drag. The point on the total drag curve where
the lift-to-drag ratio is the greatest. At this speed, total drag
is minimized.

MOCA. See minimum obstruction clearance altitude.
Mode C. Altitude reporting transponder mode.

Minimum en route altitude (MEA). The lowest published
altitude between radio fixes that ensures acceptable
navigational signal coverage and meets obstacle clearance
requirements between those fixes.
Minimum equipment list (MEL). A list developed for larger
aircraft that outlines equipment that can be inoperative for
various types of flight including IFR and icing conditions. This
list is based on the master minimum equipment list (MMEL)
developed by the FAA and must be approved by the FAA for
use. It is specific to an individual aircraft make and model.
Minimum obstruction clearance altitude (MOCA). The
lowest published altitude in effect between radio fixes on VOR
airways, off-airway routes, or route segments, which meets
obstacle clearance requirements for the entire route segment
and which ensures acceptable navigational signal coverage
only within 25 statute (22 nautical) miles of a VOR.
Minimum reception altitude (MRA). The lowest altitude
at which an airway intersection can be determined.
Minimum safe altitude (MSA). The minimum altitude
depicted on approach charts which provides at least 1,000 feet
of obstacle clearance for emergency use within a specified
distance from the listed navigation facility.

Moment. The product of the weight of an item multiplied
by its arm. Moments are expressed in pound-inches (lb-in).
Total moment is the weight of the airplane multiplied by the
distance between the datum and the CG.
Moment arm. The distance from a datum to the applied force.
Moment index (or index). A moment divided by a constant
such as 100, 1,000, or 10,000. The purpose of using a moment
index is to simplify weight and balance computations of
airplanes where heavy items and long arms result in large,
unmanageable numbers.
Monocoque. A shell-like fuselage design in which the
stressed outer skin is used to support the majority of imposed
stresses. Monocoque fuselage design may include bulkheads
but not stringers.
Monoplanes. Airplanes with a single set of wings.
Movable slat. A movable auxiliary airfoil on the leading edge
of a wing. It is closed in normal flight but extends at high
angles of attack. This allows air to continue flowing over the
top of the wing and delays airflow separation.
MRA. See minimum reception altitude.

Minimum vectoring altitude (MVA). An IFR altitude lower
than the minimum en route altitude (MEA) that provides
terrain and obstacle clearance.

MSA. See minimum safe altitude.
MSL. See mean sea level.

G-20

National Transportation Safety Board (NTSB). A United
States Government independent organization responsible for
Multi-function display (MFD). Small screen (CRT or LCD)
 investigations of accidents involving aviation, highways,
in an aircraft that can be used to display information to the
 waterways, pipelines, and railroads in the United States.
pilot in numerous configurable ways. Often an MFD will be
 NTSB is charged by congress to investigate every civil
aviation accident in the United States.
used in concert with a primary flight display.

MVA. See minimum vectoring altitude.

NAVAID. Navigational aid.
MTR. See military training route.

N

N1. Rotational speed of the low pressure compressor in a
turbine engine.
N2. Rotational speed of the high pressure compressor in a
turbine engine.
Nacelle. A streamlined enclosure on an aircraft in which
an engine is mounted. On multiengine propeller-driven
airplanes, the nacelle is normally mounted on the leading
edge of the wing.
NACG. See National Aeronautical Charting Group.
NAS. See National Airspace System.
National Airspace System (NAS). The common network of
United States airspace—air navigation facilities, equipment
and services, airports or landing areas; aeronautical charts,
information and services; rules, regulations and procedures,
technical information; and manpower and material.
National Aeronautical Charting Group (NACG). A
Federal agency operating under the FAA, responsible for
publishing charts such as the terminal procedures and en
route charts.
National Route Program (NRP). A set of rules and
procedures designed to increase the flexibility of user flight
planning within published guidelines.
National Security Area (NSA). Areas consisting of airspace of
defined vertical and lateral dimensions established at locations
where there is a requirement for increased security and safety
of ground facilities. Pilots are requested to voluntarily avoid
flying through the depicted NSA. When it is necessary to
provide a greater level of security and safety, flight in NSAs
may be temporarily prohibited. Regulatory prohibitions are
disseminated via NOTAMs.

NAV/COM. Navigation and communication radio.
NDB. See nondirectional radio beacon.
Negative static stability. The initial tendency of an aircraft
to continue away from the original state of equilibrium after
being disturbed.
Neutral static stability. The initial tendency of an aircraft
to remain in a new condition after its equilibrium has been
disturbed.
NM. Nautical mile.
NOAA. National Oceanic and Atmospheric Administration.
No-gyro approach. A radar approach that may be used in
case of a malfunctioning gyro-compass or directional gyro.
Instead of providing the pilot with headings to be flown,
the controller observes the radar track and issues control
instructions "turn right/left" or "stop turn," as appropriate.
Nondirectional radio beacon (NDB). A ground-based radio
transmitter that transmits radio energy in all directions.
Nonprecision approach. A standard instrument approach
procedure in which only horizontal guidance is provided.
No procedure turn (NoPT). Term used with the appropriate
course and altitude to denote that the procedure turn is not
required.
NoPT. See no procedure turn.
NOTAM. See Notice to Airmen.
Notice to Airmen (NOTAM). A notice filed with an aviation
authority to alert aircraft pilots of any hazards en route or at
a specific location. The authority in turn provides means of
disseminating relevant NOTAMs to pilots.

G-21

NRP. See National Route Program.
NSA. See National Security Area.
NTSB. See National Transportation Safety Board.
NWS. National Weather Service.

O
Obstacle departure procedures (ODP). A preplanned
instrument flight rule (IFR) departure procedure printed for
pilot use in textual or graphic form to provide obstruction
clearance via the least onerous route from the terminal area
to the appropriate en route structure. ODPs are recommended
for obstruction clearance and may be flown without ATC
clearance unless an alternate departure procedure (SID or
radar vector) has been specifically assigned by ATC.

Outside air temperature (OAT). The measured or indicated
air temperature (IAT) corrected for compression and friction
heating. Also referred to as true air temperature.
Overcontrolling. Using more movement in the control
column than is necessary to achieve the desired pitch-and­
bank condition.
Overboost. A condition in which a reciprocating engine
has exceeded the maximum manifold pressure allowed by
the manufacturer. Can cause damage to engine components.
Overpower. To use more power than required for the purpose
of achieving a faster rate of airspeed change.

P
P-static. See precipitation static.
PAPI. See precision approach path indicator.

Obstruction lights. Lights that can be found both on and off
an airport to identify obstructions.

PAR. See precision approach radar.

Occluded front. A frontal occlusion occurs when a fastmoving cold front catches up with a slow moving warm front.
The difference in temperature within each frontal system is
a major factor in determining whether a cold or warm front
occlusion occurs.

Parallels. Lines of latitude.

ODP. See obstacle departure procedures.

Payload (GAMA). The weight of occupants, cargo, and
baggage.

Parasite drag. Drag caused by the friction of air moving
over the aircraft structure; its amount varies directly with
the airspeed.

OM. Outer marker.
Omission error. The failure to anticipate significant
instrument indications following attitude changes; for
example, concentrating on pitch control while forgetting
about heading or roll information, resulting in erratic control
of heading and bank.
Optical illusion. A misleading visual image. For the
purpose of this handbook, the term refers to the brain's
misinterpretation of features on the ground associated
with landing, which causes a pilot to misread the spatial
relationships between the aircraft and the runway.
Orientation. Awareness of the position of the aircraft and of
oneself in relation to a specific reference point.
Otolith organ. An inner ear organ that detects linear
acceleration and gravity orientation.
Outer marker. A marker beacon at or near the glideslope
intercept altitude of an ILS approach. It is normally located
four to seven miles from the runway threshold on the
extended centerline of the runway.
G-22

Personality. The embodiment of personal traits and
characteristics of an individual that are set at a very early
age and extremely resistant to change.
P-factor. A tendency for an aircraft to yaw to the left due to
the descending propeller blade on the right producing more
thrust than the ascending blade on the left. This occurs when
the aircraft's longitudinal axis is in a climbing attitude in
relation to the relative wind. The P-factor would be to the
right if the aircraft had a counterclockwise rotating propeller.
PFD. See primary flight display.
Phugoid oscillations. Long-period oscillations of an
aircraft around its lateral axis. It is a slow change in pitch
accompanied by equally slow changes in airspeed. Angle
of attack remains constant, and the pilot often corrects for
phugoid oscillations without even being aware of them.
PIC. See pilot in command.
Pilotage. Navigation by visual reference to landmarks.

Pilot in command (PIC). The pilot responsible for the
operation and safety of an aircraft.
Pilot report (PIREP). Report of meteorological phenomena
encountered by aircraft.
Pilot's Operating Handbook/Airplane Flight Manual
(POH/AFM). FAA-approved documents published by the
airframe manufacturer that list the operating conditions for
a particular model of aircraft.

Power. Implies work rate or units of work per unit of time,
and as such, it is a function of the speed at which the force is
developed. The term "power required" is generally associated
with reciprocating engines.
Powerplant. A complete engine and propeller combination
with accessories.
Precession. The characteristic of a gyroscope that causes an
applied force to be felt, not at the point of application, but
90° from that point in the direction of rotation.

PIREP. See pilot report.
Pitot pressure. Ram air pressure used to measure airspeed.
Pitot-static head. A combination pickup used to sample pitot
pressure and static air pressure.
Plan view. The overhead view of an approach procedure on
an instrument approach chart. The plan view depicts the routes
that guide the pilot from the en route segments to the IAF.
Planform. The shape or form of a wing as viewed from
above. It may be long and tapered, short and rectangular, or
various other shapes.
Pneumatic. Operation by the use of compressed air.
POH/AFM. See Pilot's Operating Handbook/Airplane
Flight Manual.
Point-in-space approach. A type of helicopter instrument
approach procedure to a missed approach point more than
2,600 feet from an associated helicopter landing area.
Poor judgment chain. A series of mistakes that may lead
to an accident or incident. Two basic principles generally
associated with the creation of a poor judgment chain are:
(1) one bad decision often leads to another; and (2) as a
string of bad decisions grows, it reduces the number of
subsequent alternatives for continued safe flight. ADM is
intended to break the poor judgment chain before it can
cause an accident or incident.
Position error. Error in the indication of the altimeter, ASI,
and VSI caused by the air at the static system entrance not
being absolutely still.

Precipitation. Any or all forms of water particles (rain,
sleet, hail, or snow) that fall from the atmosphere and reach
the surface.
Precipitation static (P-static). A form of radio interference
caused by rain, snow, or dust particles hitting the antenna and
inducing a small radio-frequency voltage into it.
Precision approach. A standard instrument approach
procedure in which both vertical and horizontal guidance
is provided.
Precision approach path indicator (PAPI). A system of
lights similar to the VASI, but consisting of one row of lights
in two- or four-light systems. A pilot on the correct glideslope
will see two white lights and two red lights. See VASI.
Precision approach radar (PAR). A type of radar used
at an airport to guide an aircraft through the final stages of
landing, providing horizontal and vertical guidance. The
radar operator directs the pilot to change heading or adjust
the descent rate to keep the aircraft on a path that allows it
to touch down at the correct spot on the runway.
Precision runway monitor (PRM). System allows
simultaneous, independent instrument flight rules (IFR)
approaches at airports with closely spaced parallel runways.
Preferred IFR routes. Routes established in the major
terminal and en route environments to increase system
efficiency and capacity. IFR clearances are issued based
on these routes, listed in the Chart Supplement U.S. except
when severe weather avoidance procedures or other factors
dictate otherwise.

Position report. A report over a known location as
transmitted by an aircraft to ATC.

Preignition. Ignition occurring in the cylinder before the time
of normal ignition. Preignition is often caused by a local hot
spot in the combustion chamber igniting the fuel-air mixture.

Positive static stability. The initial tendency to return to a
state of equilibrium when disturbed from that state.

Pressure altitude. Altitude above the standard 29.92 "Hg
plane.
G-23

Pressure demand oxygen system. A demand oxygen system
that supplies 100 percent oxygen at sufficient pressure above
the altitude where normal breathing is adequate. Also referred
to as a pressure breathing system.
Prevailing visibility. The greatest horizontal visibility
equaled or exceeded throughout at least half the horizon
circle (which is not necessarily continuous).
Preventive maintenance. Simple or minor preservative
operations and the replacement of small standard parts
not involving complex assembly operation as listed in 14
CFR part 43, appendix A. Certificated pilots may perform
preventive maintenance on any aircraft that is owned or
operated by them provided that the aircraft is not used in air
carrier service.
Primary and supporting. A method of attitude instrument
flying using the instrument that provides the most direct
indication of attitude and performance.
Primary flight display (PFD). A display that provides
increased situational awareness to the pilot by replacing the
traditional six instruments used for instrument flight with
an easy-to-scan display that provides the horizon, airspeed,
altitude, vertical speed, trend, trim, and rate of turn among
other key relevant indications.
PRM. See precision runway monitor.
Procedure turn. A maneuver prescribed when it is necessary
to reverse direction to establish an aircraft on the intermediate
approach segment or final approach course.
Profile view. Side view of an IAP chart illustrating the vertical
approach path altitudes, headings, distances, and fixes.
Prohibited area. Designated airspace within which flight
of aircraft is prohibited.
Propeller. A device for propelling an aircraft that, when
rotated, produces by its action on the air, a thrust approximately
perpendicular to its plane of rotation. It includes the control
components normally supplied by its manufacturer.
Propeller/rotor modulation error. Certain propeller
rpm settings or helicopter rotor speeds can cause the VOR
course deviation indicator (CDI) to fluctuate as much as
±6°. Slight changes to the rpm setting will normally smooth
out this roughness.

G-24

R
Rabbit, the. High-intensity flasher system installed at many
large airports. The flashers consist of a series of brilliant
blue-white bursts of light flashing in sequence along the
approach lights, giving the effect of a ball of light traveling
toward the runway.
Radar. A system that uses electromagnetic waves to identify
the range, altitude, direction, or speed of both moving and
fixed objects such as aircraft, weather formations, and terrain.
The term RADAR was coined in 1941 as an acronym for
Radio Detection and Ranging. The term has since entered
the English language as a standard word, radar, losing the
capitalization in the process.
Radar approach. The controller provides vectors while
monitoring the progress of the flight with radar, guiding
the pilot through the descent to the airport/heliport or to a
specific runway.
Radar services. Radar is a method whereby radio waves are
transmitted into the air and are then received when they have
been reflected by an object in the path of the beam. Range is
determined by measuring the time it takes (at the speed of light)
for the radio wave to go out to the object and then return to the
receiving antenna. The direction of a detected object from a
radar site is determined by the position of the rotating antenna
when the reflected portion of the radio wave is received.
Radar summary chart. A weather product derived from the
national radar network that graphically displays a summary
of radar weather reports.
Radar weather report (SD). A report issued by radar
stations at 35 minutes after the hour, and special reports
as needed. Provides information on the type, intensity, and
location of the echo tops of the precipitation.
Radials. The courses oriented from a station.
Radio or radar altimeter. An electronic altimeter that
determines the height of an aircraft above the terrain by
measuring the time needed for a pulse of radio-frequency
energy to travel from the aircraft to the ground and return.
Radio frequency (RF). A term that refers to alternating
current (AC) having characteristics such that, if the current is
input to antenna, an electromagnetic (EM) field is generated
suitable for wireless broadcasting and/or communications.

Radio magnetic indicator (RMI). An electronic navigation
instrument that combines a magnetic compass with an ADF or
VOR. The card of the RMI acts as a gyro-stabilized magnetic
compass, and shows the magnetic heading the aircraft is flying.
Radiosonde. A weather instrument that observes and reports
meteorological conditions from the upper atmosphere. This
instrument is typically carried into the atmosphere by some
form of weather balloon.
Radio wave. An electromagnetic (EM) wave with frequency
characteristics useful for radio transmission.
RAIM. See receiver autonomous integrity monitoring.
RAM recovery. The increase in thrust as a result of ram air
pressures and density on the front of the engine caused by
air velocity.
Random RNAV routes. Direct routes, based on area
navigation capability, between waypoints defined in terms
of latitude/longitude coordinates, degree-distance fixes, or
offsets from established routes/airways at a specified distance
and direction.
Ranging signals. Transmitted from the GPS satellite, signals
allowing the aircraft's receiver to determine range (distance)
from each satellite.
Rapid decompression. The almost instantaneous loss of
cabin pressure in aircraft with a pressurized cockpit or cabin.
RB. See relative bearing.
RBI. See relative bearing indicator.
RCO. See remote communications outlet.
Receiver autonomous integrity monitoring (RAIM). A
system used to verify the usability of the received GPS signals
and warns the pilot of any malfunction in the navigation
system. This system is required for IFR-certified GPS units.
Recommended altitude. An altitude depicted on an instrument
approach chart with the altitude value neither underscored nor
overscored. The depicted value is an advisory value.
Receiver-transmitter (RT). A system that receives and
transmits a signal and an indicator.

Reduced vertical separation minimum (RVSM). Reduces
the vertical separation between flight levels (FL) 290 and 410
from 2,000 feet to 1,000 feet, and makes six additional FLs
available for operation. Also see DRVSM.
Reference circle (also, distance circle). The circle depicted
in the plan view of an IAP chart that typically has a 10 NM
radius, within which chart the elements are drawn to scale.
Regions of command. The "regions of normal and reversed
command" refers to the relationship between speed and the
power required to maintain or change that speed in flight.
Region of reverse command. Flight regime in which flight
at a higher airspeed requires a lower power setting and a
lower airspeed requires a higher power setting in order to
maintain altitude.
REIL. See runway end identifier lights.
Relative bearing (RB). The angular difference between the
aircraft heading and the direction to the station, measured
clockwise from the nose of the aircraft.
Relative bearing indicator (RBI). Also known as the fixedcard ADF, zero is always indicated at the top of the instrument
and the needle indicates the relative bearing to the station.
Relative humidity. The ratio of the existing amount of
water vapor in the air at a given temperature to the maximum
amount that could exist at that temperature; usually expressed
in percent.
Relative wind. Direction of the airflow produced by an object
moving through the air. The relative wind for an airplane in
flight flows in a direction parallel with and opposite to the
direction of flight; therefore, the actual flight path of the
airplane determines the direction of the relative wind.
Remote communications outlet (RCO). An unmanned
communications facility that is remotely controlled by air
traffic personnel.
Required navigation performance (RNP). A specified level
of accuracy defined by a lateral area of confined airspace in
which an RNP-certified aircraft operates.
Restricted area. Airspace designated under 14 CFR part
73 within which the flight of aircraft, while not wholly
prohibited, is subject to restriction.
Reverse sensing. The VOR needle appearing to indicate the
reverse of normal operation.

G-25

RF. Radio frequency.
Rhodopsin. The photosensitive pigments that initiate the
visual response in the rods of the eye.
Rigging. The final adjustment and alignment of an aircraft
and its flight control system that provides the proper
aerodynamic characteristics.
Rigidity. The characteristic of a gyroscope that prevents its
axis of rotation tilting as the Earth rotates.
Rigidity in space. The principle that a wheel with a heavily
weighted rim spinning rapidly will remain in a fixed position
in the plane in which it is spinning.
Rime ice. Rough, milky, opaque ice formed by the
instantaneous freezing of small supercooled water droplets.

Runway edge lights. A component of the runway lighting
system that is used to outline the edges of runways at night
or during low visibility conditions. These lights are classified
according to the intensity they are capable of producing.
Runway end identifier lights (REIL). A pair of synchronized
flashing lights, located laterally on each side of the runway
threshold, providing rapid and positive identification of the
approach end of a runway.
Runway visibility value (RVV). The visibility determined
for a particular runway by a transmissometer.
Runway visual range (RVR). The instrumentally derived
horizontal distance a pilot should be able to see down the
runway from the approach end, based on either the sighting
of high-intensity runway lights, or the visual contrast of
other objects.
RVR. See runway visual range.

Risk. The future impact of a hazard that is not eliminated
or controlled.
Risk elements. There are four fundamental risk elements
in aviation: the pilot, the aircraft, the environment, and the
type of operation that comprise any given aviation situation.
Risk management. The part of the decision-making
process which relies on situational awareness, problem
recognition, and good judgment to reduce risks associated
with each flight.
RMI. See radio magnetic indicator.
RNAV. See area navigation.
RNP. See required navigation performance.
RT. See receiver-transmitter.
Rudder. The movable primary control surface mounted on
the trailing edge of the vertical fin of an airplane. Movement
of the rudder rotates the airplane about its vertical axis.
Ruddervator. A pair of control surfaces on the tail of an
aircraft arranged in the form of a V. These surfaces, when
moved together by the control wheel, serve as elevators,
and when moved differentially by the rudder pedals, serve
as a rudder.
Runway centerline lights. Runway lighting which consists
of flush centerline lights spaced at 50-foot intervals beginning
75 feet from the landing threshold.

G-26

RVV. See runway visibility value.

S
SA. See selective availability.
St. Elmo's Fire. A corona discharge which lights up the
aircraft surface areas where maximum static discharge occurs.
Satellite ephemeris data. Data broadcast by the GPS
satellite containing very accurate orbital data for that satellite,
atmospheric propagation data, and satellite clock error data.
Sea breeze. A coastal breeze blowing from sea to land
caused by the temperature difference when the land surface
is warmer than the sea surface. The sea breeze usually occurs
during the day and alternates with the land breeze that blows
in the opposite direction at night.
Sea level engine. A reciprocating aircraft engine having a
rated takeoff power that is producible only at sea level.
Scan. The first fundamental skill of instrument flight,
also known as "cross-check;" the continuous and logical
observation of instruments for attitude and performance
information.
Sectional aeronautical charts. Designed for visual
navigation of slow- or medium-speed aircraft. Topographic
information on these charts features the portrayal of relief,
and a judicious selection of visual check points for VFR
flight. Aeronautical information includes visual and radio
aids to navigation, airports, controlled airspace, restricted
areas, obstructions and related data.

SDF. See simplified directional facility.
Selective availability (SA). A satellite technology permitting
the Department of Defense (DOD) to create, in the interest
of national security, a significant clock and ephemeris error
in the satellites, resulting in a navigation error.
Semicircular canal. An inner ear organ that detects angular
acceleration of the body.
Semimonocoque. A fuselage design that includes a
substructure of bulkheads and/or formers, along with stringers,
to support flight loads and stresses imposed on the fuselage.
Sensitive altimeter. A form of multipointer pneumatic
altimeter with an adjustable barometric scale that allows the
reference pressure to be set to any desired level.
Service ceiling. The maximum density altitude where the best
rate-of-climb airspeed will produce a 100-feet-per-minute
climb at maximum weight while in a clean configuration
with maximum continuous power.
Servo. A motor or other form of actuator which receives a
small signal from the control device and exerts a large force
to accomplish the desired work.
Servo tab. An auxiliary control mounted on a primary control
surface, which automatically moves in the direction opposite
the primary control to provide an aerodynamic assist in the
movement of the control.
SIDS. See standard instrument departure procedures.
SIGMET. The acronym for Significant Meteorological
information. A weather advisory in abbreviated plain
language concerning the occurrence or expected occurrence
of potentially hazardous en route weather phenomena that may
affect the safety of aircraft operations. SIGMET is warning
information, hence it is of highest priority among other types
of meteorological information provided to the aviation users.
Signal-to-noise ratio. An indication of signal strength
received compared to background noise, which is a measure
of the adequacy of the received signal.
Significant weather prognostic. Presents four panels
showing forecast significant weather.
Simplex. Transmission and reception on the same frequency.

Simplified directional facility (SDF). A NAVAID used
for nonprecision instrument approaches. The final approach
course is similar to that of an ILS localizer; however, the
SDF course may be offset from the runway, generally not
more than 3°, and the course may be wider than the localizer,
resulting in a lower degree of accuracy.
Single-pilot resource management (SRM). The ability
for a pilot to manage all resources effectively to ensure the
outcome of the flight is successful.
Situational awareness. Pilot knowledge of where the aircraft
is in regard to location, air traffic control, weather, regulations,
aircraft status, and other factors that may affect flight.
Skidding turn. An uncoordinated turn in which the rate of
turn is too great for the angle of bank, pulling the aircraft to
the outside of the turn.
Skills and procedures. The procedural, psychomotor, and
perceptual skills used to control a specific aircraft or its
systems. They are the airmanship abilities that are gained
through conventional training, are perfected, and become
almost automatic through experience.
Skin friction drag. Drag generated between air molecules
and the solid surface of the aircraft.
Slant range. The horizontal distance from the aircraft antenna
to the ground station, due to line-of-sight transmission of the
DME signal.
Slaved compass. A system whereby the heading gyro is
"slaved to," or continuously corrected to bring its direction
readings into agreement with a remotely located magnetic
direction sensing device (usually a flux valve or flux gate
compass).
Slipping turn. An uncoordinated turn in which the aircraft
is banked too much for the rate of turn, so the horizontal lift
component is greater than the centrifugal force, pulling the
aircraft toward the inside of the turn.
Small airplane. An airplane of 12,500 pounds or less
maximum certificated takeoff weight.
Somatogravic illusion. The misperception of being
in a nose-up or nose-down attitude, caused by a rapid
acceleration or deceleration while in flight situations that
lack visual reference.

G-27

Spatial disorientation. The state of confusion due to
misleading information being sent to the brain from various
sensory organs, resulting in a lack of awareness of the aircraft
position in relation to a specific reference point.

Stability. The inherent quality of an airplane to correct for
conditions that may disturb its equilibrium, and to return
or to continue on the original flight path. It is primarily an
airplane design characteristic.

Special flight permit. A flight permit issued to an aircraft
that does not meet airworthiness requirements but is capable
of safe flight. A special flight permit can be issued to move
an aircraft for the purposes of maintenance or repair, buyer
delivery, manufacturer flight tests, evacuation from danger,
or customer demonstration. Also referred to as a ferry permit.

Stagnant hypoxia. A type of hypoxia that results when the
oxygen-rich blood in the lungs is not moving to the tissues
that need it.

Special use airspace. Airspace in which flight activities are
subject to restrictions that can create limitations on the mixed
use of airspace. Consists of prohibited, restricted, warning,
military operations, and alert areas.
Special fuel consumption. The amount of fuel in pounds
per hour consumed or required by an engine per brake
horsepower or per pound of thrust.
Speed. The distance traveled in a given time.
Spin. An aggravated stall that results in an airplane
descending in a helical, or corkscrew path.
Spiral instability. A condition that exists when the static
directional stability of the airplane is very strong as compared
to the effect of its dihedral in maintaining lateral equilibrium.
Spiraling slipstream. The slipstream of a propeller-driven
airplane rotates around the airplane. This slipstream strikes
the left side of the vertical fin, causing the aircraft to yaw
slightly. Rudder offset is sometimes used by aircraft designers
to counteract this tendency.
Spoilers. High-drag devices that can be raised into the air
flowing over an airfoil, reducing lift and increasing drag.
Spoilers are used for roll control on some aircraft. Deploying
spoilers on both wings at the same time allows the aircraft
to descend without gaining speed. Spoilers are also used to
shorten the ground roll after landing.
SRM. See single-pilot resource management.
SSR. See secondary surveillance radar.
SSV. See standard service volume.
Stabilator. A single-piece horizontal tail surface on an
airplane that pivots around a central hinge point. A stabilator
serves the purposes of both the horizontal stabilizer and the
elevators.
G-28

Stall. A rapid decrease in lift caused by the separation of
airflow from the wing's surface, brought on by exceeding
the critical angle of attack. A stall can occur at any pitch
attitude or airspeed.
Standard atmosphere. At sea level, the standard atmosphere
consists of a barometric pressure of 29.92 inches of mercury
("Hg) or 1013.2 millibars, and a temperature of 15 °C (59
°F). Pressure and temperature normally decrease as altitude
increases. The standard lapse rate in the lower atmosphere for
each 1,000 feet of altitude is approximately 1 "Hg and 2 °C
(3.5 °F). For example, the standard pressure and temperature
at 3,000 feet mean sea level (MSL) are 26.92 "Hg (29.92
"Hg – 3 "Hg) and 9 °C (15 °C – 6 °C).
Standard empty weight (GAMA). This weight consists of
the airframe, engines, and all items of operating equipment
that have fixed locations and are permanently installed in the
airplane including fixed ballast, hydraulic fluid, unusable
fuel, and full engine oil.
Standard holding pattern. A holding pattern in which all
turns are made to the right.
Standard instrument departure procedures (SIDS).
Published procedures to expedite clearance delivery and to
facilitate transition between takeoff and en route operations.
Standard rate turn. A turn in which an aircraft changes its
direction at a rate of 3° per second (360° in 2 minutes) for
low- or medium-speed aircraft. For high-speed aircraft, the
standard rate turn is 1½° per second (360° in 4 minutes).
Standard service volume (SSV). Defines the limits of the
volume of airspace which the VOR serves.
Standard terminal arrival route (STAR). A preplanned
IFR ATC arrival procedure published for pilot use in graphic
and/or textual form.
Standard weights. Weights established for numerous items
involved in weight and balance computations. These weights
should not be used if actual weights are available.

STAR. See standard terminal arrival route.
Static longitudinal stability. The aerodynamic pitching
moments required to return the aircraft to the equilibrium
angle of attack.
Static pressure. Pressure of air that is still or not moving,
measured perpendicular to the surface of the aircraft.
Static stability. The initial tendency an aircraft displays
when disturbed from a state of equilibrium.
Station. A location in the airplane that is identified by a
number designating its distance in inches from the datum.
The datum is, therefore, identified as station zero. An item
located at station +50 would have an arm of 50 inches.

Supercooled water droplets. Water droplets that have been
cooled below the freezing point, but are still in a liquid state.
Surface analysis chart. A report that depicts an analysis of
the current surface weather. Shows the areas of high and low
pressure, fronts, temperatures, dewpoints, wind directions
and speeds, local weather, and visual obstructions.
Synchro. A device used to transmit indications of angular
movement or position from one location to another.
Synthetic vision. A realistic display depiction of the aircraft
in relation to terrain and flight path.

T
TAA. See terminal arrival area.

Stationary front. A front that is moving at a speed of less
than 5 knots.

TACAN. See tactical air navigation.

Steep turns. In instrument flight, any turn greater than standard
rate; in visual flight, anything greater than a 45° bank.

Tactical air navigation (TACAN). An electronic navigation
system used by military aircraft, providing both distance and
direction information.

Stepdown fix. The point after which additional descent is
permitted within a segment of an IAP.
Strapdown system. An INS in which the accelerometers
and gyros are permanently "strapped down" or aligned with
the three axes of the aircraft.

Takeoff decision speed (V1). Per 14 CFR section 23.51:
"the calibrated airspeed on the ground at which, as a result
of engine failure or other reasons, the pilot assumed to have
made a decision to continue or discontinue the takeoff."

Stress. The body's response to demands placed upon it.

Takeoff distance. The distance required to complete an
all-engines operative takeoff to the 35-foot height. It must
be at least 15 percent less than the distance required for a
one-engine inoperative engine takeoff. This distance is not
normally a limiting factor as it is usually less than the oneengine inoperative takeoff distance.

Stress management. The personal analysis of the kinds of
stress experienced while flying, the application of appropriate
stress assessment tools, and other coping mechanisms.

Takeoff safety speed (V2). Per 14 CFR part 1: "A referenced
airspeed obtained after lift-off at which the required one­
engine-inoperative climb performance can be achieved."

Structural icing. The accumulation of ice on the exterior
of the aircraft.

TAWS. See terrain awareness and warning system.

Stratoshere. A layer of the atmosphere above the tropopause
extending to a height of approximately 160,000 feet.

Sublimation. Process by which a solid is changed to a gas
without going through the liquid state.
Suction relief valve. A relief valve in an instrument vacuum
system required to maintain the correct low pressure inside
the instrument case for the proper operation of the gyros.
Supercharger. An engine- or exhaust-driven air compressor
used to provide additional pressure to the induction air so the
engine can produce additional power.

Taxiway lights. Omnidirectional lights that outline the edges
of the taxiway and are blue in color.
Taxiway turnoff lights. Lights that are flush with the runway
which emit a steady green color.
TCAS. See traffic alert collision avoidance system.
TCH. See threshold crossing height.
TDZE. See touchdown zone elevation.

G-29

TEC. See Tower En Route Control.
Technique. The manner in which procedures are executed.
Telephone information briefing service (TIBS). An FSS
service providing continuously updated automated telephone
recordings of area and/or route weather, airspace procedures,
and special aviation-oriented announcements.
Temporary flight restriction (TFR). Restriction to flight
imposed in order to:
1.	 Protect persons and property in the air or on the surface
from an existing or imminent flight associated hazard;
2.	 Provide a safe environment for the operation of
disaster relief aircraft;
3.	 Prevent an unsafe congestion of sightseeing aircraft
above an incident;
4.	 Protect the President, Vice President, or other public
figures; and,
5.	 Provide a safe environment for space agency
operations.
Pilots are expected to check appropriate NOTAMs during
flight planning when conducting flight in an area where a
temporary flight restriction is in effect.
Tension. Maintaining an excessively strong grip on the control
column, usually resulting in an overcontrolled situation.
Terminal aerodrome forecast (TAF). A report established
for the 5 statute mile radius around an airport. Utilizes the
same descriptors and abbreviations as the METAR report.
Terminal arrival area (TAA). A procedure to provide a
new transition method for arriving aircraft equipped with
FMS and/or GPS navigational equipment. The TAA contains
a "T" structure that normally provides a NoPT for aircraft
using the approach.
Terminal instrument approach procedure (TERP).
Prescribes standardized methods for use in designing
instrument flight procedures.
TERP. See terminal instrument approach procedure.
Terminal radar service areas (TRSA). Areas where
participating pilots can receive additional radar services. The
purpose of the service is to provide separation between all
IFR operations and participating VFR aircraft.

G-30

Terrain awareness and warning system (TAWS). A
timed-based system that provides information concerning
potential hazards with fixed objects by using GPS positioning
and a database of terrain and obstructions to provide true
predictability of the upcoming terrain and obstacles.
TFR. See temporary flight restriction.
Thermosphere. The last layer of the atmosphere that begins
above the mesosphere and gradually fades away into space.
Threshold crossing height (TCH). The theoretical height
above the runway threshold at which the aircraft's glideslope
antenna would be if the aircraft maintained the trajectory
established by the mean ILS glideslope or MLS glidepath.
Thrust. The force which imparts a change in the velocity of a
mass. This force is measured in pounds but has no element of
time or rate. The term "thrust required" is generally associated
with jet engines. A forward force which propels the airplane
through the air.
Thrust (aerodynamic force). The forward aerodynamic
force produced by a propeller, fan, or turbojet engine as it
forces a mass of air to the rear, behind the aircraft.
Thrust line. An imaginary line passing through the center of
the propeller hub, perpendicular to the plane of the propeller
rotation.
Time and speed table. A table depicted on an instrument
approach procedure chart that identifies the distance from the
FAF to the MAP, and provides the time required to transit
that distance based on various groundspeeds.
Timed turn. A turn in which the clock and the turn
coordinator are used to change heading a definite number
of degrees in a given time.
TIS. See traffic information service.
Title 14 of the Code of Federal Regulations (14 CFR).
Includes the federal aviation regulations governing the
operation of aircraft, airways, and airmen.
Torque. (1) A resistance to turning or twisting. (2) Forces that
produce a twisting or rotating motion. (3) In an airplane, the
tendency of the aircraft to turn (roll) in the opposite direction
of rotation of the engine and propeller. (4) In helicopters with
a single, main rotor system, the tendency of the helicopter
to turn in the opposite direction of the main rotor rotation.

Torquemeter. An instrument used with some of the larger
reciprocating engines and turboprop or turboshaft engines to
measure the reaction between the propeller reduction gears
and the engine case.

Transponder code. One of 4,096 four-digit discrete codes
ATC assigns to distinguish between aircraft.
Trend. Immediate indication of the direction of aircraft
movement, as shown on instruments.

Total drag. The sum of the parasite drag and induced drag.
Touchdown zone elevation (TDZE). The highest elevation
in the first 3,000 feet of the landing surface, TDZE is
indicated on the instrument approach procedure chart when
straight-in landing minimums are authorized.
Touchdown zone lights. Two rows of transverse light bars
disposed symmetrically about the runway centerline in the
runway touchdown zone.

Tricycle gear. Landing gear employing a third wheel located
on the nose of the aircraft.
Trim. To adjust the aerodynamic forces on the control
surfaces so that the aircraft maintains the set attitude without
any control input.
Trim tab. A small auxiliary hinged portion of a movable
control surface that can be adjusted during flight to a position
resulting in a balance of control forces.

Tower En Route Control (TEC). The control of IFR
en route traffic within delegated airspace between two
or more adjacent approach control facilities, designed to
expedite traffic and reduce control and pilot communication
requirements.

Tropopause. The boundary layer between the troposphere
and the stratosphere which acts as a lid to confine most of the
water vapor, and the associated weather, to the troposphere.

TPP. See United States Terminal Procedures Publication.

Troposphere. The layer of the atmosphere extending from
the surface to a height of 20,000 to 60,000 feet, depending
on latitude.

Track. The actual path made over the ground in flight.
Tracking. Flying a heading that will maintain the desired
track to or from the station regardless of crosswind conditions.
Traffic Alert Collision Avoidance System (TCAS).
An airborne system developed by the FAA that operates
independently from the ground-based Air Traffic Control
system. Designed to increase flight deck awareness of
proximate aircraft and to serve as a "last line of defense" for
the prevention of midair collisions.
Traffic information service (TIS). A ground-based service
providing information to the flight deck via data link using
the S-mode transponder and altitude encoder to improve the
safety and efficiency of "see and avoid" flight through an
automatic display that informs the pilot of nearby traffic.
Trailing edge. The portion of the airfoil where the airflow
over the upper surface rejoins the lower surface airflow.
Transcribed Weather Broadcast (TWEB). An FSS service,
available in Alaska only, providing continuously updated
automated broadcast of meteorological and aeronautical data
over selected L/MF and VOR NAVAIDs.

True airspeed. Actual airspeed, determined by applying a
correction for pressure altitude and temperature to the CAS.
True altitude. The vertical distance of the airplane above
sea level—the actual altitude. It is often expressed as feet
above mean sea level (MSL). Airport, terrain, and obstacle
elevations on aeronautical charts are true altitudes.
Truss. A fuselage design made up of supporting structural
members that resist deformation by applied loads. The trusstype fuselage is constructed of steel or aluminum tubing.
Strength and rigidity is achieved by welding the tubing
together into a series of triangular shapes, called trusses.
T-tail. An aircraft with the horizontal stabilizer mounted on
the top of the vertical stabilizer, forming a T.
Turbine discharge pressure. The total pressure at the
discharge of the low-pressure turbine in a dual-turbine axialflow engine.
Turbine engine. An aircraft engine which consists of an
air compressor, a combustion section, and a turbine. Thrust
is produced by increasing the velocity of the air flowing
through the engine.

Transponder. The airborne portion of the ATC radar
beacon system.

G-31

Turbocharger. An air compressor driven by exhaust gases,
which increases the pressure of the air going into the engine
through the carburetor or fuel injection system.

Underpower. Using less power than required for the purpose
of achieving a faster rate of airspeed change.

Turbofan engine. A fanlike turbojet engine designed to
create additional thrust by diverting a secondary airflow
around the combustion chamber.

United States Terminal Procedures Publication (TPP).
Booklets published in regional format by FAA Aeronautical
Navigation Products (AeroNav Products) that include DPs,
STARs, IAPs, and other information pertinent to IFR flight.

Turbojet engine. A turbine engine which produces its thrust
entirely by accelerating the air through the engine.

Unusual attitude. An unintentional, unanticipated, or
extreme aircraft attitude.

Turboprop engine. A turbine engine which drives a
propeller through a reduction gearing arrangement. Most of
the energy in the exhaust gases is converted into torque, rather
than using its acceleration to drive the aircraft.

Useful load. The weight of the pilot, copilot, passengers,
baggage, usable fuel, and drainable oil. It is the basic empty
weight subtracted from the maximum allowable gross weight.
This term applies to general aviation aircraft only.

Turboshaft engine. A gas turbine engine that delivers power
through a shaft to operate something other than a propeller.

User-defined waypoints. Waypoint location and other data
which may be input by the user, this is the only GPS database
information that may be altered (edited) by the user.

Turn-and-slip indicator. A flight instrument consisting
of a rate gyro to indicate the rate of yaw and a curved glass
inclinometer to indicate the relationship between gravity and
centrifugal force. The turn-and-slip indicator indicates the
relationship between angle of bank and rate of yaw. Also
called a turn-and-bank indicator.
Turn coordinator. A rate gyro that senses both roll and
yaw due to the gimbal being canted. Has largely replaced
the turn-and-slip indicator in modern aircraft.
TWEB. See Transcribed Weather Broadcast.

U
UHF. See ultra-high frequency.
Ultra-high frequency (UHF). The range of electromagnetic
frequencies between 300 MHz and 3,000 MHz.
Ulitimate load factor. In stress analysis, the load that
causes physical breakdown in an aircraft or aircraft
component during a strength test, or the load that according
to computations, should cause such a breakdown.
Uncaging. Unlocking the gimbals of a gyroscopic instrument,
making it susceptible to damage by abrupt flight maneuvers
or rough handling.
Uncontrolled airspace. Class G airspace that has not been
designated as Class A, B, C, D, or E. It is airspace in which
air traffic control has no authority or responsibility to control
air traffic; however, pilots should remember there are VFR
minimums which apply to this airspace.

G-32

V
V1. See takeoff decision speed.
V2. See takeoff safety speed.
VA. See maneuvering speed.
Vapor lock. A problem that mostly affects gasoline-fuelled
internal combustion engines. It occurs when the liquid fuel
changes state from liquid to gas while still in the fuel delivery
system. This disrupts the operation of the fuel pump, causing
loss of feed pressure to the carburetor or fuel injection system,
resulting in transient loss of power or complete stalling.
Restarting the engine from this state may be difficult. The
fuel can vaporize due to being heated by the engine, by the
local climate or due to a lower boiling point at high altitude.
Variation. Compass error caused by the difference in
the physical locations of the magnetic north pole and the
geographic north pole.
VASI. See visual approach slope indicator.
VDP. See visual descent point.
Vector. A force vector is a graphic representation of a force
and shows both the magnitude and direction of the force.
Vectoring. Navigational guidance by assigning headings.
VEF. Calibrated airspeed at which the critical engine of a
multi-engine aircraft is assumed to fail.

Velocity. The speed or rate of movement in a certain
direction.
Venturi tube. A specially shaped tube attached to the outside
of an aircraft to produce suction to allow proper operation
of gyro instruments.
Vertical axis. An imaginary line passing vertically through
the center of gravity of an aircraft. The vertical axis is called
the z-axis or the yaw axis.
Vertical card compass. A magnetic compass that consists of
an azimuth on a vertical card, resembling a heading indicator
with a fixed miniature airplane to accurately present the
heading of the aircraft. The design uses eddy current damping
to minimize lead and lag during turns.
Vertical speed indicator (VSI). A rate-of-pressure change
instrument that gives an indication of any deviation from a
constant pressure level.
Vertical stability. Stability about an aircraft's vertical axis.
Also called yawing or directional stability.
Very-high frequency (VHF). A band of radio frequencies
falling between 30 and 300 MHz.
Very-high frequency omnidirectional range (VOR).
Electronic navigation equipment in which the flight deck
instrument identifies the radial or line from the VOR station,
measured in degrees clockwise from magnetic north, along
which the aircraft is located.
Vestibule. The central cavity of the bony labyrinth of the
ear, or the parts of the membranous labyrinth that it contains.
VFE. The maximum speed with the flaps extended. The upper
limit of the white arc.
VFR. See visual flight rules.
VFR on top. ATC authorization for an IFR aircraft to operate
in VFR conditions at any appropriate VFR altitude.
VFR over the top. A VFR operation in which an aircraft
operates in VFR conditions on top of an undercast.
VFR terminal area chart. At a scale of 1:250,000, a chart
that depicts Class B airspace, which provides for the control
or segregation of all the aircraft within the Class B airspace.
The chart depicts topographic information and aeronautical
information including visual and radio aids to navigation,
airports, controlled airspace, restricted areas, obstructions,
and related data.

V-G diagram. A chart that relates velocity to load factor. It
is valid only for a specific weight, configuration and altitude
and shows the maximum amount of positive or negative lift
the airplane is capable of generating at a given speed. Also
shows the safe load factor limits and the load factor that the
aircraft can sustain at various speeds.
Victor airways. Airways based on a centerline that extends
from one VOR or VORTAC navigation aid or intersection,
to another navigation aid (or through several navigation aids
or intersections); used to establish a known route for en route
procedures between terminal areas.
Visual approach slope indicator (VASI). A visual aid of
lights arranged to provide descent guidance information
during the approach to the runway. A pilot on the correct
glideslope will see red lights over white lights.
Visual descent point (VDP). A defined point on the final
approach course of a nonprecision straight-in approach
procedure from which normal descent from the MDA to the
runway touchdown point may be commenced, provided the
runway environment is clearly visible to the pilot.
Visual flight rules (VFR). Flight rules adopted by the
FAA governing aircraft flight using visual references. VFR
operations specify the amount of ceiling and the visibility the
pilot must have in order to operate according to these rules.
When the weather conditions are such that the pilot cannot
operate according to VFR, he or she must use instrument
flight rules (IFR).
Visual meteorological conditions (VMC). Meteorological
conditions expressed in terms of visibility, distance from
cloud, and ceiling meeting or exceeding the minimums
specified for VFR.
VLE. Landing gear extended speed. The maximum speed at
which an airplane can be safely flown with the landing gear
extended.
VLO. Landing gear operating speed. The maximum speed for
extending or retracting the landing gear if using an airplane
equipped with retractable landing gear.
VMC. Minimum control airspeed. This is the minimum
flight speed at which a light, twin-engine airplane can be
satisfactorily controlled when an engine suddenly becomes
inoperative and the remaining engine is at takeoff power.
VMC. See visual meteorological conditions.

G-33

VNE. The never-exceed speed. Operating above this speed is
prohibited since it may result in damage or structural failure.
The red line on the airspeed indicator.
V NO. The maximum structural cruising speed. Do not
exceed this speed except in smooth air. The upper limit of
the green arc.
VOR. See very-high frequency omnidirectional range.
VORTAC. A facility consisting of two components, VOR
and TACAN, which provides three individual services: VOR
azimuth, TACAN azimuth, and TACAN distance (DME)
at one site.
VOR test facility (VOT). A ground facility which emits a
test signal to check VOR receiver accuracy. Some VOTs are
available to the user while airborne, while others are limited
to ground use only.
VOT. See VOR test facility.
VSI. See vertical speed indicator.
VS0. The stalling speed or the minimum steady flight speed
in the landing configuration. In small airplanes, this is the
power-off stall speed at the maximum landing weight in the
landing configuration (gear and flaps down). The lower limit
of the white arc.
VS1. The stalling speed or the minimum steady flight speed
obtained in a specified configuration. For most airplanes, this
is the power-off stall speed at the maximum takeoff weight
in the clean configuration (gear up, if retractable, and flaps
up). The lower limit of the green arc.
V-tail. A design which utilizes two slanted tail surfaces to
perform the same functions as the surfaces of a conventional
elevator and rudder configuration. The fixed surfaces act as
both horizontal and vertical stabilizers.
VX. Best angle-of-climb speed. The airspeed at which an
airplane gains the greatest amount of altitude in a given
distance. It is used during a short-field takeoff to clear an
obstacle.
VY. Best rate-of-climb speed. This airspeed provides the
most altitude gain in a given period of time.
VYSE. Best rate-of-climb speed with one engine inoperative.
This airspeed provides the most altitude gain in a given
period of time in a light, twin-engine airplane following an
engine failure.
G-34

W
WAAS. See wide area augmentation system.
Wake turbulence. Wingtip vortices that are created when
an airplane generates lift. When an airplane generates lift,
air spills over the wingtips from the high pressure areas
below the wings to the low pressure areas above them. This
flow causes rapidly rotating whirlpools of air called wingtip
vortices or wake turbulence.
Warm front. The boundary area formed when a warm air
mass contacts and flows over a colder air mass. Warm fronts
cause low ceilings and rain.
Warning area. An area containing hazards to any aircraft
not participating in the activities being conducted in the
area. Warning areas may contain intensive military training,
gunnery exercises, or special weapons testing.
WARP. See weather and radar processing.
Waste gate. A controllable valve in the tailpipe of an aircraft
reciprocating engine equipped with a turbocharger. The valve
is controlled to vary the amount of exhaust gases forced
through the turbocharger turbine.
Waypoint. A designated geographical location used for route
definition or progress-reporting purposes and is defined in
terms of latitude/longitude coordinates.
WCA. See wind correction angle.
Weather and radar processor (WARP). A device that
provides real-time, accurate, predictive, and strategic weather
information presented in an integrated manner in the National
Airspace System (NAS).
Weather depiction chart. Details surface conditions as
derived from METAR and other surface observations.
Weight. The force exerted by an aircraft from the pull of
gravity.
Wide area augmentation system (WAAS). A differential
global positioning system (DGPS) that improves the accuracy
of the system by determining position error from the GPS
satellites, then transmitting the error, or corrective factors,
to the airborne GPS receiver.
Wind correction angle (WCA). The angle between the
desired track and the heading of the aircraft necessary to
keep the aircraft tracking over the desired track.

Wind direction indicators. Indicators that include a
wind sock, wind tee, or tetrahedron. Visual reference will
determine wind direction and runway in use.
Wind shear. A sudden, drastic shift in windspeed, direction,
or both that may occur in the horizontal or vertical plane.
Winds and temperature aloft forecast (FB). A twice daily
forecast that provides wind and temperature forecasts for
specific locations in the contiguous United States.

Z
Zone of confusion. Volume of space above the station where
a lack of adequate navigation signal directly above the VOR
station causes the needle to deviate.
Zulu time. A term used in aviation for coordinated universal
time (UTC) which places the entire world on one time standard.

Wing area. The total surface of the wing (in square feet),
which includes control surfaces and may include wing area
covered by the fuselage (main body of the airplane), and
engine nacelles.
Wings. Airfoils attached to each side of the fuselage and are
the main lifting surfaces that support the airplane in flight.
Wing root. The wing root is the part of the wing on a fixedwing aircraft that is closest to the fuselage. Wing roots
usually bear the highest bending forces in flight and during
landing, and they often have fairings to reduce interference
drag between the wing and the fuselage. The opposite end
of a wing from the wing root is the wing tip.
Wing span. The maximum distance from wingtip to wingtip.
Wingtip vortices. The rapidly rotating air that spills over an
airplane's wings during flight. The intensity of the turbulence
depends on the airplane's weight, speed, and configuration.
Also referred to as wake turbulence. Vortices from heavy
aircraft may be extremely hazardous to small aircraft.
Wing twist. A design feature incorporated into some wings
to improve aileron control effectiveness at high angles of
attack during an approach to a stall.
Work. A measurement of force used to produce movement.
World Aeronautical Charts (WAC). A standard series
of aeronautical charts covering land areas of the world at a
size and scale convenient for navigation (1:1,000,000) by
moderate speed aircraft. Topographic information includes
cities and towns, principal roads, railroads, distinctive
landmarks, drainage, and relief. Aeronautical information
includes visual and radio aids to navigation, airports, airways,
restricted areas, obstructions and other pertinent data.

G-35

G-36


Index
A
Adjustable stabilizer.....................................................6-12

Adverse balance ...........................................................10-3

Adverse conditions.......................................................13-5

Adverse yaw...................................................................6-3

Advisory circular (AC) ................................................1-10

Aeromedical factors .....................................................17-1

Aeronautical charts ............................................ 14-3, 16-2

Aeronautical decision-making .......................................2-1

History of ADM .........................................................2-2

Aeronautical Information Manual (AIM) ......................1-9

After-landing..............................................................14-34

Ailerons..........................................................................6-3

Coupled ailerons.........................................................6-4

Differential ailerons....................................................6-4

Frise-type ailerons ......................................................6-4

Airborne radar..............................................................13-4

Aircraft documents.........................................................9-6

Aircraft engine ...............................................................7-1

Aircraft inspections........................................................9-8

100-hour Inspection....................................................9-8

Altimeter system inspection .......................................9-9

Annual inspection.......................................................9-8

Preflight inspections ...................................................9-9

Transponder inspection ..............................................9-9

Aircraft maintenance......................................................9-8

Aircraft owner/operator responsibilities ......................9-13

Aircraft Owners and Pilots Association (AOPA) ........3-13

Aircraft types and categories .......................................1-15

Air data computer (ADC) ............................................8-14

Airfoil..................................................................... 4-6, 4-7

Leading edge ..............................................................4-6

Trailing edge...............................................................4-6

Airframe systems .........................................................7-25

Airline Deregulation Act of 1978 ..................................1-7

Air masses..................................................................12-17

Air navigation ..............................................................16-1

Airplane........................................................................1-15

Airplane flight manuals (AFM) ......................... 9-1, 16-17

Airport beacon ...........................................................14-16

Airport lighting ..........................................................14-18

Airport markings..........................................................14-5


Other markings .......................................................14-15

Runway markings.....................................................14-5

Taxiway markings ..................................................14-11

Airports ........................................................................14-2

Civil airports.............................................................14-2

Military/federal government airports .......................14-2

Private airports .........................................................14-2

Towered....................................................................14-2

Nontowered ..............................................................14-2

Airport signs...............................................................14-15

Destination signs ........................................ 14-12, 14-16

Direction signs........................................................14-16

Information signs....................................................14-16

Location signs ........................................................14-15

Mandatory instruction signs ...................................14-15

Runway distance remaining signs ..........................14-16

Airport surveillance radar ............................................13-4

Air Route surveillance radar (ARSR) ..........................13-2

Air Route traffic control center (ARTCC)...................13-2

Airship..........................................................................1-15

Airspace .......................................................................15-1

Airspeed ............................................................. 9-2, 16-10

Airspeed indicator (ASI)..............................................11-2

Airspeed indicator markings ..........................................8-9

Airspeed limitations .......................................................8-9

Airspeed tape ...............................................................8-12

Air traffic control (ATC) ................................. 14-24, 15-7

Airworthiness certificate................................................9-7

Airworthiness directives (ADs) ...................................9-12

Alcohol.......................................................................17-15

Alert areas ....................................................................15-4

Alternator .....................................................................7-30

Altimeter .............................................................. 8-3, 8-13

Setting the altimeter....................................................8-5

Altimeter operation ........................................................8-6

Altitude ................................................................ 8-6, 12-6

Absolute altitude.........................................................8-7

Density altitude...........................................................8-7

Indicated altitude ........................................................8-6

Pressure altitude .........................................................8-7

True altitude ...............................................................8-6

Altitude-induced decompression sickness (DCS)......17-18

I-1

Ammeter ......................................................................7-31

Aneroid barometer .......................................................12-5

Aneroid wafer ................................................................8-3

Angle of attack (AOA)...................................................6-4

Anti-ice ........................................................................7-40

Antiservo tab........................................................ 3-6, 6-11

Approach light systems..............................................14-16

Arm ..............................................................................10-4

Assembling necessary material..................................16-17

ATC Automation ...........................................................1-6

ATC delays ..................................................................13-5

ATC radar beacon system (ATCRBS).......................14-24

ATC radar weather displays.......................................13-16

Atmosphere ................................................. 4-1, 11-2, 12-2

Atmospheric circulation...............................................12-3

Atmospheric pressure........................................... 4-3, 11-2

Atmospheric stability .................................................12-12

Attitude indicator ............................................... 8-13, 8-18

Autokinesis ................................................................17-26

Automatic decision-making .........................................2-21

Operational pitfalls ...................................................2-21

Automatic direction finder (ADF) .............................16-29

Automation ..................................................................2-25

Automation management .............................................2-31

Autopilot ......................................................................6-12

Autopilot systems.........................................................2-27

Aviation forecasts ........................................................13-9

Area forecasts (FA) ...................................... 13-9, 13-10

Terminal aerodrome forecasts (TAF).......................13-9

Forecast change group .........................................13-9

Forecast significant weather ................................13-9

Forecast sky condition..........................................13-9

Forecast visibility .................................................13-9

Forecast wind .......................................................13-9

ICAO station identifier .........................................13-9

Probability forecast ..............................................13-9

Type of report .......................................................13-9

Aviation medical examiner (AME) .............................17-2

Aviation routine weather report (METAR) ...............12-17

Aviation safety inspector (ASI) .....................................1-9

Aviation weather reports..............................................13-5

Aviation routine weather report (METAR)
Altimeter setting....................................................13-7

Modifier ................................................................13-6

Remarks ................................................................13-7

Sky condition.........................................................13-7

Station identifier ...................................................13-6

Temperature and dew point ..................................13-7

Type of report .......................................................13-6

Visibility................................................................13-6

Weather.................................................................13-6

I-2

Wind......................................................................13-6

Zulu time ...............................................................13-7

Pilot weather reports (PIREPs).................................13-8

Axes of an aircraft........................................................5-12

Axes of rotation..............................................................6-3


B
Balance.........................................................................10-2

Balance tabs .................................................................6-11

Balloon.........................................................................1-15

Barbs ..........................................................................12-12

Basic aerodynamics .......................................................3-2

Drag ............................................................................3-2

Lift ..............................................................................3-2

Thrust .........................................................................3-2

Weight ........................................................................3-2

Basic empty weight......................................................10-4

Bernoulli, Daniel............................................................4-6

Bernoulli's Principle of Differential Pressure................4-6

Best angle-of-climb speed (VX) ...................................8-10

Best rate-of-climb speed (VY) ......................................8-10

Binocular cues............................................................17-26

Blade angle...................................................................5-28

Bleed air heating systems.............................................7-30

Blocked pitot system....................................................8-10

Blocked static system...................................................8-11

Boundary layer..................................................... 5-6, 5-46

Brake horsepower (BHP)..................................... 7-6, 7-24

Brakes ..........................................................................7-34

Bus bar .........................................................................7-31


C
Cabin pressure control system .....................................7-35

Cabin pressurization system ........................................7-35

Calibrated airspeed (CAS) ................................. 8-9, 11-18

Canard ............................................................................6-7

Carbon monoxide (CO) poisoning.............................17-12

Carburetor air temperature gauge ................................7-11

Carburetor heat.............................................................7-10

Carburetor icing .............................................................7-9

Carburetor systems.........................................................7-8

Float-type carburetor ..................................................7-8

Pressure-type carburetor.............................................7-9

Ceiling........................................................................12-17

Center of gravity (CG) ....................................... 10-2, 10-4

CG limits ..................................................................10-5

CG range...................................................................10-5

Central blind spot.......................................................17-21

Certificated flight instructor (CFI)..................... 1-19, 1-23

Certificate of aircraft registration...................................9-6


Chandelles....................................................................5-36

Chart Supplement U.S. ..............................................16-17

Clearing procedures ...................................................14-28

Before takeoff.........................................................14-28

Climbs and descents ...............................................14-28

Straight and Level ..................................................14-28

Traffic at VOR sites ...............................................14-28

Traffic patterns .......................................................14-28

Training operations.................................................14-28

Climb performance ......................................................11-6

Angle of climb (AOC)..............................................11-7

Climb Performance Factors......................................11-8

Rate of climb (ROC) ................................................11-7

Clouds ........................................................................12-15

Cloud classification ................................................12-17

Alto......................................................................12-17

Castellanus .........................................................12-17

Cirrus..................................................................12-17

Cumulus ..............................................................12-17

Fracto .................................................................12-17

Lenticularus ........................................................12-17

Nimbus ................................................................12-17

Stratus .................................................................12-17

Code of Federal Regulations (CFR) ..............................1-7

Collision avoidance....................................................14-28

Combustion ..................................................................7-18

Combustion heater systems..........................................7-29

Compass heading .......................................................16-16

Composite materials in aircraft......................................3-9

Composites.....................................................................3-9

Compressor stalls .........................................................7-23

Control .........................................................................10-3

Control instruments......................................................3-13

Controllability .................................................... 5-15, 5-42

Controlled airport.........................................................14-2

Controlled airspace ......................................................15-2

Class A airspace .......................................................15-2

Class B airspace........................................................15-2

Class C airspace........................................................15-2

Class D airspace .......................................................15-2

Class E airspace........................................................15-2

Controlled firing areas (CFAs) ....................................15-4

Convective currents .....................................................12-7

Convective significant meteorological

information (WST).....................................................13-12

Converting KTS to MPH ...........................................16-11

Converting minutes to equivalent hours ....................16-11

Cooling...........................................................................7-1

Coriolis force ...............................................................12-3

Corkscrew effect ..........................................................5-31

Course ........................................................................16-10


Course deviation indicator (CDI)...............................16-23

Course intercept .........................................................16-27

Angle of intercept...................................................16-27

Rate of intercept .....................................................16-27

Cross-country flying ....................................................16-1

Crosswind and headwind component chart ...............11-25

Current conditions........................................................13-5


D
Dark adaptation..........................................................17-23

Datum...........................................................................10-5

Data link weather .......................................................13-21

Data link weather products ........................................13-23

Flight information service-broadcast (FIS-B) ........13-23

Daylight saving time ....................................................16-5

Dead reckoning ..........................................................16-13

DECIDE model............................................................2-18

Choose (a course of action) ......................................2-20

Detect (the problem).................................................2-20

Do (the necessary actions)........................................2-20

Estimate (the need to react) ......................................2-20

Evaluate (the effect of the action) ............................2-20

Identify (solutions) ...................................................2-20

Decision-making in a dynamic environment ...............2-21

Use of resources .......................................................2-21

External resources ................................................2-23

Internal resources.................................................2-23

Decision-making process .............................................2-12

Dehydration................................................................17-14

Deice system ................................................................7-40

Delta.............................................................................10-5

Density altitude ..............................................................4-4

Density altitude charts................................................11-20

Density altitude (DA)...................................................12-5

Department of transportation (DOT) .............................1-6

Deposition ..................................................................12-15

Designated pilot examiner ...........................................1-24

Design maneuvering speed (VA) ..................................8-10

Destination forecast .....................................................13-5

Determining loaded weight and CG ............................10-7

Deviation.......................................................... 16-8, 16-16

Dew point....................................................... 12-13, 13-13

Distance measuring equipment (DME) .....................16-27

Doppler radar ...............................................................13-3

Drag........................................................................ 5-1, 5-6

Form drag ...................................................................5-6

Induced drag ....................................................... 5-6, 5-7

Interference drag.........................................................5-6

Parasite drag ...............................................................5-6

Skin friction drag........................................................5-6


I-3

Drift angle ..................................................................16-10

Drugs..........................................................................17-16

Dutch roll .....................................................................5-20

Dynamic hydroplaning...............................................11-13


E
Eddy current damping..................................................8-27

Electrical ........................................................................7-1

Electrical system ..........................................................7-30

Electronic flight display (EFD)............... 3-12, 8-12, 13-18

Elevator .................................................................. 3-6, 6-5

Emergency locator transmitter (ELT)............................9-9

Empennage............................................................. 3-3, 3-6

Empty-field myopia ...................................................17-22

Engine ............................................................................7-1

Engine cooling systems................................................7-17

Engineered materials arresting systems (EMAS) ......14-35

Engine pressure ratio (EPR).........................................7-22

Engine temperature limitations ....................................7-23

Enhanced flight vision system ...................................17-28

Enhanced night vision systems ..................................17-27

Enhanced situational awareness...................................2-30

Enhanced taxiway centerline markings .....................14-12

En route forecast ..........................................................13-5

Environmental control systems......................................7-1

Equipment use..............................................................2-27

Equivalent airspeed (EAS).........................................11-18

Equivalent shaft horsepower (ESHP) ..........................7-24

Estimated time en route (ETE) ..................................16-17

Exhaust gas temperature (EGT)...................................7-22

Exhaust gas temperature (EGT) gauge ..........................7-9

Exhaust heating systems ..............................................7-29

Exhaust systems ...........................................................7-18

Explosive decompression.............................................7-36

Exposure to chemicals ...............................................17-13

Engine oil ...............................................................17-14

Fuel.........................................................................17-14

Hydraulic fluid .......................................................17-13


F
Fairings ..........................................................................5-6

False horizon..............................................................17-26

Fascination (fixation) .................................................17-27

Fatigue........................................................................17-13

FDC NOTAMs.............................................................1-13

Featureless terrain illusion .........................................17-10

Federal Aviation Administration (FAA)........................1-3

Federal certification of pilots and mechanics ................1-4

Federal Communications Commission (FCC). 9-13, 14-22

Field offices ...................................................................1-8

I-4

Filing a VFR flight plan.............................................16-21

Flameout ......................................................................7-24

Flaperons........................................................................6-5

Flaps...............................................................................6-8

Fowler flaps................................................................6-9

Plain flap ....................................................................6-8

Split flap .....................................................................6-8

Flicker vertigo............................................................17-27

Flight ..............................................................................1-2

Flight computers ........................................................16-12

Flight controls ................................................................6-2

Flight control systems ....................................................6-1

Flight diversion ..........................................................16-34

Flight limits....................................................................9-4

Flight maneuvers..........................................................5-36

Flight planning ...........................................................16-17

Flight school.................................................................1-18

Flight service station ....................................................13-4

Flight Standards District Office (FSDO) .......................1-9

Flight Standards Service (AFS) .....................................1-8

Floor load limit ............................................................10-5

Flux gate compass system............................................8-20

Fog ................................................................. 12-15, 17-10

Advection fog .........................................................12-15

Ice fog.....................................................................12-15

Radiation fog ..........................................................12-15

Sea smoke...............................................................12-15

Steam fog................................................................12-15

Upslope fog ............................................................12-15

Forces in climbs ...........................................................5-23

Forces in descents ........................................................5-24

Forces in turns..............................................................5-22

Foreign object damage (FOD) .....................................7-23

Free directional oscillations .........................................5-20

Free-stream velocity.......................................................5-6

Friction...........................................................................4-2

Fronts .........................................................................12-18

Cold front ...............................................................12-20

Fast-moving cold front .......................................12-20

Occluded front........................................................12-21

Warm front .............................................................12-18

Fuel ................................................................................7-1

Fuel consumption.......................................................16-11

Fuel contamination.......................................................7-27

Fuel fired geaters..........................................................7-29

Fuel gauges ..................................................................7-26

Fuel grades ...................................................................7-27

Aviation gasoline (AVGAS) ....................................7-27

Supplemental type certificate (STC) ........................7-27

Fuel injection system ...................................................7-11

Fuel load.......................................................................10-5


Fuel primer...................................................................7-25

Fuel rate .....................................................................16-17

Fuel selectors ...............................................................7-26

Fuel systems.................................................................7-25

Fuel-pump system ....................................................7-25

Gravity-feed system .................................................7-25

Fuel tanks .....................................................................7-25

Full authority digital engine control (FADEC)............7-20

Fuselage .........................................................................3-3


G
General Aviation Manufacturers

Association (GAMA).....................................................9-2

Generator......................................................................7-30

Glider ...........................................................................1-15

Global positioning system (GPS)..................... 3-13, 16-30

RAIM capability.....................................................16-32

Selective availability ..............................................16-31

VFR use of GPS .....................................................16-32

Graphical METARs ...................................................13-21

Ground adjustable tabs.................................................6-11

Ground lighting illusions ...........................................17-10

Ground power unit (GPU) ...........................................7-30

Groundspeed (GS) .................................. 8-9, 16-10, 16-17

Gyroscopic action ..................................... 5-31, 5-32, 8-15

Precession.................................................................8-15

Rigidity in space.......................................................8-15

Gyroscopic attitude indicators .....................................6-12

Gyroscopic flight instruments......................................8-15

Gyroscopic principles ..................................................8-15


H
Hazard ............................................................................2-4

Hazardous attitudes........................................................2-5

Hazardous in-flight weather advisory (HIWAS) .........13-4

Haze ...........................................................................17-10

Heading ......................................................................16-10

Heading indicator............................................... 8-13, 8-19

Heatstroke ..................................................................17-14

High speed flight..........................................................5-44

Hypersonic................................................................5-45

Subsonic flow ................................................. 5-44, 5-45

Supersonic flow........................................................5-44

Transonic ..................................................................5-45

High speed flight controls............................................5-49

High speed stalls ..........................................................5-36

Horizontal situation indicator ....................................16-24

Human behavior...........................................................2-11

Humidity ............................................................ 4-5, 12-13

Relative humidity ...................................................12-13


Hydraulic systems........................................................7-31

Hydromechanical ...........................................................6-2

Hyperventilation ..........................................................17-4

Hypoxia........................................................................17-3

Histotoxic hypoxia ...................................................17-4

Hypemic hypoxia .....................................................17-3

Hypoxic hypoxia ......................................................17-3

Stagnant hypoxia ......................................................17-3


I

Ice.................................................................................5-26

Icing ...........................................................................12-24

Ignition...........................................................................7-1

Ignition system.............................................................7-15

Illusions........................................................................17-6

Impact pressure chamber and lines ................................8-2

Inches of mercury ........................................................12-4

Inclinometer .................................................................8-18

Indicated airspeed (IAS) .................................... 8-8, 11-18

Induction ........................................................................7-1

Induction system ............................................................7-7

In-flight weather advisories .......................................13-11

AIRMET.................................................................13-11

SIGMET .................................................................13-12

In-runway lighting .....................................................14-18

Instrument landing system (ILS) .................................3-13

Intelligent flight control systems (IFCS) .......................6-2

International Civil Aviation Organization

(ICAO) ............................................................... 4-3, 14-22

International Standard Atmosphere (ISA) .. 4-3, 11-2, 12-5

Inversion ....................................................................12-13

Isobars ........................................................................12-12


J
Jet-fueled piston engine .................................................7-4


K
Knowledge examination ..............................................1-21


L
Lags..............................................................................8-27

Land and hold short lights..........................................14-18

Landing ......................................................................14-34

Landing charts............................................................11-26

Landing gear ..........................................3-3, 3-7, 7-1, 7-33

Landing gear extended speed (VLE) .............................8-10

Landing gear operating speed (VLO) ............................8-10

Landing performance .................................................11-16

Landing strip indicators .............................................14-20


I-5

Lattitude .......................................................................16-3

Lazy eights ...................................................................5-36

Leading edge device ......................................................6-9

Leading edge cuffs ...................................................6-10

Leading edge flaps....................................................6-10

Leads ............................................................................8-27

Licensed empty weight ................................................10-5

Lift..................................................................................5-1

Lift/drag ratio .................................................................5-5

Lighter-than-air aircraft ...............................................1-15

Lightning....................................................................12-25

Lightning strike protection...........................................3-11

Likelihood of an event ...................................................2-6

Improbable..................................................................2-6

Occasional ..................................................................2-6

Probable......................................................................2-6

Remote .......................................................................2-6

Load distribution..........................................................5-43

Loadmeter ....................................................................7-31

Local airport advisory ..................................................15-6

Longitude .....................................................................16-3

Lost procedures..........................................................16-34

Lubrication.....................................................................7-1


M
Mach buffet..................................................................5-49

Mach number ...............................................................5-45

Magnetic compass........................................................8-23

Induced errors...........................................................8-24

Acceleration error ................................................8-26

Deviation...............................................................8-24

Dip errors .............................................................8-25

Northerly turning errors .......................................8-26

Oscillation error ...................................................8-27

Southerly turning errors .......................................8-26

Variation ...............................................................8-24

Magnetic compasses ....................................................6-12

Magnetic heading.......................................................16-16

Magnus effect.................................................................4-6

Maintenance entries .....................................................9-10

Managing aircraft automation......................................2-29

Maneuverability ...........................................................5-15

Manifold absolute pressure (MAP)................................7-6

Maximum landing weight ............................................10-5

Maximum ramp weight................................................10-5

Maximum takeoff weight.............................................10-5

Maximum weight .........................................................10-5

Maximum zero fuel weight..........................................10-5

Mean aerodynamic chord (MAC)...................... 5-13, 10-5

Measurement of direction ............................................16-5

Medical certificate .......................................................17-2

I-6

Medical certification requirements ..............................1-20

Meridians .....................................................................16-3

Mesopic vision ...........................................................17-21

Mesosphere ..................................................................12-3

Meteorologists..............................................................13-1

Microjets ......................................................................7-20

Middle ear ....................................................................17-5

Military operation areas (MOAs).................................15-4

Military training routes (MTRs) ..................................15-6

Minimum control speed (VMC) ..................................8-10

Minimum equipment lists (MEL) ..................................9-9

Mixture control ..............................................................7-9

Moisture .....................................................................12-13

Moment ........................................................................10-5

Moment arm....................................................... 10-4, 5-13

Moment index ..............................................................10-5

Monocoque ............................................................ 3-3, 3-8

Motion sickness .........................................................17-12

Multi-function display (MFD) ......................... 3-12, 13-18


N
N1 indicator ..................................................................7-23

N2 indicator ..................................................................7-23

National Aeronautics and Space Administration

(NASA) ................................................................ 6-2, 3-13

National airspace system..............................................15-7

National Oceanic and Atmospheric Administration

(NOAA) .........................................................................4-5

National security areas (NSAs)....................................15-7

National weather service (NWS) .................................13-1

Navigation instruments ................................................3-13

Negative arm..............................................................10-10

Negative dynamic stability ..........................................5-15

Negative static stability................................................5-14

Net thrust......................................................................7-24

Neutral dynamic stability.............................................5-15

Neutral static stability ..................................................5-14

Newton's Basic Laws of Motion ...................................4-5

Newton's First Law........................................................4-5

Newton's First Law of Motion ....................................5-22

Newton's Second Law ...................................................4-5

Newton's Third Law ......................................................4-6

Newton's Third Law of Physics ..................................5-31

Next generation weather radar system

(NEXRAD) ................................................................13-18

Abnormalities .........................................................13-21

Limitations .............................................................13-21

Night blind spot..........................................................17-22

Night landing illusions...............................................17-27

Night vision................................................................17-22

Night vision illusions .................................................17-26


Night vision protection ..............................................17-23

Nondirectional beacon (NDB) .....................................3-13

Notices to Airmen (NOTAM)......................................13-5

FDC NOTAMs.........................................................1-12

NOTAM composition ..............................................1-13

NOTAM (D) information.........................................1-12


O
Obstruction lights.......................................................14-19

Dual lighting...........................................................14-19

High intensity white obstruction lights ..................14-19

Red obstruction lights.............................................14-19

Obstructions on wind ...................................................12-8

Oil Systems ..................................................................7-16

Operational incidents (OI) .........................................14-31

Optical illusions .........................................................17-10

Original equipment manufacturer (OEM) .....................7-4

Outside air temperature (OAT) gauge ............... 7-11, 8-28

Oxygen masks..............................................................7-38

Oxygen systems ...........................................................7-37

Continuous-flow oxygen system..............................7-38

Diluter-demand oxygen systems ..............................7-38

Electrical pulse-demand oxygen system ..................7-38

Pressure-demand oxygen systems ............................7-38


P
Parachute jump aircraft operations ..............................15-6

Parachute jumps .........................................................15-11

Parallels........................................................................16-3

PAVE checklist..............................................................2-8

Payload.........................................................................10-5

Pennants .....................................................................12-12

Perceive, Process, Perform (3P)...................................2-15

Forming good safety habits ......................................2-18

Performance .................................................................11-5

Performance charts.....................................................11-19

Performance data .........................................................11-1

Performance instruments .............................................3-12

P factor ............................................................... 5-30, 5-32

Photopic vision...........................................................17-20

Pilotage ......................................................................16-12

Pilot certifications ........................................................1-16

Airline transport pilot ...............................................1-18

Commercial pilot......................................................1-18

Private pilot ..............................................................1-17

Recreational pilot .....................................................1-17

Sport pilot.................................................................1-16

Pilot deviations...........................................................14-31

Pilot's operating handbook (POH)..................... 9-1, 16-17


Pitch .............................................................................5-13

Pitching ........................................................................5-15

Pitot-static flight instruments.........................................8-1

Placards ..........................................................................9-4

Plotter.........................................................................16-12

Positive dynamic stability ............................................5-15

Positive static stability .................................................5-14

Postural considerations ................................................17-8

Powered-lift..................................................................1-15

Powered parachute .......................................................1-15

Powerplant ...............................................3-3, 3-7, 7-1, 9-3

Practical test .................................................................1-22

Precipitation ...............................................................12-17

Precision approach path indicator (PAPI)..................14-16

Pre-landing.................................................................14-34

Pressure ..........................................................................4-3

Pressure altimeter...........................................................8-3

Pressure altitude ................................................... 4-4, 11-3

Pressurized aircraft.......................................................7-34

Preventive maintenance ...............................................9-10

Primary flight controls ...................................................6-2

Primary flight display ..................................................3-12

Primary locations of the FAA

Field offices
Flight Standards District Office (FSDO) ...............1-9

Flight Standards Service (AFS) ..............................1-8

Primary radar .............................................................14-24

Procedures, vortex avoidance ....................................14-28

Professional Air Traffic Controllers Organization

(PATCO) strike..............................................................1-6

Prohibited areas............................................................15-3

Propeller.......................................................... 3-7, 7-1, 7-4

Adjustable-pitch propeller..........................................7-6

Fixed-pitch propeller ..................................................7-5

Propeller anti-ice..........................................................7-41

Propeller blade .............................................................5-28

Propeller principles ......................................................5-28

Published VFR routes ..................................................15-6

Pulse oximeters ............................................................7-39


R
Radar observations.......................................................13-3

Radar traffic advisories ..............................................14-26

Radio communications...............................................14-22

Radio equipment ........................................................14-22

Radio magnetic indicator (RMI)................................16-24

Radio navigation ........................................................16-22

Radius of turn...............................................................5-39

Range performance ......................................................11-9


I-7

Rapid decompression ...................................................7-36

Reciprocating engines....................................................7-2

Four-stroke engines ....................................................7-3

Horizontally-opposed engine .....................................7-2

In-line engines ............................................................7-2

Radial engines ............................................................7-2

Two-stroke engine......................................................7-3

Reference datum ..........................................................10-5

Refueling procedures ...................................................7-29

Region of reversed command ....................................11-11

Repairs and alterations.................................................9-12

Respect for onboard systems .......................................2-29

Restricted areas ............................................................15-3

Retractable landing gear ..............................................7-34

Reversible perspective illusion ..................................17-26

Risk ........................................................................ 2-4, 2-6

Mitigating risk ............................................................2-8

Risk management...........................................................2-3

Rocket ..........................................................................1-15

Roll...............................................................................5-13

Rolling..........................................................................5-17

Rotorcraft .....................................................................1-15

Gyroplane .................................................................1-15

Helicopter .................................................................1-15

Rough air......................................................................5-37

Rudder.................................................................... 3-6, 6-4

Ruddervators ..................................................................6-8

Runway and terrain slopes illusion ............................17-10

Runway approach area holding position signs and

markings.....................................................................14-14

Runway centerline lighting system (RCLS) ..............14-18

Runway confusion .....................................................14-31

Runway end identifier lights (REIL) .........................14-17

Runway holding position marking...............................14-8

Runway holding position sign .....................................14-6

Runway incursion avoidance .....................................14-30

Runway lighting.........................................................14-17

Runway surface and gradient.....................................11-12

Runway width illusion ...............................................17-10


S
Safety program airmen notification system

(SPANS) ......................................................................1-14

Satellite ........................................................................13-4

Scanning techniques...................................................17-23

Scotopic vision...........................................................17-21

Sectional charts ............................................................16-2

Segmented circle visual indicator system ..................14-20

Self-Imposed stress ....................................................17-25

Semimonocoque..................................................... 3-3, 3-8

Servicing of oxygen systems .......................................7-39

I-8

Severity of an event .......................................................2-6

Catastrophic................................................................2-6

Critical ........................................................................2-6

Marginal .....................................................................2-6

Negligible ...................................................................2-6

Shock waves.................................................................5-46

Significant Meteorological Information (SIGMET)..13-12

Single-engine best rate-of-climb (VYSE).......................8-10

Single-pilot resource management.................................2-4

Sinus problems.............................................................17-5

Situational awareness...................................................2-24

Obstacles to maintaining situational awareness .......2-24

Size-distance illusion .................................................17-27

Skidding turn................................................................5-23

Slipping turn.................................................................5-23

Slip/skid indicator ........................................................8-13

Slotted flap .....................................................................6-8

Spatial disorientation ...................................................17-6

Special airworthiness certificate ....................................9-8

Special flight permits ...................................................9-12

Special use airspace .....................................................15-3

Spins.............................................................................5-36

Spiral instability ...........................................................5-20

Spoilers ........................................................................6-10

Squall line ..................................................................12-23

SRM and the 5P check.................................................2-13

Passengers ................................................................2-14

Pilot ..........................................................................2-14

Plan...........................................................................2-14

Plane .........................................................................2-14

Programming ............................................................2-15

Stabilator ........................................................................6-7

Stability ............................................5-14, 5-42, 10-2, 10-3

Dynamic stability .....................................................5-14

Lateral stability.........................................................5-17

Dihedral................................................................5-18

Sweepback and wing location...............................5-18

Keel effect and weight distribution .......................5-18

Longitudinal stability ...............................................5-15

Static stability...........................................................5-14

Vertical stability .......................................................5-19

Stalls................................................................... 5-25, 5-36

Stall speed performance charts ..................................11-27

Standard airworthiness certificate..................................9-7

Standard datum plane (SDP)........................................11-3

Standard empty weight ................................................10-5

Standard temperature lapse rate ...................................11-2

Standard weights..........................................................10-5

Starting system.............................................................7-18

Static pressure chamber and lines ..................................8-2

Station ..........................................................................10-5


Straight and Level ............................................ 11-5, 14-28

Stratosphere..................................................................12-3

Stress ..........................................................................17-12

Stress management.......................................................2-21

Student pilot .................................................................1-20

Student pilot solo requirements ...................................1-21

Subcomponents of an airplane .......................................3-8

Airframe .....................................................................3-8

Brakes.........................................................................3-8

Electrical system.........................................................3-8

Flight controls ............................................................3-8

Sublimation ................................................................12-15

Sumps...........................................................................7-27

Superchargers...............................................................7-12

Surface aviation weather observations (METARs) .....13-2

Sweepback ...................................................................5-48

Synopsis .......................................................................13-5

Synthetic vision system..............................................17-28


T
Tachometer ..................................................................8-13

Tailwheel landing gear airplanes .................................7-33

Takeoff charts ............................................................11-20

Takeoff performance..................................................11-14

Taxiway centerline lead-on lights..............................14-18

Taxiways, marking and lighting of permanently

Closed runways..........................................................14-14

Temperature ................................................... 12-13, 13-13

Temporary flight restrictions (TFR) ............................15-6

Terminal doppler weather radar (TDWR) ...................13-4

Terminal radar service area (TRSA)................ 14-26, 15-7

Tetrahedron ................................................................14-20

Thermosphere ..............................................................12-3

Thielert, Frank................................................................7-4

Three-color visual approach path...............................14-17

Thrust ..................................................................... 5-1, 5-2

Thrust horsepower (THP) .................................... 7-6, 7-24

Thunderstorms ...........................................................12-22

Time and distance check from a station.....................16-26

Time Zones ..................................................................16-3

Central standard time................................................16-5

Eastern standard time ...............................................16-5

Mountain standard time............................................16-5

Pacific standard time ................................................16-5

Tornadoes...................................................................12-23

Torque ..........................................................................5-30

Torquemeter.................................................................7-22

Total distance .............................................................16-17

Touchdown zone lights (TDZL) ................................14-18

Track ..........................................................................16-10

Transcontinental air mail route ......................................1-4


Transcribed weather broadcast (TWEB)

(Alaska only)................................................................13-4

Transponder ...............................................................14-25

Trend vectors ...............................................................8-14

Tricycle landing gear airplanes....................................7-33

Trim systems................................................................6-10

Trim tabs .............................................................. 3-6, 6-10

Tropopause...................................................................12-3

Troposphere .................................................................12-2

True airspeed (TAS) .......................................... 8-9, 11-18

True course.................................................................16-16

True heading ..............................................................16-16

Truss structure................................................................3-8

T-tail configuration ........................................................6-6

Turbine engines............................................................7-20

Turbofan ...................................................................7-21

Turbojet ....................................................................7-20

Turboprop.................................................................7-21

Turboshaft ................................................................7-21

Turbosuperchargers............................................ 7-12, 7-13

Turbulence .................................................................12-24

Turn indicators ................................................... 8-13, 8-16

Turn-and-slip indicator.............................................8-16

Turn rate indicator........................................................8-13


U
Ultralight vehicle .........................................................1-14

Uncontrolled airspace ..................................................15-3

Class G airspace .......................................................15-3

Universal coordinated time (UTC) ..............................16-5

Unmanned free balloons ............................................15-11

Upper air observations .................................................13-2

Useful load ...................................................................10-5


V

VA ...............................................................................11-18

Variable inlet guide vane (VIGV)................................7-24

Variation .......................................................... 16-6, 16-16

Vector analysis...........................................................16-13

Vehicle (driver) deviations ........................................14-31

Vertical card magnetic compass ..................................8-27

Vertical speed indicator (VSI) ............................. 8-7, 8-13

Very high frequency (VHF).......................................16-22

Very high frequency (VHF) omni-directional

radio range (VOR) ........................................... 3-13, 16-22

Very light jets (VLJs)...................................................7-20

Vestibular illusions

Coriolis illusion ........................................................17-7

Elevator illusion .......................................................17-8

Graveyard spiral .......................................................17-7


I-9

Somatogravic illusion...............................................17-7

The leans...................................................................17-7

vestibular system..........................................................17-6

VFE ..............................................................................11-18

VFR terminal area charts .............................................16-2

VFR waypoints ..........................................................16-33

Vg diagram...................................................................5-37

Viscosity ........................................................................4-2

Visibility ....................................................................12-17

Vision in flight ...........................................................17-19

Visual approach slope indicator (VASI)....................14-16

visual flight rules (VFR) ..............................................16-1

Visual glide slope indicators......................................14-16

Visual illusions.............................................................17-8

Autokinesis...............................................................17-8

False horizon ............................................................17-8

VLE ..............................................................................11-18

VLO..............................................................................11-18

VNE..............................................................................11-19

VNO .............................................................................11-19

VOR/DME RNAV.....................................................16-28

Vortex avoidance procedures.....................................14-28

Vortex behavior .........................................................14-27

Vortex generation.......................................................14-26

Vortex strength...........................................................14-27

VS0 ..............................................................................11-18

VS1 ..............................................................................11-18

V-Tail.............................................................................6-8

VX ...............................................................................11-18

VY ...............................................................................11-18


W
Wake turbulence ................................................ 5-9, 14-26

Warning areas ..............................................................15-4

Water refraction .........................................................17-10

WCA ..........................................................................16-10

Weather ........................................................................12-1

Weather avoidance assistance....................................13-18

Weather briefings.........................................................13-5

Abbreviated briefing.................................................13-5

Outlook briefing .......................................................13-5

Standard briefing ......................................................13-5

Weather charts ...........................................................13-13

Significant weather prognostic charts ....................13-15

Surface analysis chart.............................................13-13

Dew point............................................................13-13

Present weather ..................................................13-13

Pressure change/tendency ..................................13-13

Sea level pressure ...............................................13-13

Sky cover.............................................................13-13

Temperature........................................................13-13

I-10

Wind....................................................................13-13

Weather depiction chart..........................................13-15

Weather check............................................................16-17

Weather products age and expiration.........................13-18

Weight.......................................... 5-1, 5-2, 5-8, 5-40, 10-2

Weight and balance............................................ 5-40, 10-4

Weight and balance computations............................10-5

Weight and balance restrictions ...............................10-6

Weight and loading distribution.....................................9-3

Weight control .............................................................10-1

Weight-shift-control.....................................................1-15

Wind correction angle (WCA)...................................16-16

Wind direction indicators...........................................14-20

Wind patterns ...............................................................12-7

Winds and temperature aloft forecast (FB)................13-13

Winds and temperatures aloft ......................................13-5

Wind shear

Low-level wind shear .............................................12-11

Wind shifts .................................................................12-21

Wind sock ..................................................................14-20

Wind triangle .............................................................16-13

Winglets .........................................................................4-9

Wings .............................................................................3-3

Wingtip vortices.............................................................5-8

World aeronautical charts ............................................16-2

WSR-88D NEXRAD radar..........................................13-3


Y
Yaw ..............................................................................5-13

Yawing.........................................................................5-19

Yaw String ...................................................................8-18


Z
Zero fuel weight.........................................................10-10


% ===========================================

% \begin{marginfigure}
% \includegraphics[]{Picture1.jpg}
% Notes or comments here
% \end{marginfigure}

% \begin{marginfigure}
% \qrcode[height=\marginparwidth]{https://www.youtube.com/watch?v=AxldOEOnzPA}\\
% Notes or comments here
% \end{marginfigure}

%\backmatter
%\printindex
\end{document}
